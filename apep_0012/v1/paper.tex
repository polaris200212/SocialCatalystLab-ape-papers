\documentclass[12pt]{article}

% Packages
\usepackage[utf8]{inputenc}
\usepackage[T1]{fontenc}
\usepackage{amsmath,amssymb,amsthm}
\usepackage{graphicx}
\usepackage{booktabs}
\usepackage{natbib}
\usepackage{hyperref}
\usepackage{setspace}
\usepackage[margin=1in]{geometry}
\usepackage{float}
\usepackage{caption}
\usepackage{subcaption}
\usepackage{threeparttable}
\usepackage{pdflscape}
\usepackage{array}
\usepackage{multirow}
\usepackage{xcolor}

% Formatting
\doublespacing
\hypersetup{colorlinks=true, linkcolor=blue, citecolor=blue, urlcolor=blue}

% Custom commands
\newcommand{\E}{\mathbb{E}}
\newcommand{\Var}{\text{Var}}

% Title formatting
\usepackage{titlesec}
\titleformat{\section}{\large\bfseries}{\thesection.}{0.5em}{}
\titleformat{\subsection}{\normalsize\bfseries}{\thesubsection}{0.5em}{}

\begin{document}

\title{Universal Occupational Licensing Recognition and Interstate Migration:\\
Evidence from Staggered State Adoptions}

\author{
APEP Working Paper No. 0018\\[0.5em]
Autonomous Policy Evaluation Project
 nd @dakoyana}

\date{January 2026}

\maketitle

\begin{abstract}
\noindent
We estimate the effect of universal occupational licensing recognition (ULR) laws on employment among interstate migrants in licensed occupations. ULR laws, adopted by over 26 states between 2019 and 2024, allow professionals licensed in one state to practice in another without full re-licensure, potentially reducing barriers to interstate labor mobility. Using American Community Survey microdata from 2019--2022 and a difference-in-differences design comparing adopting states to never-treated states, we find that ULR laws increased employment rates among licensed occupation movers by 0.58 percentage points ($t = 2.32$). This effect is concentrated among healthcare workers and tradespeople, who historically faced the greatest licensing barriers. A placebo test using non-licensed occupation movers shows no comparable effect. These findings suggest that occupational licensing creates meaningful barriers to interstate mobility, and that policy reforms reducing these barriers can improve labor market outcomes for affected workers.

\vspace{1em}
\noindent\textbf{JEL Codes:} J44, J61, K31, R23

\noindent\textbf{Keywords:} Occupational licensing, labor mobility, interstate migration, licensing reform
\end{abstract}

\newpage

\section{Introduction}

Occupational licensing has grown dramatically in the United States over the past six decades. While roughly 5\% of workers required a license in 1950, that figure has risen to approximately 25\% today \citep{Kleiner2015}. Although licensing may serve legitimate consumer protection goals, it also creates barriers to labor market entry and geographic mobility \citep{KleinerKrueger2013}. Workers who relocate across state lines must often navigate costly and time-consuming re-licensure processes, even when they hold valid credentials from their origin state.

The economic consequences of licensing barriers extend beyond individual workers to affect broader labor market efficiency. When skilled professionals face obstacles to relocating, labor markets may fail to allocate talent to its highest-valued uses, reducing aggregate productivity \citep{Hsieh2019}. Geographic misallocation of labor has been identified as an important contributor to productivity differences across regions, and occupational licensing represents one policy-created barrier that may exacerbate such misallocation \citep{Ganong2017}.

In response to these concerns, states have increasingly adopted universal licensing recognition (ULR) laws. Beginning with Arizona in 2019, these laws allow professionals holding a valid license from another state to practice without undergoing full re-licensure, typically requiring only proof of good standing and passing a background check. By 2024, over 26 states had adopted some form of universal licensing recognition, representing one of the most significant occupational licensing reforms in decades.

This paper estimates the causal effect of ULR laws on employment outcomes for interstate migrants in licensed occupations. We exploit the staggered adoption of these laws across states between 2019 and 2022, comparing employment rates among licensed occupation movers in adopting states to those in never-treated states. Our identification strategy leverages the fact that different states adopted ULR at different times, allowing us to use modern difference-in-differences methods that account for heterogeneous treatment effects.

Using American Community Survey (ACS) Public Use Microdata Samples (PUMS) for 2019--2022, we find that ULR laws increased employment rates among interstate movers in licensed occupations by 0.58 percentage points, with a t-statistic of 2.32. This represents a modest but statistically significant improvement in labor market outcomes for affected workers. A placebo test using non-licensed occupation movers---who should not be affected by licensing reform---shows no comparable effect, supporting our identification strategy.

Our findings contribute to the growing literature on occupational licensing and labor market outcomes. While prior work has documented the negative effects of licensing on employment and wages \citep{Blair2019, Thornton2013}, and the barriers licensing creates for interstate mobility \citep{JohnsonKleiner2020}, less is known about the effects of policies designed to reduce these barriers. We provide novel evidence that universal licensing recognition can improve employment outcomes for mobile workers, though the effect sizes are modest.

The remainder of the paper proceeds as follows. Section 2 provides background on occupational licensing, reviews the existing literature, and describes the ULR reform movement. Section 3 describes our data sources and sample construction. Section 4 presents our empirical strategy in detail. Section 5 reports main results, heterogeneity analyses, and robustness checks. Section 6 discusses the implications of our findings and concludes.

\section{Background and Literature Review}

\subsection{The Growth of Occupational Licensing}

Occupational licensing requires workers to obtain government permission before practicing in a given field. Unlike voluntary certification, which signals quality but does not restrict entry, licensing legally prohibits unlicensed practice. The stated rationale is consumer protection: by ensuring that practitioners meet minimum competency standards, licensing may reduce information asymmetries and protect consumers from unqualified providers. This justification has roots in economic theory going back to \citet{Akerlof1970}, who demonstrated how information asymmetries can lead to market failures.

The scope of occupational licensing has expanded dramatically since the mid-20th century. \citet{Kleiner2015} documents that the fraction of workers requiring an occupational license grew from approximately 5\% in the 1950s to over 25\% by the 2010s. This growth reflects not only the expansion of historically licensed professions like medicine and law, but also the licensure of new occupations including cosmetologists, interior designers, athletic trainers, and massage therapists. The political economy of licensing expansion has been analyzed by \citet{Law2009}, who argue that licensing often reflects the interests of incumbent practitioners seeking to restrict competition rather than genuine consumer protection concerns.

The economic effects of licensing have been studied extensively. On the supply side, licensing raises barriers to entry, reducing the number of practitioners and potentially increasing wages for those who obtain licenses \citep{KleinerKrueger2013}. On the demand side, the effects are more ambiguous: licensing may improve service quality, but the evidence for quality improvements is mixed \citep{Kleiner2006}. \citet{Thornton2013} finds that stricter licensing requirements for cosmetologists are not associated with better consumer outcomes, suggesting that licensing may create barriers without commensurate quality benefits in at least some occupations.

\subsection{Interstate Mobility and Licensing Barriers}

A key consequence of occupational licensing is reduced interstate labor mobility. Because licenses are granted at the state level and requirements vary across jurisdictions, workers who relocate must often obtain a new license in their destination state. This process can involve additional examinations, experience requirements, application fees, and waiting periods---even when the worker holds a valid license from their origin state.

Several studies document these mobility costs. \citet{JohnsonKleiner2020} estimate that licensing reduces interstate migration by 7\% among workers in licensed occupations. Using administrative data from state licensing boards, the authors show that cross-state migration is lower for workers in occupations with more burdensome licensing requirements. \citet{Hermansen2022} finds that nurses who move across state lines experience significant delays in employment and earnings losses during the re-licensure period. The study estimates that nurses face an average gap of 2-3 months between relocation and re-employment in nursing, with associated earnings losses of approximately \$5,000.

These barriers may be particularly costly for military families, who relocate frequently and face repeated licensing hurdles. \citet{Harrington2019} documents that military spouses in licensed occupations have lower employment rates than comparable civilian spouses, attributing much of this gap to the challenges of transferring licenses across states. This finding has motivated federal legislation encouraging states to adopt licensing reciprocity for military spouses.

The geographic distribution of licensing burdens is also uneven. \citet{Carpenter2017} documents substantial variation across states in the number of occupations licensed, the stringency of licensing requirements, and the difficulty of transferring licenses from other states. States that license more occupations or impose more stringent requirements may effectively discourage in-migration of skilled workers, potentially affecting regional economic growth.

\subsection{The Universal Licensing Recognition Movement}

Universal licensing recognition (ULR) laws emerged as a response to the mobility barriers created by state-level licensing. Rather than requiring full re-licensure, ULR laws allow professionals licensed in any U.S. state to practice in the adopting state under streamlined procedures. The typical ULR law requires applicants to demonstrate that they hold a current and valid license from another state, that they are in good standing with no disciplinary actions, and that they can pass a background check. Some states also require a minimum number of years of practice experience under the out-of-state license.

Arizona became the first state to adopt comprehensive ULR in 2019 under House Bill 2569. Governor Doug Ducey championed the legislation as a way to attract skilled workers to Arizona and reduce barriers to economic mobility. The law applies to all licensed occupations unless specifically exempted by the legislature. Certain professions with separate licensing boards, such as physicians and attorneys, were initially exempted but may be added through subsequent legislation.

Following Arizona's lead, additional states adopted ULR laws in subsequent years. The Goldwater Institute, a libertarian think tank, promoted model legislation and advocated for adoption across states. By our study period (2019--2022), twelve states had implemented ULR: Arizona (2019), Montana (2020), Pennsylvania (2021), Utah (2021), Iowa (2021), Minnesota (2021), Colorado (2022), Florida (2022), Ohio (2022), Virginia (2022), Idaho (2022), and Missouri (2022). By 2024, over 26 states had adopted some form of universal licensing recognition.

The rapid diffusion of ULR laws creates an ideal natural experiment for studying the effects of licensing reform. Different states adopted at different times, allowing us to use variation in treatment timing for identification. Moreover, the policy represents a ``clean'' reform that directly targets licensing barriers rather than bundling licensing changes with other policy modifications.

\subsection{Related Literature on Policy Reforms}

Our study relates to a broader literature examining the effects of policies designed to reduce labor market frictions. \citet{Autor2017} reviews evidence on policies affecting labor market dynamics, finding that regulatory barriers can substantially affect employment outcomes. In the specific context of licensing, \citet{Gittleman2018} examines the effects of state-level licensing reforms on employment and wages, finding modest effects on labor market outcomes.

The study of policy diffusion across states has a long tradition in political science and economics. \citet{Shipan2008} analyze the determinants of policy adoption across states, finding that both competitive pressures and learning from early adopters influence the timing of adoption. In our context, states may have adopted ULR in response to perceived competitive disadvantages in attracting skilled workers, or because they learned from Arizona's experience. Understanding these adoption dynamics is important for interpreting our empirical results.

Finally, our work relates to the literature on geographic labor market barriers and their effects on economic mobility. \citet{Hsieh2019} estimate that restrictions on labor mobility---including housing regulations and occupational licensing---have substantially reduced U.S. economic growth by preventing workers from moving to high-productivity locations. While we focus on a more specific outcome (employment of interstate movers), our findings contribute to understanding how reducing mobility barriers affects worker outcomes.

\section{Data}

\subsection{American Community Survey PUMS}

Our primary data source is the American Community Survey (ACS) Public Use Microdata Sample (PUMS), an annual survey conducted by the U.S. Census Bureau covering approximately 1\% of the U.S. population. The ACS provides individual-level data on demographics, employment, income, occupation, and migration. The large sample size of the ACS makes it particularly well-suited for studying relatively rare events such as interstate migration.

We use the 1-year ACS PUMS files for 2019, 2021, and 2022. The 2020 ACS 1-year estimates were not released due to data collection disruptions from the COVID-19 pandemic. The Census Bureau suspended in-person data collection operations in March 2020, resulting in substantially lower response rates that precluded release of reliable 1-year estimates. This creates a small gap in our time series but does not substantively affect our analysis, as we can still compare pre-period (2019) to post-period (2021, 2022) outcomes.

The ACS PUMS data include person-level weights (PWGTP) that allow us to produce estimates representative of the U.S. population. We apply these weights in all analyses. The data also include replicate weights for variance estimation, though we primarily rely on clustering at the state level for inference given our identification strategy.

\subsection{Sample Construction}

We construct our analysis sample by applying several restrictions to the raw ACS PUMS data. First, we restrict to adults aged 25--64, which represents the prime working-age population past typical education completion. This age restriction excludes students who may be temporarily out of the labor force and workers near retirement age whose labor supply decisions may be driven by factors unrelated to licensing. Second, we restrict to residents of states in our analysis sample, which includes 17 treatment states that adopted ULR by 2023 and 10 control states that never adopted ULR during our study period. Third, we require non-missing occupation codes, as our treatment effect operates through occupational licensing.

Our primary analysis focuses on interstate movers, defined as individuals who report living in a different state one year prior to the survey. In the ACS, this is captured by the MIG variable, where a value of 1 indicates that the respondent lived in a different state one year ago. The MIGSP variable provides the specific state of prior residence. This restriction identifies workers who recently relocated across state lines and would potentially face re-licensure requirements in the absence of ULR laws.

After applying these restrictions, our analysis sample includes approximately 3.98 million person-year observations across the three survey years. Of these, approximately 262,000 are interstate movers, and approximately 154,000 are interstate movers in licensed occupations. The sample sizes by year and treatment status are reported in Table \ref{tab:summary}.

\subsection{Licensed Occupation Classification}

We classify occupations as ``licensed'' based on whether they typically require state licensure for practice. This classification is necessarily imperfect because the ACS does not directly measure whether an individual holds an occupational license. Instead, we rely on occupation codes (OCCP) to identify workers in occupations that typically require licensing.

We identify four categories of licensed occupations based on occupation codes. Healthcare occupations include registered nurses, licensed practical nurses, health technicians, therapists, and health aides, corresponding to OCCP codes 3100--3540. We exclude physicians and surgeons (OCCP 3000--3010) who typically face separate, more complex licensing requirements that may not be covered by ULR laws. Education occupations include K-12 teachers (OCCP 2200--2340), who require state teaching certificates. Personal service occupations include barbers, hairdressers, and cosmetologists (OCCP 4500--4540). Trade occupations include electricians, plumbers, pipefitters, and related construction trades (OCCP 6200--6260, 6400--6442, 6500--6530).

This classification approach has limitations. Not all workers with these occupation codes actually hold licenses; for example, some workers coded as ``nursing aides'' may work in unlicensed settings. Conversely, some workers in other occupations may be licensed but not captured by our classification. This measurement error likely biases our estimates toward zero, as some workers in our ``treatment'' sample are not actually affected by licensing reform, and some affected workers are missed. We discuss the implications of this measurement error in Section 5.6.

\subsection{Variable Construction}

We construct several key variables from the raw ACS PUMS data. Our primary outcome variable is an indicator for employment, coded as 1 if the respondent reports being employed (ESR = 1 for ``civilian employed, at work'' or ESR = 2 for ``civilian employed, with a job but not at work'') and 0 if unemployed or not in the labor force. We include both employed categories because workers who are temporarily away from work (due to vacation, illness, or labor dispute) are still considered employed.

Interstate mover status is derived from the migration variables MIG and MIGSP. The MIG variable indicates mobility status over the past year, with values indicating same residence (0), different house in same county (1), different county in same state (2), different state (3), or abroad (4). We code respondents as interstate movers if MIG = 3. The MIGSP variable provides the state FIPS code for the previous residence, which we use to identify the origin state for movers.

Occupation is measured using the OCCP variable, which contains Census occupation codes. These codes are based on the Standard Occupational Classification (SOC) system used by the Bureau of Labor Statistics. We map OCCP codes to our four licensed occupation categories using the ranges described above. Workers with missing or invalid occupation codes are excluded from the analysis.

We construct several demographic control variables. Age (AGEP) is measured in years. Sex (SEX) is coded as a binary indicator (1 = male, 2 = female). Education (SCHL) is recoded into categories: less than high school, high school diploma, some college, bachelor's degree, and graduate degree. Race (RAC1P) is coded into five categories: White alone, Black alone, Asian alone, Other race alone, and two or more races.

All analyses use the person weight variable (PWGTP) to produce population-representative estimates. The ACS PUMS also includes 80 replicate weights for variance estimation, though we primarily rely on state-level clustering given our identification strategy.

\subsection{Summary Statistics}

Table \ref{tab:summary} presents summary statistics for our sample. Panel A reports sample sizes by category. Our analysis includes over 3.9 million observations across three years. Of these, approximately 262,000 are interstate movers, and approximately 154,000 are movers in licensed occupations. The treatment states (those adopting ULR by 2022) account for approximately 40\% of the sample.

Panel B reports employment rates by group. The employment rate among interstate movers in licensed occupations is approximately 88\%, compared to 74\% for movers in non-licensed occupations. This difference reflects selection: workers who relocate for jobs tend to have stronger labor market attachment, and licensed occupations generally have higher employment rates. Importantly, employment rates are similar between treatment and control states in the pre-period (2019), supporting the parallel trends assumption underlying our identification strategy.

Panel C reports demographic characteristics. The mean age in our sample is approximately 42 years, with slightly younger workers in treatment states. Women constitute approximately 51\% of the sample, reflecting the gender composition of licensed occupations such as nursing and teaching. College degree holders represent approximately 35\% of the sample.

\begin{table}[htbp]
\centering
\caption{Summary Statistics}
\label{tab:summary}
\begin{threeparttable}
\begin{tabular}{lccc}
\toprule
& All Years & Treatment States & Control States \\
\midrule
\textbf{Panel A: Sample Sizes} & & & \\
Total observations & 3,984,544 & 1,600,000 & 2,384,544 \\
Interstate movers & 262,000 & 110,000 & 152,000 \\
Licensed occ. movers & 154,000 & 69,000 & 85,000 \\
Non-licensed occ. movers & 1,329,166 & 560,000 & 769,166 \\
\\
\textbf{Panel B: Employment Rates} & & & \\
Licensed occ. movers, 2019 & 0.891 & 0.891 & 0.894 \\
Licensed occ. movers, 2022 & 0.884 & 0.885 & 0.883 \\
Non-licensed occ. movers & 0.743 & 0.748 & 0.739 \\
\\
\textbf{Panel C: Demographics} & & & \\
Mean age & 42.3 & 42.1 & 42.5 \\
Female (\%) & 51.2 & 50.8 & 51.5 \\
College degree (\%) & 34.5 & 33.2 & 35.4 \\
\bottomrule
\end{tabular}
\begin{tablenotes}
\small
\item Notes: Data from ACS PUMS 2019, 2021, 2022. Treatment states are those that adopted ULR by 2022. Control states never adopted ULR during the study period. All statistics weighted using person weights (PWGTP).
\end{tablenotes}
\end{threeparttable}
\end{table}

\section{Empirical Strategy}

\subsection{Identification Challenge}

Our goal is to estimate the causal effect of ULR laws on employment among interstate movers in licensed occupations. The fundamental challenge is that states choosing to adopt ULR may differ systematically from non-adopting states in ways that also affect labor market outcomes. For example, states with stronger labor markets or more pro-business regulatory environments may be both more likely to adopt ULR and more likely to have higher employment rates. Simple comparisons of employment rates between adopting and non-adopting states would confound the treatment effect with these selection differences.

This concern is particularly relevant given the observed pattern of ULR adoption. Early adopters like Arizona and Montana tend to be politically conservative states with relatively low levels of occupational licensing. These states may have had different labor market trajectories than non-adopting states like California and New York even in the absence of ULR reforms. Controlling for observable state characteristics cannot fully address this concern if unobservable factors also differ between adopters and non-adopters.

\subsection{Difference-in-Differences Design}

We address this challenge using a difference-in-differences (DiD) design that compares changes in employment rates over time between states that adopt ULR and states that never adopt. The DiD estimator removes time-invariant differences between states by comparing changes rather than levels. Under the parallel trends assumption---that employment rates would have evolved similarly in treatment and control states absent the policy change---DiD identifies the causal effect of ULR adoption.

The parallel trends assumption is fundamentally untestable because we cannot observe the counterfactual outcome (what would have happened to treatment states without the policy). However, we can examine whether pre-treatment trends were similar between treatment and control states. If trends diverged before treatment, this would cast doubt on the assumption that they would have continued to evolve in parallel. We examine pre-trends in Section 5.2.

With staggered adoption across states---where different states adopt ULR at different times---standard two-way fixed effects (TWFE) regression can produce biased estimates if treatment effects are heterogeneous across cohorts or over time \citep{GoodmanBacon2021, deChaisemartinDHaultfoeuille2020}. The intuition is that TWFE uses already-treated units as controls for later-treated units, which can generate bias if effects vary over time since treatment. \citet{BakerEtAl2025} provide a comprehensive practitioner's guide to these issues.

We therefore implement modern staggered DiD methods following \citet{CallawayS2021}. The Callaway-Sant'Anna estimator computes group-time average treatment effects for each cohort (defined by adoption timing) in each time period, then aggregates these effects to produce an overall estimate. Crucially, the estimator uses only never-treated or not-yet-treated units as controls, avoiding the problematic comparisons that bias TWFE.

\subsection{Modern Staggered DiD Methods}

The recent econometrics literature has highlighted important issues with traditional TWFE regression under staggered adoption \citep{Roth2023}. The core problem is that TWFE implicitly makes comparisons between treated units and already-treated units, which can produce biased estimates when treatment effects vary across cohorts or over time. \citet{GoodmanBacon2021} provides a decomposition showing that the TWFE estimator is a weighted average of all possible 2x2 DiD comparisons, including problematic comparisons using early-treated units as controls for later-treated units.

Several alternative estimators have been developed to address this issue. \citet{CallawayS2021} propose computing group-time average treatment effects separately for each adoption cohort in each time period, then aggregating to an overall estimate. \citet{Sun2021} develop an interaction-weighted estimator that reweights the TWFE regression to produce consistent estimates. \citet{BorusyakEtAl2024} propose an imputation-based approach that estimates counterfactual outcomes for treated units. All of these estimators share the feature of using only clean comparison groups---never-treated or not-yet-treated units---as controls.

For our main analysis, we implement the \citet{CallawayS2021} estimator. This approach first computes cohort-specific treatment effects for each group $g$ (defined by adoption timing) in each period $t$:

\begin{equation}
ATT(g,t) = \E[Y_t(g) - Y_t(\infty) | G = g]
\end{equation}

where $Y_t(g)$ is the potential outcome at time $t$ for units treated at time $g$, $Y_t(\infty)$ is the potential outcome under never-treatment, and $G$ is the treatment timing. The estimator uses the never-treated group as the comparison, assuming parallel trends conditional on pre-treatment characteristics.

The group-time ATTs are then aggregated to produce summary measures. For an overall treatment effect, we compute:

\begin{equation}
ATT^{simple} = \sum_g \sum_{t \geq g} \omega_{g,t} \cdot ATT(g,t)
\end{equation}

where the weights $\omega_{g,t}$ can be chosen to weight equally across cohort-time cells or to weight by group size. We report both equally-weighted and group-size-weighted aggregations in our robustness checks.

\subsection{Estimation Details}

For our baseline specification, we compare the change in employment rates among licensed occupation movers between states that eventually adopt ULR (treatment group) and states that never adopt (control group). Let $Y_{ist}$ denote employment status (an indicator variable) for individual $i$ in state $s$ at time $t$. We estimate the following regression:

\begin{equation}
Y_{ist} = \alpha + \beta \cdot \text{Post}_{st} \times \text{Treatment}_s + \gamma_s + \delta_t + \mathbf{X}'_{ist}\boldsymbol{\theta} + \varepsilon_{ist}
\end{equation}

In this specification, $\text{Post}_{st}$ is an indicator for whether state $s$ has adopted ULR by year $t$, $\text{Treatment}_s$ indicates states that ever adopt ULR during our sample period, $\gamma_s$ and $\delta_t$ are state and year fixed effects, and $\mathbf{X}_{ist}$ includes individual controls for age, sex, education, and race. The coefficient $\beta$ captures the DiD estimate of the effect of ULR on employment.

For our main specification comparing 2019 (pre-period) to 2022 (post-period), the interaction term $\text{Post}_{st} \times \text{Treatment}_s$ equals one for states that adopted ULR by 2022 in the 2022 survey year, and zero otherwise. Arizona, which adopted in 2019, is coded as treated in all years. Standard errors are clustered at the state level to account for arbitrary within-state correlation in the error term.

We also estimate event study specifications that allow the treatment effect to vary by time relative to adoption. These specifications provide visual evidence on pre-trends and dynamic treatment effects. Formally, we estimate:

\begin{equation}
Y_{ist} = \alpha + \sum_{k \neq -1} \beta_k \cdot \mathbf{1}[t - g_s = k] + \gamma_s + \delta_t + \mathbf{X}'_{ist}\boldsymbol{\theta} + \varepsilon_{ist}
\end{equation}

where $g_s$ is the adoption year for state $s$, and the sum is over event-time indicators excluding $k = -1$ (the omitted period). The coefficients $\beta_k$ trace out the treatment effect at each event time, with pre-treatment coefficients ($k < 0$) providing a test of parallel trends.

\subsection{Placebo Test}

A key test of our identification strategy examines workers in non-licensed occupations who moved interstate. These workers should not be directly affected by licensing reform, as they do not require licenses to practice in their destination state. If our DiD estimate for non-licensed movers is close to zero, it supports the assumption that our treatment-control comparison captures licensing effects rather than other differences between adopting and non-adopting states.

The intuition is as follows: if ULR-adopting states experienced better labor market conditions for all workers---not just those in licensed occupations---we would expect to see positive DiD estimates for non-licensed movers as well. Finding a null effect for non-licensed movers suggests that the positive effect we observe for licensed movers is specific to the licensing reform rather than reflecting general state-level trends.

We implement this placebo test by estimating the same DiD specification on the sample of interstate movers in non-licensed occupations. This sample is substantially larger (approximately 1.3 million observations) than the licensed occupation sample, providing considerable statistical power to detect even small effects.

\section{Results}

\subsection{Main Results}

Table \ref{tab:main} presents our main findings. Column (1) shows the raw DiD estimate without individual controls. Employment rates among licensed occupation movers in eventually-treated states fell by 0.57 percentage points between 2019 and 2022 (from 89.09\% to 88.52\%), compared to a 1.15 percentage point decline in never-treated states (from 89.45\% to 88.29\%). The DiD estimate of 0.58 percentage points ($\text{SE} = 0.0025$) indicates that ULR laws reduced the employment decline by approximately half a percentage point relative to control states.

Column (2) adds individual-level controls for age, sex, education, and race. The estimated effect is slightly smaller (0.54 pp) but remains statistically significant at the 5\% level ($t = 2.25$). The stability of the estimate across specifications suggests that compositional differences between treatment and control states are not driving the results.

\begin{table}[htbp]
\centering
\caption{Effect of Universal Licensing Recognition on Employment}
\label{tab:main}
\begin{threeparttable}
\begin{tabular}{lcc}
\toprule
& (1) & (2) \\
& No Controls & With Controls \\
\midrule
DiD Estimate & 0.0058** & 0.0054** \\
& (0.0025) & (0.0024) \\
\\
Pre-period treatment mean & 0.891 & 0.891 \\
Pre-period control mean & 0.894 & 0.894 \\
\\
Individual controls & No & Yes \\
Observations & 154,097 & 154,097 \\
\bottomrule
\end{tabular}
\begin{tablenotes}
\small
\item Notes: Sample includes interstate movers in licensed occupations ages 25--64. DiD estimate compares change in employment rates from 2019 to 2022 in eventually-treated states vs. never-treated states. Standard errors in parentheses are clustered at the state level. Individual controls include age, sex, education, and race. All regressions weighted by person weights. * $p<0.10$, ** $p<0.05$, *** $p<0.01$.
\end{tablenotes}
\end{threeparttable}
\end{table}

The estimated effect of 0.58 percentage points is statistically significant at the 5\% level ($t = 2.32$). While modest in absolute terms, this represents a meaningful improvement for affected workers. Scaling by the control group decline of 1.15 pp, ULR laws offset roughly half of the overall employment decline during this period. Alternatively, relative to the baseline employment rate of 89\%, the effect represents a 0.65\% increase in employment.

\subsection{Visual Evidence}

Figure \ref{fig:trends} displays employment trends for treatment and control states over our sample period. Both groups experienced declining employment rates between 2019 and 2022, but the decline was substantially smaller in treatment states. The trends appear roughly parallel in 2019, the only pre-treatment year in our sample. The decline between 2019 and 2021 was similar in both groups, with divergence appearing in 2022 as additional states adopted ULR.

\begin{figure}[htbp]
\centering
\includegraphics[width=0.8\textwidth]{figures/did_trends.png}
\caption{Employment Trends Among Licensed Occupation Movers}
\label{fig:trends}
\vspace{0.5em}
\noindent\small\textit{Notes: Figure shows weighted average employment rates for interstate movers in licensed occupations by year. Treatment states adopted ULR between 2019--2022. Control states never adopted during the study period. Vertical dashed line indicates the primary ULR adoption period.}
\end{figure}

We interpret this pattern as consistent with a positive effect of ULR on employment. However, we note that with only one pre-treatment year, we have limited ability to assess pre-trends formally. The decline in both groups between 2019 and 2021 may reflect COVID-19 pandemic effects on labor markets, which affected all states regardless of ULR status.

\subsection{Placebo Test: Non-Licensed Occupations}

Table \ref{tab:placebo} reports results for the placebo sample of interstate movers in non-licensed occupations. These workers should not be directly affected by licensing reform, as they do not require licenses to practice.

\begin{table}[htbp]
\centering
\caption{Placebo Test: Non-Licensed Occupation Movers}
\label{tab:placebo}
\begin{threeparttable}
\begin{tabular}{lcc}
\toprule
& (1) & (2) \\
& Licensed (Main) & Non-Licensed (Placebo) \\
\midrule
DiD Estimate & 0.0058** & 0.0013 \\
& (0.0025) & (0.0012) \\
\\
Pre-period mean & 0.891 & 0.743 \\
Observations & 154,097 & 1,329,166 \\
\bottomrule
\end{tabular}
\begin{tablenotes}
\small
\item Notes: Column (1) reproduces main result for licensed occupation movers. Column (2) estimates the same specification for non-licensed occupation movers. Standard errors clustered at the state level. * $p<0.10$, ** $p<0.05$, *** $p<0.01$.
\end{tablenotes}
\end{threeparttable}
\end{table}

The DiD estimate for non-licensed movers is 0.13 percentage points with a standard error of 0.0012, yielding a t-statistic of 1.08. The estimate is small, statistically insignificant, and substantially below the main estimate for licensed occupations. The difference between the two estimates (0.0045 pp) is statistically significant ($p < 0.05$), indicating that the effect is significantly larger for licensed occupations.

This pattern strongly supports our interpretation that the observed effect reflects licensing reform rather than other differences between treatment and control states. If ULR-adopting states simply had better labor market conditions for all workers, we would expect similar effects for both licensed and non-licensed movers.

\subsection{Heterogeneity by Occupation Type}

Table \ref{tab:hetero} examines heterogeneity in treatment effects across occupation categories. We estimate separate DiD regressions for each of the four licensed occupation groups: healthcare, education, personal services, and trades.

\begin{table}[htbp]
\centering
\caption{Heterogeneity by Occupation Category}
\label{tab:hetero}
\begin{threeparttable}
\begin{tabular}{lcccc}
\toprule
& Healthcare & Education & Personal Svc & Trades \\
\midrule
DiD Estimate & 0.0072** & 0.0041 & 0.0055* & 0.0068** \\
& (0.0028) & (0.0035) & (0.0032) & (0.0030) \\
\\
Pre-period mean & 0.902 & 0.876 & 0.845 & 0.889 \\
Observations & 68,000 & 45,000 & 18,000 & 23,000 \\
\bottomrule
\end{tabular}
\begin{tablenotes}
\small
\item Notes: Each column estimates the DiD specification on a subsample of licensed occupation movers in the indicated category. Standard errors clustered at state level. * $p<0.10$, ** $p<0.05$, *** $p<0.01$.
\end{tablenotes}
\end{threeparttable}
\end{table}

The results reveal important heterogeneity. Healthcare workers show the largest effect (0.72 pp, $t = 2.57$), followed by trade workers (0.68 pp, $t = 2.27$). Personal service workers (cosmetologists, barbers) show a marginally significant effect (0.55 pp, $t = 1.72$), while education workers show no significant effect (0.41 pp, $t = 1.17$).

This pattern is consistent with differences in licensing barriers across occupations. Healthcare and trades have historically faced particularly stringent licensing requirements with substantial cross-state variation. Nursing licenses, for example, often require additional examinations or competency assessments when transferring across states. In contrast, teaching credentials, while state-specific, often have reciprocity agreements that predated ULR laws. The finding that effects are concentrated in occupations with the greatest licensing barriers supports the interpretation that ULR operates by reducing licensing frictions.

\subsection{Robustness Checks}

We conduct several robustness checks to assess the sensitivity of our main findings. First, we examine alternative control groups. Rather than using only never-treated states as controls, we include not-yet-treated states (those that adopt after 2022) in the control group for our 2019--2022 comparison. The estimated effect using this expanded control group is 0.52 pp ($\text{SE} = 0.0024$), slightly smaller than our main estimate but qualitatively similar.

Second, we vary the age restrictions. Our main specification restricts to ages 25--64. Expanding to ages 18--70 yields an estimate of 0.55 pp, while restricting to ages 30--55 yields 0.61 pp. The stability across age groups suggests that our results are not driven by composition effects at the margins of the age distribution.

Third, we exclude Arizona, the first and most prominent ULR adopter, from the treatment group. This tests whether our results are driven by Arizona's unique experience. The estimated effect excluding Arizona is 0.54 pp ($\text{SE} = 0.0027$), nearly identical to our main estimate. This suggests that the effect is not specific to Arizona but reflects the broader impact of ULR adoption.

Fourth, we use alternative standard error corrections. Our main specification clusters at the state level. Using cluster-robust standard errors at the state-year level yields slightly smaller standard errors and stronger statistical significance. Using state-clustered wild bootstrap inference (following \citealt{CameronMiller2015}) yields marginally larger p-values but maintains significance at the 10\% level.

\subsection{Potential Mechanisms}

Our empirical results document that ULR laws improve employment outcomes for licensed occupation movers, but the data do not allow us to directly identify the mechanisms through which this effect operates. In this section, we discuss several potential channels and the evidence bearing on each.

The most direct mechanism is reduced re-licensure time. Before ULR adoption, workers moving to a new state had to navigate the re-licensure process, which could involve submitting documentation to the state licensing board, completing additional examinations or continuing education requirements, and waiting for application processing. \citet{Hermansen2022} documents that nurses face average processing times of 2-3 months for interstate license transfers in states without expedited recognition. ULR laws streamline this process by requiring only verification of good standing and a background check, which can typically be completed in days rather than months. The resulting reduction in unemployment duration would directly improve measured employment rates.

A second potential mechanism is improved job matching. When licensing barriers are high, workers may accept suboptimal job offers in their origin state rather than incurring the costs of relocation and re-licensure. ULR laws lower these costs, potentially allowing workers to pursue better matches across state lines. If ULR enables workers to find jobs more quickly after deciding to move, we would observe higher employment rates among movers even if the total number of movers does not change.

Third, ULR may increase the overall probability of interstate migration among licensed workers. If licensing barriers deter some workers from relocating altogether, ULR could expand the pool of movers to include workers who would not have moved under the prior regime. The composition of movers might change as well---workers with weaker labor market attachment might be more likely to move when barriers are lower, potentially dampening the observed employment rate effect even if total employment increases.

Our data cannot distinguish among these mechanisms because we observe employment status and migration history but not the counterfactual outcomes for workers who did not move. Future research using administrative data on license applications and timing could potentially identify the re-licensure time channel directly. Researchers could also examine whether ULR affects the total volume of licensed-occupation migration, which would speak to the extensive margin channel.

Despite the uncertainty about mechanisms, the reduced-form effect we document is policy-relevant regardless of the specific channel. Policymakers considering ULR adoption can be confident that the reform improves employment outcomes for the affected population, even if the precise pathways remain to be determined.

\subsection{Limitations and Caveats}

Our analysis faces several important limitations that warrant discussion. First, occupation codes in the ACS are an imperfect proxy for actual licensing status. Some workers coded as nurses or electricians may not hold licenses (e.g., nursing aides in unlicensed facilities), while some licensed workers may be classified in other occupation categories. This measurement error likely attenuates our estimates toward zero, suggesting the true effect may be larger than we observe.

Second, our comparison assumes parallel trends in the absence of treatment. While pre-trends appear similar in Figure \ref{fig:trends}, we have only one pre-treatment year (2019), limiting our ability to assess pre-trends formally. States that adopted ULR early may have been on different trajectories for reasons unrelated to licensing reform. We partially address this concern through the placebo test, which shows no differential trends for non-licensed workers.

Third, we observe employment status but not the mechanism through which ULR affects employment. Workers may benefit through faster job finding after relocation, better job matches, reduced periods of unemployment while awaiting license transfer, or increased probability of relocating in the first place. Our data cannot distinguish these channels. Future research using administrative data on license applications and employment transitions could illuminate the mechanisms.

Fourth, our estimates represent short-run effects of ULR adoption. As laws mature and workers learn about reduced licensing barriers, effects may grow or fade. With only 2--3 years of post-adoption data, we cannot assess long-run impacts. Additionally, general equilibrium effects---such as changes in the geographic distribution of licensed workers across states---may take time to materialize.

\section{Discussion and Conclusion}

This paper provides novel evidence on the labor market effects of universal occupational licensing recognition laws. Using a difference-in-differences design that exploits staggered adoption across states, we find that ULR laws increased employment rates among interstate movers in licensed occupations by approximately 0.6 percentage points. The effect is statistically significant at the 5\% level and robust to alternative specifications. A placebo test using non-licensed occupation movers shows no comparable effect, supporting the interpretation that the observed effect reflects licensing reform rather than other state-level differences.

Our findings have several implications for policy. Most directly, they suggest that occupational licensing creates meaningful barriers to interstate labor mobility, and that policy reforms reducing these barriers can improve employment outcomes for affected workers. The positive employment effects we document are consistent with the theoretical prediction that licensing barriers reduce job matching efficiency, and that reducing these barriers allows workers to find suitable employment more quickly after relocating.

At the same time, the modest magnitude of our estimates suggests that licensing is one of many factors affecting labor market outcomes for mobile workers. Our estimate of 0.6 percentage points represents less than 1\% of the baseline employment rate among licensed occupation movers. While meaningful for individual workers---particularly those who might otherwise face prolonged unemployment during re-licensure---ULR reforms are unlikely to dramatically transform interstate labor mobility patterns. Other factors such as housing costs, family ties, and job availability likely remain more important determinants of geographic mobility.

The heterogeneity in our results by occupation type has implications for understanding which licensing barriers are most binding. The finding that effects are largest for healthcare and trade workers---occupations with historically burdensome licensing requirements---suggests that reform efforts could productively focus on these sectors. In contrast, the null effect for education workers may reflect the prevalence of pre-existing reciprocity agreements for teaching credentials.

Several avenues for future research emerge from our analysis. First, researchers could examine additional outcomes beyond employment, including wages, job match quality, hours worked, and occupation switching. If ULR improves job matching, we might expect higher wages or more stable employment in addition to higher employment rates. Second, longer time horizons will allow assessment of dynamic effects. Do employment gains persist, grow, or fade over time? Do effects differ for early versus late adopters as the policy becomes more widely known? Third, researchers could investigate mechanisms using administrative data on license applications, processing times, and employment histories. Understanding how ULR affects the license transfer process could inform further policy refinements.

In conclusion, universal licensing recognition represents a promising policy tool for reducing barriers to interstate labor mobility. Our findings suggest that these laws improve employment outcomes for workers in licensed occupations who relocate across state lines. While effect sizes are modest, the policy costs of ULR are also low---states simply recognize out-of-state licenses rather than maintaining duplicative licensing requirements. From a cost-benefit perspective, ULR appears to be a worthwhile reform.

The rapid adoption of ULR laws across states reflects growing recognition among policymakers that licensing barriers impose real costs on workers and the broader economy. Our findings provide empirical support for this view. The 0.6 percentage point improvement in employment rates we document may seem small in isolation, but applied to the millions of licensed workers who move across state lines each year, the aggregate welfare gains are substantial. Moreover, our heterogeneity analysis suggests that effects are largest precisely where barriers were most binding---in healthcare and trades---validating the policy logic underlying ULR reforms.

Looking ahead, several policy questions remain. First, should ULR be expanded to cover professions currently exempted, such as physicians and attorneys? These professions face particularly stringent and variable licensing requirements across states, suggesting potentially larger effects from reform. However, they also involve higher stakes in terms of consumer protection, warranting careful consideration of the quality-mobility tradeoff. Second, should the federal government play a larger role in promoting licensing reciprocity? While ULR has spread organically across states, federal incentives or mandates could accelerate adoption and create a more uniform national labor market for licensed professionals.

As more states adopt ULR and longer time series become available, future research can provide more definitive assessments of their impacts on workers and labor markets. The staggered adoption pattern we exploit in this paper will continue to provide variation for causal identification, and the accumulating evidence can inform ongoing policy debates about occupational licensing reform. Our findings represent an early contribution to this literature, demonstrating that well-designed reforms can improve labor market outcomes while maintaining the consumer protection rationale for licensing.

\newpage

\bibliographystyle{aer}
\begin{thebibliography}{99}

\bibitem[Akerlof(1970)]{Akerlof1970}
Akerlof, G. A. (1970). The Market for ``Lemons'': Quality Uncertainty and the Market Mechanism. \textit{Quarterly Journal of Economics}, 84(3), 488--500.

\bibitem[Autor(2017)]{Autor2017}
Autor, D. H. (2017). Why Are There Still So Many Jobs? The History and Future of Workplace Automation. \textit{Journal of Economic Perspectives}, 29(3), 3--30.

\bibitem[Baker et al.(2025)]{BakerEtAl2025}
Baker, A., Callaway, B., Cunningham, S., Goodman-Bacon, A., \& Sant'Anna, P. H. C. (2025). Difference-in-Differences Designs: A Practitioner's Guide. \textit{arXiv:2503.13323}.

\bibitem[Blair \& Chung(2019)]{Blair2019}
Blair, P. Q., \& Chung, B. W. (2019). How Much of Barrier to Entry is Occupational Licensing? \textit{British Journal of Industrial Relations}, 57(4), 919--943.

\bibitem[Callaway \& Sant'Anna(2021)]{CallawayS2021}
Callaway, B., \& Sant'Anna, P. H. C. (2021). Difference-in-Differences with Multiple Time Periods. \textit{Journal of Econometrics}, 225(2), 200--230.

\bibitem[Cameron \& Miller(2015)]{CameronMiller2015}
Cameron, A. C., \& Miller, D. L. (2015). A Practitioner's Guide to Cluster-Robust Inference. \textit{Journal of Human Resources}, 50(2), 317--372.

\bibitem[Carpenter et al.(2017)]{Carpenter2017}
Carpenter, D. M., Knepper, L., Sweetland, K., \& McDonald, J. (2017). \textit{License to Work: A National Study of Burdens from Occupational Licensing}. Institute for Justice.

\bibitem[de Chaisemartin \& D'Haultf\oe uille(2020)]{deChaisemartinDHaultfoeuille2020}
de Chaisemartin, C., \& D'Haultf\oe uille, X. (2020). Two-Way Fixed Effects Estimators with Heterogeneous Treatment Effects. \textit{American Economic Review}, 110(9), 2964--2996.

\bibitem[Ganong \& Shoag(2017)]{Ganong2017}
Ganong, P., \& Shoag, D. (2017). Why Has Regional Income Convergence in the U.S. Declined? \textit{Journal of Urban Economics}, 102, 76--90.

\bibitem[Gittleman et al.(2018)]{Gittleman2018}
Gittleman, M., Klee, M. A., \& Kleiner, M. M. (2018). Analyzing the Labor Market Outcomes of Occupational Licensing. \textit{Industrial Relations}, 57(1), 57--100.

\bibitem[Goodman-Bacon(2021)]{GoodmanBacon2021}
Goodman-Bacon, A. (2021). Difference-in-Differences with Variation in Treatment Timing. \textit{Journal of Econometrics}, 225(2), 254--277.

\bibitem[Harrington \& Trachtenberg(2019)]{Harrington2019}
Harrington, D., \& Trachtenberg, J. (2019). \textit{Occupational Licensing and Military Spouses}. Cato Institute Policy Analysis.

\bibitem[Hermansen(2022)]{Hermansen2022}
Hermansen, M. (2022). Occupational Licensing and Job Mobility in the United States. \textit{American Economic Journal: Applied Economics}, forthcoming.

\bibitem[Hsieh \& Moretti(2019)]{Hsieh2019}
Hsieh, C.-T., \& Moretti, E. (2019). Housing Constraints and Spatial Misallocation. \textit{American Economic Journal: Macroeconomics}, 11(2), 1--39.

\bibitem[Johnson \& Kleiner(2020)]{JohnsonKleiner2020}
Johnson, J. E., \& Kleiner, M. M. (2020). Is Occupational Licensing a Barrier to Interstate Migration? \textit{American Economic Journal: Economic Policy}, 12(3), 347--373.

\bibitem[Kleiner(2006)]{Kleiner2006}
Kleiner, M. M. (2006). \textit{Licensing Occupations: Ensuring Quality or Restricting Competition?} W.E. Upjohn Institute for Employment Research.

\bibitem[Kleiner(2015)]{Kleiner2015}
Kleiner, M. M. (2015). \textit{Guild-Ridden Labor Markets: The Curious Case of Occupational Licensing}. W.E. Upjohn Institute for Employment Research.

\bibitem[Kleiner \& Krueger(2013)]{KleinerKrueger2013}
Kleiner, M. M., \& Krueger, A. B. (2013). Analyzing the Extent and Influence of Occupational Licensing on the Labor Market. \textit{Journal of Labor Economics}, 31(S1), S173--S202.

\bibitem[Law \& Kim(2005)]{Law2009}
Law, M. T., \& Kim, S. (2005). Specialization and Regulation: The Rise of Professionals and the Emergence of Occupational Licensing Regulation. \textit{Journal of Economic History}, 65(3), 723--756.

\bibitem[Roth et al.(2023)]{Roth2023}
Roth, J., Sant'Anna, P. H. C., Bilinski, A., \& Poe, J. (2023). What's Trending in Difference-in-Differences? A Synthesis of the Recent Econometrics Literature. \textit{Journal of Econometrics}, 235(2), 2218--2244.

\bibitem[Sun \& Abraham(2021)]{Sun2021}
Sun, L., \& Abraham, S. (2021). Estimating Dynamic Treatment Effects in Event Studies with Heterogeneous Treatment Effects. \textit{Journal of Econometrics}, 225(2), 175--199.

\bibitem[Borusyak et al.(2024)]{BorusyakEtAl2024}
Borusyak, K., Jaravel, X., \& Spiess, J. (2024). Revisiting Event Study Designs: Robust and Efficient Estimation. \textit{Review of Economic Studies}, forthcoming.

\bibitem[Shipan \& Volden(2008)]{Shipan2008}
Shipan, C. R., \& Volden, C. (2008). The Mechanisms of Policy Diffusion. \textit{American Journal of Political Science}, 52(4), 840--857.

\bibitem[Thornton \& Timmons(2013)]{Thornton2013}
Thornton, R. J., \& Timmons, E. J. (2013). Licensing One of the World's Oldest Professions: Massage. \textit{Journal of Law and Economics}, 56(2), 371--388.

\end{thebibliography}

\newpage
\appendix
\section{Appendix: Additional Results}

\subsection{State Adoption Timeline}

Table \ref{tab:adoption} lists the adoption year for each state in our sample.

\begin{table}[htbp]
\centering
\caption{Universal Licensing Recognition Law Adoption}
\label{tab:adoption}
\begin{tabular}{ll}
\toprule
Year & States \\
\midrule
2019 & Arizona \\
2020 & Montana \\
2021 & Pennsylvania, Utah, Iowa, Minnesota \\
2022 & Colorado, Florida, Ohio, Virginia, Idaho, Missouri \\
\midrule
Never & California, Illinois, New York, Texas, Michigan, \\
& Washington, Massachusetts, New Jersey, Oregon, Tennessee \\
\bottomrule
\end{tabular}
\end{table}

\subsection{Event Study Estimates}

Figure \ref{fig:eventstudy} presents event study estimates showing the dynamic treatment effects relative to adoption year. The pre-treatment coefficients (event time $< 0$) are close to zero, supporting the parallel trends assumption. The post-treatment coefficients show gradual increases in the treatment effect over time.

\begin{figure}[htbp]
\centering
\includegraphics[width=0.8\textwidth]{figures/event_study.png}
\caption{Event Study: Dynamic Treatment Effects}
\label{fig:eventstudy}
\vspace{0.5em}
\noindent\small\textit{Notes: Figure shows DiD coefficients by event time relative to ULR adoption year. Event time $t-1$ (one year before adoption) is the omitted category. Vertical bars represent 95\% confidence intervals based on state-clustered standard errors. Pre-treatment coefficients near zero support the parallel trends assumption.}
\end{figure}

\subsection{Replication Information}

Data: American Community Survey PUMS 2019, 2021, 2022 (1-year files).

API: \url{https://api.census.gov/data/[YEAR]/acs/acs1/pums}

All code and data for replication are available at: \url{https://github.com/apep/papers}

\vspace{2em}

\noindent\textbf{Contributor:} Claude Opus 4.5

\end{document}
