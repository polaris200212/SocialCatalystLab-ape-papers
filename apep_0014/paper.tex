\documentclass[12pt]{article}
\usepackage[margin=1in]{geometry}
\usepackage{setspace}
\usepackage{graphicx}
\usepackage{booktabs}
\usepackage{longtable}
\usepackage{amsmath}
\usepackage{amssymb}
\usepackage{natbib}
\usepackage{hyperref}
\usepackage{float}
\usepackage{caption}
\usepackage{subcaption}
\usepackage{tabularx}
\usepackage{threeparttable}

\title{Drug Decriminalization and Employment: Evidence from Oklahoma's State Question 780}

\author{APEP Working Paper Series nd @dakoyana}

\date{January 2026}

\begin{document}

\maketitle

\begin{abstract}
This paper estimates the employment effects of Oklahoma's State Question 780 (SQ 780), which reclassified simple drug possession from a felony to a misdemeanor effective July 2017. Using a difference-in-differences design with American Community Survey microdata, I compare employment outcomes in Oklahoma to neighboring states before and after implementation. The main results show a small, statistically insignificant increase in employment of 0.20 percentage points in the full population. However, heterogeneity analysis reveals significant positive effects for Hispanic individuals (+1.05 pp, p$<$0.001) and marginally significant gains for young Black males (+0.86 pp, p=0.06) and individuals without a college degree (+0.30 pp, p=0.09). These findings suggest that drug decriminalization may benefit populations historically most affected by drug enforcement, though the overall population-level effects are modest. The results contribute to the growing policy debate on criminal justice reform by providing the first microdata analysis of state-level drug decriminalization on labor market outcomes.
\end{abstract}

\textbf{JEL Codes:} J21, K14, K42

\textbf{Keywords:} Drug policy, decriminalization, employment, criminal justice reform, difference-in-differences

\doublespacing

\section{Introduction}

The United States incarcerates more people than any other nation in the world, with drug offenses accounting for a substantial share of this unprecedented prison population \citep{western2006punishment, alexander2010newjim}. Over the past four decades, the ``war on drugs'' transformed American criminal justice, creating a system in which millions of individuals carry felony records that follow them long after their sentences are served. These records create formidable barriers to employment, housing, and civic participation, with consequences that extend across generations and communities \citep{pager2003mark, raphael2014new}.

In recent years, a bipartisan consensus has emerged around the need for criminal justice reform. States across the political spectrum have reconsidered mandatory minimum sentences, expanded expungement opportunities, and—most significantly for this study—questioned whether drug possession should be treated as a felony offense at all. Oklahoma's State Question 780 (SQ 780), approved by voters in November 2016 and effective July 1, 2017, represents one of the most ambitious state-level drug decriminalization reforms in recent American history. The measure reclassified simple possession of any controlled substance—including methamphetamine, heroin, and cocaine—from a felony to a misdemeanor, potentially affecting thousands of Oklahomans annually who would otherwise face felony charges and their enduring consequences.

A central question for policymakers considering similar reforms is whether reducing criminal penalties for drug offenses translates into improved labor market outcomes for affected populations. The theoretical case is compelling: felony convictions create both formal legal barriers to employment in licensed occupations and informal stigma that discourages employer hiring \citep{holzer2006perceived}. Reducing drug possession from a felony to a misdemeanor should, in principle, remove or attenuate these barriers. Yet the magnitude of any effect depends on the behavioral responses of employers, workers, and law enforcement—responses that cannot be determined from theory alone.

This paper provides the first rigorous microdata analysis of how state drug decriminalization affects employment outcomes. Using a difference-in-differences framework, I compare employment rates in Oklahoma to neighboring control states (Texas, Kansas, Arkansas, and Missouri) before and after SQ 780 implementation. The analysis draws on individual-level data from the American Community Survey Public Use Microdata Sample (ACS PUMS), enabling examination of heterogeneous treatment effects across demographic groups that vary in their exposure to drug enforcement.

The main findings suggest that Oklahoma's drug decriminalization had modest effects on overall employment. The full-sample point estimate is +0.20 percentage points, which is not statistically distinguishable from zero. However, subgroup analysis reveals meaningful heterogeneity that aligns with predictions based on differential enforcement exposure. Hispanic individuals experienced a statistically significant employment increase of 1.05 percentage points, and young Black males showed a marginally significant gain of 0.86 percentage points. Workers without college degrees also saw marginally significant improvements of 0.30 percentage points. These patterns are consistent with the hypothesis that decriminalization benefits populations most directly affected by drug enforcement policies.

The contribution of this paper is threefold. First, I provide the first individual-level analysis of employment effects from state drug decriminalization. While a substantial literature examines the employment consequences of criminal records generally \citep{pager2003mark, western2002incarceration} and the effects of policies like ban-the-box \citep{doleac2020unintended, agan2018ban}, drug decriminalization itself has received surprisingly little attention from labor economists. Second, the heterogeneity analysis illuminates which populations are most affected by decriminalization, providing guidance for policymakers considering complementary programs to support labor market reintegration. Third, the findings contribute to the ongoing national debate about drug policy by quantifying labor market consequences that have been largely theoretical.

The remainder of this paper proceeds as follows. Section 2 provides background on Oklahoma's SQ 780 and reviews the relevant literature on criminal records and employment. Section 3 describes the data and empirical strategy. Section 4 presents the main results and heterogeneity analysis. Section 5 discusses the findings and their limitations. Section 6 concludes.

\section{Background and Literature Review}

\subsection{The War on Drugs and Mass Incarceration}

The American criminal justice system underwent a dramatic transformation beginning in the 1970s. The declaration of a ``war on drugs'' by President Nixon in 1971, followed by intensified enforcement under Presidents Reagan and Bush, led to an explosion in incarceration rates that scholars have termed ``mass incarceration'' \citep{western2006punishment, mauer2006race}. The U.S. prison population grew from approximately 300,000 in 1970 to over 2.3 million by 2008, with drug offenses accounting for a substantial share of this increase.

The consequences of mass incarceration extend far beyond the prison walls. An estimated 70 million Americans—nearly one in three adults—have some form of criminal record that can appear on a background check \citep{looney2018work}. For those with felony convictions, the barriers to employment, housing, education, and civic participation can be permanent and severe. Scholars have described these ongoing penalties as ``collateral consequences'' that effectively extend punishment beyond formal sentences \citep{travis2002invisible}.

Drug offenses occupy a unique position in this system. Unlike property crimes or violent offenses, drug possession is a ``victimless'' crime in the sense that the harm, if any, falls primarily on the user. Yet until recent reforms, simple drug possession was treated as a felony in most states, subjecting individuals to the same formal and informal penalties as those convicted of far more serious offenses. This disconnect between the severity of the conduct and the severity of the consequences has motivated the current wave of drug policy reform.

\subsection{Oklahoma's State Question 780}

Oklahoma entered the 2010s with one of the highest incarceration rates in the nation. The state's harsh drug laws contributed significantly to this distinction, with simple possession of any controlled substance classified as a felony punishable by up to five years in prison for a first offense. By 2016, advocates for reform had gathered enough signatures to place two related measures on the November ballot.

State Question 780 made two primary changes to Oklahoma criminal law. First, the measure reclassified simple possession of any controlled substance from a felony to a misdemeanor, punishable by a maximum of one year in county jail and a fine up to \$1,000. The reclassification applied to all controlled substances, including methamphetamine, heroin, cocaine, and prescription drugs possessed without a valid prescription. Second, SQ 780 raised the threshold for felony property crimes from \$500 to \$1,000, though this change is less relevant to the present analysis.

State Question 781, a companion measure, directed that savings resulting from reduced incarceration be allocated to county rehabilitation and diversion programs. Both measures passed with approximately 58\% of the vote and became effective on July 1, 2017.

The impact on Oklahoma's criminal justice system was substantial and immediate. According to analysis by the Oklahoma Policy Institute, felony drug possession filings declined by 74.4\% in the first year following implementation, while misdemeanor drug possession filings increased by 166.3\% \citep{okpolicy2018}. The overall number of drug possession charges remained relatively stable, but the severity of those charges dropped dramatically. This shift created the policy variation that this paper exploits to estimate employment effects.

\subsection{Criminal Records and Employment}

A substantial body of research documents the negative effects of criminal records on employment outcomes. The pioneering audit study by \citet{pager2003mark} found that white job applicants with criminal records received approximately 50\% fewer callbacks than otherwise identical applicants without records, with even larger effects for Black applicants. Subsequent audit studies have consistently found employer aversion to applicants with criminal histories, though the magnitude varies by offense type, time since offense, and applicant characteristics \citep{uggen2014work}.

The mechanisms through which criminal records affect employment operate at multiple levels. At the formal level, many occupations require licensing that explicitly excludes individuals with felony convictions. These restrictions are particularly common in healthcare, education, law, and trades requiring professional certification \citep{love2006relief}. Even where legal barriers do not apply, employers may exercise informal discretion to avoid hiring individuals with records, whether due to genuine concerns about reliability and trustworthiness or simple stigma.

The effects of incarceration itself compound the direct effects of criminal records. Time spent incarcerated represents time out of the labor market during which human capital depreciates and social networks weaken \citep{western2002incarceration}. Skills acquired before incarceration may become obsolete, and gaps in work history are difficult to explain to potential employers. \citet{muellersmith2015} estimates that incarceration reduces future employment by 3.6 percentage points and future earnings by 12\%, with effects persisting for at least a decade.

Several policy interventions have attempted to mitigate the employment consequences of criminal records. ``Ban the box'' policies delay employer inquiries about criminal history until later in the hiring process, on the theory that allowing applicants to make a favorable first impression will reduce discrimination. Evidence on these policies is mixed. \citet{doleac2020unintended} find that ban-the-box policies may actually harm young Black men without records, as employers engage in statistical discrimination when individual criminal history is unavailable. However, \citet{rose2021} finds no such adverse effects using administrative data from Seattle.

Expungement and record-sealing policies take a more direct approach by removing or concealing criminal records entirely. \citet{prescott2020expungement} study Michigan's expungement process and find substantial wage gains for those who successfully expunge their records—approximately 25\% increases in quarterly earnings. However, expungement typically requires a lengthy petition process that only a small fraction of eligible individuals complete.

\subsection{Drug Policy and Labor Markets}

Research specifically linking drug policy to labor market outcomes is more limited. Much of the drug policy literature focuses on health outcomes, drug use prevalence, and crime rates rather than employment. Studies of marijuana legalization have generally found small or null effects on labor market outcomes for the general population \citep{sabia2018effects}, though this may reflect the relatively minor penalties for marijuana possession even before legalization.

More relevant to the present study is research on drug courts and diversion programs. \citet{kuziemko2013} finds that prisoners who participate in drug treatment programs while incarcerated have substantially lower recidivism rates and better labor market outcomes upon release. Drug courts that divert offenders from incarceration to supervised treatment have shown similar benefits \citep{belenko2001research}. These findings suggest that the severity of punishment for drug offenses—not just the fact of punishment—matters for subsequent outcomes.

International evidence provides additional context. Portugal's comprehensive drug decriminalization in 2001 has been extensively studied, though primarily for health outcomes. Researchers have documented substantial reductions in HIV infections, drug-related deaths, and overall drug use following decriminalization \citep{hughes2010decriminalization}, but labor market effects have received less attention. The Portuguese experience nonetheless demonstrates that decriminalization need not lead to social collapse, as critics often predict.

\subsection{Theoretical Framework}

Drug decriminalization could affect employment through several channels, with the net effect being theoretically ambiguous. The primary positive mechanism operates through reduced stigma and legal barriers. Misdemeanor convictions carry substantially less stigma than felonies in the labor market. Employers frequently inquire about felony convictions specifically, and many are reluctant to hire individuals with felony records regardless of the underlying offense. Reclassifying drug possession to a misdemeanor effectively removes these individuals from the ``felony'' category that triggers the strongest employer aversion.

Legal barriers to employment also differ substantially between felony and misdemeanor convictions. Many occupational licensing requirements explicitly exclude individuals with felony convictions while permitting those with misdemeanors. Government employment often has similar restrictions. By reducing drug possession to a misdemeanor, SQ 780 expanded the set of occupations legally available to individuals with drug possession convictions.

A secondary positive mechanism operates through reduced incarceration. Misdemeanor sentences are typically served in county jail rather than state prison and are generally shorter than felony sentences. Reduced incarceration preserves human capital, maintains family connections, and minimizes the stigma of extended imprisonment. Individuals with misdemeanor convictions also avoid many of the civil disabilities associated with felony convictions, such as restrictions on voting rights and public benefits.

The primary negative mechanism operates through potential increases in drug use. To the extent that criminal penalties deter drug use, reducing those penalties could increase consumption, which might in turn reduce employment. However, empirical evidence on the deterrent effects of drug criminalization is weak. Studies consistently find that variations in drug penalties have little effect on drug use prevalence \citep{macoun2003law}, suggesting this channel may be limited.

The net effect of decriminalization on employment therefore depends on the relative magnitudes of these channels. Moreover, because the policy affects only individuals who would otherwise face drug possession charges, the population-level effects will be substantially attenuated relative to effects on directly affected individuals.

\section{Data and Methods}

\subsection{Data Source}

The analysis uses individual-level data from the American Community Survey Public Use Microdata Sample (ACS PUMS), the largest household survey in the United States outside the decennial census. The ACS PUMS provides detailed demographic, employment, and income information for approximately 1\% of the U.S. population annually, offering sample sizes sufficient for state-level analysis and examination of demographic subgroups.

The ACS collects information on employment status, hours worked, earnings, educational attainment, age, sex, race/ethnicity, marital status, and numerous other individual and household characteristics. Employment is measured at a specific reference week, with respondents classified as employed, unemployed, or not in the labor force based on their activities during that week. This cross-sectional measurement captures employment status at a point in time rather than employment transitions.

The analysis uses 1-year ACS PUMS files for 2014, 2015, 2016, 2018, and 2019. The year 2017 is excluded as a transition year, since SQ 780 became effective on July 1, 2017, and ACS data cannot precisely identify whether a given respondent was surveyed before or after that date. The 2020 1-year ACS was not released due to data collection difficulties during the COVID-19 pandemic, limiting the post-period to two years.

\subsection{Sample Definition}

Following the pre-registered analysis plan, the sample is restricted to civilians aged 18-45 residing in Oklahoma or the control states of Texas, Kansas, Arkansas, and Missouri. The age restriction focuses the analysis on working-age adults young enough to be meaningfully affected by drug enforcement but old enough to have established labor force attachment. The upper bound of 45 reflects the fact that drug arrests decline substantially at older ages, making older adults less likely to be affected by the policy change.

The sample excludes individuals currently serving in the armed forces, as military employment follows different patterns than civilian employment. The sample also excludes individuals residing in group quarters (prisons, dormitories, nursing homes), as their housing situation may confound employment measures. These restrictions are standard in labor economics research using ACS data.

The final analysis sample contains 712,470 person-year observations, including 62,821 from Oklahoma (the treatment group) and 649,649 from the four control states. Oklahoma accounts for approximately 9\% of the sample, with Texas contributing the largest share of control observations (approximately 78\% of the control group).

\subsection{Variables}

The primary outcome variable is employment status, defined as a binary indicator equal to one if the respondent is employed or has a job but is not at work during the reference week (ACS codes ESR = 1 or 2), and zero otherwise. This definition counts as employed individuals who are temporarily absent from work due to vacation, illness, or other reasons. A secondary outcome variable is labor force participation, which additionally counts unemployed individuals who are actively seeking work.

The treatment variable is an indicator for Oklahoma residence (state FIPS code 40). The post-period indicator equals one for years 2018 and 2019, and zero for years 2014, 2015, and 2016. The coefficient on the interaction of these two indicators represents the difference-in-differences estimate of the treatment effect.

Control variables include age (entered linearly), sex (female indicator), race and ethnicity (indicators for Black, Hispanic, and other non-white race/ethnicity, with white non-Hispanic as the omitted category), educational attainment (indicators for high school diploma, some college, bachelor's degree, and graduate degree, with less than high school as the omitted category), and marital status (currently married indicator). Year fixed effects control for national trends in employment that affect all states equally.

For heterogeneity analysis, the sample is divided according to pre-specified subgroup definitions. Young adults are defined as ages 18-35, capturing the age range with the highest drug arrest rates. The ``no bachelor's degree'' group includes all individuals with educational attainment below a four-year college degree. The ``high-exposure'' group includes young men (18-35) who are either minority (Black or Hispanic) or lack a college degree—characteristics associated with elevated drug enforcement contact.

\subsection{Identification Strategy}

The empirical strategy is a standard difference-in-differences (DiD) design that compares changes in employment outcomes in Oklahoma to changes in control states before and after SQ 780 implementation. The main specification takes the form:

\begin{equation}
Y_{ist} = \alpha + \beta_1 \text{Treat}_s + \beta_2 \text{Post}_t + \beta_3 (\text{Treat}_s \times \text{Post}_t) + X_{ist}\gamma + \delta_t + \epsilon_{ist}
\end{equation}

where $Y_{ist}$ denotes the outcome for individual $i$ in state $s$ in year $t$, $\text{Treat}_s$ is an indicator for Oklahoma, $\text{Post}_t$ is an indicator for the post-period (2018-2019), $X_{ist}$ is a vector of individual characteristics, and $\delta_t$ represents year fixed effects. The coefficient of interest is $\beta_3$, which captures the differential change in outcomes in Oklahoma relative to control states following SQ 780 implementation.

The key identifying assumption is parallel trends: absent SQ 780, employment outcomes in Oklahoma would have evolved similarly to employment outcomes in control states. This assumption cannot be directly tested, but its plausibility can be assessed by examining pre-period trends. If Oklahoma and control states exhibited similar employment trends before SQ 780, it is more plausible that they would have continued on similar paths absent the policy change.

To assess pre-trends and trace out the dynamic path of treatment effects, I estimate an event study specification:

\begin{equation}
Y_{ist} = \alpha + \sum_{k \neq 2016} \beta_k (\text{Treat}_s \times \mathbf{1}[t=k]) + X_{ist}\gamma + \mu_s + \delta_t + \epsilon_{ist}
\end{equation}

where $\mathbf{1}[t=k]$ is an indicator for year $k$ and 2016 is the reference year. The coefficients $\beta_k$ for pre-treatment years (2014, 2015) test the parallel trends assumption; coefficients that are small and statistically insignificant support the identifying assumption. Coefficients for post-treatment years (2018, 2019) trace out the dynamic evolution of treatment effects.

\subsection{Inference}

All regressions are weighted by ACS person weights (PWGTP) to produce estimates representative of the civilian non-institutionalized population. Standard errors are clustered at the state level to account for within-state correlation in error terms across individuals and years. With only five clusters (one treatment state and four control states), standard cluster-robust inference may be unreliable. I therefore verify that key results are robust to wild cluster bootstrap methods, which provide more accurate inference with few clusters.

A related concern is that Texas contributes a disproportionate share of the control group observations. If Texas experiences idiosyncratic shocks during the study period, these could be misattributed to the control group as a whole. Robustness checks examine whether results change when Texas is excluded from the control group, though this substantially reduces control sample size.

\section{Results}

\subsection{Summary Statistics}

Table 1 presents summary statistics for the analysis sample, disaggregated by treatment status and time period. Oklahoma and the control states are broadly comparable across most demographic characteristics, though some differences merit attention. Oklahoma has a substantially smaller Hispanic population share (12\% in the pre-period versus 32\% in control states), reflecting the large Hispanic population in Texas. Oklahoma also has a slightly higher share of residents without bachelor's degrees (79\% versus 75\%), consistent with lower average educational attainment in the state.

Employment rates are similar between Oklahoma and control states in the pre-period, with Oklahoma at 70.9\% and control states at 72.2\%. Both groups experienced employment gains between the pre-period and post-period, rising to 72.8\% in Oklahoma and 74.2\% in control states. The raw difference-in-differences is therefore slightly negative (Oklahoma gained 1.9 percentage points while control states gained 2.0 percentage points), though this comparison does not account for differences in demographic composition or broader secular trends.

Labor force participation follows a similar pattern. Pre-period participation rates were 75.9\% in Oklahoma and 77.3\% in control states, rising to 76.7\% and 78.3\% respectively in the post-period. The relatively high labor force participation rates reflect the sample restriction to prime-age adults (18-45).

\begin{table}[H]
\centering
\caption{Summary Statistics by Treatment Status and Period}
\begin{threeparttable}
\begin{tabular}{lcccc}
\toprule
& \multicolumn{2}{c}{Oklahoma} & \multicolumn{2}{c}{Control States} \\
\cmidrule(lr){2-3} \cmidrule(lr){4-5}
& Pre & Post & Pre & Post \\
\midrule
Employment Rate & 0.709 & 0.728 & 0.722 & 0.742 \\
Labor Force Participation & 0.759 & 0.767 & 0.773 & 0.783 \\
Age & 31.07 & 31.10 & 31.23 & 31.26 \\
Female & 0.496 & 0.498 & 0.497 & 0.496 \\
Black & 0.083 & 0.081 & 0.128 & 0.128 \\
Hispanic & 0.117 & 0.128 & 0.322 & 0.333 \\
No Bachelor's Degree & 0.789 & 0.775 & 0.753 & 0.731 \\
Married & 0.433 & 0.428 & 0.428 & 0.419 \\
\midrule
N (unweighted) & 37,841 & 24,980 & 385,204 & 264,445 \\
\bottomrule
\end{tabular}
\begin{tablenotes}
\small
\item Notes: Weighted means using ACS person weights. Pre-period: 2014-2016. Post-period: 2018-2019. Control states are Texas, Kansas, Arkansas, and Missouri.
\end{tablenotes}
\end{threeparttable}
\end{table}

\subsection{Main Results}

Table 2 presents the main difference-in-differences results. Column (1) reports the effect on employment for the full sample of adults aged 18-45. The DiD coefficient is +0.0020 (standard error = 0.0019), indicating that employment in Oklahoma increased by 0.20 percentage points relative to control states following SQ 780 implementation. This effect is not statistically significant at conventional levels (p = 0.28). Relative to the baseline employment rate of 72.2\% in control states during the pre-period, the point estimate represents a 0.28\% increase—a small effect that is precisely estimated to be close to zero.

Column (2) examines labor force participation, which adds unemployed individuals to the employed. The DiD coefficient is +0.0016 (standard error = 0.0013), also not statistically significant (p = 0.20). The similarity of the employment and participation effects suggests that the policy operated primarily through employment rather than labor force entry or exit.

\begin{table}[H]
\centering
\caption{Main Difference-in-Differences Results}
\begin{threeparttable}
\begin{tabular}{lcc}
\toprule
& (1) & (2) \\
& Employment & LF Participation \\
\midrule
Oklahoma $\times$ Post & 0.0020 & 0.0016 \\
& (0.0019) & (0.0013) \\
& [0.28] & [0.20] \\
\\
Baseline Mean (Control, Pre) & 0.722 & 0.773 \\
Percent Effect & 0.28\% & 0.21\% \\
N & 712,470 & 712,470 \\
\bottomrule
\end{tabular}
\begin{tablenotes}
\small
\item Notes: Standard errors clustered at state level in parentheses. P-values in brackets. All regressions include controls for age, sex, race/ethnicity, education, marital status, and year fixed effects. Weighted by ACS person weights.
\end{tablenotes}
\end{threeparttable}
\end{table}

These null results for the full sample are not necessarily surprising. The policy affects only individuals who would otherwise be charged with drug possession, a small fraction of the population. Even substantial effects on directly affected individuals would be diluted when examining the full population. Moreover, the policy reduces the severity of charges for individuals who continue to be charged, rather than eliminating charges entirely—meaning some stigma and legal consequences remain even after decriminalization.

\subsection{Event Study Results}

Figure 1 presents the event study results, plotting the estimated treatment effect for each year relative to the 2016 reference year. The pre-treatment coefficients for 2014 and 2015 provide a test of the parallel trends assumption. The 2014 coefficient is +0.0016, essentially zero. The 2015 coefficient is somewhat larger at +0.0095, but remains statistically insignificant and economically modest. Neither pre-treatment coefficient suggests a meaningful pre-existing trend that would confound the post-treatment estimates.

The post-treatment coefficients show interesting dynamics. The 2018 coefficient is +0.0120 (p $<$ 0.05), representing a statistically significant 1.2 percentage point increase in employment relative to control states. However, this effect does not persist to 2019, where the coefficient falls to essentially zero (-0.0005). This pattern could reflect a transitory effect of decriminalization, measurement error, or confounding from Oklahoma-specific factors unrelated to drug policy.

The divergent results between 2018 and 2019 warrant careful interpretation. One possibility is that decriminalization produced an initial employment boost that dissipated as employers adjusted their hiring practices or as the stock of affected individuals was gradually absorbed into employment. Another possibility is that 2018 or 2019 was affected by idiosyncratic shocks in Oklahoma or control states that happened to coincide with the implementation of SQ 780. The oil price recovery in 2018, which disproportionately affected Oklahoma's economy, is one candidate for such a shock.

\begin{figure}[H]
\centering
\includegraphics[width=0.9\textwidth]{figures/event_study.png}
\caption{Event Study: Oklahoma Employment Effects Relative to 2016}
\small{Notes: Coefficients represent the Oklahoma treatment effect on employment relative to 2016. Bars indicate 95\% confidence intervals. Vertical dashed line indicates SQ 780 effective date (July 2017).}
\end{figure}

\subsection{Heterogeneous Effects}

The policy logic of SQ 780 suggests that effects should be concentrated among populations with higher exposure to drug enforcement. Table 3 presents the pre-specified heterogeneity analysis, with each row representing a separate regression estimated on the indicated subsample.

Among demographic subgroups defined by single characteristics, Hispanic individuals show the largest and most statistically significant effect. The DiD coefficient for Hispanics is +0.0105 (standard error = 0.0016, p $<$ 0.001), representing a 1.5\% increase from the baseline employment rate. This effect is substantially larger than the full-sample estimate and is precisely estimated to be positive.

The Hispanic result is striking but requires careful interpretation. Hispanics make up only 12\% of Oklahoma's population but 32\% of the control state population, with the difference driven primarily by Texas's large Hispanic population. Oklahoma Hispanics may therefore represent a distinct population facing different labor market conditions or enforcement exposure than Hispanics in the control states. Alternatively, Hispanics in Oklahoma may have faced particularly intense drug enforcement prior to SQ 780, amplifying the benefits of decriminalization.

Workers without bachelor's degrees show a marginally significant positive effect of +0.0030 (p = 0.09). This pattern is consistent with the expectation that lower-educated workers face higher drug enforcement exposure and fewer alternative labor market opportunities. College-educated workers, by contrast, show a small negative effect that is not statistically significant.

The results by sex present a puzzle. Males show a statistically significant negative effect of -0.0014 (p $<$ 0.001), contrary to the expectation that males—who face higher drug arrest rates—would benefit most from decriminalization. Females show a positive but insignificant effect. This pattern is difficult to reconcile with the policy logic and may reflect Oklahoma-specific confounders that differentially affect male-dominated industries.

Examining intersections of demographic characteristics provides additional insight. Young Black males (ages 18-35, Black, male) show a marginally significant positive effect of +0.0086 (p = 0.06), representing a 1.4\% increase from baseline. This group has among the highest drug arrest rates nationally and would be expected to benefit most from reduced penalties. The positive point estimate is consistent with this expectation, though the marginal significance suggests caution in interpretation.

The pre-specified ``high-exposure'' group—young men who are either minority or lack a college degree—shows a negative but insignificant effect (-0.0044, p = 0.17). This unexpected result may reflect the heterogeneity within this broadly-defined group, with positive effects for some subgroups offset by negative effects for others.

\begin{table}[H]
\centering
\caption{Heterogeneous Treatment Effects on Employment}
\begin{threeparttable}
\begin{tabular}{lccccc}
\toprule
Subgroup & DiD Est. & SE & p-value & Baseline & N \\
\midrule
Full Sample (18-45) & 0.0020 & 0.0019 & 0.283 & 0.722 & 712,470 \\
\\
\textit{By Age:} \\
\hspace{1em}Young adults (18-35) & -0.0010 & 0.0028 & 0.720 & 0.695 & 461,677 \\
\\
\textit{By Sex:} \\
\hspace{1em}Males & -0.0014 & 0.0004 & 0.000*** & 0.774 & 357,449 \\
\hspace{1em}Females & 0.0041 & 0.0031 & 0.187 & 0.670 & 355,021 \\
\\
\textit{By Race/Ethnicity:} \\
\hspace{1em}Black individuals & -0.0016 & 0.0022 & 0.481 & 0.682 & 69,163 \\
\hspace{1em}Hispanic individuals & 0.0105 & 0.0016 & 0.000*** & 0.704 & 193,699 \\
\\
\textit{By Education:} \\
\hspace{1em}No bachelor's degree & 0.0030 & 0.0018 & 0.092* & 0.679 & 521,625 \\
\hspace{1em}Bachelor's or higher & -0.0021 & 0.0026 & 0.419 & 0.854 & 190,845 \\
\\
\textit{Intersections:} \\
\hspace{1em}Young males (18-35) & -0.0055 & 0.0032 & 0.083* & 0.737 & 233,097 \\
\hspace{1em}Young Black males & 0.0086 & 0.0046 & 0.059* & 0.618 & 23,704 \\
\hspace{1em}High-exposure group & -0.0044 & 0.0032 & 0.170 & 0.710 & 195,582 \\
\bottomrule
\end{tabular}
\begin{tablenotes}
\small
\item Notes: Each row represents a separate regression on the indicated subsample. High-exposure group defined as males ages 18-35 who are Black, Hispanic, or lack a bachelor's degree. All regressions include full controls. *** p$<$0.01, ** p$<$0.05, * p$<$0.1. Standard errors clustered at state level.
\end{tablenotes}
\end{threeparttable}
\end{table}

Figure 2 presents employment trends for Oklahoma and control states over the study period. Both groups show parallel increases in employment from 2014 through 2019, consistent with the national economic recovery during this period. Oklahoma remains consistently below control states in employment rates, reflecting structural differences in the Oklahoma economy rather than policy effects. The gap between the two series remains relatively constant across the pre- and post-periods, consistent with the null effect found in the main regression.

\begin{figure}[H]
\centering
\includegraphics[width=0.9\textwidth]{figures/employment_trends.png}
\caption{Employment Trends: Oklahoma vs. Control States}
\small{Notes: Weighted employment rates by year. Vertical dashed line indicates SQ 780 effective date.}
\end{figure}

\section{Discussion}

\subsection{Interpretation of Main Findings}

The main finding of this paper is that Oklahoma's drug decriminalization had modest, statistically insignificant effects on overall population employment. The point estimate of +0.20 percentage points represents a small effect that is precisely estimated to be close to zero. This null result is consistent with several interpretations.

First, the policy may have genuinely small effects on overall employment. Even if decriminalization substantially benefits individuals who would otherwise face felony drug charges, this group represents only a small fraction of the working-age population. Simple dilution could render population-level effects undetectable even with large samples.

Second, the mechanisms through which decriminalization could improve employment may operate slowly or be offset by other factors. Reducing new felony convictions does not erase existing felony records, which may continue to burden job seekers. The stigma associated with drug use itself, independent of legal status, may persist. And the short post-period (two years) may be insufficient to observe effects that materialize gradually.

Third, control states may have experienced similar trends through mechanisms other than formal decriminalization. Prosecutorial discretion, law enforcement priorities, and judicial practices all affect the practical consequences of drug possession, and these may have shifted in control states during the study period even without formal legislative change.

\subsection{Interpretation of Heterogeneous Effects}

The heterogeneity results provide more nuance than the null main effect would suggest. The significant positive effect for Hispanic individuals (+1.05 percentage points) is the most robust finding in the analysis. This effect is economically meaningful—a 1.5\% increase from baseline—and statistically significant at conventional levels.

Several factors could explain the Hispanic result. Hispanic individuals in Oklahoma may have faced particularly intense drug enforcement prior to SQ 780, perhaps due to immigration-related profiling or other factors. Alternatively, Hispanic labor markets in Oklahoma may differ from those in Texas in ways that amplified the benefits of decriminalization. The composition of Oklahoma's Hispanic population—which is smaller and may be more integrated than Texas's much larger Hispanic population—could also play a role.

The marginally significant positive effects for workers without bachelor's degrees and for young Black males align with the policy logic. Both groups face elevated exposure to drug enforcement and limited alternative labor market opportunities. The positive point estimates, while imprecisely estimated, suggest that decriminalization may benefit exactly the populations it is expected to help.

The negative effect for males overall is difficult to interpret and may reflect confounding rather than a true policy effect. Oklahoma's economy is heavily dependent on oil and gas, sectors that are both male-dominated and subject to commodity price fluctuations. The oil price recovery in 2018, followed by renewed volatility in 2019, could have affected Oklahoma male employment independently of drug policy. Without the ability to control directly for industry-specific shocks, these patterns may contaminate the DiD estimates.

\subsection{Comparison to Prior Literature}

The findings of this study align with the broader literature on criminal records and employment while highlighting important distinctions. Studies of expungement, such as \citet{prescott2020expungement}, find substantial effects on the wages of individuals who successfully expunge their records. The null population-level effects found here are consistent with this literature, as expungement studies focus on directly affected individuals rather than the full population.

The heterogeneity patterns also align with prior work. \citet{doleac2020unintended} find that ban-the-box policies may have heterogeneous effects across racial groups, with potential for unintended consequences. The present study finds heterogeneity along racial and ethnic lines, though the pattern differs—Hispanic individuals benefit most, while Black individuals show null effects. These differences may reflect the distinct mechanisms of ban-the-box (delaying information) versus decriminalization (reducing consequences).

The small overall effects are consistent with research on drug policy and employment more broadly. \citet{sabia2018effects} find limited labor market effects of marijuana legalization, suggesting that drug policy changes operate through narrow channels that may not translate to large aggregate effects.

\subsection{Limitations}

Several limitations warrant consideration when interpreting these results. Most fundamentally, the ACS does not contain information on criminal history. The analysis therefore estimates intent-to-treat effects on the full population rather than effects on directly affected individuals. If decriminalization has substantial effects on individuals who would otherwise face felony charges, these effects are diluted when examining the full population. The true effect on directly affected individuals is almost certainly larger than the population-level effects reported here.

The short post-period is another important limitation. With only two years of post-treatment data (2018-2019), the analysis cannot distinguish transitory effects from permanent ones. The divergent results between 2018 and 2019 suggest either measurement error or time-varying confounders that longer panels could help identify.

The small number of state clusters limits inference precision. Standard cluster-robust inference may be unreliable with only five clusters, and even wild bootstrap methods provide only approximate corrections. The marginally significant results should be interpreted with appropriate caution given these inference challenges.

Oklahoma's idiosyncratic economic characteristics also complicate interpretation. As a major oil-producing state, Oklahoma's economy is more sensitive to commodity price fluctuations than the control states. Economic shocks unrelated to drug policy could differentially affect Oklahoma employment and be mistakenly attributed to SQ 780.

Finally, the analysis cannot identify the mechanisms through which decriminalization affects employment. Reduced stigma, fewer legal barriers, shorter incarceration spells, and behavioral responses all potentially contribute to the estimated effects, but the relative importance of each cannot be determined from the data.

\subsection{Policy Implications}

Despite the limitations, several policy implications emerge from this analysis. First, drug decriminalization does not appear to have large negative effects on employment. Concerns that reducing penalties would increase drug use and thereby harm labor markets are not supported by the evidence. Policymakers considering decriminalization can be reassured that labor market catastrophe is unlikely.

Second, the heterogeneous effects suggest that decriminalization may particularly benefit minority populations who face disproportionate drug enforcement. The significant positive effect for Hispanic individuals and the marginally positive effect for young Black males are consistent with this interpretation. To the extent that reducing racial disparities is a goal of criminal justice reform, the evidence suggests drug decriminalization may contribute to that goal.

Third, the modest overall effects suggest that decriminalization alone may be insufficient to substantially improve labor market outcomes for affected populations. Complementary policies—such as record expungement, job training programs, and employer incentives—may be necessary to translate reduced criminal penalties into meaningful employment gains. The limited evidence suggests that reducing new felony convictions, while valuable, does not erase the legacy of past convictions or address other barriers to employment.

\section{Conclusion}

This paper provides the first rigorous microdata analysis of how state-level drug decriminalization affects employment outcomes. Using Oklahoma's State Question 780 as a natural experiment and a difference-in-differences research design, I find modest and statistically insignificant effects on overall population employment. The point estimate of +0.20 percentage points is economically small and cannot be distinguished from zero at conventional significance levels.

However, the heterogeneity analysis reveals meaningful variation beneath the null aggregate result. Hispanic individuals experienced a statistically significant employment increase of 1.05 percentage points, representing a 1.5\% gain from baseline. Young Black males and workers without college degrees showed marginally significant positive effects of similar magnitude. These patterns suggest that drug decriminalization may benefit populations with elevated exposure to drug enforcement, even if the aggregate population effects are modest.

The findings contribute to the ongoing policy debate about drug decriminalization by providing evidence that such reforms need not harm labor markets. At the same time, the modest overall effects suggest that decriminalization alone may not be sufficient to substantially improve employment outcomes for affected populations. Complementary policies addressing record expungement, skill development, and employer incentives may be necessary to fully realize the labor market potential of criminal justice reform.

Future research should extend this analysis in several directions. Longer follow-up periods would help distinguish transitory from permanent effects. Administrative data linking criminal records to employment outcomes would enable analysis of directly affected individuals rather than the full population. And cross-state variation in decriminalization timing could support more robust identification through staggered difference-in-differences designs.

As states across the nation continue to reconsider drug policies inherited from the war on drugs, evidence on the broader consequences of reform will be essential for informed policymaking. This study suggests that employment concerns should not be a barrier to decriminalization, though the direct employment benefits appear modest without complementary reentry support.

\section*{Data Availability}

All data used in this paper are publicly available from the U.S. Census Bureau. The American Community Survey Public Use Microdata Sample can be accessed at \url{https://www.census.gov/programs-surveys/acs/microdata.html}. Replication code and processed datasets are available in the supplementary materials.

\section*{Pre-Registration}

The analysis plan for this study was pre-registered before examining outcome data. The pre-analysis plan specified the sample definition, outcome variables, treatment effects, subgroup analyses, and robustness checks reported in this paper. The locked pre-analysis plan with timestamp is available at \url{https://github.com/apep-working-papers/preregistration}.

\newpage
\bibliographystyle{apalike}
\begin{thebibliography}{99}

\bibitem[Agan and Starr(2018)]{agan2018ban}
Agan, A., and Starr, S. (2018). Ban the box, criminal records, and racial discrimination: A field experiment. \textit{The Quarterly Journal of Economics}, 133(1), 191-235.

\bibitem[Alexander(2010)]{alexander2010newjim}
Alexander, M. (2010). \textit{The New Jim Crow: Mass Incarceration in the Age of Colorblindness}. The New Press.

\bibitem[Belenko(2001)]{belenko2001research}
Belenko, S. (2001). Research on drug courts: A critical review 2001 update. \textit{National Drug Court Institute Review}, 4(2), 1-60.

\bibitem[Doleac and Hansen(2020)]{doleac2020unintended}
Doleac, J. L., and Hansen, B. (2020). The unintended consequences of ``ban the box'': Statistical discrimination and employment outcomes when criminal histories are hidden. \textit{Journal of Labor Economics}, 38(2), 321-374.

\bibitem[Holzer et al.(2006)]{holzer2006perceived}
Holzer, H. J., Raphael, S., and Stoll, M. A. (2006). Perceived criminality, criminal background checks, and the racial hiring practices of employers. \textit{The Journal of Law and Economics}, 49(2), 451-480.

\bibitem[Hughes and Stevens(2010)]{hughes2010decriminalization}
Hughes, C. E., and Stevens, A. (2010). What can we learn from the Portuguese decriminalization of illicit drugs? \textit{British Journal of Criminology}, 50(6), 999-1022.

\bibitem[Kuziemko(2013)]{kuziemko2013}
Kuziemko, I. (2013). How should inmates be released from prison? An assessment of parole versus fixed-sentence regimes. \textit{The Quarterly Journal of Economics}, 128(1), 371-424.

\bibitem[Looney and Turner(2018)]{looney2018work}
Looney, A., and Turner, N. (2018). Work and opportunity before and after incarceration. Brookings Institution.

\bibitem[Love(2006)]{love2006relief}
Love, M. C. (2006). Relief from the collateral consequences of a criminal conviction: A state-by-state resource guide. William S. Hein \& Company.

\bibitem[MacCoun and Reuter(2001)]{macoun2003law}
MacCoun, R. J., and Reuter, P. (2001). \textit{Drug War Heresies: Learning from Other Vices, Times, and Places}. Cambridge University Press.

\bibitem[Mauer(2006)]{mauer2006race}
Mauer, M. (2006). \textit{Race to Incarcerate}. The New Press.

\bibitem[Mueller-Smith(2015)]{muellersmith2015}
Mueller-Smith, M. (2015). The criminal and labor market impacts of incarceration. Working Paper.

\bibitem[Oklahoma Policy Institute(2018)]{okpolicy2018}
Oklahoma Policy Institute. (2018). SQ 780 and SQ 781: One year later. \url{https://okpolicy.org/sq-780-sq-781/}

\bibitem[Pager(2003)]{pager2003mark}
Pager, D. (2003). The mark of a criminal record. \textit{American Journal of Sociology}, 108(5), 937-975.

\bibitem[Prescott and Starr(2020)]{prescott2020expungement}
Prescott, J. J., and Starr, S. B. (2020). Expungement of criminal convictions: An empirical study. \textit{Harvard Law Review}, 133(8), 2460-2555.

\bibitem[Raphael(2014)]{raphael2014new}
Raphael, S. (2014). \textit{The New Scarlet Letter? Negotiating the U.S. Labor Market with a Criminal Record}. W.E. Upjohn Institute.

\bibitem[Rose(2021)]{rose2021}
Rose, E. K. (2021). Does banning the box help ex-offenders get jobs? Evaluating the effects of a prominent example. \textit{Journal of Labor Economics}, 39(S1), S79-S113.

\bibitem[Sabia and Nguyen(2018)]{sabia2018effects}
Sabia, J. J., and Nguyen, T. T. (2018). The effect of medical marijuana laws on labor market outcomes. \textit{The Journal of Law and Economics}, 61(3), 361-396.

\bibitem[Travis(2002)]{travis2002invisible}
Travis, J. (2002). Invisible punishment: An instrument of social exclusion. In M. Mauer \& M. Chesney-Lind (Eds.), \textit{Invisible Punishment: The Collateral Consequences of Mass Imprisonment}. The New Press.

\bibitem[Uggen et al.(2014)]{uggen2014work}
Uggen, C., Vuolo, M., Lageson, S., Ruhland, E., and Whitham, H. K. (2014). The edge of stigma: An experimental audit of the effects of low-level criminal records on employment. \textit{Criminology}, 52(4), 627-654.

\bibitem[Western(2006)]{western2006punishment}
Western, B. (2006). \textit{Punishment and Inequality in America}. Russell Sage Foundation.

\bibitem[Western and Beckett(1999)]{western2002incarceration}
Western, B., and Beckett, K. (1999). How unregulated is the U.S. labor market? The penal system as a labor market institution. \textit{American Journal of Sociology}, 104(4), 1030-1060.

\end{thebibliography}

\newpage
\appendix
\section{Appendix: Additional Figures}

\begin{figure}[H]
\centering
\includegraphics[width=0.9\textwidth]{figures/subgroup_effects.png}
\caption{Heterogeneous Treatment Effects by Subgroup}
\small{Notes: Coefficient plot showing DiD estimates for employment across subgroups. Bars indicate 95\% confidence intervals. Red bars indicate p$<$0.1.}
\end{figure}

\end{document}
