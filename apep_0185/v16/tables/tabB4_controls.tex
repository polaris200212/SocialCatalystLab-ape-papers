\begin{table}[H]
\centering
\caption{Robustness: County-Specific Linear Trends}
\label{tab:robustB4}
\small
\begin{tabular}{l cccc}
\toprule
 & (1) OLS & (2) OLS & (3) 2SLS & (4) 2SLS \\
 & Baseline & + County Trends & Baseline & + County Trends \\
\midrule
\multicolumn{5}{l}{\textit{Dependent Variable: Log Employment}} \\[3pt]
Network MW & 0.7086*** & 0.0099 & 0.8858*** & -0.1054 \\
 & (0.1526) & (0.0759) & (0.1641) & (0.1550) \\
\midrule
County FE & Yes & Yes & Yes & Yes \\
State $\times$ Time FE & Yes & Yes & Yes & Yes \\
County Linear Trend & No & Yes & No & Yes \\
\bottomrule
\end{tabular}
\begin{figurenotes}
Notes: Columns (1)--(2) report OLS; columns (3)--(4) report 2SLS. Adding county-specific
linear trends absorbs most of the variation, as expected when the identifying variation is
gradual. The attenuation is 98.6\% for OLS, consistent with the slow-moving
nature of network exposure changes.
Standard errors clustered at state level in parentheses.
*** p$<$0.01, ** p$<$0.05, * p$<$0.1.
\end{figurenotes}
\end{table}

