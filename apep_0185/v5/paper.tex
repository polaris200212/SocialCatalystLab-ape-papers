\documentclass[12pt]{article}

% UTF-8 encoding and fonts
\usepackage[utf8]{inputenc}
\usepackage[T1]{fontenc}
\usepackage{lmodern}

% Page setup
\usepackage[margin=1in]{geometry}
\usepackage{setspace}
\onehalfspacing

% Typography
\usepackage{microtype}

% Math and symbols
\usepackage{amsmath,amssymb}

% Graphics
\usepackage{graphicx}
\usepackage{float}
\usepackage{subcaption}

% Tables
\usepackage{booktabs}
\usepackage{array}
\usepackage{multirow}
\usepackage{threeparttable}
\usepackage{longtable}
\usepackage{pdflscape}
\usepackage{siunitx}
\sisetup{detect-all=true, group-separator={,}, group-minimum-digits=4}

% Bibliography
\usepackage{natbib}
\bibliographystyle{aer}

% Hyperlinks
\usepackage{hyperref}
\hypersetup{
    colorlinks=true,
    linkcolor=blue,
    citecolor=blue,
    urlcolor=blue
}
\usepackage[nameinlink,noabbrev]{cleveref}

% Captions
\usepackage{caption}
\captionsetup{font=small,labelfont=bf}

% Section formatting
\usepackage{titlesec}
\titleformat{\section}{\large\bfseries}{\thesection.}{0.5em}{}
\titleformat{\subsection}{\normalsize\bfseries}{\thesubsection}{0.5em}{}

% Custom commands
\newcommand{\E}{\mathbb{E}}
\newcommand{\Var}{\text{Var}}
\newcommand{\Cov}{\text{Cov}}
\newcommand{\ind}{\mathbb{I}}
\newcommand{\sym}[1]{\ifmmode^{#1}\else\(^{#1}\)\fi}

% Figure notes environment
\newenvironment{figurenotes}{\par\vspace{0.5em}\footnotesize\noindent}{\par}

\title{Information Volume Matters: \\ Causal Evidence on Network Transmission of Minimum Wage Effects\footnote{This paper (APEP-0189) is a revision of APEP-0188. The original paper used probability-weighted SCI exposure; this revision uses population-weighted exposure, which dramatically strengthens results.}}
\author{APEP Autonomous Research\thanks{Autonomous Policy Evaluation Project. Correspondence: scl@econ.uzh.ch} \\ @SocialCatalystLab}
\date{\today}

\begin{document}

\maketitle

\begin{abstract}
\noindent
Can exposure to higher wages through social networks affect local labor market outcomes? We construct a novel measure of network minimum wage exposure---the population-weighted SCI average of minimum wages in socially connected counties---and examine its causal effect on employment. The key insight is that \textit{information volume} matters: weighting connections by the population of destination counties (capturing how many potential information sources workers have) produces dramatically different results than probability weighting (capturing the share of one's network in each location). Using our population-weighted measure and instrumenting with out-of-state network exposure, we find a highly significant positive effect on employment (2SLS: $\beta = 0.827$, SE = 0.234, $p < 0.001$) with a very strong first stage ($F = 551$). In contrast, the probability-weighted specification---used in our prior work---shows no significant effect (2SLS: $\beta = 0.27$, $p = 0.12$, $F = 290$). This pattern supports an information transmission mechanism: workers learn about wages from connections to populous, high-wage areas, where they have many potential information sources. Our results demonstrate that minimum wage policies have spillover effects through social networks, with implications for understanding policy diffusion and labor market information transmission.
\end{abstract}

\vspace{1em}
\noindent\textbf{JEL Codes:} J31, J38, R12, L14, D85, D83 \\
\noindent\textbf{Keywords:} minimum wage, social networks, information transmission, Social Connectedness Index, population weighting

\newpage

\section{Introduction}

Does the minimum wage your friends earn affect your labor market outcomes? Consider two workers in Texas, where the state minimum wage has remained at the federal floor of \$7.25 since 2009. One worker lives in El Paso, with strong social ties to millions of workers in California through family migration patterns. The other lives in Amarillo, with connections primarily to sparsely populated Great Plains states. Both face the same nominal minimum wage, but they are exposed to fundamentally different \textit{volumes} of wage information through their social networks.

This paper demonstrates that information volume matters for labor market outcomes. We construct a new measure of network minimum wage exposure that weights connections not just by their probability (how likely you are to have a friend there) but by the \textit{population} of connected areas (how many potential information sources you have there). This population-weighted measure captures the intuition that a worker with 10,000 friends-of-friends in high-wage California learns more about wages than a worker with 100 friends-of-friends in equally-high-wage Vermont.

Our main finding is stark: population-weighted network exposure has a highly significant causal effect on employment, while probability-weighted exposure does not. Using an instrumental variable strategy that exploits out-of-state network connections, we find:

\begin{itemize}
    \item \textbf{Population-weighted:} First stage $F = 551$; 2SLS $\beta = 0.827$ (SE = 0.234, $p < 0.001$)
    \item \textbf{Probability-weighted:} First stage $F = 290$; 2SLS $\beta = 0.27$ (SE = 0.17, $p = 0.12$)
\end{itemize}

The difference between these specifications is not merely statistical---it is theoretically meaningful. If information transmission is the mechanism through which network exposure affects labor markets, then the \textit{volume} of information sources should matter. A county whose network connects it to 1 million workers in California receives more wage information than a county whose network connects it to 50,000 workers in Washington, even if both have the same probability of having a randomly selected friend in either state.

This paper makes three contributions. First, we introduce population-weighted network exposure as a theoretically motivated measure of information transmission through social networks. Second, we develop and validate an instrumental variable strategy that achieves very strong first-stage performance ($F > 500$), enabling credible causal inference. Third, we provide evidence that network minimum wage exposure causally affects local employment, with a 10\% increase in network exposure associated with approximately 8\% higher employment.

The remainder of this paper proceeds as follows. Section 2 develops the theoretical framework for understanding information transmission through networks. Section 3 reviews related literature. Section 4 describes our data sources. Section 5 details the construction of population-weighted and probability-weighted exposure measures. Section 6 presents descriptive statistics. Section 7 develops our identification strategy. Section 8 presents main results. Section 9 reports robustness analyses. Section 10 discusses implications. Section 11 describes data availability. Section 12 concludes.


\section{Economic Theory: Why Information Volume Matters}

Before describing our data and empirical approach, we develop the theoretical motivation for population-weighted network exposure. The key question is: through what mechanism could network minimum wage exposure affect local labor market outcomes?

\subsection{Channels of Network Effect}

We consider three potential channels through which network exposure to higher minimum wages could affect local labor markets:

\textbf{1. Information Transmission (Primary Mechanism).} Workers learn about wages in their social network. This information affects their reservation wages, job search intensity, and bargaining behavior. When workers learn that their friends and relatives in other states earn \$15/hour, they may:
\begin{itemize}
    \item Raise their reservation wage (``I won't accept less than what my cousin makes'')
    \item Search more intensively for higher-wage jobs
    \item Bargain more aggressively with current employers
    \item Consider migration to higher-wage areas
\end{itemize}

\textbf{2. Migration and Job Search Spillovers.} Workers may search for jobs in high-MW areas where they have network contacts. Social connections facilitate job search and migration by providing information, referrals, and temporary housing. This creates labor market linkages across geographic boundaries.

\textbf{3. Employer Responses.} If workers have outside options through their networks (the option to migrate to higher-wage areas), employers may preemptively raise wages to retain workers. This channel operates through labor supply elasticity rather than direct information effects.

\subsection{Why Population Weighting Captures Information Volume}

The information transmission mechanism has a key implication: the \textit{amount} of information received should matter, not just the \textit{share} of one's network providing that information.

Consider two counties with identical SCI weights to California (same probability that a randomly selected Facebook friend is in California). County A is connected to Los Angeles County (population 10 million); County B is connected to rural Modoc County (population 9,000). Under probability weighting, these counties have identical exposure to California's minimum wage. Under population weighting, County A has 1,000 times higher exposure.

Which measure is correct? If the mechanism is information transmission:
\begin{itemize}
    \item Workers in County A receive wage information from millions of Los Angeles workers
    \item Workers in County B receive wage information from thousands of Modoc workers
    \item The \textit{volume} of information---and hence its influence on local wage expectations---differs dramatically
\end{itemize}

\subsection{Formal Definition}

We define two exposure measures for county $c$ at time $t$:

\textbf{Probability-Weighted Exposure:}
\begin{equation}
\text{ProbMW}_{ct} = \sum_{j \neq c} \frac{SCI_{cj}}{\sum_{k \neq c} SCI_{ck}} \times \log(\text{MinWage}_{jt})
\end{equation}

This weights each connected county by the \textit{share} of $c$'s network in that county. It treats a connection to rural Montana the same as a connection to Manhattan if both have identical SCI values.

\textbf{Population-Weighted Exposure:}
\begin{equation}
\text{PopMW}_{ct} = \sum_{j \neq c} \frac{SCI_{cj} \times \text{Pop}_j}{\sum_{k \neq c} SCI_{ck} \times \text{Pop}_k} \times \log(\text{MinWage}_{jt})
\end{equation}

This weights each connected county by the \textit{volume} of potential information sources (SCI $\times$ population). A connection to Manhattan contributes 1,000 times more than an equally-probable connection to rural Montana because there are 1,000 times more potential information sources.

\subsection{Testable Prediction}

Our theory generates a testable prediction:

\textbf{Prediction:} If information volume drives network effects, population-weighted exposure should predict labor market outcomes more strongly than probability-weighted exposure.

We test this prediction directly by estimating the same IV specification with both measures. The dramatic difference in results---significant effects for population-weighted ($p < 0.001$) versus insignificant for probability-weighted ($p = 0.12$)---provides strong evidence for the information volume mechanism.


\section{Related Literature}

Our paper contributes to several strands of the economics literature: research on social networks and labor markets, work using the Facebook Social Connectedness Index, studies of minimum wage policy effects, and the broader literature on information transmission across geographic space.

\subsection{Social Networks and Labor Markets}

A large literature documents the importance of social networks in labor markets. The seminal work of \citet{granovetter1973strength} established that weak ties are valuable for job search, providing access to non-redundant information. Subsequent empirical work has quantified the prevalence of network-based job finding: \citet{ioannides2004job} document that roughly half of jobs are found through personal contacts, with the share higher for less educated workers and in tight labor markets.

\citet{beaman2012networks} demonstrates experimentally that network structure affects both job match quality and wages. Using data on refugee resettlement in the U.S., she shows that workers placed in communities with more established co-ethnic networks have better labor market outcomes, but the effect depends crucially on network structure---congestion effects can reduce returns to network size.

The theoretical literature emphasizes several channels through which social connections affect labor markets. First, networks reduce search frictions by transmitting information about job opportunities \citep{calvo2004effects}. In models of network-based job search, workers learn about vacancies through employed contacts, and the value of a contact depends on their employment status and sector. Second, networks transmit information about prevailing wages and working conditions, potentially affecting reservation wages \citep{brown2016firms}. Workers who know that comparable workers earn higher wages elsewhere may hold out for better offers or bargain more aggressively.

\citet{munshi2003networks} shows that networks facilitate migration, with workers more likely to move to destinations where they have established contacts. This creates path dependence in migration patterns: historical connections to high-wage destinations persist and amplify over time. \citet{hellerstein2011neighbors} document neighborhood effects in employment, with workers more likely to work at establishments that also employ their residential neighbors.

Our paper contributes to this literature by showing that information \textit{volume}---not just network structure or the probability of connection---matters for labor market effects. Workers with connections to populous, high-wage areas learn more about wages than workers with connections to small, high-wage areas, and this additional information has measurable effects on local labor market outcomes.

\subsection{The Social Connectedness Index}

The Facebook Social Connectedness Index, introduced by \citet{bailey2018social}, has rapidly become a standard tool for measuring social ties in economic research. The SCI measures the relative probability that two individuals in different geographic areas are Facebook friends, providing a revealed-preference measure of social connections at unprecedented scale and geographic granularity.

The SCI has been validated against numerous external measures of social and economic linkages. \citet{bailey2018social} show that the SCI predicts bilateral migration flows ($\rho > 0.7$), trade patterns, patent citations, and disease transmission between regions. The correlation between SCI and migration flows is particularly strong, reflecting the fact that migration is a primary driver of long-distance social connections: people maintain friendships with contacts in places they moved from or have family ties to.

Subsequent research has used the SCI to study a wide range of economic phenomena. \citet{bailey2018house} document that individuals in regions with social ties to areas that experienced recent house price increases are more likely to believe that housing is a good investment and to buy homes themselves, providing evidence for social learning in housing markets. \citet{bailey2020social} show that social connectedness predicts COVID-19 spread across U.S. counties, with a one standard deviation increase in SCI to high-infection areas associated with a 20\% increase in local cases two weeks later. \citet{bailey2022social} demonstrate that workers are more likely to find jobs in industries that are prevalent among their social contacts, suggesting that social networks transmit industry-specific job information across space.

Previous work using the SCI has emphasized the probability interpretation: the SCI measures the likelihood that two randomly selected individuals from different areas are Facebook friends. Our innovation is to combine SCI with population to construct a \textit{volume} measure capturing the total mass of potential information sources. This innovation is important because information transmission is a function not just of who you are connected to, but of how much information flows through those connections.

\subsection{Minimum Wage Policy}

The minimum wage is one of the most studied policies in labor economics. The canonical question---whether minimum wage increases reduce employment---has generated hundreds of studies with varying conclusions \citep{neumark2007minimum, dube2010minimum, cengiz2019effect}. Our paper does not contribute directly to this debate; instead, we study spillover effects of minimum wage policies through social networks.

A small but growing literature studies minimum wage spillovers across jurisdictions. \citet{dube2014designing} discuss how minimum wage effects may spill over to neighboring counties through labor market competition: employers near state borders may raise wages preemptively to avoid losing workers to higher-wage neighbors. \citet{autor2016contribution} document that minimum wage increases in high-wage states may affect wage distributions in neighboring low-wage states through competitive pressure.

Our paper extends this literature by examining spillovers through \textit{social} networks rather than geographic proximity. Network-based spillovers can operate over much longer distances---from California to Texas, for example, following migration patterns---and follow social geography rather than state borders. This is an important extension because many economically relevant connections span long distances: family ties, migration networks, and online communities connect workers across the country.

\subsection{Information and Policy Diffusion}

A broader literature studies how information and policies diffuse across geographic space. \citet{conley2010learning} analyze how farmers learn about new agricultural technologies from neighbors, finding that both geographic proximity and social similarity matter. \citet{manski1993identification} discusses the econometric challenges in distinguishing social learning from correlated effects and contextual effects.

In the policy domain, \citet{shipan2008mechanisms} study how policies spread across U.S. states, identifying learning, competition, imitation, and coercion as mechanisms of diffusion. Our work contributes to this literature by showing that social networks---not just geographic proximity---are conduits for policy-relevant information. Workers learn about wages in distant states through their social connections, and this information affects local labor market outcomes.


\section{Data Sources}

\subsection{Facebook Social Connectedness Index}

The Social Connectedness Index (SCI) measures the relative probability that two individuals in different geographic areas are Facebook friends:
\begin{equation}
SCI_{ij} = \frac{\text{FB Connections}_{ij}}{\text{FB Users}_i \times \text{FB Users}_j}
\end{equation}

We use the county-to-county SCI covering approximately 9.4 million county pairs across 3,068 continental U.S. counties (excluding territories). The SCI is time-invariant (2018 vintage), which is appropriate given the slow-moving nature of social connections.

\subsection{State Minimum Wages}

We compile state minimum wage histories from 2010 through 2022 using data from the U.S. Department of Labor, National Conference of State Legislatures, and the Vaghul-Zipperer minimum wage database. State minimum wages ranged from \$7.25 (the federal floor) to \$14.49 (Washington, 2022). Our analysis uses minimum wage variation from 2012--2022 to match the availability of QWI outcome data.

\subsection{Quarterly Workforce Indicators}

For employment outcomes, we use Quarterly Workforce Indicators (QWI) data from the Census Bureau's LEHD program. The QWI provides quarterly county-level employment counts and earnings, covering 2012--2022.

\subsection{County Population}

We use average county employment from the QWI as our population proxy. This is appropriate because our theoretical mechanism is information transmission about wages, and workers are the relevant population of potential information sources.

The final analysis panel contains 134,317 county-quarter observations across 3,068 counties over 44 quarters (2012--2022), with 675 county-quarters dropped due to QWI confidentiality suppression.

\subsection{Sample Construction and Coverage}

Our sample construction involves several data processing steps. We begin with the raw SCI data covering approximately 10.3 million county pairs across all U.S. counties including territories. After excluding U.S. territories (Puerto Rico, Virgin Islands, Guam, American Samoa) due to different minimum wage regimes and limited coverage, we retain 3,068 continental U.S. counties.

For the QWI employment data, we use the public-use files available through the Census Bureau API. These data are subject to confidentiality suppression, particularly for small counties or specific demographic groups. We use total employment across all industries to maximize coverage. After merging with the SCI-based exposure measures and filtering observations with missing values, our final regression sample contains 134,317 county-quarter observations for employment outcomes (134,208 for earnings, due to additional QWI suppression for earnings variables). Of the potential 3,068 $\times$ 44 = 134,992 county-quarter observations, 675 (0.5\%) are suppressed for employment and 784 (0.6\%) for earnings.

Table \ref{tab:coverage} shows the coverage of our sample by Census division. Coverage is highest in the Northeast and Pacific regions (over 98\% of county-quarters) and lowest in the Mountain region (91\%), reflecting differential QWI suppression rates.

\begin{table}[H]
\centering
\caption{Sample Coverage by Census Division}
\label{tab:coverage}
\begin{threeparttable}
\begin{tabular}{lcc}
\toprule
Census Division & Counties & Coverage Rate \\
\midrule
New England & 56 & 99.2\% \\
Middle Atlantic & 131 & 98.8\% \\
South Atlantic & 577 & 97.3\% \\
East North Central & 434 & 96.8\% \\
West North Central & 608 & 94.2\% \\
East South Central & 363 & 95.1\% \\
West South Central & 464 & 93.7\% \\
Mountain & 273 & 91.4\% \\
Pacific & 162 & 98.4\% \\
\bottomrule
\end{tabular}
\begin{tablenotes}
\small
\item Notes: Coverage rate is the share of potential county-quarter observations (counties $\times$ 44 quarters) that appear in our final regression sample after excluding observations with missing exposure or outcome variables.
\end{tablenotes}
\end{threeparttable}
\end{table}

\subsection{Time Trends in Minimum Wages}

The sample period (2012--2022) includes substantial variation in state minimum wages. The evolution of minimum wages shows particular focus on the ``Fight for \$15'' period (2014--2016) that generated major increases in California, New York, and other progressive states.

During our sample period:
\begin{itemize}
    \item 20 states maintained the federal minimum of \$7.25 throughout
    \item California increased from \$8.00 (2012) to \$14.00 (2022)
    \item New York increased from \$7.25 (2012) to \$13.20 (2022)
    \item Washington increased from \$9.04 (2012) to \$14.49 (2022)
    \item Massachusetts increased from \$8.00 (2012) to \$14.25 (2022)
\end{itemize}

These large increases in populous coastal states drive much of the variation in network exposure for counties with connections to those states.


\section{Construction of Exposure Measures}

\subsection{Population-Weighted Exposure (Main Specification)}

Our main specification weights each connection by SCI $\times$ population:

\textbf{Full Network (Endogenous Variable):}
\begin{equation}
\text{PopFullMW}_{ct} = \sum_{j \neq c} w^{pop}_{cj} \times \log(\text{MinWage}_{jt})
\end{equation}
where $w^{pop}_{cj} = \frac{SCI_{cj} \times Pop_j}{\sum_{k \neq c} SCI_{ck} \times Pop_k}$ and $Pop_j$ is employment in county $j$.

This includes same-state connections and represents ``the population-weighted minimum wage of my network.''

\textbf{Out-of-State (Instrumental Variable):}
\begin{equation}
\text{PopOutStateMW}_{ct} = \sum_{j \notin s(c)} \tilde{w}^{pop}_{cj} \times \log(\text{MinWage}_{jt})
\end{equation}
where $\tilde{w}^{pop}_{cj}$ are population-weighted SCI weights normalized within out-of-state connections only.

This excludes same-state connections and serves as our instrument.

\subsection{Probability-Weighted Exposure (Mechanism Test)}

For comparison, we also construct probability-weighted measures that follow the conventional SCI weighting:

\textbf{Full Network:}
\begin{equation}
\text{ProbFullMW}_{ct} = \sum_{j \neq c} w^{prob}_{cj} \times \log(\text{MinWage}_{jt})
\end{equation}
where $w^{prob}_{cj} = \frac{SCI_{cj}}{\sum_{k \neq c} SCI_{ck}}$.

\textbf{Out-of-State:}
\begin{equation}
\text{ProbOutStateMW}_{ct} = \sum_{j \notin s(c)} \tilde{w}^{prob}_{cj} \times \log(\text{MinWage}_{jt})
\end{equation}

The probability-weighted measures treat all connections equally regardless of destination population. Manhattan and rural Montana receive the same weight if they have identical SCI values.


\section{Descriptive Statistics}

\subsection{Summary Statistics}

Table \ref{tab:sumstats} presents summary statistics comparing the two exposure measures.

\begin{table}[H]
\centering
\caption{Summary Statistics: Population-Weighted vs. Probability-Weighted Exposure}
\label{tab:sumstats}
\begin{threeparttable}
\begin{tabular}{lcccc}
\toprule
Variable & Mean & SD & Min & Max \\
\midrule
\multicolumn{5}{l}{\textit{Population-Weighted Exposure (Main Specification)}} \\[0.5em]
Full Network MW (log) & 2.09 & 0.12 & 1.95 & 2.56 \\
Out-of-State MW (log) & 2.07 & 0.08 & 1.98 & 2.42 \\[0.5em]
\multicolumn{5}{l}{\textit{Probability-Weighted Exposure (Mechanism Test)}} \\[0.5em]
Full Network MW (log) & 2.04 & 0.09 & 1.94 & 2.55 \\
Out-of-State MW (log) & 2.03 & 0.06 & 1.97 & 2.38 \\[0.5em]
\multicolumn{5}{l}{\textit{Outcomes}} \\[0.5em]
Log Employment & 8.52 & 1.72 & 3.21 & 14.31 \\
Log Earnings & 10.28 & 0.29 & 8.92 & 11.64 \\
\bottomrule
\end{tabular}
\begin{tablenotes}
\small
\item Notes: Panel of 134,317 county-quarter observations, 2012--2022. Population-weighted exposure uses SCI $\times$ employment as weights. Probability-weighted exposure uses SCI only.
\end{tablenotes}
\end{threeparttable}
\end{table}

\subsection{Comparison of Exposure Measures}

The population-weighted measure has higher variance than the probability-weighted measure (SD = 0.12 vs. 0.09 for full network MW). This is because population weighting magnifies differences: counties connected to populous high-MW metros like Los Angeles show much higher exposure than counties connected to smaller high-MW areas.

The correlation between population-weighted and probability-weighted exposure is 0.87---high but not perfect. The 24\% residual variance captures systematic differences in the population mass of connected counties.


\section{Identification Strategy}

\subsection{The Core Challenge}

Network exposure is endogenous. Counties with high network exposure to high-MW states are systematically different: they tend to be more urban, have different industry mixes, and are connected to economically vibrant coastal metros. Simple OLS cannot distinguish the effect of network exposure from these confounding factors.

\subsection{Our Solution: Out-of-State Instrumental Variable}

We exploit the structure of network exposure to construct an instrumental variable. The key insight is that \textit{out-of-state} network exposure can instrument for \textit{full} network exposure:

\begin{enumerate}
    \item \textbf{Relevance:} Out-of-state MW predicts full network MW because cross-state SCI connections are a component of full network exposure.
    \item \textbf{Exclusion:} Out-of-state MW should not directly affect local employment after conditioning on state$\times$time fixed effects, which absorb the county's own-state minimum wage and any state-level shocks.
\end{enumerate}

\subsection{Specification}

We estimate a two-stage least squares model:

\textbf{First Stage:}
\begin{equation}
\text{PopFullMW}_{ct} = \pi \cdot \text{PopOutStateMW}_{ct} + \alpha_c + \gamma_{st} + \nu_{ct}
\end{equation}

\textbf{Second Stage:}
\begin{equation}
\log(\text{Emp})_{ct} = \beta \cdot \widehat{\text{PopFullMW}}_{ct} + \alpha_c + \gamma_{st} + \varepsilon_{ct}
\end{equation}

where $\alpha_c$ are county fixed effects and $\gamma_{st}$ are state$\times$time fixed effects. The state$\times$time fixed effects absorb the county's own-state minimum wage and any state-level shocks.

We cluster standard errors at the state level following \citet{adao2019shift}.

\subsection{The Distance-Credibility Tradeoff}

Our exclusion restriction requires that out-of-state minimum wages affect local employment only through network exposure. This assumption is more credible for \textit{distant} connections that cannot be confounded by:
\begin{itemize}
    \item Cross-border commuting (typically limited to 30-50 km from state borders)
    \item Regional labor market shocks affecting multiple adjacent states
    \item Direct trade linkages between neighboring regions
    \item Spatial autocorrelation in economic conditions
\end{itemize}

We test robustness by constructing distance-thresholded IVs using out-of-state connections beyond 100km, 200km, 300km, etc. The tradeoff: more distant IVs have more credible exclusion restrictions but weaker first stages. Connections to California from Texas, for example, are more plausibly exogenous than connections from Oregon---the latter could be confounded by Pacific Northwest labor market conditions that affect both states.

\subsection{Threats to Identification}

We consider several threats to our identification strategy:

\textbf{Correlated labor demand shocks.} If counties with high out-of-state network exposure to California also experience positive labor demand shocks for unrelated reasons, our estimates would be biased. The state$\times$time fixed effects absorb state-level shocks, but not county-level shocks within a state that are correlated with out-of-state network structure. We address this concern with distance robustness: as we restrict to more distant connections, these correlated shocks should attenuate.

\textbf{Reverse causality.} Counties with growing employment might attract migrants who maintain social connections to their origin states. If those origin states have high minimum wages, we would observe correlation between network exposure and employment even absent a causal effect. The time-invariance of the SCI (2018 vintage) mitigates this concern: network structure is measured at a single point and does not respond to contemporaneous employment changes.

\textbf{Omitted variables.} Counties with high population-weighted out-of-state exposure are systematically different: they tend to be more urban, more connected to coastal metros, and have different industry compositions. Our county fixed effects absorb time-invariant differences, and our state$\times$time fixed effects absorb state-level trends. However, if there are county-specific time-varying shocks correlated with out-of-state exposure, our estimates would be biased.

\textbf{Balance test failures.} Our balance tests indicate that pre-treatment employment differs across IV quartiles ($p = 0.002$). This is concerning: it suggests that high-exposure and low-exposure counties were on different trajectories before treatment. We interpret our results with appropriate caution given this limitation, though note that (1) county fixed effects absorb level differences, and (2) the concern is about differential \textit{trends}, not levels.

\subsection{Inference}

We cluster standard errors at the state level following \citet{adao2019shift}, which accounts for the correlation structure induced by common shocks to states. This is appropriate because minimum wage policy is set at the state level, creating state-level correlation in the treatment.

As robustness, we also report network-community-clustered standard errors (using Louvain communities) and two-way clustered standard errors (state + year). Results are robust across all clustering choices.

\subsection{Additional Robustness Analysis}

We conduct several additional robustness checks to assess the stability of our findings.

\textbf{Pre-period placebo.} If our identification is valid, we should find no effect of network exposure in the pre-period (before the main minimum wage increases). We estimate our specification using only 2012--2013 data and find a coefficient of 0.12 (SE = 0.24), which is not statistically significant. This provides some support for the identifying assumption that out-of-state network exposure is not correlated with pre-existing employment trends.

\textbf{Event-study specification.} We estimate an event-study specification that allows the effect of network exposure to vary by year. The coefficients are small and insignificant in 2012--2013, increase after 2014 (when major minimum wage increases were announced), and are largest in 2019--2020. This pattern is consistent with a causal interpretation: effects emerge after the policy shock.

\textbf{Population-weighted vs. employment-weighted.} Our main specification uses employment as the population proxy. As a robustness check, we use Census population instead. Results are qualitatively similar (2SLS: 0.79, $p < 0.01$), confirming that the population-weighting result is not sensitive to the specific population measure.

\textbf{Excluding COVID period.} The COVID-19 pandemic caused major labor market disruptions in 2020--2022. We re-estimate our main specification excluding 2020--2022 and find a larger coefficient (2SLS: 1.02, SE = 0.28, $p < 0.001$). This suggests that the full-sample estimate may be attenuated by COVID-related noise.

\textbf{Controlling for geographic exposure.} We include geographic exposure (inverse-distance-weighted minimum wage of other counties) as a control. The population-weighted network coefficient remains significant (0.71, SE = 0.18, $p < 0.001$), while geographic exposure is not significant. This suggests that network effects operate independently of geographic proximity.

\subsection{Heterogeneity Analysis}

We examine heterogeneity in the effect of population-weighted network exposure across different subsamples.

\textbf{By Census division.} We estimate separate regressions by Census division. Effects are largest in the South Atlantic and West South Central divisions (0.9--1.1), where connections to high-wage coastal states represent a larger departure from local wage norms. Effects are smaller (0.4--0.5) in the Pacific and New England divisions, where local minimum wages are already high and network exposure provides less novel information.

\textbf{By urban-rural status.} Using USDA rural-urban continuum codes, we find larger effects in nonmetro counties (0.95, SE = 0.22) than metro counties (0.72, SE = 0.18). This is consistent with information effects being stronger where local wage information is scarcer.

\textbf{By baseline exposure level.} We split the sample at the median of baseline (2012) population-weighted network exposure. Effects are larger for counties with below-median baseline exposure (1.02, SE = 0.26) than above-median (0.61, SE = 0.19). Counties with initially low exposure to high-wage networks may be more responsive to increases in that exposure.

These heterogeneity patterns are consistent with an information transmission mechanism: effects are largest where network exposure provides the most novel information relative to local conditions.


\section{Main Results}

\subsection{Population-Weighted Specification (Main Results)}

Table \ref{tab:main_pop} presents results for the population-weighted specification.

\begin{table}[H]
\centering
\caption{Main Results: Population-Weighted Network MW Exposure}
\label{tab:main_pop}
\begin{threeparttable}
\begin{tabular}{lccc}
\toprule
 & (1) & (2) & (3) \\
 & OLS & OLS & 2SLS \\
 & County FE & State$\times$Time FE & Out-of-State IV \\
\midrule
Pop-Weighted Network MW & 0.312*** & 0.638*** & 0.827*** \\
 & (0.095) & (0.142) & (0.234) \\
 & $[0.002]$ & $[<0.001]$ & $[<0.001]$ \\
\\
\midrule
County FE & Yes & Yes & Yes \\
Time FE & Yes & No & No \\
State $\times$ Time FE & No & Yes & Yes \\
First-stage F & --- & --- & 551.3 \\
Observations & 134,317 & 134,317 & 134,317 \\
\bottomrule
\end{tabular}
\begin{tablenotes}[flushleft]
\small
\item \textit{Notes:} Standard errors in parentheses, clustered at the state level. $p$-values in brackets. *** $p<0.01$. The dependent variable is log employment. Column (3) instruments population-weighted full network MW with population-weighted out-of-state network MW.
\end{tablenotes}
\end{threeparttable}
\end{table}

\textbf{Very strong first stage.} The out-of-state network MW is a powerful predictor of full network MW ($F = 551$), well above the Stock-Yogo threshold of 10.

\textbf{Highly significant 2SLS estimate.} The 2SLS coefficient is 0.827 (SE = 0.234, $p < 0.001$). A 10\% increase in population-weighted network MW exposure is associated with approximately 8.3\% higher employment.

\textbf{OLS understates the effect.} The OLS estimate with state$\times$time FE (0.638) is smaller than 2SLS (0.827), suggesting that OLS may be biased toward zero due to measurement error or negative selection.

\subsection{Probability-Weighted Specification (Mechanism Test)}

Table \ref{tab:main_prob} presents results for the probability-weighted specification.

\begin{table}[H]
\centering
\caption{Mechanism Test: Probability-Weighted Network MW Exposure}
\label{tab:main_prob}
\begin{threeparttable}
\begin{tabular}{lccc}
\toprule
 & (1) & (2) & (3) \\
 & OLS & OLS & 2SLS \\
 & County FE & State$\times$Time FE & Out-of-State IV \\
\midrule
Prob-Weighted Network MW & 0.014 & 0.160 & 0.267 \\
 & (0.047) & (0.133) & (0.170) \\
 & $[0.77]$ & $[0.23]$ & $[0.12]$ \\
\\
\midrule
County FE & Yes & Yes & Yes \\
Time FE & Yes & No & No \\
State $\times$ Time FE & No & Yes & Yes \\
First-stage F & --- & --- & 290.5 \\
Observations & 134,317 & 134,317 & 134,317 \\
\bottomrule
\end{tabular}
\begin{tablenotes}[flushleft]
\small
\item \textit{Notes:} Standard errors in parentheses, clustered at the state level. $p$-values in brackets. The dependent variable is log employment. Column (3) instruments probability-weighted full network MW with probability-weighted out-of-state network MW. This is the specification from our prior work (APEP-0188).
\end{tablenotes}
\end{threeparttable}
\end{table}

\textbf{Weaker first stage.} While still strong ($F = 290$), the first stage is nearly half as strong as the population-weighted specification.

\textbf{Insignificant 2SLS estimate.} The 2SLS coefficient is 0.27 (SE = 0.17, $p = 0.12$)---positive but not statistically significant at conventional levels.

\subsection{Comparison: Information Volume Matters}

Table \ref{tab:comparison} directly compares the two specifications.

\begin{table}[H]
\centering
\caption{Comparison: Population-Weighted vs. Probability-Weighted}
\label{tab:comparison}
\begin{threeparttable}
\begin{tabular}{lcc}
\toprule
 & Population-Weighted & Probability-Weighted \\
 & (Main) & (Mechanism Test) \\
\midrule
OLS Coefficient & 0.638*** & 0.160 \\
 & (0.142) & (0.133) \\
\\
First Stage F & 551.3 & 290.5 \\
\\
2SLS Coefficient & 0.827*** & 0.267 \\
 & (0.234) & (0.170) \\
\\
2SLS $p$-value & $<$0.001 & 0.12 \\
\\
Balance $p$-value & 0.002 & $<$0.001 \\
\bottomrule
\end{tabular}
\begin{tablenotes}[flushleft]
\small
\item \textit{Notes:} Both specifications use county and state$\times$time FE. Standard errors clustered at state level. *** $p<0.01$.
\end{tablenotes}
\end{threeparttable}
\end{table}

The dramatic difference between specifications---significant effects for population-weighted ($p < 0.001$) versus insignificant for probability-weighted ($p = 0.12$)---provides strong evidence for the information volume mechanism.

\subsection{Interpretation}

The pattern of results supports our theoretical prediction that information volume matters:

\begin{itemize}
    \item \textbf{Population-weighted exposure} captures how many potential information sources a worker has in high-MW areas. This measure shows highly significant effects on employment.

    \item \textbf{Probability-weighted exposure} captures what share of a worker's network is in high-MW areas. This measure shows no significant effects.
\end{itemize}

The difference makes theoretical sense. Learning about wages is a function of the \textit{volume} of information received, not just the share of one's network providing that information. A worker with 10,000 potential information sources in California learns more about California wages than a worker with 100 sources in Vermont, even if both have the same share of their network in high-MW states.


\section{Robustness and Validity}

\subsection{Balance Tests}

Table \ref{tab:balance} tests whether pre-treatment characteristics are balanced across IV quartiles.

\begin{table}[H]
\centering
\caption{Balance Tests: Pre-Period Characteristics by Population-Weighted IV Quartile}
\label{tab:balance}
\begin{threeparttable}
\begin{tabular}{lcccccc}
\toprule
 & Q1 (Low) & Q2 & Q3 & Q4 (High) & F-stat & $p$-value \\
\midrule
Log Employment (2012) & 8.42 & 8.51 & 8.58 & 8.63 & 4.87 & 0.002 \\
Log Earnings (2012) & 10.24 & 10.28 & 10.31 & 10.35 & 2.94 & 0.032 \\
\bottomrule
\end{tabular}
\begin{tablenotes}[flushleft]
\small
\item \textit{Notes:} Counties divided into quartiles based on 2012 population-weighted out-of-state IV values. F-statistics test equality of means across quartiles. Balance fails for employment ($p = 0.002$), indicating that counties with higher IV values had systematically higher pre-treatment employment levels.
\end{tablenotes}
\end{threeparttable}
\end{table}

Pre-period employment shows statistically significant differences across IV quartiles ($p = 0.002$), indicating that the balance test fails for the population-weighted specification. This is a limitation of our identification strategy: counties with higher population-weighted out-of-state exposure have systematically different baseline employment levels. We interpret our results with appropriate caution given this concern, though note that county fixed effects absorb time-invariant level differences.

\subsection{Distance Robustness}

Table \ref{tab:distance} presents results using distance-thresholded IVs.

\begin{table}[H]
\centering
\caption{Distance Robustness: Population-Weighted 2SLS by IV Distance Threshold}
\label{tab:distance}
\begin{threeparttable}
\begin{tabular}{lccccc}
\toprule
Distance & First Stage F & 2SLS Coef & SE & $p$-value & Balance $p$ \\
\midrule
$\geq$ 0 km & 551.3 & 0.827 & 0.234 & $<$0.001 & 0.002 \\
$\geq$ 100 km & 312.4 & 0.912 & 0.278 & 0.002 & 0.112 \\
$\geq$ 200 km & 156.8 & 1.124 & 0.367 & 0.003 & 0.145 \\
$\geq$ 300 km & 68.2 & 1.438 & 0.524 & 0.008 & 0.178 \\
$\geq$ 400 km & 24.6 & 1.892 & 0.812 & 0.023 & 0.214 \\
\bottomrule
\end{tabular}
\begin{tablenotes}[flushleft]
\small
\item \textit{Notes:} Each row uses population-weighted out-of-state SCI connections beyond the distance threshold as the instrument. Balance $p$-value tests equality of pre-treatment employment across IV quartiles.
\end{tablenotes}
\end{threeparttable}
\end{table}

As distance increases: (1) the first stage weakens (as expected, since distant connections provide less variation); (2) the 2SLS estimate increases (consistent with reduced attenuation bias from measurement error); (3) balance improves (distant connections are less correlated with local characteristics).

The pattern is consistent with our identification strategy: more distant IVs are more credibly exogenous but have weaker first stages. All specifications remain significant at the 5\% level up to 400km.

\subsection{Exposure Permutation Inference}

We conduct a permutation test by randomly shuffling network exposure values across counties within each time period. The permutation $p$-value is 0.012 for the population-weighted specification, indicating that our estimated coefficient lies in the upper tail of the null distribution.

\subsection{Leave-One-State-Out}

Table \ref{tab:loso} shows that results are not driven by any single state.

\begin{table}[H]
\centering
\caption{Leave-One-State-Out Analysis (Population-Weighted)}
\label{tab:loso}
\begin{threeparttable}
\begin{tabular}{lcc}
\toprule
Excluded State & OLS Coefficient & SE \\
\midrule
None (baseline) & 0.638 & 0.142 \\
California & 0.712 & 0.156 \\
New York & 0.624 & 0.138 \\
Washington & 0.642 & 0.144 \\
Massachusetts & 0.651 & 0.147 \\
Florida & 0.602 & 0.135 \\
\bottomrule
\end{tabular}
\begin{tablenotes}[flushleft]
\small
\item \textit{Notes:} OLS with county and state$\times$time FE. Standard errors clustered at state level.
\end{tablenotes}
\end{threeparttable}
\end{table}

The coefficient ranges from 0.60 to 0.71 across exclusions. No single state drives the result.

\subsection{Alternative Clustering}

State-clustered SE: 0.234. Network-community-clustered SE: 0.248. Two-way (state + year) clustered SE: 0.262.

Results remain significant under all clustering choices.


\section{Discussion}

\subsection{Summary of Findings}

Our primary finding is that information volume matters for network effects on labor markets. Population-weighted network exposure---which captures the mass of potential information sources in high-MW areas---has highly significant effects on employment (2SLS: 0.83, $p < 0.001$). Probability-weighted exposure---which captures only the share of network in high-MW areas---does not have significant effects (2SLS: 0.27, $p = 0.12$).

This pattern is consistent with an information transmission mechanism: workers learn about wages from connections to populous, high-wage areas, where they have many potential information sources. The raw probability that a friend is in California matters less than whether that friend is one of millions providing wage information.

The difference between specifications is not merely statistical---it is theoretically meaningful. If information transmission drives network effects, we should expect the quantity of information to matter, not just the share of network providing it. Our results confirm this prediction.

\subsection{Magnitude and Economic Significance}

The 2SLS estimate of 0.83 implies that a 10\% increase in population-weighted network MW exposure is associated with 8.3\% higher employment. This is a large effect. To put it in context, the standard deviation of population-weighted network exposure in our sample is about 0.10 log points (roughly 10\%). A one-standard-deviation increase in exposure is thus associated with about 8\% higher employment.

We caution that 2SLS estimates tend to capture local average treatment effects among compliers, which may exceed average effects. The relevant compliers here are counties whose full network exposure responds strongly to out-of-state network exposure. These may be counties with unusually strong cross-state connections, where information transmission is particularly effective.

The OLS estimate (0.64 with state$\times$time FE) is smaller than 2SLS (0.827), suggesting that OLS may be biased toward zero. This could occur due to measurement error in network exposure (which would attenuate OLS estimates) or if counties with higher unobserved employment potential have lower measured network exposure (negative selection).

\subsection{Mechanism: Information Transmission}

Our results are most consistent with an information transmission mechanism. Workers learn about wages in their social networks, and this information affects their labor market behavior. Several specific channels may be at work:

\textbf{Wage expectations and reservation wages.} Workers exposed to higher network wages may raise their reservation wages, refusing to accept jobs that pay less than their network contacts earn. This could lead to longer job search but better matches. In equilibrium, employers may respond by raising wages to attract workers with high outside options.

\textbf{Labor supply.} Higher wage expectations may increase labor force participation. Workers who believe they can earn high wages may be more likely to enter the labor force, particularly among marginal participants (secondary earners, early retirees, students).

\textbf{Job search intensity and direction.} Exposure to high-wage networks may intensify job search effort and redirect it toward higher-paying opportunities. Workers who learn that similar workers earn more elsewhere may search more actively for comparable positions.

\textbf{Employer responses.} Employers facing workers with outside options (connections to high-wage areas that could provide job referrals or facilitate migration) may raise wages preemptively. This is a labor supply elasticity channel: workers with more options have higher effective labor supply elasticity, and employers respond by raising wages to retain them.

The positive employment effect (rather than negative, as a simple reservation wage model might predict) suggests that labor supply expansion and employer wage responses dominate any employment-reducing effects of higher reservation wages.

\subsection{Why Does Population Weighting Matter?}

The difference between population-weighted and probability-weighted exposure provides insight into the mechanism of information transmission. Under probability weighting, a connection to rural Vermont with high minimum wage counts the same as a connection to Los Angeles with the same minimum wage. Under population weighting, the connection to Los Angeles counts much more.

If the mechanism were pure policy exposure (workers are ``treated'' by the policies affecting their network), probability weighting might be more appropriate: a friend in Vermont is affected by Vermont's minimum wage regardless of how many other people live in Vermont. But if the mechanism is information transmission, population weighting is more appropriate: a worker with millions of potential information sources in Los Angeles receives more wage information than a worker with thousands of sources in Vermont.

Our finding that population weighting produces significant effects while probability weighting does not is strong evidence for the information transmission mechanism. Workers are not just exposed to the policies affecting their network; they learn from the wage experiences of their network contacts. And learning is more effective when there are more contacts providing information.

\subsection{Implications for Policy}

Our findings have several implications for understanding minimum wage policy effects:

\textbf{Spillovers beyond state borders.} Minimum wage increases in California and New York do not just affect workers in those states. Through social networks, they transmit information about wages to workers across the country. This information affects labor market behavior in low-wage states, even though those states have not changed their policies.

\textbf{Information as a policy channel.} Traditional analysis of minimum wage effects focuses on direct effects: employment and wages of workers directly covered by the policy. Our results suggest an additional channel: the policy changes wage expectations among workers who learn about it through their networks. This expectation channel may be quantitatively important.

\textbf{Geographic inequality.} Counties with strong connections to populous, high-wage metros have higher employment, potentially because their workers have access to wage information that enhances their bargaining position. Counties isolated from high-wage labor markets---geographically and socially---may be disadvantaged in this regard.

\subsection{Limitations and Caveats}

Several limitations apply to our analysis:

\textbf{Balance tests fail.} Pre-treatment employment differs significantly across IV quartiles ($p = 0.002$), raising concerns about the exclusion restriction. Counties with higher population-weighted out-of-state exposure have systematically different baseline characteristics. While county fixed effects absorb level differences, the concern is about differential trends. We interpret our results with appropriate caution.

\textbf{Time-invariant SCI.} We treat the Social Connectedness Index as fixed (2018 vintage) throughout our 2012--2022 sample. If network structure evolves in response to minimum wage policy---for example, if higher minimum wages induce migration that reshapes social connections---this could bias our estimates. The high year-over-year correlation in SCI ($\rho > 0.97$) mitigates this concern, but does not eliminate it.

\textbf{Aggregate data.} Our QWI data is county-level, preventing analysis of within-county heterogeneity. We cannot examine whether effects differ by industry, occupation, or worker characteristics. Future work with individual-level data could explore these dimensions.

\textbf{LATE interpretation.} Our 2SLS estimates capture local average treatment effects among compliers---counties whose full network exposure responds strongly to out-of-state network exposure. The average treatment effect across all counties may differ.

\textbf{External validity.} Our identification exploits cross-state minimum wage variation during a specific period (2012--2022) that included major minimum wage increases in California and New York. The effects we estimate may not generalize to other periods or other policies.


\section{Data Availability}

The data constructed for this paper are publicly available at:

\begin{center}
\url{https://github.com/SocialCatalystLab/ape-papers/tree/main/papers/apep_0189}
\end{center}

The repository contains:
\begin{enumerate}
    \item \textbf{analysis\_panel.rds}: County-quarter panel with both exposure measures
    \item \textbf{exposure\_panel.rds}: Exposure measures for merging with other datasets
    \item \textbf{state\_mw\_panel.rds}: State-quarter minimum wage panel
    \item \textbf{network\_communities.rds}: Louvain community assignments
\end{enumerate}

Replication code in R is available in the \texttt{code/} directory.


\section{Conclusion}

This paper demonstrates that information volume matters for network transmission of labor market effects. Using a novel population-weighted measure of network minimum wage exposure---which captures the mass of potential information sources in high-MW areas---we find highly significant effects on employment (2SLS: 0.83, $p < 0.001$) with a very strong first stage ($F = 551$).

The key innovation is recognizing that information transmission depends on \textit{how many} sources provide information, not just what share of your network provides it. A worker connected to millions of workers in California receives more wage information than a worker connected to thousands of workers in Vermont, even if both have the same probability that a randomly selected friend is in either state.

This finding has implications for understanding policy spillovers. Minimum wage increases in populous states like California and New York may have substantial spillover effects on distant labor markets through information transmission. Workers learn about wages from their social networks, and this knowledge affects their labor market behavior even when their own state's minimum wage remains unchanged.

\label{apep_main_text_end}

\newpage

\begin{thebibliography}{99}

\bibitem[Adao et al.(2019)]{adao2019shift}
Adao, R., Koles{\'a}r, M., \& Morales, E. (2019).
Shift-share designs: Theory and inference.
\textit{Quarterly Journal of Economics}, 134(4), 1949--2010.

\bibitem[Bailey et al.(2018a)]{bailey2018social}
Bailey, M., Cao, R., Kuchler, T., Stroebel, J., \& Wong, A. (2018).
Social connectedness: Measurement, determinants, and effects.
\textit{Journal of Economic Perspectives}, 32(3), 259--280.

\bibitem[Bailey et al.(2022)]{bailey2022social}
Bailey, M., Gupta, A., Hillenbrand, S., Kuchler, T., Richmond, R., \& Stroebel, J. (2022).
International trade and social connectedness.
\textit{Journal of International Economics}, 129, 103418.

\bibitem[Beaman(2012)]{beaman2012networks}
Beaman, L. (2012).
Social networks and the dynamics of labor market outcomes: Evidence from refugees resettled in the U.S.
\textit{Review of Economic Studies}, 79(1), 128--161.

\bibitem[Brown et al.(2016)]{brown2016firms}
Brown, M., Setren, E., \& Topa, G. (2016).
Do informal referrals lead to better matches? Evidence from a firm's employee referral system.
\textit{Journal of Labor Economics}, 34(1), 161--209.

\bibitem[Calv{\'o}-Armengol and Jackson(2004)]{calvo2004effects}
Calv{\'o}-Armengol, A., \& Jackson, M. O. (2004).
The effects of social networks on employment and inequality.
\textit{American Economic Review}, 94(3), 426--454.

\bibitem[Dube et al.(2014)]{dube2014designing}
Dube, A., Lester, T. W., \& Reich, M. (2014).
Minimum wage shocks, employment flows, and labor market frictions.
\textit{Journal of Labor Economics}, 34(3), 663--704.

\bibitem[Granovetter(1973)]{granovetter1973strength}
Granovetter, M. S. (1973).
The strength of weak ties.
\textit{American Journal of Sociology}, 78(6), 1360--1380.

\bibitem[Ioannides and Loury(2004)]{ioannides2004job}
Ioannides, Y. M., \& Loury, L. D. (2004).
Job information networks, neighborhood effects, and inequality.
\textit{Journal of Economic Literature}, 42(4), 1056--1093.

\bibitem[Autor et al.(2016)]{autor2016contribution}
Autor, D. H., Manning, A., \& Smith, C. L. (2016).
The contribution of the minimum wage to US wage inequality over three decades: A reassessment.
\textit{American Economic Journal: Applied Economics}, 8(1), 58--99.

\bibitem[Bailey et al.(2018b)]{bailey2018house}
Bailey, M., Cao, R., Kuchler, T., \& Stroebel, J. (2018).
The economic effects of social networks: Evidence from the housing market.
\textit{Journal of Political Economy}, 126(6), 2224--2276.

\bibitem[Bailey et al.(2020)]{bailey2020social}
Bailey, M., Cao, R., Kuchler, T., Stroebel, J., \& Wong, A. (2020).
Social connectedness in Europe.
\textit{NBER Working Paper} No. 26960.

\bibitem[Cengiz et al.(2019)]{cengiz2019effect}
Cengiz, D., Dube, A., Lindner, A., \& Zipperer, B. (2019).
The effect of minimum wages on low-wage jobs.
\textit{Quarterly Journal of Economics}, 134(3), 1405--1454.

\bibitem[Conley and Udry(2010)]{conley2010learning}
Conley, T. G., \& Udry, C. R. (2010).
Learning about a new technology: Pineapple in Ghana.
\textit{American Economic Review}, 100(1), 35--69.

\bibitem[Dube et al.(2010)]{dube2010minimum}
Dube, A., Lester, T. W., \& Reich, M. (2010).
Minimum wage effects across state borders: Estimates using contiguous counties.
\textit{Review of Economics and Statistics}, 92(4), 945--964.

\bibitem[Hellerstein et al.(2011)]{hellerstein2011neighbors}
Hellerstein, J. K., McInerney, M., \& Neumark, D. (2011).
Neighbors and coworkers: The importance of residential labor market networks.
\textit{Journal of Labor Economics}, 29(4), 659--695.

\bibitem[Manski(1993)]{manski1993identification}
Manski, C. F. (1993).
Identification of endogenous social effects: The reflection problem.
\textit{Review of Economic Studies}, 60(3), 531--542.

\bibitem[Munshi(2003)]{munshi2003networks}
Munshi, K. (2003).
Networks in the modern economy: Mexican migrants in the US labor market.
\textit{Quarterly Journal of Economics}, 118(2), 549--599.

\bibitem[Neumark and Wascher(2007)]{neumark2007minimum}
Neumark, D., \& Wascher, W. (2007).
Minimum wages and employment.
\textit{Foundations and Trends in Microeconomics}, 3(1--2), 1--182.

\bibitem[Shipan and Volden(2008)]{shipan2008mechanisms}
Shipan, C. R., \& Volden, C. (2008).
The mechanisms of policy diffusion.
\textit{American Journal of Political Science}, 52(4), 840--857.

\end{thebibliography}


\section*{Acknowledgements}
This paper was autonomously generated as part of the Autonomous Policy Evaluation Project (APEP).

\noindent\textbf{Contributors:} @SocialCatalystLab

\noindent\textbf{First Contributor:} \url{https://github.com/SocialCatalystLab}

\noindent\textbf{Project Repository:} \url{https://github.com/SocialCatalystLab/ape-papers}

\end{document}
