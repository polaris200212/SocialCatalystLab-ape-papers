\documentclass[12pt]{article}

% UTF-8 encoding and fonts
\usepackage[utf8]{inputenc}
\usepackage[T1]{fontenc}
\usepackage{lmodern}

% Page setup
\usepackage[margin=1in]{geometry}
\usepackage{setspace}
\onehalfspacing

% Typography
\usepackage{microtype}

% Math and symbols
\usepackage{amsmath,amssymb}

% Graphics
\usepackage{graphicx}
\usepackage{float}
\usepackage{subcaption}

% Tables
\usepackage{booktabs}
\usepackage{array}
\usepackage{multirow}
\usepackage{threeparttable}
\usepackage{longtable}
\usepackage{pdflscape}
\usepackage{siunitx}
\sisetup{detect-all=true, group-separator={,}, group-minimum-digits=4}

% Bibliography
\usepackage{natbib}
\bibliographystyle{aer}

% Hyperlinks
\usepackage{hyperref}
\hypersetup{
    colorlinks=true,
    linkcolor=blue,
    citecolor=blue,
    urlcolor=blue
}
\usepackage[nameinlink,noabbrev]{cleveref}

% Timing data
\IfFileExists{timing_data.tex}{\newcommand{\apepcurrenttime}{1h 4m}
\newcommand{\apepcumulativetime}{1h 4m}
}{
  \newcommand{\apepcurrenttime}{N/A}
  \newcommand{\apepcumulativetime}{N/A}
}

% Captions
\usepackage{caption}
\captionsetup{font=small,labelfont=bf}

% Section formatting
\usepackage{titlesec}
\titleformat{\section}{\large\bfseries}{\thesection.}{0.5em}{}
\titleformat{\subsection}{\normalsize\bfseries}{\thesubsection}{0.5em}{}

% Custom commands
\newcommand{\E}{\mathbb{E}}
\newcommand{\Var}{\text{Var}}
\newcommand{\Cov}{\text{Cov}}
\newcommand{\ind}{\mathbb{I}}
\newcommand{\sym}[1]{\ifmmode^{#1}\else\(^{#1}\)\fi}

\title{Can Clean Cooking Save Lives? Evidence from India's Ujjwala Yojana}
\author{APEP Autonomous Research\thanks{Autonomous Policy Evaluation Project. This paper was generated autonomously. Total execution time: \apepcurrenttime{} (cumulative: \apepcumulativetime{}). Correspondence: scl@econ.uzh.ch} \and @olafdrw}
\date{\today}

\begin{document}

\maketitle

\begin{abstract}
\noindent
India's Pradhan Mantri Ujjwala Yojana distributed 100 million free LPG connections to below-poverty-line women---the world's largest clean cooking intervention. I exploit district-level variation in baseline solid fuel dependence, comparing NFHS-4 (2015--16) with NFHS-5 (2019--21) across 708 districts. The first stage is strong: high-exposure districts gained 14 percentage points more clean fuel adoption within states ($F = 19$). Reduced-form estimates show significant reductions in stunting ($-$8.3 pp) and underweight ($-$6.1 pp), but effects attenuate substantially when controlling for contemporaneous water and sanitation improvements. IV estimates suggest each percentage point of clean fuel adoption reduces stunting by 0.59 points. These results highlight both the promise and the identification challenges of evaluating infrastructure programs amid multiple concurrent interventions.
\end{abstract}

\vspace{1em}
\noindent\textbf{JEL Codes:} I15, O13, Q53, J16 \\
\noindent\textbf{Keywords:} clean cooking, indoor air pollution, child health, Ujjwala Yojana, India, LPG

\newpage

\section{Introduction}

A woman cooking with dung cakes in rural Bihar inhales the equivalent of 400 cigarettes per day in particulate matter exposure \citep{smith2014}. Her children, confined to the same smoke-filled kitchen, face heightened risks of acute respiratory infection, stunting, low birth weight, and premature death. Globally, household air pollution from cooking with solid fuels---wood, dung, crop residues, and coal---kills an estimated 3.2 million people per year, mostly women and children in developing countries \citep{who2022}. The burden falls hardest on India, where over 60 percent of households relied on biomass cooking fuels at the start of the last decade \citep{iea2017}.

The causal link between indoor air pollution (IAP) and health is biologically well-established but empirically difficult to isolate. Randomized trials of improved cookstoves have produced disappointingly mixed results, largely because adoption and sustained use remain low \citep{hanna2016, mortimer2017}. The policy frontier has therefore shifted from technology upgrades within biomass to wholesale fuel switching---from solid fuels to liquefied petroleum gas (LPG), electricity, or biogas. Yet no country has attempted fuel switching at the scale India did with Pradhan Mantri Ujjwala Yojana.

Launched in May 2016, Ujjwala provided free LPG connections to women in below-poverty-line (BPL) households. By 2021, over 100 million connections had been distributed, making it the world's largest clean cooking intervention by an order of magnitude. The program's stated goal was to ``safeguard the health of women and children by providing them with clean cooking fuel.'' Whether it achieved this goal is the question this paper addresses.

I exploit the program's targeting structure for identification. Because Ujjwala prioritized districts with low baseline clean fuel adoption, districts that were more dependent on solid fuels at baseline received greater treatment intensity. I measure this intensity using district-level clean fuel usage from the National Family Health Survey Round 4 (NFHS-4, 2015--16), which was collected immediately before Ujjwala's launch. The key identifying variation is the interaction between baseline solid fuel dependence and the post-Ujjwala period, comparing outcomes in NFHS-5 (2019--21) to NFHS-4 within states.

The first-stage relationship is strong and precisely estimated. A one-standard-deviation increase in baseline solid fuel dependence (approximately 25 percentage points) is associated with a 14 percentage point larger increase in clean fuel adoption within states (first-stage $F = 19$, well above conventional weak-instrument thresholds). This dramatic convergence in clean fuel access is visible across all states and robust to controlling for baseline electricity, sanitation, water access, and female literacy.

The reduced-form health results are more nuanced. With state fixed effects and baseline controls, districts with greater Ujjwala exposure experienced significant reductions in child stunting ($-$8.3 percentage points, $p < 0.01$) and underweight prevalence ($-$6.1 pp, $p < 0.01$). The effect on diarrhea prevalence, while negative ($-$2.8 pp), is not statistically significant. Women's anemia shows no detectable response.

However, robustness checks reveal an important identification challenge. When I control for contemporaneous improvements in sanitation (Swachh Bharat Mission) and water access (Jal Jeevan Mission), the clean fuel coefficient attenuates substantially for stunting (from $-$8.3 to $-$0.9 pp) and becomes essentially zero for diarrhea. A ``horse race'' specification including all four infrastructure gaps simultaneously confirms that water access, not cooking fuel, is the dominant predictor of diarrhea reduction, while cooking fuel and water both predict nutritional improvements. Placebo tests are mixed: the electricity gap (placebo treatment) does not predict diarrhea changes, but it does predict stunting and underweight changes, suggesting that general development trends contaminate the nutritional outcomes.

These findings yield two contributions. First, I provide the first district-level causal evidence on Ujjwala's health effects, documenting both a strong supply-side success (fuel switching) and more ambiguous demand-side outcomes (health). The IV estimate suggests that each percentage point increase in clean fuel adoption reduces stunting by 0.59 points ($p < 0.01$), a meaningful effect size, but the exclusion restriction is threatened by correlated development trends.

Second, the paper illustrates a general methodological point about evaluating large-scale infrastructure programs in developing countries. India simultaneously rolled out Ujjwala (cooking fuel), Swachh Bharat (sanitation), Jal Jeevan (water), Saubhagya (electricity), and Ayushman Bharat (health insurance) during 2015--2021. These programs target the same disadvantaged populations and their effects are deeply confounded in any district-level analysis. My horse-race specification---entering all infrastructure gaps simultaneously---offers one strategy for disentangling them, but the fundamental limitation of a two-period design without true pre-treatment data remains.

This paper contributes to three literatures. First, it speaks to the large literature on indoor air pollution and health in developing countries \citep{duflo2008, smith2011, greenstone2015, jeuland2015, gould2018}. The randomized evidence on cookstoves is notably mixed. \citet{hanna2016} found that improved cookstoves distributed in Orissa were abandoned within two years, with no lasting health benefits. \citet{mortimer2017} conducted a large cluster-randomized trial of cleaner cookstoves in Malawi and found no reduction in childhood pneumonia despite meaningful reductions in carbon monoxide exposure. \citet{smith2011} found a 33\% reduction in physician-diagnosed pneumonia in the RESPIRE trial in Guatemala, but the intervention was an improved biomass stove rather than a fuel switch. The key insight from this literature is that technology improvements within biomass may be insufficient---wholesale fuel switching to LPG or electricity is the policy frontier. Ujjwala represents the first attempt to achieve fuel switching at national scale, and our results---while confounded---provide the first district-level evidence on health effects.

Second, it contributes to research on India's development programs and their interactions \citep{pattanayak2019, kar2019, gould2019}. The existing literature evaluates each program in isolation. \citet{kar2019} used commercial LPG sales data to estimate Ujjwala's impact on refill consumption and found that the program increased average refills per connection but that many beneficiaries remained below the threshold for meaningful health gains (typically estimated at 4--6 refills per year). \citet{gould2019} documented persistent fuel stacking---households using LPG for some cooking tasks while continuing to burn biomass for others---even among subsidized LPG users. Our finding that health effects attenuate when controlling for concurrent programs adds an important dimension: even if Ujjwala successfully increases LPG adoption, isolating its health contribution from the broader development transformation is empirically challenging.

Third, it contributes methodologically to the growing literature on identification challenges in multi-treatment settings \citep{goldsmith2022, roth2023, goldsmithpinkham2020}. The baseline-gap-times-post design, while intuitive and widely used, is vulnerable to mean reversion and correlated treatment when multiple programs target the same populations. \citet{goldsmith2022} formalize the ``contamination bias'' that arises when treatment effects are heterogeneous and the implicit comparison weights placed on different units are not aligned with the target estimand. Our horse-race specification is a practical response to this concern, but it cannot fully resolve the collinearity problem when multiple treatments are highly correlated.

Section 2 describes the institutional background. Section 3 presents the data. Section 4 develops the empirical strategy. Section 5 reports results. Section 6 discusses implications and limitations. Section 7 concludes.


\section{Institutional Background}

\subsection{Indoor Air Pollution in India}

India's cooking fuel landscape at the start of Ujjwala was starkly divided. According to NFHS-4 (2015--16), only 30 percent of districts had clean fuel adoption rates above 50 percent, and these were concentrated in urbanized states like Delhi, Goa, and Kerala. In the median district, fewer than one in four households cooked with clean fuel. The remaining households relied on firewood (68 percent of rural households), dung cakes (12 percent), crop residues (7 percent), and coal or charcoal (3 percent) \citep{nfhs4}.

The health consequences are severe. Burning biomass in traditional \textit{chulhas} (clay stoves) produces concentrations of PM$_{2.5}$ that are 10--40 times the WHO guideline of 15 $\mu$g/m$^3$ \citep{smith2014}. Women, who do the vast majority of cooking in Indian households, experience chronic exposure lasting 3--7 hours daily. Children under five, who remain near their mothers, are the most vulnerable: their smaller airways and developing lungs make them disproportionately susceptible to particulate matter.

The epidemiological literature links biomass smoke exposure to acute respiratory infections (ARI) in children, chronic obstructive pulmonary disease (COPD) in women, low birth weight, cataracts, and cardiovascular disease \citep{smith2011, fullerton2008}. For child nutrition specifically, the pathway operates through two channels: (i) direct morbidity from respiratory infection, which reduces nutrient absorption and increases metabolic demands, and (ii) time savings from faster cooking, which can be reallocated to childcare, breastfeeding, or income generation.

\subsection{Pradhan Mantri Ujjwala Yojana (PMUY)}

The Government of India launched PMUY on May 1, 2016, with an initial target of providing 50 million free LPG connections to BPL women by 2019. The program was subsequently expanded: the target was raised to 80 million by March 2020, and under ``Ujjwala 2.0'' (August 2021), eligibility was extended to all poor households, including migrant families, with a target of 10 million additional connections.

The program's design had several key features relevant to identification:

\textbf{Targeting:} Connections were provided to adult women in households identified as below the poverty line in the Socio-Economic and Caste Census (SECC) 2011. This targeting means that districts with more BPL households---which are precisely the districts with lower baseline clean fuel adoption---received more connections per capita.

\textbf{Connection vs. usage:} PMUY provided the LPG connection (cylinder, regulator, hose) for free. However, refilling a 14.2 kg cylinder costs approximately Rs 800--1,000 (roughly \$10--12), which represents a significant expense for BPL households. Studies have documented substantial ``refill dropout,'' with 30--40 percent of Ujjwala beneficiaries consuming fewer than three refills per year \citep{kar2019, gould2018}. This incomplete compliance is important for interpreting our estimates as intention-to-treat effects.

\textbf{Scale and pace:} The program distributed 1.5 million connections in FY 2016--17 (its first partial year), 30.7 million in FY 2017--18, 32.6 million in FY 2018--19, and 14.5 million in FY 2019--20. The bulk of distribution occurred between April 2017 and March 2019, which falls squarely between the NFHS-4 (2015--16) and NFHS-5 (2019--21) survey rounds.

\textbf{Concurrent programs:} Ujjwala was not the only infrastructure program targeting disadvantaged districts during this period. The Swachh Bharat Mission (SBM, October 2014) built 100 million toilets; the Jal Jeevan Mission (JJM, August 2019) provided piped water connections; the Saubhagya scheme (September 2017) provided universal electricity access. These programs targeted overlapping populations and confound any attempt to isolate Ujjwala's health effects. I address this challenge directly in the empirical strategy.


\section{Data}

\subsection{National Family Health Survey (NFHS)}

The primary data source is the National Family Health Survey, India's implementation of the Demographic and Health Survey (DHS) program. I use district-level factsheet data from NFHS-4 (2015--16) and NFHS-5 (2019--21), which provide comparable estimates for over 50 health, nutrition, and development indicators across 708 districts.

The key advantage of the NFHS factsheets is comprehensive district coverage with standardized indicators across both rounds. The key limitation is that each round produces a single cross-sectional estimate per district, yielding only two time periods. This precludes event-study designs or pre-trend tests within the NFHS data.

\textbf{Treatment variable:} Baseline solid fuel dependence is measured as $(100 - \text{NFHS-4 clean fuel \%}) / 100$. This ``Ujjwala exposure'' variable ranges from 0.03 (nearly universal clean fuel at baseline) to 1.0 (zero clean fuel at baseline), with a mean of 0.70 and standard deviation of 0.25.

\textbf{First-stage outcome:} Change in clean fuel usage (NFHS-5 minus NFHS-4), measured in percentage points. The mean increase is 24.2 pp (SD = 22.8), reflecting Ujjwala's dramatic impact on fuel adoption.

\textbf{Health outcomes:} Changes in (i) diarrhea prevalence among children under 5 in the two weeks preceding the survey, (ii) stunting (height-for-age below $-$2 SD), and (iii) underweight (weight-for-age below $-$2 SD). The NFHS also reports acute respiratory infection (ARI) symptoms and women's anemia, but high rates of missing data across districts limit their inclusion in the main analysis.

\textbf{Placebo outcomes:} Changes in full vaccination coverage and institutional birth rates, which should not respond to cooking fuel changes.

\textbf{Baseline controls:} NFHS-4 district-level values for electricity access, improved sanitation, improved water source, female literacy, and institutional births.

\textbf{Concurrent program measures:} To address confounding from other infrastructure programs, I construct analogous ``gap'' variables for electricity ($100 - \text{NFHS-4 electricity \%}$), sanitation ($100 - \text{NFHS-4 improved sanitation \%}$), and water ($100 - \text{NFHS-4 improved water \%}$). These gaps are highly correlated with the fuel gap: pairwise correlations range from 0.65 (fuel-electricity) to 0.82 (fuel-water), reflecting the systematic disadvantage of districts with low infrastructure access. I also construct changes in these indicators between NFHS rounds to use as controls in augmented specifications.

\subsection{Geographic Coverage and District Matching}

The analysis requires matching districts across NFHS-4 and NFHS-5, which is complicated by administrative redistricting between survey rounds. NFHS-4 covers approximately 640 districts, while NFHS-5 covers over 700, reflecting the creation of new districts through splitting of existing ones. The GitHub-hosted dataset harmonizes these by mapping new districts back to their parent NFHS-4 district boundaries where possible. The matched dataset yields 708 district-level observations with non-missing clean fuel data in both rounds.

The sample spans all 36 states and union territories. The largest states by number of districts are Uttar Pradesh (75), Madhya Pradesh (52), and Maharashtra (36). Small union territories (Chandigarh, Lakshadweep, Daman and Diu) contribute 1--2 districts each. State fixed effects absorb cross-state variation, so our identification comes from within-state differences across districts---effectively asking whether districts within Bihar or within Tamil Nadu that had lower clean fuel usage experienced differentially larger health improvements.

\subsection{Timing Considerations}
\label{sec:timing}

A critical detail for identification is that NFHS-4 fieldwork spanned January 2015 to December 2016, overlapping Ujjwala's May 2016 launch. However, three features mitigate contamination. First, most districts were surveyed before mid-2016, as the state-by-state survey schedule began in 2015. Second, Ujjwala distributed only 1.5 million connections in its first year---less than 2 percent of its eventual total. Third, any contamination biases our estimates \textit{toward zero}: if NFHS-4 partially captures early Ujjwala effects, the pre/post difference shrinks.

NFHS-5 fieldwork similarly spans a wide window (2019--2021), during which Ujjwala continued to distribute connections under the 2.0 expansion. The COVID-19 pandemic disrupted fieldwork in many states, with survey teams resuming data collection at different times. This introduces measurement noise into the post-period outcomes but does not systematically bias the cross-district comparison if pandemic disruptions were uncorrelated with baseline fuel dependence conditional on state fixed effects.

The effective treatment period between NFHS rounds is approximately 3--5 years, during which the bulk of Ujjwala connections were distributed (2017--2019). This window is long enough for clean fuel adoption to be observed but potentially too short for downstream health effects---particularly nutritional outcomes like stunting---to fully materialize. The biological literature suggests that improvements in indoor air quality require sustained exposure reductions over 1--2 years to affect chronic nutritional outcomes \citep{pope2010}.

\subsection{Summary Statistics}

\Cref{tab:summary} presents summary statistics for key variables. Clean fuel usage increased dramatically between survey rounds, from a mean of 29.9\% (NFHS-4) to 54.1\% (NFHS-5), a 24.2 percentage point gain. This gain was not uniform: the standard deviation of changes (22.8 pp) reflects large heterogeneity across districts, which is the variation our identification exploits.

Health indicators present a more complex picture. Diarrhea prevalence declined slightly ($-$0.4 pp), while stunting and underweight prevalence actually \textit{increased} on average (+4.2 pp and +2.9 pp). This aggregate worsening likely reflects measurement changes between NFHS rounds and the COVID-19 pandemic's impact on nutrition during NFHS-5 fieldwork (2019--2021). The question is whether districts with greater Ujjwala exposure experienced relatively better outcomes despite these aggregate trends.

\begin{table}[htbp]
\centering
\caption{Summary Statistics by Treatment Status}
\label{tab:summary_stats}
\begin{tabular}{lccc}
\hline\hline
Variable & Not Afraid & Afraid & Overall \\
\hline
$N$ & 28,704 & 18,384 & 47,088 \\
\hline
Age & 46.5 & 47.4 & 46.9 \\
Female (%) & 43.6 & 74.1 & 55.5 \\
Black (%) & 12.0 & 18.2 & 14.4 \\
Education (years) & 13.3 & 12.8 & 13.1 \\
College+ (%) & 26.7 & 22.2 & 24.9 \\
Married (%) & 54.9 & 45.8 & 51.4 \\
Parent Educ (years) & 11.4 & 11.0 & 11.3 \\
Real Income ($) & \$35,466 & \$28,156 & \$32,647 \\
Conservative (%) & 34.3 & 31.9 & 33.4 \\
Urban (%) & 27.7 & 42.1 & 33.3 \\
\hline
Death Pen. Support (%) & 70.4 & 67.2 & 69.1 \\
Courts Lenient (%) & 74.4 & 80.9 & 77.0 \\
More Crime Spend (%) & 65.3 & 72.2 & 68.1 \\
\hline\hline
\end{tabular}
\begin{tablenotes}
\small
\item \textit{Notes:} Data from the General Social Survey, 1973--2024.
Treatment is defined as reporting fear of walking alone at night near home.
Real income in 2024 dollars.
\end{tablenotes}
\end{table}



\section{Empirical Strategy}

\subsection{Identification}

The identification strategy exploits cross-district variation in baseline solid fuel dependence as a continuous measure of Ujjwala treatment intensity. The key insight is that districts with lower clean fuel adoption at baseline had more BPL households eligible for Ujjwala connections and therefore received higher treatment intensity per capita.

The primary specification estimates the reduced-form relationship:
\begin{equation}
\Delta Y_d = \alpha + \beta \cdot \text{FuelGap}_d + X_d' \gamma + \mu_s + \varepsilon_d
\label{eq:reduced_form}
\end{equation}
where $\Delta Y_d$ is the change in a health outcome from NFHS-4 to NFHS-5 in district $d$; $\text{FuelGap}_d = (100 - \text{NFHS-4 clean fuel}_d) / 100$ is the baseline treatment intensity; $X_d'$ is a vector of baseline controls (electricity, sanitation, water access, female literacy); and $\mu_s$ are state fixed effects. Standard errors are heteroskedasticity-robust (HC1). Results are qualitatively similar with state-level clustering, though inference weakens for marginally significant estimates given only 36 clusters \citep{cameron2008}.

The coefficient $\beta$ captures the differential change in outcomes per unit of Ujjwala exposure, conditional on state and baseline development levels. A negative $\beta$ for health outcomes (diarrhea, stunting) indicates that higher-exposure districts improved more.

\subsection{Instrumental Variables}

To estimate the causal effect of clean fuel adoption itself (rather than the intention-to-treat effect of exposure), I use an IV specification:
\begin{align}
\text{First stage:} \quad \Delta \text{CleanFuel}_d &= \pi \cdot \text{FuelGap}_d + X_d' \delta + \mu_s + \nu_d \label{eq:first_stage} \\
\text{Second stage:} \quad \Delta Y_d &= \alpha + \beta^{IV} \cdot \widehat{\Delta \text{CleanFuel}}_d + X_d' \gamma + \mu_s + \varepsilon_d \label{eq:iv}
\end{align}
The IV estimate $\beta^{IV}$ represents the local average treatment effect (LATE) of clean fuel adoption on health outcomes, identified off the Ujjwala-induced component of fuel switching. The exclusion restriction requires that baseline solid fuel dependence affects health changes \textit{only} through its effect on clean fuel adoption, conditional on controls and state fixed effects.

This exclusion restriction is potentially violated if baseline fuel dependence correlates with other determinants of health improvement---precisely the concurrent intervention problem discussed above. Two features of the design partially address this concern. First, state fixed effects remove state-level confounding from programs administered at the state level. Second, baseline controls for electricity, sanitation, and water access absorb level differences in infrastructure that might predict differential health trajectories. However, if the \textit{changes} in these infrastructure measures are correlated with baseline fuel dependence (as they are---see the horse-race results in \Cref{tab:horse_race}), the exclusion restriction fails. The IV estimates should therefore be interpreted as upper bounds on the true causal effect of clean fuel adoption, acknowledging that some of the instrumented variation may operate through correlated development channels.

\subsection{Panel Specification}

As a complement to the first-differences approach, I estimate a two-period panel with district and period fixed effects:
\begin{equation}
Y_{dt} = \alpha_d + \lambda_t + \delta \cdot \text{FuelGap}_d \times \text{Post}_t + \varepsilon_{dt}
\label{eq:panel}
\end{equation}
where $\alpha_d$ absorbs time-invariant district characteristics and $\lambda_t$ absorbs common time shocks. Standard errors are clustered at the state level.

\subsection{Threats to Validity}

The design faces three principal threats.

\textbf{Correlated treatments.} The most serious concern is that districts with low baseline clean fuel also had low sanitation, water, and electricity access, and all four infrastructure gaps narrowed simultaneously during 2015--2021 as the government pursued multiple development programs targeting the same populations. I address this with a ``horse race'' specification that enters all four infrastructure gaps simultaneously, and by directly controlling for changes in water and sanitation coverage.

\textbf{Mean reversion.} Districts starting from lower clean fuel levels may have experienced larger gains mechanically (bounded above at 100\%). State fixed effects help by comparing districts within the same state, but residual mean reversion remains a concern. The placebo tests---using electricity gap as a fake treatment---assess whether the results simply reflect catch-up growth in all development indicators.

\textbf{NFHS-4 timing overlap.} As discussed in \Cref{sec:timing}, some NFHS-4 fieldwork occurred after Ujjwala's launch. This attenuates estimates toward zero and does not threaten internal validity, but it means our estimates may understate the true effect.


\section{Results}

\subsection{First Stage: Ujjwala Exposure and Clean Fuel Adoption}

The program moved the needle on fuel adoption. Without controls, a one-unit increase in Ujjwala exposure predicts a 44.2 pp increase in clean fuel adoption between rounds---a mechanical mean-reversion relationship that is unsurprising (\Cref{tab:first_stage}, Column 1).

The more informative estimate conditions on state fixed effects and baseline infrastructure. Within states, a one-standard-deviation increase in Ujjwala exposure (0.25 units) produced approximately 3.5 additional percentage points of clean fuel adoption ($\beta = 14.1$, $SE = 3.0$, $t = 4.7$; \Cref{tab:first_stage}, Column 3). This was not a marginal shift---it represented a wholesale transformation of how the most fuel-deprived districts cooked.

\Cref{fig:first_stage} visualizes the first-stage relationship. The positive slope is evident: districts with greater baseline solid fuel dependence experienced larger increases in clean fuel usage. \Cref{fig:fuel_by_tercile} shows the convergence pattern by exposure tercile---high-exposure districts started with clean fuel usage below 10\% and climbed to approximately 40\%, while low-exposure districts started near 60\% and reached about 80\%.

\begin{table}
\centering
\caption{First Stage: Effect of Nitrogen Ruling on Building Permits}
\centering
\begin{tabular}[t]{lcccc}
\toprule
  & (1) Basic & (2) Prov x Qtr FE & (3) Permits/1000 & (4) Pre-COVID\\
\midrule
N2000 Share $\times$ Post & -13.415** & -7.218 & -0.041 & -3.207\\
 & (6.220) & (7.562) & (0.277) & (8.288)\\
\midrule
Num.Obs. & 17704 & 17704 & 17704 & 10014\\
R2 & 0.497 & 0.512 & 0.118 & 0.557\\
FE: muni_code & X & X & X & X\\
FE: yq & X & X & X & X\\
\bottomrule
\multicolumn{5}{l}{\rule{0pt}{1em}* p $<$ 0.1, ** p $<$ 0.05, *** p $<$ 0.01}\\
\multicolumn{5}{l}{\rule{0pt}{1em}Clustered standard errors at municipality level in parentheses. Post = 2019Q2 onwards. N2000 Share is the fraction of municipality area designated as Natura 2000. Column (4) restricts to pre-COVID period (through 2019Q4).}\\
\end{tabular}
\end{table}


\begin{figure}[H]
\centering
\includegraphics[width=0.85\textwidth]{figures/fig2_first_stage.pdf}
\caption{First Stage: Baseline Solid Fuel Dependence Predicts Clean Fuel Adoption}
\label{fig:first_stage}
\end{figure}

\begin{figure}[H]
\centering
\includegraphics[width=0.85\textwidth]{figures/fig4_fuel_by_tercile.pdf}
\caption{Clean Fuel Adoption by Ujjwala Exposure Tercile}
\label{fig:fuel_by_tercile}
\end{figure}


\subsection{Reduced-Form Health Effects}

Did this fuel switching translate into healthier children? The reduced-form estimates with state fixed effects and baseline controls suggest it may have (\Cref{tab:reduced_form}).

Stunting---a marker of chronic nutritional deprivation---fell by 8.3 percentage points more in high-exposure districts ($SE = 2.4$, $p < 0.01$). For a typical one-standard-deviation increase in exposure, the implied effect is approximately 2.1 pp, or roughly 6 percent of the mean stunting rate. In a district with 50,000 children under five, this represents roughly 1,000 fewer children with permanently impaired growth. Underweight prevalence shows a comparable pattern ($-$6.1 pp, $SE = 2.4$, $p < 0.01$).

Diarrhea, by contrast, did not respond meaningfully: $-$2.8 pp ($SE = 2.3$, $p = 0.22$). This null is unsurprising. Diarrhea is primarily waterborne, transmitted through contaminated water and inadequate sanitation, and the direct causal pathway from cooking fuel to gastrointestinal disease is weak.


\begin{table}[htbp]
   \caption{\label{tab:reduced_form} Reduced Form: Ujjwala Exposure and Child Health Outcomes}
   \centering
   \begin{tabular}{lccc}
      \tabularnewline \midrule \midrule
      Dependent Variables:  & delta\_diarrhea    & delta\_stunting    & delta\_underweight\\   
                            & $\Delta$ Diarrhea  & $\Delta$ Stunting  & $\Delta$ Underweight \\    
      Model:                & (1)                & (2)                & (3)\\  
      \midrule
      \emph{Variables}\\
      Ujjwala exposure      & -2.822             & -8.290$^{***}$     & -6.129$^{***}$\\   
                            & (2.280)            & (2.406)            & (2.372)\\   
      Baseline electricity  & -1.283             & -13.71$^{***}$     & -22.21$^{***}$\\   
                            & (2.164)            & (3.354)            & (3.093)\\   
      Baseline sanitation   & -3.498             & 1.641              & 9.306$^{***}$\\   
                            & (2.284)            & (2.790)            & (2.751)\\   
      Baseline water access & -6.472$^{***}$     & -29.66$^{***}$     & -19.14$^{***}$\\   
                            & (1.907)            & (2.931)            & (2.481)\\   
      \midrule
      \emph{Fixed-effects}\\
      state\_code           & Yes                & Yes                & Yes\\  
      \midrule
      \emph{Fit statistics}\\
      Observations          & 708                & 708                & 708\\  
      R$^2$                 & 0.49159            & 0.82889            & 0.80422\\  
      Within R$^2$          & 0.16858            & 0.67734            & 0.60971\\  
      \midrule \midrule
      \multicolumn{4}{l}{\emph{Heteroskedasticity-robust standard-errors in parentheses}}\\
      \multicolumn{4}{l}{\emph{Signif. Codes: ***: 0.01, **: 0.05, *: 0.1}}\\
   \end{tabular}
   
   \par \raggedright 
   HC1 robust standard errors in parentheses. State fixed effects included. Dependent variables: change in health indicator (pp) from NFHS-4 to NFHS-5.
\end{table}




\Cref{fig:health_by_tercile} visualizes the reduced-form pattern. For all three health outcomes, high-exposure districts show convergence toward the levels of low-exposure districts between NFHS-4 and NFHS-5.

\begin{figure}[H]
\centering
\includegraphics[width=\textwidth]{figures/fig3_health_by_tercile.pdf}
\caption{Child Health Outcomes by Ujjwala Exposure Tercile}
\label{fig:health_by_tercile}
\end{figure}


\subsection{IV Estimates}

\Cref{tab:iv} presents the IV estimates, using baseline fuel gap as an instrument for actual clean fuel adoption changes. The IV coefficient for stunting is $-$0.59 ($SE = 0.21$, $p < 0.01$): each percentage point increase in clean fuel adoption, instrumented by baseline fuel gap, reduces stunting by 0.59 points. The first-stage $F$-statistic is 18.9 (reported in \Cref{tab:iv}), well above the \citet{oster2019} threshold of 10 for weak instruments.

For diarrhea, the IV coefficient is $-$0.20 ($SE = 0.16$, $p = 0.22$), correctly signed but not significant. The IV estimates are larger in magnitude than the reduced-form coefficients divided by the first stage, reflecting the scaling by incomplete first-stage compliance.


\begin{table}[htbp]
   \caption{\label{tab:iv} IV Estimates: Clean Fuel Adoption and Child Health}
   \centering
   \begin{tabular}{lcc}
      \tabularnewline \midrule \midrule
      Dependent Variables:                     & delta\_diarrhea  & delta\_stunting\\   
                                               & Diarrhea         & Stunting \\   
      Model:                                   & (1)              & (2)\\  
      \midrule
      \emph{Variables}\\
      $\Delta$ Clean fuel                      & -0.2009          & -0.5901$^{***}$\\   
                                               & (0.1642)         & (0.2057)\\   
      Baseline electricity                     & -2.832           & -18.26$^{***}$\\   
                                               & (2.705)          & (4.622)\\   
      Baseline sanitation                      & -4.348           & -0.8565\\   
                                               & (2.843)          & (4.050)\\   
      Baseline water access                    & -10.08$^{***}$   & -40.25$^{***}$\\   
                                               & (3.755)          & (5.129)\\   
      \midrule
      \emph{Fixed-effects}\\
      state\_code                              & Yes              & Yes\\  
      \midrule
      \emph{Fit statistics}\\
      Observations                             & 708              & 708\\  
      R$^2$                                    & 0.43802          & 0.73104\\  
      F-test (1st stage), $\Delta$ Clean fuel  & 18.938           & 18.938\\  
      \midrule \midrule
      \multicolumn{3}{l}{\emph{Heteroskedasticity-robust standard-errors in parentheses}}\\
      \multicolumn{3}{l}{\emph{Signif. Codes: ***: 0.01, **: 0.05, *: 0.1}}\\
   \end{tabular}
   
   \par \raggedright 
   Instrument: baseline Ujjwala exposure. State FE and baseline controls included. HC1 robust SE.
\end{table}





\subsection{Robustness and Placebo Tests}

The preceding results must be interpreted with caution, as several robustness checks reveal identification challenges.

\textbf{Placebo treatments.} Using the baseline electricity gap as a placebo treatment (\Cref{tab:placebos}, Row 2), I find a null effect on diarrhea changes ($\beta = 1.3$, $p = 0.55$), supporting identification for this outcome. However, unreported regressions show that the electricity gap significantly predicts stunting and underweight changes, suggesting that general development convergence---not just clean fuel adoption---drives the nutritional results. This is the most concerning finding for the paper's main claim.

\textbf{Placebo outcomes.} The Ujjwala exposure variable significantly predicts vaccination improvements ($\beta = 16.6$, $p = 0.04$), which should not respond to cooking fuel. It does not predict institutional birth changes ($\beta = -2.1$, $p = 0.46$). The vaccination result indicates that Ujjwala exposure proxies for general development intensity, not just fuel switching.

\textbf{Controlling for concurrent programs.} When I add changes in sanitation and water coverage as controls (\Cref{tab:horse_race}), the Ujjwala coefficient for diarrhea attenuates from $-$2.8 to $+$0.9 (essentially zero), while the stunting coefficient attenuates from $-$8.3 to $-$0.9 (an 89\% reduction). Changes in water access strongly predict changes in all health outcomes ($p < 0.01$ across columns). This dramatic attenuation suggests that the apparent Ujjwala health effect in the baseline specification is confounded by correlated improvements in water and sanitation. The coefficients on $\Delta$ water and $\Delta$ sanitation should not be interpreted causally---they are endogenous contemporaneous changes---but their inclusion effectively absorbs the variation that previously loaded onto the Ujjwala coefficient.


\begin{table}[htbp]
   \caption{\label{tab:horse_race} Controlling for Concurrent Infrastructure Changes}
   \centering
   \begin{tabular}{lccc}
      \tabularnewline \midrule \midrule
      Dependent Variables:   & delta\_diarrhea    & delta\_stunting    & delta\_underweight\\   
                             & $\Delta$ Diarrhea  & $\Delta$ Stunting  & $\Delta$ Underweight \\    
      Model:                 & (1)                & (2)                & (3)\\  
      \midrule
      \emph{Variables}\\
      Ujjwala exposure       & 0.8919             & -0.9025            & -1.679\\   
                             & (2.020)            & (2.064)            & (2.058)\\   
      $\Delta$ Sanitation    & -0.0298            & -0.0640$^{*}$      & -0.0648$^{**}$\\   
                             & (0.0253)           & (0.0348)           & (0.0323)\\   
      $\Delta$ Water access  & 0.1002$^{***}$     & 0.4352$^{***}$     & 0.3754$^{***}$\\   
                             & (0.0121)           & (0.0205)           & (0.0207)\\   
      \midrule
      \emph{Fixed-effects}\\
      state\_code            & Yes                & Yes                & Yes\\  
      \midrule
      \emph{Fit statistics}\\
      Observations           & 708                & 708                & 708\\  
      R$^2$                  & 0.49178            & 0.83839            & 0.80197\\  
      Within R$^2$           & 0.16889            & 0.69525            & 0.60523\\  
      \midrule \midrule
      \multicolumn{4}{l}{\emph{Heteroskedasticity-robust standard-errors in parentheses}}\\
      \multicolumn{4}{l}{\emph{Signif. Codes: ***: 0.01, **: 0.05, *: 0.1}}\\
   \end{tabular}
   
   \par \raggedright 
   HC1 robust SE. State FE included. Controls for changes in sanitation and water coverage between NFHS rounds to address confounding from Swachh Bharat Mission and Jal Jeevan Mission.
\end{table}




\textbf{Dose-response.} Quartile analysis shows a monotonic pattern for diarrhea (Q2: $-$0.5, Q3: $-$1.3, Q4: $-$1.7 pp relative to Q1), but none of the quartile coefficients are individually significant. The first stage also shows a monotonic pattern, though the magnitude does not increase proportionally at the highest exposure levels, consistent with diminishing returns or refill dropout among the poorest households.

\textbf{Leave-one-state-out.} The reduced-form diarrhea coefficient ranges from $-$3.8 to +0.1 across 36 state exclusions, bracketing the full-sample estimate of $-$2.8 (\Cref{fig:loso}). The narrow range confirms that no single state drives the result. The first-stage coefficient is stable across all exclusions (range: 10.4 to 15.8 pp).

\Cref{fig:placebo} visualizes the placebo test for diarrhea, confirming that the real treatment (clean fuel gap) produces a negative coefficient while the placebo (electricity gap) is null. \Cref{fig:loso} shows the leave-one-state-out sensitivity.

\begin{figure}[H]
\centering
\includegraphics[width=0.75\textwidth]{figures/fig6_placebo_test.pdf}
\caption{Placebo Test: Real vs. Placebo Treatment Effect on Diarrhea}
\label{fig:placebo}
\end{figure}

\begin{figure}[H]
\centering
\includegraphics[width=0.75\textwidth]{figures/fig7_loso_sensitivity.pdf}
\caption{Leave-One-State-Out Sensitivity: Diarrhea Coefficient}
\label{fig:loso}
\end{figure}

\begin{table}[htbp]
\centering
\caption{Placebo Tests}
\label{tab:placebos}
\begin{tabular}{lcc}
\hline\hline
Test & Coefficient & SE \\
\hline
Fuel gap $\rightarrow$ $\Delta$ Diarrhea (REAL) & -2.822 & 2.28 \\
Electricity gap $\rightarrow$ $\Delta$ Diarrhea (PLACEBO TREATMENT) & 1.3 & 2.177 \\
Fuel gap $\rightarrow$ $\Delta$ Vaccination (PLACEBO OUTCOME) & 16.64 & 8.066 \\
Fuel gap $\rightarrow$ $\Delta$ Institutional births (PLACEBO OUTCOME) & -2.149 & 2.871 \\
\hline\hline
\end{tabular}
\vspace{0.5em}
{\small \textit{Notes:} All specifications include state FE and baseline controls. HC1 robust SE. Row 1 is the real treatment effect. Rows 2--4 are placebos that should show null effects.}
\end{table}



\subsection{Heterogeneity}

Tercile-level heterogeneity in the reduced form reveals a dose-response pattern: high-exposure districts experienced the largest health improvements relative to low-exposure districts within states, consistent with a treatment effect that increases with exposure intensity. However, this pattern is also consistent with correlated development trends, as the most fuel-deprived districts received the most intense concurrent infrastructure investments.

As a complement, Appendix~\ref{app:panel_did} reports a na\"ive two-way fixed effects panel DiD. That specification absorbs district fixed effects but does \textit{not} condition on state-level trends, and the resulting coefficients are positive---indicating that high-exposure districts experienced \textit{worse} health trajectories in the raw data. This sign reversal relative to the within-state reduced form reflects the concentration of high-exposure districts in states with the worst aggregate health trends and underscores the critical importance of within-state identification for this design.

The heterogeneity across health outcomes is itself informative. Stunting and underweight---which reflect chronic nutritional status accumulated over months or years---respond to Ujjwala exposure, while diarrhea and ARI---which are acute conditions measured at a point in time---do not. This pattern aligns with the biological mechanisms: reduced indoor air pollution improves appetite, reduces infection-related nutrient loss, and frees maternal time for childcare, all of which operate through chronic nutritional pathways rather than immediate disease incidence. It also aligns with the finding from randomized cookstove trials that point-in-time respiratory and gastrointestinal outcomes are harder to move than cumulative nutritional status \citep{hanna2016, mortimer2017}.

The geographic heterogeneity is also notable. States with the highest Ujjwala exposure---Bihar, Jharkhand, Chhattisgarh, Madhya Pradesh---are also states with the worst baseline health indicators and the most active government development programs. The leave-one-state-out analysis confirms that no single state drives the results: excluding Bihar (the state with the highest mean exposure at 0.91) shifts the diarrhea coefficient noticeably, while excluding smaller states has minimal impact. The first-stage coefficient is remarkably stable across all state exclusions, ranging from 10.4 to 15.8 pp for clean fuel changes.


\section{Discussion}

\subsection{Interpreting the Results}

The evidence from Ujjwala presents a clear narrative for supply-side success and a more ambiguous picture for demand-side health impacts.

On the supply side, the program unambiguously achieved its primary goal: clean fuel adoption surged by 24 percentage points on average, with the largest gains in the most disadvantaged districts. The first stage is robust across specifications and the convergence pattern---high-exposure districts narrowing the gap with low-exposure districts---is visually striking and statistically overwhelming.

On the demand side, the health story is more nuanced. The IV estimates suggest meaningful effects of clean fuel adoption on child stunting ($-$0.59 pp per percentage point of adoption), which translates to roughly a 3--4 percentage point reduction in stunting for a district that gained 5--7 additional clean fuel percentage points due to Ujjwala. This is consistent with the biological literature on IAP and child nutrition.

However, the robustness analysis raises important caveats. The fact that the electricity gap---which has no direct connection to cooking---predicts stunting changes at similar magnitudes suggests that our estimates capture general development convergence, not fuel-specific effects. The attenuation when controlling for changes in water and sanitation (\Cref{tab:horse_race})---from $-$8.3 to $-$0.9 for stunting, an 89\% reduction---further supports this interpretation.

\subsection{Comparison with Existing Evidence}

Our IV estimate of $-$0.59 percentage points of stunting per percentage point of clean fuel adoption can be benchmarked against existing estimates. \citet{smith2011}, in the RESPIRE trial in Guatemala, found that randomized assignment to improved (chimney) stoves reduced physician-diagnosed pneumonia by 33\%, corresponding to approximately 5 percentage points on a base of 15\%. However, that trial measured acute respiratory outcomes, not nutritional status. The RESPIRE follow-up studies found no significant effects on birth weight, suggesting that the biological pathway from IAP to nutrition may require longer exposure periods than the 18-month trial window.

\citet{hanna2016} found essentially zero long-run effects of improved cookstoves in Orissa, but this reflected abandonment of the technology rather than ineffectiveness of clean cooking \textit{per se}. Their key finding---that adoption declines sharply after the initial distribution---is directly relevant to Ujjwala's refill dropout problem. If 30--40\% of Ujjwala beneficiaries consume fewer than three LPG refills per year, the effective treatment is substantially diluted relative to the connection count.

Our reduced-form estimates are of comparable magnitude to other district-level studies of Indian development programs. \citet{imbert2015} estimated that MGNREGA reduced rural poverty by 3--4 percentage points in early-phase districts. \citet{asher2020} found that PMGSY road construction increased economic activity in connected villages by 10--20\%. The magnitudes are broadly consistent: large-scale infrastructure programs in India produce effects that are economically meaningful but modest relative to baseline deprivation.

\subsection{Why Diarrhea Does Not Respond}

The null result for diarrhea is actually reassuring for our identification strategy. Diarrhea is primarily a waterborne disease, transmitted through contaminated water, food, and fecal-oral contact. The direct pathway from cooking fuel to diarrhea is weak (primarily through contaminated hands during cooking or reduced water boiling). Our finding that water access---not cooking fuel---is the dominant predictor of diarrhea changes (\Cref{tab:horse_race}) aligns with the epidemiological evidence and helps validate the instrument: Ujjwala exposure should not and does not predict diarrhea changes after conditioning on water access.

\subsection{Mechanisms: How Clean Cooking Might Affect Child Nutrition}

The reduced-form finding that Ujjwala exposure predicts improvements in stunting and underweight---but not diarrhea---is consistent with several biological mechanisms linking indoor air pollution to chronic nutritional status.

\textbf{Respiratory morbidity and nutrient absorption.} Children exposed to high levels of PM$_{2.5}$ from biomass cooking experience repeated bouts of acute respiratory infection. Each infection episode reduces appetite, increases metabolic demands, and impairs nutrient absorption in the gut. Over time, this cycle of infection and malnutrition produces chronic stunting. A shift from biomass to LPG reduces PM$_{2.5}$ exposure by 80--90 percent in controlled settings \citep{smith2014}, potentially breaking this cycle. The fact that stunting (a cumulative indicator reflecting months of nutritional deprivation) responds more strongly than diarrhea (a point-in-time acute condition) is consistent with this chronic pathway.

\textbf{Maternal time reallocation.} Collecting firewood and cooking with biomass is time-intensive---rural Indian women spend an estimated 2--4 hours daily on fuel collection and cooking with traditional stoves. LPG cooking reduces this burden substantially: lighting a gas burner takes seconds compared to the 20--30 minutes required to start and tend a biomass fire. The time savings can be reallocated to childcare, breastfeeding, food preparation with more nutritious ingredients, or income-generating activities. \citet{hanna2016} hypothesized this pathway but found no evidence in their trial due to technology abandonment. With Ujjwala achieving higher sustained adoption (due to the LPG subsidy infrastructure), the time-reallocation channel may be more relevant.

\textbf{Appetite and food contamination.} Smoke-filled cooking environments can directly reduce children's appetite and contaminate food during preparation. LPG stoves produce virtually no smoke, creating a cleaner eating environment. While this mechanism is difficult to measure directly in survey data, it provides an additional pathway through which clean cooking could improve nutritional outcomes without operating through the diarrhea channel.

\textbf{Why not diarrhea?} The null result for diarrhea is biologically coherent. Diarrheal disease in children is overwhelmingly waterborne, transmitted through contaminated drinking water, inadequate handwashing, and poor sanitation. The direct causal pathway from cooking fuel to diarrhea is weak. One indirect channel---that boiling water for drinking is more convenient with LPG than with biomass---could theoretically reduce waterborne disease, but water boiling behavior is not measured in the NFHS factsheets, and any such effect would be second-order relative to the direct impact of improved water supply (Jal Jeevan Mission).

\subsection{The Concurrent Intervention Problem}

Perhaps the most important contribution of this paper is documenting the severity of the concurrent intervention problem. India's government launched at least five major infrastructure programs targeting disadvantaged populations between 2014 and 2019: Ujjwala (cooking fuel), Swachh Bharat (sanitation), Jal Jeevan (water), Saubhagya (electricity), and Ayushman Bharat (health insurance). All five programs were explicitly targeted at the poorest districts and households, creating near-perfect collinearity in treatment intensity.

This poses a fundamental challenge for any baseline-gap-times-post identification strategy. The ``fuel gap'' is mechanically correlated with the ``sanitation gap,'' ``water gap,'' and ``electricity gap'' (correlation coefficients range from 0.65 to 0.82 across districts). Controlling for these correlated gaps helps, but multicollinearity limits our ability to cleanly separate effects.

The specification controlling for concurrent infrastructure changes (\Cref{tab:horse_race}) provides suggestive evidence, but should be interpreted cautiously rather than causally---the $\Delta$ sanitation and $\Delta$ water controls are themselves endogenous. The key finding is that once contemporaneous infrastructure improvements are absorbed, the Ujjwala coefficient becomes statistically and economically zero across all outcomes. This pattern is consistent with the public health literature ranking water quality as the single most important environmental determinant of child survival in low-income settings \citep{kremer2011}.

\subsection{Limitations}

Several limitations should be acknowledged. First, the two-period structure of NFHS prevents formal pre-trend testing. We cannot verify that high- and low-exposure districts were on parallel trajectories before Ujjwala. The covariate balance analysis reveals large differences across exposure terciles in all observable characteristics, and while state fixed effects and baseline controls address level differences, differential trends remain a concern.

Second, NFHS-4 fieldwork partially overlaps with Ujjwala's launch. Although we argue this biases estimates toward zero, the extent of contamination is unknown without individual-level interview dates.

Third, our treatment measure---baseline solid fuel dependence---is a proxy for Ujjwala intensity, not a direct measure of LPG connections distributed. Districts may have gained clean fuel access through urbanization, income growth, or private market expansion rather than Ujjwala.

Fourth, we measure cooking fuel adoption, not sustained usage. The ``refill dropout'' problem means that many Ujjwala beneficiaries received connections but did not use LPG as their primary cooking fuel. Our estimates are intention-to-treat effects that reflect this incomplete compliance.

Fifth, the district-level analysis necessarily aggregates across heterogeneous households. Within any district, the households that switched to LPG may differ systematically from those that did not---they may be wealthier, more educated, or more health-conscious within the BPL population. The district-level average treatment effect conflates the experiences of compliers (who switched) and non-compliers (who received connections but continued using biomass). Household-level analysis using NFHS unit records would enable sharper estimation of the complier average causal effect, but such microdata was not available for this study.

\subsection{External Validity}

The generalizability of these findings depends on the extent to which India's experience is representative of other developing countries pursuing clean cooking transitions. Several features of the Indian context are distinctive. First, the scale of Ujjwala---100 million connections in five years---is unmatched globally. Indonesia's ``Three Million Connections'' program (2007--2012) and Brazil's \textit{Aux\'{i}lio G\'{a}s} subsidy are the closest comparators, but neither approached India's scale relative to population. Scale matters because it determines the density of the LPG distribution network (refill stations, delivery infrastructure), which in turn affects sustained usage.

Second, India's concurrent rollout of multiple infrastructure programs creates a uniquely challenging identification environment. Countries pursuing clean cooking interventions in isolation---for instance, through World Bank-financed LPG subsidy programs in Sub-Saharan Africa---may face cleaner identification conditions but also different contextual factors (lower population density, weaker supply chains, different cooking practices).

Third, the ``fuel stacking'' phenomenon documented in India---where households use LPG for some meals but continue burning biomass for others---may be more or less prevalent in other contexts depending on cultural cooking practices, the relative cost of LPG refills, and the availability of free biomass. In settings where firewood is purchased rather than freely collected (as in many urban African contexts), the economic incentive to switch fully to LPG is stronger, potentially yielding larger health effects than observed here.

\section{Conclusion}

India's Ujjwala Yojana achieved a remarkable supply-side success: 100 million LPG connections distributed in five years, with clean fuel adoption surging by 24 percentage points across 708 districts. The program's targeting---prioritizing the poorest, most biomass-dependent districts---produced dramatic convergence in clean fuel access within states.

The health dividends are more elusive to pin down. While the point estimates suggest meaningful reductions in child stunting and underweight, these effects are confounded by contemporaneous improvements in water, sanitation, and electricity that targeted the same populations. The honest conclusion is that clean cooking fuel likely contributes to improved child nutrition, but we cannot credibly separate its effect from the broader development transformation India experienced during 2015--2021.

This finding has direct policy implications. First, the strong first stage confirms that supply-side interventions (free connections) effectively increase clean fuel adoption, though refill dropout remains a challenge for sustained use. The contrast between connection distribution (which was rapid) and sustained LPG usage (which remains low in many areas) suggests that supply-side subsidies are necessary but not sufficient---complementary demand-side interventions like refill subsidies, behavioral nudges, or community-level cooking demonstrations may be needed to translate connections into health benefits.

Second, the difficulty of isolating health effects suggests that India's strategy of bundling multiple infrastructure programs may create more health impact than any single program, but also makes evaluation of individual programs nearly impossible in non-experimental settings. This creates a tension between policy design (bundling is efficient) and policy evaluation (bundling defeats identification). Countries designing similar multi-program development strategies should consider staggering rollout dates or introducing randomized variation to enable credible evaluation.

Third, the finding that water access dominates cooking fuel in predicting diarrhea reductions, while both matter for nutritional outcomes, suggests a hierarchy of infrastructure investments for child health. If governments face budget constraints, the evidence here tentatively supports prioritizing clean water over clean cooking fuel for immediate child survival, while clean cooking fuel may have larger effects on chronic nutritional status.

For future research, household-level analysis using NFHS microdata with precise interview dates would enable sharper identification through regression discontinuity designs around survey date cutoffs. District-level analysis of this type would also benefit from earlier pre-treatment data (DLHS-4 or AHS) to test parallel trends. Finally, the horse-race approach---simultaneous evaluation of multiple infrastructure investments---deserves further methodological development, as countries increasingly pursue bundled development strategies.

The growing availability of administrative data on program beneficiaries---including Ujjwala connection records, SBM toilet construction dates, and JJM water connection timelines---could enable more granular matching of treatment timing to health outcomes. Such analysis would require negotiating data access with multiple government ministries, but would represent a significant advance over the survey-based approach pursued here.

The question ``Can clean cooking save lives?'' has a likely affirmative answer on biological grounds. Whether we can measure the answer precisely in the presence of India's development revolution remains an open challenge.


\section*{Acknowledgements}

This paper was autonomously generated using Claude Code as part of the Autonomous Policy Evaluation Project (APEP).

\noindent\textbf{Project Repository:} \url{https://github.com/SocialCatalystLab/ape-papers}

\noindent\textbf{Contributors:} @olafdrw

\noindent\textbf{First Contributor:} \url{https://github.com/olafdrw}

\label{apep_main_text_end}
\newpage
\bibliography{references}

\newpage
\appendix

\section{Data Appendix}

\subsection{NFHS District Factsheet Data}

The NFHS district factsheet data was obtained from a publicly available GitHub repository (\url{https://github.com/SaiSiddhardhaKalla/NFHS}), which compiles district-level indicators parsed from the official NFHS reports published by the International Institute for Population Sciences (IIPS) and hosted on \url{rchiips.org}.

The raw dataset contains 73,632 rows covering approximately 708 districts (some districts have missing values for specific indicators). Each row represents one indicator for one district, with columns for NFHS-4 and NFHS-5 values.

\textbf{Key indicators used:}
\begin{itemize}
\item \textbf{Clean fuel:} ``Households using clean fuel for cooking (\%)'' -- includes LPG, biogas, electricity, and natural gas
\item \textbf{Diarrhea:} ``Prevalence of diarrhoea in the 2 weeks preceding the survey (\%)'' -- children under 5
\item \textbf{Stunting:} ``Children under 5 years who are stunted (height for age) (\%)'' -- below $-$2 SD of WHO reference
\item \textbf{Underweight:} ``Children under 5 years who are underweight (weight for age) (\%)'' -- below $-$2 SD
\item \textbf{ARI:} ``Prevalence of symptoms of acute respiratory infection (ARI) in the 2 weeks preceding the survey (\%)''
\item \textbf{Anemia:} ``All women age 15-49 years who are anaemic (\%)''
\end{itemize}

\subsection{Variable Construction}

The treatment intensity variable is constructed as:
\[
\text{Ujjwala Exposure}_d = \frac{100 - \text{NFHS-4 Clean Fuel}_d}{100}
\]

This ranges from 0.03 (97\% clean fuel at baseline) to 1.0 (0\% clean fuel at baseline). Exposure terciles are defined by the 33rd and 67th percentiles of this distribution.

First-difference outcomes are computed as NFHS-5 value minus NFHS-4 value for each indicator. Baseline controls are NFHS-4 values divided by 100 (rescaled to 0--1 range).

\subsection{Sample and Missing Data}

The final analysis sample includes 708 districts with non-missing clean fuel data in both NFHS rounds. For specific health outcomes, the sample varies slightly due to missing values (e.g., some districts lack ARI or anemia data). The panel dataset contains 19,824 observations (708 districts $\times$ 2 periods $\times$ 14 outcome variables, with district$\times$period combinations dropped for missing outcomes).


\section{Identification Appendix}

\subsection{Covariate Balance}

\Cref{tab:balance_app} shows baseline characteristics by Ujjwala exposure tercile. High-exposure districts (lowest baseline clean fuel) differ dramatically from low-exposure districts: they have lower electricity (33\% vs. 97\%), lower sanitation (12\% vs. 66\%), lower water access (38\% vs. 95\%), and lower institutional births (28\% vs. 90\%). All differences are statistically significant ($p < 0.001$). This imbalance motivates the use of state fixed effects and baseline controls, but residual confounding from differential trends remains a concern.

\begin{table}[H]
\centering
\caption{Covariate Balance by Ujjwala Exposure Tercile}
\label{tab:balance_app}
\begin{threeparttable}
\begin{tabular}{lccc}
\toprule
& Low Exposure & Medium Exposure & High Exposure \\
& ($N = 236$) & ($N = 236$) & ($N = 236$) \\
\midrule
Electricity (\%) & 97.4 & 84.4 & 33.3 \\
Improved sanitation (\%) & 65.5 & 40.4 & 12.3 \\
Improved water (\%) & 95.3 & 89.3 & 37.7 \\
Institutional births (\%) & 89.5 & 75.4 & 28.0 \\
Full vaccination (\%) & 67.4 & 58.8 & 25.7 \\
\bottomrule
\end{tabular}
\begin{tablenotes}[flushleft]
\small
\item \textit{Notes:} NFHS-4 baseline values. All differences across terciles are significant at $p < 0.001$ (ANOVA F-test).
\end{tablenotes}
\end{threeparttable}
\end{table}

\subsection{Binned Scatter Plots}

\Cref{fig:binned} presents binned scatter plots for the reduced-form relationship between Ujjwala exposure and changes in diarrhea prevalence (Panel A) and clean fuel adoption (Panel B). Each bin represents the mean of approximately 35 districts. Panel B confirms the strong first stage across the full distribution of exposure. Panel A shows a weakly negative relationship for diarrhea, with considerable noise.

\begin{figure}[H]
\centering
\includegraphics[width=\textwidth]{figures/fig5_binned_scatter.pdf}
\caption{Binned Scatter: Ujjwala Exposure vs. Outcome Changes}
\label{fig:binned}
\end{figure}


\section{Panel DiD Estimates}
\label{app:panel_did}

\Cref{tab:panel_did} presents the na\"ive two-way fixed effects estimates with district and period fixed effects but \textit{without} state$\times$period interactions. The Exposure $\times$ Post coefficients are positive for health outcomes (diarrhea $+$8.9 pp, stunting $+$30.1 pp). These positive signs---indicating \textit{worse} health trajectories in high-exposure districts---reflect the omission of state-level trend controls. High-exposure districts are concentrated in states (Bihar, Jharkhand, Chhattisgarh) with the worst aggregate health trends. Without conditioning on state-level trajectories, the interaction term captures these differential state trends rather than the Ujjwala treatment effect. This diagnostic motivates the within-state identification strategy used in the main analysis (\Cref{eq:reduced_form}), where state fixed effects absorb these confounding state-level trends.


\begin{table}[htbp]
   \caption{\label{tab:panel_did} Na\"ive Panel DiD (Without State$\times$Period FE)}
   \centering
   \begin{tabular}{lccc}
      \tabularnewline \midrule \midrule
      Dependent Variable: & \multicolumn{3}{c}{outcome}\\
                              & Clean Fuel    & Diarrhea      & Stunting \\   
      Model:                  & (1)           & (2)           & (3)\\  
      \midrule
      \emph{Variables}\\
      Exposure $\times$ Post  & 44.14$^{***}$ & 8.892$^{***}$ & 30.12$^{***}$\\   
                              & (13.61)       & (2.025)       & (7.432)\\   
      \midrule
      \emph{Fixed-effects}\\
      district\_id            & Yes           & Yes           & Yes\\  
      period                  & Yes           & Yes           & Yes\\  
      \midrule
      \emph{Fit statistics}\\
      Observations            & 1,416         & 1,416         & 1,416\\  
      R$^2$                   & 0.86347       & 0.57270       & 0.70554\\  
      Within R$^2$            & 0.22647       & 0.10336       & 0.20673\\  
      \midrule \midrule
      \multicolumn{4}{l}{\emph{Clustered (state\_code) standard-errors in parentheses}}\\
      \multicolumn{4}{l}{\emph{Signif. Codes: ***: 0.01, **: 0.05, *: 0.1}}\\
   \end{tabular}
   
   \par \raggedright 
   District and period FE included but not state$\times$period interactions. Positive health coefficients reflect concentration of high-exposure districts in states with worst aggregate trends. See Section 5 for discussion.
\end{table}





\section{Robustness Appendix}

\subsection{Non-Linear Exposure Effects}

Quartile analysis of the diarrhea reduced form (with state FE and controls) shows:
\begin{itemize}
\item Q2 vs. Q1: $-$0.5 pp ($SE = 0.7$)
\item Q3 vs. Q1: $-$1.3 pp ($SE = 1.1$)
\item Q4 vs. Q1: $-$1.7 pp ($SE = 1.8$)
\end{itemize}
The monotonic pattern is consistent with a dose-response relationship, but none of the individual quartile coefficients are statistically significant.

For the first stage (clean fuel change), quartile effects are:
\begin{itemize}
\item Q2 vs. Q1: $+$4.8 pp ($SE = 1.2$, $p < 0.01$)
\item Q3 vs. Q1: $+$3.7 pp ($SE = 1.5$, $p < 0.05$)
\item Q4 vs. Q1: $+$2.2 pp ($SE = 2.3$, $p = 0.34$)
\end{itemize}
The first stage is strongest for the second quartile and attenuates at the highest exposure levels, consistent with diminishing marginal returns or refill dropout among the poorest households.


\section{Baseline Fuel Distribution}

\begin{figure}[H]
\centering
\includegraphics[width=0.85\textwidth]{figures/fig1_baseline_fuel_distribution.pdf}
\caption{Distribution of Baseline Clean Fuel Usage Across Districts}
\label{fig:distribution}
\end{figure}

\Cref{fig:distribution} shows the distribution of baseline clean fuel usage (NFHS-4) across 708 districts, colored by Ujjwala exposure tercile. The distribution is right-skewed, with a large mass of districts below 20\% clean fuel usage. The median is approximately 22\%, indicating that the typical Indian district was heavily dependent on solid fuels at baseline.


\end{document}
