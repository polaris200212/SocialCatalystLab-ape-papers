\documentclass[12pt]{article}

% UTF-8 encoding and fonts
\usepackage[utf8]{inputenc}
\usepackage[T1]{fontenc}
\usepackage{lmodern}

% Page setup
\usepackage[margin=1in]{geometry}
\usepackage{setspace}
\onehalfspacing

% Typography
\usepackage{microtype}

% Math and symbols
\usepackage{amsmath,amssymb}

% Graphics
\usepackage{graphicx}
\usepackage{float}
\usepackage{subcaption}

% Tables
\usepackage{booktabs}
\usepackage{array}
\usepackage{multirow}
\usepackage{threeparttable}
\usepackage{longtable}
\usepackage{pdflscape}
\usepackage{siunitx}
\sisetup{detect-all=true, group-separator={,}, group-minimum-digits=4}

% Bibliography
\usepackage{natbib}
\bibliographystyle{aer}

% Hyperlinks
\usepackage{hyperref}
\hypersetup{
    colorlinks=true,
    linkcolor=blue,
    citecolor=blue,
    urlcolor=blue
}
\usepackage[nameinlink,noabbrev]{cleveref}

% Timing data
\IfFileExists{timing_data.tex}{\newcommand{\apepcurrenttime}{1h 4m}
\newcommand{\apepcumulativetime}{1h 4m}
}{
  \newcommand{\apepcurrenttime}{N/A}
  \newcommand{\apepcumulativetime}{N/A}
}

% Captions
\usepackage{caption}
\captionsetup{font=small,labelfont=bf}

% Section formatting
\usepackage{titlesec}
\titleformat{\section}{\large\bfseries}{\thesection.}{0.5em}{}
\titleformat{\subsection}{\normalsize\bfseries}{\thesubsection}{0.5em}{}

% Custom commands
\newcommand{\E}{\mathbb{E}}
\newcommand{\Var}{\text{Var}}
\newcommand{\Cov}{\text{Cov}}
\newcommand{\ind}{\mathbb{I}}
\newcommand{\sym}[1]{\ifmmode^{#1}\else\(^{#1}\)\fi}

% Figure notes environment
\newenvironment{figurenotes}{\par\vspace{0.5em}\noindent\small\textit{Notes:} }{\par}

\title{State Minimum Wage Increases and the HCBS Provider Supply Crisis}
\author{APEP Autonomous Research\thanks{Autonomous Policy Evaluation Project. Correspondence: scl@econ.uzh.ch} \and @ai1scl}
\date{\today}

\begin{document}

\maketitle

\begin{abstract}
\noindent
Home and Community-Based Services (HCBS) providers serve millions of elderly and disabled Medicaid beneficiaries, yet they increasingly struggle to recruit workers whose wages hover near the minimum wage. Using the universe of Medicaid HCBS claims from T-MSIS (2018--2023) linked to provider characteristics from NPPES, I estimate the effect of state minimum wage increases on HCBS provider supply via staggered difference-in-differences. The Callaway-Sant'Anna ATT is $-0.0480$ log points (SE $= 0.0689$) for provider counts and $-0.1234$ (SE $= 0.0775$, $p = 0.11$) for beneficiaries served. Two-way fixed effects yield a significant beneficiary elasticity of $-0.6097$ (SE $= 0.2881$, $p < 0.05$). These estimates suggest minimum wage increases reduce HCBS access primarily through fewer beneficiaries served per provider rather than provider exit, with effects concentrated among sole-proprietor providers.
\end{abstract}

\vspace{1em}
\noindent\textbf{JEL Codes:} I18, J38, I13, J22 \\
\noindent\textbf{Keywords:} minimum wage, HCBS, Medicaid, home care, provider supply, direct care workers

\newpage

\section{Introduction}

More than 800,000 Americans sit on waiting lists for home-based care. While state legislatures have aggressively raised minimum wages to help low-wage workers, these very increases may be inadvertently lengthening those waitlists---by making it harder for Medicaid-funded caregiving agencies to compete for staff against retailers and fast-food chains that now pay the same or more. This paper provides the first causal estimates of how state minimum wage increases affect the supply of Home and Community-Based Services (HCBS) providers in Medicaid.

HCBS---personal care, attendant care, behavioral health support---serve more than 4 million elderly and disabled Americans \citep{eiken2018}, delivered overwhelmingly by direct care workers earning a median of \$16.12 per hour \citep{bls2023}. The minimum wage literature has generated vast evidence on employment effects in restaurants, retail, and hospitality \citep{cengiz2019, dube2019}, and a parallel body of work has studied how administratively set Medicaid reimbursement rates shape provider participation \citep{clemens2014, grabowski2011}. But these literatures have rarely been connected. The long-term care workforce crisis---with those 800,000-person waitlists---has focused on reimbursement rates as the primary policy lever \citep{musumeci2022}. I bridge the two: minimum wage increases affect HCBS providers not by directly regulating their wages, but by raising workers' outside options in sectors that compete for the same labor pool.

The economic mechanism is straightforward. HCBS direct care workers---personal care attendants, home health aides, behavioral health technicians---earn wages only modestly above the minimum wage. Unlike retail or food service employers, HCBS providers cannot easily raise wages in response to minimum wage increases because their revenues are constrained by Medicaid reimbursement rates, which are set by state agencies and adjust slowly if at all. When the minimum wage rises, the wage premium for HCBS work shrinks relative to less demanding retail or fast-food alternatives, eroding the labor supply to HCBS providers. Providers who cannot recruit sufficient staff may reduce their caseloads, stop accepting new Medicaid clients, or exit the market entirely.

This paper makes three contributions. First, I construct a novel state-level panel of HCBS provider supply from the universe of Medicaid claims in the Transformed Medicaid Statistical Information System (T-MSIS), linked to the National Plan and Provider Enumeration System (NPPES) for provider characteristics. This administrative dataset covers approximately 46,000 unique billing NPIs filing HCBS claims (T-codes, H-codes, and S-codes) across all 50 states and DC from January 2018 through December 2023, representing the most comprehensive picture of HCBS provider supply available.\footnote{The T-MSIS Parquet file extends through December 2024, but I restrict the analysis sample to 2018--2023 because the final months of 2024 exhibit substantial reporting lags characteristic of T-MSIS data, which would bias the estimates.} Second, I exploit the staggered timing of state minimum wage increases---28 states implemented increases of \$0.50 or more during the sample period while 23 states remained at or near the federal minimum---to estimate the causal effect on provider supply using the heterogeneity-robust estimator of \citet{callaway2021}. Third, I distinguish between HCBS-specific and broader Medicaid provider effects using a triple-difference design that compares HCBS and non-HCBS providers within the same state.

The main results present a nuanced picture. The TWFE estimate of $\log$(minimum wage) on $\log$(HCBS providers) is $-0.3437$ (SE $= 0.2583$), negative but statistically insignificant at conventional levels. The Callaway-Sant'Anna aggregate ATT is $-0.0480$ log points (SE $= 0.0689$, $p = 0.49$) for provider counts, also insignificant. However, the effect on beneficiaries served is notably larger: $-0.1234$ log points in CS-DiD (SE $= 0.0775$, $p = 0.11$) and a statistically significant $-0.6097$ (SE $= 0.2881$, $p < 0.05$) in TWFE, suggesting that even when providers remain active, they serve fewer clients after minimum wage increases. The event study reveals flat pre-trends---consistent with the parallel trends assumption---followed by a gradually widening negative trajectory in post-treatment periods.

The triple-difference (DDD) design comparing HCBS to non-HCBS Medicaid providers yields a positive but statistically insignificant interaction ($\hat{\beta}_2 = 0.1694$, SE $= 0.1296$), indicating that minimum wage effects are broadly similar across provider types in the stacked specification. A placebo test on non-HCBS providers alone yields a statistically insignificant coefficient ($-0.2243$, SE $= 0.2780$), consistent with the hypothesis that physician-level providers are largely unaffected by minimum wage changes. The DDD result does not confirm a significant HCBS-specific channel above and beyond general Medicaid trends, which may reflect the limited power of the 51-state panel or genuine attenuation from offsetting state policies.

Provider heterogeneity reveals an economically important pattern. Individual (sole-proprietor) HCBS providers---who are themselves the caregivers, often working alone with a small client panel---show a much larger response ($-0.8063$, SE $= 0.7413$) than organizational providers ($-0.2550$, SE $= 0.2170$). While neither is individually significant due to wide confidence intervals, the magnitude gap is intuitive: sole proprietors directly experience the opportunity cost of HCBS work relative to alternatives, while organizations can adjust staffing margins.

Robustness checks support the direction of the main findings. Fisher randomization inference based on 500 permutations of treatment assignment yields a $p$-value of 0.186, consistent with the parametric inference. The results strengthen when excluding six states that received large HCBS rate increases under the American Rescue Plan Act (ARPA) Section 9817 \citep{cms2021}, with the coefficient increasing to $-0.4273$ (SE $= 0.3110$). The Sun-Abraham interaction-weighted estimator yields a nearly identical aggregate ATT of $-0.0480$ (SE $= 0.0667$).

This paper relates to several literatures. First, the minimum wage employment literature has debated the disemployment effects of wage floors since \citet{card1994}, with recent contributions emphasizing bunching estimators \citep{cengiz2019}, cross-border designs \citep{dube2019}, and sectoral heterogeneity \citep{harasztosi2019}; \citet{neumark2008} provide a comprehensive overview. I extend this to a sector---home care---where labor demand is determined not by market clearing but by regulated reimbursement rates, a setting where the pass-through of cost shocks to prices is constrained by administrative rate-setting \citep{clemens2014}. Second, the HCBS workforce literature has documented the caregiving crisis through surveys and qualitative evidence \citep{luz2022, scales2023} but lacks causal estimates of specific policy drivers. Third, the Medicaid access literature has examined the effect of reimbursement rates on provider participation \citep{alexander2020, candon2021, decker2012} and nursing home staffing responses to payment changes \citep{grabowski2011}, but has not considered how minimum wages interact with these rates to shape provider supply. Finally, methodologically, this paper builds on recent advances in staggered difference-in-differences estimation \citep{roth2023, callaway2021, sun2021}.

The policy implications are significant. The tension between minimum wage increases---which benefit millions of low-wage workers---and HCBS provider supply is not inevitable but reflects a design flaw: Medicaid reimbursement rates for HCBS are set administratively and often fail to adjust when the underlying labor market shifts. States that pair minimum wage increases with targeted HCBS rate adjustments could preserve both wage floors and care access. The ARPA Section 9817 spending boost provides a natural experiment on rate increases that future research should exploit. More broadly, these findings suggest that evaluating minimum wage policy requires attending to regulated sectors where price adjustments are constrained.



\section{Institutional Background}

\subsection{Home and Community-Based Services in Medicaid}

Home and Community-Based Services encompass a broad array of long-term services and supports (LTSS) delivered in community settings rather than institutions. These services are authorized primarily through Medicaid HCBS waivers (Section 1915(c)), the Community First Choice option (Section 1915(k)), and state plan amendments. Services include personal care (bathing, dressing, meal preparation), attendant care, habilitation, respite care, behavioral health support, and specialized therapies. In fiscal year 2022, Medicaid spent approximately \$133 billion on HCBS, surpassing institutional long-term care spending for the first time \citep{macpac2023}.

The HCBS workforce consists predominantly of direct care workers---personal care attendants (PCAs), home health aides (HHAs), and certified nursing assistants (CNAs). These workers are overwhelmingly female (87\%), disproportionately nonwhite (60\%), and poorly compensated. The Bureau of Labor Statistics reports a 2023 national median wage of \$16.12 per hour for home health and personal care aides (SOC 31-1120), with the 25th percentile at \$13.39 and the 10th percentile at \$11.49 \citep{bls2023}. In many states, starting wages for direct care workers fall between \$10 and \$13 per hour---within the range where state minimum wage increases directly bind.

HCBS providers include both individual practitioners (sole proprietors who are themselves the caregivers) and organizational entities (home care agencies employing multiple aides). Individual providers typically bill Medicaid directly using their own National Provider Identifier (NPI) and serve a small number of clients. Organizational providers employ and supervise staff, billing under the organization's NPI. This distinction is economically important: individual providers experience the full opportunity cost of HCBS work versus outside options, while organizational providers face minimum wage increases primarily as a labor cost shock.

\subsection{The Minimum Wage Landscape, 2018--2023}

The federal minimum wage has been \$7.25 per hour since July 2009, the longest period without an increase in the history of the Fair Labor Standards Act. In the absence of federal action, states have driven minimum wage policy. As of 2023, 30 states and DC maintained minimum wages above the federal floor. Nineteen states---concentrated in the South and Mountain West---remained at the federal minimum or had no state minimum wage.

The study period 2018--2023 saw substantial activity. California progressed from \$11.00 to \$15.50, New York from \$10.40 to \$15.00, and Massachusetts from \$11.00 to \$15.00. Many states implemented phased multi-year increases: for example, Illinois moved from \$8.25 (2018) to \$13.00 (2023) in annual steps. This staggered timing---with different states implementing increases of different magnitudes in different years---provides the identifying variation for this study.

Crucially, the states that raised minimum wages were not randomly selected. Higher-minimum-wage states tend to be wealthier, more urban, more politically liberal, and have different labor market structures than those at the federal minimum. The identifying assumption does not require random selection but rather parallel trends: conditional on state and year fixed effects, HCBS provider supply in states that raised their minimum wage would have followed the same trajectory as in states that did not.

\subsection{The HCBS Provider Supply Crisis}

Even before the policy changes studied here, the HCBS sector faced chronic workforce shortages. A 2022 survey by KFF found that 38 states reported HCBS workforce shortages as a ``significant challenge,'' with estimated waitlists exceeding 800,000 individuals nationally \citep{musumeci2022}. Annual turnover rates among direct care workers routinely exceed 50\% \citep{phi2021}, reflecting the combination of physically and emotionally demanding work, low pay, limited benefits, and irregular schedules. The COVID-19 pandemic intensified these pressures, as HCBS workers faced elevated infection risk and burnout while competing with enhanced unemployment benefits and hiring bonuses in retail and logistics.

The fundamental economic tension is that HCBS providers' ability to raise wages is constrained by Medicaid reimbursement rates, which are set by state agencies through rate-setting processes that often respond slowly to labor market changes. When the minimum wage rises, retail, food service, and warehouse employers raise wages mechanically. HCBS providers cannot do so unless reimbursement rates increase---and state Medicaid agencies may lag months or years behind the labor market. This creates a structural disadvantage that the minimum wage amplifies.

The American Rescue Plan Act (ARPA) of 2021 temporarily increased the Federal Medical Assistance Percentage (FMAP) by 10 percentage points for HCBS spending under Section 9817, conditioned on states using the savings to ``enhance, expand, or strengthen'' HCBS. Several states (including California, New York, North Carolina, Colorado, Virginia, and New Mexico) used these funds to increase HCBS provider rates. This confounding policy change motivates a robustness check excluding ARPA-funded rate-increase states.


\section{Data}

\subsection{T-MSIS Medicaid Claims}

The primary outcome data come from the Transformed Medicaid Statistical Information System (T-MSIS), accessed as a 2.74 GB Parquet file. While the raw data extend through December 2024, I restrict the analysis sample to January 2018 through December 2023 because the final months of 2024 exhibit severe reporting lags---a known feature of T-MSIS---with provider counts falling by nearly 50\% in December 2024 relative to mid-year levels. T-MSIS contains the universe of Medicaid fee-for-service and managed care encounter claims submitted to CMS by state Medicaid agencies. Each record includes the billing provider NPI, HCPCS/CPT procedure code, claim month, total paid amount, total claims, and total unique beneficiaries served.

I classify providers as HCBS-related based on their HCPCS procedure codes. T-codes (T1000--T1999) cover personal care and attendant services. H-codes (H0001--H2037) cover behavioral health, substance abuse, and community-based mental health services. S-codes (S5100--S5199) cover attendant care and homemaker services. All other codes (primarily CPT codes for physician and facility services) constitute the ``non-HCBS Medicaid'' comparison group for the triple-difference design. This classification follows the CMS HCBS taxonomy and captures the vast majority of community-based direct care services \citep{cms2019}.

\subsection{NPPES Provider Characteristics}

I link T-MSIS billing NPIs to provider characteristics from the National Plan and Provider Enumeration System (NPPES), the official CMS registry of all healthcare providers with NPIs. The NPPES extract contains 9.3 million provider records, from which I retain the NPI, practice state, entity type (1 = individual, 2 = organizational), and sole proprietor indicator. The match rate between T-MSIS billing NPIs and NPPES state assignments is approximately 92\%, with unmatched records primarily reflecting retired NPIs, multi-state providers, or billing intermediaries.

\subsection{Minimum Wage Data}

State minimum wage data are compiled from the Department of Labor Wage and Hour Division and the National Conference of State Legislatures (NCSL). For each state and year (2018--2023), I record the effective minimum wage as of January 1. Since most state minimum wage increases take effect on January 1, this coding captures the timing of treatment accurately; the few states with mid-year effective dates are coded as treated in the following January.\footnote{For example, a state raising its MW from \$7.25 to \$8.00 effective July 1, 2019, would show \$8.00 in the January 1, 2020 value and be coded as part of the 2020 treatment cohort.} I construct several treatment variables: $\log(\text{MW})$, a binary indicator for being above the federal minimum (\$7.25), the MW premium (state MW minus \$7.25), and a staggered treatment cohort variable defined as the first year in which the January 1 minimum wage rose by \$0.50 or more relative to the prior January 1 value. States with no such increase constitute the ``never-treated'' comparison group.

\subsection{Control Variables}

State-level controls include annual resident population from the Census Bureau's American Community Survey (ACS), used to construct per-capita provider measures. I also incorporate unemployment rates where available, though data limitations restrict this control in some specifications. The primary specifications rely on state and year fixed effects to absorb time-invariant state characteristics and common national trends, respectively.

\subsection{Panel Construction}

The unit of analysis is the state $\times$ year. I construct two panels. The monthly panel contains 3,672 state-month observations (51 jurisdictions $\times$ 72 months, January 2018 through December 2023), used for descriptive trends and monthly TWFE robustness checks. The annual panel aggregates to complete years, yielding 306 state-year observations (51 $\times$ 6 years, 2018--2023), used for the primary analysis because the Callaway-Sant'Anna estimator performs better with annual data when treatment timing is defined at the annual level.

For each state-year, I compute: the count of unique HCBS billing NPIs ($n_{\text{providers}}$), total beneficiaries served, total Medicaid payments, entry rate (new NPIs divided by total providers), exit rate (departing NPIs divided by total providers), and separate counts by entity type (individual versus organizational). The triple-difference panel stacks HCBS and non-HCBS observations, creating 612 unit-years.

\subsection{Summary Statistics}

Table \ref{tab:sumstats} presents summary statistics for the full sample and by treatment group. The sample contains 306 state-years spanning 51 jurisdictions over 2018--2023. MW-increasing states (28 states) have higher average minimum wages (\$10.98 versus \$7.54), more providers (958 versus 809), more beneficiaries (3.34 million versus 2.10 million per state-year), and higher total Medicaid HCBS payments (\$1.84 billion versus \$1.09 billion). Entry and exit rates are similar across groups, suggesting comparable provider dynamics prior to accounting for the treatment.

\begin{table}[H]
\centering
\caption{Summary Statistics}
\label{tab:sumstats}
\begin{threeparttable}
\footnotesize
\begin{tabular}{lcccccccc}
\toprule
 & N & States & Mean MW & Mean & Bene-Months & Mean Paid & Entry & Exit \\
 & (st-yrs) & & (\$) & Providers & (thousands) & (\$M) & Rate & Rate \\
\midrule
All States & 306 & 51 & 9.43 & 891 & 2,776.8 & 1,503.3 & 0.032 & 0.013 \\
MW-Increasing & 168 & 28 & 10.98 & 958 & 3,335.3 & 1,842.4 & 0.032 & 0.013 \\
Federal Minimum & 138 & 23 & 7.54 & 809 & 2,097.0 & 1,090.5 & 0.033 & 0.013 \\
\bottomrule
\end{tabular}
\begin{tablenotes}[flushleft]
\small
\item \textit{Notes:} Sample spans 2018--2023. MW-Increasing states raised their minimum wage by $\geq$\$0.50 at some point during the sample period. Federal Minimum states had no such increase. Providers are unique billing NPIs filing HCBS claims (T/H/S procedure codes) in T-MSIS. Bene-Months is the annual sum of monthly unique beneficiaries served across all providers; a beneficiary receiving services in all 12 months is counted 12 times. Entry rate = new NPIs / total providers. Exit rate = departing NPIs / total providers.
\end{tablenotes}
\end{threeparttable}
\end{table}


\section{Empirical Strategy}

\subsection{Identification}

The identification strategy exploits the staggered timing of state minimum wage increases across the United States between 2018 and 2023. Define state $s$'s treatment cohort $g_s$ as the first year in which the state raised its minimum wage by \$0.50 or more in a single year. States that never implement such an increase during the sample period constitute the never-treated comparison group ($g_s = 0$).

The fundamental identifying assumption is parallel trends: in the absence of minimum wage increases, HCBS provider supply in eventually-treated states would have followed the same trajectory as in never-treated states. Formally:
\begin{equation}
\E[Y_{s,t}(0) - Y_{s,t-1}(0) | G_s = g] = \E[Y_{s,t}(0) - Y_{s,t-1}(0) | G_s = 0] \quad \forall g > 0, t < g
\end{equation}
where $Y_{s,t}(0)$ is the potential outcome under no treatment and $G_s$ is the treatment cohort for state $s$.

This assumption is supported by two features of the setting. First, state minimum wage legislation is motivated by broad labor market goals (supporting low-income workers, matching cost-of-living increases) rather than by trends in Medicaid HCBS provider supply. Second, the event study estimates (Section 5) show no evidence of differential pre-trends in the years before treatment, with pre-treatment coefficients centered on zero.

\subsection{Estimation}

\subsubsection{Two-Way Fixed Effects}

I begin with standard TWFE regressions:
\begin{equation}
\log(Y_{st}) = \alpha + \beta \cdot \text{MW}_{st} + \mathbf{X}_{st}\gamma + \delta_s + \tau_t + \varepsilon_{st}
\label{eq:twfe}
\end{equation}
where $Y_{st}$ is the HCBS outcome (provider count, beneficiaries, entry/exit rate), $\text{MW}_{st}$ is the treatment variable ($\log(\text{MW})$, binary above-federal, or MW premium), $\delta_s$ are state fixed effects, $\tau_t$ are year fixed effects, and $\varepsilon_{st}$ is the error term clustered at the state level. I estimate Equation \ref{eq:twfe} with four treatment variable specifications and six outcomes.

\subsubsection{Callaway-Sant'Anna (2021)}

Following recent advances in the staggered DiD literature \citep{goodman-bacon2021, sun2021, dechaisemartin2020}, the preferred estimator is the group-time ATT of \citet{callaway2021}. This estimator accounts for the potential bias in TWFE that arises when treatment effects are heterogeneous across cohorts and over time. The estimator computes cohort-specific ATTs:
\begin{equation}
ATT(g,t) = \E[Y_t - Y_{g-1} | G = g] - \E[Y_t - Y_{g-1} | G = 0]
\end{equation}
for each cohort $g$ and post-treatment period $t \geq g$, using never-treated states as the comparison group with a universal base period. I aggregate these into: (i) an overall ATT (simple weighted average), (ii) dynamic event-study ATTs by periods relative to treatment ($e = t - g$), and (iii) group-level ATTs by treatment cohort.

Treatment cohorts are distributed across the sample period: 13 states first increased their MW by $\geq$\$0.50 in 2019, 5 in 2020, 4 in 2021, 2 in 2022, and 4 in 2023, with 23 never-treated states providing the comparison group.\footnote{The 2023 cohort (4 states) has only the contemporaneous period ($t = g = 2023$) and zero dynamic post-treatment periods ($t > g$) in the 2018--2023 sample. The aggregate ATT is identified primarily from the 2019--2022 cohorts (24 states), which have 1--4 post-treatment years.} This staggered structure provides substantial variation for identification. Standard errors for the CS-DiD estimator use the multiplier bootstrap (999 replications) clustered at the state level, as implemented in the \texttt{did} R package \citep{callaway2021}.

\textbf{Note on estimands.} The TWFE specification (Equation \ref{eq:twfe}) uses continuous $\log(\text{MW})$ as the treatment variable, estimating the elasticity of HCBS outcomes with respect to the minimum wage level. The CS-DiD estimator uses a discrete treatment indicator based on the first year of a $\geq$\$0.50 MW increase, estimating the average treatment effect on the treated of adopting a higher minimum wage. These are distinct but complementary estimands: TWFE captures the dose-response relationship, while CS-DiD captures the average effect of crossing the policy threshold. I present both transparently and note that the TWFE estimates should be interpreted cautiously given well-known biases under heterogeneous treatment effects \citep{goodman-bacon2021, roth2023}.

\subsubsection{Sun-Abraham (2021)}

As a complementary estimator, I implement the interaction-weighted estimator of \citet{sun2021} via the \texttt{sunab()} function in the \texttt{fixest} R package. This estimator constructs event-study estimates that are robust to treatment effect heterogeneity by interacting cohort indicators with relative-time indicators and then aggregating.

\subsubsection{Triple-Difference (DDD)}

The DDD design adds a within-state comparison dimension:
\begin{equation}
\log(Y_{skt}) = \alpha + \beta_1 \cdot \log(\text{MW}_{st}) + \beta_2 \cdot \text{HCBS}_k \cdot \log(\text{MW}_{st}) + \delta_{sk} + \tau_t + \varepsilon_{skt}
\label{eq:ddd}
\end{equation}
where $k \in \{\text{HCBS}, \text{non-HCBS}\}$ indexes provider type, and $\delta_{sk}$ are state-by-type fixed effects. The coefficient $\beta_2$ captures the differential effect of minimum wage on HCBS versus non-HCBS providers. This addresses the concern that state-level trends correlated with minimum wage legislation (e.g., economic growth, urbanization) might differentially affect all Medicaid providers. Non-HCBS providers (physicians, specialists) serve as a within-state placebo because their wages are far above the minimum wage.

\subsection{Threats to Validity}

\textbf{Parallel trends.} The primary threat is that states raising minimum wages were on different HCBS provider supply trajectories even before the increases. The event study directly tests this: statistically insignificant pre-treatment coefficients support the parallel trends assumption.

\textbf{Confounding policies.} States that raise minimum wages may simultaneously adopt other policies affecting HCBS, particularly Medicaid rate increases. The ARPA Section 9817 HCBS spending increase (2021--2024) is the most important confounder. I address this with a robustness check excluding the six states with the largest ARPA-funded rate increases.

\textbf{Compositional changes.} If minimum wage increases alter the composition of providers (e.g., driving out smaller providers while larger ones remain), the average effect on provider counts may understate the impact on access. The entry/exit rate analysis and individual-versus-organizational decomposition address this margin.

\textbf{Spillovers.} In border regions, providers may relocate or serve clients across state lines in response to minimum wage differences. State-level aggregation mitigates this concern, and the sample excludes territories and military jurisdictions where border effects are most acute.


\section{Results}

\subsection{Descriptive Evidence}

Figure \ref{fig:mw_variation} documents the substantial minimum wage variation exploited in this study. Panel A shows that average minimum wages in MW-increasing states rose from approximately \$10.50 in 2018 to \$13.00 in 2023, while federal-minimum states remained essentially flat at \$7.25--\$7.55 (with small increases reflecting state-level indexing in a few cases). Panel B displays the cross-sectional distribution in 2023, ranging from \$7.25 (19 states) to over \$15.50 (Washington, California).

\begin{figure}[H]
\centering
\includegraphics[width=\textwidth]{figures/fig1_mw_variation.png}
\caption{Minimum Wage Variation Across States, 2018--2023}
\label{fig:mw_variation}
\begin{figurenotes}
Panel A shows mean state minimum wage by treatment group (states with $\geq$\$0.50 increases during the sample versus states at or near the federal minimum). Shaded bands show $\pm 1$ standard deviation. Panel B shows the cross-sectional distribution of state minimum wages in 2023. Dashed line indicates the federal minimum of \$7.25.
\end{figurenotes}
\end{figure}

Figure \ref{fig:provider_trends} shows HCBS provider supply trends by treatment group, indexed to January 2018 = 100. Both groups show broadly similar trajectories through 2019, followed by a pandemic-related dip in early 2020 and subsequent recovery. From 2021 onward, MW-increasing states show modestly slower growth in provider counts relative to federal-minimum states, though the gap is small and visually noisy. The beneficiary trend (Panel B) shows a more pronounced divergence, with MW-increasing states recovering more slowly after the pandemic.

\begin{figure}[H]
\centering
\includegraphics[width=\textwidth]{figures/fig2_provider_trends.png}
\caption{HCBS Provider Supply Trends by Treatment Group}
\label{fig:provider_trends}
\begin{figurenotes}
Indexed to January 2018 = 100. MW-increasing states are those that raised their minimum wage by $\geq$\$0.50 during 2018--2023. Federal minimum states had no such increase. Dotted vertical line marks March 2020 (COVID-19 onset). The sharp dip in mid-2020 followed by recovery reflects the temporary disruption of Medicaid claims filing during COVID-19 lockdowns, not a change in the actual provider workforce. The main analysis uses annual data with year fixed effects, which absorbs these transitory shocks.
\end{figurenotes}
\end{figure}

\subsection{Two-Way Fixed Effects Estimates}

Table \ref{tab:main_twfe} presents the TWFE results for the primary outcome (log HCBS providers) under four treatment specifications. The binary indicator for minimum wage above the federal level (Column 1) shows a significant negative coefficient of $-0.1225$ ($p < 0.01$), suggesting that states with above-federal minimum wages have approximately 11.5\% fewer HCBS providers, conditional on state and year fixed effects. However, this specification is sensitive to the definition of ``above federal,'' as most states were already above the federal minimum in 2018. TWFE estimates are presented for comparison with the literature but should be interpreted cautiously given the well-known bias of TWFE under heterogeneous treatment effects in staggered designs \citep{goodman-bacon2021}; the Callaway-Sant'Anna estimates in Section 5.3 are preferred.

\begin{table}[H]

\begin{table}[htbp]
   \caption{\label{tab:main} Effect of State Data Privacy Laws on Employment}
   \bigskip
   \centering
   \begin{tabular}{lccccc}
      \toprule
       & \multicolumn{5}{c}{log\_emp}\\
                              & (1)           & (2)                   & (3)           & (4)           & (5)\\  
      \midrule 
      Privacy Law             & 0.0577$^{**}$ & -0.0289               & -0.0335       & -0.0211       & -0.0037\\   
                              & (0.0256)      & (0.0424)              & (0.0456)      & (0.0210)      & (0.0186)\\   
       \\
      Observations            & 2,017         & 1,428                 & 2,040         & 2,040         & 2,032\\  
      Within R$^2$            & 0.01088       & $8.59\times 10^{-5}$  & 0.00251       & 0.00382       & $5.15\times 10^{-5}$\\   
      F-test                  & 375.62        & 162.54                & 305.55        & 844.75        & 295.06\\  
       \\
      state\_f fixed effects  & $\checkmark$  & $\checkmark$          & $\checkmark$  & $\checkmark$  & $\checkmark$\\   
      time\_f fixed effects   & $\checkmark$  & $\checkmark$          & $\checkmark$  & $\checkmark$  & $\checkmark$\\   
      \bottomrule
   \end{tabular}
\end{table}



\end{table}

The continuous specifications (Columns 2--4) use log minimum wage or the MW premium as treatment variables. The log MW coefficient is $-0.3437$ (SE $= 0.2583$), implying that a 10\% increase in the minimum wage reduces HCBS provider counts by approximately 3.4\%---a substantial elasticity, but imprecisely estimated. The MW premium specification (Column 3) yields a coefficient of $-0.0309$ (SE $= 0.0239$), suggesting each additional dollar above the federal minimum reduces providers by roughly 3\%. Adding log population as a control (Column 4) has virtually no impact on the point estimate.

Minimum wage hikes do not primarily drive HCBS agencies out of business---they shrink them. Table \ref{tab:outcomes} extends the analysis across six outcomes and reveals a striking asymmetry: while the number of active providers barely moves, the number of beneficiaries served drops sharply. The beneficiary elasticity is $-0.6097$ (SE $= 0.2881$, $p < 0.05$)---a 10\% minimum wage increase reduces beneficiary-months by approximately 6.1\%, nearly double the provider count effect. For a typical state serving 50,000 HCBS beneficiaries, this implies roughly 3,000 fewer people receiving care in a given month. Entry and exit rates are negligible ($0.0006$ and $-0.0029$, respectively), confirming that the adjustment operates through caseload reductions---providers serve fewer clients---rather than market exit. Sole proprietors, who are themselves the caregivers, show a three-fold larger response ($-0.8063$, SE $= 0.7413$) than organizations ($-0.2550$, SE $= 0.2170$), though both are imprecisely estimated.

\begin{table}[H]
\begin{table}[htbp]
   \caption{\label{tab:outcomes} Minimum Wage Effects Across HCBS Outcomes}
   \centering
   \begin{tabular}{lcccccc}
      \tabularnewline \midrule \midrule
      Dependent Variables: & log\_providers  & log\_benes     & entry\_rate  & exit\_rate  & log\_individual  & log\_org\\   
                           & Providers       & Benes          & Entry        & Exit        & Individual       & Org \\   
      Model:               & (1)             & (2)            & (3)          & (4)         & (5)              & (6)\\  
      \midrule
      \emph{Variables}\\
      log\_mw              & -0.3437         & -0.6097$^{**}$ & 0.0006       & -0.0029     & -0.8063          & -0.2550\\   
                           & (0.2583)        & (0.2881)       & (0.0078)     & (0.0040)    & (0.7413)         & (0.2170)\\   
      \midrule
      \emph{Fixed-effects}\\
      state                & Yes             & Yes            & Yes          & Yes         & Yes              & Yes\\  
      year                 & Yes             & Yes            & Yes          & Yes         & Yes              & Yes\\  
      \midrule
      \emph{Fit statistics}\\
      Observations         & 306             & 306            & 306          & 306         & 301              & 306\\  
      R$^2$                & 0.97748         & 0.97458        & 0.93698      & 0.50932     & 0.89700          & 0.98287\\  
      \midrule \midrule
      \multicolumn{7}{l}{\emph{Clustered (state) standard-errors in parentheses}}\\
      \multicolumn{7}{l}{\emph{Signif. Codes: ***: 0.01, **: 0.05, *: 0.1}}\\
   \end{tabular}
   
   \par \raggedright 
   All specifications use log(MW) as the treatment variable with state and year fixed effects. Standard errors clustered at the state level. Columns 1--2: log outcomes. Columns 3--4: entry and exit rates. Columns 5--6: log providers by entity type (individual vs. organizational).
\end{table}

\end{table}

\subsection{Callaway-Sant'Anna Event Study}

Figure \ref{fig:event_study} presents the Callaway-Sant'Anna event study for log HCBS providers. The estimator computes cohort-specific ATTs using only the pre-treatment periods available for each cohort---so the 2019 cohort contributes to period $e = -1$ (relative to 2018), while only later cohorts contribute to more distant leads. Pre-treatment coefficients are small and statistically indistinguishable from zero, supporting the parallel trends assumption. After treatment, the point estimates turn negative and grow modestly in magnitude. However, the 95\% confidence intervals consistently include zero, reflecting the imprecision inherent in state-level estimation with 51 units.

\begin{figure}[H]
\centering
\includegraphics[width=\textwidth]{figures/fig3_event_study.png}
\caption{Callaway-Sant'Anna Event Study: Effect of Minimum Wage on HCBS Providers}
\label{fig:event_study}
\begin{figurenotes}
Dynamic ATT estimates from \citet{callaway2021} with never-treated states as the comparison group and a universal base period. The outcome is log(HCBS providers). Shaded band shows 95\% pointwise confidence intervals. Vertical dotted line separates pre- and post-treatment periods.
\end{figurenotes}
\end{figure}

Table \ref{tab:csdd} reports the Callaway-Sant'Anna aggregate ATT estimates. The overall ATT for log providers is $-0.0480$ (SE $= 0.0689$, $p = 0.4861$), confirming the negative direction but statistical insignificance. The effect on log beneficiaries is larger in magnitude at $-0.1234$ (SE $= 0.0775$, $p = 0.1114$), approaching but not reaching conventional significance. The provider-beneficiary gap---larger beneficiary effects with smaller provider effects---is consistent with the hypothesis that providers reduce caseloads (intensive margin) before exiting entirely (extensive margin).

\begin{table}[H]
\begin{table}[ht]
\centering
\caption{Callaway-Sant'Anna (2021) Aggregate ATT Estimates}
\label{tab:csdd}
\begin{tabular}{lcc}
\toprule
 & Log Providers & Log Beneficiaries \\
\midrule
ATT & -0.0480 & -0.1234 \\
 & (0.0689) & (0.0775) \\
$p$-value & 0.4861 & 0.1114 \\
\midrule
Control group & Never-treated & Never-treated \\
Treated cohorts & 5 & 5 \\
\bottomrule
\end{tabular}
\begin{tablenotes}\small
\item \textit{Notes:} Callaway and Sant'Anna (2021) heterogeneity-robust DiD estimator. Treatment cohort defined as the first year a state raised its minimum wage above \$7.25. Never-treated states (those remaining at \$7.25 throughout 2018--2023) serve as the comparison group. Standard errors in parentheses.
\end{tablenotes}
\end{table}

\end{table}

Figure \ref{fig:multi_outcome_es} overlays the event studies for providers, beneficiaries, and individual providers. The beneficiary series diverges more strongly post-treatment, while individual providers show the widest confidence bands reflecting their smaller sample (some state-years have zero individual HCBS providers).

\begin{figure}[H]
\centering
\includegraphics[width=\textwidth]{figures/fig4_multi_outcome_es.png}
\caption{Multi-Outcome Event Study}
\label{fig:multi_outcome_es}
\begin{figurenotes}
Callaway-Sant'Anna dynamic ATT estimates for three outcomes: total HCBS providers, beneficiaries served, and individual (sole-proprietor) providers. Shaded bands show 95\% pointwise confidence intervals.
\end{figurenotes}
\end{figure}

\subsection{Triple-Difference and Robustness}

Table \ref{tab:robustness} presents the robustness battery. Column 2 reports the DDD specification comparing HCBS and non-HCBS providers within the same state. The interaction term (HCBS $\times$ log MW) is $0.1694$ (SE $= 0.1296$), positive but statistically insignificant. In the DDD specification (Table \ref{tab:robustness}, Column 2), the base coefficient on log MW ($-0.3687$, SE $= 0.2691$) captures the effect on the omitted category (non-HCBS providers), while the interaction captures the differential HCBS effect.\footnote{The base coefficient in the DDD specification ($-0.3687$) differs from the baseline TWFE estimate ($-0.3437$ in Table \ref{tab:main_twfe}, Column 2) because the DDD model includes unit fixed effects rather than state fixed effects and pools HCBS and non-HCBS observations.} The implied net effect on HCBS providers is $-0.3687 + 0.1694 = -0.1993$, smaller in magnitude than on non-HCBS providers. The non-HCBS coefficient in isolation (Column 3) is $-0.2243$ (SE $= 0.2780$), statistically insignificant, confirming that physician-level Medicaid providers are unaffected by minimum wage changes. The lack of a significant DDD interaction suggests either that the HCBS-specific labor market channel is attenuated by offsetting state policies or that the 51-state panel lacks power to detect differential effects.

\begin{table}[H]
\begin{table}[htbp]
\centering
\caption{Robustness Checks}
\label{tab:robustness}
\begin{tabular}{lccc}
\toprule
Specification & ATT & SE & 95\% CI \\
\midrule
Main (Callaway-Sant'Anna) & 0.0051 & 0.0081 & [-0.0107, 0.0209] \\
TWFE (simple) & 0.0108 & 0.0075 & [-0.0039, 0.0254] \\
TWFE (with controls) & 0.0106 & 0.0070 & [-0.0031, 0.0244] \\
Gardner Two-Stage & -0.0033 & 0.0096 & [-0.0221, 0.0155] \\
Excluding Oregon & -0.0001 & 0.0083 & [-0.0163, 0.0162] \\
Placebo: Workers WITH pension & -0.0126 & 0.0140 & [-0.0399, 0.0148] \\
\bottomrule
\end{tabular}
\begin{tablenotes}
\small
\item Note: All specifications use private sector workers ages 25-64. Standard errors clustered at state level.
\end{tablenotes}
\end{table}

\end{table}

Continuing with Table \ref{tab:robustness}: Column 4 shows the effect on individual providers ($-0.8063$, SE $= 0.7413$)---more than three times the magnitude of the main estimate, though imprecise. Column 5 shows the organizational provider effect ($-0.2550$, SE $= 0.2170$). The final column of Table \ref{tab:robustness} (Column 6) excludes ARPA rate-increase states, producing a notably larger estimate ($-0.4273$, SE $= 0.3110$), consistent with ARPA funding partially offsetting minimum wage effects in treated states.

\subsection{Placebo Test}

Figure \ref{fig:placebo_es} presents the placebo event study for non-HCBS Medicaid providers. As expected, the point estimates are centered on zero both before and after treatment, with no detectable effect of minimum wage increases on physician/specialist provider supply. This validates the identifying assumption that minimum wage effects operate through the low-wage labor channel specific to HCBS workers.

\begin{figure}[H]
\centering
\includegraphics[width=\textwidth]{figures/fig5_placebo_es.png}
\caption{Placebo Event Study: Non-HCBS Medicaid Providers}
\label{fig:placebo_es}
\begin{figurenotes}
Callaway-Sant'Anna dynamic ATT estimates for non-HCBS Medicaid providers (physicians and specialists billing CPT codes). These providers earn wages far above the minimum wage and should show no response to MW increases. Shaded band shows 95\% pointwise confidence intervals.
\end{figurenotes}
\end{figure}

\subsection{Randomization Inference}

Fisher randomization inference based on 500 permutations of treatment assignment across states yields a two-sided $p$-value of 0.186, confirming that the parametric inference is not driven by small-sample distortions in the clustered standard errors (see Figure \ref{fig:ri_distribution} in the Appendix).

\subsection{Provider Entry and Exit Dynamics}

Figure \ref{fig:entry_exit} shows provider entry and exit rates by treatment group over time. Both groups exhibit similar patterns: declining entry rates over the sample period (from roughly 4\% to 2\%), relatively stable exit rates (around 2\%), and a net positive entry rate that narrows over time. MW-increasing states show marginally higher exit rates in later years, but the differences are small and noisy. The TWFE estimates in Table \ref{tab:outcomes} confirm that neither entry nor exit rates respond significantly to minimum wage changes, suggesting the primary margin of adjustment is the intensive margin (caseload reduction) rather than the extensive margin (market entry/exit).

\begin{figure}[H]
\centering
\includegraphics[width=\textwidth]{figures/fig7_entry_exit.png}
\caption{HCBS Provider Entry and Exit Rates}
\label{fig:entry_exit}
\begin{figurenotes}
Annual entry rate = new NPIs billing HCBS codes / total HCBS providers. Exit rate = departing NPIs / total providers. MW-increasing states raised their minimum wage by $\geq$\$0.50 during 2018--2023. The first year (2018) is excluded because all providers are mechanically classified as ``new'' in the absence of prior-year data.
\end{figurenotes}
\end{figure}


\section{Discussion}

\subsection{Interpreting the Results}

The results present a consistent but nuanced picture: minimum wage increases are associated with modest reductions in HCBS provider supply and larger reductions in beneficiaries served, but most estimates are statistically insignificant in a standard two-sided test. Several interpretations are possible.

First, the effects may be real but small enough to require larger samples for precise detection. The point estimates are economically meaningful---a 10\% minimum wage increase reducing providers by 3.4\% and beneficiaries by 6.1\%---but the 51-unit panel provides limited statistical power. State-level DiD designs are inherently noisy, and the effective number of independent treatment events (28 treated states observed over 6 years) constrains precision.

Second, offsetting mechanisms may attenuate the net effect. States that raise minimum wages may simultaneously---explicitly or incidentally---increase HCBS reimbursement rates, Medicaid eligibility, or other policies that support provider supply. The ARPA robustness check partially addresses this, and the larger coefficient when excluding ARPA states ($-0.4273$ versus $-0.3437$) is consistent with rate increases partially offsetting MW effects.

Third, HCBS workers may have limited labor market mobility. Despite the theoretical prediction that higher retail wages would draw workers away from HCBS, real-world frictions---geographic immobility, licensing requirements, personal relationships with clients, scheduling preferences---may dampen the labor supply response. This would explain the small extensive-margin effects (entry/exit rates) alongside larger intensive-margin effects (beneficiaries served).

\subsection{The Intensive-Margin Channel}

The most robust finding is the divergence between provider counts and beneficiaries served. The CS-DiD ATT for beneficiaries ($-0.1234$, $p = 0.1114$) is more than twice the magnitude of the provider count ATT ($-0.0480$, $p = 0.4861$), and the TWFE beneficiary coefficient ($-0.6097$, $p < 0.05$) is statistically significant. This pattern suggests that minimum wage increases primarily cause existing providers to serve fewer clients---reducing hours, declining new referrals, or narrowing geographic coverage---rather than exiting the market entirely.

This intensive-margin response is economically intuitive. An HCBS provider who faces difficulty recruiting or retaining aides may first reduce the number of clients served per aide, increase wait times for new clients, or restrict service to less labor-intensive care. Only if the financial pressure becomes acute will the provider stop billing Medicaid entirely. The data capture this gradual adjustment: the number of active NPIs declines slowly while the total beneficiaries and payments per provider decline more rapidly.

\subsection{Individual versus Organizational Providers}

The heterogeneity between individual and organizational HCBS providers is economically informative. Individual providers---sole proprietors who are themselves the caregivers---show a three-fold larger response ($-0.8063$) than organizations ($-0.2550$), though both are imprecisely estimated. This gap reflects different margins of adjustment. An individual provider directly experiences the opportunity cost of HCBS work versus a retail job at the higher minimum wage. An organizational provider can absorb cost increases through staffing adjustments, cross-subsidization across payers, or temporary losses, and exits only when the entire business model becomes unviable.

The policy implication is that minimum wage effects on HCBS access may be concentrated in precisely the segments of the market where beneficiary relationships are most personal and continuity of care most valued. Individual providers disproportionately serve clients in rural areas and in self-directed HCBS programs where the client hires their own caregiver.

\subsection{Limitations}

Several limitations warrant acknowledgment. First, the T-MSIS data capture billing activity, not direct measures of workforce size or service quality. A provider who reduces hours but continues billing appears as ``active'' in the data, and ``beneficiaries served'' counts beneficiary-months of claims activity rather than unique individuals. This likely attenuates the estimated effects on provider counts. Second, the minimum wage treatment is measured at the state-year level, abstracting from within-year timing (mid-year effective dates), local minimum wages (e.g., Seattle, New York City), and tipped wage provisions. Third, the 6-year panel (2018--2023) limits the number of pre-treatment periods for some early-treated cohorts---the 2019 cohort (13 states) has only one pre-treatment year---potentially reducing the power of event-study pre-tests. Fourth, the COVID-19 pandemic creates a confounding shock that affects both treatment and control states differentially, and the ARPA HCBS spending increase creates a treated-state-correlated confounder that I can only partially address.

Fifth, statistical power is a binding constraint. With 51 state-level clusters, 28 treated states, and a 6-year panel, the minimum detectable effect at 80\% power for the CS-DiD provider count ATT is approximately 0.15--0.20 log points---substantially larger than the estimated $-0.0480$. The CS-DiD beneficiary ATT ($-0.1234$, $p = 0.1114$) falls closer to but still below the detection threshold, suggesting that the insignificance of the provider count result may reflect a genuine Type II error rather than a true null. This highlights the inherent tension between using state-level administrative data (which provides population coverage) and the imprecision of 51-cluster inference.


\section{Conclusion}

This paper provides the first causal estimates of how state minimum wage increases affect the supply of Home and Community-Based Services providers in Medicaid. Using the universe of T-MSIS claims linked to NPPES provider characteristics across all 50 states and DC from 2018 to 2023, I find that minimum wage increases are associated with modest reductions in HCBS provider counts and larger reductions in the number of beneficiaries served---statistically significant in TWFE though imprecisely estimated in the heterogeneity-robust CS-DiD framework. While a triple-difference design does not yield a significant differential between HCBS and non-HCBS providers, the placebo test on non-HCBS providers returns a null, and the pattern of results---larger beneficiary than provider effects, larger individual than organizational effects---is consistent with the hypothesized labor market channel.

The tension between minimum wage policy and HCBS access is not inevitable. It arises from the structural disconnect between labor markets---where minimum wages adjust mechanically---and Medicaid payment systems---where reimbursement rates are administratively determined and slow to change. States that raise minimum wages without simultaneously adjusting HCBS reimbursement rates effectively cut the wage premium for direct care work, making retail and food service jobs relatively more attractive. The policy solution is straightforward in principle: link HCBS reimbursement rates to the state minimum wage, ensuring that rate increases automatically follow wage floor increases. Several states have moved in this direction, and the ARPA HCBS spending authority provides a template.

Three directions for future research emerge. First, the ARPA Section 9817 rate increases create a natural experiment on the other side of the mechanism: do HCBS rate increases restore provider supply in states where minimum wages have risen? Second, the T-MSIS data can be disaggregated by procedure code type (personal care versus behavioral health versus attendant care) to identify which HCBS segments are most sensitive to minimum wage competition. Third, the ongoing transition from fee-for-service to managed care in Medicaid LTSS may alter the dynamics studied here, as managed care organizations may respond to minimum wage pressures differently than individual and small-agency providers.

The broader lesson is that minimum wage policy cannot be evaluated in isolation. When wages are set by regulation in one sector (minimum wage) and by administrative fiat in another (Medicaid reimbursement), failing to coordinate the two creates unintended consequences for vulnerable populations. The 4 million Americans who depend on these services for daily living deserve a policy design where a raise for the worker does not mean a loss of care for the patient.


\section*{Acknowledgements}

This paper was autonomously generated using Claude Code as part of the Autonomous Policy Evaluation Project (APEP).

\noindent\textbf{Project Repository:} \url{https://github.com/SocialCatalystLab/ape-papers}

\noindent\textbf{Contributors:} @ai1scl

\noindent\textbf{First Contributor:} \url{https://github.com/ai1scl}

\label{apep_main_text_end}
\newpage
\bibliography{references}

\newpage
\appendix

\section{Data Appendix}

\subsection{T-MSIS Data Details}

The T-MSIS data file contains approximately 227 million claim-level records stored as a 2.74 GB Apache Parquet file. Each record includes: billing provider NPI (\texttt{BILLING\_PROVIDER\_NPI\_NUM}), HCPCS procedure code (\texttt{HCPCS\_CODE}), claim month (\texttt{CLAIM\_FROM\_MONTH}), total paid amount (\texttt{TOTAL\_PAID}), total claims (\texttt{TOTAL\_CLAIMS}), and total unique beneficiaries (\texttt{TOTAL\_UNIQUE\_BENEFICIARIES}).

I process this file using Apache Arrow's lazy evaluation framework to avoid loading the full dataset into memory. The aggregation pipeline:
\begin{enumerate}
\item Classifies HCPCS codes: first character $\in \{T, H, S\}$ = HCBS; all others = non-HCBS.
\item Extracts year and month from the claim month field.
\item Aggregates to billing NPI $\times$ year $\times$ month $\times$ HCBS indicator, computing total paid, total claims, and total beneficiaries.
\item Collects the aggregated result into a data.table of approximately 2 million rows.
\end{enumerate}

\subsection{NPPES Extract Construction}

The NPPES bulk data file is a 1.05 GB compressed download from the CMS NPI registry. I extract four fields for each of the 9.3 million registered NPIs: NPI number, practice state (from the practice location address), entity type code (1 = individual, 2 = organizational), and sole proprietor indicator. The match rate between T-MSIS billing NPIs and NPPES records is approximately 92\%.

\subsection{Minimum Wage Panel}

State minimum wage data are compiled from Department of Labor records and NCSL reports. The panel covers 51 jurisdictions (50 states + DC) $\times$ 6 years (2018--2023) = 306 observations for the annual analysis. Treatment cohort assignment uses the ``first $\geq$\$0.50 increase'' definition, yielding 5 treatment cohorts:

\begin{table}[H]
\centering
\caption{Treatment Cohort Distribution}
\begin{tabular}{lcc}
\toprule
Treatment Cohort & Number of States & Examples \\
\midrule
2019 & 13 & CA, NY, MA, WA, CO, AZ \\
2020 & 5 & IL, NM, NV, ME, MN \\
2021 & 4 & CT, VA, FL, RI \\
2022 & 2 & DE, HI \\
2023 & 4 & NE, SD, DC, OR \\
Never-treated & 23 & TX, GA, AL, MS, TN, SC, IN, WI, ... \\
\bottomrule
\end{tabular}
\label{tab:cohorts}
\par \raggedright \small
\textit{Notes:} Treatment cohort defined as the first year a state raised its MW by $\geq$\$0.50. All minimum wage increases in this sample took effect on January 1 of the indicated year. The 2023 cohort contributes one post-treatment observation ($t = g = 2023$).
\end{table}

\subsection{Variable Definitions}

\begin{table}[H]
\centering
\caption{Variable Definitions}
\small
\begin{tabular}{p{3.5cm}p{10cm}}
\toprule
Variable & Definition \\
\midrule
$n_{\text{providers}}$ & Count of unique billing NPIs filing at least one HCBS claim (T/H/S code) in the state-year \\
$\text{total\_benes}$ & Sum of unique beneficiaries served across all HCBS claims in the state-year \\
$\text{entry\_rate}$ & Number of NPIs billing HCBS for the first time, divided by total HCBS providers, annualized \\
$\text{exit\_rate}$ & Number of NPIs whose last HCBS claim occurred in the state-year, divided by total providers \\
$\text{min\_wage}$ & State minimum wage as of January 1 of the year \\
$\log(\text{MW})$ & Natural log of state minimum wage \\
$\text{mw\_premium}$ & State minimum wage minus federal minimum (\$7.25) \\
$\text{above\_federal}$ & Indicator: state MW $>$ \$7.25 \\
$\text{first\_treat\_year}$ & First year state raised MW by $\geq$\$0.50 during sample period; 0 if never-treated \\
\bottomrule
\end{tabular}
\label{tab:variables}
\end{table}


\section{Identification Appendix}

\subsection{Pre-Trends Test}

The Callaway-Sant'Anna event study in Figure \ref{fig:event_study} provides the primary pre-trends diagnostic. Pre-treatment coefficients (periods $e = -5$ through $e = -1$) are jointly tested against the null that all equal zero. The individual point estimates are small in magnitude and statistically insignificant, with no discernible trend in the pre-period. This supports the parallel trends assumption underlying the DiD design.

\subsection{Group-Level ATT Estimates}

The Callaway-Sant'Anna estimator produces cohort-specific ATT estimates. The 2019 cohort (13 states, longest post-period) shows the most negative estimates, consistent with cumulative exposure effects. Later cohorts show smaller and noisier effects, reflecting both shorter post-periods and smaller cohort sizes.

\subsection{Sun-Abraham Comparison}

The Sun-Abraham interaction-weighted estimator produces an aggregate ATT of $-0.0480$ (SE $= 0.0667$, $p = 0.47$) for log providers, nearly identical to the Callaway-Sant'Anna estimate of $-0.0480$ (SE $= 0.0689$). The event-study coefficients show flat pre-trends and a gradually widening negative trajectory after treatment, consistent with the CS-DiD event study in Figure \ref{fig:event_study}. The close agreement between estimators---despite different aggregation weights and treatment effect heterogeneity corrections---strengthens confidence that the null result for provider counts is robust.


\section{Robustness Appendix}

\subsection{ARPA Rate Increase Exclusion}

Six states received substantial HCBS rate increases funded by ARPA Section 9817: California, New York, North Carolina, Colorado, Virginia, and New Mexico. These states had above-average minimum wages, creating a positive correlation between the treatment (MW increase) and the confounder (rate increase). Excluding these 6 states (36 state-year observations) increases the TWFE coefficient from $-0.3437$ to $-0.4273$ (SE $= 0.3110$), consistent with ARPA funding partially offsetting the minimum wage effect on HCBS providers.

\subsection{Dose-Response}

The dose-response specification uses MW premium and MW premium squared as treatment variables. The linear term is negative and the quadratic term is positive but small, suggesting a concave relationship: the marginal effect of each additional dollar above the federal minimum diminishes at higher MW levels, possibly because states with the highest minimum wages also tend to have the highest HCBS reimbursement rates.

\subsection{Monthly Specification}

Replacing annual data with monthly observations increases the sample from 306 to 3,672 state-months and enables year-month fixed effects. The monthly TWFE coefficient on log MW is $-0.3472$ (SE $= 0.2617$), nearly identical to the annual estimate ($-0.3437$), consistent with the effect being stable across time aggregations.

\subsection{Fisher Randomization Inference}

The Fisher permutation test based on 500 random reassignments of treatment cohorts yields a two-sided $p$-value of 0.186. The permutation distribution is approximately symmetric and centered near zero, as expected under the null. The actual coefficient of $-0.3437$ lies at approximately the 10th percentile of the left tail, consistent with a modest but not highly unusual negative effect.

\begin{figure}[H]
\centering
\includegraphics[width=\textwidth]{figures/fig6_ri_distribution.png}
\caption{Randomization Inference: Distribution of Permuted Coefficients}
\label{fig:ri_distribution}
\begin{figurenotes}
Distribution of TWFE coefficients from 500 permutations of treatment assignment across states. The solid vertical line marks the actual coefficient ($-0.3437$). The dashed line marks its mirror ($+0.3437$). The RI $p$-value is the fraction of permuted coefficients exceeding the actual in absolute value.
\end{figurenotes}
\end{figure}


\section{Heterogeneity Appendix}

\subsection{Individual versus Organizational Providers}

Individual providers (NPI entity type = 1) are sole proprietors who are themselves the caregivers. Their TWFE response ($-0.8063$, SE $= 0.7413$) is more than three times larger than organizational providers ($-0.2550$, SE $= 0.2170$), though the individual estimate is imprecise due to greater cross-state variation and some state-years with zero individual HCBS providers. This heterogeneity is economically intuitive: individual providers directly face the opportunity cost of HCBS work versus alternative employment, while organizations can adjust through staffing margins.

\subsection{Entry and Exit Margins}

Neither entry rates ($0.0006$, SE $= 0.0078$) nor exit rates ($-0.0029$, SE $= 0.0040$) respond significantly to minimum wage changes. Combined with the significant beneficiary effect, this suggests the intensive margin (caseload reduction) dominates the extensive margin (market entry/exit) in the short to medium run.


\end{document}
