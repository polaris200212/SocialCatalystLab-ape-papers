\begin{table}[H]
\centering
\caption{Robustness of Minimum Wage Effects on Parental Co-residence}
\label{tab:robustness}
\begin{threeparttable}
\begin{tabular}{lcc}
\toprule
Specification & ATT & SE \\
\midrule
Baseline (MW gap $\geq$ \$1.00, never-treated) & -0.540 & (0.446) \\
MW gap $\geq$ \$0.50 & -1.750 & (1.883) \\
MW gap $\geq$ \$1.50 & -0.287 & (0.754) \\
MW gap $\geq$ \$2.00 & 0.082 & (0.475) \\
Not-yet-treated control group & -0.496 & (0.794) \\
Excluding 2021 & -0.540 & (0.439) \\
Pre-pandemic only (2015--2019) & -0.053 & (0.461) \\
Additional outcome: \% other arrangements & 0.019 & (0.445) \\
Continuous MW gap (linear, TWFE) & -0.111 & (0.120) \\
Alternative DV: \% independent & 0.522 & (0.564) \\
\bottomrule
\end{tabular}
\begin{tablenotes}[flushleft]
\small
\item \textit{Notes:} Each row reports the overall ATT from a separate CS-DiD estimation unless otherwise noted.
The baseline uses the \$1.00 threshold (MW gap $\geq$ \$1.00) with never-treated states as controls, as reported in Table~\ref{tab:main_results}.
Alternative threshold rows redefine treatment cohorts using the specified gap.
The other-arrangements outcome uses the share of 18--34 year-olds living with other relatives or other nonrelatives of the householder (B09021\_013E + B09021\_014E). Because parental co-residence, independent living, and other arrangements sum to 100\%, this is not an independent test.
The baseline sample is 357 state-year observations (51 jurisdictions $\times$ 7 years: 2015--2019, 2021--2022; 2020 not released).
``Excluding 2021'' drops the first post-pandemic year ($N = 306$). ``Pre-pandemic only'' retains 2015--2019 ($N = 255$).
Standard errors (in parentheses) are clustered at the state level.
* $p<0.10$, ** $p<0.05$, *** $p<0.01$.
\end{tablenotes}
\end{threeparttable}
\end{table}
