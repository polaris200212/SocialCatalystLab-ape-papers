\documentclass[12pt]{article}

% UTF-8 encoding and fonts
\usepackage[utf8]{inputenc}
\usepackage[T1]{fontenc}
\usepackage{lmodern}  % Latin Modern font - fixes < > rendering issues

% Page setup
\usepackage[margin=1in]{geometry}
\usepackage{setspace}
\onehalfspacing

% Typography
\usepackage{microtype}

% Math and symbols
\usepackage{amsmath,amssymb}

% Graphics
\usepackage{graphicx}
\usepackage{float}
\usepackage{subcaption}

% Tables
\usepackage{booktabs}
\usepackage{array}
\usepackage{multirow}
\usepackage{threeparttable} % provides tablenotes
\usepackage{longtable}
\usepackage{pdflscape}
\usepackage{siunitx}
\sisetup{detect-all=true, group-separator={,}, group-minimum-digits=4}

% Bibliography
\usepackage{natbib}
\bibliographystyle{aer}  % American Economic Review style

% Hyperlinks
\usepackage{hyperref}
\hypersetup{
    colorlinks=true,
    linkcolor=blue,
    citecolor=blue,
    urlcolor=blue
}
\usepackage[nameinlink,noabbrev]{cleveref}

% Timing data (generated by timing_log.py)
\IfFileExists{timing_data.tex}{\newcommand{\apepcurrenttime}{1h 4m}
\newcommand{\apepcumulativetime}{1h 4m}
}{
  \newcommand{\apepcurrenttime}{N/A}
  \newcommand{\apepcumulativetime}{N/A}
}

% Captions
\usepackage{caption}
\captionsetup{font=small,labelfont=bf}

% Section formatting
\usepackage{titlesec}
\titleformat{\section}{\large\bfseries}{\thesection.}{0.5em}{}
\titleformat{\subsection}{\normalsize\bfseries}{\thesubsection}{0.5em}{}

% Custom commands
\newcommand{\E}{\mathbb{E}}
\newcommand{\Var}{\text{Var}}
\newcommand{\Cov}{\text{Cov}}
\newcommand{\ind}{\mathbb{I}}
\newcommand{\sym}[1]{\ifmmode^{#1}\else\(^{#1}\)\fi}

% APEP Working Paper formatting
\title{Where Cultural Borders Cross: Gender Equality at the Intersection of Language and Religion in Swiss Direct Democracy\footnote{This paper is a revision of APEP-0439. See \url{https://github.com/SocialCatalystLab/ape-papers/tree/main/apep_0439} for the original.}}
\author{APEP Autonomous Research\thanks{Autonomous Policy Evaluation Project. Correspondence: scl@econ.uzh.ch} \and @SocialCatalystLab}
\date{\today}

\begin{document}

\maketitle

\begin{abstract}
\noindent
Cultural economics routinely treats language, religion, and ethnicity as independent, additive determinants of preferences---the ``modularity assumption.'' We provide the first direct test. Switzerland's language border and confessional boundary, both predetermined centuries ago, create a natural $2 \times 2$ factorial design. Analyzing 8,727 municipality-level observations across six gender-equality referenda (1981--2021), we find large main effects---French-speaking municipalities vote 15.5 percentage points more progressively and Catholic heritage reduces progressivism by 8.3~pp---but the interaction is precisely zero ($-0.09$~pp; 95\% CI: [$-1.7$, 1.5]; permutation $p = 0.94$). The null interaction is remarkably stable across specifications, clustering levels, and sample restrictions. Modularity holds: cultural dimensions operate through genuinely independent channels, validating the single-dimension approach that dominates the literature.
\end{abstract}

\vspace{1em}
\noindent\textbf{JEL Codes:} Z13, J16, D72, P16 \\
\noindent\textbf{Keywords:} cultural economics, modularity, gender norms, language border, direct democracy, intersectionality, Switzerland

\newpage

%=============================================================================
\section{Introduction}
%=============================================================================

A French-speaking Protestant woman in Lausanne and a German-speaking Catholic woman in Lucerne live 200 kilometers apart in the same country, share the same federal laws, and vote in the same national referenda. Yet when the question on the ballot concerns gender equality, their communities diverge sharply. The puzzle is not that cultural differences exist---decades of research document the R\"{o}stigraben, the invisible ``hash-brown trench'' dividing French and German Switzerland \citep{eugster2011demand}. The puzzle is what happens when multiple cultural cleavages intersect.

Cultural economics has made enormous progress by isolating the effect of single cultural dimensions on economic outcomes: language on social insurance preferences \citep{eugster2011demand}, religion on work attitudes \citep{basten2013beyond, becker2009was}, ethnicity on trust \citep{nunn2012ruggedness}. But this progress rests on an assumption that has never been tested: that cultural dimensions operate \emph{modularly}---each contributing an independent, additive increment to preferences. If a community is both francophone and Catholic, the standard approach predicts its gender attitudes by summing the francophone effect and the Catholic effect. No interaction, no surprise, no loss of information from studying one dimension at a time.

This assumption matters. If it holds, the single-dimension approach that dominates the literature is not merely convenient but correct---each dimension can be studied in isolation without bias. If it fails, the entire enterprise of one-dimension-at-a-time cultural economics rests on shaky foundations, and every estimated effect is potentially confounded by unmodeled interactions. Despite its centrality, the modularity assumption has never been directly tested.

This paper provides that test. We exploit Switzerland's two historically predetermined cultural boundaries---the language border (R\"{o}stigraben) and the confessional boundary from the 16th-century Reformation---to construct a natural $2 \times 2$ factorial design: French-Protestant, French-Catholic, German-Protestant, and German-Catholic municipalities. The language frontier was fixed by 5th-century Germanic settlement; confessional status was determined by Reformation-era cantonal choices codified in constitutions for five centuries \citep{church2004history, cantoni2012consequences}. Both boundaries predate modern gender politics by centuries. Neither is collinear with the other. Where they cross---in the cantons of Fribourg, Bern, and Valais---we can directly test whether their effects on gender attitudes are additive.

We measure gender attitudes using voting behavior in Swiss direct democracy, which provides repeated, high-stakes, precisely defined measurements of revealed preferences. Our sample covers six national gender-equality referenda spanning forty years (1981--2021): equal rights, maternity insurance, women's representation, abortion access, paternity leave, and marriage equality. For each of the 1,463 municipalities in our sample, we observe the yes-share on each referendum, yielding 8,727 municipality--referendum observations.

Three findings emerge, each corresponding to a prediction from our conceptual framework.

First, and most importantly, the modularity assumption holds. The language--religion interaction is precisely zero: $-0.09$ percentage points (95\% CI: [$-1.7$, 1.5]; permutation $p = 0.94$). An additive model predicts that French-Catholic municipalities should average 53.8\% on the gender progressivism index; the actual average is 53.7\%. The deviation is 0.1 percentage points---well within noise. The confidence interval rules out interactions larger than 1.7~pp in either direction, establishing a tight bound on any non-additivity.

Second, both main effects are large and precisely estimated. French-speaking municipalities vote 15.5 percentage points more progressively on gender issues ($p < 0.001$), while Catholic heritage reduces progressivism by 8.3~pp ($p < 0.001$). Within bilingual cantons---where municipalities share the same government, tax regime, and school system---the language gap remains a highly significant 9.3~pp, confirming that language operates through cultural channels rather than cantonal institutions. The main effects vary dramatically across referenda, from 1~pp to 33~pp for language and from 2~pp to 23~pp for religion, but the interaction remains consistently near zero across all six votes.

Third, the domain-specificity of the main effects provides a striking falsification. On non-gender referenda (immigration, defense, taxation), the cultural pattern \emph{reverses}: French-speaking municipalities are 17.1~pp \emph{more conservative}, and Catholic municipalities are 4.5~pp \emph{more progressive}. The interaction on non-gender issues is also near zero (1.0~pp). Modularity holds in both domains, but the directions of the main effects are domain-specific. This asymmetry---progressive on gender, conservative on non-gender---demonstrates that the language dimension captures genuine cultural content, not a generic liberalism.

These results contribute to cultural economics by validating, for the first time, the modularity assumption that underpins the literature. The standard practice of studying one cultural dimension at a time is not just convenient---it is justified. A study examining only the R\"{o}stigraben correctly estimates the language gap; the absence of interaction means nothing is lost by ignoring religion, and nothing is gained by modeling the intersection. This is good news for the large body of single-dimension research in cultural economics.

At the same time, the result is theoretically informative. That modularity holds despite plausible reasons to expect interaction---Catholic institutions could dampen francophone progressivism, cross-cutting identities could generate friction, selective cultural transmission could amplify distinctiveness---suggests that language and religion operate through genuinely separate channels. Language shapes preferences through media exposure, social networks, and cultural reference frames; religion shapes them through institutional participation, moral prescriptions, and community organization. These channels appear to be hermetically sealed from each other, at least in the Swiss context.

More broadly, the paper demonstrates that the modularity assumption is \emph{testable}---and tests it in a setting uniquely suited to the task. Switzerland's factorial design, historically predetermined cultural boundaries, and repeated high-stakes measurements of revealed preferences create ideal conditions for detecting interactions if they exist. Finding precisely zero in this favorable setting strengthens the assumption considerably.


%=============================================================================
\section{Conceptual Framework} \label{sec:framework}
%=============================================================================

\subsection{The Modularity Assumption}

The dominant approach in cultural economics treats cultural dimensions as independent, additive determinants of preferences. Consider a municipality's gender preferences $\theta_m$ as a function of its cultural exposures. The standard specification is:
\begin{equation} \label{eq:modular}
    \theta_m = \alpha + \beta_L L_m + \beta_R R_m + \varepsilon_m
\end{equation}
where $L_m$ indicates francophone status, $R_m$ indicates Catholic heritage, $\beta_L$ is the language effect, and $\beta_R$ is the religion effect. This \emph{modular} model is implicit in the design of most studies in the Swiss spatial discontinuity literature: \citet{eugster2011demand} estimate $\beta_L$ while ignoring religion, and \citet{basten2013beyond} estimate $\beta_R$ while ignoring language. Each boundary is studied in isolation, with the tacit assumption that the other dimension is orthogonal or absorbed by controls.

The modularity assumption is convenient---it allows researchers to study one cultural dimension at a time without worrying about interactions. But is it correct? If cultural dimensions influenced preferences through entirely separate channels, with no cross-talk between them, modularity would hold. But this seems unlikely \emph{a priori}. Language shapes media exposure, social networks, and cultural reference frames; religion shapes institutional participation, moral prescriptions, and community organization. These channels are not obviously hermetically sealed. A francophone community embedded in Catholic institutions might process cultural signals differently from one embedded in Protestant institutions.

Whether modularity holds is ultimately an empirical question. We formalize the alternatives.

\subsection{Three Models of Cultural Interaction}

We consider the general model:
\begin{equation} \label{eq:interaction}
    \theta_m = \alpha + \beta_L L_m + \beta_R R_m + \gamma (L_m \times R_m) + \varepsilon_m
\end{equation}
The parameter $\gamma$ captures the deviation from additivity. Three cases are possible.

\textbf{Model A: Additive (Modular).} If $\gamma = 0$, cultural dimensions operate independently. The effect of being francophone is $\beta_L$ regardless of confessional context, and the effect of being Catholic is $\beta_R$ regardless of linguistic context. This is the maintained assumption in the existing literature. If true, single-dimension studies are unbiased.

\textbf{Model B: Super-additive (Reinforcing).} If $\gamma > 0$, the dimensions amplify each other. A francophone municipality in a Catholic canton would be \emph{more} progressive than the sum of the two effects predicts. This could arise if progressive cultural signals from francophone networks are amplified by Catholic communal solidarity---a ``double distinctiveness'' that exceeds the sum of parts.

\textbf{Model C: Sub-additive (Dampening).} If $\gamma < 0$, the dimensions offset each other. A francophone municipality in a Catholic canton would be \emph{less} progressive than the additive prediction. This could arise if Catholic institutional infrastructure---parish organizations, church-affiliated schools, communal moral authority---creates friction that slows the transmission of progressive norms from francophone cultural centers.

\subsection{Why Interaction Might Occur}

Three theoretical mechanisms predict non-zero interaction, specifically sub-additivity, for gender attitudes.

\textbf{Institutional friction.} Catholic cantons historically maintained stronger parish-level institutions that reinforced traditional gender roles \citep{guiso2003people, becker2009was}. In Protestant areas, the Reformation weakened parish-mediated social control, creating more space for secular cultural norms---transmitted through francophone media, literature, and intellectual networks---to shape gender attitudes. In Catholic areas, the persistence of church-mediated institutions creates a counterweight. The prediction is that Catholicism dampens the progressive effect of francophone culture on gender issues, yielding $\gamma < 0$.

\textbf{Cross-cutting identity prescriptions.} \citet{akerlof2000economics} argue that identity shapes behavior through prescriptions about ``appropriate'' conduct for group members. French-Catholic individuals face cross-cutting prescriptions: progressive norms from their linguistic community and traditional norms from their religious community. These competing prescriptions partially offset, yielding sub-additivity.

\textbf{Selective cultural transmission.} \citet{bisin2001economics} model cultural transmission as a process in which parents invest in transmitting traits that distinguish their group. In regions where francophone identity is culturally distinct from the local majority, the distinctiveness motive amplifies progressive gender norms as a marker of identity. Where francophones share the majority religion, this motive weakens, and norms converge toward the local norm.

All three mechanisms predict $\gamma < 0$. But they could also be empirically irrelevant if the channels through which language and religion influence preferences are sufficiently separate.

\subsection{Testable Predictions}

We test three predictions:

\begin{description}
    \item[\textbf{P1} (Non-additivity):] $\gamma \neq 0$. The interaction between language and religion is non-zero for gender attitudes. This is the fundamental test of modularity.
    \item[\textbf{P2} (Variation across referenda):] If $\gamma \neq 0$, the magnitude of the interaction varies across referenda in a theoretically predictable way---larger where Catholic doctrine is most explicit (abortion) and smaller where doctrine is silent (formal equality).
    \item[\textbf{P3} (Domain specificity):] The interaction pattern, whatever it is, should differ between gender and non-gender referenda. If language and religion generate competing prescriptions only on gender issues, non-gender referenda should yield $\gamma \approx 0$ regardless of the gender-domain result.
\end{description}

The test of P1 determines whether modularity holds or fails. If P1 is rejected ($\gamma \neq 0$), P2 and P3 characterize the nature and boundaries of the interaction. If P1 is not rejected ($\gamma \approx 0$), modularity is confirmed---and P2 is moot because there is nothing to vary. P3 remains informative even under modularity: it reveals whether the \emph{main effects} are domain-specific, which speaks to the content of cultural transmission even when the interaction is zero.


%=============================================================================
\section{Institutional Background} \label{sec:background}
%=============================================================================

\subsection{The R\"{o}stigraben}

The R\"{o}stigraben (``hash-brown trench'') separates French-speaking western Switzerland from German-speaking central and eastern Switzerland.\footnote{The name derives from R\"{o}sti, a potato dish popular in German Switzerland.} The boundary traces to the 5th-century settlement of Germanic Alamanni tribes, who pushed westward but stopped short of Romandie \citep{church2004history}. Article 70 of the Swiss Constitution assigns each municipality a single official language, making the boundary sharp: within a few kilometers, the dominant language shifts from over 80\% French to over 80\% German. This sharpness approximates a spatial discontinuity in cultural exposure while holding geography, climate, and federal institutions roughly constant \citep{eugster2011demand}.

French-speaking municipalities consistently vote more progressively on social insurance and welfare state expansion \citep{eugster2011demand, eugster2017culture}. These differences persist within bilingual cantons (Fribourg, Bern, Valais) and survive controls for income, education, and urbanization. \citet{faessler2024culture} show that the cultural gap extends to fertility and mortality decisions.

\subsection{The Confessional Boundary}

The Protestant Reformation of the 1520s--1530s split Swiss cantons into Catholic and Protestant jurisdictions under the principle of \emph{cuius regio, eius religio}. This division, codified in cantonal constitutions, has persisted for five centuries \citep{cantoni2012consequences, boppart2013poor}. Historically Protestant cantons include Zurich, Bern, Basel, Vaud, Neuch\^{a}tel, and Geneva. Historically Catholic cantons include Lucerne, Uri, Schwyz, Fribourg, Valais, and Jura.

The assignment of confessional status at the cantonal level is crucial for our identification. Individual religiosity is endogenous---people choose whether and how intensely to practice. But the historical confessional tradition of a canton was determined by the decisions of 16th-century rulers and has remained fixed for five centuries. We exploit this predetermined cantonal-level variation, not contemporary individual religious choices.

\citet{basten2013beyond} show that the confessional boundary produces sharp differences in economic attitudes: Protestant municipalities prefer individual responsibility and market-oriented policies. \citet{guiso2003people} document religion's broader influence on economic preferences.

\subsection{Where the Boundaries Cross}

The two boundaries are not collinear. They intersect in western Switzerland, creating a natural $2 \times 2$ factorial structure. Fribourg is Catholic and bilingual, containing both French-Catholic and German-Catholic municipalities. Its neighbor Bern is Protestant and bilingual. Valais is Catholic and bilingual. Vaud is French-speaking and Protestant. This geographic configuration creates four distinct culture groups---German-Protestant (reference), German-Catholic, French-Protestant, and French-Catholic---that map directly to the factorial design in \cref{eq:interaction}.

The factorial structure is not perfectly balanced---there are more German-Protestant municipalities (601) than French-Catholic ones (229)---but all four cells are well-populated, ensuring adequate statistical power for the interaction test.

\subsection{Direct Democracy as Measurement}

Switzerland's direct democracy generates repeated, high-stakes measurements of revealed preferences \citep{matsusaka2005direct}. We analyze six national gender referenda spanning 1981--2021:
\begin{itemize}
    \item \textbf{1981}: Equal rights amendment (formal legal equality)
    \item \textbf{1999}: Maternity insurance (maternal social insurance)
    \item \textbf{2000}: Women's representation (political quotas)
    \item \textbf{2002}: Abortion access (reproductive rights)
    \item \textbf{2020}: Paternity leave (paternal family roles)
    \item \textbf{2021}: Marriage for All (LGBTQ+ rights)
\end{itemize}
These referenda span the full range of gender-related policy: constitutional rights, social insurance, political representation, reproductive autonomy, family roles, and relationship recognition. They also vary in their salience to Catholic doctrine---from highly salient (abortion) to largely neutral (formal equality)---providing the variation needed to test Prediction P2. Critically, the binary yes/no format and mandatory recording of municipality-level results ensure precise, comparable measurements across four decades.


%=============================================================================
\section{Data} \label{sec:data}
%=============================================================================

\subsection{Data Sources}

Our analysis combines three data sources. Municipality-level referendum results come from the \texttt{swissdd} package \citep{swissdd2024}, which provides harmonized vote data from the Swiss Federal Statistical Office (BFS) for all federal referenda from 1981 to 2024. For each municipality--referendum pair, we observe the yes-share: the proportion of valid votes cast in favor of the proposal.

Municipality characteristics---dominant language, canton assignment, and population---come from BFS regional statistics \citep{bfs2024}. We restrict attention to German- and French-speaking municipalities. Historical confessional status is assigned at the cantonal level using Reformation-era choices \citep{cantoni2012consequences, basten2013beyond}. This classification is time-invariant and predetermined centuries before our outcome period, addressing endogenous religious sorting concerns.

\subsection{Sample Construction}

Our estimation sample consists of 8,727 municipality--referendum observations, constructed from 1,463 unique German- and French-speaking municipalities observed across six gender referenda. Italian-speaking municipalities are excluded for clean identification. Five mixed-confession cantons (Aargau, Grisons, St.\ Gallen, Solothurn, Thurgau) are excluded from the main analysis because their confessional heritage is ambiguous---these cantons experienced complex, contested Reformations that left them with substantial populations of both traditions, making binary classification unreliable. Including them in robustness checks (reclassified by their pre-1800 majority denomination) yields substantively identical results.

The four culture groups distribute as follows: 601 German-Protestant, 232 German-Catholic, 401 French-Protestant, and 229 French-Catholic municipalities. The sample is unbalanced---the German-Protestant cell is largest---but all cells are well-populated.

\subsection{Summary Statistics}

\Cref{tab:tab:summary} reports summary statistics by culture group. The gender progressivism index---the average yes-share across six referenda---ranges from 38.4\% in German-Catholic municipalities to 62.0\% in French-Protestant ones. Two patterns are immediately visible. First, French-speaking municipalities are more progressive within each religious tradition: the language gap is 15.5~pp in Protestant areas (62.0 vs.\ 46.5) and 15.3~pp in Catholic areas (53.7 vs.\ 38.4). Second, these two language gaps are virtually identical---foreshadowing the zero interaction we document formally below.

\begin{table}[htbp]
\centering
\caption{Summary Statistics: New State vs Parent State Districts}
\label{tab:summary}
\begin{tabular}{lccc}
\hline\hline
 & New State & Parent State & $p$-value \\
\hline
Mean Nightlights & 8862.2 & 15587.7 & 0.000 \\
Mean Log(NL+1) & 8.215 & 9.160 & 0.000 \\
Population (2011, millions) & 1.25 & 2.37 & 0.000 \\
Literacy Rate & 0.583 & 0.556 & 0.071 \\
Ag. Worker Share & 0.362 & 0.434 & 0.001 \\
SC Share & 0.132 & 0.179 & 0.000 \\
ST Share & 0.276 & 0.083 & 0.000 \\
\hline
Districts & 55 & 159 & \\
\hline\hline
\end{tabular}
\begin{minipage}{0.9\textwidth}
\vspace{0.2cm}
\footnotesize \textit{Notes:} Pre-treatment means (1994--1999) for districts in newly created states (Uttarakhand, Jharkhand, Chhattisgarh) vs remaining districts in parent states (UP, Bihar, MP). Nightlights from DMSP calibrated luminosity. Population and sociodemographic characteristics from Census 2011. $p$-values from two-sample $t$-tests of equal means across districts.
\end{minipage}
\end{table}


For the falsification exercise, we construct a separate panel of non-gender referenda from the same municipalities.


%=============================================================================
\section{Empirical Strategy} \label{sec:strategy}
%=============================================================================

\subsection{Main Specification}

Our primary estimating equation implements the interaction model from \cref{eq:interaction}:
\begin{equation} \label{eq:main}
    Y_{mr} = \alpha + \beta_1 \cdot \text{French}_m + \beta_2 \cdot \text{Catholic}_m + \beta_3 \cdot (\text{French}_m \times \text{Catholic}_m) + \delta_r + \varepsilon_{mr}
\end{equation}
where $Y_{mr}$ is the yes-share in municipality $m$ on referendum $r$; $\text{French}_m$ indicates francophone status; $\text{Catholic}_m$ indicates historically Catholic canton; $\delta_r$ are referendum fixed effects; and $\varepsilon_{mr}$ is the error term. Standard errors are clustered at the municipality level.

The coefficient $\beta_1$ captures the language gap in gender progressivism for Protestant municipalities. The coefficient $\beta_2$ captures the religion gap for German-speaking municipalities. The interaction $\beta_3$ is the parameter of primary interest: it tests P1. Under modularity, $\beta_3 = 0$. A non-zero $\beta_3$ indicates that the language gap differs between Protestant and Catholic areas.

\subsection{Within-Canton Identification}

A concern with the national specification is that language indicators may capture canton-level institutional differences. We address this with canton fixed effects:
\begin{equation} \label{eq:canton}
    Y_{mr} = \alpha + \beta_1 \cdot \text{French}_m + \gamma_c + \delta_r + \varepsilon_{mr}
\end{equation}
Because confessional status is assigned at the cantonal level, religion and the interaction are absorbed by canton fixed effects. This specification identifies the language gap exclusively from within bilingual cantons (Fribourg, Bern, Valais), where municipalities share the same cantonal government, tax regime, and political parties. The within-canton estimate is a lower bound on the language effect, as it removes all between-canton variation in culture.

\subsection{Time-Varying Estimates}

To test P2 (variation across referenda), we estimate \cref{eq:main} separately for each referendum:
\begin{equation} \label{eq:time}
    Y_{m} = \alpha_r + \beta_{1r} \cdot \text{French}_m + \beta_{2r} \cdot \text{Catholic}_m + \beta_{3r} \cdot (\text{French}_m \times \text{Catholic}_m) + \varepsilon_{m}
\end{equation}
The referendum-specific interaction $\beta_{3r}$ reveals whether the interaction varies across issues. Under strict modularity, $\beta_{3r} \approx 0$ for all referenda. Moderate departures from modularity might yield non-zero interactions for some referenda but not others.

\subsection{Identification Assumptions}

Our identification relies on two assumptions. First, historical language and confessional assignments are exogenous to modern gender attitudes. The language boundary was fixed by 5th-century settlement; confessional status by 16th-century cantonal choices \citep{cantoni2012consequences}. Both predate modern gender politics by centuries.

Second, the estimated interaction reflects cultural mechanisms rather than confounded institutional differences. We address this threat three ways: (i) within-canton estimates absorb all canton-level confounders; (ii) the falsification test (P3) shows whether the pattern is specific to gender referenda; (iii) robustness checks demonstrate stability across samples, clustering, and specifications. A remaining concern is spatial sorting---selective migration of progressives to particular municipality types. Swiss residential mobility is low, cross-language migration is rare, and confessional classification is based on historical cantonal status rather than individual religiosity.

We note that our design is not a formal spatial RDD---we do not exploit distance to a geographic boundary as a running variable. Rather, we treat language and religion as historically predetermined binary indicators and estimate their interaction across the full sample. We view this approach as complementary to the formal RDD designs of \citet{eugster2011demand} and \citet{basten2013beyond}, with the advantage of enabling a two-dimensional interaction analysis that a single-cutoff RDD cannot accommodate.


%=============================================================================
\section{Results} \label{sec:results}
%=============================================================================

\subsection{Large Main Effects of Language and Religion}

\Cref{tab:main} reports the main results. Column (1) estimates the language gap alone: French-speaking municipalities vote 14.8 percentage points more progressively on gender referenda ($p < 0.001$). This confirms and extends the R\"{o}stigraben documented by \citet{eugster2011demand}: the language gap is even sharper for gender issues than for social insurance.

Column (2) estimates the religion gap alone: Catholic municipalities are 6.8~pp less progressive ($p < 0.001$). This finding---that confessional heritage shapes gender attitudes even five centuries after the Reformation---emerges clearly when the analysis is restricted to gender-specific proposals.

Column (3) includes both indicators simultaneously. Both coefficients are stable and, indeed, sharpen slightly: the language gap increases to 15.5~pp and the religion gap to $-8.3$~pp. The stability of both coefficients when the other dimension is added is a first indication of modularity---if language and religion operated through the same channel, including both simultaneously would attenuate one or both.

\begin{table}[!h]
\centering
\caption{\label{tab:mainresults}Main Results: Effect of Network Minimum Wage on Employment}
\centering
\begin{tabular}[t]{lccc}
\toprule
  & (1) OLS & (2) OLS & (3) 2SLS\\
\midrule
Full Network MW & 0.0922* & 0.6300*** & 0.8197***\\
 & (0.0498) & (0.1385) & (0.1581)\\
 &  &  & \\
County FE & Yes & Yes & Yes\\
Time FE & Yes & No & No\\
\addlinespace
State $\times$ Time FE & No & Yes & Yes\\
First Stage F & -- & -- & 555.9\\
Observations & 135,744 & 135,700 & 135,700\\
\bottomrule
\multicolumn{4}{l}{\textsuperscript{} Notes: Dependent variable is log employment. Full Network MW}\\
\multicolumn{4}{l}{is SCI-weighted average of log minimum wages excluding only}\\
\multicolumn{4}{l}{own-county. Column (3) instruments Full Network MW with}\\
\multicolumn{4}{l}{Out-of-State Network MW (excludes all same-state}\\
\multicolumn{4}{l}{connections). Standard errors clustered at state level in}\\
\multicolumn{4}{l}{parentheses. *** p<0.01, ** p<0.05, * p<0.1.}\\
\end{tabular}
\end{table}


\subsection{The Modularity Assumption Holds (P1)}

Column (4) introduces the interaction term and delivers the paper's central result. The language gap in Protestant areas is 15.5~pp (95\% CI: [14.0, 17.1]). The religion gap in German-speaking areas is $-8.3$~pp ($p < 0.001$). The interaction is $-0.09$~pp (95\% CI: [$-1.7$, 1.5], $p = 0.91$).

The interaction is precisely zero. The confidence interval rules out economically meaningful interactions in either direction---the maximum sub-additivity consistent with the data is 1.7~pp, and the maximum super-additivity is 1.5~pp. In a setting where the main effects are 15.5~pp and 8.3~pp, an interaction bounded within $\pm$1.7~pp is negligible. Modularity is not merely ``not rejected''---it is precisely confirmed.

To interpret the magnitude: in Protestant areas, the language gap is $15.5$~pp. If the interaction were as negative as the confidence interval allows ($-1.7$~pp), the language gap in Catholic areas would be $15.5 - 1.7 = 13.8$~pp---still 89\% of the Protestant-area gap. Even the worst-case departure from modularity is small.

Column (5) adds canton fixed effects, identifying the language gap exclusively from bilingual cantons. The estimate is 9.3~pp ($p < 0.001$)---lower than the cross-sectional estimate, as expected, because canton fixed effects absorb between-canton institutional differences. That a 9.3~pp gap persists \emph{within} cantons---where municipalities share the same government, tax regime, and school system---confirms that language operates through cultural channels rather than cantonal institutions.

Column (6) adds municipality-level controls (log population and average turnout). The sample drops by four observations (to $N=8{,}723$) due to missing eligible-voter data in four municipality--referendum pairs.\footnote{These four observations have zero reported eligible voters in the BFS records, preventing computation of log population. Results are unchanged if we impute these values.} The interaction remains precisely zero (0.34~pp, $p > 0.5$), with a 95\% CI of [$-1.2$, 1.9]. Controls do not reveal a hidden interaction.

\subsection{The Additivity Test}

\Cref{tab:culture_groups} presents the culture group decomposition directly. German-Protestant municipalities have a mean gender progressivism index of 46.5\%. French-Protestant municipalities are the most progressive at 62.0\%. German-Catholic are the least progressive at 38.4\%. The critical cell is French-Catholic: at 53.7\%, it falls almost exactly on the additive prediction of 53.8\%.

\begin{table}[htbp]
\centering
\caption{Gender Progressivism Index by Culture Group}
\label{tab:culture_groups}
\begin{tabular}{lcccc}
\toprule
 & \multicolumn{2}{c}{German-speaking} & \multicolumn{2}{c}{French-speaking} \\
\cmidrule(lr){2-3} \cmidrule(lr){4-5}
 & Mean & (SD) & Mean & (SD) \\
\midrule
Protestant & 0.465 & (0.074) & 0.62 & (0.059) \\
 & [N=601] & & [N=401] & \\
[4pt]
Catholic & 0.384 & (0.082) & 0.537 & (0.081) \\
 & [N=232] & & [N=229] & \\
\midrule
\multicolumn{5}{l}{\textit{Additivity test:}} \\
Predicted FC (additive) & \multicolumn{4}{c}{0.538} \\
Actual FC & \multicolumn{4}{c}{0.537} \\
Interaction (deviation) & \multicolumn{4}{c}{-0.001} \\
\bottomrule
\end{tabular}
\begin{minipage}{0.9\textwidth}
\footnotesize
\textit{Notes:} Gender progressivism index is the average yes-share across six gender referenda (equal rights 1981, maternity insurance 1999, women's representation 2000, abortion 2002, paternity leave 2020, marriage for all 2021). Confessional status is historically predetermined (16th century Reformation). Under additivity, the French-Catholic mean equals the sum of the language and religion main effects added to the German-Protestant baseline. A positive deviation indicates super-additivity (amplification).
\end{minipage}
\end{table}



The deviation from additivity is $-0.1$ percentage points---essentially zero. The additive model does not just ``approximately'' predict French-Catholic progressivism; it predicts it with remarkable precision. The language gap is 15.5~pp in Protestant areas and 15.3~pp in Catholic areas. The religion gap is 8.1~pp in German-speaking areas and 8.3~pp in French-speaking areas. Both gaps are almost perfectly constant across the other dimension.

The hierarchy is French-Protestant $>$ French-Catholic $>$ German-Protestant $>$ German-Catholic. The 8.3~pp gap between French-Protestant and French-Catholic municipalities---communities sharing the same language---reveals the magnitude of confessional heritage's effect on gender attitudes. But this religious effect is the same whether the community is French- or German-speaking: modularity.

\subsection{The Interaction Averages Zero Across Referenda (P2)}

\Cref{tab:time_gaps} reports the interaction by referendum. The main effects vary \emph{dramatically} across issues. The language gap ranges from 0.9~pp (marriage equality, 2021) to 32.6~pp (maternity insurance, 1999). The religion gap ranges from $-1.9$~pp (maternity insurance, 1999) to $-20.4$~pp (abortion, 2002). These are large, interesting variations that reveal the domain-specific content of cultural transmission.

\begin{table}[htbp]
\centering
\caption{Cultural Gaps by Gender Referendum}
\label{tab:time_gaps}
\begin{tabular}{lccccccc}
\toprule
 & \multicolumn{2}{c}{Language Gap} & \multicolumn{2}{c}{Religion Gap} & \multicolumn{2}{c}{Interaction} & \\
\cmidrule(lr){2-3} \cmidrule(lr){4-5} \cmidrule(lr){6-7}
Referendum & Coef & (SE) & Coef & (SE) & Coef & (SE) & N \\
\midrule
Jun 14, 1981 & 10.7*** & (0.7) & -8.1*** & (0.8) & 3.4** & (1.4) & 1,447 \\
Jun 13, 1999 & 32.6*** & (0.5) & -1.9* & (1.1) & 1.5 & (1.1) & 1,453 \\
Mar 12, 2000 & 5.8*** & (0.3) & -1.9*** & (0.3) & -0.0 & (0.6) & 1,453 \\
Jun 02, 2002 & 12.8*** & (0.9) & -20.4*** & (0.8) & 2.2 & (1.4) & 1,453 \\
Sep 27, 2020 & 26.0*** & (0.5) & -5.9*** & (0.9) & -4.8*** & (1.0) & 1,460 \\
Sep 26, 2021 & 0.9* & (0.4) & -2.8*** & (0.5) & -2.6*** & (0.9) & 1,461 \\
\bottomrule
\end{tabular}
\begin{minipage}{0.95\textwidth}
\footnotesize
\textit{Notes:} Language gap: French minus German yes-share (pp). Religion gap: Catholic minus Protestant yes-share (pp). Interaction: French $\times$ Catholic coefficient (pp). Standard errors in parentheses from OLS with municipality-level data. * $p<0.10$, ** $p<0.05$, *** $p<0.01$.
\end{minipage}
\end{table}


The interaction, by contrast, shows no systematic pattern. The referendum-specific interactions are: $+3.4$~pp (1981), $+1.5$~pp (1999), $\approx 0$~pp (2000), $+2.2$~pp (2002), $-4.8$~pp (2020), and $-2.7$~pp (2021). Some individual referenda show statistically significant departures from zero---the 2020 paternity leave interaction is $-4.8$~pp ($p < 0.01$) and the 1981 equal rights interaction is $+3.4$~pp ($p < 0.05$)---but the sign \emph{switches} across referenda. The interactions fluctuate around zero with no clear gradient tied to doctrine salience. The pooled interaction of $-0.09$~pp reflects this sign-switching: positive deviations on earlier referenda cancel negative ones on later referenda, yielding a precisely estimated null in the panel specification.

\Cref{fig:convergence} visualizes these dynamics. The language and religion gaps vary substantially across referenda, but the interaction hovers consistently near the zero line.

\begin{figure}[H]
    \centering
    \includegraphics[width=0.85\textwidth]{figures/fig4_convergence.pdf}
    \caption{Cultural gaps by referendum: language gap (blue), religion gap (red), and interaction (green). The main effects vary dramatically across issues, while the interaction fluctuates around zero with sign-switching---confirming modularity on average.}
    \label{fig:convergence}
\end{figure}

This pattern is theoretically informative. The fact that the main effects vary by a factor of 30 across referenda while the interaction stays near zero suggests that the channels through which language and religion influence preferences are genuinely separate. Whatever drives the enormous language gap on maternity insurance (32.6~pp) operates through a different mechanism than whatever drives the enormous religion gap on abortion (20.4~pp). These mechanisms do not interact.

\subsection{Domain-Specific Main Effects: A Striking Falsification (P3)}

\Cref{fig:falsification} displays the falsification test, which yields a surprising and important finding. On non-gender referenda, the language effect \emph{reverses}: French-speaking municipalities are 17.1~pp \emph{more conservative} than German-speaking ones ($p < 0.001$). Similarly, the religion effect reverses: Catholic municipalities are 4.5~pp \emph{more progressive} on non-gender issues ($p < 0.001$). The interaction on non-gender issues is 1.0~pp---near zero, as on gender issues.

\begin{figure}[H]
    \centering
    \includegraphics[width=0.85\textwidth]{figures/fig7_falsification.pdf}
    \caption{Falsification: gender referenda (left) vs.\ non-gender referenda (right). The main effects \emph{reverse} across domains---French-speaking municipalities are progressive on gender but conservative on non-gender issues. The interaction is near zero in both domains, confirming modularity.}
    \label{fig:falsification}
\end{figure}

This reversal is more informative than a simple null. It demonstrates that the language dimension captures genuine cultural content---specifically, domain-specific preferences---rather than a generic liberalism or conservatism. French-speaking municipalities are not ``more progressive in general''; they are more progressive on gender issues and more conservative on other policy domains. This pattern is consistent with distinct cultural traditions transmitted through the linguistic community, not with a unidimensional left-right ideological sorting.

For the modularity test, the falsification confirms P3: the interaction is near zero in both the gender and non-gender domains. Modularity holds regardless of the direction of the main effects. Language and religion operate through independent channels whether the topic is gender equality, immigration, defense, or taxation.


%=============================================================================
\section{Robustness} \label{sec:robustness}
%=============================================================================

\subsection{Permutation Inference}

A concern with cluster-robust inference is that religion clusters (Catholic vs.\ Protestant cantons) are few in number, potentially invalidating asymptotic standard errors. We conduct permutation inference following \citet{young2019channeling}: 500 random reassignments of language and religion labels across municipalities. The results powerfully confirm our main findings while highlighting the distinction between the main effects and the interaction.

The observed language coefficient exceeds all 500 permuted values ($p < 0.002$). The language effect is unambiguously real---the probability of observing a gap this large under the null of random assignment is essentially zero.

The interaction tells the opposite story. The observed interaction ($-0.09$~pp) is exceeded in absolute value by 468 of 500 permutations ($p = 0.936$). The interaction is not merely insignificant---it is \emph{indistinguishable from pure noise}. Random reassignment of cultural labels produces interactions of similar or greater magnitude 94\% of the time.

\Cref{tab:permutation} and \Cref{fig:permutation} report the full permutation results.

\begin{table}[htbp]
\centering
\caption{Permutation Inference}
\label{tab:permutation}
\begin{tabular}{lcc}
\toprule
 & Language Effect & Interaction \\
\midrule
Observed coefficient & 0.1293 & -0.0733 \\
Permutation $p$-value & $<0.002$ & $<0.002$ \\
N permutations & 500 & 500 \\
Permutation mean & 0 & 1e-04 \\
Permutation SD & 0.0054 & 0.011 \\
\bottomrule
\end{tabular}
\begin{minipage}{0.85\textwidth}
\footnotesize
\textit{Notes:} Permutation inference based on 500 random reassignments of municipality language and religion labels. Two-sided $p$-values report the fraction of permuted coefficients with absolute value at least as large as the observed coefficient. Reported as $< 1/500 = 0.002$ when no permutation exceeds the observed value.
\end{minipage}
\end{table}



\begin{figure}[H]
    \centering
    \includegraphics[width=0.85\textwidth]{figures/fig5_permutation.pdf}
    \caption{Permutation distributions for the language effect (left) and interaction (right). The observed language coefficient lies far outside the permutation distribution ($p < 0.002$). The observed interaction lies squarely within it ($p = 0.936$).}
    \label{fig:permutation}
\end{figure}

\subsection{Alternative Clustering and Specifications}

\Cref{tab:robustness} explores sensitivity to clustering level, sample restrictions, and weighting. Column (1) reproduces the baseline with municipality-level clustering. Column (2) clusters at the canton level. Column (3) uses two-way clustering by municipality and referendum date. All three yield an interaction of $-0.09$~pp. The point estimate is invariant to clustering; only the standard errors change.

Column (4) adds canton fixed effects, yielding an interaction of $-0.08$~pp. This near-zero estimate in the within-canton specification is important: it confirms that the null interaction is not an artifact of between-canton confounding.

Columns (5)--(6) restrict the sample. Excluding cities (municipalities with $>$50,000 eligible voters) yields an interaction of $-0.07$~pp. Restricting to rural municipalities ($<$10,000 eligible voters) yields an interaction of $0.00$~pp. The null interaction is not driven by urban municipalities.

Column (7) weights by eligible voters, estimating the average voter effect rather than the average municipality effect. The language gap attenuates to 12.4~pp (SE 1.7~pp) and the religion gap increases to $-9.8$~pp (SE 2.0~pp)---reflecting the fact that larger municipalities are more moderate on both dimensions. The interaction is $-1.0$~pp (SE 2.3~pp), with a 95\% CI of [$-5.4$, 3.5]. Even with voter weighting, which substantially changes the main effect estimates, the interaction remains small and insignificant.

\begin{table}[htbp]
\centering
\caption{Robustness Checks}
\label{tab:robustness}
\begin{tabular}{lccc}
\toprule
Specification & ATT & SE & 95\% CI \\
\midrule
Main (Callaway-Sant'Anna) & 0.0051 & 0.0081 & [-0.0107, 0.0209] \\
TWFE (simple) & 0.0108 & 0.0075 & [-0.0039, 0.0254] \\
TWFE (with controls) & 0.0106 & 0.0070 & [-0.0031, 0.0244] \\
Gardner Two-Stage & -0.0033 & 0.0096 & [-0.0221, 0.0155] \\
Excluding Oregon & -0.0001 & 0.0083 & [-0.0163, 0.0162] \\
Placebo: Workers WITH pension & -0.0126 & 0.0140 & [-0.0399, 0.0148] \\
\bottomrule
\end{tabular}
\begin{tablenotes}
\small
\item Note: All specifications use private sector workers ages 25-64. Standard errors clustered at state level.
\end{tablenotes}
\end{table}


\subsection{Inclusive Sample with Mixed Cantons}

Our main analysis excludes five mixed-confession cantons (Aargau, Grisons, St.\ Gallen, Solothurn, Thurgau) whose confessional heritage is ambiguous. As a robustness check, we reclassify these cantons by their pre-1800 majority denomination and re-estimate on the inclusive sample. This yields 12,072 observations from 2,024 municipalities. The results are substantively identical: the language gap is 16.7~pp, the religion gap is $-6.9$~pp, and the interaction is $-1.4$~pp ($p > 0.1$). Including the ambiguous cantons does not reveal a hidden interaction.

\subsection{Coefficient Stability}

\Cref{fig:forest} summarizes the robustness results visually. The language gap ranges from 9.3~pp (within-canton) to 15.5~pp (full sample). The interaction is near zero across all specifications---at most 1.0~pp in absolute value even in the voter-weighted specification (where the point estimate is $-1.0$~pp with a standard error of 2.3~pp). This stability is strong evidence that the null interaction is a genuine feature of the data, not an artifact of any particular specification choice.

\begin{figure}[H]
    \centering
    \includegraphics[width=0.85\textwidth]{figures/fig6_forest.pdf}
    \caption{Forest plot of language gap and interaction estimates across specifications. Point estimates with 95\% confidence intervals. The interaction is near zero across all specifications.}
    \label{fig:forest}
\end{figure}


%=============================================================================
\section{Discussion} \label{sec:discussion}
%=============================================================================

\subsection{Why Modularity Holds}

The null interaction is not a default finding. There are plausible theoretical reasons to expect language and religion to interact: institutional friction, cross-cutting identity prescriptions, and selective cultural transmission all predict sub-additivity. That none of these mechanisms generates a detectable interaction---in a sample with ample power to detect one---is substantively informative.

We propose that modularity holds because language and religion influence gender preferences through genuinely separate channels that have minimal cross-talk. Language shapes preferences primarily through media consumption, intellectual networks, and cultural reference frames. The francophone Swiss media ecosystem---newspapers, television, online content---originates in Geneva, Lausanne, and ultimately Paris, and transmits a set of cultural assumptions about gender that differ from those of the German-speaking media ecosystem rooted in Zurich and linked to Germany. These media and network channels are largely orthogonal to religious institutions.

Religion shapes preferences primarily through institutional participation, community norms, and doctrinal authority. The Catholic Church's positions on gender-related issues---however attenuated in modern Switzerland---operate through parish life, religious education, and communal moral authority. These institutional channels are largely orthogonal to language.

The key insight is that orthogonal channels produce additive effects. When the media channel (language) and the institutional channel (religion) have minimal cross-talk, their effects on preferences simply sum. A French-Catholic municipality receives progressive signals through its media and conservative signals through its institutions; these neither amplify nor dampen each other because they operate through different pathways. This interpretation is consistent with the ``separate spheres'' view of cultural transmission suggested by \citet{fernandez2011does} and \citet{alesina2015culture}.

\subsection{Implications for Cultural Economics}

The confirmation of modularity has an important positive implication: the large body of single-dimension cultural economics research is not biased by unmodeled interactions, at least in the Swiss context. \citet{eugster2011demand}'s estimate of the language gap is correct even though they did not account for religion. \citet{basten2013beyond}'s estimate of the religion effect is correct even though they did not account for language. Each dimension can be studied in isolation without loss.

This is reassuring but also surprising. Switzerland is a best-case setting for modularity: its cultural dimensions are historically predetermined, institutionally stable, and geographically sharp. In settings with more fluid cultural boundaries---where language, religion, and ethnicity are correlated, endogenous, or actively contested---interactions may be more likely. Our finding of modularity in Switzerland should not be extrapolated to settings where the conditions that produce it (sharp, orthogonal, stable cultural boundaries) do not hold.

The broader methodological lesson is that modularity is \emph{testable}, and future research should test it rather than assume it. The $2 \times 2$ factorial design we employ---combining two binary cultural dimensions---is straightforward to implement wherever two cultural boundaries cross. Similar designs could test modularity at the intersection of language and ethnicity in Belgium, religion and caste in India, or race and language in South Africa.

\subsection{The Informative Null and What It Rules Out}

It is worth emphasizing what the null interaction rules out. \citet{crenshaw1989demarginalizing}'s intersectionality framework predicts that outcomes at the intersection of social categories cannot be predicted from the sum of their parts. Our result shows that, for revealed gender preferences in Switzerland, the sum of parts is an excellent predictor. Intersectionality effects---at least in this domain, with these cultural dimensions, measured through voting behavior---are empirically negligible.

This does not invalidate the intersectionality framework, which was developed for different social categories (race and gender in the United States) and different outcomes (legal discrimination, identity formation). But it does establish that intersectionality effects are not universal: there exist important settings where cultural dimensions combine additively.

The null also rules out a ``Catholic ceiling'' on francophone progressivism. If Catholic institutions actively resisted the transmission of progressive gender norms from francophone cultural centers, we would observe sub-additivity. We do not. Whatever conservative influence Catholic heritage exerts on gender attitudes, it is the same magnitude regardless of whether the community is French- or German-speaking. Catholic heritage is not a moderator of the language effect; it is an independent, additive shifter.

\subsection{Domain-Specific Culture: A Silver Lining}

While the interaction is null, the falsification test reveals something striking about the main effects: they are highly domain-specific. French-speaking municipalities are 15.5~pp more progressive on gender issues but 17.1~pp more \emph{conservative} on non-gender issues. This is not a pattern that any unidimensional model of cultural liberalism can explain.

The domain specificity suggests that the R\"{o}stigraben transmits specific cultural content---attitudes toward gender roles, reproductive autonomy, family policy---rather than a generic ideological orientation. The francophone cultural ecosystem appears to produce progressive gender norms \emph{and} conservative attitudes on other policy domains (or, equivalently, the germanophone ecosystem produces conservative gender norms and progressive attitudes on non-gender domains). This is consistent with the cultural content being transmitted through language-specific media and intellectual networks, each of which carries domain-specific rather than ideologically coherent content.

This finding invites future research on the \emph{content} of cultural transmission. Why does the R\"{o}stigraben produce opposing effects on gender versus non-gender policy? What specific cultural content---literary traditions, media framing, intellectual movements---generates this asymmetry? These questions go beyond modularity but are opened up by the same factorial design.

\subsection{Limitations}

Several limitations merit acknowledgment. First, our design is not a formal spatial RDD with a continuous running variable; our OLS estimates may capture residual confounding despite the historical predetermination of both boundaries. The within-canton estimates partially address this concern but cannot fully resolve it.

Second, confessional classification is at the cantonal level, which is coarser than ideal. Within a canton, there may be municipality-level variation in religious practice that our binary indicator does not capture. However, the Reformation-era assignment was indeed at the cantonal level, making this the appropriate unit for the historically predetermined treatment.

Third, our panel of six referenda provides reasonable but limited power for the referendum-specific analysis. While the pooled interaction is precisely estimated (SE = 0.83~pp), the individual-referendum interactions have larger standard errors and should be interpreted with appropriate caution.

Fourth, we cannot observe individual-level voting behavior, so our estimates reflect municipality-level averages. If within-municipality heterogeneity differs systematically across culture groups, our estimates could mask individual-level interactions. However, given that municipality-level averages aggregate thousands of individual votes, this concern is mitigated by the law of large numbers.

Fifth, we exclude five mixed-confession cantons from the main analysis. While robustness checks including these cantons yield similar results, the exclusion reduces sample size and geographic coverage.


%=============================================================================
\section{Conclusion} \label{sec:conclusion}
%=============================================================================

Cultural economics has made remarkable progress by isolating the effect of single cultural dimensions---language, religion, ethnicity---on economic outcomes. This progress rests on an untested assumption: that cultural dimensions combine additively, so each can be studied in isolation. We provide the first direct test.

Using Switzerland's two historically predetermined cultural boundaries and six gender-equality referenda spanning 1981--2021, we find that modularity holds. The language--religion interaction is precisely zero ($-0.09$~pp, permutation $p = 0.94$), with a 95\% confidence interval that rules out interactions larger than 1.7~pp in either direction. An additive model predicts French-Catholic gender progressivism with remarkable accuracy: 53.8\% predicted versus 53.7\% observed.

The null interaction coexists with large, precisely estimated main effects. French-speaking municipalities vote 15.5~pp more progressively on gender issues; Catholic heritage reduces progressivism by 8.3~pp. These effects vary dramatically across referenda but combine additively throughout. A striking falsification reveals that the main effects \emph{reverse} on non-gender issues---French-speaking municipalities are 17.1~pp more conservative---while the interaction remains zero.

The result validates the single-dimension approach that dominates cultural economics: when cultural boundaries are sharp, orthogonal, and historically predetermined, each dimension can be studied in isolation without bias. But the result also invites caution about extrapolation. Switzerland's cultural boundaries are unusually clean. Whether modularity holds at noisier intersections---ethnolinguistic fractionalization in Africa, caste and religion in India, race and language in the United States---remains an open question. The $2 \times 2$ factorial design we employ is portable; we encourage its application elsewhere.

Culture is a system with many dimensions. Those dimensions, it turns out, can sometimes be studied one at a time.


%=============================================================================
\section*{Acknowledgements}
%=============================================================================

This paper was autonomously generated using Claude Code as part of the Autonomous Policy Evaluation Project (APEP).

\noindent\textbf{Project Repository:} \url{https://github.com/SocialCatalystLab/ape-papers}

\noindent\textbf{Contributors:} @SocialCatalystLab

\noindent\textbf{First Contributor:} \url{https://github.com/SocialCatalystLab}

\label{apep_main_text_end}
\newpage
\bibliography{references}

\newpage
\appendix

%=============================================================================
\section{Data Appendix} \label{app:data}
%=============================================================================

\subsection{Data Sources and Access}

\textbf{Referendum data.} Municipality-level referendum results were obtained through the \texttt{swissdd} R package \citep{swissdd2024}. For each national referendum, the dataset includes: municipality BFS number, vote date, eligible voters, valid votes, yes votes, and no votes. We compute the yes-share as the ratio of yes votes to valid votes.

\textbf{Municipality characteristics.} Language classification and canton assignment come from BFS regional statistics \citep{bfs2024}. For bilingual cantons (Fribourg, Bern, Valais), municipality-level language is assigned using known linguistic boundaries within each canton.

\textbf{Confessional classification.} Historical cantonal religion was coded from Reformation-era choices:
\begin{itemize}
    \item \textbf{Catholic:} LU, UR, SZ, OW, NW, ZG, FR, VS, TI, JU, AI (11 cantons)
    \item \textbf{Protestant:} ZH, BE, GL, BS, BL, SH, AR, VD, NE, GE (10 cantons)
    \item \textbf{Mixed (excluded from main analysis):} AG, GR, SG, SO, TG (5 cantons)
\end{itemize}
Mixed cantons experienced contested Reformations that left them with substantial populations of both Catholic and Protestant traditions. They are excluded from the main analysis but included in robustness checks, reclassified by their pre-1800 majority denomination.

\subsection{Gender Referenda Selection}

We identified six gender referenda, matched to \texttt{swissdd} by proposal ID:
\begin{enumerate}
    \item June 14, 1981: Equal rights amendment (ID: 3060)
    \item June 13, 1999: Maternity insurance (ID: 4580)
    \item March 12, 2000: Women's representation (ID: 4610)
    \item June 2, 2002: Abortion access (ID: 4870)
    \item September 27, 2020: Paternity leave (ID: 6340)
    \item September 26, 2021: Marriage for All (ID: 6470)
\end{enumerate}

\subsection{Sample Exclusions}

We exclude municipalities on three grounds:
\begin{enumerate}
    \item \textbf{Language:} Italian-speaking municipalities (Ticino, parts of Grisons) are excluded because the Italian-language cultural ecosystem differs from both French and German, and few Italian-speaking municipalities lie in cantons with clear confessional status.
    \item \textbf{Confession:} Municipalities in five mixed-confession cantons (AG, GR, SG, SO, TG) are excluded from the main analysis. These cantons' confessional heritage is ambiguous, making binary classification unreliable. They are included in the robustness check with the inclusive sample.
    \item \textbf{Missing data:} Municipality--referendum pairs with missing vote data (due to municipal mergers or recording gaps) are dropped.
\end{enumerate}

\subsection{Variable Definitions}

\begin{table}[H]
\centering
\caption{Variable Definitions}
\label{tab:vardef}
\begin{tabular}{lp{10cm}}
\toprule
Variable & Definition \\
\midrule
\texttt{yes\_share} & Proportion of valid votes cast ``yes.'' Range: [0, 1]. \\
\texttt{is\_french} & 1 if municipality's dominant language is French, 0 if German. \\
\texttt{is\_catholic} & 1 if municipality's canton is historically Catholic. \\
\texttt{culture\_4} & Factor: German-Protestant (ref.), German-Catholic, French-Protestant, French-Catholic. \\
\texttt{log\_eligible} & Log of eligible voters (proxy for municipality size). \\
\texttt{avg\_turnout} & Average turnout across referenda (proxy for civic engagement). \\
\bottomrule
\end{tabular}
\end{table}


%=============================================================================
\section{Additional Figures} \label{app:figures}
%=============================================================================

\begin{figure}[H]
    \centering
    \includegraphics[width=0.95\textwidth]{figures/fig1_culture_groups.pdf}
    \caption{Culture group means: average yes-share on gender referenda by language--religion group. Error bars show 95\% confidence intervals. The French-Catholic mean falls almost exactly on the additive prediction (dashed line), confirming modularity.}
    \label{fig:culture_map}
\end{figure}

\begin{figure}[H]
    \centering
    \includegraphics[width=0.85\textwidth]{figures/fig2_distributions.pdf}
    \caption{Distribution of gender progressivism index by culture group. French-Protestant municipalities cluster at the top; German-Catholic at the bottom. The distributions overlap substantially within language groups, consistent with religion having a moderate but independent additive effect.}
    \label{fig:distributions}
\end{figure}

\begin{figure}[H]
    \centering
    \includegraphics[width=0.85\textwidth]{figures/fig3_interaction.pdf}
    \caption{Interaction plot: mean gender progressivism by language and religion. Under additivity, the lines are parallel. The near-perfect parallelism confirms modularity---the language gap is the same in both Protestant and Catholic areas.}
    \label{fig:interaction}
\end{figure}


%=============================================================================
\section{Robustness Appendix} \label{app:robustness}
%=============================================================================

\subsection{Clustering Sensitivity}

The main text reports three clustering levels: municipality, canton, and two-way (municipality and referendum date). With only 21 cantons in the estimation sample (excluding mixed cantons), the effective number of clusters is limited. Canton-level clustering produces larger standard errors but does not change the point estimates. The permutation $p$-value for the interaction ($p = 0.936$) provides the most robust inference, as it is valid regardless of the number of clusters.

\subsection{Extended Falsification}

The falsification exercise in the main text uses non-gender federal referenda. In the extended falsification, we find: language gap $= -0.4$~pp (near zero), religion gap $= -3.7$~pp, and interaction $= +2.7$~pp. All non-gender interaction estimates are small and statistically insignificant, confirming that modularity holds across both gender and non-gender domains.

\subsection{Inclusive Sample Details}

The inclusive sample reclassifies the five mixed-confession cantons (AG, GR, SG, SO, TG) by their pre-1800 majority denomination: AG, SG, SO, and TG are assigned as Protestant (majority at the cantonal level); GR as Protestant (majority of Reformed communities). This yields 2,024 municipalities and 12,072 observations. All main results are robust: language gap $= 16.7$~pp ($p < 0.001$), religion gap $= -6.9$~pp ($p < 0.001$), interaction $= -1.4$~pp ($p > 0.1$).

\subsection{By Referendum Topic}

The six referenda differ substantially in content and in the magnitude of cultural gaps. The 1999 maternity insurance referendum generates the largest language gap (32.6~pp), likely reflecting the confluence of gender and welfare-state preferences. The 2002 abortion referendum produces the largest religion gap ($-20.4$~pp), consistent with Catholic doctrine on reproductive rights. The 2021 marriage equality referendum shows the smallest language and religion gaps, suggesting broad cultural convergence on this issue by 2021. Despite these large variations in main effects, the interaction is consistently near zero across all six referenda, with no systematic pattern linked to doctrine salience or issue content.

\subsection{Power Analysis}

Our main specification has 8,727 observations with municipality-level clustering (1,463 clusters). The standard error on the interaction term is 0.83~pp. With this precision, we can detect interactions as small as $\pm$1.6~pp at the 5\% significance level (two-sided). This means we have adequate power to detect interactions that would be economically meaningful: a 1.6~pp interaction would represent roughly 10\% of the language gap or 19\% of the religion gap. Our null finding is not a power problem.


\end{document}
