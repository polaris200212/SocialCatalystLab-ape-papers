\begin{table}[!h]
\centering
\caption{\label{tab:tab:summary}Summary Statistics by Culture Group}
\centering
\begin{threeparttable}
\begin{tabular}[t]{lrrrrrr}
\toprule
\multicolumn{1}{c}{ } & \multicolumn{1}{c}{ } & \multicolumn{2}{c}{Gender Progressivism} & \multicolumn{2}{c}{Participation} & \multicolumn{1}{c}{ } \\
\cmidrule(l{3pt}r{3pt}){3-4} \cmidrule(l{3pt}r{3pt}){5-6}
Culture Group & Municipalities & Gender Index & SD & Turnout (\textbackslash{}\%) & Eligible Voters & Referenda\\
\midrule
French-Catholic & 284 & 0.503 & 0.104 & 42.5 & 1376 & 6\\
French-Protestant & 372 & 0.626 & 0.052 & 47.1 & 1979 & 6\\
German-Catholic & 359 & 0.411 & 0.082 & 49.8 & 2719 & 6\\
German-Protestant & 711 & 0.461 & 0.078 & 47.2 & 2958 & 6\\
\bottomrule
\end{tabular}
\begin{tablenotes}
\item \textit{Note: } 
\item Gender progressivism index is the average yes-share across six gender referenda: equal rights (1981), maternity insurance (1999), women's representation (2000), abortion (2002), paternity leave (2020), marriage for all (2021). Culture groups defined by municipal language (BFS) and historically predetermined confessional status (16th century Reformation). Italian-speaking and mixed-religion municipalities excluded.
\end{tablenotes}
\end{threeparttable}
\end{table}
