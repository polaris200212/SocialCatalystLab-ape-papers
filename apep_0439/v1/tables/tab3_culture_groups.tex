\begin{table}[htbp]
\centering
\caption{Gender Progressivism Index by Culture Group}
\label{tab:culture_groups}
\begin{tabular}{lcccc}
\toprule
 & \multicolumn{2}{c}{German-speaking} & \multicolumn{2}{c}{French-speaking} \\
\cmidrule(lr){2-3} \cmidrule(lr){4-5}
 & Mean & (SD) & Mean & (SD) \\
\midrule
Protestant & 0.461 & (0.078) & 0.626 & (0.052) \\
 & [N=711] & & [N=372] & \\
[4pt]
Catholic & 0.411 & (0.082) & 0.503 & (0.104) \\
 & [N=359] & & [N=284] & \\
\midrule
\multicolumn{5}{l}{\textit{Additivity test:}} \\
Predicted FC (additive) & \multicolumn{4}{c}{0.576} \\
Actual FC & \multicolumn{4}{c}{0.503} \\
Interaction (deviation) & \multicolumn{4}{c}{-0.074} \\
\bottomrule
\end{tabular}
\begin{minipage}{0.9\textwidth}
\footnotesize
\textit{Notes:} Gender progressivism index is the average yes-share across six gender referenda (equal rights 1981, maternity insurance 1999, women's representation 2000, abortion 2002, paternity leave 2020, marriage for all 2021). Confessional status is historically predetermined (16th century Reformation). Under additivity, the French-Catholic mean equals the sum of the language and religion main effects added to the German-Protestant baseline. A positive deviation indicates super-additivity (amplification).
\end{minipage}
\end{table}

