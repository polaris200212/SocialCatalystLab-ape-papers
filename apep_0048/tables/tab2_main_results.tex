
\begin{table}[htbp]
\centering
\caption{Effect of Women's Suffrage on Female Labor Force Participation}
\label{tab:main_results}
\begin{threeparttable}
\begin{tabular}{lccc}
\toprule
& (1) & (2) & (3) \\
& TWFE & Triple-Diff & + Controls \\
\midrule
Post-Suffrage & 0.023* & 0.023** & 0.022** \\
 & (0.012) & (0.011) & (0.010) \\
Urban & & 0.001 & 0.001 \\
 & & (0.001) & (0.001) \\
Post $\times$ Urban & & 0.001 & 0.001 \\
 & & (0.001) & (0.001) \\
\midrule
State FE & Yes & Yes & Yes \\
Year FE & Yes & Yes & Yes \\
Individual Controls & No & No & Yes \\
\midrule
Observations & 6,655,507 & 6,655,507 & 6,655,507 \\
\bottomrule
\end{tabular}
\begin{tablenotes}
\small
\item \textit{Notes:} Dependent variable is an indicator for labor force participation (having an occupation, OCC1950 codes 1-979). Sample includes women ages 18-64, 1880-1920 (10\% sample of full-count census). Column (1) presents basic two-way fixed effects. Column (2) adds the triple-difference interaction with urban residence (imputed from state-year urbanization rates). Column (3) adds controls for age, age squared, and race. Standard errors clustered at state level in parentheses. We do not report Sun-Abraham estimator results because the coarse decennial timing (only 4 census years) provides insufficient variation to identify cohort-specific effects reliably. *** p$<$0.01, ** p$<$0.05, * p$<$0.1.
\end{tablenotes}
\end{threeparttable}
\end{table}

