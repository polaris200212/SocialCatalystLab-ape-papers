
\begin{table}[htbp]
\centering
\caption{Summary Statistics by Treatment Status and Urban/Rural Residence}
\label{tab:summary}
\begin{threeparttable}
\begin{tabular}{lcccc}
\toprule
& \multicolumn{2}{c}{Control States} & \multicolumn{2}{c}{Treated States} \\
\cmidrule(lr){2-3} \cmidrule(lr){4-5}
& Rural & Urban & Rural & Urban \\
\midrule
N (observations) & 3,266,556 & 1,935,124 & 674,958 & 778,869 \\
 & & & & \\
Labor Force Participation & 23.2\% & 24.3\% & 19.9\% & 24.5\% \\
 & (0.42) & (0.43) & (0.40) & (0.43) \\
Age & 34.6 & 35.4 & 35.3 & 35.7 \\
 & (12.4) & (12.5) & (12.4) & (12.4) \\
White (\%) & 82.4\% & 91.5\% & 96.9\% & 97.8\% \\
 & (0.38) & (0.28) & (0.17) & (0.15) \\
\bottomrule
\end{tabular}
\begin{tablenotes}
\small
\item \textit{Notes:} Sample consists of women ages 18-64 from IPUMS USA full-count census data, 1880-1920 (10\% random sample). Treated states adopted women's suffrage before the 19th Amendment (1920). Standard deviations in parentheses. Labor force participation is defined as having an occupation (OCC1950 codes 1-979). Observations pooled across all census years.
\end{tablenotes}
\end{threeparttable}
\end{table}

