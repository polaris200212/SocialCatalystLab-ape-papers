\section{Historical Background}
\label{sec:background}

The struggle for women's suffrage in the United States spanned more than seven decades, from the Seneca Falls Convention of 1848 to the ratification of the Nineteenth Amendment in 1920. This prolonged campaign unfolded against a backdrop of profound economic transformation---industrialization, urbanization, and the emergence of a modern service economy---that reshaped women's relationship to paid labor in ways that varied dramatically between cities and the countryside. Understanding the historical context of suffrage adoption and the structure of Progressive Era labor markets is essential for interpreting the heterogeneous effects we document in this paper. This section provides that context, tracing the geographic and temporal patterns of enfranchisement, describing the divergent labor market conditions facing urban and rural women, and developing a theoretical framework for why the effects of suffrage on female labor force participation might differ by location.

\subsection{The Path to Women's Suffrage}
\label{subsec:suffrage_history}

The movement for women's voting rights in America began in earnest at the Seneca Falls Convention of 1848, where Elizabeth Cady Stanton introduced a resolution declaring that ``it is the duty of the women of this country to secure to themselves their sacred right to the elective franchise'' \citep{flexner1975}. Yet more than two decades would pass before any American jurisdiction granted women the right to vote. The path from Seneca Falls to the Nineteenth Amendment was neither straight nor swift; it proceeded through a patchwork of state-level victories that created the variation in treatment timing essential to our empirical strategy.

Wyoming Territory became the first jurisdiction to enfranchise women in 1869, a decision motivated partly by a desire to attract female settlers to the sparsely populated frontier and partly by genuine commitment to equal rights among territorial legislators \citep{keyssar2000}. Utah Territory followed in 1870, though Congress revoked Utah women's suffrage in 1887 as part of federal efforts to suppress polygamy; the franchise was restored when Utah achieved statehood in 1896. These early Western adoptions established a pattern that would persist throughout the suffrage campaign: the frontier West proved far more receptive to women's voting rights than the established states of the East and South.

The next wave of suffrage expansion came in the 1890s, when Colorado (1893) and Idaho (1896) joined Wyoming and Utah as full-suffrage states. A long period of stasis followed---suffrage referenda failed repeatedly in state after state during the early 1900s---before a renewed surge of victories beginning in 1910. Washington State adopted women's suffrage in 1910, followed by California in 1911 and three states (Oregon, Kansas, and Arizona) in 1912. Montana and Nevada enfranchised women in 1914, New York in 1917, and Michigan, Oklahoma, and South Dakota in 1918. By the time the Nineteenth Amendment was ratified in August 1920, thirteen states had already granted women full voting rights in state elections, while the remaining thirty-five states---concentrated in the East, Midwest, and especially the South---had resisted until federal action forced their hand \citep{moehling2020}.

The geographic pattern of suffrage adoption reflected deeper regional differences in political culture, economic structure, and gender ideology. Western states adopted suffrage earliest and most readily, a pattern scholars have attributed to several factors: the relative scarcity of women on the frontier, which elevated their value and status; the weaker grip of traditional gender hierarchies in newly settled communities; the alliance between suffragists and Progressive reformers who were especially influential in Western politics; and the practical experience Western women had gained in homesteading, ranching, and community building \citep{teele2018}. Eastern states, with their established political machines and entrenched interests, proved more resistant. Southern states were most opposed of all, fearing that women's suffrage would complicate the elaborate legal machinery constructed to disenfranchise Black voters after Reconstruction \citep{keyssar2000}.

Table \ref{tab:suffrage_dates} presents the complete chronology of state-level suffrage adoption. For our empirical analysis, we focus on states that adopted between 1893 and 1918---the period for which full-count census data are available both before and after treatment. Wyoming and Utah are excluded from our primary analysis because no pre-treatment census data exist for these earliest adopters; the first available census (1880) postdates Wyoming's adoption, and Utah's 1870 adoption left only a single pre-treatment observation before the 1887 revocation complicated the treatment definition. The remaining eleven treatment states span a range of adoption years, providing the variation in treatment timing that enables our staggered difference-in-differences design.

The timing of suffrage adoption was not random, and the determinants of early versus late adoption have implications for our identification strategy. States that adopted suffrage earlier tended to be more rural, more Western, and have smaller populations than later adopters \citep{lott1999}. Later adopters like New York (1917) and Michigan (1918) were more urbanized and industrialized, with larger female labor forces working in manufacturing and services. This correlation between adoption timing and state characteristics does not threaten identification---our design compares changes over time within states, differenced against changes in never-treated states---but it does suggest caution in extrapolating effects from early adopters to the broader population. We address this concern through robustness checks that restrict attention to late-adopting states and through event-study analyses that examine whether effects vary systematically with time since treatment.

The suffrage movement itself was deeply intertwined with the broader Progressive reform agenda that animated American politics between 1900 and 1920. Suffragists forged alliances with temperance advocates, labor organizers, settlement house workers, and public health reformers, all of whom saw women's enfranchisement as instrumental to their respective causes \citep{skocpol1992}. These coalitions were especially effective in urban areas, where women's clubs, labor unions, and reform organizations provided the organizational infrastructure for sustained political mobilization. The National American Woman Suffrage Association, the dominant suffrage organization by 1900, pursued a state-by-state strategy that concentrated resources on winnable campaigns while building momentum for an eventual federal amendment. This strategy produced the pattern of staggered adoption we exploit: a gradual expansion of the suffrage frontier from West to East, punctuated by high-profile victories in large states like California (1911) and New York (1917) that shifted national momentum toward final passage of the Nineteenth Amendment.

\subsection{Urban and Rural Labor Markets in the Progressive Era}
\label{subsec:labor_markets}

The period from 1880 to 1920 witnessed a fundamental transformation of American economic life, driven by industrialization, urbanization, and the rise of large-scale enterprise. At the beginning of this period, the United States remained predominantly agricultural; by its end, the majority of Americans lived in urban areas and worked in manufacturing, commerce, or services \citep{haines2000}. This structural transformation created two starkly different labor market environments for women---one urban, one rural---that would shape how the political change of enfranchisement translated into economic outcomes.

Urban labor markets offered women an expanding array of employment opportunities during the Progressive Era. The manufacturing sector, though dominated by male workers in heavy industry, provided substantial employment for women in textile mills, garment factories, food processing plants, and light manufacturing \citep{kessler1982}. More dramatic was the growth of employment in retail trade and clerical work, sectors that would become increasingly feminized over the twentieth century. Department stores, which emerged as major employers in American cities during this period, hired young women as salesgirls, cashiers, and office workers. The expansion of business administration created demand for typists, stenographers, filing clerks, and telephone operators---occupations that were explicitly marketed to young women as respectable alternatives to factory work or domestic service \citep{goldin1990}. By 1920, clerical work had become the second-largest occupational category for urban women after domestic service, representing a fundamental shift in the structure of female employment.

The working conditions facing urban women varied enormously across industries and occupations. Factory work often meant long hours, dangerous machinery, poor ventilation, and wages that barely covered subsistence. The Triangle Shirtwaist Factory fire of 1911, which killed 146 workers---mostly young immigrant women---became a galvanizing symbol of the hazards facing industrial workers and spurred demands for workplace safety regulation. At the other end of the spectrum, clerical positions offered shorter hours, cleaner working environments, and the social status associated with ``white-collar'' employment, though wages were often no higher than in manufacturing \citep{kessler1982}. What united these diverse urban occupations was their visibility: women who worked for wages did so in settings that were observable, regulable, and subject to the reforms that Progressive activists demanded.

The formalization of urban employment relationships made women's labor a target for state intervention. Beginning in the 1890s and accelerating after 1900, state legislatures enacted an array of ``protective'' legislation governing the terms on which women could be employed. Maximum hours laws limited the length of the working day for women in manufacturing, retail, and other covered industries; by 1920, forty-three states had enacted such restrictions \citep{goldin1988}. Minimum wage laws for women, first adopted by Massachusetts in 1912, spread to fifteen states by 1920. Other regulations prohibited night work for women, required rest periods and meal breaks, mandated seating for saleswomen, and established sanitary standards for workplaces employing women. The constitutionality of sex-specific protective legislation was affirmed in \textit{Muller v. Oregon} (1908), where the Supreme Court accepted the argument that women's physical differences and maternal responsibilities justified special state protection even when equivalent regulations for men would be struck down as unconstitutional interference with freedom of contract.

Rural labor markets operated on entirely different principles. The majority of rural women lived on farms, where they worked within household economies that blurred the distinctions between production and consumption, paid labor and unpaid household work, that structured urban employment. Farm women's productive contributions were substantial: they raised poultry and tended dairy cows, maintained vegetable gardens and preserved food for winter, made butter and cheese for home consumption or local sale, and often managed the farm's accounts \citep{haines2000}. Yet this work was largely invisible to the formal economy and to the census enumerators who would classify such women as having no occupation or as ``keeping house.'' The undercounting of women's labor was especially severe in agricultural areas, where the boundary between ``gainful employment'' and ``household duties'' was genuinely ambiguous and where enumerator practice varied widely \citep{folbre1991}.

The organization of agricultural labor meant that protective legislation---the primary channel through which suffrage might affect urban women's working conditions---had little relevance for farm women. Maximum hours laws could not regulate work that had no employer and no time clock. Minimum wage legislation could not set floors for labor compensated through shared household consumption rather than individual wages. Workplace safety standards could not reach production that occurred in the home, the barn, and the field. Even where rural women did engage in wage labor---as seasonal workers in canning factories, for instance, or as domestic servants in nearby towns---the geographic dispersion and small scale of rural employment made regulation difficult and enforcement nearly impossible.

The demographic composition of the female labor force also differed markedly between urban and rural areas. Urban women who worked for wages were disproportionately young and unmarried; labor force participation rates for married women remained low in cities until well into the twentieth century, reflecting both social norms against married women's employment and practical constraints including the demands of household management without modern conveniences \citep{goldin1990}. Young women typically entered the urban workforce between school-leaving age and marriage, withdrawing from paid employment when they wed. This pattern created a strong life-cycle dimension to urban female labor supply that the suffrage movement might have affected by changing the age at marriage, the probability of marriage, or the expectations of young women about their future roles.

Rural women's labor force participation followed different patterns. Married women on farms were almost universally engaged in productive work, but this work was classified as unpaid family labor rather than labor force participation. Young unmarried women in rural areas faced limited local opportunities for wage employment; those who wished to work for wages often migrated to nearby towns or cities, joining the great rural-to-urban migration that transformed American demographics during this period \citep{costa2000}. The decision to migrate was itself a labor market choice that suffrage might have influenced, though one that is difficult to disentangle from the myriad other factors drawing young people to cities during this era of rapid urbanization.

These structural differences in urban and rural labor markets suggest that any effects of women's suffrage on female employment might manifest quite differently across locations. The policy channel---suffrage leading to protective legislation leading to improved working conditions---had clear purchase in urban labor markets characterized by formal employment, regulatory oversight, and concentrated advocacy efforts, but limited relevance in rural areas where work was informal, dispersed, and embedded in household production. Understanding these differences is essential for interpreting the patterns we document in subsequent sections.

\subsection{Theoretical Framework: Why Suffrage Effects Might Differ by Location}
\label{subsec:theory}

Why might the extension of voting rights to women produce different effects on female labor force participation in urban versus rural areas? We consider three channels through which suffrage could affect women's labor market outcomes, each with distinct implications for urban-rural heterogeneity.

\subsubsection{The Policy Channel: Protective Legislation and Workplace Regulation}

The most direct mechanism connecting suffrage to labor market outcomes operates through changes in public policy. Women voters and women's organizations demonstrated distinct policy preferences that, once enfranchised, could influence the political equilibrium. Historical evidence documents that women voters were more likely than men to support temperance legislation, public health spending, educational investment, and labor market regulations \citep{lott1999, miller2008}. Women's organizations---including the National Consumers' League, the Women's Trade Union League, and the General Federation of Women's Clubs---lobbied actively for protective labor legislation before and after suffrage, but their political leverage increased substantially once women could vote \citep{skocpol1992}.

The policy channel predicts that suffrage effects on female labor force participation should be concentrated in labor markets where regulatory interventions could meaningfully alter working conditions. Urban labor markets meet this criterion: women working in factories, department stores, and offices were subject to state regulation, and improvements in their working conditions---shorter hours, higher wages, safer workplaces---could make formal employment more attractive relative to home production or non-participation. If suffrage accelerated the adoption and enforcement of protective legislation, we would expect to see increased female labor force participation in regulated industries and, by extension, in urban areas where such industries were concentrated.

The policy channel also predicts that effects should emerge with a lag, as the political process of translating votes into legislation and legislation into enforcement takes time. \citet{miller2008} documents that the effects of suffrage on public health spending and child mortality materialized over several years following adoption, consistent with a policy mechanism that operated through gradual institutional change rather than immediate behavioral response. Our event-study analysis can shed light on this timing, examining whether effects on labor force participation appear immediately upon suffrage adoption or emerge more gradually in subsequent years.

For rural women, the policy channel predicts weak or null effects. Protective labor legislation could not reach work that was informal, household-based, and outside the regulatory reach of state government. Even policies that did affect rural areas---public health spending, educational investment, agricultural extension services---would operate through channels other than women's own labor force participation. Rural women might benefit from suffrage-induced policy changes in many ways, but changes in their own employment patterns would not be among the primary margins of adjustment.

\subsubsection{The Norms Channel: Voting as a Signal of Citizenship}

A second channel operates through social norms about appropriate gender roles. The right to vote was not merely an instrumental political tool; it carried profound symbolic significance as a marker of full citizenship and public participation \citep{keyssar2000}. To the extent that opposition to women's labor force participation reflected ideological commitments about women's proper sphere---the doctrine of ``separate spheres'' that confined women to home and family while reserving public and economic life for men---the extension of voting rights might have disrupted these ideological foundations. If women could be trusted to exercise the franchise responsibly, perhaps they could also be trusted to participate in the workforce, run businesses, or pursue professional careers.

The norms channel predicts a different pattern of urban-rural heterogeneity than the policy channel. Traditional gender ideologies were, if anything, more deeply entrenched in rural areas than in cities. Farm families depended on clear divisions of labor within the household, and rural communities often maintained conservative social mores that resisted changing gender roles. The symbolic meaning of women's suffrage might therefore have been especially transgressive---and potentially transformative---in rural contexts where traditional gender hierarchies were most firmly established. If the norms channel dominated, we might expect to see suffrage effects that were as strong or stronger in rural areas than in urban ones.

The norms channel also has different implications for timing. Changes in social norms are notoriously difficult to date precisely, but one might expect normative effects to materialize relatively quickly as the symbolic meaning of suffrage diffused through communities. The act of voting itself---women lining up at polling places, casting ballots, participating in the rituals of democratic citizenship---could shift perceptions of women's public role. Alternatively, normative change might operate through generational replacement, with effects concentrated among younger women who came of age in a post-suffrage environment where female political participation was normalized.

\subsubsection{Labor Market Structure: Wage Work versus Family Production}

A third consideration involves the structure of labor markets themselves, independent of policy or norms. Urban and rural labor markets offered fundamentally different kinds of economic opportunities for women, and suffrage might have affected women's ability or willingness to access those opportunities differentially.

In urban labor markets, entry into wage employment required overcoming various barriers: acquiring relevant skills, obtaining information about job opportunities, navigating hiring processes, and managing the practical logistics of combining work with household responsibilities. To the extent that suffrage reduced discrimination against women by employers, expanded women's networks and information channels, or increased women's bargaining power within households, it might have lowered these barriers to urban employment. The concentration of female workers in urban industries also created possibilities for collective action---labor organizing, union membership, workplace advocacy---that could improve conditions for all women workers. Suffrage might have empowered these collective efforts by giving women workers a political voice that employers and legislators could not ignore.

In rural labor markets, the relevant margin was not entry into wage employment but the allocation of effort within household production. Farm women faced a portfolio of tasks---some directly productive (dairy, poultry, gardening), some household maintenance (cooking, cleaning, childcare), some market-oriented (selling butter and eggs), and some purely domestic. Suffrage might have affected how women allocated time across these activities, but such shifts would not show up in labor force participation as measured by the census. Indeed, the census categories themselves were poorly suited to capture the economic contribution of farm women, whose work defied easy classification into ``employed'' versus ``not in labor force.''

This structural difference suggests that even if suffrage had equivalent effects on women's economic empowerment across locations, these effects would manifest in measurable labor force participation only in urban areas where wage employment provided a clear, observable indicator of economic activity. The null rural effects we document might therefore reflect measurement limitations as much as true differences in suffrage's impact. We return to this issue in our discussion of results, considering whether alternative measures or channels might reveal suffrage effects on rural women that our primary analysis cannot detect.

\subsection{Prior Evidence on the Economic Effects of Women's Suffrage}
\label{subsec:prior_evidence}

The economic literature on women's suffrage has established that enfranchisement produced meaningful changes in public policy, government spending, and social outcomes, though direct evidence on labor market effects remains limited. Our analysis builds on and extends this literature by examining an outcome---female labor force participation---that has received relatively little attention in previous work.

The foundational contribution is \citet{lott1999}, who documented that women's suffrage was associated with substantial increases in state government expenditure and revenue. Using a difference-in-differences framework that exploited variation in the timing of suffrage adoption across states, Lott and Kenny found that per capita state government expenditure increased by 28 percent following suffrage, with corresponding increases in tax revenue. They interpreted these findings through the lens of median voter theory: women's preferences for government spending differed systematically from men's, and enfranchisement shifted the political equilibrium toward larger government. The expenditure effects were concentrated in social spending categories---education, public health, welfare---where women's preferences were thought to diverge most sharply from men's.

\citet{miller2008} extended this analysis to examine the effects of suffrage on public health outcomes. Using a similar difference-in-differences design with individual-level data on child mortality, Miller found that suffrage led to immediate and substantial increases in local public health spending, which in turn produced large reductions in infant and child mortality. The effects were concentrated among children under age nine and were driven primarily by declines in infectious disease mortality---outcomes directly connected to public health interventions like sanitation, water purification, and milk safety programs. Miller's analysis demonstrated that suffrage effects operated through specific policy channels rather than through diffuse changes in economic conditions or social organization.

Subsequent research has documented suffrage effects on other policy domains. \citet{aidt2015} examined women's suffrage in Western Europe and found effects on social spending similar to those documented for the United States. \citet{bertocchi2011} developed a theoretical model of how female enfranchisement affects the size and composition of the welfare state, with predictions that align with observed patterns of post-suffrage policy change. \citet{klassen2023} examined the effects of suffrage on political participation itself, finding that enfranchisement increased women's engagement with the political process beyond the simple act of voting.

Research on suffrage and education has produced mixed findings. \citet{carruthers2014} examined the effects of suffrage on schooling provision, finding evidence of increased educational investment following enfranchisement. However, the effects appear to have been mediated by racial politics, with suffrage producing different outcomes in states with different racial compositions. \citet{naidu2012} examined suffrage in the context of the post-bellum South, where the interaction of gender and race created a particularly complex political environment. These studies suggest that suffrage effects were context-dependent, varying with the political and demographic characteristics of adopting states.

Direct evidence on suffrage effects on female labor force participation is scarce. The existing literature has focused primarily on policy outcomes---what governments did after suffrage---rather than on behavioral outcomes---what women themselves did. Our analysis fills this gap by examining whether the political empowerment of women translated into changes in women's own economic activity. The evidence from the policy literature provides theoretical motivation for expecting labor market effects: if suffrage changed protective labor legislation, working conditions, and the political environment facing women workers, these changes should have affected women's labor supply decisions. The urban-rural heterogeneity we document helps distinguish between mechanisms, as different channels predict different patterns of geographic variation.

The labor history literature provides complementary perspective on how women's economic circumstances evolved during this period. \citet{goldin1990} documented the long-run evolution of female labor force participation, noting that participation rates increased substantially between 1880 and 1920 even as married women remained largely out of the formal workforce. \citet{kessler1982} traced the expansion of women's work opportunities in manufacturing, retail, and clerical sectors, emphasizing both the opportunities and exploitation that characterized Progressive Era labor markets. \citet{costa2000} examined the shift from domestic service to clerical and sales occupations, documenting a fundamental restructuring of the female occupational distribution. These accounts provide essential context for interpreting our findings: women's labor force participation was changing for many reasons during this period, and identifying the causal contribution of suffrage requires careful attention to identification strategy and alternative explanations.

The methodological advances of the past decade have also reshaped how researchers approach questions about suffrage. Early studies using two-way fixed effects estimators may have produced biased estimates if treatment effects varied across states or over time since treatment---a concern that the recent econometrics literature has shown to be empirically relevant in many settings \citep{goodman2021}. Our analysis implements modern staggered difference-in-differences methods that address these concerns, providing estimates that are robust to heterogeneous treatment effects and that can be interpreted as appropriately weighted averages of causal effects across treated units. The application of these methods to the suffrage question represents a methodological contribution in addition to our substantive findings on urban-rural heterogeneity.

In summary, the prior literature establishes that women's suffrage produced meaningful changes in government policy and social outcomes, with effects operating through political channels that gave newly enfranchised women voice in the democratic process. Our contribution is to examine whether these political changes translated into changes in women's own labor market behavior, and to document that any such effects were concentrated in urban areas where the policy channel had greatest relevance. The historical background developed in this section---the geographic pattern of suffrage adoption, the structure of Progressive Era labor markets, and the theoretical channels linking political enfranchisement to economic outcomes---provides the foundation for the empirical analysis that follows.
