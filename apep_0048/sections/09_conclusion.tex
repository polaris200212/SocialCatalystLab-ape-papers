\section{Conclusion}
\label{sec:conclusion}

The extension of voting rights to women represented one of the most consequential expansions of political participation in American history. Between 1869 and 1920, the suffrage movement transformed half the adult population from political outsiders to full citizens with the power to shape electoral outcomes. This paper has asked whether that political transformation translated into economic change, specifically whether women's suffrage affected female labor force participation, and whether any such effects differed between urban and rural areas.

Our findings reveal a surprising and robust pattern: women's suffrage increased female labor force participation by approximately 2.8 percentage points in rural areas---a statistically significant effect---while producing a smaller and statistically insignificant effect of 1.5 percentage points in urban areas. This rural dominance pattern persists across multiple econometric approaches, including the modern staggered difference-in-differences estimators of \citet{callaway2021} and \citet{sun2021} that address concerns about heterogeneous treatment effects, and it survives sensitivity analyses using alternative estimators, urban classification thresholds, and sample restrictions. The concentration of effects in rural labor markets challenges the prevailing policy channel hypothesis, which predicted urban effects would dominate because protective labor legislation operated primarily in formal wage labor markets concentrated in cities.

This paper makes three contributions to the literature. First, we provide new evidence on the labor market effects of democratic inclusion. While prior research has established that women's suffrage changed government policy---increasing public expenditure, expanding public health programs, and shifting the composition of state budgets \citep{lott1999, miller2008}---the effects of suffrage on women's own economic behavior have received less attention. Our finding that suffrage increased female labor force participation, with an overall effect of approximately 2.3 percentage points representing roughly 10 percent of baseline participation, demonstrates that political empowerment translated into measurable changes in women's labor market outcomes.

Second, we document surprising heterogeneity in the effects of political inclusion by urban-rural residence. This heterogeneity is substantive evidence that challenges conventional accounts of mechanism. The concentration of effects in rural areas is inconsistent with a policy channel that operated primarily through protective labor legislation and workplace regulation---interventions that had scope for impact only in formal wage labor markets concentrated in cities. Instead, our findings suggest that suffrage may have affected women's labor market outcomes through alternative channels: measurement changes that shifted how census enumerators and women themselves understood farm women's economic contributions; local programs and county-level government services responsive to newly enfranchised women voters; or compositional effects from selective migration. The mechanisms through which political empowerment translates into economic opportunity appear more complex than the protective legislation channel that has dominated prior scholarship.

Third, we demonstrate the application of modern staggered difference-in-differences methods to an important historical question. The suffrage literature has largely relied on earlier econometric approaches now known to produce potentially biased estimates when treatment effects vary across units or over time. By implementing the estimators of \citet{callaway2021} and \citet{sun2021}, along with the sensitivity analyses of \citet{rambachan2023}, we provide estimates robust to heterogeneous treatment effects and offer researchers a template for applying these methods to other historical questions. The consistency of our findings across estimators provides confidence that the urban-rural divergence we document reflects genuine patterns in the data rather than artifacts of any particular econometric approach.

The implications of our findings extend beyond the specific historical episode we study. The relationship between political representation and economic outcomes remains a central concern of political economy, and understanding the channels through which democratic inclusion affects material welfare has relevance for contemporary debates about political participation and economic inequality. Our results suggest that the effects of political empowerment on labor market outcomes operated differently than existing theory predicts. The policy channel operating through protective legislation---with its emphasis on formal wage employment and state regulation---appears less central than previously assumed. Instead, political voice may have translated into economic opportunity through more decentralized channels: local programs, county-level government, measurement changes, and compositional effects. This reframing has implications for how we understand the political economy of democratic inclusion.

Our findings have implications for understanding how political institutions might affect women's economic opportunities in contemporary settings. Many economies feature labor markets with substantial informal sectors where women's work, like that of Progressive Era farm women, is embedded in household production and difficult to measure through standard surveys. Our surprising finding that effects were larger in rural areas suggests that political empowerment may affect women's economic outcomes through channels other than formal labor market regulation. The measurement channel we propose---where political recognition of women's capacity shifted how their economic contributions were perceived and reported---may be relevant wherever women's work is ambiguously classified. The local programs channel---where women's political voice influenced county-level services and extension programs---may be relevant wherever local government provides services of direct relevance to women's economic activities.

Several directions for future research emerge from our analysis. Most immediately, researchers might examine whether the urban-rural divergence we document extends to other outcomes: did suffrage produce heterogeneous effects on women's educational attainment, marriage patterns, or fertility decisions? The mechanisms we propose---the policy channel operating through protective legislation, the norms channel operating through shifting gender ideologies---would predict different patterns for these outcomes, and examining them would help distinguish between explanations. Researchers might also examine within-urban variation, asking whether suffrage effects differed by industry, occupation, or neighborhood characteristics in ways that shed further light on mechanism.

More ambitiously, future research might examine the intergenerational effects of suffrage. If enfranchisement changed the economic opportunities available to women in the early twentieth century, these changes may have affected the human capital investments, aspirations, and labor market trajectories of subsequent generations. The daughters of women who entered the labor force after suffrage may have grown up with different expectations about their own economic roles; the sons of these women may have developed different attitudes about gender and work. Tracing these intergenerational dynamics could illuminate how political change echoes through family structures over time.

The women's suffrage movement fundamentally reshaped American political life. Our findings suggest it also reshaped women's economic lives, though in surprising ways that challenge conventional accounts. We expected urban effects to dominate because protective labor legislation operated primarily in formal wage labor markets concentrated in cities. Instead, we find larger effects in rural areas, where women's work was embedded in household production ostensibly beyond the reach of state regulation. This unexpected pattern speaks to a broader truth about the relationship between political institutions and economic outcomes: the mechanisms through which political empowerment affects material welfare are more complex than simple models suggest. The visible policy changes that dominate historical accounts and scholarly attention may not be the channels through which democratic inclusion most powerfully affects people's lives. Understanding this complexity is essential for drawing lessons from history that inform contemporary efforts to expand political participation and promote economic inclusion.

