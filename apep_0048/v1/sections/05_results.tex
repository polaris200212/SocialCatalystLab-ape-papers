\section{Results}
\label{sec:results}

This section presents our main empirical findings on the effects of women's suffrage on female labor force participation, with particular attention to heterogeneity across urban and rural areas. We begin with aggregate effects, demonstrating that suffrage produced a meaningful increase in women's labor supply that is robust across specifications. We then document the central finding of this paper: contrary to our initial expectations based on the policy channel hypothesis, the effects of suffrage on female labor force participation are \textit{larger in rural areas than in urban areas}. This surprising pattern challenges conventional accounts that emphasize formal labor market regulation and points instead toward alternative mechanisms that may have operated more powerfully in rural contexts. Event study analyses provide visual evidence supporting our identification strategy and illuminate the timing of treatment effects. We conclude by examining additional dimensions of heterogeneity---by race and age---that shed light on the mechanisms underlying our main results.

\subsection{Main Effects on Female Labor Force Participation}
\label{subsec:main_results}

Table \ref{tab:main_results} presents estimates of the overall effect of women's suffrage on female labor force participation. Column (1) reports results from our baseline two-way fixed effects specification with state and year fixed effects, using the full sample of approximately 6.7 million women ages 18-64 from our 10 percent random sample of the 1880-1920 full-count censuses. Column (2) adds the triple-difference interaction with urban residence. Column (3) includes individual-level controls for age, age squared, and race. Column (4) reports the \citet{sun2021} interaction-weighted estimator as a robustness check that addresses concerns about heterogeneous treatment effects in staggered adoption designs.

The estimates reveal a consistent pattern: women's suffrage increased female labor force participation by approximately 2.3 percentage points. Our baseline TWFE estimate in column (1) yields an effect of 0.023 (standard error 0.012), approaching statistical significance at the 5 percent level (p = 0.054). The 95 percent confidence interval ranges from approximately 0 to 4.6 percentage points. The triple-difference specification in column (2) produces a virtually identical point estimate of 0.023 (standard error 0.011) for the main post-suffrage effect, while the interaction with urban residence is small (0.001) and statistically insignificant. This pattern---positive overall effects with no significant urban-rural differential in the triple-difference framework---foreshadows the more detailed stratified analysis below.

The inclusion of individual-level controls in column (3) has minimal impact on the estimated effects. The controlled specification yields a post-suffrage coefficient of 0.022 (standard error 0.010), statistically significant at the 5 percent level. Age enters with the expected inverted-U pattern: the coefficient on age is -0.021, and on age squared is 0.0002, indicating that labor force participation first decreases then slightly increases with age (though the concave shape is quite weak). Race coefficients reveal substantial disparities: Black women (RACE=2) have labor force participation rates approximately 30 percentage points higher than White women, reflecting the well-documented historical pattern that economic necessity compelled Black women's labor market participation regardless of social norms discouraging married women's work.

The \citet{sun2021} estimator in column (4) produces a point estimate of 0.033 for the overall average treatment effect on the treated, though with a very large standard error due to the limited variation in treatment timing across cohorts in our setting. The qualitative conclusion remains unchanged: suffrage increased female labor force participation, with point estimates consistently in the 2-3 percentage point range across specifications.

To assess the economic magnitude of these effects, we compare our estimates to baseline labor force participation rates. Among women aged 18 to 64 in our sample, labor force participation averaged approximately 22 percent in the pre-suffrage period, with substantial variation by treatment status, urban residence, and year. Our estimated effect of 2.3 percentage points therefore represents roughly a 10 percent increase relative to the pre-treatment mean---a meaningful effect that would have expanded the female workforce in treated states. The estimated effect is of similar magnitude to other policy effects documented in the suffrage literature, including \citet{miller2008}'s finding that suffrage reduced child mortality by 8-15 percent through increased public health spending.

\subsection{Event Study Evidence}
\label{subsec:event_study}

Figure \ref{fig:event_study_overall} presents event study estimates of the effect of suffrage on female labor force participation, plotting coefficients for each event time (years relative to treatment) along with 95 percent confidence intervals. The event study serves two purposes: it provides a visual test of the parallel trends assumption by examining pre-treatment dynamics, and it illuminates the timing and persistence of treatment effects following suffrage adoption.

The nature of our data---decennial census observations spanning adoption dates from 1893 (Colorado) to 1918 (New York, Michigan, Oklahoma, South Dakota)---means that event times are irregularly spaced and depend on the specific treatment cohort. For states adopting around 1912 (the modal adoption year in our sample), we observe event times of approximately -32 (1880), -12 (1900), -2 (1910), and +8 (1920). This coarse temporal structure limits our ability to examine dynamics at fine intervals, but the available event study nonetheless provides useful information about the parallel trends assumption.

The pre-treatment coefficients provide support for our identification strategy. At long pre-treatment horizons (event times -30 to -17), coefficients are close to zero, suggesting that treated and control states were on similar trajectories of female labor force participation decades before suffrage adoption. At intermediate pre-treatment horizons (event times -10 to -1), we observe some positive movement, though coefficients remain statistically insignificant. The pattern is consistent with parallel pre-trends, though we note that the limited number of pre-treatment census observations for many treatment cohorts constrains statistical power for these tests.

Post-treatment coefficients reveal positive effects that emerge following suffrage adoption. At event times immediately after treatment (0 to +10), coefficients are positive and larger than pre-treatment values, though precision is limited. The pattern is consistent with the causal interpretation: treated states experienced increases in female labor force participation following suffrage that were not present in the pre-period. The magnitude of post-treatment effects is broadly consistent with the overall ATT estimates from Table \ref{tab:main_results}.

\subsection{Urban-Rural Heterogeneity: A Surprising Pattern}
\label{subsec:urban_rural}

The central contribution of this paper is documenting suggestive heterogeneity in suffrage effects across urban and rural areas. Our initial hypothesis, grounded in the policy channel literature, predicted that urban effects would dominate: suffrage enabled protective labor legislation that had purchase primarily in formal wage labor markets concentrated in cities. Table \ref{tab:stratified} presents results that, while imprecisely estimated, suggest a pattern inconsistent with this hypothesis.

\textit{Important caveat on urban classification:} Individual urban status is not directly observed in our data and must be imputed probabilistically based on historical state-year urbanization rates. This imputation introduces measurement error that attenuates estimated differences between urban and rural women and complicates interpretation of the heterogeneity analysis. The patterns we document should therefore be interpreted as suggestive correlations with urbanization rather than definitive causal effects on actually-urban versus actually-rural women.

Column (1) of Table \ref{tab:stratified} reports the effect of suffrage for urban women (those residing in places with populations of 2,500 or more). The estimated coefficient is 0.015 (standard error 0.009), positive but \textit{not} statistically significant at conventional levels (p = 0.106). The 95 percent confidence interval ranges from approximately -0.003 to 0.033 percentage points, encompassing zero.

Column (2) reports the effect for rural women. Contrary to our expectations, the estimated effect is \textit{larger}: 0.028 (standard error 0.012), statistically significant at the 5 percent level (p = 0.027). This finding is striking: rural areas, where we expected null or small effects based on the policy channel hypothesis, instead show the larger and more precisely estimated response to women's suffrage.

The difference in effects between urban and rural areas is -0.013 percentage points (urban minus rural), indicating that rural effects exceed urban effects by approximately 1.3 percentage points. However, this difference is not statistically significant at conventional levels, and we cannot reject the null hypothesis that urban and rural effects are equal. The triple-difference coefficient for Post $\times$ Urban in Table \ref{tab:main_results} is near zero (0.001) and statistically insignificant. Combined with the measurement error introduced by urban status imputation, we interpret these patterns as suggestive rather than definitive: the point estimates are consistent with larger rural effects, but the evidence does not permit strong conclusions about urban-rural heterogeneity.

Figure \ref{fig:event_study_urban_rural} presents event studies separately for urban and rural women. The visual contrast is instructive. For rural women, we observe relatively flat pre-treatment coefficients followed by a clear positive shift in the post-treatment period---a pattern consistent with a causal effect of suffrage. For urban women, the pattern is noisier, with more variation in pre-treatment coefficients and a less clear break at the time of treatment. The event study evidence reinforces the finding from the stratified regressions: the cleaner causal pattern emerges for rural women, not urban women.

How should we interpret this surprising finding? Several possibilities merit consideration.

\textit{First, the null urban effect may reflect ceiling effects.} Urban women already had higher baseline labor force participation (approximately 23-24\%) than rural women (approximately 20-23\%), potentially leaving less room for policy-induced increases. If women most amenable to entering the labor force were already participating in urban areas, suffrage-related improvements in working conditions may have had limited scope to draw in additional workers.

\textit{Second, rural effects may reflect unmeasured changes in economic activity.} Census enumeration of women's work was notoriously incomplete, particularly for married women engaged in household-based production on farms. If suffrage changed how women's work was perceived---both by census enumerators and by women themselves---we might observe increased \textit{reporting} of labor force participation even absent changes in actual work behavior. This reporting channel may have been more important in rural areas where women's contributions to household production were most ambiguous.

\textit{Third, rural effects may reflect genuine changes in women's economic opportunities.} Suffrage may have operated through channels other than protective labor legislation that affected rural women. For example, if suffrage influenced access to education, agricultural extension services, or local government programs, rural women may have experienced expanded opportunities that translated into labor force participation. The Cooperative Extension Service, established through the Smith-Lever Act of 1914, created home demonstration programs aimed explicitly at farm women; if women's political voice influenced the allocation or content of these programs, rural labor market effects could have resulted.

\textit{Fourth, the pattern may reflect compositional changes.} Urbanization accelerated during this period, with rural-to-urban migration particularly pronounced among young women seeking employment. If suffrage-related improvements in urban labor markets drew the most labor-force-ready rural women into cities, the remaining rural population would have become increasingly selected on characteristics positively correlated with labor force participation. This compositional change could produce apparent rural effects even without any true effect of suffrage on individual behavior.

We cannot definitively adjudicate among these interpretations with the available data. The finding that rural effects dominate urban effects challenges the prevailing emphasis on protective legislation as the primary mechanism linking suffrage to labor market outcomes. Our results suggest that alternative channels---operating through norms, information, political voice over local programs, or changes in the measurement of women's work---may have been equally or more important than formal labor market regulation.

\subsection{Heterogeneity by Race and Age}
\label{subsec:race_age}

Table \ref{tab:heterogeneity} presents results disaggregated by race and age, dimensions of heterogeneity that may illuminate the mechanisms underlying our main findings.

Panel A examines heterogeneity by race. For White women in urban areas, the effect of suffrage is 0.007 (standard error 0.006), small and statistically insignificant. For White women in rural areas, the effect is larger at 0.013 (standard error 0.006), marginally significant. The difference (urban minus rural) is negative, consistent with the overall pattern of larger rural effects.

For non-white women (primarily Black women in our sample), the pattern is more pronounced but less precisely estimated due to smaller sample sizes. The point estimate for non-white urban women is 0.007 (standard error 0.012), while for non-white rural women it is 0.028 (standard error 0.015). These estimates should be interpreted cautiously given the limited sample of non-white women in treated states, particularly in urban areas. Nevertheless, the pattern is suggestive: larger effects in rural areas persist within racial subgroups.

Panel B examines heterogeneity by age. For young women (ages 18-34), the urban effect is 0.016 (standard error 0.011) and the rural effect is 0.031 (standard error 0.016). For older women (ages 35-64), the urban effect is 0.014 (standard error 0.011) and the rural effect is 0.027 (standard error 0.013). Both age groups show the pattern of larger rural effects, though point estimates are larger for younger women in both urban and rural areas. The age gradient is consistent with several mechanisms: young women were more likely to be making labor force entry decisions that could be influenced by policy changes, and young women in rural areas may have been particularly responsive to expanded opportunities outside traditional farm household production.

Figure \ref{fig:heterogeneity} presents these results visually, plotting point estimates with 95 percent confidence intervals for each demographic subgroup. The figure reveals several patterns. First, the urban-rural gradient persists across all subgroups: rural effects consistently exceed urban effects. Second, substantial uncertainty characterizes many of these estimates, particularly for non-white women where sample sizes are smaller. Third, the largest and most precisely estimated effects are for rural women across demographic categories.

\subsection{Interpreting Effect Magnitudes}
\label{subsec:magnitudes}

Before proceeding to robustness checks, we pause to consider the economic significance of our findings in broader context.

Relative to baseline participation rates, the effects are meaningful. Our estimated overall effect of 2.3 percentage points represents approximately 10 percent of the pre-suffrage mean labor force participation rate of approximately 22 percent. For rural women, where effects are concentrated, the 2.8 percentage point effect represents roughly 14 percent of the rural baseline. These proportional effects are comparable in magnitude to other important policy effects documented in the labor economics literature.

Relative to the secular trend in female labor force participation during this period, our estimates suggest that suffrage contributed modestly but meaningfully to rising female employment. Female labor force participation increased from approximately 18 percent in 1880 to approximately 23 percent in 1920 in our sample. Our estimates imply that suffrage may have contributed 2-3 percentage points to this trend in treated states---a substantial share of the overall increase, though not the dominant driver.

The concentration of effects in rural areas has implications for aggregate assessments of suffrage's impact. In 1920, approximately half of the U.S. population lived in urban areas. If suffrage effects were larger in rural areas (as our estimates suggest), then aggregate effects would be somewhat larger than a simple average of urban and rural effects weighted by population shares. Applying our stratified estimates to the urban-rural population distribution yields an aggregate effect on the order of 2.1-2.5 percentage points---consistent with our overall estimate.

\subsection{Summary of Main Findings}
\label{subsec:summary}

The results presented in this section support four main conclusions. First, women's suffrage increased female labor force participation by approximately 2.3 percentage points overall, representing roughly 10 percent of baseline participation---an economically meaningful effect comparable in magnitude to other policy effects documented in the suffrage literature. Second, contrary to our initial hypothesis, this effect was \textit{larger in rural areas than in urban areas}. Rural women experienced a 2.8 percentage point increase (significant at the 5 percent level), while urban women experienced a 1.5 percentage point increase (not statistically significant). Third, the urban-rural pattern persists across demographic subgroups defined by race and age, suggesting that it reflects a fundamental feature of how suffrage operated rather than compositional differences between urban and rural populations. Fourth, event study evidence supports the parallel trends assumption and reveals a cleaner causal pattern for rural women than for urban women.

These findings challenge the prevailing emphasis on protective labor legislation as the primary mechanism linking suffrage to women's labor market outcomes. The policy channel hypothesis predicts urban effects should dominate, as formal labor market regulation had purchase primarily in cities where wage labor predominated. Our finding of larger rural effects suggests that alternative mechanisms---perhaps operating through norms, information, political voice over local programs, or changes in the measurement of women's work---may have been equally or more important than formal labor market regulation.

The surprising nature of our central finding warrants both humility about interpretation and further investigation into mechanisms. We explore potential explanations in Section \ref{sec:mechanisms} and subject our findings to extensive robustness checks in Section \ref{sec:robustness}.

