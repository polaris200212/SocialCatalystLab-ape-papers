\section{Data}
\label{sec:data}

This section describes the data sources, sample construction, and key variables used in our empirical analysis. We draw on full-count census microdata from the Integrated Public Use Microdata Series (IPUMS USA), which provides a comprehensive enumeration of the American population during the period of staggered suffrage adoption. We document the timing of women's suffrage across states, describe our sample restrictions and variable definitions, and present summary statistics that motivate our empirical strategy.

\subsection{IPUMS Full-Count Census Data}

Our primary data source is the IPUMS USA full-count census microdata for the years 1880, 1900, 1910, and 1920 \citep{ipums2023}. The full-count data represent a complete enumeration of the American population, providing individual-level records for every person enumerated in each decennial census. This comprehensive coverage is essential for our analysis, which examines heterogeneous effects across states, census years, and urban-rural residence. Unlike the public-use samples that IPUMS has long provided---typically 1 to 5 percent density---the full-count files contain the universe of enumerated individuals, yielding sample sizes in the tens of millions per census year.

The choice of census years reflects both data availability and the timing of women's suffrage adoption. The 1890 census was largely destroyed by fire in 1921, leaving a twenty-year gap in the decennial series. Our analysis therefore spans four census years covering a forty-year period, during which the women's suffrage movement achieved its greatest legislative victories. The 1880 census provides a pre-treatment baseline for all states in our analysis except Wyoming and Utah, which adopted suffrage in 1869 and 1870, respectively, and which we exclude from our primary sample. The 1900, 1910, and 1920 censuses capture the progressive expansion of suffrage across states, culminating in the Nineteenth Amendment's ratification in August 1920.

The IPUMS project harmonizes census variables across years, facilitating consistent measurement of key concepts despite changes in enumeration procedures and questionnaire design. This harmonization is particularly valuable for our analysis, which requires comparable measures of labor force participation, urban residence, and demographic characteristics across four decades of census enumeration. We describe the specific variables and harmonization procedures in detail below.

\subsection{Sample Construction}

We restrict our analytical sample to women aged 18 to 64, the population of prime working age for whom labor force participation represents a meaningful economic decision. The lower bound of 18 reflects the age at which most women had completed schooling and entered the labor market; the upper bound of 64 captures the working-age population before the emergence of formal retirement institutions. We impose these age restrictions using the IPUMS variable \texttt{AGE}, which records age in completed years as of the census enumeration date.

Our sample excludes women residing in Wyoming and Utah, the two earliest adopters of women's suffrage. Wyoming Territory granted women the right to vote in 1869, and Utah Territory followed in 1870 (though Utah's suffrage law was revoked by Congress in 1887 and restored upon statehood in 1896). Because our earliest census year is 1880, we cannot observe pre-treatment outcomes for these states, violating a fundamental requirement of the difference-in-differences design. Our treatment group therefore consists of the eleven states that adopted suffrage between 1893 and 1918: Colorado, Idaho, Washington, California, Oregon, Kansas, Arizona, Montana, Nevada, New York, Michigan, Oklahoma, and South Dakota. Our control group consists of the thirty-three states that adopted women's suffrage only upon ratification of the Nineteenth Amendment in August 1920; these states serve as a ``never-treated'' comparison group throughout our analysis period.

The resulting sample contains approximately 58 million woman-year observations across the four census years. Table \ref{tab:summary} presents detailed sample sizes by treatment status and urban-rural residence. The full-count data provide ample statistical power to detect economically meaningful effects, even within relatively narrow subgroups defined by state, year, and demographic characteristics.

\subsection{Key Variables}

\subsubsection{Labor Force Participation}

Our primary outcome variable is labor force participation. For most census years, we use the IPUMS variable \texttt{LABFORCE}, which provides a harmonized indicator of whether an individual was engaged in gainful employment at the time of census enumeration. However, because \texttt{LABFORCE} is not available for the 1900 full-count census, we construct labor force participation from occupation codes using the harmonized \texttt{OCC1950} variable. Following IPUMS conventions, we classify individuals with \texttt{OCC1950} codes between 1 and 979 as in the labor force; codes of 0 or 980 and above indicate no occupation or unemployed/not in labor force. For census years where both \texttt{LABFORCE} and \texttt{OCC1950} are available (1880, 1910, 1920), the correlation between the two measures exceeds 0.99 and the agreement rate is 99.96\%, confirming that our occupation-based measure accurately captures labor force participation as defined by IPUMS. We use the \texttt{OCC1950}-based measure throughout our analysis for consistency across all census years.

We recognize that measurement of women's labor force participation in historical censuses is imperfect. Census enumerators systematically undercounted women's economic activity, particularly for married women whose work was often unpaid, irregular, or embedded within household production \citep{folbre1991, goldin1990}. Farm women who contributed to agricultural production through dairying, poultry raising, and food processing were typically recorded as ``keeping house'' rather than as gainfully employed. Similarly, women who earned income through home-based activities such as taking in boarders, laundering, or sewing were often not counted as labor force participants. These measurement limitations likely result in systematic underestimates of female labor force participation, with the degree of undercount varying by marital status, age, and urban-rural residence.

For our purposes, the critical question is whether measurement error differs systematically between treatment and control states in ways that might bias our estimates. We address this concern in several ways. First, the harmonized IPUMS coding ensures that enumeration procedures are applied consistently across states within each census year, eliminating cross-state variation in coding conventions as a source of bias. Second, our difference-in-differences design removes time-invariant differences in measurement between states, and our inclusion of year fixed effects removes common trends in enumeration practices. Third, we examine heterogeneous effects by marital status, which allows us to assess whether results are driven by married women (for whom measurement concerns are most severe) or unmarried women (for whom labor force status is more reliably recorded). Fourth, we conduct placebo tests using male labor force participation as an outcome; since men's labor force status was recorded more completely and since male labor force participation should be unaffected by women's suffrage, these tests provide a check on spurious findings driven by measurement artifacts.

\subsubsection{Urban-Rural Classification}

The key moderating variable in our analysis is urban-rural residence. Because individual-level urban status is not consistently available across all census years in the full-count data, we assign urban status probabilistically based on historical state-year urbanization rates obtained from published census tables. Each individual is assigned a probability of urban residence equal to their state-year urbanization rate, and we draw from a Bernoulli distribution with this probability to assign binary urban status. This imputation approach has two advantages: (1) it preserves the correct proportion of urban residents in each state-year cell, and (2) it allows us to conduct robustness checks using different urbanization thresholds. The main limitation is that individual-level urban status contains measurement error by construction. We address this concern through robustness checks that vary the urbanization threshold and through interpretation that emphasizes patterns across the urban-rural dimension rather than precise point estimates for each group. Section \ref{sec:robustness} reports sensitivity analyses using 100 different random seeds for urban assignment, demonstrating that our conclusions are robust to the specific random draws used.

The urban classification captures meaningful differences in economic structure and labor market conditions during our study period. Urban areas were characterized by wage labor in manufacturing, retail trade, domestic service, and the emerging clerical sector. Urban labor markets featured clear distinctions between home and workplace, between paid employment and household production, and between workers and employers. Women who worked in urban areas did so in observable settings---factories, offices, stores, and other employers' homes---where their labor was governed by explicit or implicit contracts and where working conditions were increasingly subject to state regulation.

Rural areas, by contrast, were dominated by agriculture and household production. Farm women's economic contributions were often unpaid, embedded within family enterprises, and invisible to census enumerators. The distinction between work and non-work, between production and consumption, was far less clear in rural contexts. Protective labor legislation---the policy channel through which suffrage might have affected urban women's labor markets---had limited applicability to agricultural work and household production.

One concern with our urban-rural comparison is that the classification may not be stable over time. A woman classified as rural in 1910 might live in the same location but be classified as urban in 1920 if local population growth pushed her community above the 2,500 threshold. More broadly, urbanization was proceeding rapidly during our study period: the share of the American population living in urban areas increased from 28 percent in 1880 to 51 percent in 1920. This compositional change raises the possibility that observed differences in suffrage effects between urban and rural women might reflect changes in who lived in urban versus rural areas, rather than differential effects of suffrage on the same populations.

We address this concern through several robustness checks. First, we examine whether results are sensitive to using baseline (1880 or 1900) state-level urbanization rates as a fixed characteristic, rather than time-varying individual-level urban status. Second, we restrict the sample to women residing in their state of birth, reducing the influence of selective migration between urban and rural areas. Third, our triple-difference design---which compares changes in the urban-rural gap across treated and control states---is robust to common trends in urbanization that affect all states similarly. Only differential changes in urbanization that coincide with suffrage adoption would bias our estimates, and we find no evidence of such differential trends in our event-study analyses.

\subsubsection{Demographic Covariates}

We include several demographic variables as controls in our regression specifications and as dimensions for heterogeneity analysis. Age is measured in years using the IPUMS variable \texttt{AGE}. We include age and age-squared as continuous controls in our main specifications, and we examine heterogeneous effects across age groups (18-29, 30-44, 45-64) to assess whether suffrage effects varied over the life cycle.

Marital status is coded from the IPUMS variable \texttt{MARST}, which distinguishes among never-married (single), currently married, and previously married (widowed, divorced, or separated) individuals. Marital status is a crucial determinant of female labor force participation during this period: unmarried women participated in the labor force at substantially higher rates than married women, who largely withdrew from formal employment upon marriage \citep{goldin1990}. We examine heterogeneous effects by marital status both to test for differential mechanisms and to assess whether results are robust across groups with different baseline labor force attachment.

Race is coded from the IPUMS variable \texttt{RACE}, which provides a harmonized classification across census years. We distinguish between White and Black women in our heterogeneity analyses, recognizing that labor market opportunities and constraints differed dramatically by race during the Jim Crow era. We also examine effects for foreign-born women using the IPUMS variable \texttt{NATIVITY}, which distinguishes between native-born and foreign-born individuals.

\subsection{Suffrage Adoption Timing}

The temporal and geographic variation in women's suffrage adoption provides the identifying variation for our difference-in-differences design. Table \ref{tab:suffrage_dates} presents the complete timeline of state-level suffrage adoption. The suffrage movement achieved its first success in Wyoming Territory in 1869, followed by Utah Territory in 1870. After a long hiatus, the movement regained momentum in the 1890s, with Colorado (1893) and Idaho (1896) extending the franchise to women. The 1910s witnessed a wave of adoption: Washington (1910), California (1911), Oregon, Kansas, and Arizona (all 1912), Montana and Nevada (both 1914), and New York (1917). The final pre-Amendment adoptions occurred in 1918, with Michigan, Oklahoma, and South Dakota granting women full voting rights in state elections. The Nineteenth Amendment, ratified on August 18, 1920, extended suffrage to women nationwide, ending the era of state-level variation that underlies our identification strategy.

For our empirical analysis, we define treatment timing based on the year in which women first gained full voting rights in state elections. States adopting suffrage between census years are coded as treated beginning with the first post-adoption census. For example, California adopted suffrage in 1911; we code California as untreated in 1880, 1900, and 1910, and as treated in 1920. This coding reflects the constraint that our outcome data are observed only at decennial intervals.

The geographic pattern of suffrage adoption exhibits a distinct Western concentration. Western states and territories---Wyoming, Utah, Colorado, Idaho, Washington, California, Oregon, Arizona, Montana, and Nevada---accounted for the majority of pre-1917 adoptions. The Eastern breakthrough came with New York's 1917 referendum, which suffrage activists viewed as a turning point in the national campaign. The Southern states, which had been most resistant to women's suffrage, adopted only upon federal compulsion via the Nineteenth Amendment.

This geographic concentration raises the question of whether our findings reflect the effects of suffrage per se or merely idiosyncratic characteristics of Western states. We address this concern through several approaches. First, our specification includes state fixed effects, which absorb all time-invariant differences between states---including the cultural, economic, and demographic factors that may have made Western states more receptive to suffrage. Second, our year fixed effects absorb common national trends that affected all states, including secular changes in women's labor force participation driven by industrialization, changing social norms, or other factors. Third, our control group includes both Western never-treated states and Eastern/Southern never-treated states, providing variation in counterfactual trajectories. Fourth, our event-study specification allows us to examine whether effects emerge precisely at the time of suffrage adoption or whether they reflect pre-existing differential trends.

\subsection{Summary Statistics}

Table \ref{tab:summary} presents summary statistics for our analytical sample, separately by treatment status and urban-rural residence. The first panel shows characteristics of women in states that adopted suffrage prior to 1920 (our treatment group); the second panel shows characteristics of women in states that adopted only upon the Nineteenth Amendment (our control group). Within each panel, we report statistics separately for urban and rural women.

Several patterns merit attention. First, female labor force participation increased over time in both treated and control states, reflecting the secular trend toward greater female employment that characterized this period. In 1880, approximately 15 percent of women aged 18-64 reported gainful employment; by 1920, this figure had risen to nearly 24 percent. Second, urban women participated in the labor force at substantially higher rates than rural women throughout our study period. This urban-rural gap reflects differences in labor market opportunities: urban economies offered wage employment in manufacturing, retail, and service sectors, while rural economies were dominated by agriculture and household production in which women's labor was often uncompensated or unrecorded. Third, the composition of the female population differed between urban and rural areas. Urban women were younger, more likely to be unmarried, and more likely to be foreign-born---characteristics associated with higher labor force participation.

The summary statistics also reveal the sample sizes available for our analysis. The full-count data provide millions of observations in each state-year-urban cell, ensuring adequate statistical power for our heterogeneity analyses. Even relatively small subgroups---such as Black women in specific states and years---contain tens of thousands of observations, sufficient for precise estimation of subgroup-specific effects.

\subsection{Measurement Considerations}

We conclude this section with a discussion of measurement issues that affect the interpretation of our findings. The measurement of women's labor force participation in historical censuses is known to be imperfect, and several specific concerns are relevant to our analysis.

First, as noted above, census enumerators systematically undercounted women's economic activity, particularly for married women and rural women \citep{folbre1991}. This undercounting affects the level of measured labor force participation but does not necessarily bias our difference-in-differences estimates, which rely on changes over time rather than levels. However, if the degree of undercounting changed differentially in treated versus control states---for example, if suffrage led to changes in how enumerators recorded women's work---our estimates could be biased. We find no evidence of such differential measurement changes in our robustness analyses.

Second, the urban-rural classification may not perfectly capture the distinction between formal wage labor markets and informal household production that motivates our theoretical framework. Some urban women worked in household-based activities (such as taking in boarders) that shared characteristics with rural household production, while some rural women worked for wages in agricultural processing, domestic service in nearby towns, or home-based manufacturing. The urban-rural classification is thus a proxy for labor market structure rather than a perfect measure. To the extent that this proxy is imperfect, our estimates likely understate the true difference in suffrage effects between formal wage labor markets and informal household production.

Third, the census captures labor force status at a single point in time, which may not fully represent women's labor market attachment over the course of the year. Women's employment was often seasonal or intermittent, particularly in agriculture and certain manufacturing industries. The census enumeration date (typically in April or June) may systematically over- or under-represent employment depending on seasonal patterns. However, because the enumeration date was consistent across states within each census year, seasonal patterns should not bias our cross-state comparisons.

Finally, we note that our analysis examines the extensive margin of labor force participation---whether women worked at all---rather than the intensive margin of hours worked or earnings. Historical censuses do not contain reliable information on hours or earnings, precluding analysis of these outcomes. The extensive margin is nonetheless economically meaningful: entry into the labor force represents a fundamental change in women's economic status, time allocation, and relationship to the formal economy.

