\section{Mechanisms}
\label{sec:mechanisms}

The finding that suffrage effects on female labor force participation were larger in rural areas than in urban areas requires explanation. In this section, we explore potential mechanisms that might account for this pattern, acknowledging upfront that our data do not permit definitive identification of any single channel. We organize the discussion around four broad categories of mechanisms: policy channels, measurement channels, compositional channels, and norms channels.

\subsection{Policy Channels}

The prevailing explanation in the suffrage literature emphasizes policy channels: suffrage enabled women to advocate for legislation and government programs that affected their economic opportunities. The concentration of effects in rural areas suggests that the relevant policy channels may have differed from the protective labor legislation emphasized in prior work.

\textit{Local programs and county-level government.} While state-level protective labor legislation may have had limited reach into rural areas, county-level government provided services of direct relevance to farm families. County agricultural agents, home demonstration programs, and local public health services were responsive to local political pressures. If women's enfranchisement influenced the allocation or quality of these services, rural women may have experienced expanded economic opportunities. The Cooperative Extension Service, established through the Smith-Lever Act of 1914, created home demonstration programs that taught skills with commercial applications---food preservation, poultry raising, dairy management. Counties where women could vote may have prioritized these programs differently than counties (in non-suffrage states) where women had no formal political voice.

\textit{Education and human capital.} Suffrage may have affected female labor force participation through educational channels. \citet{carruthers2018} documented that suffrage increased educational investment and schooling outcomes. If these effects extended to adult education programs or vocational training, the human capital channel could explain labor market effects. Rural areas had fewer formal educational institutions, potentially leaving more room for policy-induced improvements to affect outcomes.

\textit{Transportation and market access.} Road quality and transportation infrastructure affected rural women's ability to participate in market transactions. If women's political voice influenced county road expenditures, rural women may have gained improved access to markets for their products. This channel is speculative---we have no direct evidence on road expenditures by suffrage status---but it illustrates how political voice might have operated through channels other than state-level protective legislation.

\subsection{Measurement Channels}

The rural dominance pattern may partly reflect how census enumeration captured women's economic activity rather than changes in actual behavior.

\textit{Changing recognition of farm women's work.} Census enumeration of women's work was notoriously incomplete during this period, particularly for married women engaged in household-based production. The conceptual boundary between ``gainful occupation'' and ``household duties'' was especially ambiguous for farm women. When a woman tended the family garden, raised chickens for sale, kept the household accounts, or helped with harvest, was she engaged in an occupation? The answer depended on how enumerators and respondents understood the question.

Suffrage may have shifted these understandings. The public recognition of women's civic capacity through the vote may have encouraged census enumerators to take women's economic contributions more seriously. Similarly, women themselves may have been more willing to describe their activities as ``work'' rather than dismissing them as household duties. If this interpretive shift was concentrated in rural areas---where women's economic contributions were most ambiguous---we would observe apparent effects on rural labor force participation even absent behavioral change.

\textit{Differential measurement error.} More generally, if urban women's work was easier to observe and classify (they held jobs with clear titles at identifiable employers), while rural women's work required interpretation, then measurement error was likely greater in rural areas. Any factor that reduced measurement error in rural areas would manifest as increased reported labor force participation. Suffrage, by raising the salience of women's public roles, may have had exactly this effect.

\subsection{Compositional Channels}

The period 1880--1920 witnessed substantial demographic change, including accelerated urbanization. These compositional shifts may affect the interpretation of our estimates.

\textit{Selective migration.} Rural-to-urban migration during this period was particularly pronounced among young women seeking employment in cities. If suffrage-related improvements in urban labor markets attracted the most labor-force-ready rural women to migrate to cities, the remaining rural population would become selected on characteristics associated with labor force participation. Women who chose to remain in rural areas despite urban opportunities may have been those with stronger attachment to agricultural production or local enterprises. This selection could produce apparent rural effects without any causal effect of suffrage on individual behavior.

The migration interpretation has ambiguous implications. On one hand, it suggests that the apparent rural effects might be compositional rather than behavioral. On the other hand, if suffrage induced economically active women to remain in rural areas (by improving rural opportunities), the rural effects would reflect genuine causal effects operating through location decisions.

\textit{Changing household composition.} The composition of rural households changed during this period as family size declined and household production shifted. If these compositional changes differed by suffrage status, they could confound our estimates. We address this concern in our robustness checks by controlling for household characteristics, but the available controls are limited.

\subsection{Norms Channels}

Suffrage may have affected female labor force participation through normative channels that shifted attitudes about women's appropriate roles.

\textit{Symbolic effects of enfranchisement.} The extension of the franchise to women represented a symbolic recognition of women's civic capacity and public presence. This symbolism may have affected labor market outcomes even in the absence of policy change. If suffrage signaled that women's public participation was legitimate, social sanctions against women's market work may have diminished.

The concentration of effects in rural areas is puzzling from a norms perspective. Traditionally, rural communities are thought to have been more conservative and resistant to changing gender norms. If normative change drove the labor market effects, we would expect urban areas---more exposed to Progressive Era ideas and social movements---to show larger effects. The opposite pattern we observe suggests that norms channels, at least operating through direct attitude change, are unlikely to explain our findings.

\textit{Information and role models.} Suffrage campaigns may have exposed women to information about economic opportunities and to role models of economically active women. Suffrage organizations included working women and professional women whose examples may have influenced others' labor market decisions. However, suffrage organizations were concentrated in urban areas, making this channel unlikely to explain the rural dominance pattern.

\subsection{Limitations and Interpretation}

We cannot definitively distinguish among these mechanisms with the available data. The measurement channel is concerning because it suggests that our findings might reflect changes in how women's work was recorded rather than changes in women's actual economic activity. The compositional channel is concerning because it suggests that migration patterns might confound our estimates.

Several considerations argue against entirely attributing our findings to measurement or composition. First, the event study shows a clear break in the trajectory of rural female labor force participation at the time of suffrage adoption, not a gradual drift that might reflect measurement or compositional changes. Second, the pattern is robust to controls for observable household characteristics that might proxy for compositional change. Third, the magnitude of the estimated effects (approximately 2.8 percentage points) seems too large to be explained entirely by measurement changes in how enumerators classified existing work activities.

The most likely interpretation is that multiple mechanisms operated simultaneously. Policy channels may have expanded rural women's economic opportunities through local programs and services. Measurement channels may have increased recognition of farm women's existing economic contributions. Compositional channels may have contributed through selective migration patterns. The relative importance of these channels remains uncertain.

\subsection{Implications for the Policy Channel Hypothesis}

Our findings do not refute the policy channel hypothesis linking suffrage to labor market outcomes, but they suggest that the relevant policies may have differed from the protective labor legislation emphasized in prior work. State-level protective legislation---maximum hours laws, minimum wages, workplace safety standards---had purchase primarily in urban labor markets. The concentration of effects in rural areas suggests that other policy channels were at least as important: county-level programs, agricultural extension services, local infrastructure investments, or educational programs.

This reframing has implications for how we understand the political economy of suffrage. If the labor market effects of enfranchisement operated primarily through local programs rather than state-level legislation, the mechanisms through which women's political voice affected policy may have been more decentralized and less visible than the dramatic legislative victories that dominate historical accounts. Women's influence over county budgets, school board decisions, and extension program priorities may have mattered as much for economic outcomes as the high-profile protective legislation that generated contemporary debate and subsequent scholarly attention.

