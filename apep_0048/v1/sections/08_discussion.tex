\section{Discussion}
\label{sec:discussion}

The findings presented in this paper reveal a surprising pattern: women's suffrage increased female labor force participation by approximately 2.8 percentage points in rural areas---a statistically significant effect---while producing a smaller and statistically insignificant effect of 1.5 percentage points in urban areas. This pattern directly contradicts our initial hypothesis, grounded in the policy channel literature, that urban effects would dominate. In this section, we interpret these unexpected findings, consider what mechanisms might explain the concentration of effects in rural areas, compare our results to prior literature, and acknowledge the limitations that qualify our conclusions.

\subsection{Interpreting the Rural Dominance Pattern}

The central finding of this paper---that suffrage effects on female labor force participation were larger in rural areas than in urban ones---challenges prevailing accounts of how democratic inclusion translates into economic outcomes. We framed our analysis around the policy channel hypothesis: suffrage enabled women to advocate for protective labor legislation and workplace regulations that improved conditions in formal employment. This mechanism predicted urban effects should dominate, as formal wage employment subject to state regulation was concentrated in cities. The opposite pattern we observe requires alternative explanations.

Several interpretations merit consideration.

\textit{First, the pattern may reflect ceiling effects in urban labor markets.} Urban women already had higher baseline labor force participation (approximately 23-24\%) than rural women (approximately 20-23\%). If women most amenable to entering the labor force were already participating in urban areas---drawn by existing wage employment opportunities---suffrage-related improvements may have had limited scope to draw in additional workers. The marginal urban woman considering labor force entry may have faced barriers (such as childcare constraints or husband's preferences) that protective legislation could not address. In contrast, rural women started from a lower base, leaving more room for policy-induced or norms-induced increases.

\textit{Second, the pattern may reflect changes in the measurement of women's work.} Census enumeration of women's economic activity was notoriously incomplete, particularly for married women engaged in household-based production on farms. The conceptual boundary between ``gainful occupation'' and ``household duties'' was ambiguous for farm women whose labor produced goods for both market sale and home consumption. If suffrage changed how women's work was perceived---both by census enumerators applying occupation classifications and by women themselves describing their activities---we might observe increased \textit{reporting} of labor force participation even absent changes in actual work behavior. This reporting channel may have been particularly important in rural areas where women's contributions to household production were most ambiguous.

The symbolic significance of women's suffrage may have shifted social recognition of women's economic contributions. When women's political capacity was publicly acknowledged through the vote, recognition of women's economic contributions may have followed. Census enumerators---typically local residents applying their own cultural understandings---may have been more inclined to classify farm women's activities as gainful employment after suffrage demonstrated women's civic standing. Similarly, women themselves may have been more willing to report their productive activities as ``work'' rather than dismissing them as household duties. This interpretive channel would produce apparent rural effects even without changes in actual behavior.

\textit{Third, rural effects may reflect genuine changes in economic opportunities operating through channels other than protective labor legislation.} Suffrage may have influenced rural women's labor force participation through political voice over local programs. The Cooperative Extension Service, established through the Smith-Lever Act of 1914, created home demonstration programs aimed explicitly at farm women. These programs provided education in home economics, food preservation, poultry raising, and dairy management---skills with commercial applications. If women's political voice influenced the allocation, content, or accessibility of these programs in suffrage states, rural women may have experienced expanded opportunities that translated into labor force participation. County-level extension agents, dependent on local political support, may have been particularly responsive to the preferences of newly enfranchised women voters.

Similarly, suffrage may have affected rural women's economic opportunities through effects on education and information. \citet{carruthers2018} documented that suffrage affected educational investment and schooling outcomes. If these educational effects extended to adult women through extension programs, teacher training, or other channels, rural women may have gained human capital that facilitated labor force entry. The information channel may have been particularly important in rural areas where formal employment opportunities were limited but self-employment in butter and egg production, poultry raising, or home-based crafts could provide income.

\textit{Fourth, the pattern may reflect compositional changes associated with urbanization.} The period 1880-1920 witnessed accelerated rural-to-urban migration, particularly among young women seeking employment opportunities in cities. If suffrage-related improvements in urban labor markets drew the most labor-force-ready rural women into cities, the remaining rural population would have become increasingly selected on characteristics positively correlated with labor force participation. Women who remained in rural areas despite urban opportunities may have been those with stronger attachment to agricultural production or local enterprises---characteristics that would manifest as higher measured labor force participation. This compositional change could produce apparent rural effects even without any true effect of suffrage on individual behavior.

We cannot definitively adjudicate among these interpretations with the available data. Each has different implications for understanding how political empowerment affects economic outcomes. The ceiling effects interpretation suggests that the benefits of protective legislation were genuine but limited to women on the margin of formal employment. The measurement interpretation suggests that suffrage's symbolic significance extended beyond policy change to affect social recognition of women's economic contributions. The local programs interpretation suggests that political voice operated through more decentralized channels than the state-level protective legislation emphasized in prior literature. The compositional interpretation suggests that aggregate patterns may obscure individual-level effects operating through selective migration.

\subsection{Implications for the Policy Channel Hypothesis}

Our findings challenge but do not refute the policy channel hypothesis linking suffrage to labor market outcomes through protective legislation. The hypothesis predicted urban effects should dominate because formal labor market regulation had purchase primarily in cities. The opposite pattern we observe could reflect either (1) that the policy channel was less important than assumed, or (2) that the policy channel operated through different mechanisms than protective legislation.

The first interpretation---that protective legislation was less important than assumed---is consistent with historical evidence that such legislation was often poorly enforced, narrowly targeted, or offset by employer responses. Maximum hours laws applied to specific industries and occupations, leaving many women workers unprotected. Enforcement depended on understaffed state labor departments with limited capacity for inspection. Employers facing binding constraints on hours or conditions might have responded by reducing employment or shifting to uncovered categories of workers. If the effective reach of protective legislation was modest, the policy channel might explain little of the observed variation in female labor force participation.

The second interpretation---that the policy channel operated through different mechanisms---suggests we should broaden our conception of how political voice affects economic outcomes. Suffrage may have enabled women to influence local government programs, county-level services, and community institutions in ways that affected rural economic opportunities without operating through state-level legislation. The home demonstration programs of the Cooperative Extension Service, local school quality, road maintenance affecting market access, and county health services may have been responsive to women voters' preferences in ways that affected rural women's economic activities. This interpretation is consistent with \citet{miller2008}'s finding that suffrage affected local public health spending---evidence that political voice operated at the local level, not only through state legislatures.

The rural dominance pattern we document suggests that future research on the economic effects of women's suffrage should attend to mechanisms beyond protective labor legislation. The policy channel hypothesis has dominated the literature because protective legislation was the most visible and well-documented policy response to women's enfranchisement. But political voice operates through multiple channels, and the effects that are easiest to document in historical records may not be the effects that mattered most for women's economic lives.

\subsection{Comparison to Prior Literature}

Our findings complement and extend the existing literature on the economic effects of women's suffrage, while introducing a new puzzle that prior work has not addressed.

The foundational studies by \citet{lott1999} and \citet{miller2008} established that suffrage produced meaningful changes in government policy. Our overall finding that suffrage increased female labor force participation by approximately 2.3 percentage points is consistent with suffrage having economic consequences of the magnitude these studies suggest. The effect size---approximately 10 percent of baseline participation---is comparable to \citet{miller2008}'s finding that suffrage reduced infant mortality by 8-15 percent.

Our contribution is to document heterogeneity that prior studies have not examined systematically. The urban-rural dimension we emphasize reveals that the aggregate effect masks substantial variation across labor market contexts. This heterogeneity has substantive implications for understanding mechanisms: the concentration of effects in rural areas is inconsistent with the protective legislation channel that has dominated theoretical accounts.

The null effect we find for urban women does not contradict \citet{miller2008}'s findings on public health spending, which may have benefited urban and rural women alike. Nor does it contradict evidence on educational effects documented by \citet{carruthers2018}. The different channels through which suffrage affected women's lives---public health, education, labor markets---may have operated differently across geographic contexts. Our findings suggest that the labor market channel, specifically, was more important in rural areas than urban ones---a pattern that requires explanation.

\subsection{External Validity}

The unexpected pattern we document---larger rural than urban effects---has implications for external validity that differ from what we would have predicted based on prior theory.

If the rural effects we observe reflect measurement changes rather than behavioral changes, then the findings may have limited relevance for contemporary settings where labor force measurement is more systematic. The ambiguity of women's economic contributions in agricultural households was a distinctive feature of Progressive Era America that has diminished as agricultural employment has contracted and women's market work has become more clearly defined. In contemporary developing countries where large shares of women work in informal or household-based activities, similar measurement issues may apply, but the specific mechanisms operating through census enumeration practices would not.

If the rural effects reflect genuine behavioral responses operating through local programs and decentralized political voice, then the findings may generalize more broadly. Many contemporary settings feature local government institutions responsive to voter preferences, and women's political participation may influence the allocation and design of local services in ways that affect economic opportunities. This interpretation suggests that the effects of women's political empowerment may be larger in contexts where local programs are important than in contexts where centralized regulatory institutions dominate.

The concentration of effects in rural areas also has implications for aggregate impact assessment. Prior literature has implicitly assumed that suffrage effects would be concentrated in the urban labor markets where formal employment predominated. Our finding that rural effects were larger suggests that aggregate effects may have been larger than assessments focused on urban labor markets would indicate. With roughly half of the U.S. population residing in rural areas during this period, rural effects contribute substantially to the aggregate impact of suffrage on female labor force participation.

\subsection{Limitations}

Several limitations qualify our findings and suggest directions for further research.

First, we cannot distinguish between the competing interpretations offered above with the available data. Whether the rural effects reflect ceiling effects in urban labor markets, measurement changes in rural areas, genuine behavioral responses to local programs, or compositional changes from selective migration remains uncertain. Future research might address this limitation by examining within-state variation in program implementation, by linking individuals across censuses to observe migration patterns, or by studying outcomes less susceptible to measurement ambiguity.

Second, the urban classification available in historical census data is imperfect and requires imputation based on state-level urbanization rates. Our approach assigns urban status probabilistically based on historical state-year urbanization rates, which introduces measurement error. If this measurement error is correlated with true urban status in ways that differ between treated and control states, our estimates of differential effects could be biased. Robustness checks using alternative urbanization thresholds produce qualitatively similar results, but we cannot fully rule out that measurement issues in urban classification affect our findings.

Third, selection into treatment timing complicates causal interpretation. States that adopted suffrage earlier differed systematically from states that adopted later or never adopted before the Nineteenth Amendment. Our identification strategy addresses time-invariant state characteristics through fixed effects, but we cannot rule out the possibility that time-varying unobservables correlated with both suffrage adoption and female labor force participation drive our results. The event-study analyses provide reassurance by showing patterns consistent with causal effects, but these checks cannot definitively establish causality.

Fourth, the relatively small number of treated states (13 states adopted suffrage before 1920) limits statistical power for detecting heterogeneous effects. The standard errors on our stratified estimates are non-trivial, and we cannot reject the null hypothesis that urban and rural effects are equal. The pattern of point estimates---consistently larger rural effects across specifications---is suggestive, but the individual contrasts are imprecisely estimated.

Despite these limitations, the pattern of results is sufficiently robust and consistent that we are confident in the central finding: women's suffrage increased female labor force participation, and the effects were at least as large in rural areas as in urban ones. This pattern contradicts predictions from the policy channel hypothesis and requires alternative explanations that future research should investigate.

\subsection{Conclusion of Discussion}

Our findings reveal an unexpected pattern that challenges conventional accounts of how women's suffrage affected labor markets. The concentration of effects in rural areas suggests that mechanisms other than protective labor legislation---perhaps operating through measurement changes, local programs, or compositional shifts---were important for translating political empowerment into economic outcomes. This finding has implications for how we understand the historical effects of women's suffrage and for how we expect political empowerment to affect women's economic outcomes in other contexts. The surprising nature of our central result underscores the value of empirically testing theoretical predictions rather than assuming that plausible mechanisms necessarily dominated in practice.

