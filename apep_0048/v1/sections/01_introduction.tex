\section{Introduction}
\label{sec:introduction}

In 1910, a woman living in Seattle could cast a ballot in state elections; her counterpart in rural Kansas could not. By 1912, the Kansas woman had gained the franchise as well. Yet these two women inhabited starkly different labor markets. The Seattle woman lived in a rapidly industrializing city where women worked in department stores, telephone exchanges, and clerical offices---wage labor governed by formal contracts, subject to state regulation, and visible to policymakers. The Kansas woman lived on a farm where her labor, though economically essential, was unpaid, informal, and embedded within household production. When women gained the right to vote, did their newfound political power translate into economic opportunity? And did this transformation unfold differently across the urban-rural divide that defined American economic life in the early twentieth century?

This paper investigates the heterogeneous effects of women's suffrage on female labor force participation across urban and rural areas during the Progressive Era. We exploit the staggered adoption of women's suffrage across U.S. states between 1869 and 1920 using a triple-difference design that compares changes in women's labor supply in treated versus control states, before versus after suffrage adoption, and in urban versus rural areas within states. Drawing on full-count census data from the Integrated Public Use Microdata Series \citep{ipums2023}, we analyze the labor market decisions of over 50 million women across four census years (1880, 1900, 1910, and 1920). Our identification strategy leverages modern difference-in-differences methods robust to heterogeneous treatment effects \citep{callaway2021, sun2021, goodman2021}, addressing concerns about bias in canonical two-way fixed effects estimators when treatment timing varies.

Our findings reveal a striking and unexpected pattern: the labor market effects of women's suffrage were \textit{larger in rural areas than in urban areas}---the opposite of what the policy channel hypothesis predicts. Among rural women, suffrage adoption increased labor force participation by approximately 2.8 percentage points, a statistically significant effect. Among urban women, the effect was smaller at 1.5 percentage points and not statistically significant. This rural dominance pattern persists across a battery of robustness checks, including alternative estimators robust to heterogeneous treatment effects \citep{sun2021}, restrictions to late-adopting states with multiple pre-treatment periods, alternative urbanization thresholds, and analyses stratified by race and age. The pattern is not driven by compositional changes in who lived in urban versus rural areas, nor by differential measurement error across contexts.

These results speak to a fundamental question in political economy: through what channels does democratic inclusion translate into economic change? The theoretical literature offers two broad mechanisms. First, extending the franchise may enable newly enfranchised groups to advocate for policies that directly improve their economic position---what we term the ``policy channel.'' Women's organizations in the Progressive Era lobbied vigorously for protective labor legislation, minimum wages, workplace safety standards, and equal pay provisions \citep{kessler1982, skocpol1992}. If suffrage empowered these advocacy efforts, effects should concentrate in labor markets where such policies could bite: urban areas with formal wage employment, regulatory oversight, and observable working conditions. Second, suffrage may have shifted social norms about appropriate gender roles---a ``norms channel'' that could operate independently of policy. If voting symbolized women's capacity for public participation and rational deliberation, this ideological shift might have expanded the sphere of acceptable female activity, potentially reducing social sanctions against women who worked outside the home. The norms channel predicts effects that could be as strong or stronger in rural areas, where traditional gender ideologies were more deeply entrenched and where the symbolic meaning of enfranchisement might have been particularly transgressive.

Our finding of larger rural effects challenges the prevailing emphasis on protective legislation as the mechanism linking suffrage to labor market outcomes. This surprising pattern requires alternative explanations. One possibility is that suffrage operated through measurement channels: the symbolic recognition of women's civic capacity may have shifted how census enumerators and women themselves understood farm women's economic contributions, leading to increased \textit{reporting} of labor force participation even absent behavioral change. Another possibility is that suffrage affected rural women through policy channels other than protective legislation---county-level programs, agricultural extension services, and local government responsiveness to newly enfranchised voters \citep{miller2008, carruthers2018}. A third interpretation emphasizes ceiling effects: urban women already participated in the labor force at higher rates, leaving less room for policy-induced increases. We cannot definitively adjudicate among these mechanisms with our data, but our findings suggest that the channels through which democratic inclusion translates into economic opportunity were more complex than the protective legislation hypothesis implies.

\subsection*{Historical Context: Women's Labor and the Urban-Rural Divide}

The period from 1880 to 1920 witnessed a profound transformation in American women's labor force participation, a transformation that played out quite differently in cities and on farms. At the opening of our study period, female labor force participation stood at roughly 15 percent nationwide, though this figure obscures substantial heterogeneity \citep{goldin1990}. In rural areas, women's work was largely invisible to census enumerators: farm wives labored in dairy production, poultry raising, vegetable gardening, and food preservation---activities that were economically essential but classified as ``keeping house'' rather than gainful employment. Urban women, by contrast, increasingly entered formal wage labor in the burgeoning sectors of manufacturing, domestic service, retail trade, and clerical work. By 1920, female labor force participation had risen to approximately 24 percent, with gains concentrated among young, unmarried, urban women \citep{goldin2006}.

This urban-rural divergence in women's labor markets reflected deeper structural differences in economic organization. Urban labor markets featured clear distinctions between home and workplace, between paid labor and household production. Women who worked for wages did so in observable settings---factories, offices, stores---where their hours, conditions, and compensation were matters of public record. This visibility made urban women's labor a natural target for Progressive Era reform. State legislatures debated and enacted laws governing maximum hours for women workers, minimum wages for female employees, prohibitions on night work, and requirements for rest breaks and sanitary facilities \citep{goldin1990, kessler1982}. The constitutionality of such sex-specific protective legislation was affirmed by the Supreme Court in \textit{Muller v. Oregon} (1908), which accepted the argument that women's physical differences and maternal responsibilities justified special state protection.

Rural labor markets operated on entirely different principles. Farm women's productive activities were embedded within household economies where distinctions between work and leisure, production and consumption, blurred into irrelevance. A woman who spent her morning milking cows, her afternoon preserving vegetables, and her evening mending clothes was engaged in continuous economic production, yet census enumerators recorded her occupation as ``none'' or ``keeping house.'' This invisibility had both measurement implications---female labor force participation in rural areas was systematically understated---and policy implications. Protective labor legislation could not reach work that took place within the household, that was organized by family relationships rather than employment contracts, and that was compensated through shared consumption rather than individual wages.

The women's suffrage movement navigated these distinct terrains with varying success. Urban suffragists drew on networks of women's clubs, settlement houses, and labor organizations that provided organizational infrastructure for political mobilization \citep{flexner1975}. Rural suffragists faced the challenges of geographic dispersion, limited communication networks, and the resistance of agricultural communities to federal interference of any kind. Yet the suffrage movement ultimately succeeded in both contexts: by the time the Nineteenth Amendment was ratified in 1920, women in every state had gained the franchise. The question we address is whether this uniform political change produced uniform economic effects.

\subsection*{Related Literature}

Our paper contributes to three interconnected literatures: the economic history of women's suffrage, the economics of gender gaps in labor markets, and the econometrics of difference-in-differences designs with heterogeneous treatment effects.

The foundational contribution to the economics of women's suffrage is \citet{lott1999}, who documented that suffrage adoption was associated with significant increases in state government expenditure and revenue. They interpreted this finding through the lens of median voter theory: women's policy preferences differed systematically from men's, and enfranchisement shifted the political equilibrium toward policies favored by the new median voter. \citet{miller2008} extended this insight in an influential study showing that suffrage led to large increases in public health spending and corresponding decreases in child mortality. Miller's identification strategy exploited within-state variation in the timing of suffrage adoption using a difference-in-differences framework, and his results demonstrated that women's political voice translated into concrete policy outcomes that improved child welfare. More recently, \citet{moehling2020} provided a comprehensive synthesis of the economic literature on women's enfranchisement, documenting effects on public expenditure composition, prohibition legislation, and various measures of government responsiveness.

Our contribution to this literature is to examine labor market outcomes rather than government policy or public health. While previous work has established that suffrage changed what governments did, we ask whether suffrage changed what women themselves did in the labor market. This distinction matters for understanding the channels through which democratic inclusion affects economic outcomes. If suffrage improved women's labor market position primarily through policy changes---protective legislation, equal pay laws, anti-discrimination enforcement---then effects should be concentrated in labor markets where such policies have purchase. Our urban-rural comparison provides a direct test of this mechanism.

The economics of gender gaps in labor markets has documented both the long-run evolution of female labor force participation and the persistence of gender differences in earnings, occupations, and employment \citep{goldin1990, goldin2006, olivetti2016}. \citet{goldin1990} provided the canonical account of how American women's labor force participation evolved from the nineteenth century through the present, emphasizing the role of structural transformation, educational expansion, and changing social norms. Her framework distinguishes between the extensive margin---whether women work at all---and the intensive margin---what kinds of jobs women hold and how much they earn conditional on working. We focus on the extensive margin, examining whether suffrage affected the probability that women participated in the labor force.

A crucial insight from this literature is that female labor force participation patterns differed dramatically by marital status, urban residence, and nativity during our study period \citep{goldin1990, kessler1982}. Young, unmarried women entered the labor force at high rates, particularly in urban areas where opportunities in manufacturing, retail, and clerical work were expanding. Married women, by contrast, largely withdrew from formal employment upon marriage---a pattern that prevailed until the mid-twentieth century. Our empirical strategy accounts for these compositional differences by including detailed controls for age, marital status, and race, and by examining heterogeneous effects across these dimensions.

Related work on women's labor in the Progressive Era has emphasized the importance of protective legislation in shaping female employment patterns. \citet{goldin1988} found that maximum hours laws for women reduced female employment in regulated industries while potentially increasing employment in unregulated sectors. \citet{moehling2009} documented the effects of Progressive Era child labor laws on family labor supply decisions, showing how regulations targeting one family member could affect the labor supply of others. These findings suggest that the policy environment mattered considerably for women's labor market outcomes in this period, and that changes in that environment---potentially including changes brought about by women's suffrage---could have had meaningful effects on female employment.

The recent econometric literature on difference-in-differences with staggered treatment adoption has fundamentally reshaped empirical practice in economics. \citet{goodman2021} demonstrated that canonical two-way fixed effects estimators can be severely biased when treatment effects are heterogeneous across units or over time. In staggered adoption designs, the TWFE estimator implicitly uses already-treated units as controls for newly-treated units, and when treatment effects vary with time since treatment, this comparison yields a weighted average of effects with potentially negative weights. \citet{callaway2021} and \citet{sun2021} proposed alternative estimators that aggregate only ``clean'' comparisons between treated and not-yet-treated (or never-treated) units, producing consistent estimates under weaker assumptions. \citet{roth2023} provided a comprehensive synthesis of this literature, offering practical guidance for applied researchers.

We implement these modern methods throughout our analysis. Our preferred specification uses the \citet{callaway2021} estimator with never-treated states as the comparison group, aggregating group-time average treatment effects into event-study plots that visualize treatment effect dynamics. We report results from multiple estimators---including standard TWFE, the \citet{sun2021} interaction-weighted estimator, and stratified analyses---to assess robustness. Following \citet{rambachan2023}, we also report sensitivity analyses that allow for violations of parallel trends, showing how our conclusions change under assumptions that permit pre-existing differential trends of various magnitudes.

\subsection*{Our Contribution}

This paper makes three contributions. First, we provide new evidence on the labor market effects of women's suffrage, complementing existing work on public expenditure and public health outcomes. While \citet{miller2008} and \citet{moehling2020} have documented that suffrage changed government policy, we examine whether suffrage changed women's own economic behavior. Our finding of positive effects on female labor force participation---with an overall effect of approximately 2.3 percentage points representing roughly 10 percent of baseline participation---demonstrates that political empowerment translated into measurable changes in women's labor market outcomes.

Second, we document surprising heterogeneity in the effects of democratic inclusion by urban-rural residence. This finding has important implications for understanding how political empowerment translates into economic opportunity. The concentration of effects in rural areas contradicts the policy channel hypothesis, which predicted urban effects would dominate because protective legislation operated primarily in formal wage labor markets. Instead, our results suggest that alternative mechanisms---measurement changes, local programs, or compositional effects---may have been equally or more important than state-level protective legislation. This reframing has implications for how we understand the political economy of suffrage: women's influence over county budgets, school board decisions, and extension program priorities may have mattered as much for economic outcomes as the high-profile protective legislation that has dominated scholarly attention.

Third, we demonstrate the application of modern staggered difference-in-differences methods to an important historical question. The women's suffrage literature has largely relied on pre-2020 econometric methods that are now known to produce potentially biased estimates under treatment effect heterogeneity. By implementing the estimators of \citet{callaway2021}, \citet{sun2021}, and the sensitivity analyses of \citet{rambachan2023}, we show how these modern tools can be brought to bear on historical questions while providing researchers with a template for similar applications.

Our findings also connect to broader debates about the relationship between political institutions and economic development. \citet{acemoglu2001} and subsequent work in political economy has emphasized that democratic institutions can promote economic prosperity by constraining extractive elites and providing voice to broader segments of the population. The extension of suffrage to women represents one of the largest expansions of political participation in American history, and understanding its economic consequences illuminates the mechanisms through which democratic inclusion might affect development. Our results suggest that these mechanisms operated differently than existing theory predicts: political voice translated into economic opportunity more powerfully in rural areas than in urban ones, challenging accounts that emphasize formal regulatory channels as the primary link between political empowerment and economic outcomes.

The paper also speaks to ongoing debates about gender inequality and political representation. Contemporary research has documented that female political representation---women serving as legislators, mayors, or executives---can affect policy outcomes in domains relevant to women's interests \citep{chattopadhyay2004, ferreira2014}. Our historical analysis examines a more fundamental form of political inclusion: the extension of voting rights to half the population. If voting rights alone could produce meaningful changes in women's economic outcomes, this suggests a powerful channel through which political institutions shape economic inequality. Conversely, if voting rights produced limited or heterogeneous effects, this might suggest that formal political inclusion is insufficient without complementary changes in economic and social institutions.

\subsection*{Data, Methods, and Empirical Strategy}

Our empirical analysis draws on full-count census data from the Integrated Public Use Microdata Series (IPUMS), covering the census years 1880, 1900, 1910, and 1920 \citep{ipums2023}. The full-count data provide a comprehensive enumeration of the American population, allowing us to examine labor force participation patterns across states, years, urban-rural residence, and demographic subgroups with minimal sampling error. We restrict attention to women aged 18-64, the population of prime working age for whom labor force participation is a meaningful margin of adjustment.

Our identification strategy exploits the staggered adoption of women's suffrage across states prior to the Nineteenth Amendment. Between 1869 (Wyoming Territory) and 1918 (Michigan, Oklahoma, South Dakota), thirteen states or territories extended full voting rights to women in state elections. The remaining states adopted women's suffrage only upon ratification of the Nineteenth Amendment in August 1920. This variation in treatment timing provides the foundation for our difference-in-differences design.

We exclude the earliest adopters---Wyoming (1869) and Utah (1870)---from our primary analysis because the available census data do not include pre-treatment observations for these states. Our treatment group therefore consists of eleven states that adopted suffrage between 1893 (Colorado) and 1918, with comparison provided by the thirty-three states that never adopted state-level suffrage and serve as a ``never-treated'' control group. This design follows best practices in the modern difference-in-differences literature, which emphasizes the value of never-treated comparison groups for avoiding contamination by heterogeneous treatment effects \citep{callaway2021, roth2023}.

The triple-difference design adds a third dimension of comparison: urban versus rural residence within states. This extension allows us to separate state-wide changes in female labor force participation---which might reflect suffrage effects operating through policy, norms, or other channels---from differential changes by urban-rural residence---which speak specifically to mechanisms that operate more strongly in one context than the other. The key identifying assumption is that, absent suffrage, the difference in female labor force participation between urban and rural women would have evolved similarly in treated and control states. We assess this assumption through event-study analyses that examine pre-treatment dynamics, and through sensitivity analyses that allow for bounded violations of parallel trends.

Our primary outcome measure is labor force participation, coded from the IPUMS variable \texttt{LABFORCE}, which harmonizes information on gainful employment across historical census years. We recognize that measurement of women's labor force status is imperfect in historical data, particularly for married women and rural women whose work may have been undercounted or misclassified \citep{goldin1990, folbre1991}. We address this concern in several ways: by examining sensitivity to alternative definitions of labor force participation, by considering heterogeneous effects by marital status (where measurement concerns are most acute for married women), and by conducting placebo tests using male labor force participation (which should be unaffected by women's suffrage).

\subsection*{Roadmap}

The remainder of this paper proceeds as follows. Section \ref{sec:background} provides historical background on the women's suffrage movement and the evolution of women's labor markets during the Progressive Era. We document the geographic and temporal patterns of suffrage adoption, describe the policy and social changes that accompanied enfranchisement, and develop hypotheses about why suffrage effects might differ between urban and rural areas.

Section \ref{sec:data} describes our data sources, sample construction, and variable definitions. We present summary statistics on female labor force participation by state, year, and urban-rural residence, and document the variation in suffrage timing that underlies our identification strategy.

Section \ref{sec:methods} presents our econometric framework. We begin with the canonical two-way fixed effects specification and explain why this approach may produce biased estimates under treatment effect heterogeneity. We then describe our implementation of modern staggered difference-in-differences estimators, including the \citet{callaway2021} group-time aggregation procedure and the \citet{sun2021} interaction-weighted estimator. We explain our approach to inference, which clusters standard errors at the state level to account for serial correlation within states and potential correlation of errors across individuals within state-year cells.

Section \ref{sec:results} presents our main findings. We begin with overall effects of suffrage on female labor force participation, then introduce the urban-rural heterogeneity that is our central contribution. Event-study figures visualize the dynamics of treatment effects, providing evidence on pre-trends and the timing of effects relative to suffrage adoption. We report results from multiple estimators and across a range of specifications to assess robustness.

Section \ref{sec:mechanisms} explores mechanisms underlying the urban-rural divergence. We examine whether effects differ by age, marital status, and race---dimensions that may help distinguish between competing explanations for the patterns we observe. We also examine effects on occupational composition among working women, asking whether suffrage changed not only whether women worked but also what kinds of work they did.

Section \ref{sec:robustness} presents additional robustness checks. We implement the sensitivity analysis of \citet{rambachan2023}, which allows for bounded violations of parallel trends; we restrict attention to late-adopting states with multiple pre-treatment periods; we examine placebo outcomes that should be unaffected by suffrage; and we assess sensitivity to alternative control group definitions and sample restrictions.

Section \ref{sec:discussion} interprets our findings in light of the historical record and existing literature. We discuss the policy implications of our results for understanding how democratic inclusion affects economic outcomes, and we consider limitations of our analysis and directions for future research.

Section \ref{sec:conclusion} concludes with a summary of our contributions and their implications for the economics of political institutions, gender, and labor markets.

