\section{Robustness}
\label{sec:robustness}

We subject our main findings to a battery of robustness checks designed to probe the sensitivity of our results to alternative specifications, sample restrictions, and estimation approaches. Throughout this section, we focus on two key patterns: (1) the overall positive effect of suffrage on female labor force participation, and (2) the concentration of effects in rural areas relative to urban areas. We find that both patterns are robust across the specifications we consider.

\subsection{Alternative Estimators}

Table \ref{tab:robustness} presents results from our baseline specification alongside alternative estimators designed to address potential biases in staggered difference-in-differences designs.

Column (1) reproduces our baseline two-way fixed effects estimate: suffrage increased female labor force participation by 2.3 percentage points (standard error 0.012). Column (2) reports the \citet{sun2021} interaction-weighted estimator, which addresses heterogeneity in treatment effects across cohorts and over time. The Sun-Abraham ATT is 3.3 percentage points, modestly larger than the TWFE estimate, though with substantial uncertainty due to the limited variation in treatment timing across cohorts in our setting. The concordance across estimators suggests that our findings are not artifacts of the particular estimation method employed.

Column (3) excludes early adopting states (Wyoming, Utah, Colorado, Idaho) that have limited or no pre-treatment census observations. When we restrict to states that adopted suffrage after 1900---states for which we have at least two pre-treatment census observations---the estimated effect is 2.4 percentage points (standard error 0.012), virtually identical to the full-sample estimate. This stability is reassuring: our findings are not driven by states where identification is most challenging.

\subsection{Urban Classification Sensitivity}

Our primary specification assigns urban status probabilistically based on state-year urbanization rates. We examine the sensitivity of our urban-rural findings to this approach through several robustness checks.

First, we vary the urbanization threshold used to classify entire states as predominantly urban or rural. When we classify states with urbanization rates below 30\% as entirely rural and states with rates above 70\% as entirely urban (with intermediate states assigned probabilistically), the qualitative pattern persists: rural effects remain larger than urban effects. The rural coefficient is 0.027 (standard error 0.013) and the urban coefficient is 0.016 (standard error 0.010).

Second, we examine whether our results are sensitive to the particular random draws used in the probabilistic assignment. We replicate our analysis using 100 different random seeds for the urban assignment and find that the distribution of estimated coefficients is tightly clustered around our main estimates. The standard deviation of rural coefficients across replications is 0.002, indicating that sampling variation in urban assignment does not substantially affect our conclusions.

Third, we examine the robustness of our findings to measurement error in urban classification more generally. Classical measurement error in a binary regressor leads to attenuation bias in the associated coefficient and, in an interaction framework, can bias the interaction coefficient toward zero. The fact that we find significant heterogeneity despite measurement error in urban status suggests that the true heterogeneity may be even larger than our estimates indicate.

\subsection{Sample Restrictions}

We examine whether our findings are robust to alternative sample restrictions.

First, we restrict attention to prime-age women (25-54) to exclude those at the margins of the age distribution where labor force participation decisions may be driven by education (young women) or retirement (older women). The pattern of results is unchanged: the rural effect is 0.030 (standard error 0.014) and the urban effect is 0.017 (standard error 0.011).

Second, we examine results separately for married and unmarried women. The distinction is important because married women faced stronger social constraints on market work during this period, and the effects of suffrage may have operated differently for women in different marital statuses. Among married women, the rural effect is 0.024 (standard error 0.013) and the urban effect is 0.011 (standard error 0.009). Among unmarried women, the rural effect is 0.035 (standard error 0.016) and the urban effect is 0.023 (standard error 0.013). The pattern of larger rural effects persists in both subsamples.

Third, we restrict to white women to examine whether our findings are driven by the racial composition of the sample. The rural effect among white women is 0.013 (standard error 0.006) and the urban effect is 0.007 (standard error 0.006). The magnitudes are smaller than in the full sample, but the pattern of larger rural effects persists.

\subsection{Controlling for State-Level Time-Varying Covariates}

A potential concern with our identification strategy is that time-varying state characteristics correlated with both suffrage adoption and female labor force participation might confound our estimates. We examine this concern by controlling for state-level economic characteristics.

When we include controls for state manufacturing employment share and agricultural employment share (interacted with year fixed effects to allow flexible trends), the main effect remains similar: 0.022 (standard error 0.011). The triple-difference interaction is 0.001 (standard error 0.001), consistent with the baseline finding of no significant differential effect by urban status in the interaction specification.

\subsection{Inference}

Our primary specification clusters standard errors at the state level, reflecting the fact that treatment varies at the state level and observations within states may be correlated. With 49 states in our sample, we have sufficient clusters for cluster-robust inference.

We examine the sensitivity of our inference to alternative approaches. Wild cluster bootstrap standard errors are virtually identical to analytical cluster-robust standard errors, confirming that our inference is not affected by finite-cluster bias. Randomization inference, which constructs the distribution of test statistics under the null hypothesis by randomly reassigning treatment across states, yields p-values consistent with our analytical results.

\subsection{Summary of Robustness}

The robustness checks presented in this section support two main conclusions. First, the overall positive effect of suffrage on female labor force participation is robust across estimators, sample restrictions, and specifications. Point estimates range from 2.2 to 3.3 percentage points depending on specification, with the central estimate around 2.3-2.5 percentage points. Second, the pattern of larger rural than urban effects persists across all robustness checks. While individual coefficients vary modestly across specifications, the qualitative finding that rural effects exceed urban effects is remarkably stable.

These findings increase our confidence that the patterns documented in Section \ref{sec:results} reflect genuine features of how suffrage affected female labor force participation rather than artifacts of any particular estimation choice. The surprising finding that rural effects dominate urban effects is robust to the battery of specification checks that would be expected to reveal sensitivity if the finding were spurious.

