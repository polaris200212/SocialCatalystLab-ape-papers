
\begin{table}[htbp]
\centering
\caption{Heterogeneous Effects by Race and Age}
\label{tab:heterogeneity}
\begin{threeparttable}
\begin{tabular}{lcccc}
\toprule
& \multicolumn{2}{c}{By Race} & \multicolumn{2}{c}{By Age} \\
\cmidrule(lr){2-3} \cmidrule(lr){4-5}
& (1) & (2) & (3) & (4) \\
& White & Non-white & Young (18-34) & Older (35-64) \\
\midrule
\multicolumn{5}{l}{\textit{Panel A: Urban Areas}} \\
Post-Suffrage & 0.006 & 0.027 & 0.015 & 0.017** \\
 & (0.006) & (0.025) & (0.011) & (0.008) \\
\midrule
\multicolumn{5}{l}{\textit{Panel B: Rural Areas}} \\
Post-Suffrage & 0.014*** & 0.024 & 0.034** & 0.024** \\
 & (0.005) & (0.029) & (0.014) & (0.012) \\
\midrule
\multicolumn{5}{l}{\textit{Panel C: Urban-Rural Difference}} \\
Difference & -0.008 & 0.002 & -0.019 & -0.008 \\
 & (0.008) & (0.038) & (0.018) & (0.014) \\
\midrule
State FE & Yes & Yes & Yes & Yes \\
Year FE & Yes & Yes & Yes & Yes \\
\bottomrule
\end{tabular}
\begin{tablenotes}
\small
\item \textit{Notes:} Dependent variable is labor force participation. Columns (1)-(2) stratify by race (White vs Non-white). Columns (3)-(4) stratify by age group. Panel A shows estimates for urban residents only. Panel B shows estimates for rural residents only. Panel C reports the difference. Standard errors clustered at state level. *** p$<$0.01, ** p$<$0.05, * p$<$0.1.
\end{tablenotes}
\end{threeparttable}
\end{table}

