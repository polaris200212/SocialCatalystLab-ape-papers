\documentclass[12pt]{article}

% UTF-8 encoding and fonts
\usepackage[utf8]{inputenc}
\usepackage[T1]{fontenc}
\usepackage{lmodern}

% Page setup
\usepackage[margin=1in]{geometry}
\usepackage{setspace}
\onehalfspacing

% Math and symbols
\usepackage{amsmath,amssymb}

% Graphics
\usepackage{graphicx}
\usepackage{float}

% Tables
\usepackage{booktabs}
\usepackage{array}
\usepackage{multirow}
\usepackage{tabularx}
\usepackage{threeparttable}

% Bibliography
\usepackage{natbib}
\bibliographystyle{aer}

% Hyperlinks
\usepackage{hyperref}
\hypersetup{
    colorlinks=true,
    linkcolor=blue,
    citecolor=blue,
    urlcolor=blue
}

% Captions
\usepackage{caption}
\captionsetup{font=small,labelfont=bf}

% Section formatting
\usepackage{titlesec}
\titleformat{\section}{\large\bfseries}{\thesection.}{0.5em}{}
\titleformat{\subsection}{\normalsize\bfseries}{\thesubsection}{0.5em}{}

% Custom commands
\newcommand{\E}{\mathbb{E}}
\newcommand{\Var}{\text{Var}}
\newcommand{\Cov}{\text{Cov}}

% Title
\title{Does EITC Eligibility Affect Employment for Childless Workers?\\Evidence from the Age-25 Threshold}
\author{APEP Autonomous Research\thanks{Autonomous Policy Evaluation Project. Data from IPUMS CPS and O*NET. All errors are our own. nd @dakoyana}}
\date{January 2026}

\begin{document}

\maketitle

\begin{abstract}
\noindent
We examine whether the Earned Income Tax Credit (EITC) affects employment for childless workers using the sharp eligibility threshold at age 25. Using real microdata from the Current Population Survey (2015-2024), we implement a regression discontinuity design comparing childless adults just above and below the cutoff. Our main finding is null: we detect no robust discontinuity in employment at the age-25 threshold. While employment increases with age throughout the 22-28 range, the pattern is consistent with smooth age-employment dynamics rather than a discrete EITC effect. Our preferred quadratic specification yields an economically small and statistically insignificant estimate of -0.3 percentage points (SE = 1.8 pp). We also find no heterogeneous effects by education level or occupation automation exposure. These null results are consistent with prior work suggesting that the small credit amount ($\sim$\$600 maximum) and low awareness among childless filers limit EITC's labor supply effects for this population. Our findings inform debates about expanding the childless EITC, suggesting that simply extending eligibility without increasing the credit amount may have limited employment effects.
\end{abstract}

\vspace{1em}
\noindent\textbf{JEL Codes:} J23, H24, J62, O33 \\
\noindent\textbf{Keywords:} EITC, labor supply, regression discontinuity, childless workers

\newpage

\section{Introduction}

The Earned Income Tax Credit (EITC) is the largest federal cash transfer program for low-income working families, distributing over \$60 billion annually. A robust literature documents that the EITC increases labor force participation among single mothers, but much less is known about effects on childless workers. Childless workers receive a much smaller maximum credit (approximately \$600 versus \$7,000+ for families with children) and face a unique age restriction: they are ineligible for any benefit until age 25.

This age restriction creates a sharp eligibility threshold that could, in principle, provide clean identification of EITC effects on childless worker labor supply. A worker aged 24 years and 364 days is ineligible for any credit, while the same worker one day later qualifies for up to \$600. If the EITC work incentive operates for childless workers as theory predicts, we should observe a discontinuous jump in employment at this threshold.

We test this hypothesis using individual-level data from the Current Population Survey Annual Social and Economic Supplement (CPS ASEC) for 2015-2024, obtained via IPUMS. Our analysis sample includes 100,182 childless adults aged 22-28, providing a three-year bandwidth on each side of the cutoff. We implement a regression discontinuity design, estimating the change in employment probability at the age-25 threshold while controlling flexibly for underlying age trends.

Our main finding is null. We detect no robust positive discontinuity in employment at the age-25 EITC eligibility threshold. While employment rates increase with age throughout the 22-28 range---from 65\% at age 22 to 81\% at age 28---this pattern is consistent with smooth age-employment dynamics rather than a discrete EITC effect. Our preferred quadratic specification yields an estimate of -0.3 percentage points (SE = 1.8 pp), which is economically small and statistically insignificant.

Motivated by debates about automation and the labor market, we also test whether EITC effects differ by occupational automation exposure or education level. We find no significant heterogeneity. Workers across all education levels and automation quartiles show similar null effects at the eligibility threshold.

Our null findings are consistent with prior work. Bastian and Jones (2022) also fail to find significant employment effects using a similar design. Several factors may explain why EITC eligibility has limited effects on childless worker employment:

\begin{enumerate}
\item \textbf{Small credit amount:} The maximum childless EITC (\$600) provides much weaker work incentives than the family credit (\$7,000+).
\item \textbf{Low awareness:} Many childless filers are unaware they qualify for the credit and fail to claim it.
\item \textbf{High baseline participation:} Young childless workers already have relatively high employment rates, limiting the pool of marginal workers.
\item \textbf{Discrete running variable:} Age measured in years provides only seven support points in our bandwidth, limiting statistical precision.
\end{enumerate}

This paper contributes to debates about expanding the EITC for childless workers. Recent policy proposals would lower the eligibility age to 19 or increase the maximum credit. Our findings suggest that simply extending eligibility to younger workers---without increasing the credit amount---may have limited employment effects. More generous reforms, such as substantially increasing the maximum credit for childless workers, may be necessary to generate meaningful labor supply responses.

The remainder of the paper proceeds as follows. Section 2 describes the institutional background. Section 3 discusses our data and empirical strategy. Section 4 presents results. Section 5 discusses mechanisms and implications. Section 6 concludes.

\section{Institutional Background}

\subsection{EITC for Childless Workers}

The Earned Income Tax Credit provides refundable tax credits to low- and moderate-income workers. The credit phases in with earned income, reaches a maximum, and then phases out. For tax year 2024, the maximum credit ranges from approximately \$3,700 (one child) to \$7,400 (three or more children).

For childless workers, the EITC is substantially less generous. The maximum credit is approximately \$600, and the phase-in and phase-out ranges are much narrower. Critically, eligibility is restricted to workers aged 25-64. This age restriction has been part of the EITC since its inception.

The policy rationale for the age-25 threshold is that younger childless workers have higher labor force participation rates and lower poverty rates than other groups, reducing the need for income support. However, this reasoning has been questioned, particularly as young workers face increasing labor market challenges.

The American Rescue Plan temporarily lowered the eligibility age to 19 and increased the maximum credit to approximately \$1,500 for tax year 2021. This provision expired, and the age-25 threshold is again in effect.

\subsection{Automation and Labor Markets}

A growing literature documents the effects of automation on employment. Acemoglu and Restrepo (2020) find that industrial robots reduce employment and wages, particularly for routine manual occupations. Autor, Dorn, and Hanson (2015) document persistent labor market disruptions in trade-exposed communities.

We use automation exposure data from O*NET, the Occupational Information Network maintained by the U.S. Department of Labor. O*NET provides a ``Degree of Automation'' measure rating each occupation on a 1-5 scale. We use this to test whether EITC effects differ for workers in high- versus low-automation occupations.

\section{Data and Empirical Strategy}

\subsection{Data}

We use individual-level data from the CPS Annual Social and Economic Supplement (ASEC) for 2015-2024, accessed via IPUMS. The CPS ASEC provides detailed information on employment, hours, earnings, and demographic characteristics.

Our sample restrictions are:
\begin{itemize}
\item Ages 22-28 (three-year bandwidth around cutoff)
\item Childless (no children in household)
\item Valid employment status (not NIU)
\item Positive survey weight
\end{itemize}

These restrictions yield an analysis sample of 100,182 observations.

Table 1 presents summary statistics. The sample is 45\% female, 55\% white non-Hispanic, and 37\% college graduates. The overall employment rate is 75.6\%, with labor force participation at 80.2\%. Employment rises with age: from 65.3\% at age 22 to 81.4\% at age 28.

\subsection{Empirical Strategy}

Our identification strategy exploits the sharp discontinuity in EITC eligibility at age 25. We estimate:

\begin{equation}
Y_i = \alpha + \tau D_i + f(Age_i - 25) + D_i \cdot f(Age_i - 25) + \gamma_t + \delta_s + \epsilon_i
\end{equation}

where $Y_i$ is employment status, $D_i = \mathbf{1}[Age_i \geq 25]$ indicates EITC eligibility, $f(\cdot)$ is a polynomial in age centered at 25, $\gamma_t$ are year fixed effects, and $\delta_s$ are state fixed effects. The coefficient $\tau$ measures the discontinuity at the eligibility threshold.

We report specifications with linear and quadratic polynomials. Given that age is measured in years (a discrete running variable with only 7 support points in our bandwidth), we present parametric estimates using OLS with heteroskedasticity-robust standard errors.

The key identifying assumption is that potential outcomes evolve smoothly through age 25 absent EITC eligibility. We test this by examining covariate balance at the threshold and conducting placebo tests at alternative cutoffs.

\subsection{Validity Tests}

\textbf{Covariate balance:} Figure 2 shows differences in covariates between workers aged 24 versus 25. Most covariates are well-balanced. However, we observe statistically significant differences in college graduation rates (41.2\% at 25 vs. 38.6\% at 24) and marital status. These differences likely reflect life-cycle transitions (college completion, marriage) that occur in this age range rather than selection into EITC eligibility.

\textbf{Density test:} With age measured in years, formal McCrary density tests have limited power. We observe 46,785 observations below age 25 and 53,397 at/above age 25, a ratio of 0.88. This imbalance reflects natural cohort sizes and sample composition rather than manipulation.

\textbf{Placebo cutoffs:} We estimate effects at alternative age cutoffs (23, 24, 26, 27) where no policy change occurs. Significant effects at these placebo cutoffs would cast doubt on our design.

\section{Results}

\subsection{Main Results}

Table 2 presents our main RDD estimates. Column 1 shows the simple linear specification without fixed effects. The coefficient on EITC eligibility is -3.1 percentage points (SE = 0.71 pp), which is statistically significant but negative. This counterintuitive result reflects the change in slope of the age-employment relationship: employment rises steeply from ages 22-24 but more slowly from ages 25-28. The linear model interprets this slope change as a negative discontinuity.

Columns 2 and 3 add year and state fixed effects, yielding similar estimates (-3.0 and -2.9 pp). Column 4 presents our preferred specification with a quadratic polynomial, which allows the age-employment relationship to curve. This specification yields an estimate of -0.3 percentage points (SE = 1.8 pp), which is economically small and statistically insignificant.

Figure 1 visualizes the data. Employment rates increase smoothly with age, from 65.3\% at age 22 to 81.4\% at age 28. There is no visible jump at the age-25 threshold. The pattern is consistent with smooth age-employment dynamics.

\subsection{Robustness}

\textbf{Bandwidth sensitivity:} Table 3 Panel A shows estimates across different bandwidths. Results are qualitatively similar across bandwidth choices, with point estimates ranging from -1.9 pp (±2 years) to -2.9 pp (±3 years).

\textbf{Placebo cutoffs:} Table 3 Panel B shows estimates at placebo cutoffs. We find significant negative effects at ages 26 and 27 (where no policy changes) and a significant positive effect at age 23. This pattern---significant effects at multiple ages where no policy discontinuity exists---suggests that the ``effects'' reflect the underlying curvature of the age-employment relationship rather than true discontinuities.

\textbf{Donut RDD:} Excluding observations at ages 24 and 25 yields an estimate of -4.1 pp (SE = 1.6 pp), larger in magnitude but consistent with the main findings.

\subsection{Heterogeneity}

\textbf{By education:} Table 4 shows estimates by education level. College graduates show a small negative effect (-1.6 pp, SE = 1.1 pp). Workers with some college show a larger negative effect (-3.9 pp, SE = 1.3 pp). Workers with high school or less show a small positive but insignificant effect (0.6 pp, SE = 1.4 pp). None of these estimates are robustly different from zero, and the differences across groups are not statistically significant.

\textbf{By automation exposure:} Figure 6 shows estimates by occupation automation quartile among employed workers. All four quartiles show estimates close to zero with overlapping confidence intervals. We find no evidence that EITC effects differ by automation exposure.

\subsection{Additional Outcomes}

We examine labor force participation, finding a similar null result: the discontinuity is -3.5 pp (SE = 0.66 pp) in the linear specification but reflects slope changes rather than a true jump. Among employed workers, we find no effect on full-time status (-0.7 pp, SE = 0.84 pp) or hours worked (-7.7 hours, SE = 3.9).

\section{Discussion}

\subsection{Interpreting the Null Result}

Our failure to detect a positive employment effect at the EITC eligibility threshold is consistent with prior work and has several possible explanations:

\textbf{Small credit amount:} The maximum childless EITC (\$600) provides weak work incentives. At the federal minimum wage, \$600 represents approximately 80 hours of work---a modest income supplement that may not be salient enough to affect labor supply decisions.

\textbf{Low take-up:} Many childless workers are unaware they qualify for the EITC. The IRS estimates that take-up rates are lower for childless filers than for families with children. If workers don't know they're eligible, eligibility cannot affect behavior.

\textbf{High baseline employment:} Young childless workers already have relatively high labor force participation. Those not working may be students, disabled, or facing other barriers that the EITC cannot address.

\textbf{Methodological limitations:} Age measured in years provides only 7 support points in our bandwidth, limiting our ability to detect small effects. A more precise measure (age in months or days) might reveal effects that our analysis misses.

\subsection{Policy Implications}

Our null findings have implications for EITC expansion proposals:

\textbf{Age threshold:} Simply lowering the eligibility age from 25 to 19 may have limited employment effects if the credit amount remains small. Our estimates suggest that younger workers would respond similarly to current eligible workers---which is to say, not much.

\textbf{Credit amount:} Increasing the maximum credit for childless workers may be necessary to generate meaningful labor supply responses. The American Rescue Plan's temporary \$1,500 credit was roughly three times the normal amount and warrants evaluation.

\textbf{Awareness:} Improving awareness of the childless EITC could increase take-up, potentially generating employment effects. Outreach campaigns and simplified filing may be valuable complements to benefit expansion.

\subsection{Limitations}

Several limitations warrant mention:

\textbf{Discrete running variable:} With age measured in years, we have limited ability to implement standard RDD inference procedures. Our results rely on parametric assumptions about the age-employment relationship.

\textbf{Post-treatment occupation:} Our automation heterogeneity analysis conditions on current occupation, which could respond to EITC eligibility. This limits our ability to identify how pre-existing automation exposure moderates EITC effects.

\textbf{Confounding policies:} Other policies may change at age 25, including some state minimum wage laws and health insurance provisions. While we control for state fixed effects, we cannot fully rule out confounding.

\textbf{External validity:} Our sample covers 2015-2024. Results may differ for earlier or later periods, or under different macroeconomic conditions.

\section{Conclusion}

We examine whether EITC eligibility affects employment for childless workers using the sharp age-25 threshold. Using real microdata from the CPS ASEC (2015-2024), we find no robust evidence of a positive employment effect. Employment increases smoothly with age, consistent with normal age-employment dynamics rather than a discrete EITC effect.

Our null findings suggest that the current childless EITC---with its small credit amount and low awareness---has limited effects on labor supply. Policy proposals to expand the childless EITC should consider not just extending eligibility to younger workers, but also substantially increasing the credit amount and improving take-up. Without such reforms, eligibility expansion alone may produce disappointing results.

\newpage

\section*{References}

\begin{description}
\item Acemoglu, D., \& Restrepo, P. (2020). Robots and jobs: Evidence from US labor markets. \textit{Journal of Political Economy}, 128(6), 2188-2244.

\item Autor, D. H., Dorn, D., \& Hanson, G. H. (2015). Untangling trade and technology: Evidence from local labour markets. \textit{Economic Journal}, 125(584), 621-646.

\item Bastian, J., \& Jones, M. R. (2022). Do EITC expansions pay for themselves? Effects on tax revenue and government transfers. \textit{Journal of Public Economics}, 208, 104604.

\item Calonico, S., Cattaneo, M. D., \& Titiunik, R. (2014). Robust nonparametric confidence intervals for regression-discontinuity designs. \textit{Econometrica}, 82(6), 2295-2326.

\item Cattaneo, M. D., Idrobo, N., \& Titiunik, R. (2019). \textit{A Practical Introduction to Regression Discontinuity Designs: Foundations}. Cambridge University Press.

\item Eissa, N., \& Liebman, J. B. (1996). Labor supply response to the earned income tax credit. \textit{Quarterly Journal of Economics}, 111(2), 605-637.

\item Frey, C. B., \& Osborne, M. A. (2017). The future of employment: How susceptible are jobs to computerisation? \textit{Technological Forecasting and Social Change}, 114, 254-280.

\item Imbens, G., \& Lemieux, T. (2008). Regression discontinuity designs: A guide to practice. \textit{Journal of Econometrics}, 142(2), 615-635.

\item Kolesár, M., \& Rothe, C. (2018). Inference in regression discontinuity designs with a discrete running variable. \textit{American Economic Review}, 108(8), 2277-2304.

\item Lee, D. S., \& Lemieux, T. (2010). Regression discontinuity designs in economics. \textit{Journal of Economic Literature}, 48(2), 281-355.

\item Meyer, B. D., \& Rosenbaum, D. T. (2001). Welfare, the earned income tax credit, and the labor supply of single mothers. \textit{Quarterly Journal of Economics}, 116(3), 1063-1114.
\end{description}

\newpage

\appendix
\section*{Appendix: Tables and Figures}

% Table 1: Summary Statistics
\begin{table}[H]
\centering
\caption{Summary Statistics}
\label{tab:summary}
\begin{threeparttable}
\begin{tabular}{lccc}
\toprule
& Full Sample & Age 22-24 & Age 25-28 \\
& & (Ineligible) & (Eligible) \\
\midrule
N & 100,182 & 46,785 & 53,397 \\
Employment Rate & 0.756 & 0.708 & 0.799 \\
Labor Force Participation & 0.802 & 0.757 & 0.842 \\
Female & 0.451 & 0.461 & 0.443 \\
White (non-Hispanic) & 0.554 & 0.560 & 0.549 \\
Black & 0.137 & 0.131 & 0.142 \\
Hispanic & 0.202 & 0.201 & 0.203 \\
High School Graduate & 0.951 & 0.949 & 0.952 \\
Some College & 0.687 & 0.677 & 0.695 \\
College Graduate & 0.373 & 0.348 & 0.395 \\
Not Married & 0.875 & 0.893 & 0.860 \\
Metropolitan Area & 0.897 & 0.895 & 0.899 \\
Mean Age & 24.8 & 23.0 & 26.4 \\
\bottomrule
\end{tabular}
\begin{tablenotes}
\small
\item \textit{Notes:} Data from CPS ASEC 2015-2024 via IPUMS. Sample restricted to childless adults ages 22-28. Statistics weighted using ASEC survey weights.
\end{tablenotes}
\end{threeparttable}
\end{table}

% Table 2: Main Results
\begin{table}[H]
\centering
\caption{Effect of EITC Eligibility on Employment: RDD Estimates}
\label{tab:main}
\begin{threeparttable}
\begin{tabular}{lcccc}
\toprule
& (1) & (2) & (3) & (4) \\
& Linear & Year FE & Year+State FE & Quadratic \\
\midrule
EITC Eligible & -0.0307*** & -0.0304*** & -0.0291*** & -0.0026 \\
& (0.0071) & (0.0071) & (0.0071) & (0.0184) \\
\\
Year FE & No & Yes & Yes & Yes \\
State FE & No & No & Yes & Yes \\
Polynomial & Linear & Linear & Linear & Quadratic \\
\\
Observations & 100,182 & 100,182 & 100,182 & 100,182 \\
\bottomrule
\end{tabular}
\begin{tablenotes}
\small
\item \textit{Notes:} Regression discontinuity estimates. Running variable is age in years, centered at 25. Heteroskedasticity-robust standard errors in parentheses. The negative coefficients in linear specifications reflect the change in slope of the age-employment relationship, not a true discontinuity.
\item * p $<$ 0.10, ** p $<$ 0.05, *** p $<$ 0.01
\end{tablenotes}
\end{threeparttable}
\end{table}

% Table 3: Robustness
\begin{table}[H]
\centering
\caption{Robustness Checks}
\label{tab:robust}
\begin{threeparttable}
\begin{tabular}{lcc}
\toprule
Specification & Estimate & SE \\
\midrule
\multicolumn{3}{l}{\textit{Panel A: Bandwidth Sensitivity}} \\
$\pm$2 years (N = 71,702) & -0.0195 & (0.0098) \\
$\pm$3 years (N = 100,182) & -0.0291 & (0.0071) \\
\\
\multicolumn{3}{l}{\textit{Panel B: Placebo Cutoffs}} \\
Age 23 (Placebo) & 0.0540 & (0.0058) \\
Age 24 (Placebo) & -0.0140 & (0.0100) \\
Age 25 (Actual) & -0.0291 & (0.0071) \\
Age 26 (Placebo) & -0.0253 & (0.0062) \\
Age 27 (Placebo) & -0.0369 & (0.0059) \\
\bottomrule
\end{tabular}
\begin{tablenotes}
\small
\item \textit{Notes:} All specifications include year and state fixed effects with heteroskedasticity-robust standard errors. Significant effects at placebo cutoffs suggest estimates reflect curvature in the age-employment relationship rather than true discontinuities.
\end{tablenotes}
\end{threeparttable}
\end{table}

% Table 4: Heterogeneity
\begin{table}[H]
\centering
\caption{Heterogeneous Effects by Education Level}
\label{tab:hetero}
\begin{threeparttable}
\begin{tabular}{lccc}
\toprule
& College & Some & HS Graduate \\
& Graduate & College & or Less \\
\midrule
EITC Eligible & -0.0163 & -0.0391 & 0.0060 \\
& (0.0114) & (0.0126) & (0.0135) \\
\\
N & 35,189 & 32,889 & 32,104 \\
\bottomrule
\end{tabular}
\begin{tablenotes}
\small
\item \textit{Notes:} RDD estimates by education level. All specifications include year and state fixed effects with heteroskedasticity-robust standard errors. None of the estimates are robustly different from zero.
\end{tablenotes}
\end{threeparttable}
\end{table}

\newpage

% Figures
\begin{figure}[H]
\centering
\includegraphics[width=0.9\textwidth]{figures/fig1_rdd_main.pdf}
\caption{Employment Rate by Age: No Discontinuity at EITC Eligibility Threshold}
\label{fig:main}
\begin{figurenotes}
Points show weighted employment rates with 95\% confidence intervals. Lines show linear fits on either side of the age-25 cutoff. Employment increases smoothly with age; no visible jump at the eligibility threshold.
\end{figurenotes}
\end{figure}

\begin{figure}[H]
\centering
\includegraphics[width=0.9\textwidth]{figures/fig2_balance.pdf}
\caption{Covariate Balance at Age-25 Cutoff}
\label{fig:balance}
\begin{figurenotes}
Differences in covariate means between workers aged 25 versus 24. Red points indicate statistically significant differences (p $<$ 0.05). Imbalances in education and marital status likely reflect life-cycle transitions.
\end{figurenotes}
\end{figure}

\begin{figure}[H]
\centering
\includegraphics[width=0.8\textwidth]{figures/fig3_placebo.pdf}
\caption{Placebo Cutoff Tests}
\label{fig:placebo}
\begin{figurenotes}
RDD estimates at alternative age cutoffs. Significant effects at placebo ages suggest that ``effects'' reflect curvature in the age-employment relationship rather than true policy discontinuities.
\end{figurenotes}
\end{figure}

\begin{figure}[H]
\centering
\includegraphics[width=0.8\textwidth]{figures/fig4_bandwidth.pdf}
\caption{Bandwidth Sensitivity}
\label{fig:bw}
\begin{figurenotes}
RDD estimates across different bandwidth choices. Error bars show 95\% confidence intervals. Estimates are stable but reflect slope changes rather than discontinuities.
\end{figurenotes}
\end{figure}

\begin{figure}[H]
\centering
\includegraphics[width=0.8\textwidth]{figures/fig5_heterogeneity_education.pdf}
\caption{Heterogeneous Effects by Education Level}
\label{fig:educ}
\begin{figurenotes}
RDD estimates by educational attainment. Error bars show 95\% confidence intervals. No education group shows a significant positive effect.
\end{figurenotes}
\end{figure}

\begin{figure}[H]
\centering
\includegraphics[width=0.8\textwidth]{figures/fig6_heterogeneity_automation.pdf}
\caption{Heterogeneous Effects by Automation Exposure}
\label{fig:auto}
\begin{figurenotes}
RDD estimates among employed workers by occupation automation quartile. Error bars show 95\% confidence intervals. Note: occupation is potentially post-treatment. No quartile shows a significant effect.
\end{figurenotes}
\end{figure}

\begin{figure}[H]
\centering
\includegraphics[width=0.9\textwidth]{figures/fig7_trends.pdf}
\caption{Employment Rate Trends by Eligibility Group}
\label{fig:trends}
\begin{figurenotes}
Employment rates over time for workers below (age 22-24) and at/above (age 25-28) the EITC eligibility threshold. The gap between groups is stable over time, consistent with age effects rather than EITC effects.
\end{figurenotes}
\end{figure}

\end{document}
