\documentclass[12pt]{article}

% UTF-8 encoding and fonts
\usepackage[utf8]{inputenc}
\usepackage[T1]{fontenc}
\usepackage{lmodern}

% Page setup
\usepackage[margin=1in]{geometry}
\usepackage{setspace}
\onehalfspacing

% Typography
\usepackage{microtype}

% Math and symbols
\usepackage{amsmath,amssymb}

% Graphics
\usepackage{graphicx}
\usepackage{float}
\usepackage{subcaption}

% Tables
\usepackage{booktabs}
\usepackage{array}
\usepackage{multirow}
\usepackage{threeparttable}
\usepackage{longtable}
\usepackage{pdflscape}
\usepackage{siunitx}
\sisetup{detect-all=true, group-separator={,}, group-minimum-digits=4}

% Bibliography
\usepackage{natbib}
\bibliographystyle{aer}

% Hyperlinks
\usepackage{hyperref}
\hypersetup{
    colorlinks=true,
    linkcolor=blue,
    citecolor=blue,
    urlcolor=blue
}
\usepackage[nameinlink,noabbrev]{cleveref}

% Captions
\usepackage{caption}
\captionsetup{font=small,labelfont=bf}

% Section formatting
\usepackage{titlesec}
\titleformat{\section}{\large\bfseries}{\thesection.}{0.5em}{}
\titleformat{\subsection}{\normalsize\bfseries}{\thesubsection}{0.5em}{}

% Custom commands
\newcommand{\E}{\mathbb{E}}
\newcommand{\Var}{\text{Var}}

\title{Roads, Crashes, and Substances: A Geocoded Atlas of Western US Traffic Fatalities}
\author{APEP Autonomous Research\thanks{Autonomous Policy Evaluation Project. Correspondence: scl@econ.uzh.ch} \\ @anonymous}
\date{\today}

\begin{document}

\maketitle

\begin{abstract}
\noindent
We construct and document a novel integrated dataset combining fatal traffic crashes in Western US states from the Fatality Analysis Reporting System (FARS) with OpenStreetMap road network attributes and marijuana legalization policy timing. The resulting dataset of approximately 70,000 crashes (of which roughly 90\% have valid geocoding) enables unprecedented granularity in studying the geography of impaired driving. We document three key patterns: (1) among fatal crashes with any drug finding reported in 2018--2019, the share with THC detected is approximately 20\% in legalized states versus approximately 10\% in comparison states (illegal during our study period); (2) THC detection rates show visible discontinuities at several state borders in 2018--2019, with patterns varying across border pairs (motivating spatial RDD designs); (3) alcohol involvement exhibits a secular decline from approximately 40\% in the early 2000s to under 30\% in recent years. Our maps demonstrate crash-level precision suitable for spatial regression discontinuity designs at policy borders. We provide complete replication code to enable researchers to extend this analysis to additional states, time periods, and policy questions. This data infrastructure paper establishes a foundation for rigorous causal research on marijuana policy and traffic safety.
\end{abstract}

\vspace{1em}
\noindent\textbf{JEL Codes:} I18, K32, R41 \\
\noindent\textbf{Keywords:} traffic fatalities, marijuana legalization, geocoded data, FARS, spatial analysis

\newpage

\section{Introduction}

Traffic crashes kill approximately 40,000 Americans annually, making motor vehicle accidents a leading cause of death, particularly among young adults \citep{nhtsa2023}. Understanding the role of substance impairment in these fatalities is critical for effective traffic safety policy. The past decade has witnessed dramatic policy shifts as states legalized recreational marijuana, raising important questions about the relationship between cannabis availability and traffic safety.

Despite the policy significance of this question, empirical research faces substantial data limitations. Most existing studies rely on state-level aggregated crash counts, which cannot exploit the geographic precision available in administrative data \citep{anderson2013,hansen2015}. The Fatality Analysis Reporting System (FARS), maintained by the National Highway Traffic Safety Administration, provides latitude and longitude coordinates for each fatal crash, yet this spatial dimension remains largely unexploited in the literature. Similarly, detailed information on road characteristics, speed limits, and highway classifications could inform analyses but requires integration with separate geographic information systems.

This paper addresses these gaps by constructing and documenting a comprehensive integrated dataset suitable for high-resolution spatial analysis of impaired driving. We combine three data sources: (1) FARS crash records with geocoded locations for 2001--2005 and 2016--2019; (2) OpenStreetMap road network attributes including highway type, speed limits, and lane counts; and (3) marijuana legalization policy timing at the state-day level. The final dataset covers approximately 70,000 fatal crashes (of which 90\% are geocoded) in Western marijuana states and their comparison neighbors (states that were illegal during our study period).

Our contribution is methodological and descriptive rather than causal. We demonstrate the research potential of this integrated dataset through extensive visualization and pattern documentation. The core finding is that crash-level geocoded data reveal striking spatial patterns that aggregate data obscure. At marijuana legalization borders, THC-positive rates in 2018--2019 differ sharply between legal and illegal jurisdictions---differences that emerge precisely at the border crossing. These patterns motivate spatial regression discontinuity designs that exploit the sharp policy contrast at state borders.

We document several additional patterns of interest. First, among crashes with drug findings reported, THC detection rates in legalizing states exceed 20\% compared to approximately 10\% in comparison states (illegal during study period) (THC identified via drug name matching, which is reliable from 2018 onward). Second, alcohol involvement continues its long-term secular decline from 40\% to 28\%, with limited evidence of substitution between alcohol and cannabis. Third, substance involvement varies dramatically by time of day, with alcohol peaking in late-night hours and THC showing a flatter distribution.

The paper proceeds as follows. Section 2 describes our data sources and the integration methodology in detail. Section 3 presents national-level descriptive patterns including temporal trends and geographic distributions. Section 4---the core showcase---presents high-resolution zoom maps demonstrating granularity at state borders, highway corridors, and metropolitan areas. Section 5 documents substance involvement patterns in detail. Section 6 analyzes patterns at marijuana legalization borders. Section 7 discusses data quality limitations, particularly the substantial missingness in drug testing. Section 8 outlines research applications including spatial RDD and difference-in-differences designs that our dataset enables. Section 9 concludes.

Our replication package includes all code necessary to reproduce the dataset from raw FARS downloads, enabling researchers to extend the analysis to additional states, incorporate newer FARS releases, or adapt the methodology for other policy applications.

\section{Data Sources and Integration}

This section describes each data source, our integration methodology, and the resulting analysis dataset.

\subsection{FARS: The Universe of Fatal Crashes}

The Fatality Analysis Reporting System (FARS) is a census of all motor vehicle crashes in the United States resulting in at least one fatality within 30 days of the crash. Maintained by NHTSA since 1975, FARS provides detailed information on crash circumstances, vehicle characteristics, and person-level outcomes including drug and alcohol test results.

We use FARS data from two time periods: 2001--2005 (pre-legalization baseline, before any state had legalized recreational marijuana) and 2016--2019 (recent period, after early-wave states legalized). Note that within 2016--2019, states legalized at different times (see Table 1), so this is not a uniform post-treatment period. This two-period design allows comparison of alcohol involvement trends over fifteen years. The geocoding quality of FARS improved substantially over this period. By 2001, approximately 85\% of crashes have valid coordinates; by 2016, this exceeds 95\%.

For each crash, FARS provides the accident file (crash-level characteristics including location, time, weather, and road conditions), the person file (individual-level data on drivers, passengers, pedestrians, and cyclists including injury severity and substance test results), and the drugs file (detailed drug test results with specific drug codes). We merge these files at the crash level, creating variables for THC involvement, alcohol involvement, and poly-substance use.

The key substance variables require careful interpretation. The FARS drugs file records drug \textbf{findings} at the person level---it reports substances detected, not comprehensive test panels. This means we can observe THC-positive results but cannot definitively identify THC-negative tests (absence of a THC record may mean not tested, tested but unreported, or tested negative). We restrict our analysis to \textbf{drivers only} (excluding passengers, pedestrians, and cyclists). We define:
\begin{itemize}
    \item \textbf{Crash with drug record:} A crash where at least one driver has any record in the FARS drugs file (indicating some drug finding was reported)
    \item \textbf{Crash THC-positive:} A crash where at least one driver has a positive THC result, identified via text pattern matching in the drug result name field (e.g., ``Tetrahydrocannabinols (THC)'', ``DELTA 9'')
\end{itemize}
\textbf{Important limitation:} Because FARS reports findings rather than test panels, our ``THC-positive rate among crashes with drug records'' should be interpreted as the share of crashes with reported drug findings that include THC---not as the share of drug-tested crashes that are THC-positive. This limitation is inherent to FARS and affects all research using these data. THC detection is only reliable from 2018 onward when comprehensive drug name data became available. For alcohol, we use the FARS accident file \texttt{drunk\_dr} variable, which records the number of drivers with alcohol impairment.

A critical limitation is that drug reporting is not universal. States vary substantially in their testing and reporting protocols, and the share of crashes with drug records differs by crash severity, driver survival, and other factors. We document these patterns extensively in Section 7. The key implication is that THC detection rates among crashes with drug records may not represent true THC-positive rates among all crashes due to selection into reporting.

\subsection{OpenStreetMap Road Network}

OpenStreetMap (OSM) is a collaborative mapping project providing detailed road network data for the entire United States. We extract road networks for key geographic regions using the osmnx Python package \citep{boeing2017}.

For each road segment, OSM provides attributes including:
\begin{itemize}
    \item Highway type (motorway, trunk, primary, secondary, tertiary, residential)
    \item Speed limit (where tagged)
    \item Number of lanes
    \item Road name and route number
\end{itemize}

We snap each FARS crash to the nearest road segment using spatial join operations, requiring crashes to fall within 200 meters of a road. This threshold balances matching accuracy against the inherent imprecision in FARS coordinates. We record the snap distance for each crash to enable robustness checks excluding poorly matched crashes.

The OSM highway classification maps approximately to functional road classes:
\begin{itemize}
    \item \textbf{Motorway}: Interstate highways
    \item \textbf{Trunk}: US highways
    \item \textbf{Primary}: State highways
    \item \textbf{Secondary/Tertiary}: County and local roads
    \item \textbf{Residential}: Neighborhood streets
\end{itemize}

Speed limit data in OSM is incomplete, with coverage varying by state and road type. We use speed limits where available but do not impute missing values.

\textbf{Temporal validity caveat:} OSM represents contemporary road network conditions at the time of extraction (2024). Road attributes such as speed limits, lane counts, and highway classifications may have changed since 2001--2005. We therefore recommend using OSM-linked road characteristics only for the 2016--2019 period, where temporal mismatch is minimal. For 2001--2005 crashes, researchers should treat OSM attributes as approximate or restrict analysis to time-invariant characteristics (e.g., highway vs.\ local road classification, which changes rarely).

\subsection{Marijuana Policy Data}

We compile marijuana legalization dates from the Harvard Dataverse marijuana policy database, supplemented by manual verification against news archives and state government sources \citep{pacula2015}. For each state, we record:

\begin{itemize}
    \item Medical marijuana effective date
    \item Recreational ballot initiative date
    \item Recreational possession effective date
    \item Retail sales opening date
\end{itemize}

Table \ref{tab:policy} presents the legalization timeline for Western states. Colorado and Washington legalized recreational marijuana by ballot initiative in November 2012, with possession becoming legal in December 2012 and retail sales opening in January and July 2014, respectively. Oregon and Alaska followed in 2014--2015, then California and Nevada in 2016--2017. Arizona, Montana, and New Mexico legalized more recently (2020--2021).

Our comparison states---Wyoming, Nebraska, Kansas, Idaho, and Utah---maintained prohibition throughout the study period. Arizona, Montana, and New Mexico legalized recreational marijuana after our sample ends (2020--2021) and are therefore also classified as comparison states for our 2001--2019 analysis. \textbf{Comparison group definition:} Throughout this paper, ``comparison states'' refers to all eight states that were illegal during our study period (WY, NE, KS, ID, UT, AZ, MT, NM), creating sharp policy discontinuities at borders with legal states.

\begin{table}[H]
\centering
\caption{Marijuana Legalization Timeline: Western States}
\begin{threeparttable}
\begin{tabular}{llllc}
\toprule
State & Abbr & Legalization Date & Retail Opens & Category \\
\midrule
Colorado & CO & 2012-12-10 & 2014-01-01 & Pioneer (2012) \\
Washington & WA & 2012-12-06 & 2014-07-08 & Pioneer (2012) \\
Oregon & OR & 2015-07-01 & 2015-10-01 & Wave 2 (2014-15) \\
Alaska & AK & 2015-02-24 & 2016-10-29 & Wave 2 (2014-15) \\
California & CA & 2016-11-09 & 2018-01-01 & Wave 3 (2016-17) \\
Nevada & NV & 2017-01-01 & 2017-07-01 & Wave 3 (2016-17) \\
Wyoming & WY & -- & -- & Illegal (study period) \\
Nebraska & NE & -- & -- & Illegal (study period) \\
Kansas & KS & -- & -- & Illegal (study period) \\
Idaho & ID & -- & -- & Illegal (study period) \\
Utah & UT & -- & -- & Illegal (study period) \\
Arizona & AZ & 2020-11-30 & 2021-01-22 & Wave 4 (2020+) \\
Montana & MT & 2021-01-01 & 2022-01-01 & Wave 4 (2020+) \\
New Mexico & NM & 2021-06-29 & 2022-04-01 & Wave 4 (2020+) \\
\bottomrule
\end{tabular}
\begin{tablenotes}[flushleft]
\small
\item Notes: Legalization date is when recreational possession became legal. Retail date is when legal sales to recreational users began (Oregon: early sales at medical dispensaries; others: licensed retail). -- indicates recreational marijuana not legalized during the study period. AZ, MT, and NM legalized after our sample period (2019) and are included as comparison states.
\end{tablenotes}
\end{threeparttable}
\label{tab:policy}
\end{table}

\subsection{Integration Pipeline}

Our integration pipeline proceeds as follows:

\begin{enumerate}
    \item Download FARS national files for 2001--2005 and 2016--2019 from NHTSA
    \item Filter to Western focus states (14 states)
    \item Merge accident, person, and drugs files
    \item Create crash-level substance involvement indicators
    \item Convert coordinates to spatial format (EPSG:5070 Albers Equal Area projection)
    \item Download state boundaries from Census TIGER
    \item Compute distance from each crash to nearest marijuana legalization border. \textbf{Important:} This variable uses a \textbf{fixed 2018--2019 legal/illegal classification} and is therefore only valid for crashes in 2018--2019 (when policy status is stable within our sample). For 2016--2017 crashes, policy status changed within that period (CA, NV), and for 2001--2005 crashes, no recreational-legal states existed. We set \texttt{dist\_to\_border\_km = NA} for all crashes outside 2018--2019.
    \item Download OSM road networks for key regions
    \item Snap crashes to nearest road segments
    \item Merge policy timing variables based on crash date and state
\end{enumerate}

The resulting analysis dataset contains approximately 70,000 fatal crashes, of which roughly 63,000 (90\%) have valid geocoding. Among geocoded crashes, those within 200m of an OSM road are snapped to road segments and receive OSM-derived attributes (highway type, speed limit). Crashes outside 200m of any road or lacking geocoding have missing road attributes. All crashes have FARS-derived substance involvement indicators and policy exposure variables; spatial analyses (border distances, maps) use the geocoded subset.

\subsection{Summary Statistics}

Table \ref{tab:summary} presents summary statistics for our analysis dataset, separately for the two time periods and by state legalization status (as of 2018--2019).

\begin{table}[H]
\centering
\caption{Summary Statistics: Fatal Crashes in Western States}
\label{tab:summary}
\begin{threeparttable}
\begin{tabular}{lcccc}
\toprule
& \multicolumn{2}{c}{2001--2005} & \multicolumn{2}{c}{2016--2019} \\
\cmidrule(lr){2-3} \cmidrule(lr){4-5}
& Legal & Comparison & Legal & Comparison \\
\midrule
\textbf{Panel A: Crash Characteristics} \\
Total crashes & 21,847 & 16,203 & 18,624 & 13,291 \\
Annual average & 4,369 & 3,241 & 4,656 & 3,323 \\
Crashes with geocoding (\%) & 86.2 & 84.7 & 95.8 & 94.1 \\
\addlinespace
\textbf{Panel B: Substance Involvement} \\
Alcohol-involved (\%) & 40.1 & 33.5 & 28.4 & 24.8 \\
\addlinespace
\textbf{Panel D: Crash Context (FARS variables)} \\
Interstate/expressway (\%)$^{\ddagger}$ & 18.3 & 22.1 & 17.9 & 21.8 \\
Nighttime crashes (\%)$^{\ddagger}$ & 34.2 & 33.8 & 35.1 & 34.3 \\
\bottomrule
\end{tabular}
\vspace{0.5em}

\noindent\textit{Panel C: Drug Data (2018--2019 only---separate time window)}
\vspace{0.3em}

\begin{tabular}{lcc}
\toprule
& \multicolumn{2}{c}{2018--2019} \\
\cmidrule(lr){2-3}
& Legal & Comparison \\
\midrule
Crashes with any drug record (\%) & 33.8 & 31.4 \\
THC detected (among w/ records) (\%) & 19.1 & 10.0 \\
\bottomrule
\end{tabular}
\begin{tablenotes}[flushleft]
\small
\item Notes: Legal states = CO, WA, OR, AK, CA, NV. Comparison states = WY, NE, KS, ID, UT, AZ, MT, NM (illegal during study period). \textbf{Timing caveat for 2016--2019:} CA legalized Nov 2016, NV in Jan 2017; their 2016 crashes are technically pre-legalization but included in ``Legal'' for simplicity. Alcohol involvement from FARS \texttt{drunk\_dr} variable. Panel C uses 2018--2019 only because text-based THC identification (drug name matching) is only reliable from 2018. \\
$^\ddagger$Panel D uses FARS variables: \texttt{func\_sys} for Interstate/expressway (codes 1--2), \texttt{lgt\_cond} for nighttime (dark conditions).
\end{tablenotes}
\end{threeparttable}
\end{table}

Several patterns emerge from Table \ref{tab:summary}. First, fatal crash counts are relatively stable across time periods, with legalized states having more crashes due to larger populations (California alone accounts for roughly 40\% of legal-state crashes). Second, alcohol involvement declined substantially from approximately 40\% in 2001--2005 to under 30\% in 2016--2019, consistent with national trends. Third, among crashes with drug records in 2018--2019, THC detection is approximately twice as common in legalized states (19\%) compared to comparison states (10\%). Fourth, road characteristics are similar across state groups, with about 20\% of crashes on Interstate highways and 35\% occurring at night.

\section{Descriptive Patterns: National Overview}

\subsection{Temporal Trends}

Figure \ref{fig:annual} presents annual fatal crash counts for Western states in our two sample periods: 2001--2005 (before any state legalization) and 2016--2019 (recent period). Total crashes averaged approximately 7,600 annually in 2001--2005 and 7,900 annually in 2016--2019, showing relative stability across this fifteen-year span.

\begin{figure}[H]
\centering
\includegraphics[width=0.9\textwidth]{figures/fig02_annual_crashes.pdf}
\caption{Fatal Traffic Crashes by Year, Western States}
\label{fig:annual}
\end{figure}

\subsection{Substance Involvement Trends}

Figures \ref{fig:thc} and \ref{fig:alcohol} present THC detection rates (among crashes with drug findings) and alcohol-involved rates over time, separately for states that legalized recreational marijuana and comparison states (illegal during our study period).

THC detection rates (at the crash level, measuring whether any driver has a THC-positive record) are substantially elevated in states that legalized recreational marijuana. Using drug data available from 2018, we observe that among crashes with any drug finding reported, THC is detected in approximately 20\% of such crashes in legalizing states compared to 10\% in comparison states (illegal during study period). This cross-sectional difference is consistent with marijuana legalization increasing THC involvement in fatal crashes, though pre-legalization trend data is limited by drug data availability.

Alcohol involvement exhibits a secular decline from approximately 40\% in the early 2000s to under 28\% by 2019. This decline reflects decades of drunk driving policy interventions including 0.08 BAC laws, administrative license revocation, and ignition interlock requirements. The parallel trends between legalizing and non-legalizing states suggest alcohol declines are driven by national factors rather than marijuana-specific substitution.

\begin{figure}[H]
\centering
\includegraphics[width=0.9\textwidth]{figures/fig03_thc_rate_over_time.pdf}
\caption{THC-Positive Rate Among Crashes with Any Drug Finding Reported, 2018--2019. Numerator = crashes with any THC-positive driver finding; denominator = crashes with any driver drug record in FARS drugs file.}
\label{fig:thc}
\end{figure}

\begin{figure}[H]
\centering
\includegraphics[width=0.9\textwidth]{figures/fig04_alcohol_rate_over_time.pdf}
\caption{Alcohol-Involved Fatal Crashes Over Time. Note: ``Legalized States'' (CO, WA, OR, AK, CA, NV) and ``Comparison States'' (WY, NE, KS, ID, UT, AZ, MT, NM) use fixed groupings based on eventual legalization status, not crash-date legal status. For 2016, CA and NV were not yet legal; their crashes are included in ``Legalized States'' for consistent grouping across years.}
\label{fig:alcohol}
\end{figure}

\subsection{Time-of-Day Patterns}

Figure \ref{fig:hour} shows crash counts by hour of day, colored by substance involvement category. Fatal crashes peak during evening rush hours (4--7 PM) and late night (10 PM--2 AM). Substance-involved crashes are disproportionately concentrated in late-night hours: over 60\% of crashes between midnight and 3 AM involve alcohol, compared to under 20\% during daytime hours. THC involvement shows a flatter distribution, elevated at night but also present throughout the day.

\begin{figure}[H]
\centering
\includegraphics[width=0.9\textwidth]{figures/fig05_crashes_by_hour.pdf}
\caption{Fatal Crashes by Hour of Day and Substance Involvement, 2018--2019}
\label{fig:hour}
\end{figure}

\section{Showcase: Zooming In}

This section demonstrates the granularity of our geocoded data through progressively detailed maps of key regions.

\subsection{Colorado-Wyoming Border}

Figure \ref{fig:cowy} presents fatal crashes in the Colorado-Wyoming border region from 2018--2019 (when drug name data is available), colored by THC test result. Colorado legalized recreational marijuana in December 2012; Wyoming maintains prohibition. The state border running east-west creates a sharp policy discontinuity.

Visual inspection reveals several patterns. First, crashes cluster along major highways: I-25 running north-south, I-80 running east-west through Wyoming, and I-70 in Colorado. Second, THC-positive crashes (red points) appear more common on the Colorado side, while crashes with no THC positive recorded (gray points) predominate in Wyoming. Third, the border itself is clearly visible as a line separating the two patterns.

This visualization motivates the spatial regression discontinuity designs we discuss in Section 8: researchers can compare outcomes on either side of the border, controlling for distance, to estimate causal effects of marijuana legalization.

\begin{figure}[H]
\centering
\includegraphics[width=\textwidth]{figures/fig10_co_wy_border.pdf}
\caption{Fatal Crashes at the Colorado-Wyoming Border, 2018--2019. Points show crashes with any driver drug record in FARS. ``THC Finding Present'' = THC-positive finding recorded for any driver; ``No THC Finding'' = drug record present but no THC detected.}
\label{fig:cowy}
\end{figure}

\subsection{I-25 Corridor Detail}

Figure \ref{fig:i25} zooms further into the I-25 corridor from Denver to the Wyoming border. Each point represents a single fatal crash snapped to the highway. The clustering of crashes at specific locations---interchanges, curves, and areas with high traffic volume---is apparent. This level of detail enables researchers to control for road segment characteristics in analyzing substance involvement.

\begin{figure}[H]
\centering
\includegraphics[width=0.7\textwidth]{figures/fig11_i25_corridor.pdf}
\caption{Fatal Crashes Along the I-25 Corridor, Denver to Wyoming Border, 2018--2019}
\label{fig:i25}
\end{figure}

\subsection{Denver Metropolitan Area}

Figure \ref{fig:denver} presents the Denver metropolitan area, showing the urban crash distribution. Fatal crashes occur throughout the metro area but cluster along major arterials and at intersections. The density of crashes in urban areas enables analysis of urban-specific patterns including proximity to commercial areas.

\begin{figure}[H]
\centering
\includegraphics[width=0.9\textwidth]{figures/fig12_denver_metro.pdf}
\caption{Fatal Crashes in the Denver Metropolitan Area, 2018--2019}
\label{fig:denver}
\end{figure}

\subsection{Rural Border: Oregon-Idaho}

Figure \ref{fig:orid} shows the Oregon-Idaho border region, demonstrating patterns in rural areas. Unlike the urban density of Denver, rural crashes are sparse and cluster along the few major highways. I-84 running through the Columbia River Gorge shows a clear concentration of crashes. Crashes are colored by state marijuana status (Oregon legalized in 2015; Idaho maintains prohibition), illustrating the sharp policy boundary in this rural corridor.

\begin{figure}[H]
\centering
\includegraphics[width=\textwidth]{figures/fig14_or_id_border.pdf}
\caption{Fatal Crashes at the Oregon-Idaho Border, 2016--2019}
\label{fig:orid}
\end{figure}

\section{Substance Involvement Patterns}

\subsection{Alcohol Involvement}

Alcohol involvement in fatal crashes is measured using the FARS \texttt{drunk\_dr} variable, which records the number of alcohol-impaired drivers in each crash. This measure captures law enforcement's determination of alcohol involvement based on BAC tests, officer observation, or other evidence. Prior research documents that fatal alcohol-related crashes typically involve heavily impaired drivers with BAC levels well above the 0.08 legal limit \citep{nhtsa2023}.

\subsection{Poly-Substance Involvement}

Figure \ref{fig:poly} shows the breakdown of substance involvement for 2018--2019, when comprehensive drug name data is available. Among crashes with driver drug records (crashes where at least one driver has a record in the FARS drugs file), we combine THC status (from the drugs file) with alcohol involvement (from the \texttt{drunk\_dr} variable, which records law enforcement's determination of alcohol impairment). The share involving THC only is substantial in legalizing states (11\% vs 6\% in comparison states). Crashes involving both THC and alcohol represent approximately 8\% in legalized states and 4\% in comparison states.

The presence of meaningful poly-substance involvement (both THC and alcohol) suggests that some drivers use multiple impairing substances. The relatively low poly-substance rate compared to single-substance involvement indicates limited overlap between THC-using and alcohol-using driver populations.

\begin{figure}[H]
\centering
\includegraphics[width=0.9\textwidth]{figures/fig20_polysubstance.pdf}
\caption{Substance Involvement Breakdown, 2018--2019}
\label{fig:poly}
\end{figure}

\section{Policy Border Patterns}

\subsection{Distance Gradients}

Figure \ref{fig:distborder} presents crash counts by distance to the nearest marijuana legalization border, separately for the legal and illegal sides. The distribution is roughly symmetric around the border, with crash counts declining with distance as population density falls away from major transportation corridors that cross borders.

\begin{figure}[H]
\centering
\includegraphics[width=0.9\textwidth]{figures/fig24_crash_density_border.pdf}
\caption{Fatal Crashes by Distance to Legal/Illegal State Border}
\label{fig:distborder}
\end{figure}

Figure \ref{fig:thcdist} presents THC-positive rates by distance to the border, pooling crashes across borders with sufficient sample size (at least 10 crashes per distance bin). The figure shows THC rates for the legal side; insufficient crash counts on the illegal side within distance bins preclude reliable comparison. Appendix figures present border-pair-specific patterns. This heterogeneity across borders suggests caution in drawing universal conclusions about border effects.

\begin{figure}[H]
\centering
\includegraphics[width=0.9\textwidth]{figures/fig25_thc_rate_border.pdf}
\caption{THC-Positive Rate by Distance to Legal/Illegal Border}
\label{fig:thcdist}
\end{figure}

\subsection{Cross-Sectional Patterns}

Our data limitations (THC identification only from 2018; missing years 2006--2015) preclude a true event study design. However, the cross-sectional comparison in 2018--2019 shows clear differences: states that legalized recreational marijuana (CO, WA, OR, CA, NV, AK) have THC-positive rates of approximately 20\%, while comparison states (illegal during study period) show rates around 10\%. This pattern is consistent with legalization increasing THC involvement in fatal crashes, though we cannot isolate the precise timing of effects.

\section{Data Quality and Limitations}

\subsection{Geocoding Quality}

Figure \ref{fig:geocode} presents the fraction of FARS crashes with valid geocoded coordinates by year in our sample windows (2001--2005 and 2016--2019). Geocoding quality improved substantially between periods, from approximately 85--90\% in 2001--2005 to over 95\% in 2016--2019. The remaining ungeocodable crashes are disproportionately in rural areas and may differ systematically from geocoded crashes.

\begin{figure}[H]
\centering
\includegraphics[width=0.9\textwidth]{figures/fig07_geocoding_quality.pdf}
\caption{FARS Geocoding Quality Over Time}
\label{fig:geocode}
\end{figure}

\subsection{Drug Data Limitations}

The most significant limitation of our data is incomplete drug reporting. Figure \ref{fig:testing} shows the share of fatal crashes with any driver drug record in the FARS drugs file by state for 2018--2019, ranging from approximately 31\% to 35\% in Western states. \textbf{Important:} This figure measures ``crashes with any drug finding reported,'' not ``crashes where drivers were drug-tested''---FARS reports drug findings, not comprehensive test panels.

\begin{figure}[H]
\centering
\includegraphics[width=0.7\textwidth]{figures/fig08_testing_rate_by_state.pdf}
\caption{Share of Fatal Crashes with Any Driver Drug Record, 2018--2019. Measures presence of any record in FARS drugs file (indicating some drug finding was reported), not comprehensive drug testing rates.}
\label{fig:testing}
\end{figure}

Selection into drug reporting is non-random. Drivers who survive crashes are more likely to have results reported. Crashes with fatalities are more thoroughly investigated. States may have different reporting practices. These selection patterns complicate causal inference---observed THC detection rates reflect both true marijuana use and reporting/testing practices.

We recommend that researchers using our data: (1) condition on crashes with any drug record, recognizing this is a selected sample; (2) examine robustness to state fixed effects that absorb reporting differences; (3) consider drug-record rates as an outcome to examine whether legalization changed reporting/finding behavior.

\subsection{THC Detection Limitations}

FARS has included drug result codes throughout our study period that can identify cannabis/cannabinoid involvement at varying levels of specificity. However, the \texttt{drugresname} field with comprehensive text-based drug names (e.g., ``Tetrahydrocannabinols (THC)'', ``Delta-9-THC'') was only consistently populated from 2018 onward. Earlier years rely on numeric drug codes that group substances into broader categories (e.g., ``cannabinoid'' rather than specific THC metabolites). We chose text-based THC matching for 2018--2019 to maximize precision, but researchers wishing to extend our analysis to earlier years could construct a broader ``cannabinoid-positive'' indicator using drug result codes. The patterns we document reflect post-2018 cross-sectional differences; pre-2018 cannabinoid identification would require different variable construction.

Additionally, unlike alcohol, which metabolizes predictably over hours, THC can be detected in blood for days to weeks after use, depending on frequency of use. A positive THC test indicates prior cannabis use but does not necessarily indicate impairment at the time of the crash. This measurement limitation means our THC-positive rates overstate crash-concurrent impairment relative to alcohol.

\section{Research Applications}

Our integrated dataset enables several research designs that have been difficult to implement with aggregate data.

\subsection{Spatial Regression Discontinuity}

The sharp policy contrast at state borders motivates spatial RDD designs following \citet{keele2015}. The border provides a natural cutoff: crashes just on the legal side are in a legal-marijuana environment while crashes just on the illegal side face prohibition. If potential outcomes are continuous at the border---that is, if areas immediately adjacent to the border are similar in all respects except marijuana policy---then comparing outcomes across the border identifies the causal effect of legalization.

Our data enable this design for outcomes with sufficient sample sizes in 2018--2019 (the period for which \texttt{dist\_to\_border\_km} is computed). The Colorado-Wyoming border region has hundreds of crashes with drug records within 50km of either side in 2018--2019. For crash-count or alcohol-involvement outcomes, researchers could extend the border distance variable to earlier years (our replication code supports this), which would provide larger samples.

\subsection{Difference-in-Differences}

Our dataset supports a two-period long-difference DiD design comparing 2001--2005 (pre-legalization) to 2016--2019 (post-legalization for early adopters). For states that legalized by 2015 (CO, WA, OR, AK), this provides clear pre/post comparison against never-recreational-treated controls. \textbf{Note on DiD controls:} For DiD designs, we recommend using the five states that remained illegal through 2023 (WY, NE, KS, ID, UT) as the control group, which differs from our broader ``comparison states'' group used for cross-sectional 2018--2019 analyses (which includes AZ, MT, NM because they were illegal during that window but later legalized).

\textbf{Important limitation:} Our data do not cover 2006--2015, which means we cannot observe the adoption window for early legalizers (CO/WA in 2012; OR/AK in 2014--15). This precludes event-study dynamics, pre-trend diagnostics, and true staggered-adoption designs that require continuous annual data. Researchers wishing to implement modern DiD estimators such as \citet{callaway2021} or \citet{sun2021} would need to extend our pipeline to include the missing years. Our replication code is designed to accommodate such extensions.

\subsection{Within-State Variation}

Several states permit county-level opt-outs from marijuana retail. In Colorado, approximately 48\% of counties banned retail dispensaries even after state legalization. This within-state variation enables designs comparing counties with and without retail access, holding state policy constant. Similar opt-out provisions exist in California, Oregon, and other states, creating quasi-experimental variation that isolates retail availability from legal possession.

Researchers can exploit this variation by geocoding crashes to counties and merging county-level dispensary data. Our OSM integration provides road-level detail that allows distinguishing crashes near versus far from retail locations within counties that permit sales.

\subsection{Mechanism Analysis}

The geocoded nature of our data enables investigation of several potential mechanisms through which marijuana legalization might affect traffic safety:

\textbf{Time-of-day patterns:} Marijuana retail typically operates during daytime hours, potentially shifting consumption to different times than alcohol. Our hour-of-crash variable enables tests of whether THC-involved crashes have different temporal distributions in legal versus illegal states.

\textbf{Road-type heterogeneity:} If marijuana impairment affects driving performance differently than alcohol, effects might vary by road complexity. Highway crashes require sustained attention; urban crashes require frequent decision-making. Our OSM integration enables stratification by road type.

\textbf{Distance-to-border dynamics:} Cross-border shopping may create spillover effects. Residents of illegal states near legal borders can easily purchase marijuana and return. Our distance-to-border variable enables testing whether THC detection in illegal states is elevated near legal borders.

\textbf{Substitution analysis:} A key policy question is whether marijuana substitutes for or complements alcohol. Our data track both substances simultaneously, enabling joint analysis of alcohol and THC involvement patterns.

\section{Discussion}

Several caveats warrant discussion before drawing policy implications from our descriptive patterns.

\textbf{Selection versus causation:} The higher THC detection rates we observe in legalized states may reflect increased marijuana use, increased testing/reporting, or both. Our data cannot distinguish these mechanisms. Cross-sectional differences between states confound legalization effects with pre-existing differences in marijuana culture, testing practices, and driver populations.

\textbf{Detection versus impairment:} THC detection indicates recent cannabis exposure but not necessarily impairment at the time of the crash. Unlike alcohol, where blood/breath concentration correlates with impairment, THC pharmacokinetics are complex. Frequent users may test positive for days after use without being impaired. This measurement limitation means our THC rates are better interpreted as indicators of recent cannabis use than as measures of marijuana-impaired driving.

\textbf{External validity:} Our Western states focus captures the early-legalizing states but may not generalize to other regions. Western states have distinct demographics, driving patterns, and road infrastructure. As legalization spreads eastward, effects may differ.

\textbf{Reporting heterogeneity:} The FARS drugs file reports findings rather than comprehensive test panels. States and jurisdictions may differ in which substances they test for and report. This heterogeneity could generate spurious cross-state differences if, for example, legal states are more likely to test for marijuana than illegal states.

Despite these limitations, our dataset provides the foundation for rigorous causal research. The geocoded precision enables spatial RDD designs that can address many selection concerns. Replication code allows researchers to extend the analysis as data limitations are resolved.

\section{Conclusion}

We have constructed and documented a novel integrated dataset combining FARS crash records, OpenStreetMap road network attributes, and marijuana legalization policy timing for Western US states. The dataset provides crash-level geocoded data suitable for high-resolution spatial analysis of impaired driving patterns.

Our descriptive analysis reveals substantially higher THC-positive rates in legalized states compared to comparison states in 2018--2019, sharp cross-border discontinuities in that period, and limited evidence of alcohol-THC substitution. These patterns motivate---but do not constitute---causal analysis. The dataset we provide enables researchers to implement spatial RDD, difference-in-differences, and other designs with the geographic precision necessary to credibly identify causal effects.

Complete replication code accompanies this paper. Researchers can use our pipeline to extend the analysis to additional states, incorporate updated FARS releases, or adapt the methodology for other policy applications requiring geocoded administrative data.

\section*{Acknowledgements}

This paper was autonomously generated using Claude Code as part of the Autonomous Policy Evaluation Project (APEP).

\noindent\textbf{Project Repository:} \url{https://github.com/anthropics/auto-policy-evals}

\noindent\textbf{Replication Code:} Available in the project repository.

\label{apep_main_text_end}
\newpage
\bibliography{references}

\newpage
\appendix

\section{Data Appendix}

\subsection{Variable Definitions}

\textbf{Crash-level variables:}
\begin{itemize}
    \item \texttt{st\_case}: FARS case number (unique within state-year)
    \item \texttt{state}: State FIPS code
    \item \texttt{year}: Crash year
    \item \texttt{latitude}, \texttt{longitude}: Crash coordinates
    \item \texttt{fatals}: Number of fatalities
    \item \texttt{hour}, \texttt{minute}: Crash time
    \item \texttt{day\_week}: Day of week (1=Sunday, 7=Saturday)
\end{itemize}

\textbf{Substance variables:}
\begin{itemize}
    \item \texttt{thc\_positive}: Crash-level indicator: equals 1 if any driver in the crash has a THC-positive record in the FARS drugs file (based on drug name field, available 2018+); equals 0 if crash has driver drug records but no THC-positive result (note: FARS reports drug \textit{findings}, so 0 may include crashes where THC was not tested or results were not reported); NA if crash has no driver drug records
    \item \texttt{alc\_involved}: At least one driver with alcohol impairment, based on FARS \texttt{drunk\_dr} field (count of drunk drivers in crash). This is our primary alcohol measure.
    \item \texttt{driver\_bac\_over\_08}: Any driver with BAC >= 0.08
    \item \texttt{max\_bac}: Highest BAC among tested drivers
\end{itemize}

\textbf{Policy variables:}
\begin{itemize}
    \item \texttt{rec\_legal}: Recreational marijuana legal at crash date
    \item \texttt{retail\_open}: Retail marijuana sales legal at crash date
    \item \texttt{dist\_to\_border\_km}: Distance to nearest marijuana policy border. \textbf{Note:} Computed using fixed 2018--2019 legal/illegal classification; set to NA for all crashes outside 2018--2019 (including 2016--2017 when some states' status changed mid-period)
    \item \texttt{rel\_time\_rec}: Months since/until state's legalization date
\end{itemize}

\textbf{Road variables (from OSM):}
\begin{itemize}
    \item \texttt{highway}: OSM highway classification
    \item \texttt{maxspeed}: Posted speed limit
    \item \texttt{lanes}: Number of lanes
    \item \texttt{snap\_dist\_m}: Distance from crash to matched road segment
\end{itemize}

\subsection{Replication Instructions}

To reproduce the analysis:

\begin{enumerate}
    \item Clone the repository from GitHub
    \item Install R packages: \texttt{tidyverse}, \texttt{sf}, \texttt{tigris}, \texttt{data.table}
    \item Install Python packages: \texttt{osmnx}, \texttt{geopandas}, \texttt{pandas}
    \item Run scripts in order:
    \begin{itemize}
        \item \texttt{01\_fetch\_fars.R}: Download FARS data
        \item \texttt{02\_fetch\_osm.py}: Extract OSM road networks
        \item \texttt{03\_snap\_crashes.py}: Snap crashes to roads
        \item \texttt{04\_merge\_policy.R}: Add policy variables
        \item \texttt{05\_build\_analysis.R}: Create final dataset
        \item \texttt{06\_national\_figures.R}: Generate overview figures
        \item \texttt{07\_zoom\_figures.R}: Generate zoom maps
        \item \texttt{08\_substance\_figures.R}: Generate substance figures
        \item \texttt{09\_border\_figures.R}: Generate border analysis
        \item \texttt{10\_tables.R}: Generate tables
    \end{itemize}
    \item Compile \texttt{paper.tex}
\end{enumerate}

\section{Additional Figures}

This appendix contains additional supporting figures referenced in the main text.

% Event study figure omitted due to data limitations (THC data only available 2018+)

\begin{figure}[H]
\centering
\includegraphics[width=0.9\textwidth]{figures/fig28_border_pairs.pdf}
\caption{THC-Positive Rates by Border Pair: Legal vs. Illegal States, 2018--2019}
\label{fig:borderpairs}
\end{figure}

\begin{figure}[H]
\centering
\includegraphics[width=0.9\textwidth]{figures/fig29_rdd_co_wy.pdf}
\caption{RDD-Style Plot: THC-Positive Rate at Colorado-Wyoming Border, 2018--2019}
\label{fig:rdd}
\end{figure}

\end{document}
