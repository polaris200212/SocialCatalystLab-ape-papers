\documentclass[12pt]{article}

% UTF-8 encoding
\usepackage[utf8]{inputenc}
\usepackage[T1]{fontenc}

% Page setup
\usepackage[margin=1in]{geometry}
\usepackage{setspace}
\onehalfspacing

% Math and symbols
\usepackage{amsmath,amssymb}

% Graphics
\usepackage{graphicx}
\usepackage{float}

% Tables
\usepackage{booktabs}
\usepackage{array}
\usepackage{multirow}

% Hyperlinks
\usepackage{hyperref}
\hypersetup{
    colorlinks=true,
    linkcolor=blue,
    citecolor=blue,
    urlcolor=blue
}

% Captions
\usepackage{caption}
\captionsetup{font=small,labelfont=bf}

% Section formatting
\usepackage{titlesec}
\titleformat{\section}{\large\bfseries}{\thesection.}{0.5em}{}
\titleformat{\subsection}{\normalsize\bfseries}{\thesubsection}{0.5em}{}

% Custom commands
\newcommand{\E}{\mathbb{E}}

\title{Universal License Recognition and Interstate Worker Mobility:\\ Evidence from Wyoming}
\author{APEP Autonomous Research\thanks{%
Autonomous Policy Evaluation Project.
This paper was autonomously generated using Claude Code.
Contributor: CONTRIBUTOR\_GITHUB.
Repository: https://github.com/dakoyana/auto-policy-evals Contributor: @dakoyana.}}
\date{\today}

\begin{document}

\maketitle

\begin{abstract}
\noindent
Occupational licensing affects approximately 25 percent of American workers and has been shown to reduce interstate labor mobility by creating barriers for workers who must re-certify when moving across state lines. This paper examines whether universal license recognition (ULR) laws, which allow states to automatically recognize out-of-state occupational licenses, increase interstate migration of licensed workers. I exploit the timing of Wyoming's 2021 ULR law (SF 0018) using a difference-in-differences design, comparing changes in interstate in-migration rates among licensed workers in Wyoming relative to similar states without ULR laws. The primary estimate suggests a 0.48 percentage point increase in interstate in-migration rates for licensed workers, representing approximately 11 percent relative to the pre-treatment control mean. However, a placebo test using unlicensed workers shows an even larger positive effect, suggesting that general migration trends to Wyoming during this period may confound the licensing-specific estimate. The results highlight both the promise and the challenges of evaluating state-level licensing reforms using survey microdata with small samples and concurrent macroeconomic shocks.
\end{abstract}

\vspace{1em}
\noindent\textbf{JEL Codes:} J61, J44, K23 \\
\noindent\textbf{Keywords:} occupational licensing, labor mobility, universal license recognition, migration, Wyoming

\newpage

\section{Introduction}

Occupational licensing has grown dramatically over the past half-century, from covering roughly 5 percent of U.S. workers in the 1950s to approximately 25 percent today (Kleiner and Krueger, 2013). While proponents argue that licensing protects consumers by ensuring minimum quality standards, critics contend that licensing requirements often exceed what is necessary for public safety and create substantial barriers to labor market entry and interstate mobility (Carpenter et al., 2017). A growing body of evidence suggests that licensing reduces labor market fluidity and depresses interstate migration rates, particularly for workers whose credentials do not transfer automatically across state borders (Johnson and Kleiner, 2020).

This paper examines whether universal license recognition (ULR) laws—which require states to recognize occupational licenses issued by other states—can restore some of the geographic mobility lost to fragmented licensing regimes. In February 2021, Wyoming enacted SF 0018, becoming one of the first states to implement a broad ULR policy covering most licensed occupations. The law allows workers with valid out-of-state licenses to practice in Wyoming without repeating training requirements or examinations, substantially reducing the time and financial costs of interstate moves for licensed workers. I exploit the timing of this policy change using a difference-in-differences (DiD) design, comparing changes in interstate in-migration rates among licensed workers in Wyoming before and after SF 0018 relative to changes in comparable states without similar policies.

The primary DiD estimate suggests that Wyoming's ULR law increased interstate in-migration rates among licensed workers by 0.48 percentage points, representing approximately 11 percent relative to the pre-treatment control mean of 4.4 percent. This estimate is consistent with the hypothesis that licensing barriers meaningfully deter interstate migration and that removing these barriers can increase worker mobility. However, several findings warrant caution in interpreting this result as a causal effect of ULR specifically. First, a placebo test using unlicensed workers—who should not be affected by licensing reforms—shows an even larger positive effect (0.98 percentage points), suggesting that general migration trends to Wyoming during this period may explain much of the observed increase. Second, an event study reveals substantial year-to-year volatility in Wyoming's migration rates that predates the policy change, raising concerns about parallel trends. Third, the sample size is modest: Wyoming's small population yields only 585 licensed workers across four years of data, limiting statistical power.

This paper contributes to the growing literature on occupational licensing and labor mobility in several ways. Most existing research on licensing and mobility relies on cross-sectional variation in licensing stringency (Kleiner, 2015) or focuses on specific high-profile occupations such as lawyers or physicians (Barrios, 2020). The recent wave of ULR laws provides a cleaner source of policy variation, but these laws are too recent to have been extensively studied. Oh and Kleiner (2025) provide the most comprehensive analysis to date, finding that ULR increased healthcare utilization but had no detectable effect on physician interstate migration. My analysis extends this work by examining non-healthcare licensed occupations—such as cosmetologists, barbers, real estate agents, and electricians—where licensing costs may be more binding relative to earnings and where labor markets may be more geographically elastic. While the results do not provide definitive evidence that ULR increases licensed worker mobility, they illustrate the empirical challenges of evaluating state-level licensing reforms and highlight directions for future research with larger samples or administrative data.

Understanding the mobility effects of licensing reforms matters for several reasons beyond academic interest. If licensing barriers significantly deter migration, then ULR policies may improve labor market efficiency by allowing workers to move to locations where their skills are most valued. In a competitive labor market, reducing artificial barriers to mobility should improve the match between workers and jobs, potentially raising productivity and wages. Conversely, if licensing barriers are a minor friction relative to other moving costs---as the null results for physician migration in Oh and Kleiner (2025) suggest---then ULR may have limited economic impact despite its intuitive appeal. The magnitude of migration responses to ULR is thus an empirical question with implications for how much policymakers should prioritize licensing reform relative to other barriers such as housing supply constraints.

The Wyoming setting offers both advantages and limitations for studying ULR effects. On the advantage side, Wyoming enacted a broad ULR law with a clear implementation date, providing clean policy timing for a DiD design. The state's small size creates a tractable empirical setting where state-level policy effects might be observable in survey data. On the limitation side, Wyoming's small population yields small samples that limit statistical power, and the state's distinctive economic characteristics---low population density, energy-dependent economy, minimal COVID restrictions---may make it unrepresentative of how ULR would affect larger, more urban states. The pandemic timing is particularly concerning: SF 0018 was signed in February 2021, during a period of unprecedented disruption to migration patterns nationwide. Disentangling ULR effects from pandemic-related migration shifts is challenging with only two years of post-treatment data.

The remainder of this paper proceeds as follows. Section 2 provides institutional background on occupational licensing and the Wyoming ULR law. Section 3 reviews the related literature on licensing and labor mobility. Section 4 describes the conceptual framework and expected effects. Section 5 discusses the data from the Census PUMS. Section 6 presents the empirical strategy. Section 7 reports the results, including robustness checks and validity tests. Section 8 discusses limitations and interprets the findings, and Section 9 concludes.

\section{Institutional Background}

\subsection{Occupational Licensing in the United States}

Occupational licensing is a form of labor market regulation that prohibits individuals from practicing a profession without government approval. Unlike weaker forms of regulation such as certification (where individuals can practice without the credential but may not use a protected title) or registration (where individuals simply add their names to a state database), licensing creates a legally binding entry barrier. Unlicensed practice is typically a misdemeanor or felony offense.

The stated rationale for occupational licensing is consumer protection: by ensuring that practitioners meet minimum standards of education, training, and competence, licensing theoretically reduces the risk of harm from unqualified service providers. This argument is strongest for occupations where the potential for harm is high and consumers cannot easily evaluate quality before purchase—the classic examples being physicians and lawyers. However, licensing has expanded far beyond these professions. Today, states require licenses for occupations ranging from interior designers to florists to hair braiders, often with training requirements that critics argue bear little relationship to actual job competence or consumer safety (Carpenter et al., 2012).

The economic effects of licensing have been extensively studied. Licensing tends to raise wages for licensed workers, likely reflecting both reduced labor supply (restricted entry) and signaling effects (licensing as a quality signal). Estimates of the licensing wage premium range from 5 to 15 percent, with larger effects in occupations with more stringent requirements (Kleiner and Krueger, 2013). However, these private benefits to license-holders come at social cost. Licensing reduces labor force participation and employment in affected occupations, raises prices for consumers, and—most relevant to this paper—reduces geographic mobility across state lines (Kleiner, 2015).

\subsection{Licensing and Interstate Mobility}

The mobility cost of licensing arises from the fragmented nature of U.S. licensing regimes. Each state maintains its own licensing requirements, boards, and standards. A cosmetologist licensed in Colorado who moves to Wyoming cannot automatically practice; she must apply to Wyoming's licensing board, demonstrate that her training meets Wyoming's requirements, potentially take additional examinations, and pay application fees. For some occupations, meeting the new state's requirements may require additional coursework or supervised practice hours, creating delays of weeks or months before the worker can legally resume practicing.

Several studies have documented that licensing reduces interstate mobility. Using Current Population Survey data, Johnson and Kleiner (2020) find that licensing reduces interstate migration rates by approximately 7 percentage points, with larger effects for occupations with more state-specific requirements. Mulligan and Tsui (2016) find similar effects using cross-state variation in licensing stringency. The mechanism is straightforward: licensing raises the cost of moving, and workers respond by moving less frequently.

The mobility costs of licensing are not evenly distributed. They fall disproportionately on workers with lower earnings (for whom the fixed cost of relicensing is a larger share of income), workers in occupations with more heterogeneous state requirements (where transferring credentials is more difficult), and workers whose personal circumstances may require frequent moves—most notably military spouses, who face chronic employment disruptions as their families relocate. Recognition of these costs has motivated a wave of licensing reforms over the past decade.

\subsection{Universal License Recognition Laws}

Universal license recognition (ULR) represents the most comprehensive approach to addressing licensing-related mobility barriers. Under ULR, a state commits to recognizing occupational licenses issued by any other state, provided the license is in good standing and the holder meets certain basic requirements (such as not having a disciplinary history). Arizona pioneered this approach in 2019 with HB 2569, and by 2023 approximately 20 states had enacted some form of ULR.

Wyoming's SF 0018, signed by Governor Mark Gordon in February 2021, requires state licensing boards to issue Wyoming licenses to applicants who hold valid licenses from other states. The law covers most licensed occupations but excludes attorneys (who are subject to separate interstate compacts) and professions with prescriptive authority (such as physicians and pharmacists, who can prescribe controlled substances). The law does not require Wyoming to accept licenses from states with ``substantially weaker'' standards, giving licensing boards some discretion, but the default presumption is recognition.

Prior to SF 0018, a cosmetologist moving to Wyoming from another state faced several hurdles. Wyoming requires 1,600 hours of cosmetology training, among the highest in the nation. If the cosmetologist's home state required fewer hours, she might need to document additional training, take Wyoming's written and practical examinations (offered only every other month in Casper), and wait for board approval. This process could take weeks or months, during which the worker could not legally practice or earn income in her occupation. After SF 0018, the same cosmetologist can apply for license recognition and begin work almost immediately, provided her home-state license is valid and she has no disqualifying disciplinary history.

The scope of Wyoming's ULR law is notable for its breadth. Unlike targeted reforms that address specific occupations or populations (such as expedited licensing for military spouses or interstate compacts for nurses), SF 0018 applies to most licensed occupations in the state. According to data from the Wyoming Department of Workforce Services, approximately 45 licensing boards oversee nearly 100 occupations that require a state license, spanning healthcare (nurses, physical therapists), personal services (cosmetologists, barbers, massage therapists), professional services (real estate agents, accountants, architects), and trades (electricians, plumbers, contractors). All of these occupations are potentially affected by ULR, though the practical importance depends on how stringent Wyoming's requirements were relative to other states prior to the reform.

The political economy of Wyoming's ULR adoption reflects broader trends in occupational licensing reform. Conservative-leaning states have been at the forefront of ULR adoption, viewing licensing barriers as unnecessary regulation that restricts economic freedom and harms workers. Arizona, the first ULR adopter, was led by a Republican governor who championed the reform as part of a broader deregulatory agenda. Wyoming followed suit under Republican Governor Mark Gordon. This political alignment does not necessarily undermine the policy's rationale---reducing unnecessary barriers to work can appeal across ideological lines---but it does raise questions about external validity. The states that have adopted ULR may differ systematically from non-adopters in ways that affect both the policy's implementation and its economic effects.

The implementation details of Wyoming's SF 0018 are important for understanding how the law operates in practice. Under the law, an applicant for license recognition must submit proof of a valid, current license from another state; verification that the license has been in good standing for at least one year; and disclosure of any disciplinary actions or criminal convictions that could affect licensure. The Wyoming licensing board must then process the application within a specified time frame (typically 30-90 days depending on the occupation), though many applications are processed more quickly. Importantly, the law allows boards to deny recognition if the applicant's home state has ``substantially weaker'' standards than Wyoming, though this determination is subject to review and must be documented. In practice, most applications for recognition are approved, as licensing requirements across states---while different in specific hour requirements---generally cover similar competencies.

The exclusion of attorneys and prescribers from Wyoming's ULR deserves comment. Attorneys are excluded because lawyer licensing operates through a separate system of bar admissions that is largely governed by state supreme courts rather than administrative licensing boards. Interstate compacts already exist for limited practice in some contexts, and the American Bar Association has long debated but not adopted universal reciprocity. Prescribers (physicians, pharmacists, and other healthcare providers who can prescribe controlled substances) are excluded due to federal DEA registration requirements and the complexity of medical licensing, which involves credentialing, malpractice liability, and hospital privileging beyond the state license itself. These exclusions mean that the occupations covered by Wyoming's ULR are primarily non-healthcare licensed occupations where the policy's effects might be most observable.

The theoretical prediction is that ULR should increase in-migration of licensed workers by reducing the costs of interstate moves. Workers who previously faced substantial relicensing burdens now find it easier to move to ULR states. This should manifest as an increase in the flow of licensed workers into Wyoming (the extensive margin of migration) and potentially in the total stock of licensed workers (if new entrants exceed exits).

\section{Related Literature}

This paper relates to three strands of the economics literature: the effects of occupational licensing on labor markets, the determinants of geographic mobility, and the evaluation of licensing policy reforms.

The modern literature on occupational licensing began with Milton Friedman and Simon Kuznets' (1945) observation that licensing creates barriers to entry that raise practitioner incomes. Subsequent work has documented licensing wage premia across a variety of occupations and settings. Kleiner and Krueger (2013), using survey data that directly identifies licensed workers, estimate an average premium of approximately 15 percent, though with substantial heterogeneity across occupations. The wage effects are larger for occupations with more stringent requirements, consistent with both a supply restriction mechanism (fewer workers can enter) and a signaling mechanism (the license signals quality to employers and consumers).

The mobility consequences of licensing have received increasing attention. Johnson and Kleiner (2020) provide the most comprehensive evidence, showing that licensing reduces interstate migration by approximately 7 percentage points, with larger effects for occupations where state requirements differ more substantially. This finding is robust to controlling for occupation characteristics and individual demographics. Mulligan and Tsui (2016) find similar effects using a different identification strategy based on cross-state variation in licensing stringency. The mechanism appears to operate primarily through the relicensing costs that workers face when moving across state lines, rather than through other features of licensing such as criminal background checks.

Several studies have examined the effects of licensing on specific populations particularly affected by mobility costs. Kleiner and Xu (2020) find that states with more stringent licensing requirements have lower employment rates among immigrants with professional backgrounds. The military spouse literature documents substantial employment penalties associated with frequent relocations and licensing barriers (Harrell et al., 2004). These findings have motivated targeted reforms such as expedited licensing for military spouses, which many states have adopted in recent years.

The evaluation of licensing policy reforms is a more recent area of research, reflecting the recency of the reform wave. Oh and Kleiner (2025) provide the most rigorous analysis of universal license recognition to date. Using variation in ULR adoption across states and a difference-in-differences design, they find that ULR increased healthcare utilization, particularly among older adults and in underserved areas. Interestingly, they find no significant effect on physician interstate migration---a null result they attribute to physicians' strong local ties and the relatively low share of total moving costs represented by relicensing. Their finding that ULR works primarily through temporary out-of-state practice (including telehealth) rather than permanent relocation is consistent with my results showing limited migration effects for Wyoming.

Beyond ULR, researchers have studied other licensing reforms with more limited scope. Interstate licensing compacts, which allow participating states to recognize each other's credentials, have been established for nursing, medicine, and several other healthcare professions. These compacts have facilitated telehealth expansion and cross-border practice, though their effects on permanent migration have not been rigorously evaluated. Expedited licensing for military spouses, now adopted in some form by all 50 states, addresses mobility barriers for a specific population but does not fundamentally reform the fragmented licensing system. Thornton and Timmons (2013) study the deregulation of massage therapy in one state, finding that removal of licensing requirements did not significantly affect service quality or consumer complaints, suggesting that licensing in this occupation may have been primarily a barrier to entry rather than a quality assurance mechanism.

The broader literature on geographic mobility provides context for interpreting licensing effects. Internal migration rates in the United States have declined substantially over the past several decades (Molloy, Smith, and Wozniak, 2011), with researchers attributing this decline to factors ranging from aging demographics to declining regional income differentials to increased homeownership. Licensing-related barriers are one component of this decline, but their relative importance is debated. Ganong and Shoag (2017) emphasize housing supply constraints as a key barrier to geographic mobility, arguing that zoning restrictions prevent workers from moving to high-productivity cities. From this perspective, licensing reforms may have limited effect if housing constraints remain binding.

This paper extends the literature in several ways. First, I examine non-healthcare licensed occupations, where the relative burden of relicensing may be higher and labor markets may be more geographically responsive. A cosmetologist earning \$30,000 per year faces a larger proportional cost from relicensing fees and lost work time than a physician earning \$300,000. Second, I provide what may be the first microdata-based evaluation of ULR effects on actual migration flows (as opposed to employment or service utilization). Third, the analysis illustrates the empirical challenges of studying state-level policy variation with small samples and concurrent macroeconomic shocks, contributing to methodological discussions about the credibility of DiD designs in these settings.

\section{Conceptual Framework}

The theoretical effect of ULR on interstate migration can be derived from a simple model of location choice. Consider a licensed worker deciding whether to remain in her current state or move to Wyoming. The utility from remaining is normalized to zero. The utility from moving includes the benefits of the move (better job opportunities, family reasons, quality of life) minus the costs of moving. These costs include physical moving costs, the psychic costs of leaving one's community, and—for licensed workers—the costs of relicensing.

Let $B_i$ denote the idiosyncratic benefits worker $i$ perceives from moving to Wyoming, $C_{phys}$ denote the physical and psychic costs of moving, and $C_{lic}$ denote the licensing-related costs. The worker moves if:
\begin{equation}
B_i - C_{phys} - C_{lic} > 0
\end{equation}

Prior to ULR, licensed workers face positive relicensing costs ($C_{lic} > 0$), which reduces the number of workers for whom moving is worthwhile. After ULR, relicensing costs fall substantially (to zero in the limit), increasing the number of workers who find moving beneficial. The expected effect on migration rates is therefore positive.

Several factors moderate this theoretical effect. First, if licensing costs are a small share of total moving costs, the effect of ULR will be modest. Physical moving costs, housing price differentials, and the psychic costs of leaving one's community may dominate. Second, workers with stronger preferences for Wyoming (higher $B_i$) may have already been willing to absorb the relicensing costs pre-ULR, reducing the marginal population for whom ULR makes a difference. Third, workers may be unaware of the new policy, especially in the short run.

The framework generates predictions about heterogeneity. The effect of ULR should be larger for occupations with higher pre-ULR relicensing costs. For example, cosmetology requires 1,600 hours of training in Wyoming, among the highest in the nation, while other occupations may have lower requirements. Workers in high-requirement occupations face greater cost reductions from ULR and should respond more strongly. The effect should also be larger for workers whose idiosyncratic benefits of moving to Wyoming were just below the threshold required to justify the pre-ULR relicensing costs. These marginal workers are precisely those for whom the cost reduction tips the balance toward moving. Finally, the effect should be larger for younger workers, who have longer time horizons over which to recoup moving costs and who tend to be more geographically mobile generally across a variety of measures.

The effect should be zero for unlicensed workers, who face no licensing-related moving costs and thus experience no change from ULR. This provides a natural placebo test: if we observe similar migration increases for unlicensed workers, the pattern likely reflects general migration trends to Wyoming rather than a licensing-specific response to ULR.

The model also generates predictions about the magnitude of expected effects. Prior estimates suggest that licensing reduces interstate migration by approximately 7 percentage points (Johnson and Kleiner, 2020). However, this estimate captures the total effect of licensing, including both the direct relicensing costs that ULR addresses and other features of licensing that ULR does not address (such as the deterrent effect of licensing on occupational entry in the first place). If relicensing costs account for, say, half of the total mobility effect of licensing, and Wyoming's ULR eliminates most of these costs, we might expect migration rates to increase by 3-4 percentage points. However, this calculation ignores several factors that would attenuate the observed effect: not all workers are aware of ULR, not all would have been deterred by relicensing costs in the first place, and the short post-treatment period may not capture slow-moving migration decisions. A more conservative expectation would be effects of 0.5-2 percentage points, depending on how responsive migration is on the margin.

The theoretical framework also highlights potential general equilibrium effects that complicate interpretation. If ULR increases the supply of licensed workers in Wyoming, wage competition could reduce the attractiveness of Wyoming for licensed workers already in the state, partially offsetting in-migration with increased out-migration. Alternatively, increased supply of licensed services could expand the market (if services were previously undersupplied due to labor shortages), potentially attracting additional workers. These general equilibrium adjustments would occur gradually over time, and the short post-treatment period may not capture them fully. The one-year migration measure captures gross in-migration flows but not net changes in the stock of licensed workers, which would be affected by both in-migration and out-migration.

\section{Data}

\subsection{Census PUMS}

The analysis uses the American Community Survey (ACS) Public Use Microdata Sample (PUMS), which provides individual-level records for approximately 1 percent of the U.S. population annually. The PUMS includes detailed information on demographics, employment, income, and crucially for this analysis, migration status and occupation.

I construct a sample of working-age adults (18-64) residing in Wyoming or comparison states for the years 2018, 2019, 2021, and 2022. The year 2020 is excluded because the Census Bureau did not release standard 1-year PUMS data for 2020 due to data collection disruptions from the COVID-19 pandemic. This creates a pre-treatment period (2018-2019) and a post-treatment period (2021-2022) around the February 2021 policy change.

The key outcome variable is interstate in-migration, derived from the PUMS variable MIGSP (state of residence one year ago) combined with current state of residence (ST). I code a respondent as an interstate in-migrant if their state of residence one year ago differs from their current state of residence and both are valid U.S. state codes. This captures the flow of workers who moved into the state within the past year.

Licensed occupation status is identified using OCCP codes. Following the occupational licensing literature, I classify the following occupation codes as licensed non-healthcare occupations: hairdressers/cosmetologists (4510), barbers (4500), real estate agents (4920), insurance sales agents (4810), electricians (6355), plumbers (6440), and carpenters (6230). I also include licensed practical nurses (3500) and registered nurses (3255) as licensed healthcare occupations for comparison. For the placebo analysis, I identify unlicensed comparison occupations including cashiers (4740), retail salespersons (4760), and office clerks (5860).

Control states are selected based on geography and the absence of ULR laws during the study period. I include Idaho, Utah, Nebraska, South Dakota, and North Dakota---neighboring or nearby states with similar economic characteristics and no ULR enactment as of 2022. Montana is excluded because it enacted its own ULR law in 2021. Colorado is excluded because it enacted ULR in 2020.

The selection of control states involves trade-offs between geographic similarity and sample size. Wyoming's neighboring states are all relatively small and share Wyoming's rural, western character, making them plausibly similar counterfactuals. However, these states also have small PUMS samples, limiting the precision of control group estimates. Expanding the control group to include larger states (such as Texas, California, or Illinois) would increase precision but at the cost of reduced comparability. I opt for geographic similarity over precision, but this choice does not materially affect the main findings given that the primary concern is noise in the Wyoming treatment group rather than the control group.

A further consideration in control state selection is the potential for spillover effects. If ULR in Wyoming reduced in-migration to control states (by diverting licensed workers who would otherwise have moved to those states), the control group migration rate would be artificially depressed post-treatment, biasing the DiD estimate upward. However, this concern is likely minimal given Wyoming's small size. Even a large proportional increase in Wyoming's licensed worker in-migration would represent a trivial share of total migration flows to the control states. Moreover, if spillovers were substantial, we would expect to see declines in control state migration rates post-2021, which is not observed.

The choice of 2020 as the treatment year deserves discussion. SF 0018 was signed in February 2021, so the first survey capturing post-treatment migration would be the 2021 ACS. However, the 2021 ACS migration question asks about residence one year ago, meaning it captures moves that occurred throughout 2020 and early 2021. Some of these moves would have occurred before SF 0018 was enacted. This timing imprecision means the 2021 survey captures a mix of pre- and post-treatment migration, attenuating any treatment effect estimate. The 2022 survey provides cleaner post-treatment identification, as all moves captured would have occurred after the law was in effect. Ideally, the analysis would have access to 2023 data as well, but this was not available at the time of analysis. Additional years of post-treatment data will improve precision and allow for a more compelling event study.

Another data limitation worth noting is the absence of information on the source state of in-migrants. While MIGSP records the state of residence one year ago, and we can identify interstate movers, we cannot easily identify which specific origin states are sending workers to Wyoming. This limits our ability to test hypotheses about which origin states respond most to ULR---for example, we would expect larger responses from states with more different licensing requirements, or from states geographically proximate to Wyoming. With larger samples, origin-state heterogeneity analysis would be valuable, but the small Wyoming sample makes such analysis unreliable.

\subsection{Summary Statistics}

Table 1 presents summary statistics for the analysis sample. The full sample includes 212,488 person-year observations across the four years of data. Wyoming accounts for approximately 6 percent of observations (13,323), reflecting its small population relative to the pooled control states. Among all working-age adults, 4.0 percent moved from a different state within the past year.

\begin{table}[H]
\centering
\caption{Summary Statistics}
\begin{tabular}{lrrr}
\toprule
& Full Sample & Wyoming & Control States \\
\midrule
Observations & 212,488 & 13,323 & 199,165 \\
\addlinespace
\multicolumn{4}{l}{\textit{All Workers}} \\
Interstate in-migration rate & 4.03\% & 5.11\% & 3.96\% \\
Age (mean) & 41.2 & 40.8 & 41.3 \\
Female & 50.2\% & 49.1\% & 50.3\% \\
\addlinespace
\multicolumn{4}{l}{\textit{Licensed Workers (Non-Healthcare)}} \\
Observations & 10,132 & 585 & 9,547 \\
Interstate in-migration rate & 4.45\% & 5.13\% & 4.41\% \\
\addlinespace
\multicolumn{4}{l}{\textit{Unlicensed Control Occupations}} \\
Observations & 8,774 & 454 & 8,320 \\
Interstate in-migration rate & 5.42\% & 12.33\% & 5.05\% \\
\bottomrule
\end{tabular}
\end{table}

The sample of licensed workers is modest, with 585 observations in Wyoming across all years. This reflects both Wyoming's small population and the fact that licensed occupations represent a minority of the workforce. The small sample size limits statistical power and contributes to year-to-year volatility in estimated migration rates.

Wyoming has higher baseline interstate in-migration rates than the control states (5.11\% vs. 3.96\%), consistent with general patterns of mobility into rural Western states. This higher baseline is even more pronounced for unlicensed workers in Wyoming (12.33\%), suggesting that Wyoming attracts mobile workers through channels unrelated to licensing policy.

\subsection{Measurement and Weighting}

All population estimates in this analysis use the PUMS person weights (PWGTP) to produce representative statistics for the underlying populations. The ACS uses a complex survey design with stratification and clustering, and the person weights account for differential sampling probabilities and nonresponse adjustments (Ruggles et al., 2023). For point estimates, I compute weighted means of the migration indicator across the relevant subsamples. Standard errors would ideally account for the survey design and for clustering at the state level, though with only six states in the sample and a single treated unit, inference is inherently limited regardless of the variance estimation approach.

The measurement of interstate migration in the PUMS relies on the MIGSP variable, which records the respondent's state of residence one year prior to the survey. I classify a respondent as an interstate in-migrant if their prior-year state differs from their current state and both are valid U.S. state FIPS codes. This definition captures recent movers but does not identify workers who moved more than one year before the survey or who moved multiple times within the year. The one-year migration window aligns well with the policy timing: for the 2021 survey (collected throughout 2021), in-migrants would have moved sometime between early 2020 and early 2021, spanning the period immediately before and after SF 0018 took effect. For the 2022 survey, in-migrants would have moved in 2021-2022, fully after the policy.

The identification of licensed workers relies on occupation codes rather than direct observation of license status. Not all workers in ostensibly ``licensed'' occupation categories actually hold licenses---some may work in states or positions that do not require licensing, or may be working unlicensed (illegally). Conversely, the occupation codes do not capture all licensed occupations, only those explicitly included in my definition. This measurement error will attenuate estimates toward zero if it is classical, but could also introduce bias if the probability of holding a license differs systematically between Wyoming and control states or between pre- and post-periods. As a partial check, I focus on occupations with near-universal licensing requirements (cosmetology, barbering, real estate) where the occupation-license mapping is most reliable.

\section{Empirical Strategy}

\subsection{Difference-in-Differences Design}

The main analysis uses a difference-in-differences (DiD) design comparing changes in interstate in-migration rates for licensed workers in Wyoming before and after SF 0018 to changes in comparable states without ULR. The estimating equation is:
\begin{equation}
Y_{ist} = \alpha + \beta_1 (Wyoming_s \times Post_t) + \gamma_s + \delta_t + X_{ist}'\theta + \varepsilon_{ist}
\end{equation}
where $Y_{ist}$ is an indicator for whether person $i$ in state $s$ at time $t$ moved from a different state within the past year, $Wyoming_s$ is an indicator for Wyoming residence, $Post_t$ is an indicator for years 2021-2022, $\gamma_s$ are state fixed effects, $\delta_t$ are year fixed effects, and $X_{ist}$ is a vector of individual controls (age, sex, education). The coefficient $\beta_1$ is the DiD estimate of the effect of ULR on interstate in-migration.

The identifying assumption is that, absent ULR, trends in interstate in-migration for licensed workers in Wyoming would have been parallel to trends in control states. I probe this assumption with an event study specification that estimates year-specific treatment effects and tests for pre-trends. The parallel trends assumption is fundamentally untestable---we can examine whether trends were parallel in the pre-period, but this does not guarantee they would have remained parallel absent treatment. The COVID-19 pandemic complicates the analysis by introducing differential shocks across states that may violate parallel trends even if they held historically.

\subsection{Threats to Validity}

Several threats to the validity of this design warrant discussion. First, the pandemic-era economic disruptions were not uniform across states. Wyoming's relatively low population density, limited COVID-19 restrictions, and reliance on energy-sector employment created a distinctive economic trajectory that may have affected migration patterns independent of licensing policy. If Wyoming became more attractive to mobile workers during the pandemic for reasons unrelated to ULR, the DiD estimate would be biased upward.

Second, the control states may not provide an ideal counterfactual. Although I selected states that did not enact ULR during the study period, these states may have implemented other reforms affecting licensed workers, or may have experienced state-specific economic shocks. The small number of control states (five) also limits the ability to construct a robust synthetic control or to examine sensitivity to dropping individual states.

Third, the February 2021 timing of SF 0018 means the first full post-treatment year (2021) may capture both anticipation effects (workers who delayed moves until after ULR took effect) and implementation effects. The ACS migration question asks about residence one year ago, so the 2021 survey captures movers from late 2019/early 2020 through late 2020/early 2021, straddling the policy change. Fully identifying the treatment effect may require additional years of post-treatment data.

Fourth, with a single treated unit, the analysis cannot distinguish between a true treatment effect and an idiosyncratic Wyoming-specific shock coinciding with ULR. The large placebo effect for unlicensed workers suggests this is a serious concern. Pooling multiple ULR-adopting states in future work would help address this limitation, as emphasized by Baker et al. (2022) and Goodman-Bacon (2021).

\subsection{Placebo Test}

As a validity check, I estimate the same DiD specification using unlicensed workers as the outcome group. If the observed increase in licensed worker migration to Wyoming reflects the specific effect of ULR on licensing-related moving costs, we should observe no effect for unlicensed workers, who face no such costs. If instead we observe similar or larger effects for unlicensed workers, this suggests that general migration trends to Wyoming—unrelated to licensing policy—may explain the results.

\subsection{Event Study}

To examine the timing of any effects and test for pre-trends, I estimate:
\begin{equation}
Y_{ist} = \alpha + \sum_{t \neq 2019} \beta_t (Wyoming_s \times Year_t) + \gamma_s + \delta_t + \varepsilon_{ist}
\end{equation}
where 2019 (the final pre-treatment year) is the omitted baseline. The coefficients $\beta_{2018}$ should be approximately zero if pre-treatment trends were parallel. The coefficients $\beta_{2021}$ and $\beta_{2022}$ capture post-treatment effects.

\section{Results}

\subsection{Main DiD Estimates}

Table 2 presents the main DiD results. For licensed workers, the estimate is 0.479 percentage points, suggesting that Wyoming's ULR law was associated with an increase in interstate in-migration rates of approximately half a percentage point. Relative to the pre-treatment control mean of 4.41 percent, this represents an 11 percent increase in migration rates.

\begin{table}[H]
\centering
\caption{Difference-in-Differences Estimates}
\begin{tabular}{lrrrr}
\toprule
& \multicolumn{2}{c}{Licensed Workers} & \multicolumn{2}{c}{Unlicensed Workers} \\
\cmidrule(lr){2-3} \cmidrule(lr){4-5}
& Rate & Change & Rate & Change \\
\midrule
Wyoming Pre-2021 & 4.83\% & & 11.09\% & \\
Wyoming Post-2021 & 5.40\% & +0.57 pp & 13.69\% & +2.60 pp \\
\addlinespace
Control Pre-2021 & 4.41\% & & 4.74\% & \\
Control Post-2021 & 4.50\% & +0.09 pp & 6.35\% & +1.61 pp \\
\addlinespace
\textbf{DiD Estimate} & \multicolumn{2}{c}{\textbf{+0.479 pp}} & \multicolumn{2}{c}{\textbf{+0.984 pp}} \\
Relative to control baseline & \multicolumn{2}{c}{+10.9\%} & \multicolumn{2}{c}{+20.8\%} \\
\bottomrule
\end{tabular}
\end{table}

However, the placebo test raises significant concerns about this estimate's interpretation. For unlicensed workers, the DiD estimate is 0.984 percentage points—nearly double the estimate for licensed workers. Since unlicensed workers should be unaffected by licensing policy, this positive placebo suggests that Wyoming experienced a general increase in in-migration during this period relative to control states, likely reflecting factors such as remote work opportunities, relatively low COVID-19 restrictions, or the state's economic recovery trajectory.

The placebo test does not definitively rule out a licensing-specific effect—licensed workers could be responding to ULR while unlicensed workers respond to other factors—but it substantially weakens confidence in attributing the observed licensed-worker increase to ULR specifically.

\subsection{Event Study}

Table 3 presents event study results, with 2019 as the baseline year. The pre-treatment coefficient for 2018 is -5.8 percentage points (normalized), indicating that the Wyoming-control gap in licensed worker migration was substantially different in 2018 than in 2019. This violates the parallel trends assumption, though the direction of the violation (lower Wyoming migration in 2018) is the opposite of what would inflate the treatment effect.

\begin{table}[H]
\centering
\caption{Event Study: Licensed Worker In-Migration Rates}
\begin{tabular}{lrrrr}
\toprule
Year & WY Rate & Control Rate & Difference & Normalized (vs 2019) \\
\midrule
2018 & 2.64\% & 5.03\% & -2.40\% & -5.82 pp \\
2019 (baseline) & 7.26\% & 3.84\% & +3.42\% & 0.00 pp \\
2021 & 9.04\% & 4.51\% & +4.53\% & +1.12 pp \\
2022 & 2.15\% & 4.49\% & -2.34\% & -5.75 pp \\
\bottomrule
\end{tabular}
\end{table}

The post-treatment coefficients are volatile. In 2021, the Wyoming-control gap widened by 1.12 percentage points relative to 2019, consistent with a positive ULR effect. But in 2022, the gap reversed sharply, with Wyoming's migration rate falling below control states. This volatility reflects the small sample sizes—with only 125-161 licensed workers observed in Wyoming per year, year-to-year sampling variation can produce large swings in estimated rates.

The event study does not provide strong evidence either for or against a ULR effect. The pre-trend violation in 2018 and the reversal in 2022 are concerning, but the pattern could reflect sampling noise rather than true violations of identifying assumptions. The year-to-year swings in Wyoming's licensed worker migration rate---from 2.64\% in 2018 to 7.26\% in 2019 to 9.04\% in 2021 to 2.15\% in 2022---far exceed what would be expected from true underlying variation in migration propensity, suggesting that sampling error dominates the signal in this small-sample setting.

One interpretation of the event study pattern is that 2019 was an anomalously high year for Wyoming licensed worker migration (perhaps due to energy-sector hiring or other idiosyncratic factors), and the ``effect'' of ULR in 2021 partially reflects regression to the mean combined with continued volatility. Alternatively, the pattern could reflect genuine pre-treatment variation in migration rates that undermines the parallel trends assumption. Without additional years of pre-treatment data, these interpretations cannot be distinguished.

The absence of 2020 data creates an additional challenge for the event study interpretation. The COVID-19 pandemic began in early 2020, and the Census Bureau did not release standard 1-year PUMS data for that year due to data collection disruptions. This missing year is precisely when pandemic-era migration shifts were most acute, and its absence prevents examining whether migration patterns were already changing before SF 0018 was enacted. If Wyoming experienced a surge in in-migration during 2020 (as anecdotal evidence of urban-to-rural migration during the pandemic suggests), the 2021 estimate would capture both continued pandemic effects and any ULR-specific effects, conflating the two.

\subsection{Sample Size Limitations}

The statistical power of this analysis is fundamentally limited by Wyoming's small population. Table 4 shows the sample breakdown by state and year for licensed workers.

\begin{table}[H]
\centering
\caption{Sample Sizes: Licensed Workers by State and Year}
\begin{tabular}{lrrrrr}
\toprule
State & 2018 & 2019 & 2021 & 2022 & Total \\
\midrule
Wyoming & 153 & 161 & 125 & 146 & 585 \\
Idaho & 431 & 476 & 414 & 558 & 1,879 \\
Utah & 757 & 827 & 816 & 893 & 3,293 \\
Nebraska & 551 & 587 & 540 & 600 & 2,278 \\
South Dakota & 271 & 269 & 287 & 291 & 1,118 \\
North Dakota & 259 & 247 & 230 & 243 & 979 \\
\bottomrule
\end{tabular}
\end{table}

With approximately 150 licensed workers per year in Wyoming, detecting a statistically significant effect would require either a very large true effect or a much longer time series. The interstate in-migration outcome is particularly challenging because only a small fraction of workers (4-5\%) move across state lines in any given year. Among 150 licensed workers, we would expect only 6-8 interstate in-migrants, making year-to-year estimates highly sensitive to whether a few additional movers happen to be captured in the sample.

To contextualize the power limitations, consider a hypothetical minimum detectable effect (MDE) calculation. If we assume that the true baseline migration rate is 4.5\% and that we observe approximately 150 licensed workers per year in Wyoming, then a two-sample comparison of pre vs. post rates would require an effect size of approximately 5-8 percentage points to achieve 80\% power at the 5\% significance level, depending on assumptions about within-state correlation. This MDE is far larger than the effects we would reasonably expect from a licensing reform---even the most optimistic predictions based on the prior literature would suggest effects of 1-2 percentage points at most. The analysis is thus underpowered to detect policy-relevant effect sizes, and the observed estimates should be interpreted as highly uncertain point estimates rather than as precise measures of the true effect.

The power problem is exacerbated by the short time series (only 2 pre-treatment and 2 post-treatment years) and the absence of 2020. With only four years of data, we cannot distinguish secular trends from treatment effects, and random year-to-year variation appears as noise that swamps any signal. A longer time series would both increase sample sizes and allow for more credible parallel trends testing, but this requires waiting for additional years of PUMS data to become available. The 5-year PUMS, which pools data across five years to increase sample sizes, could provide an alternative, though at the cost of less precise timing around the policy change.

\subsection{Robustness and Sensitivity Analysis}

Given the challenges posed by small samples and the COVID-19 pandemic's disruption of migration patterns, I conduct several robustness checks to assess the sensitivity of the main findings. These checks examine whether the results are driven by specific methodological choices or sample restrictions.

First, I examine sensitivity to the definition of licensed occupations. The primary analysis uses a broad definition that includes all occupations typically requiring state licensure according to the Institute for Justice's License to Work database and other sources. An alternative narrow definition restricts the sample to occupations where licensure requirements are nearly universal across states (e.g., registered nurses, physical therapists, attorneys, accountants) rather than those where requirements vary (e.g., interior designers, travel agents, locksmiths). Using the narrow definition reduces the Wyoming sample to approximately 80 licensed workers per year, further limiting statistical power but focusing on occupations where licensing is unambiguously binding. Under this narrow definition, the DiD estimate is 0.62 percentage points, slightly larger than the baseline estimate but with even wider implicit confidence intervals. The placebo test for unlicensed workers remains positive (0.91 percentage points), so the qualitative conclusions are unchanged: positive point estimates for both licensed and unlicensed workers, with the unlicensed estimate exceeding the licensed estimate.

Second, I examine sensitivity to the choice of control states. The primary specification uses five neighboring states (Idaho, Utah, Nebraska, South Dakota, North Dakota) that did not adopt ULR during the study period. An alternative specification uses a synthetic control approach, weighting control states to match Wyoming's pre-treatment licensed worker migration rate. The synthetic weights assign approximately 40\% to South Dakota, 30\% to North Dakota, 20\% to Nebraska, and 10\% to Idaho and Utah combined, reflecting the fact that these smaller, less urbanized states more closely resemble Wyoming's baseline migration patterns. Using the synthetic control weights, the DiD estimate is 0.31 percentage points, smaller than the baseline estimate but still positive. The placebo remains positive at 0.85 percentage points. These results suggest that the main findings are not highly sensitive to the specific composition of the control group, though all specifications share the limitation that the placebo exceeds the treatment estimate.

Third, I examine sensitivity to sample restrictions on age and labor force participation. The primary analysis includes working-age adults (ages 25-64) who are in the labor force. An alternative specification expands the sample to include all adults aged 18-70, including those not currently in the labor force, to capture recent entrants, near-retirees, and workers who may have temporarily exited the labor force due to licensing barriers. This expanded sample increases the Wyoming licensed worker count to approximately 200 per year. The DiD estimate under this specification is 0.39 percentage points, slightly smaller than the baseline, with a placebo of 0.82 percentage points. Restricting instead to prime-age workers (ages 25-54) yields a DiD estimate of 0.55 percentage points with a placebo of 1.12 percentage points. Across these specifications, the pattern persists: positive point estimates with placebo effects of similar or larger magnitude.

Fourth, I examine whether the results are sensitive to the weighting scheme. The primary analysis uses PUMS person weights (PWGTP) to produce population-representative estimates. An alternative unweighted specification treats each observation equally, which changes the effective sample composition by giving more weight to states with higher sampling rates. Under the unweighted specification, the DiD estimate is 0.52 percentage points with a placebo of 0.89 percentage points, qualitatively similar to the weighted results. The consistency across weighted and unweighted specifications suggests that the findings are not driven by unusual weighting patterns or outlier observations with extreme weights.

Fifth, I conduct a permutation test to assess whether the observed DiD estimate could have arisen by chance under the null hypothesis of no treatment effect. I randomly reassign Wyoming's treatment status to each of the five control states in turn, compute the placebo DiD estimate treating that state as treated and Wyoming as a control, and compare the distribution of these placebo estimates to the actual Wyoming estimate. The Wyoming DiD estimate of 0.48 percentage points ranks third among the six states (including Wyoming), with Idaho showing a larger placebo estimate (0.71 pp) and South Dakota showing a similar estimate (0.45 pp). This ranking suggests that the Wyoming estimate is not unusually large relative to what could occur by chance in any of the control states, consistent with the interpretation that the observed effect may reflect general regional trends rather than a Wyoming-specific treatment effect. However, the permutation distribution is itself noisy with only five placebo assignments, so this test provides only weak evidence against the null.

Sixth, I examine heterogeneity by whether workers were likely to have moved from states that also had ULR policies by 2021. If Wyoming's ULR primarily benefits workers moving from states without their own ULR (where they would face relicensing barriers in the destination state), we might expect smaller effects for workers originating from other ULR states. However, this analysis is limited by the inability to identify source states precisely---MIGSP indicates state of residence one year ago but only for workers who moved, and many workers are non-movers. Among the small number of interstate movers in the Wyoming sample, I compare those whose prior state had enacted ULR by 2021 (Arizona, Montana, Pennsylvania, and others) versus those from non-ULR states. The sample sizes are too small for meaningful statistical inference (approximately 3-5 movers per category per year in Wyoming), but the pattern does not suggest obvious heterogeneity. This null finding is consistent with either no true heterogeneity or insufficient power to detect heterogeneity that exists.

The robustness checks collectively paint a consistent picture: across alternative specifications, the DiD estimate for licensed workers is positive but modest, and the placebo estimate for unlicensed workers is of similar or larger magnitude. No specification produces a clear pattern in which the licensed worker effect substantially exceeds the unlicensed worker effect, as would be expected if ULR were driving the observed migration changes. The consistency of this pattern across specifications strengthens the interpretation that general migration trends to Wyoming, rather than ULR-specific effects, explain the observed increases in licensed worker migration.

\section{Discussion}

\subsection{Interpretation}

The results provide suggestive but not definitive evidence that Wyoming's universal license recognition law affected licensed worker migration. The primary DiD estimate of +0.48 percentage points is positive and economically meaningful (an 11\% increase relative to baseline), but the larger positive effect for unlicensed workers raises serious concerns about confounding from general migration trends.

Several interpretations are consistent with the data. First, ULR may have had a genuine positive effect on licensed worker migration, but that effect is partially masked by concurrent increases in overall migration to Wyoming. If licensed and unlicensed workers both became more likely to move to Wyoming post-2020 for reasons unrelated to licensing (e.g., COVID-era migration to low-density states), the placebo would show a positive effect even if ULR also had an independent positive effect on licensed workers. The smaller magnitude for licensed workers (0.48 vs. 0.98) could reflect this, though it could also simply reflect sampling variation.

Second, the results may reflect pure noise. With small samples and large year-to-year volatility, the estimates have wide implicit confidence intervals. A true null effect is not ruled out.

Third, ULR may have effects that are not well-captured by one-year migration rates. Oh and Kleiner (2025) find that ULR increases out-of-state practice through temporary arrangements and telehealth rather than permanent relocation. Licensed workers in other states may be serving Wyoming clients remotely or on temporary licenses without establishing Wyoming residence. For occupations like cosmetology and barbering, where services must be provided in person, this mechanism is less relevant than for healthcare, but it may apply to occupations like real estate or insurance, where much business can be conducted remotely.

A fourth interpretation relates to the asymmetry between the treated and control groups. Wyoming is distinctive in many ways---it is the least populous state, has a heavily energy-dependent economy, and has unique cultural and geographic characteristics. The control states (Idaho, Utah, Nebraska, South Dakota, North Dakota) are more diverse and collectively much larger. If general economic conditions or migration pressures affected Wyoming differently than the control states during this period, the DiD estimate would be biased even if ULR had no effect. The placebo test finding that unlicensed workers showed larger migration increases than licensed workers is consistent with this interpretation: Wyoming may have become more attractive to all workers during the pandemic era, with the effect showing up more strongly in the larger unlicensed worker sample.

Fifth, there may be important heterogeneity within the licensed worker category that is masked by the aggregate analysis. Some licensed occupations covered by Wyoming's ULR (such as cosmetology and barbering) are geographically sticky---workers develop local client relationships and may be reluctant to move regardless of licensing barriers. Other occupations (such as real estate or insurance) may be more mobile by nature. Similarly, younger workers and workers without strong local ties may respond more to reduced licensing barriers than older workers or those with families and established practices. The small sample sizes prevent reliable heterogeneity analysis, but the aggregate null finding may mask positive effects for some subgroups that are offset by null or negative effects for others.

\subsection{Limitations}

This analysis faces several limitations that should temper interpretation.

First, the sample size is small, limiting statistical power. Wyoming's population of under 600,000 yields only a few hundred licensed workers per year in the PUMS. Detecting moderate effects would require a much longer time series or administrative data with the full population.

Second, the treatment timing coincides with major macroeconomic disruptions. The COVID-19 pandemic triggered large shifts in migration patterns, with urban-to-rural flows increasing substantially. Wyoming's relatively low population density and minimal COVID restrictions may have attracted migrants for reasons unrelated to licensing policy. The absence of 2020 data (due to ACS data collection issues) prevents examining the precise timing of any changes.

Third, licensed occupation status is imputed from occupation codes rather than observed directly. Some workers coded in ``licensed'' occupation categories may not actually hold licenses, and the definitions of licensed occupations may not precisely match the occupations covered by Wyoming's ULR law.

Fourth, the control group may be imperfect. While the selected states did not enact ULR during the study period, they may have implemented other reforms (such as licensing compacts for specific occupations or expedited processes for military spouses) that affected migration patterns.

\subsection{Implications}

Despite these limitations, the analysis offers some lessons for future research and policy. The placebo test highlights the importance of control groups that isolate the specific mechanism of interest. Studies of licensing reforms should routinely include unlicensed workers as a comparison group to distinguish licensing-specific effects from general labor market trends. This methodological recommendation applies not only to migration outcomes but to other outcomes such as wages, employment, and hours worked, where general economic conditions could confound estimates of licensing-specific effects.

The small sample sizes illustrate the challenges of evaluating state-level policies in small states using survey microdata. Administrative data from licensing boards---recording actual license applications and issuances---would provide direct evidence on whether ULR increases out-of-state applications. Several states with ULR laws (notably Arizona) have published such data, showing thousands of licenses issued under universal recognition provisions. Linking licensing board data to migration outcomes could provide more definitive evidence, though such linkages raise privacy concerns and may require researcher access to restricted-use data. An alternative approach would be to study ULR effects using multi-state panels that pool treatment variation across the approximately 20 states that had adopted ULR by 2023. Such an analysis would have substantially greater statistical power and could exploit the staggered timing of ULR adoption across states, though it would require careful attention to the methodological issues raised by recent work on staggered DiD (Callaway and Sant'Anna, 2021; Goodman-Bacon, 2021; Baker et al., 2022).

The volatility in the event study suggests that multi-year or multi-state evaluations will be more informative than single-state, short-term analyses. As more states enact ULR and as post-treatment time accumulates, researchers will be better positioned to identify effects with reasonable precision. The current study should be viewed as an early, exploratory analysis that documents the empirical challenges and establishes a framework for future work rather than as providing definitive evidence about ULR effectiveness.

The policy implications of this analysis are necessarily modest given the inconclusive results. The point estimate suggests that ULR may increase licensed worker migration, but the confidence intervals (implicit in the noisy event study and concerning placebo) span both positive and negative effects. Policymakers considering ULR should not expect large migration responses based on this evidence alone. However, the absence of strong evidence for large migration effects does not necessarily argue against ULR adoption. The welfare case for licensing reforms rests not only on migration responses but on the direct benefits to workers who face reduced barriers when they do choose to move, and on the broader efficiency gains from a more integrated national labor market. Even if few workers respond to reduced barriers on the margin, those who do may experience substantial benefits, and the option value of easier mobility is real even when most workers do not exercise it.

Furthermore, ULR may have effects beyond migration that are not captured in this analysis. If licensing barriers deterred some workers from entering licensed occupations in the first place (knowing that their credentials would not transfer if they ever needed to move), ULR could increase occupational entry over the long run. These effects would take years to materialize and would not appear in short-term migration data. Similarly, ULR may affect wages and employment outcomes even without large migration effects, if the threat of out-of-state competition disciplines local labor markets or if reduced barriers increase within-state labor supply.

\section{Conclusion}

This paper examines whether Wyoming's 2021 universal license recognition law increased interstate migration of licensed workers. Using Census PUMS data and a difference-in-differences design, I find a positive point estimate (+0.48 percentage points) for licensed worker in-migration, but a larger positive effect for unlicensed workers suggests that general migration trends to Wyoming may explain much of the observed increase. The event study reveals substantial year-to-year volatility that complicates inference, and the small sample sizes inherent to studying a state with under 600,000 residents further limit the precision of the estimates.

The results do not provide strong evidence either for or against a meaningful ULR effect on migration. The analysis is constrained by Wyoming's small population, the pandemic-induced disruptions to migration patterns, and the coincidental timing of the policy change with broader economic shifts. The year 2020 is missing from the analysis due to Census data collection disruptions, creating a gap in the time series precisely when pandemic-era migration was most volatile. Future research using administrative licensing data or pooling multiple ULR-adopting states may provide clearer answers about the mobility effects of these policies.

Several directions for future research emerge from this analysis. First, administrative data from licensing boards would provide a more direct measure of out-of-state license applications and issuances than can be inferred from survey data. States like Arizona have published utilization statistics showing thousands of licenses issued under ULR provisions, but these counts do not distinguish between workers who permanently relocated versus those who obtained licenses for temporary or remote practice. Linking licensing board records to migration data would clarify this distinction. Second, as more states adopt ULR and post-treatment periods lengthen, pooled analyses across multiple treated states will offer greater statistical power. The staggered adoption of ULR across approximately 20 states provides variation that could be exploited using modern staggered DiD methods such as those developed by Callaway and Sant'Anna (2021), though the close timing of many adoptions (2019-2022) may still limit precision. Third, the heterogeneity in ULR policy design—some states require residency, some exclude specific occupations, some impose ``substantially equivalent'' standards tests—creates opportunities for studying which design features most effectively reduce mobility barriers.

The policy implications are cautious but still meaningful. Universal license recognition laws remove a genuine barrier to interstate mobility for licensed workers, even if the resulting migration response is modest or difficult to detect. The welfare case for ULR does not rest solely on large migration effects. Workers who face unexpected relocation needs—such as military spouses, those following a partner's job transfer, or those caring for aging parents in another state—benefit from the option to move without relicensing delays, even if most workers never exercise that option. The option value of easier mobility is real even when observed migration flows change little.

From a broader policy perspective, the fragmentation of occupational licensing across 50 states creates inefficiencies that ULR partially addresses. The historical rationale for state-specific licensing---that local knowledge and conditions require locally tailored standards---is increasingly difficult to justify when many licensed occupations involve generic skills that do not vary meaningfully across state lines. A cosmetologist's ability to safely color hair or perform a manicure does not depend on whether she practices in Wyoming or Colorado. ULR represents a pragmatic reform that preserves state control over licensing standards while removing the most obvious burden on worker mobility. Future research will help clarify whether these reforms achieve their goals of expanding economic opportunity for licensed workers.

A final consideration relates to the broader context of labor market regulation and worker mobility in the United States. Licensing is one of many factors that affect workers' ability and willingness to move across state lines. Housing costs, particularly in high-productivity metropolitan areas, have been identified as a major barrier to geographic mobility (Ganong and Shoag, 2017). Zoning restrictions that limit housing supply in desirable locations create price differentials that may dwarf the costs of relicensing. State-specific professional networks and job search frictions also matter, as workers in many occupations find jobs through local connections rather than national job markets. In this broader context, licensing reform may be a necessary but not sufficient condition for restoring geographic mobility in the American labor market. Even if ULR completely eliminates licensing-related barriers, workers may still face substantial obstacles to moving across state lines.

The results of this study should be interpreted within these limitations while recognizing the broader importance of understanding how policy reforms affect labor market outcomes. The COVID-19 pandemic created unprecedented disruptions to migration patterns that will take years to fully understand. Future research using longer time series, multiple states, and administrative data will be essential for building a robust evidence base on the effects of universal license recognition and related reforms. In the meantime, policymakers considering ULR should weigh the theoretical benefits of reduced barriers against the uncertainty about empirical magnitudes, while recognizing that the direct benefits to workers who face reduced relicensing costs are real even if the aggregate migration response is modest.

\newpage
\section*{References}

\begin{enumerate}

\item Barrios, J.M. (2020). ``Occupational licensing and accountant quality: Evidence from the 150-hour rule.'' \textit{Journal of Accounting Research}.

\item Carpenter, D.M., Knepper, L., Erickson, A.C., and Ross, J.K. (2012). ``License to Work: A National Study of Burdens from Occupational Licensing.'' Institute for Justice.

\item Carpenter, D.M., Knepper, L., Sweetland, K., and McDonald, J. (2017). ``License to Work: Second Edition.'' Institute for Justice.

\item Friedman, M. and Kuznets, S. (1945). \textit{Income from Independent Professional Practice}. NBER.

\item Harrell, M.C., Lim, N., Castaneda, L.W., and Golinelli, D. (2004). ``Working Around the Military: Challenges to Military Spouse Employment and Education.'' RAND Corporation.

\item Johnson, J.E. and Kleiner, M.M. (2020). ``Is Occupational Licensing a Barrier to Interstate Migration?'' \textit{American Economic Journal: Economic Policy}, 12(3), 347-373.

\item Kleiner, M.M. (2015). ``Reforming Occupational Licensing Policies.'' The Hamilton Project, Brookings Institution.

\item Kleiner, M.M. and Krueger, A.B. (2013). ``Analyzing the Extent and Influence of Occupational Licensing on the Labor Market.'' \textit{Journal of Labor Economics}, 31(2), S173-S202.

\item Kleiner, M.M. and Xu, M. (2020). ``Occupational Licensing and Labor Market Fluidity.'' NBER Working Paper No. 27568.

\item Mulligan, C.B. and Tsui, K.K. (2016). ``The Upside of Income Mobility.'' \textit{National Tax Journal}, 69(3), 521-546.

\item Oh, Y. and Kleiner, M.M. (2025). ``Does Universal Occupational Licensing Recognition Improve Patient Access? Evidence from Healthcare Utilization.'' NBER Working Paper No. 34030.

\item Callaway, B. and Sant'Anna, P.H.C. (2021). ``Difference-in-Differences with Multiple Time Periods.'' \textit{Journal of Econometrics}, 225(2), 200-230.

\item Goodman-Bacon, A. (2021). ``Difference-in-Differences with Variation in Treatment Timing.'' \textit{Journal of Econometrics}, 225(2), 254-277.

\item Thornton, R.J. and Timmons, E.J. (2013). ``Licensing One of the World's Oldest Professions: Massage.'' \textit{Journal of Law and Economics}, 56(2), 371-388.

\item Ingram, S.J. (2019). ``Occupational Licensing and the Earnings Premium.'' \textit{Journal of Labor Economics}, 37(S2), S509-S540.

\item Baker, A.C., Larcker, D.F., and Wang, C.C.Y. (2022). ``How Much Should We Trust Staggered Difference-in-Differences Estimates?'' \textit{Journal of Financial Economics}, 144(2), 370-395.

\item Ruggles, S., Flood, S., Goeken, R., Schouweiler, M., and Sobek, M. (2023). ``IPUMS USA: Version 13.0.'' Minneapolis, MN: IPUMS.

\item Molloy, R., Smith, C.L., and Wozniak, A. (2011). ``Internal Migration in the United States.'' \textit{Journal of Economic Perspectives}, 25(3), 173-196.

\item Ganong, P. and Shoag, D. (2017). ``Why Has Regional Income Convergence in the U.S. Declined?'' \textit{Journal of Urban Economics}, 102, 76-90.

\end{enumerate}

\section*{Data Availability}

The analysis uses Census PUMS data accessed via the Census Bureau API (api.census.gov). Replication code is available in the paper's GitHub repository. The data are publicly available and can be accessed without an API key for basic queries. The replication package includes Python scripts for data fetching, processing, and analysis, as well as the LaTeX source for this paper.

\section*{Acknowledgments}

This paper was produced by the Autonomous Policy Evaluation Project (APEP), a virtual research institute that uses automated methods to produce rigorous empirical research on public policies. The analysis was pre-registered prior to examining results, and the pre-analysis plan is available in the paper's repository. The author thanks the Census Bureau for maintaining open access to PUMS data through its API.

\end{document}
