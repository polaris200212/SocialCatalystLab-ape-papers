\begin{table}[htbp]
\centering
\caption{Effect of Must-Access PDMP Mandates on State Employment Outcomes}
\label{tab:main}
\begin{tabular}{lccc}
\toprule
 & CS-DiD & CS-DiD & TWFE \\
 & (Not-Yet-Treated) & (Never-Treated) & \\
\midrule
\multicolumn{4}{l}{\textit{Panel A: Log Employment}} \\[3pt]
ATT / Coefficient & 0.0036 & 0.0100 & 0.0033 \\
 & (0.0079) & (0.0078) & (0.0055) \\[6pt]
\multicolumn{4}{l}{\textit{Panel B: Unemployment Rate (pp)}} \\[3pt]
ATT / Coefficient & $-$0.242 & $-$0.399 & $-$0.169 \\
 & (0.293) & (0.369) & (0.214) \\[6pt]
\midrule
State FE & \checkmark & \checkmark & \checkmark \\
Year FE & \checkmark & \checkmark & \checkmark \\
N (state-years) & 850 & 850 & 850 \\
States & 50 & 50 & 50 \\
Treated states & 46 & 46 & 46 \\
Control group & Not-yet-treated & Never-treated & All \\
\bottomrule
\end{tabular}
\begin{tablenotes}[flushleft]
\small
\item \textit{Notes:} Column (1) reports Callaway and Sant'Anna (2021) doubly-robust ATT estimates
using not-yet-treated states as the comparison group (preferred specification).
Column (2) uses never-treated states (KS, MO, NE, SD) as the comparison group (sensitivity check).
Column (3) reports standard TWFE estimates with state and year fixed effects.
CS standard errors computed via multiplier bootstrap (1,000 iterations);
TWFE standard errors clustered at the state level.
The not-yet-treated comparison group is preferred because the thin four-state never-treated group
produces spurious pre-trend violations (see Section~6.1).
* $p < 0.1$, ** $p < 0.05$, *** $p < 0.01$.
\end{tablenotes}
\end{table}
