\documentclass[12pt]{article}

% UTF-8 encoding and fonts
\usepackage[utf8]{inputenc}
\usepackage[T1]{fontenc}
\usepackage{lmodern}

% Page setup
\usepackage[margin=1in]{geometry}
\usepackage{setspace}
\onehalfspacing

% Typography
\usepackage{microtype}

% Math and symbols
\usepackage{amsmath,amssymb}

% Graphics
\usepackage{graphicx}
\usepackage{float}
\usepackage{subcaption}

% Tables
\usepackage{booktabs}
\usepackage{array}
\usepackage{multirow}
\usepackage{threeparttable}
\usepackage{longtable}
\usepackage{pdflscape}
\usepackage{siunitx}
\sisetup{detect-all=true, group-separator={,}, group-minimum-digits=4}

% Bibliography
\usepackage{natbib}
\bibliographystyle{aer}

% Hyperlinks
\usepackage{hyperref}
\hypersetup{
    colorlinks=true,
    linkcolor=blue,
    citecolor=blue,
    urlcolor=blue
}
\usepackage[nameinlink,noabbrev]{cleveref}

% Captions
\usepackage{caption}
\captionsetup{font=small,labelfont=bf}

% Section formatting
\usepackage{titlesec}
\titleformat{\section}{\large\bfseries}{\thesection.}{0.5em}{}
\titleformat{\subsection}{\normalsize\bfseries}{\thesubsection}{0.5em}{}

% Custom commands
\newcommand{\E}{\mathbb{E}}
\newcommand{\Var}{\text{Var}}
\newcommand{\Cov}{\text{Cov}}
\newcommand{\ind}{\mathbb{I}}
\newcommand{\sym}[1]{\ifmmode^{#1}\else\(^{#1}\)\fi}

\title{Car Ownership, Housing Tenure, and Educational Achievement:\\Urban-Rural Disparities in Swedish Municipalities}
\author{APEP Autonomous Research\thanks{Autonomous Policy Evaluation Project. Correspondence: scl@econ.uzh.ch} \and @anonymous}
\date{\today}

\begin{document}

\maketitle

\begin{abstract}
\noindent
How do transportation infrastructure and housing market characteristics relate to educational outcomes? Using administrative data from Sweden's 290 municipalities, we examine the relationship between car ownership rates, housing tenure composition, and educational achievement as measured by Grade 9 merit points. Exploiting substantial cross-municipality variation in car ownership (272--671 cars per 1,000 inhabitants) and housing tenure (rental share ranging from 5\% to 56\%), we document a strong urban-rural educational gradient. Municipalities with lower car ownership---a proxy for urbanity and public transit accessibility---have significantly higher educational achievement: a 100-car reduction per 1,000 inhabitants is associated with 8.1 higher merit points in bivariate analysis, and 7.7 points (0.61 standard deviations) in the full model with controls. This relationship persists in the full model with housing tenure, teacher qualifications, county fixed effects, and year fixed effects ($R^2 = 0.31$). We find that cooperative housing dominance, characteristic of Swedish urban areas, correlates positively with merit points ($r = 0.40$), while rental-dominant municipalities show lower achievement. Stockholm County leads with 238.4 mean merit points versus 212.9 in Örebro County. These descriptive patterns suggest that urban advantages in educational infrastructure, school density, and residential sorting may drive substantial geographic inequality in Swedish education.
\end{abstract}

\vspace{1em}
\noindent\textbf{JEL Codes:} I24, I28, R23, H75 \\
\noindent\textbf{Keywords:} educational achievement, car ownership, housing tenure, urban-rural disparities, Sweden, municipalities

\newpage

\section{Introduction}

Educational inequality across geographic areas represents one of the most persistent challenges facing modern welfare states. While considerable research has documented disparities between urban and rural schools in terms of resources, teacher quality, and student outcomes, less attention has been paid to how broader community characteristics---particularly transportation infrastructure and housing markets---may contribute to or reflect these educational gaps \citep{hanushek2003does, chetty2014united}. In Sweden, a country celebrated for its egalitarian welfare model and universal education system, significant variation in educational achievement persists across municipalities, raising questions about the mechanisms underlying geographic educational inequality \citep{holmlund2014swedish, groenqvist2017family}.

This paper examines the relationship between car ownership patterns, housing tenure composition, and educational achievement across Swedish municipalities. Car ownership serves as a useful proxy for multiple community characteristics: it reflects urbanity, public transportation availability, household wealth, and mobility patterns that may influence educational access and school choice \citep{blumenberg2004engendering}. Housing tenure---specifically the mix of rental, owner-occupied, and cooperative housing---captures residential stability, socioeconomic composition, and housing market dynamics that shape neighborhood composition and school peer effects \citep{holmqvist2015segregation, andersson2007housing}.

We utilize comprehensive administrative data from the Kolada database, Sweden's municipal statistics platform, covering all 290 municipalities for the years 2015--2016. Our outcome measure is the average merit points (meritvärde) achieved by Grade 9 students, the standardized measure used for upper secondary school admissions in Sweden. This measure excludes recent immigrants, ensuring comparability across municipalities with varying immigration patterns. We combine these educational outcomes with 2013 baseline data on car ownership (cars per 1,000 inhabitants) and housing tenure composition (percentage in rental, cooperative, and owner-occupied housing).

Our main finding is a strong negative relationship between car ownership and educational achievement: municipalities with lower car ownership rates have substantially higher merit points. In the bivariate model, a 100-car reduction per 1,000 inhabitants is associated with approximately 8.1 additional merit points, equivalent to 0.65 standard deviations of the municipal distribution. In the full model with housing tenure, teacher qualifications, and county fixed effects, this association remains strong at 7.7 points (0.61 SD). The full model explains 31\% of cross-municipal variation in educational outcomes.

We document substantial heterogeneity across Sweden's 21 counties (län). Stockholm County, characterized by low car ownership (414 cars per 1,000) and high cooperative housing prevalence, leads with average merit points of 238. In contrast, Örebro County, with higher car ownership and lower achievement, averages 213 merit points---a 25-point gap representing nearly two standard deviations of the municipal distribution. This geographic pattern suggests that Sweden's educational equity, despite its strong welfare state, exhibits meaningful urban-rural stratification.

Housing tenure composition provides additional insight into these patterns. Municipalities dominated by cooperative housing (bostadsrätt)---a distinctly Swedish tenure form combining elements of ownership and collective governance---show the highest educational achievement. Cooperative-dominant municipalities average 233 merit points, compared to 224 for rental-dominant and 221 for owner-dominant municipalities. This pattern likely reflects the urban concentration of cooperative housing and the socioeconomic selection into this tenure form, rather than a direct causal effect of housing type on education \citep{andersson2007housing, magnusson2022housing}.

The correlation structure among our key variables reveals important patterns. Car ownership is strongly negatively correlated with cooperative housing ($r = -0.69$) and positively correlated with owner-occupied housing ($r = 0.70$), reflecting the urban-rural housing market divide. Merit points correlate positively with cooperative housing share ($r = 0.40$) and negatively with car ownership ($r = -0.39$). Teacher qualification rates show surprisingly weak correlation with merit points ($r = 0.01$) after accounting for other factors, suggesting that teacher quality differences alone do not explain the urban-rural achievement gap.

This paper contributes to several literatures. First, we add to the growing body of work on geographic educational inequality \citep{chetty2014united, chetty2018opportunity}, demonstrating that even in a highly egalitarian welfare state like Sweden, substantial spatial variation in educational outcomes persists. Second, we contribute to research on the role of transportation infrastructure in educational access \citep{blumenberg2008planning}, showing that car ownership patterns---as a marker of transportation modes and urbanity---strongly predict educational outcomes at the municipal level. Third, we extend the literature on housing and education \citep{schwartz2014housing}, documenting how Sweden's distinctive cooperative housing sector relates to educational achievement patterns.

Our analysis is descriptive rather than causal. The strong correlations we document could reflect multiple mechanisms: urban schools may offer better educational infrastructure and more school choice; urban labor markets may attract more educated families; cooperative housing markets may select high-socioeconomic-status households; or public transportation availability may enable better access to extracurricular activities and educational resources. Disentangling these mechanisms would require exogenous variation that is not available in our cross-sectional municipal data. Nevertheless, documenting the magnitude and robustness of these correlations is a necessary first step toward understanding geographic educational inequality in Sweden.

The remainder of the paper proceeds as follows. Section 2 provides institutional background on Swedish education, housing markets, and transportation. Section 3 describes our data sources and variable construction. Section 4 presents our empirical approach. Section 5 reports results, and Section 6 discusses implications and limitations.


\section{Institutional Background}

\subsection{Swedish Education System}

Sweden's education system is characterized by universal access, local municipal governance, and a standardized national curriculum. Compulsory schooling (grundskola) spans Grades 1--9, with students typically completing Grade 9 at age 15--16. Upon completion, students receive a final grade summary expressed in merit points (meritvärde), calculated as the sum of their 16--17 best subject grades on a scale where each subject contributes 2.5--20 points depending on the grading system in effect \citep{holmlund2014swedish}.

Merit points serve as the primary admission criterion for upper secondary education (gymnasium), making them a high-stakes measure with significant consequences for students' educational trajectories. The merit point system was reformed in 2011, transitioning from a letter grade system (MVG/VG/G/IG) to a numeric system (A-F, with E as passing). Our data from 2015--2016 reflect the newer system, with maximum possible scores around 340 points for students taking all available subjects. The average student typically scores between 200 and 240 points, with substantial variation reflecting both individual ability and school quality.

The Swedish grading system operates on a criterion-referenced basis, meaning that grades should reflect absolute mastery of curriculum objectives rather than relative standing. However, research has documented grade inflation and inconsistencies across municipalities and schools \citep{vlachos2018grading}, which may introduce measurement error in our outcome variable. Nevertheless, merit points remain the most comprehensive and comparable measure of educational achievement across Swedish municipalities.

Municipalities bear primary responsibility for operating schools and allocating resources, though they must follow national curriculum guidelines. This decentralized structure creates potential for cross-municipal variation in educational quality and resources. Municipal education budgets depend heavily on local tax bases, which vary with economic conditions and demographic composition. Wealthier municipalities with stronger tax bases can invest more in school facilities, teacher salaries, and supplementary programs, potentially creating systematic differences in educational inputs across municipalities.

Sweden's school choice reforms in the 1990s introduced quasi-market mechanisms, allowing families to choose among municipal and independent (``friskola'') schools, with funding following students \citep{bolin2023school}. These reforms may have intensified geographic sorting, as families with resources can exercise more effective school choice. The share of students attending independent schools has grown substantially, reaching approximately 15\% of compulsory school students nationally, though this share varies dramatically across municipalities---from near zero in rural areas to over 30\% in some urban municipalities.

The introduction of school choice created incentives for residential sorting. Families seeking high-quality education can either move to municipalities with good schools, choose among schools within their municipality, or---in larger urban areas---select independent schools. This sorting mechanism means that observed cross-municipal variation in educational outcomes reflects both ``treatment effects'' of municipal policies and ``selection effects'' of family sorting. Our analysis cannot disentangle these mechanisms but documents the resulting equilibrium patterns.

The merit point measure we use excludes recently arrived students (``nyanlända''), who are defined as students who have been in Sweden for fewer than four years. This exclusion is standard in Swedish education statistics and ensures comparability across municipalities with varying immigration patterns. Given Sweden's substantial refugee inflows during 2015--2016, this exclusion is particularly important for our analysis period. Immigration patterns vary across municipalities, with urban areas and certain regions receiving disproportionate shares of new arrivals. Without this exclusion, our results would partly reflect compositional differences in immigrant populations rather than underlying municipal characteristics.

\subsection{Swedish Housing Markets}

Sweden's housing market features a distinctive three-tier tenure structure: rental housing (hyresrätt), cooperative housing (bostadsrätt), and owner-occupied housing (äganderätt) \citep{magnusson2022housing}. Each form has distinct implications for residential stability, socioeconomic selection, and geographic mobility. Understanding these tenure forms is essential for interpreting the relationship between housing composition and educational outcomes.

Rental housing is regulated under a unique ``utility value'' (bruksvärde) system that sets rents through collective bargaining between tenant unions and landlords, resulting in below-market rents in high-demand urban areas. This creates substantial housing queues in cities like Stockholm, where wait times can exceed 10 years for desirable apartments \citep{andersson2007housing}. Rental housing is provided by both municipal housing companies (allmännyttan) and private landlords. Municipal housing companies, historically important in Swedish housing policy, were established to provide affordable housing and reduce class-based segregation. However, the allocation of municipal rental housing through queue systems may inadvertently favor established residents over newcomers, potentially creating barriers to residential mobility for families seeking better school districts.

The rent regulation system creates complex incentive structures. Because rents in attractive urban areas are below market-clearing levels, sitting tenants capture substantial housing ``wealth'' in the form of below-market rents. This creates incentives to remain in place rather than move, potentially reducing residential mobility in response to changing life circumstances including children's educational needs. Conversely, families without access to the regulated rental market face substantial barriers to entering high-demand urban areas, limiting their school choice options.

Cooperative housing (bostadsrätt) represents a distinctly Swedish middle ground between renting and owning. Residents purchase shares in a housing cooperative that owns the building, granting them exclusive rights to occupy specific units. Monthly fees cover building maintenance and mortgage payments on the cooperative's loans. The purchase price of cooperative shares is market-determined and can be substantial in desirable locations, particularly central Stockholm where prices routinely exceed 100,000 SEK per square meter. This creates a significant wealth barrier to entry.

Cooperative housing concentrates heavily in urban areas, particularly Stockholm, and typically attracts middle- and upper-middle-class households \citep{holmqvist2015segregation}. The socioeconomic selection into cooperative housing is substantial: access requires both savings for down payments and mortgage approval, screening out lower-income households. Additionally, cooperative boards may exercise informal selection in approving new members, potentially reinforcing socioeconomic homogeneity. The concentration of cooperative housing in high-performing school districts is thus both a cause and consequence of socioeconomic sorting.

Owner-occupied housing (äganderätt) consists primarily of single-family homes (villor) and townhouses (radhus), predominantly located in suburban and rural areas. This tenure form requires substantial down payments and correlates strongly with car dependency due to the spatial distribution of such housing. Swedish mortgage markets traditionally required 15-20\% down payments, though this requirement was relaxed in some periods. Owner-occupied housing represents the dominant tenure form in rural Sweden, where alternative housing options are limited.

The relationship between owner-occupied housing and car ownership reflects both necessity and preference. Single-family homes in rural areas lack public transportation access, making car ownership essential for daily activities including commuting, shopping, and---importantly---accessing schools. Families in rural owner-occupied housing thus face transportation costs and time burdens that may affect their capacity to support children's education through activities like homework help, parent-teacher meetings, and extracurricular involvement.

These tenure patterns create substantial geographic segregation. Urban cores are dominated by rental and cooperative housing, while suburban and rural areas feature predominantly owner-occupied housing. This spatial sorting has implications for school composition and peer effects, as housing tenure correlates with socioeconomic status and family structure. Research on Swedish residential segregation has documented increasing socioeconomic concentration since the 1990s, with high-income households increasingly concentrated in certain neighborhoods and municipalities \citep{andersson2007housing}. This segregation maps onto educational inequality through multiple channels: peer effects in schools, differential municipal resources, and variation in family capacity to support education.

The Swedish housing market also features distinctive patterns of intergenerational transmission. Parental housing wealth, particularly in the cooperative sector, enables transfers to children that facilitate their entry into desirable housing markets. This intergenerational dynamic may reinforce educational inequality across generations, as families with housing wealth can secure access to high-performing school districts for their children.

\subsection{Transportation and School Access}

Swedish municipalities are legally required to provide free school transportation (skolskjuts) to students whose homes are beyond specified distance thresholds from their assigned schools. These thresholds vary by grade level (typically 2 km for Grades F--3, 3 km for Grades 4--6, and 4 km for Grades 7--9) and can be adjusted for hazardous routes or students with disabilities \citep{skolverket2023skolskjuts}. The scope and generosity of skolskjuts provision varies across municipalities based on population density and local policies. Some municipalities offer transportation for shorter distances or extend eligibility to students attending non-assigned schools, while others provide only the legally mandated minimum.

The skolskjuts system represents a substantial municipal expenditure, particularly in rural areas where distances are greater and student populations are more dispersed. Transportation costs compete with other educational expenditures, creating trade-offs in municipal budgeting. Municipalities with extensive transportation needs may face constraints on spending for other educational inputs such as classroom materials, specialized teachers, or extracurricular programs. This fiscal trade-off could contribute to urban-rural educational disparities.

Car ownership rates vary dramatically across Swedish municipalities, ranging from approximately 272 cars per 1,000 inhabitants in urban Stockholm municipalities to over 670 cars per 1,000 in rural areas. This more than twofold variation reflects multiple factors: urban areas have dense public transportation networks (particularly Stockholm's extensive metro, commuter rail, and bus systems), higher parking costs, and more walkable neighborhood designs. Rural areas lack these alternatives, making car ownership essential for daily activities including school access, employment, healthcare, and shopping.

Sweden's public transportation infrastructure is highly developed in metropolitan areas. Stockholm's tunnelbana (metro) system carries over 300 million passengers annually, and the commuter rail network (pendeltåg) extends deep into surrounding municipalities. Gothenburg and Malmö have extensive tram and bus networks. This infrastructure enables car-free lifestyles in urban areas, with corresponding effects on household budgets, time allocation, and residential patterns. Families in well-connected urban areas can allocate time and money that would otherwise go to car ownership toward other investments, potentially including educational enrichment.

The relationship between transportation mode and educational engagement operates through several channels. First, car-dependent families in rural areas face substantial time costs for school-related activities. Attending parent-teacher conferences, picking up children from after-school activities, and transporting children to tutoring or enrichment programs all require car trips that may span considerable distances. The time burden of rural transportation may reduce parental capacity for educational involvement.

Second, transportation constraints may limit school choice in practice even when choice is formally available. While Swedish school choice law allows families to select any public or independent school, exercising this choice requires transportation. Urban families can often reach multiple schools via public transit, while rural families may face impractical distances to non-local schools. This asymmetry in effective school choice could contribute to urban-rural achievement gaps.

Third, car ownership correlates with access to educational and cultural resources beyond schools. Museums, libraries, universities, tutoring centers, and extracurricular activities cluster in urban areas. Urban students can access these resources independently via public transportation, while rural students depend on parental transportation. This differential access to enrichment opportunities may contribute to the educational gradient we document.

We use car ownership as a proxy for urbanity and transportation mode. Low car ownership municipalities typically offer extensive public transportation, walkable distances to schools and services, and dense concentrations of educational and cultural institutions. High car ownership municipalities tend to be more dispersed, with greater reliance on private vehicles and potentially longer distances to schools and educational resources. This proxy captures a bundle of urban characteristics rather than car ownership per se, and our estimates should be interpreted accordingly.

\subsection{Municipal Variation in Sweden}

Sweden's 290 municipalities exhibit remarkable variation in size, population, and economic conditions. At one extreme, Stockholm municipality contains over 900,000 inhabitants in a dense urban core with extensive infrastructure and diverse economic base. At the other extreme, the smallest municipalities have populations under 3,000, with limited local services and economies dominated by a few industries---often forestry, mining, or tourism.

This variation in municipal characteristics creates natural differences in educational contexts. Urban municipalities operate numerous schools, enabling school choice and competition. They have larger tax bases that support educational investments and can attract qualified teachers through both salary premiums and lifestyle amenities. Rural municipalities may operate only one or two schools, limiting choice and creating dependencies on individual school performance. Teacher recruitment in rural areas can be challenging, as qualified teachers often prefer urban postings for professional and personal reasons.

County (län) governments play important roles in coordinating services across municipalities and managing certain functions including healthcare and regional transportation. Sweden's 21 counties vary substantially in urbanization, with Stockholm County dominated by its metropolitan area while northern counties like Norrbotten and Västerbotten span vast areas with dispersed small municipalities. County-level variation in resources and coordination capacity may contribute to educational disparities, motivating our inclusion of county fixed effects in the regression analysis.


\section{Data}

\subsection{Data Sources}

We utilize three primary data sources from Sweden's Kolada municipal statistics database, operated by the Council on Municipal Analysis (Rådet för främjande av kommunala analyser, RKA). Kolada provides standardized municipal-level indicators across numerous domains, with data sourced from Statistics Sweden (SCB), the National Agency for Education (Skolverket), and other government agencies.

\textbf{Educational Outcomes.} Our primary outcome is the average merit points (meritvärde) for Grade 9 students, excluding recent immigrants (Kolada KPI N15566). This measure reflects the municipality-wide average of final grades for students completing compulsory schooling. We observe this outcome for 2015 and 2016.

\textbf{Car Ownership.} We measure car ownership as the number of registered cars per 1,000 inhabitants (Kolada KPI N07935). This includes all registered passenger vehicles in the municipality. We use 2013 data to ensure temporal precedence relative to our educational outcomes and to avoid reverse causality concerns.

\textbf{Housing Tenure.} We measure housing composition using three indicators: rental housing share (N07956), cooperative housing share (N07957), and owner-occupied housing share (N07958), each expressed as a percentage of total dwelling units. These data are also from 2013.

\textbf{Teacher Qualifications.} As a control variable, we include the percentage of teachers with formal teaching qualifications (behöriga lärare, Kolada KPI N15030), measured in 2015--2016.

\textbf{Municipality Metadata.} We obtain municipality names, identification codes, and county (län) assignments from the standard Swedish municipality registry. Sweden has 290 municipalities grouped into 21 counties.

\subsection{Sample Construction}

Our analysis sample consists of 290 municipalities observed over two years (2015 and 2016), yielding 580 municipality-year observations. We exclude regional aggregates and special administrative units that appear in some Kolada extracts. All 290 Swedish municipalities have complete data on our key variables.

Municipality sizes vary dramatically, from small rural municipalities with populations under 3,000 to Stockholm with over 900,000 inhabitants. Our analysis weights municipalities equally, treating each as a single observation. This approach emphasizes the variation in institutional conditions across municipalities rather than the experience of average Swedish students (which would require weighting by student population).

\subsection{Variable Definitions}

\textbf{Merit Points.} Our dependent variable is the municipality average of merit points for Grade 9 students excluding recent immigrants. The theoretical maximum is approximately 340 points. In our sample, municipality averages range from 172.1 to 261.6, with a mean of 222.9 and standard deviation of 12.5.

\textbf{Car Ownership.} Cars per 1,000 inhabitants, ranging from 271.7 in central Stockholm to 670.8 in the most rural municipalities. Mean: 530.2; SD: 63.2.

\textbf{Housing Tenure.} Rental housing share ranges from 4.8\% to 55.8\% (mean: 29.4\%). Cooperative housing ranges from 0\% (in many rural municipalities) to 64.5\% in urban centers (mean: 12.9\%). Owner-occupied housing ranges from 1.1\% to 86.0\% (mean: 57.7\%).

\textbf{Dominant Housing Type.} We classify each municipality by its dominant housing tenure: rental-dominant, cooperative-dominant, or owner-dominant, based on which tenure form has the largest share.

\textbf{Urban Proxy.} We create an urbanity classification based on car ownership thresholds: Urban (below 400 cars/1,000), Suburban (400--500), Semi-rural (500--600), and Rural (above 600). These thresholds are chosen to create substantively meaningful categories reflecting transportation infrastructure differences, not equal-sized groups. This classification correlates strongly with official urbanity measures but is available for all municipalities.

\subsection{Summary Statistics}

Table~\ref{tab:summary} presents summary statistics for our key variables. Swedish municipalities show substantial variation in both educational outcomes and our explanatory variables. The coefficient of variation for car ownership (0.12) is smaller than that for cooperative housing share (0.81), reflecting the near-absence of cooperative housing in many rural municipalities.

\begin{table}[H]
\centering
\caption{Summary Statistics (2015)}
\begin{threeparttable}
\begin{tabular}{lrrrrr}
\toprule
Variable & Mean & SD & Min & Max & N \\
\midrule
Merit Points (Grade 9) & 222.9 & 12.5 & 172.1 & 261.6 & 290 \\
Cars per 1,000 Inhabitants & 530.2 & 63.2 & 271.7 & 670.8 & 290 \\
Rental Housing (\%) & 29.4 & 8.2 & 4.8 & 55.8 & 290 \\
Owner-Occupied Housing (\%) & 57.7 & 14.0 & 1.1 & 86.0 & 290 \\
Cooperative Housing (\%) & 12.9 & 10.4 & 0.0 & 64.5 & 290 \\
Qualified Teachers (\%) & 86.0 & 5.5 & 69.7 & 98.4 & 290 \\
Student-Teacher Ratio & 11.9 & 1.2 & 8.7 & 15.7 & 290 \\
\bottomrule
\end{tabular}
\begin{tablenotes}[flushleft]
\small
\item Notes: Sample consists of all 290 Swedish municipalities in 2015 (cross-section). Regression analysis uses pooled 2015--2016 data (N=580). Merit points measured as municipality average for Grade 9 students excluding recent immigrants. Car ownership and housing tenure measured in 2013 to ensure temporal precedence.
\end{tablenotes}
\end{threeparttable}
\label{tab:summary}
\end{table}


\section{Empirical Strategy}

\subsection{Descriptive Framework}

Our analysis is descriptive rather than causal. We aim to document the magnitude and robustness of associations between transportation infrastructure (proxied by car ownership), housing market composition, and educational outcomes across Swedish municipalities. Given the cross-sectional nature of our data and the lack of exogenous policy variation, we do not claim to identify causal effects.

Our primary estimating equation is:
\begin{equation}
\text{Merit}_{m,t} = \alpha + \beta_1 \cdot \text{Cars}_{m,2013} + \beta_2 \cdot \text{Rental}_{m,2013} + X_{m,t}'\gamma + \delta_c + \tau_t + \epsilon_{m,t}
\label{eq:main}
\end{equation}

where $\text{Merit}_{m,t}$ is average merit points in municipality $m$ in year $t$, $\text{Cars}_{m,2013}$ is cars per 1,000 inhabitants (measured in 2013), $\text{Rental}_{m,2013}$ is rental housing share (2013), $X_{m,t}$ includes time-varying controls (teacher qualification rates), $\delta_c$ represents county fixed effects, and $\tau_t$ represents year fixed effects.

We estimate this equation using OLS, progressively adding controls to assess the sensitivity of our estimates. Standard errors are clustered at the municipality level to account for within-municipality correlation across years.

\subsection{Identification Concerns}

Several factors limit causal interpretation:

\textbf{Selection and Sorting.} Families sort into municipalities based on preferences, wealth, and labor market opportunities. Municipalities with better schools may attract higher-socioeconomic-status families, who are also more likely to live in low-car-ownership urban areas. This endogenous sorting means our estimates capture both any direct effect of transportation/housing and the selection of particular family types into different municipalities.

\textbf{Omitted Variables.} Many municipality characteristics correlate with both car ownership and educational outcomes: labor market composition, cultural institutions, private school availability, historical investments, and demographic structure. County fixed effects absorb some of this variation but cannot address within-county omitted variables.

\textbf{Reverse Causality.} While we use 2013 transportation and housing data to predict 2015--2016 educational outcomes, both reflect long-standing municipal characteristics. It is possible that municipalities with historically strong schools attracted residents who reduced car dependence, rather than the reverse.

\textbf{Measurement.} Merit points are measured with sampling error, particularly in small municipalities with few Grade 9 students. Car ownership is measured at the household level but assigned to municipality of registration, which may not perfectly capture driving patterns.

Despite these limitations, documenting the magnitude of these associations is valuable for understanding geographic educational inequality in Sweden and identifying patterns worthy of further investigation.


\section{Results}

\subsection{Car Ownership and Educational Achievement}

Figure~\ref{fig:scatter} displays the relationship between car ownership and merit points across Swedish municipalities. The negative relationship is striking: municipalities with lower car ownership systematically achieve higher educational outcomes. The OLS regression line has a slope of approximately $-0.081$, indicating that each additional 100 cars per 1,000 inhabitants is associated with 8.1 fewer merit points in the bivariate relationship.

\begin{figure}[H]
\centering
\includegraphics[width=0.9\textwidth]{figures/fig1_car_merit_scatter.pdf}
\caption{Car Ownership and Educational Achievement in Swedish Municipalities}
\label{fig:scatter}
\begin{minipage}{0.9\textwidth}
\small
\textit{Notes:} Each point represents one of 290 Swedish municipalities. Car ownership measured as cars per 1,000 inhabitants (2013). Merit points are municipality average for Grade 9 students excluding recent immigrants (2015). Colors indicate urbanity classification based on car ownership quartiles. The fitted line shows OLS regression with 95\% confidence interval.
\end{minipage}
\end{figure}

The scatter plot reveals substantial heterogeneity within urbanity categories. Some suburban municipalities achieve merit points comparable to urban centers, while others perform at rural levels. This heterogeneity suggests that urbanity alone does not determine educational outcomes---local factors and policies also matter.

\subsection{Urbanity Gradient}

Figure~\ref{fig:urban} presents mean merit points by our car-ownership-based urbanity classification. The gradient is pronounced and monotonic: Urban municipalities (below 400 cars/1,000) average 234 merit points, compared to 230 for Suburban (400--500), 221 for Semi-rural (500--600), and 219 for Rural municipalities (above 600).

\begin{figure}[H]
\centering
\includegraphics[width=0.85\textwidth]{figures/fig3_urbanity_gradient.pdf}
\caption{Educational Achievement by Municipality Urbanity}
\label{fig:urban}
\begin{minipage}{0.85\textwidth}
\small
\textit{Notes:} Bars show mean merit points for each urbanity category (2015 cross-section, N=290 municipalities). Urbanity classified by car ownership: Urban (below 400 cars/1,000), Suburban (400--500), Semi-rural (500--600), Rural (above 600). Error bars show 95\% confidence intervals. Numbers above bars indicate municipality count in each category.
\end{minipage}
\end{figure}

The 15-point gap between Urban and Rural municipalities represents 1.2 standard deviations of the municipal merit point distribution. In a system where merit points determine upper secondary school admission, this difference can translate into substantially different educational trajectories.

\subsection{Housing Tenure Patterns}

Figure~\ref{fig:housing} shows merit points by dominant housing tenure type. Cooperative-dominant municipalities---overwhelmingly located in the Stockholm metropolitan area---achieve the highest merit points (mean: 233.4). Rental-dominant municipalities average 223.5 points, while owner-dominant municipalities (primarily rural) average 221.2 points.

\begin{figure}[H]
\centering
\includegraphics[width=0.85\textwidth]{figures/fig2_housing_boxplot.pdf}
\caption{Educational Achievement by Dominant Housing Tenure Type}
\label{fig:housing}
\begin{minipage}{0.85\textwidth}
\small
\textit{Notes:} Boxplots show distribution of merit points by dominant housing tenure. Dominant type is the housing category with the largest share in each municipality. Points show individual municipalities with jitter for visibility.
\end{minipage}
\end{figure}

The relationship between housing tenure and achievement is complex. Cooperative housing is strongly concentrated in urban Stockholm, so the ``cooperative effect'' largely reflects the Stockholm urban advantage. Rental-dominant municipalities include both urban centers with substantial social housing and smaller towns with moderate rental shares.

\subsection{Correlation Structure}

Table~\ref{tab:corr} presents the correlation matrix for our key variables. Several patterns emerge:

\begin{table}[H]
\centering
\caption{Correlation Matrix: Key Variables}
\begin{threeparttable}
\begin{tabular}{lrrrrrr}
\toprule
& Merit & Cars & Rental & Owner & Coop & Teachers \\
\midrule
Merit Points & 1.00 & & & & & \\
Cars per 1,000 & $-$0.39 & 1.00 & & & & \\
Rental Housing (\%) & $-$0.09 & $-$0.33 & 1.00 & & & \\
Owner-Occupied (\%) & $-$0.24 & 0.70 & $-$0.68 & 1.00 & & \\
Cooperative (\%) & 0.40 & $-$0.69 & 0.13 & $-$0.82 & 1.00 & \\
Qualified Teachers (\%) & 0.01 & 0.20 & $-$0.13 & 0.25 & $-$0.23 & 1.00 \\
\bottomrule
\end{tabular}
\begin{tablenotes}[flushleft]
\small
\item Notes: Pearson correlation coefficients for 290 municipalities (2015). All housing and car variables measured in 2013.
\end{tablenotes}
\end{threeparttable}
\label{tab:corr}
\end{table}

Car ownership shows a strong negative correlation with merit points ($r = -0.39$) and cooperative housing ($r = -0.69$), and a strong positive correlation with owner-occupied housing ($r = 0.70$). This reflects the urban-rural divide in Swedish housing markets: urban areas have low car ownership, high cooperative housing, and low owner-occupied housing; rural areas show the opposite pattern.

The weak correlation between teacher qualifications and merit points ($r = 0.01$) is notable. While teacher quality likely matters for individual student outcomes, cross-municipal variation in qualification rates does not explain achievement differences once urbanity is accounted for. This may reflect compensating mechanisms (urban schools attract qualified teachers but face other challenges) or measurement limitations in our teacher quality proxy.

\subsection{Regression Results}

Table~\ref{tab:regression} presents our main regression results, progressively adding controls.

\begin{table}[H]
\centering
\caption{Regression Results: Merit Points on Car Ownership and Housing Tenure}
\begin{threeparttable}
\begin{tabular}{lccccc}
\toprule
& (1) & (2) & (3) & (4) & (5) \\
& Bivariate & +Housing & +Teachers & +County FE & Full \\
\midrule
Cars per 1,000 & $-$0.081*** & $-$0.096*** & $-$0.097*** & $-$0.079*** & $-$0.077*** \\
& (0.011) & (0.012) & (0.012) & (0.013) & (0.013) \\
\\
Rental Housing (\%) & & $-$0.349*** & $-$0.345*** & $-$0.246*** & $-$0.229*** \\
& & (0.088) & (0.089) & (0.084) & (0.083) \\
\\
Qualified Teachers (\%) & & & 0.059 & 0.134 & 0.228** \\
& & & (0.128) & (0.112) & (0.108) \\
\\
Constant & 266.0*** & 286.9*** & 281.8*** & 267.1*** & 251.4*** \\
& (5.9) & (7.8) & (13.4) & (13.1) & (14.2) \\
\\
County Fixed Effects & No & No & No & Yes & Yes \\
Year Fixed Effects & No & No & No & No & Yes \\
\\
Observations & 580 & 580 & 580 & 580 & 580 \\
$R^2$ & 0.153 & 0.196 & 0.197 & 0.292 & 0.312 \\
\bottomrule
\end{tabular}
\begin{tablenotes}[flushleft]
\small
\item Notes: Dependent variable is municipality average merit points (Grade 9, excluding recent immigrants). Standard errors clustered at municipality level in parentheses. * p$<$0.10, ** p$<$0.05, *** p$<$0.01. Car ownership and housing tenure measured in 2013; educational outcomes and teacher qualifications measured in 2015--2016.
\end{tablenotes}
\end{threeparttable}
\label{tab:regression}
\end{table}

Column (1) shows the bivariate relationship: each additional car per 1,000 inhabitants is associated with 0.081 lower merit points. This estimate is highly significant and explains 15.3\% of cross-municipal variation.

Adding rental housing share (Column 2) increases the car coefficient to $-0.096$, suggesting that conditional on housing composition, car ownership has an even stronger association with achievement. Rental housing share is significantly negatively associated with merit points: a 10 percentage point increase in rental share corresponds to 3.5 fewer merit points.

Teacher qualifications (Column 3) add minimal explanatory power. The coefficient on teacher qualifications is positive but statistically insignificant, and $R^2$ barely increases from 0.196 to 0.197.

County fixed effects (Column 4) reduce the car ownership coefficient to $-0.079$ and substantially increase $R^2$ to 0.292. This indicates that much of the car-merit relationship operates between counties, though significant within-county variation remains.

The full model (Column 5) including year fixed effects yields our preferred estimate: each additional 100 cars per 1,000 inhabitants is associated with 7.7 fewer merit points. This represents approximately 0.61 standard deviations of the merit point distribution. The rental housing coefficient remains significant at $-0.23$, and teacher qualifications become significant at the 5\% level with a coefficient of 0.23.

To put these magnitudes in perspective, consider a hypothetical shift from a typical rural municipality (600 cars per 1,000) to a typical urban municipality (400 cars per 1,000). Using the full model coefficient of $-0.077$, the 200-car difference implies $200 \times 0.077 = 15.4$ additional merit points---enough to substantially affect secondary school admission prospects. Similarly, a 10 percentage point reduction in rental housing share (from 35\% to 25\%) would be associated with 2.3 additional merit points, holding other factors constant.

The $R^2$ values indicate that our models explain substantial but not complete variation in municipal achievement. The full model explains 31.2\% of variation, leaving considerable unexplained heterogeneity. This unexplained variation likely reflects unmeasured factors including parental education levels, local economic conditions, school-specific quality variation, and historical institutional factors that vary across municipalities.

The increase in $R^2$ from adding county fixed effects (from 0.196 to 0.292) suggests that approximately 10 percentage points of explained variation operates between counties rather than within them. County-level factors---regional economies, historical development patterns, coordination of services---appear to matter beyond what our municipality-level variables capture. This motivates examining county-level patterns directly.

\subsection{Interpreting the Teacher Qualification Result}

The weak relationship between teacher qualification rates and merit points deserves further discussion. Teacher quality is widely believed to be the most important school-level factor in educational outcomes \citep{hanushek2003does}, yet our cross-municipal analysis shows minimal correlation.

Several factors may explain this pattern. First, our qualification measure captures formal credentials (behöriga lärare) rather than effective teaching quality. Teachers with credentials may not be better teachers; alternatively, uncredentialed teachers may compensate through other qualities. Second, teacher sorting across municipalities may generate offsetting effects: urban municipalities may attract more qualified teachers but face larger class sizes, more challenging student populations, or other resource constraints that offset the teacher quality advantage. Third, within-municipality variation in teacher quality may dwarf cross-municipality variation, making municipality-level averages poor predictors of student outcomes.

Fourth, and perhaps most importantly, teacher qualifications may be endogenous to municipal characteristics. Municipalities with higher-achieving students (due to family backgrounds and resources) may be more attractive to qualified teachers, creating a spurious positive correlation that offsets any negative relationship arising from urban-rural teacher allocation patterns. The near-zero correlation we observe may reflect the net of these opposing forces.

\subsection{County-Level Analysis}

Figure~\ref{fig:county} presents county-level averages, revealing substantial geographic variation in educational achievement.

\begin{figure}[H]
\centering
\includegraphics[width=0.95\textwidth]{figures/fig4_county_comparison.pdf}
\caption{Educational Achievement Across Swedish Counties}
\label{fig:county}
\begin{minipage}{0.95\textwidth}
\small
\textit{Notes:} Each point represents a county (län). Point size indicates number of municipalities; color indicates average car ownership (darker = fewer cars = more urban). Dashed line shows national mean (222.9 points).
\end{minipage}
\end{figure}

Stockholm County leads with average merit points of 238.4, 15.5 points above the national mean. Stockholm's 26 municipalities have the lowest average car ownership (414 cars/1,000) and highest cooperative housing prevalence in Sweden. At the other end, Örebro County averages 212.9 merit points, 10 points below the national mean, with car ownership of 543 cars/1,000.

The county ranking generally follows urbanity patterns, with exceptions. Gotland, a large island municipality, achieves 229.5 merit points despite high car ownership (583/1,000), likely reflecting its unique institutional context and small size. Norrbotten in northern Sweden achieves 227.0 points despite very high car ownership (578/1,000), possibly due to concentrated urban populations in Luleå and Kiruna.

\subsection{Rental Housing and Achievement}

Figure~\ref{fig:rental} examines the relationship between rental housing share and merit points, with car ownership indicated by color.

\begin{figure}[H]
\centering
\includegraphics[width=0.9\textwidth]{figures/fig6_rental_merit_scatter.pdf}
\caption{Rental Housing and Educational Achievement}
\label{fig:rental}
\begin{minipage}{0.9\textwidth}
\small
\textit{Notes:} Each point represents one municipality. Rental housing share measured as percentage of dwelling units (2013). Color indicates car ownership (darker = fewer cars = more urban). Fitted line shows OLS regression with 95\% CI.
\end{minipage}
\end{figure}

The relationship between rental housing and achievement is weakly negative in the raw data ($r = -0.09$), but this masks substantial heterogeneity. Urban municipalities (darker colors) span the full range of rental shares but achieve consistently higher merit points. The negative rental coefficient in our regressions reflects that, conditional on urbanity (proxied by car ownership), higher rental shares are associated with lower achievement. This likely reflects the concentration of lower-income households in rental housing within municipalities.


\section{Discussion}

\subsection{Interpretation of Results}

Our findings document a robust urban-rural gradient in Swedish educational achievement. Municipalities with lower car ownership---corresponding to greater urbanity, better public transportation, and higher population density---achieve substantially higher Grade 9 merit points. This 15-point gap between the most urban and most rural municipalities represents meaningful differences in students' educational trajectories and post-secondary opportunities.

Several mechanisms could explain this pattern, operating through both school-level and family-level channels. We discuss each in turn.

\textbf{School Choice and Competition.} Urban areas offer greater school density, enabling families to select schools that match their children's needs and aptitudes. Sweden's school choice reforms in the 1990s created quasi-markets where funding follows students, generating competitive pressure on schools to attract families. This competitive mechanism operates more powerfully in urban areas with multiple school options than in rural areas where a single municipal school may serve the entire community. The availability of independent schools (friskolor) is also much greater in urban areas, further expanding choice options. Research has documented both positive and negative effects of Swedish school choice, including potential cream-skimming by independent schools and increased segregation \citep{bolin2023school}, but the net effect on average achievement in choice-rich environments remains debated.

\textbf{Labor Market Sorting.} Urban labor markets concentrate high-skill, high-wage employment in sectors such as technology, finance, professional services, and higher education. These labor markets attract and retain educated workers who tend to invest heavily in their children's education. The concentration of human capital in urban areas creates positive externalities: children grow up surrounded by educated adults, role models for professional careers, and networks that facilitate educational and career advancement. Rural labor markets, by contrast, often depend on extraction industries, agriculture, or manufacturing that require different skill profiles and may not generate the same educational aspirations or investments.

\textbf{Cultural and Educational Infrastructure.} Museums, libraries, theaters, research universities, and other cultural institutions cluster in urban centers. These institutions provide enrichment opportunities---exhibitions, lectures, programs, research experiences---that complement formal schooling. Urban students can access these resources independently via public transportation, while rural students depend on parental initiative and transportation capacity. The density of cultural infrastructure may also affect community norms about education and intellectual achievement.

\textbf{Peer Effects and Social Networks.} Urban schools may have higher-achieving peer groups on average due to the socioeconomic sorting described above. Peer effects in education are well-documented: students learn from each other, and classroom composition affects both individual achievement and teacher effectiveness. Beyond the classroom, urban students have access to larger social networks that may include academically successful peers, mentors, and connections to educational and career opportunities. The sparse populations of rural areas limit these network effects.

\textbf{Parental Time and Investment.} Transportation mode affects parental capacity for educational investment. Car-dependent rural families spend substantial time driving---to work, to shops, to schools, to activities. This time could otherwise be allocated to homework help, reading with children, or attending school events. Urban families with public transit access may have more discretionary time for educational investment. The time costs of transportation may be particularly binding for dual-earner families, who must coordinate work schedules with school drop-offs, pick-ups, and activities.

The role of cooperative housing deserves attention. The strong positive correlation between cooperative housing share and achievement ($r = 0.40$) reflects both the urban concentration of this tenure form and socioeconomic selection. Cooperative housing requires down payments and mortgage capacity, selecting higher-income households. In urban areas with substantial queues for rental housing, cooperative housing represents the primary avenue for middle-class families seeking to establish residence in good school catchment areas \citep{holmqvist2015segregation}. The association between cooperative housing and achievement thus likely reflects socioeconomic selection rather than any direct effect of housing tenure on educational outcomes.

The weak relationship between teacher qualifications and achievement is puzzling but consistent with prior research suggesting that measurable teacher characteristics explain little of the variation in educational outcomes \citep{hanushek2003does}. Teacher quality may matter, but our qualification measure may not capture the relevant dimensions---enthusiasm, pedagogical skill, subject matter expertise beyond formal credentials. Alternatively, teacher sorting may partly offset qualification effects: qualified teachers may prefer urban schools with better working conditions and career prospects, but urban schools may face other challenges (larger class sizes, more diverse student populations) that offset this advantage.

\subsection{Comparison with International Patterns}

Our findings on Swedish urban-rural educational disparities resonate with research in other countries, though the magnitudes and mechanisms may differ. The influential work of \citet{chetty2014united, chetty2018opportunity} documents substantial geographic variation in intergenerational mobility across U.S. commuting zones, with dense metropolitan areas often (but not always) offering better mobility prospects. Their findings suggest that place matters for long-term outcomes, consistent with our documentation of place-based educational achievement differences.

However, Sweden's institutional context differs from the U.S. in important ways that may attenuate or amplify geographic effects. Sweden's more compressed income distribution and generous welfare state may reduce socioeconomic gaps that drive educational inequality. Universal access to healthcare and childcare removes some barriers that affect U.S. children's educational readiness. Yet Sweden's school choice system may generate sorting effects that partially offset these equalizing features.

Comparisons with other Nordic countries would be informative. Denmark, Norway, and Finland share many institutional features with Sweden---comprehensive welfare states, universal education, similar demographic structures---but differ in details of housing markets, school governance, and regional development policies. Understanding whether the Swedish urban-rural gradient is replicated, attenuated, or amplified in these similar contexts would shed light on which institutional features drive geographic educational inequality.

\subsection{Implications for Policy}

Our descriptive findings have several policy implications, though causal claims require additional research with stronger identification strategies.

\textbf{Place-Based Educational Investment.} Sweden's substantial urban-rural educational gradient suggests that place-based educational policies may be warranted. Rural municipalities face structural disadvantages in school choice availability, population density, resource concentration, and access to enrichment opportunities. Targeted interventions might include enhanced funding for rural schools, incentive programs to attract qualified teachers to rural postings, investments in digital infrastructure to expand educational access, or regional coordination to increase school choice options across municipal boundaries.

\textbf{Transportation Policy.} The skolskjuts (school transportation) system may play a compensating role in rural areas by enabling access to distant schools. Our results do not directly measure skolskjuts effects, but they suggest that transportation access is linked to educational opportunity. Policies that expand transportation subsidies, reduce distance thresholds, or extend eligibility to non-assigned schools might increase effective school choice in rural areas. Conversely, public transportation investments in rural areas---though costly and potentially infeasible---could reduce car dependency and its associated time and financial burdens on families.

\textbf{Housing Policy.} Housing policy interacts with educational outcomes through residential sorting. Policies affecting cooperative housing construction, rental housing allocation, or owner-occupied housing development will influence the socioeconomic composition of municipalities and thereby school composition. Sweden's rent regulation system, by creating housing queues, may limit residential mobility in ways that constrain school choice. Reforms to increase housing supply in high-demand areas could improve access to high-performing school districts, though such reforms face substantial political resistance.

\textbf{Regional Development.} Broader regional development policies---investments in rural infrastructure, incentives for employers to locate outside metropolitan areas, support for local cultural institutions---could address root causes of urban-rural disparities. However, such policies involve trade-offs between efficiency (concentrating resources where returns may be highest) and equity (spreading resources to disadvantaged areas). The optimal balance depends on value judgments about geographic equity that our descriptive analysis cannot resolve.

\subsection{Limitations}

Our analysis faces several important limitations that should temper causal interpretation and policy prescription.

\textbf{Causal Identification.} We cannot identify causal effects. The correlations we document could reflect selection, sorting, omitted variables, or reverse causality. Families with strong preferences for education may simultaneously choose low-car-ownership municipalities and invest heavily in their children's schooling. Our regression coefficients should be interpreted as conditional associations, not treatment effects.

\textbf{Municipal-Level Aggregation.} Our municipal-level analysis cannot address within-municipality variation. Large municipalities like Stockholm contain substantial internal heterogeneity in car ownership, housing tenure, and educational achievement that our data cannot capture. Stockholm encompasses wealthy inner-city neighborhoods with very low car ownership alongside suburban areas with higher ownership and potentially different educational outcomes. Our analysis treats Stockholm as a single observation, obscuring this internal variation.

\textbf{Compositional Effects.} Our outcome measure (municipality average merit points) is affected by compositional changes. If higher-achieving students systematically move to certain municipalities---through parental job changes, school choice across municipal boundaries, or family formation decisions---our results would partly reflect this mobility rather than the effects of municipal characteristics on given students. We cannot distinguish ``treatment'' of students by their municipality from ``selection'' of students into municipalities.

\textbf{Population Coverage.} The merit point measure excludes recent immigrants (students in Sweden for fewer than four years), which is appropriate for cross-municipal comparability but means our results may not generalize to this growing population. Recent immigrants face distinctive educational challenges---language acquisition, educational gaps, adjustment stress---that municipal-level factors may affect differently than they affect native-born students. Our findings thus apply primarily to established residents.

\textbf{Unobserved Mechanisms.} We lack data on important potential mechanisms: school quality beyond teacher qualifications, extracurricular investments, private tutoring, parental education levels, local labor market conditions, and individual-level transportation mode. Future research with richer data could decompose the car ownership association into its constituent mechanisms.

\textbf{Temporal Scope.} Our data cover only 2015--2016 educational outcomes, with 2013 car ownership and housing variables. We cannot assess whether the patterns we document are stable over time or whether they reflect period-specific conditions. Sweden experienced substantial immigration in 2015--2016, economic recovery from the 2008 financial crisis, and ongoing debates about school quality. Our snapshot may not generalize to other periods.

\subsection{Future Research}

Several extensions would strengthen our understanding of geographic educational inequality in Sweden.

\textbf{Individual-Level Analysis.} Data linking students to neighborhood characteristics---perhaps through register linkages between educational records and residential addresses---would enable within-municipality analysis. Such data could distinguish school effects from neighborhood effects and identify which specific local characteristics predict individual outcomes.

\textbf{Longitudinal Analysis.} Longitudinal data following students across residential moves could address selection concerns. If students who move from high-car to low-car municipalities show improved outcomes relative to stayers, this would suggest causal effects of municipal characteristics beyond selection. The Swedish register system potentially enables such analysis.

\textbf{Natural Experiments.} Natural experiments in transportation infrastructure or housing policy could provide causal identification. New transit lines, school closures forcing redistricting, housing development projects, or policy reforms affecting transportation subsidies could generate quasi-experimental variation. Exploiting such variation requires finding specific events with sufficient scale and sharp implementation.

\textbf{Comparative Analysis.} Cross-country comparisons with Denmark, Norway, and Finland---countries sharing Sweden's Nordic institutional context---would assess generalizability. Differences in housing markets, school governance, or regional development policies across these similar countries could identify institutional features that moderate or amplify geographic educational inequality.


\section{Conclusion}

This paper documents a substantial urban-rural gradient in educational achievement across Swedish municipalities. Using comprehensive administrative data from the Kolada database, we show that municipalities with lower car ownership---a proxy for urbanity and public transportation availability---achieve significantly higher Grade 9 merit points. A 100-car reduction per 1,000 inhabitants is associated with 7.7 higher merit points, roughly 0.61 standard deviations of the municipal distribution. This relationship is robust to controlling for housing tenure composition, teacher qualification rates, and county fixed effects.

Housing tenure composition provides additional explanatory power. Cooperative housing dominance, characteristic of urban Sweden, correlates positively with achievement, while rental housing share shows a negative conditional association. The three housing tenure types---rental, cooperative, and owner-occupied---exhibit strong geographic clustering that mirrors the urban-rural divide. These patterns likely reflect residential sorting rather than direct effects of housing type on education: families select into housing and municipalities based on preferences, constraints, and resources that also affect educational investments and outcomes.

Our findings challenge the perception of Sweden as an educationally egalitarian society. Despite universal education, free school transportation, and substantial public investment, where children grow up substantially shapes their educational outcomes. The 25-point gap between Stockholm County and the lowest-performing counties represents meaningful differences in life trajectories. In a system where merit points determine upper secondary school admissions, these geographic disparities translate into differential access to academic programs, vocational training, and ultimately career opportunities.

Several mechanisms may underlie the urban-rural gradient: school choice and competition, labor market sorting of educated families, access to cultural and educational infrastructure, peer effects in schools and communities, and parental time constraints associated with car-dependent lifestyles. Our cross-sectional data cannot disentangle these mechanisms, but their combined effect produces substantial and robust geographic stratification.

Understanding the mechanisms underlying this geographic inequality---and designing policies to address it---represents an important challenge for Swedish education policy. Place-based interventions targeting rural schools, transportation policy reforms expanding access to school choice, housing policies affecting residential sorting, and regional development investments could all potentially address the disparities we document. However, the appropriate policy response depends on which mechanisms dominate---a question requiring additional research with causal identification strategies.

Our analysis has important limitations. We cannot establish causality, we observe only municipal averages rather than individual outcomes, we lack data on many potential mechanisms, and our temporal scope is narrow. Future research using individual-level register data, exploiting natural experiments in housing or transportation policy, and conducting comparative analysis across Nordic countries could address these limitations and move toward actionable policy guidance.

Despite these limitations, our descriptive analysis serves an important purpose: documenting the magnitude, robustness, and patterns of geographic educational inequality in Sweden. Before designing interventions, policymakers need to understand what variation exists and which factors correlate with outcomes. This paper provides that foundation. The substantial urban-rural gradient we document---unexpected in a country renowned for egalitarianism---demands attention from researchers and policymakers alike. Whether Sweden's educational system can deliver on its promise of equal opportunity regardless of birthplace remains an open question, but our findings suggest considerable distance yet to travel.


\section*{Acknowledgements}

This paper was autonomously generated using Claude Code as part of the Autonomous Policy Evaluation Project (APEP).

\noindent\textbf{Project Repository:} \url{https://github.com/SocialCatalystLab/auto-policy-evals}

\noindent\textbf{Contributors:} @anonymous

\noindent\textbf{First Contributor:} \url{https://github.com/anonymous}

\label{apep_main_text_end}
\newpage

\begin{thebibliography}{99}

\bibitem[Andersson(2007)]{andersson2007housing}
Andersson, R. (2007).
\newblock Segregation dynamics and urban policy in Sweden.
\newblock \textit{Housing, Theory and Society}, 24(1), 1--19.

\bibitem[Blumenberg and Manville(2004)]{blumenberg2004engendering}
Blumenberg, E., \& Manville, M. (2004).
\newblock Beyond the spatial mismatch: Welfare recipients and transportation policy.
\newblock \textit{Journal of Planning Literature}, 19(2), 182--205.

\bibitem[Blumenberg(2008)]{blumenberg2008planning}
Blumenberg, E. (2008).
\newblock Planning for the transportation needs of welfare participants: Institutional challenges to collaborative planning.
\newblock \textit{Journal of Planning Education and Research}, 27(4), 391--407.

\bibitem[Bolin and Holmlund(2023)]{bolin2023school}
Bolin, K., \& Holmlund, H. (2023).
\newblock School choice and educational outcomes in Sweden.
\newblock \textit{Journal of Human Resources}, 58(3), 812--844.

\bibitem[Chetty et al.(2014)]{chetty2014united}
Chetty, R., Hendren, N., Kline, P., \& Saez, E. (2014).
\newblock Where is the land of opportunity? The geography of intergenerational mobility in the United States.
\newblock \textit{Quarterly Journal of Economics}, 129(4), 1553--1623.

\bibitem[Chetty and Hendren(2018)]{chetty2018opportunity}
Chetty, R., \& Hendren, N. (2018).
\newblock The impacts of neighborhoods on intergenerational mobility I: Childhood exposure effects.
\newblock \textit{Quarterly Journal of Economics}, 133(3), 1107--1162.

\bibitem[Grönqvist et al.(2017)]{groenqvist2017family}
Grönqvist, E., Öckert, B., \& Vlachos, J. (2017).
\newblock The intergenerational transmission of cognitive and noncognitive abilities.
\newblock \textit{Journal of Human Resources}, 52(4), 887--918.

\bibitem[Hanushek and Rivkin(2003)]{hanushek2003does}
Hanushek, E. A., \& Rivkin, S. G. (2003).
\newblock Does public school competition affect teacher quality?
\newblock In C. Hoxby (Ed.), \textit{The Economics of School Choice} (pp. 23--47). University of Chicago Press.

\bibitem[Holmlund et al.(2014)]{holmlund2014swedish}
Holmlund, H., Häggblom, J., Lindahl, E., Martinson, S., Sjögren, A., Vikman, U., \& Öckert, B. (2014).
\newblock Decentralisering, skolval och fristående skolor: Resultat och likvärdighet i svensk skola.
\newblock IFAU Report 2014:25.

\bibitem[Holmqvist(2015)]{holmqvist2015segregation}
Holmqvist, E. (2015).
\newblock Tenure mix as a way to counteract segregation? The case of Sweden.
\newblock \textit{European Journal of Housing Policy}, 15(3), 344--364.

\bibitem[Magnusson Turner and Andersson(2022)]{magnusson2022housing}
Magnusson Turner, L., \& Andersson, R. (2022).
\newblock Swedish housing policy: Past, present, and future.
\newblock In M. Stephens \& M. Elsinga (Eds.), \textit{Routledge Handbook of Housing Policy and Planning}. Routledge.

\bibitem[Schwartz(2014)]{schwartz2014housing}
Schwartz, A. E. (2014).
\newblock Housing policy is school policy.
\newblock \textit{Cityscape}, 16(1), 53--56.

\bibitem[Skolverket(2023)]{skolverket2023skolskjuts}
Skolverket. (2023).
\newblock Skolskjuts i grundskola och anpassad grundskola.
\newblock Swedish National Agency for Education.

\bibitem[Vlachos(2018)]{vlachos2018grading}
Vlachos, J. (2018).
\newblock Trust-based evaluation in a market-oriented school system.
\newblock IFN Working Paper No. 1217.

\bibitem[Black and Machin(2011)]{black2011housing}
Black, S. E., \& Machin, S. (2011).
\newblock Housing valuations of school performance.
\newblock In E. A. Hanushek, S. Machin, \& L. Woessmann (Eds.), \textit{Handbook of the Economics of Education} (Vol. 3, pp. 485--519). Elsevier.

\bibitem[Card and Rothstein(2007)]{card2007racial}
Card, D., \& Rothstein, J. (2007).
\newblock Racial segregation and the black-white test score gap.
\newblock \textit{Journal of Public Economics}, 91(11--12), 2158--2184.

\bibitem[Hoxby(2000)]{hoxby2000peer}
Hoxby, C. (2000).
\newblock Peer effects in the classroom: Learning from gender and race variation.
\newblock NBER Working Paper No. 7867.

\bibitem[Kain(1968)]{kain1968housing}
Kain, J. F. (1968).
\newblock Housing segregation, negro employment, and metropolitan decentralization.
\newblock \textit{Quarterly Journal of Economics}, 82(2), 175--197.

\bibitem[Sacerdote(2011)]{sacerdote2011peer}
Sacerdote, B. (2011).
\newblock Peer effects in education: How might they work, how big are they and how much do we know thus far?
\newblock In E. A. Hanushek, S. Machin, \& L. Woessmann (Eds.), \textit{Handbook of the Economics of Education} (Vol. 3, pp. 249--277). Elsevier.

\bibitem[Reardon(2016)]{reardon2016patterns}
Reardon, S. F. (2016).
\newblock School segregation and racial academic achievement gaps.
\newblock \textit{RSF: The Russell Sage Foundation Journal of the Social Sciences}, 2(5), 34--57.

\bibitem[Ladd and Fiske(2001)]{ladd2001schools}
Ladd, H. F., \& Fiske, E. B. (2001).
\newblock The uneven playing field of school choice: Evidence from New Zealand.
\newblock \textit{Journal of Policy Analysis and Management}, 20(1), 43--64.

\bibitem[Böhlmark and Lindahl(2015)]{bohlmark2015independent}
Böhlmark, A., \& Lindahl, M. (2015).
\newblock Independent schools and long-run educational outcomes: Evidence from Sweden's large-scale voucher reform.
\newblock \textit{Economica}, 82(327), 508--551.

\bibitem[Gibbons and Machin(2003)]{gibbons2003valuing}
Gibbons, S., \& Machin, S. (2003).
\newblock Valuing English primary schools.
\newblock \textit{Journal of Urban Economics}, 53(2), 197--219.

\bibitem[Rothstein(2006)]{rothstein2006good}
Rothstein, J. M. (2006).
\newblock Good principals or good peers? Parental valuation of school characteristics, Tiebout equilibrium, and the incentive effects of competition among jurisdictions.
\newblock \textit{American Economic Review}, 96(4), 1333--1350.

\bibitem[Epple and Romano(2011)]{epple2011peer}
Epple, D., \& Romano, R. E. (2011).
\newblock Peer effects in education: A survey of the theory and evidence.
\newblock In J. Benhabib, A. Bisin, \& M. O. Jackson (Eds.), \textit{Handbook of Social Economics} (Vol. 1B, pp. 1053--1163). North-Holland.

\bibitem[Downes and Greenstein(1996)]{downes1996understanding}
Downes, T. A., \& Greenstein, S. M. (1996).
\newblock Understanding the supply decisions of nonprofits: Modelling the location of private schools.
\newblock \textit{RAND Journal of Economics}, 27(2), 365--390.

\end{thebibliography}

\newpage
\appendix

\section{Data Appendix}

\subsection{Data Sources}

All data were obtained from the Kolada database (\url{https://www.kolada.se}), Sweden's official municipal statistics platform. The following Key Performance Indicators (KPIs) were used:

\begin{itemize}
\item \textbf{N15566}: Average merit points, Grade 9, excluding recent immigrants
\item \textbf{N15030}: Share of teachers with formal qualifications (\%)
\item \textbf{N07935}: Registered cars per 1,000 inhabitants
\item \textbf{N07936}: Gasoline cars per 1,000 inhabitants
\item \textbf{N07937}: Diesel cars per 1,000 inhabitants
\item \textbf{N07938}: Electric cars per 1,000 inhabitants
\item \textbf{N07956}: Rental housing share (\%)
\item \textbf{N07957}: Cooperative housing share (\%)
\item \textbf{N07958}: Owner-occupied housing share (\%)
\end{itemize}

Data were accessed via the Kolada REST API in February 2026.

\subsection{Sample Construction}

Starting sample: All municipality-year observations in Kolada for 2015--2016.

\begin{enumerate}
\item Restricted to 290 standard Swedish municipalities (excluding regional aggregates and special administrative codes): N = 580 municipality-years.
\item Required non-missing merit points: N = 580 (no observations dropped).
\item Required non-missing car ownership (2013): N = 580.
\item Required non-missing housing tenure (2013): N = 580.
\item Final sample: 580 municipality-year observations (290 municipalities $\times$ 2 years).
\end{enumerate}

\subsection{Variable Construction}

\textbf{Urban Proxy Classification:}
Municipalities were classified into four urbanity categories based on car ownership per 1,000 inhabitants (2013 values):
\begin{itemize}
\item Urban: Below 400 cars/1,000
\item Suburban: 400--500 cars/1,000
\item Semi-rural: 500--600 cars/1,000
\item Rural: Above 600 cars/1,000
\end{itemize}

\textbf{Dominant Housing Type:}
Municipalities were classified by their dominant housing tenure based on whichever category had the largest share:
\begin{itemize}
\item Rental dominant: Rental \% $>$ max(Cooperative \%, Owner \%)
\item Cooperative dominant: Cooperative \% $>$ max(Rental \%, Owner \%)
\item Owner dominant: Owner \% $>$ max(Rental \%, Cooperative \%)
\end{itemize}


\section{Robustness Appendix}

\subsection{Alternative Specifications}

Table~\ref{tab:robust} presents alternative specifications to assess robustness.

\begin{table}[H]
\centering
\caption{Robustness Checks}
\begin{threeparttable}
\begin{tabular}{lcccc}
\toprule
& (1) & (2) & (3) & (4) \\
& Baseline & No 2016 & Add Coop & Weighted \\
\midrule
Cars per 1,000 & $-$0.077*** & $-$0.078*** & $-$0.069*** & $-$0.082*** \\
& (0.013) & (0.011) & (0.015) & (0.014) \\
\\
Rental Housing (\%) & $-$0.229*** & $-$0.231** & $-$0.142 & $-$0.198** \\
& (0.083) & (0.089) & (0.102) & (0.091) \\
\\
Cooperative Housing (\%) & & & 0.112 & \\
& & & (0.098) & \\
\\
Qualified Teachers (\%) & 0.228** & 0.241** & 0.225** & 0.219** \\
& (0.108) & (0.115) & (0.109) & (0.112) \\
\\
Observations & 580 & 290 & 580 & 580 \\
$R^2$ & 0.312 & 0.318 & 0.316 & 0.334 \\
\bottomrule
\end{tabular}
\begin{tablenotes}[flushleft]
\small
\item Notes: All specifications include county and year fixed effects (except Column 2, which uses 2015 only). Column 3 adds cooperative housing share. Column 4 weights municipalities by student population. Standard errors clustered at municipality level. * p$<$0.10, ** p$<$0.05, *** p$<$0.01.
\end{tablenotes}
\end{threeparttable}
\label{tab:robust}
\end{table}

The car ownership coefficient is stable across specifications, ranging from $-0.069$ to $-0.082$. Using only 2015 data (Column 2) produces nearly identical results to the pooled sample. Adding cooperative housing share (Column 3) reduces the car coefficient slightly and renders rental housing insignificant, consistent with multicollinearity between housing types. Population weighting (Column 4) slightly increases the magnitude of the car coefficient.


\section{Heterogeneity Appendix}

\subsection{Results by Car Ownership Tercile}

Table~\ref{tab:tercile} presents summary statistics by car ownership tercile.

\begin{table}[H]
\centering
\caption{Summary Statistics by Car Ownership Tercile (2015)}
\begin{threeparttable}
\begin{tabular}{lccc}
\toprule
& Low Car & Medium Car & High Car \\
& (T1) & (T2) & (T3) \\
\midrule
N Municipalities & 97 & 97 & 96 \\
Mean Cars/1,000 & 385.2 & 530.5 & 596.8 \\
Mean Merit Points & 228.2 & 222.3 & 218.2 \\
SD Merit Points & 13.8 & 11.0 & 10.6 \\
Mean Rental \% & 33.5 & 29.8 & 24.8 \\
Mean Coop \% & 19.7 & 11.1 & 7.9 \\
\bottomrule
\end{tabular}
\begin{tablenotes}[flushleft]
\small
\item Notes: Terciles defined by car ownership per 1,000 inhabitants. Low = bottom tercile, High = top tercile.
\end{tablenotes}
\end{threeparttable}
\label{tab:tercile}
\end{table}

The gradient is monotonic: low car ownership municipalities average 10 more merit points than high car ownership municipalities (228.2 vs. 218.2). This 10-point gap is roughly 0.8 standard deviations of the merit point distribution within the high-car tercile.


\section{Additional Figures}

\begin{figure}[H]
\centering
\includegraphics[width=0.85\textwidth]{figures/fig5_correlation_heatmap.pdf}
\caption{Correlation Heatmap: Key Variables}
\label{fig:heatmap}
\begin{minipage}{0.85\textwidth}
\small
\textit{Notes:} Pearson correlations for 290 municipalities (2015). All variables measured at municipality level. Blue indicates negative correlation; yellow indicates positive correlation.
\end{minipage}
\end{figure}


\end{document}
