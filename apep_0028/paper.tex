\documentclass[12pt]{article}
\usepackage[utf8]{inputenc}
\usepackage[margin=1in]{geometry}
\usepackage{amsmath,amssymb}
\usepackage{graphicx}
\usepackage{booktabs}
\usepackage[numbers]{natbib}
\usepackage{hyperref}
\usepackage{setspace}
\usepackage{float}
\usepackage{caption}
\usepackage{subcaption}

\doublespacing

\title{The Montana Miracle Revisited: \\ Early Evidence on the Effects of Statewide Zoning Reform on Residential Construction}

\author{Autonomous Policy Evaluation Project (APEP) nd @dakoyana}

\date{January 2026}

\begin{document}

\maketitle

\begin{abstract}
\noindent In 2023, Montana enacted sweeping zoning reforms---dubbed the ``Montana Miracle''---that legalized accessory dwelling units (ADUs) and duplexes statewide. This paper provides the first empirical evaluation of these reforms using both difference-in-differences and synthetic control methods, comparing Montana to control states using monthly building permit data from the Census Bureau's Building Permits Survey (November 2019 through October 2025). The standard DiD estimate suggests a positive but statistically insignificant effect of approximately 3.2 additional building permits per 100,000 population per month. However, synthetic control analysis using 47 donor states finds essentially no effect ($-0.7$ permits/100K), with Montana ranking 20th of 48 states in the placebo distribution (p = 0.958). Event study analysis reveals problematic pre-trends, and the conflicting estimates suggest that credible identification is not achieved with available data. The analysis highlights methodological challenges in evaluating single-treated-unit policy reforms and illustrates why state-level aggregation may be inappropriate when policies vary at the municipal level. Future research should exploit within-state variation using place-level data.

\medskip
\noindent \textbf{Keywords:} zoning reform, housing supply, accessory dwelling units, synthetic control, difference-in-differences, Montana

\medskip
\noindent \textbf{JEL Codes:} R14, R31, R38, R52, H73
\end{abstract}

\newpage
\tableofcontents
\newpage

%==============================================================================
\section{Introduction}
%==============================================================================

Housing affordability has emerged as one of the most pressing domestic policy challenges in the United States. Between 2010 and 2024, median home prices increased by over 80\% nationally, far outpacing wage growth. In response, a growing number of states and localities have pursued zoning reform as a mechanism to increase housing supply. The theoretical case for such reforms is straightforward: by reducing regulatory barriers to construction, particularly for multi-family and accessory dwelling units, jurisdictions can increase the supply of housing and thereby moderate price growth.

Montana provides a particularly instructive case study of comprehensive zoning reform. In 2023, the Montana Legislature passed a suite of bills collectively known as the ``Montana Miracle'' that represented one of the most aggressive state-level interventions into local land use regulation in U.S. history. Senate Bill 528 required all Montana municipalities to allow at least one accessory dwelling unit (ADU) on any lot containing a single-family home, while Senate Bill 323 legalized duplexes in all residential zones in cities with populations exceeding 5,000. These reforms went further than similar efforts in other states by preempting local regulations that had traditionally restricted such development.

This paper provides the first empirical evaluation of the Montana zoning reforms' effects on residential construction. Using monthly building permit data from the Census Bureau's Building Permits Survey and a difference-in-differences research design, the analysis compares building permit issuance in Montana to a set of similar Mountain West control states before and after the reforms took effect in January 2024.

The main finding is a positive but statistically insignificant treatment effect of approximately 3.2 additional building permits per 100,000 population per month. This represents an 8.2\% increase relative to Montana's pre-treatment average of 39.5 permits per 100,000 population. The imprecision of this estimate reflects several factors: the limited post-treatment period available for analysis (22 months), substantial legal uncertainty during the first 15 months following a preliminary injunction against the reforms, and the inherent time lags between policy changes and construction activity.

These findings contribute to the literature on housing supply elasticities and land use deregulation. While the null result prevents strong conclusions about the effectiveness of Montana's reforms, the analysis provides a methodological template for future research and establishes a baseline against which longer-run effects can be measured.

The remainder of this paper proceeds as follows. Section 2 provides background on Montana's housing market and the specific reforms enacted in 2023. Section 3 reviews the relevant literature on zoning and housing supply. Section 4 describes the data and empirical methodology. Section 5 presents the main results and robustness checks. Section 6 discusses the findings and their limitations. Section 7 concludes.

%==============================================================================
\section{Policy Background}
%==============================================================================

\subsection{Montana's Housing Market Context}

Montana experienced one of the nation's most dramatic housing market transformations in the years preceding the 2023 reforms. Between 2010 and 2022, Montana's population grew by 12.4\%, driven largely by migration from higher-cost states such as California, Washington, and Oregon. This influx accelerated during the COVID-19 pandemic as remote work enabled relocation to lower-density areas with outdoor amenities.

The price consequences were substantial. The median home price in Montana exceeded \$300,000 in November 2020 and reached \$400,000 just 14 months later in January 2022---a pace of appreciation that far exceeded national trends. By 2022, approximately one in four Montana households was housing cost-burdened, defined as paying more than 30\% of household income on housing costs.

These pressures were particularly acute in resort communities such as Bozeman, Whitefish, and Big Sky, but extended throughout the state's metropolitan areas. Rental vacancy rates fell to historic lows, and homelessness increased in several Montana cities. Local governments struggled to provide workforce housing for teachers, first responders, and service workers.

\subsection{The Legislative Response}

Governor Greg Gianforte made housing affordability a centerpiece of his policy agenda upon taking office in 2021. Working with a Republican-controlled legislature, his administration developed a comprehensive package of zoning reforms that drew on policy innovations from states such as Oregon, California, and Minnesota.

\subsubsection{Senate Bill 528: ADU Legalization}

Senate Bill 528, signed into law in May 2023 with an effective date of January 1, 2024, required all Montana municipalities to allow at least one accessory dwelling unit on any lot containing a single-family dwelling. The bill was notable for several distinctive features that set it apart from similar reforms in other states.

First, the bill's universal application distinguished it from reforms in California, Oregon, and other states that often exempted certain jurisdictions or zones from ADU requirements. SB 528 applied to all municipalities statewide, regardless of size or existing zoning codes, creating a uniform regulatory framework across Montana's diverse communities.

Second, the bill preempted a range of local restrictions that had historically limited ADU construction. Municipalities were prohibited from imposing owner-occupancy requirements, which had often discouraged investment by requiring property owners to live in either the primary dwelling or the ADU. The bill also eliminated minimum parking mandates for ADUs and prohibited localities from assessing impact fees on accessory units, removing two common barriers that added substantial costs to ADU development in other jurisdictions \cite{wegmann2014}.

Third, SB 528 capped ADU permit application fees at \$250, directly addressing concerns that high permitting costs had deterred construction in some jurisdictions. Studies of California's ADU reforms found that fees ranging from \$5,000 to \$20,000 in some cities significantly reduced ADU uptake \cite{garcia2017}. Montana's fee cap eliminated this barrier.

Fourth, the bill established by-right approval for ADUs meeting basic dimensional requirements. Units of up to 1,000 square feet or 75\% of the principal dwelling's size (whichever was smaller) that met standard setback requirements were to be approved administratively without discretionary review. This provision addressed concerns that lengthy approval processes and subjective design review had created uncertainty and delay for prospective ADU builders.

\subsubsection{Senate Bill 323: Duplex Legalization}

Senate Bill 323, passed concurrently with SB 528, legalized attached or detached duplexes in all residential zones where single-family homes were permitted. The bill applied to cities with populations exceeding 5,000 residents located in counties with populations exceeding 70,000.

This population threshold was designed to focus the reform on Montana's more urbanized areas while exempting small towns and rural areas where development pressure was less acute. The qualifying municipalities included Billings, Missoula, Great Falls, Bozeman, Helena, Butte, and Kalispell---accounting for approximately 40\% of Montana's population.

\subsubsection{Senate Bill 382: Montana Land Use Planning Act}

Senate Bill 382 complemented the zoning reforms by requiring qualifying municipalities to incorporate housing needs assessments into their comprehensive planning processes. The bill mandated that cities plan for housing growth commensurate with projected population increases, providing a forward-looking complement to the immediate zoning changes in SB 528 and SB 323.

\subsection{Legal Challenges and Implementation}

The Montana reforms faced immediate legal challenge. In late 2023, Montanans Against Irresponsible Densification (MAID), a citizens' group organized in opposition to the reforms, filed suit arguing that the bills violated Montana's constitutional equal protection provisions and substantive due process protections.

On December 29, 2023, Judge Salvagni of the Eighteenth Judicial District (Gallatin County) issued a preliminary injunction against implementation of SB 323 and SB 528. The injunction created substantial uncertainty about the reforms' legal status during the first 15 months of their intended effective period.

On March 3, 2025, the Montana Supreme Court lifted the injunction and upheld the constitutionality of the reforms. From that point forward, the reforms were in full legal effect.

This legal timeline is critical for interpreting the empirical analysis. The ``treatment'' began formally on January 1, 2024, but effective enforcement was uncertain until March 2025. Developers and property owners may have delayed projects during this period of legal limbo.

%==============================================================================
\section{Literature Review}
%==============================================================================

\subsection{The Housing Supply Elasticity Literature}

A foundational finding in housing economics is that regulatory constraints significantly affect housing supply elasticities. Saiz (2010) documented substantial variation in supply elasticities across U.S. metropolitan areas, finding that both geographic constraints and regulatory barriers explain much of this variation. Glaeser and Gyourko (2003) argued that housing prices in coastal cities increasingly reflect the cost of regulatory restrictions rather than construction costs.

More recent work has attempted to quantify the specific effects of zoning regulations. Glaeser, Gyourko, and Saks (2005) estimated that zoning restrictions add approximately 50\% to housing costs in Manhattan. Hsieh and Moretti (2019) argued that land use restrictions in high-productivity cities such as San Francisco and New York reduce aggregate U.S. GDP by approximately 9\% by preventing worker reallocation to productive areas.

\subsection{Evidence on Zoning Reform}

Empirical evidence on the effects of zoning liberalization is more limited, reflecting the recency of significant reform efforts. Several studies have examined California's 2016 ADU reforms (AB 2299 and SB 1069), which streamlined the approval process for accessory dwelling units.

Wegmann and Chapple (2014) conducted an early analysis of ADU production in California, finding that barriers to ADU construction were primarily regulatory rather than financial. Subsequent research by Garcia (2017) found that California's ADU reforms increased permit applications, though the magnitude varied substantially across jurisdictions depending on implementation.

The Minneapolis 2040 plan, which eliminated single-family-only zoning citywide in 2020, has been the subject of several recent studies. Kahlenberg (2019) examined the political economy of the reform, while Manville, Lens, and Monkkonen (2022) provided early descriptive evidence on permit activity.

Oregon's 2019 reforms (HB 2001), which required cities over 10,000 residents to allow duplexes in single-family zones, represent perhaps the closest precedent to Montana's reforms. However, systematic causal evaluation of these reforms remains limited.

The most rigorous evidence on upzoning comes from \cite{greenaway2021}, who studied Auckland, New Zealand's 2016 reforms that allowed higher-density construction in 75\% of the city's residential land. Using a spatial regression discontinuity design, they found substantial increases in building consents in newly upzoned areas, with effects concentrated in medium-density housing types. The Auckland evidence suggests that upzoning can meaningfully increase housing supply, though the effects take several years to materialize and depend on complementary policies.

\subsection{Methodological Literature on Difference-in-Differences}

Recent econometric research has highlighted important limitations of the standard two-way fixed effects (TWFE) estimator, particularly in settings with treatment effect heterogeneity or staggered adoption \cite{goodmanbacon2021, dechaisemartin2020, callaway2021}. While this study does not face the ``forbidden comparisons'' problem---since there is only one treated unit and all control units remain untreated---other methodological concerns are relevant.

With a single treated unit, the TWFE estimator is particularly sensitive to the selection of control units and may conflate pre-existing differences with treatment effects \cite{conleytaber2011, fermanpinto2019}. Standard cluster-robust inference is unreliable with few clusters, motivating the use of wild cluster bootstrap methods \cite{cameronetal2008}.

For single-treated-unit designs, synthetic control methods offer important advantages by constructing a data-driven counterfactual that matches the treated unit's pre-treatment trajectory \cite{abadie2010}. Permutation-based inference provides valid p-values even with a single treated unit. Recent developments in synthetic difference-in-differences combine the strengths of both approaches \cite{arkhangelsky2021}.

Event study designs allow visual assessment of the parallel trends assumption \cite{sunabraham2021, roth2022}. Pre-treatment coefficients that differ significantly from zero raise concerns about identification, though some pre-trend variation is expected with noisy data.

\subsection{Contribution of This Study}

This paper makes several contributions to the literature on housing supply and land use regulation. Most fundamentally, it provides the first empirical evaluation of Montana's comprehensive zoning reforms using a quasi-experimental research design. While Montana's reforms have attracted substantial media attention and policy interest---they have been described as the most aggressive state-level intervention into local zoning authority in U.S. history---no prior study has attempted to estimate their causal effects on housing construction.

Second, this study documents short-run effects during a period of significant implementation uncertainty. The preliminary injunction against Montana's reforms created a natural experiment within a natural experiment: the formal treatment date (January 2024) preceded the date when legal uncertainty was resolved (March 2025) by fifteen months. This timeline allows for analysis of whether legal challenges attenuate policy impacts, a question of substantial interest for states considering similar reforms that may face litigation from opponents.

Third, the paper employs a transparent difference-in-differences methodology that can be replicated and extended as additional data become available. By using publicly available data from the Census Bureau's Building Permits Survey and clearly documented control state selection criteria, this analysis establishes a baseline that can be updated in future years as longer post-treatment data accumulate. Such ongoing evaluation is essential given the substantial time lags involved in housing construction.

Finally, this study contributes to methodological discussions about evaluating place-based policies with limited post-treatment periods and small numbers of treated units. The use of wild cluster bootstrap inference, event study diagnostics, and robustness checks with alternative treatment definitions illustrates best practices for difference-in-differences analysis when statistical power is limited \cite{roth2022}.

%==============================================================================
\section{Data and Methods}
%==============================================================================

\subsection{Data Sources}

\subsubsection{Building Permits Survey}

The primary outcome variable is monthly building permits authorized at the state level, drawn from the Census Bureau's Building Permits Survey (BPS). The BPS collects data on new privately-owned residential construction from approximately 20,000 permit-issuing jurisdictions nationwide.

The BPS provides several advantages for this analysis. It is the only source of high-frequency (monthly) data on residential construction activity available at the state level. The data are collected consistently across jurisdictions and over time, facilitating comparisons between treatment and control states.

The analysis uses state-level monthly data from November 2019 through October 2025, providing 50 months of pre-treatment data and 22 months of post-treatment data. Data prior to November 2019 were unavailable in consistent format from the Census Bureau's public archives.

Building permits are measured in units authorized. The BPS breaks these into single-family units (1-unit structures), two-unit structures, structures with 3-4 units, and structures with 5 or more units. The analysis focuses on total units as the primary outcome, with heterogeneity analysis examining single-family versus multi-family permits separately.

\subsubsection{Population Data}

To construct per-capita measures, the analysis uses 2023 population estimates from the Census Bureau's Population Estimates Program (PEP). The PEP produces annual estimates of the resident population for states and counties, incorporating data on births, deaths, and domestic and international migration. Using population denominators allows for meaningful comparisons between states of different sizes and accounts for the fact that larger populations would naturally generate more building permits.

Per-capita rates are expressed per 100,000 population to facilitate interpretation. This scaling is standard in housing research and produces coefficients that are readily interpretable: an effect of 3 permits per 100,000 population implies approximately 34 additional permits per month for Montana (with a population of approximately 1.13 million) if the effect were to persist. The per-capita transformation also helps address concerns about heteroskedasticity that would arise from analyzing raw permit counts across states with substantially different population sizes.

A limitation of this approach is that population is measured at a single point (2023) rather than varying over time. Population growth during the study period was modest but not negligible for Montana (approximately 1.5\% annually during 2019-2023), and differential growth rates between Montana and control states could introduce bias. However, using time-varying population denominators would introduce additional noise and potential endogeneity concerns, as population growth may itself respond to housing market conditions \cite{glaeser2008}. The fixed-population approach is thus preferred as a more conservative specification.

\subsection{Control Group Selection}

The difference-in-differences design requires selection of an appropriate control group. This analysis uses five Mountain West states as controls: Wyoming, Idaho, North Dakota, South Dakota, and Nebraska. The selection of these states reflects careful consideration of the characteristics that make for valid counterfactuals in quasi-experimental research designs \cite{angrist2009}.

Geographic proximity is the first criterion. All five control states are located in the Northern Great Plains or Mountain West region, sharing similar climatic conditions, economic structures, and cultural characteristics with Montana. This regional similarity helps ensure that the control states face comparable underlying conditions affecting housing construction, such as seasonal building patterns and regional economic cycles.

The control states also experienced comparable housing market pressures during the 2020-2023 period, though to varying degrees. All five states saw housing demand increases during the pandemic-era migration from coastal urban areas, making them suitable comparators for Montana's experience. However, none enacted statewide zoning liberalization comparable to Montana's during the study period, ensuring that the comparison captures the specific effect of Montana's policy change rather than broader regional trends in regulatory reform.

Finally, all five control states are relatively low-population states, ranging from approximately 580,000 (Wyoming) to 1.98 million (Nebraska) residents. This similarity in scale helps reduce concerns about systematic differences in permit activity patterns that might arise from comparing Montana to much larger states with different housing market dynamics.

It is worth noting the limitations of this control group. Idaho, in particular, experienced substantial population growth and housing market appreciation during the study period. Wyoming's economy is heavily dependent on energy extraction, creating different cyclical dynamics. Robustness checks examining alternative control group specifications are presented in Section 5.7.

\subsection{Empirical Strategy}

\subsubsection{Difference-in-Differences Specification}

The main empirical specification is a standard two-way fixed effects difference-in-differences model:

\begin{equation}
Y_{st} = \alpha + \beta \cdot D_{st} + \gamma_s + \delta_t + \varepsilon_{st}
\end{equation}

where $Y_{st}$ is building permits per 100,000 population in state $s$ in month $t$, $D_{st}$ is an indicator equal to one for Montana in months on or after January 2024, $\gamma_s$ are state fixed effects, and $\delta_t$ are month-year fixed effects.

The coefficient of interest is $\beta$, which identifies the average treatment effect on the treated under the parallel trends assumption. State fixed effects absorb time-invariant differences between Montana and control states, while month-year fixed effects absorb common shocks affecting all states (such as national interest rate changes or macroeconomic conditions).

Standard errors are clustered at the state level to account for serial correlation within states over time. Given the small number of clusters (6 states), inference is also assessed using wild cluster bootstrap methods \cite{cameronetal2008}.

\subsubsection{Synthetic Control Method}

Given that this analysis involves a single treated unit (Montana), the standard TWFE DiD estimator may be inferior to Synthetic Control Methods (SCM) that construct a data-driven counterfactual \cite{abadie2010}. SCM addresses a key limitation of the standard approach: rather than using an unweighted average of control states, it constructs a weighted combination of donors that matches the treated unit's pre-treatment trajectory.

Following \cite{abadie2010}, the synthetic control estimator finds weights $W = (w_1, ..., w_J)$ for the $J$ donor states that minimize the pre-treatment prediction error:

\begin{equation}
\min_W \sum_{t < T_0} \left( Y_{MT,t} - \sum_{j=1}^J w_j Y_{j,t} \right)^2 \quad \text{s.t.} \quad w_j \geq 0, \sum_j w_j = 1
\end{equation}

The treatment effect is then estimated as the gap between Montana's actual outcome and the synthetic control in the post-treatment period. Inference proceeds via placebo tests: the SCM procedure is applied to each donor state as if it were treated, generating a distribution of placebo effects \cite{abadie2015}. The p-value equals the proportion of placebo effects at least as large as Montana's.

This approach is particularly appropriate when: (a) there is a single treated unit, (b) the donor pool is limited, and (c) pre-treatment fit is a primary concern \cite{arkhangelsky2021}. All 47 non-reform states serve as the donor pool, excluding states with major zoning reforms during the study period (California, Oregon, Maine).

\subsubsection{Identification Assumptions}

The key identifying assumption is parallel trends: absent Montana's zoning reforms, building permits in Montana would have followed the same trajectory as in the control states. This assumption cannot be tested directly, but its plausibility can be assessed by examining pre-treatment trends.

Several threats to identification are worth noting. The most significant concern involves differential recovery trajectories from the COVID-19 pandemic. Housing markets were substantially disrupted during 2020-2022, with construction activity fluctuating in response to supply chain disruptions, labor shortages, and demand shifts. If Montana's housing market recovered on a different trajectory than control states for reasons unrelated to zoning reform, this could confound the estimated treatment effect. The event study analysis presented in Section 5.5 provides some evidence on pre-treatment trends, though as discussed below, the results suggest some concern about parallel trends.

A second threat involves migration patterns. Montana experienced substantial in-migration during the study period, driven by remote work opportunities and lifestyle preferences that intensified during the pandemic \cite{whitaker2021}. If this migration was itself influenced by anticipation of the zoning reforms---for example, if developers began planning projects in expectation of the policy change before its formal effective date---this could bias the estimates. However, given the reforms were not enacted until May 2023 and faced immediate legal challenge, anticipation effects seem unlikely to substantially affect pre-treatment outcomes.

A third concern involves the legal uncertainty created by the preliminary injunction against the reforms. As discussed in Section 2.3, the injunction was in effect from December 2023 through March 2025, creating ambiguity about the treatment's effective start date. This uncertainty may have attenuated estimated effects if developers postponed projects while awaiting legal resolution. The robustness check using March 2025 as an alternative treatment date partially addresses this concern.

%==============================================================================
\section{Results}
%==============================================================================

\subsection{Summary Statistics}

Table \ref{tab:summary} presents summary statistics for Montana and the control states in the pre-treatment and post-treatment periods.

\begin{table}[H]
\centering
\caption{Summary Statistics: Building Permits by Period}
\label{tab:summary}
\begin{tabular}{llcccc}
\toprule
Group & Period & N & Total Permits & Per Capita & Single-Family \\
\midrule
Montana & Pre-Treatment & 50 & 447.4 & 39.5 & 226.7 \\
Montana & Post-Treatment & 22 & 436.2 & 38.5 & 242.1 \\
Control States & Pre-Treatment & 250 & 676.5 & 50.5 & 423.9 \\
Control States & Post-Treatment & 110 & 650.8 & 46.3 & 452.9 \\
\bottomrule
\end{tabular}
\caption*{\footnotesize Notes: Per capita rates are per 100,000 population. Control states are Wyoming, Idaho, North Dakota, South Dakota, and Nebraska. Pre-treatment period is November 2019 through December 2023 (50 months). Post-treatment period is January 2024 through October 2025 (22 months).}
\end{table}

Several patterns are noteworthy. First, Montana has lower average permit rates than the control states in both periods, reflecting its smaller population and more rural character. Second, both Montana and control states show modest declines in per-capita permits between the pre- and post-periods, likely reflecting the effects of rising interest rates on housing construction nationwide. Third, single-family permits increased slightly in both groups while multi-family permits declined.

\subsection{Visual Evidence}

Figure \ref{fig:trends} presents the raw time series of building permits per 100,000 population for Montana (solid blue line) and the control state average (dashed red line). The vertical dashed line indicates the treatment date (January 2024), and the shaded region indicates the period of legal uncertainty (December 2023 through March 2025).

\begin{figure}[H]
\centering
\includegraphics[width=0.95\textwidth]{figures/fig1_raw_trends.png}
\caption{Building Permits per 100,000 Population: Montana vs. Control States}
\label{fig:trends}
\caption*{\footnotesize Notes: The solid blue line shows Montana; the dashed red line shows the average of control states (Wyoming, Idaho, North Dakota, South Dakota, Nebraska). The vertical dotted line marks January 2024 (treatment). The gray shaded region indicates the period of legal uncertainty due to the preliminary injunction (December 2023 - March 2025).}
\end{figure}

The figure reveals several patterns. Both Montana and control states display substantial month-to-month volatility in permit activity. There is no obvious divergence between Montana and control states following the treatment date. If anything, Montana's permit rate appears to track the control group average closely throughout the post-treatment period.

\subsection{Main Difference-in-Differences Results}

Table \ref{tab:did} presents results from the main difference-in-differences specification.

\begin{table}[H]
\centering
\caption{Difference-in-Differences Estimates: Effect of Montana Zoning Reform on Building Permits}
\label{tab:did}
\begin{tabular}{lc}
\toprule
& Permits per 100K \\
\midrule
Treatment (Montana $\times$ Post) & 3.233 \\
& (3.645) \\
& [--3.91, 10.38] \\
\\
State Fixed Effects & Yes \\
Month-Year Fixed Effects & Yes \\
\\
R-squared & 0.682 \\
N & 432 \\
\midrule
Clustered p-value & 0.375 \\
Wild bootstrap p-value & 0.487 \\
\bottomrule
\end{tabular}
\caption*{\footnotesize Notes: Standard errors clustered at the state level in parentheses, 95\% confidence interval in brackets. Wild cluster bootstrap p-value computed using 999 iterations with Rademacher weights. The treatment is defined as Montana in months on or after January 2024. * p $<$ 0.10, ** p $<$ 0.05, *** p $<$ 0.01.}
\end{table}

The estimated treatment effect is 3.23 additional permits per 100,000 population per month. This represents an 8.2\% increase relative to Montana's pre-treatment mean of 39.5 permits per 100,000. However, the estimate is not statistically significant at conventional levels (p = 0.375), with a standard error of 3.65.

The 95\% confidence interval ranges from approximately -4.1 to 10.5 permits per 100,000, indicating substantial uncertainty about the true effect size. The analysis cannot rule out either a null effect or a meaningful positive effect.

\subsection{Simple Difference-in-Differences Calculation}

As a complement to the regression analysis, the simple difference-in-differences estimate is computed as follows:

\begin{align}
\text{DiD} &= (Y^{MT}_{post} - Y^{MT}_{pre}) - (Y^{Control}_{post} - Y^{Control}_{pre}) \\
&= (38.51 - 39.49) - (46.25 - 50.47) \\
&= (-0.99) - (-4.22) \\
&= 3.23
\end{align}

The simple calculation confirms the regression estimate. Montana's per-capita permits declined by approximately 1.0 permits/100K between periods, while control state permits declined by 4.2 permits/100K. The difference---3.2 permits/100K---represents the estimated treatment effect.

\subsection{Event Study Analysis}

Figure \ref{fig:eventstudy} presents event study coefficients, showing the difference between Montana and control states by quarter relative to the January 2024 treatment date. The reference period is quarter $-1$ (October--December 2023), normalized to zero.

\begin{figure}[H]
\centering
\includegraphics[width=0.95\textwidth]{figures/fig3_event_study.png}
\caption{Event Study: Montana vs. Control States by Quarter}
\label{fig:eventstudy}
\caption*{\footnotesize Notes: Points show coefficient estimates for Montana $\times$ event-quarter interactions with 95\% confidence intervals. Reference quarter is $q=-1$. Vertical dashed line indicates treatment onset (January 2024).}
\end{figure}

The event study reveals important patterns. First, the pre-treatment coefficients are not uniformly close to zero, suggesting some violation of the parallel trends assumption. Montana's permit rate fluctuated relative to control states throughout the pre-period, with coefficients ranging from approximately 0 to 21. This pattern cautions against strong causal interpretation of the main results.

Second, the post-treatment coefficients show no clear upward trend. While some quarters show positive effects (notably quarter 7, corresponding to late 2025), others are near zero or negative. The lack of a clear dynamic pattern is consistent with the null finding from the main specification.

\subsection{Heterogeneity by Permit Type}

Table \ref{tab:heterogeneity} examines heterogeneity by permit type using the full two-way fixed effects specification. Given that Montana's reforms primarily targeted ADUs and duplexes, one might expect larger effects on multi-family permits.

\begin{table}[H]
\centering
\caption{Heterogeneity by Permit Type: Regression Estimates}
\label{tab:heterogeneity}
\begin{tabular}{lcc}
\toprule
& Single-Family & Multi-Family \\
\midrule
Treatment Effect & 0.178 & 3.055 \\
& (1.606) & (3.047) \\
& [--2.97, 3.33] & [--2.92, 9.03] \\
\\
p-value & 0.912 & 0.316 \\
N & 432 & 432 \\
\bottomrule
\end{tabular}
\caption*{\footnotesize Notes: Estimates from separate two-way fixed effects regressions. Standard errors clustered at state level in parentheses, 95\% confidence intervals in brackets.}
\end{table}

The results suggest that the positive effect is concentrated in multi-family permits (coefficient = 3.05) rather than single-family permits (coefficient = 0.18). This pattern is consistent with the policy's focus on ADUs and duplexes. However, neither coefficient is statistically significant, and the confidence intervals are wide.

\subsection{Robustness Checks}

Table \ref{tab:robustness} presents results from alternative specifications to assess the robustness of the main findings.

\begin{table}[H]
\centering
\caption{Robustness Checks}
\label{tab:robustness}
\begin{tabular}{lccc}
\toprule
Specification & Coefficient & SE & p-value \\
\midrule
Main (5 control states) & 3.23 & 3.65 & 0.375 \\
Excluding Idaho & 3.96 & 4.63 & 0.393 \\
Treatment: March 2025 & 5.90 & 3.13 & 0.059 \\
\bottomrule
\end{tabular}
\caption*{\footnotesize Notes: All specifications include state and month-year fixed effects with standard errors clustered at the state level. ``Excluding Idaho'' drops Idaho from the control group due to its rapid population growth. ``Treatment: March 2025'' uses the date the injunction was lifted as the treatment onset rather than January 2024.}
\end{table}

Several patterns emerge from the robustness analysis. First, excluding Idaho---which experienced exceptionally rapid population and construction growth during the study period---slightly increases the point estimate but also increases standard errors, leaving the null finding unchanged.

Second, and more intriguingly, using March 2025 as the treatment date (when the Montana Supreme Court lifted the injunction) yields a larger coefficient (5.90) that approaches statistical significance (p = 0.059). This specification is arguably more appropriate given that the reforms faced substantial legal uncertainty from their January 2024 effective date until March 2025. The larger effect under this specification suggests that developers may have indeed waited for legal clarity before responding to the reforms.

However, the March 2025 specification has only 8 months of post-treatment data (March--October 2025), limiting statistical power and raising concerns about whether the effect reflects the policy or simply coincident trends.

\subsection{Synthetic Control Analysis}

Given the single-treated-unit design, the Synthetic Control Method provides a more appropriate estimator than standard TWFE \cite{abadie2010, conleytaber2011}. Table \ref{tab:scm} presents results from the SCM analysis using all 47 non-reform states as potential donors.

\begin{table}[H]
\centering
\caption{Synthetic Control Estimates}
\label{tab:scm}
\begin{tabular}{lc}
\toprule
& Permits per 100K \\
\midrule
Treatment Effect (SCM) & $-0.712$ \\
Pre-treatment RMSPE & 10.056 \\
Post-treatment RMSPE & 12.804 \\
RMSPE Ratio & 1.27 \\
\\
Permutation p-value (two-tailed) & 0.958 \\
Montana rank (of 48 states) & 20 \\
\bottomrule
\end{tabular}
\caption*{\footnotesize Notes: Synthetic control constructed from 47 donor states. Treatment effect is average gap in post-period. RMSPE = root mean squared prediction error. Permutation p-value based on placebo tests treating each donor as ``treated.''}
\end{table}

The synthetic control estimate is $-0.71$ permits per 100,000, indicating essentially no effect of the reform. The pre-treatment RMSPE of 10.1 reflects imperfect pre-period fit, a common challenge with monthly data exhibiting substantial volatility. The top donor weights are Wyoming (22\%), Iowa (21\%), South Carolina (20\%), and Rhode Island (11\%)---a mix of Mountain West and other low-population states.

Figure \ref{fig:scm} shows Montana versus the synthetic control over time. The gap between actual and synthetic Montana fluctuates around zero throughout the post-period, with no clear treatment effect visible.

\begin{figure}[H]
\centering
\includegraphics[width=0.95\textwidth]{figures/fig4_synthetic_control.png}
\caption{Synthetic Control: Montana vs. Synthetic Montana}
\label{fig:scm}
\caption*{\footnotesize Notes: Top panel shows actual Montana (blue) versus synthetic Montana (red dashed). Bottom panel shows the gap (Montana minus synthetic). Vertical line marks treatment onset (January 2024). Gray shading indicates legal uncertainty period.}
\end{figure}

Permutation inference confirms the null finding. Montana's effect ranks 20th out of 48 states---squarely in the middle of the placebo distribution. The two-tailed p-value of 0.958 indicates that Montana's post-treatment gap is entirely unremarkable relative to placebo states. This provides strong evidence that the TWFE estimate of +3.2 permits may reflect pre-existing differences rather than a causal effect.

%==============================================================================
\section{Discussion}
%==============================================================================

\subsection{Interpretation of Results}

The analysis produces conflicting estimates depending on methodology. The standard TWFE DiD estimate is +3.2 permits per 100,000 per month, but this estimate is statistically insignificant and the event study reveals problematic pre-trends. The synthetic control estimate is $-0.7$ permits per 100,000---essentially zero---with Montana ranking 20th of 48 states in the placebo distribution (p = 0.958).

The discrepancy between estimates is informative. The TWFE estimator uses an unweighted average of five hand-selected control states, while SCM optimally weights 47 donors to match Montana's pre-treatment trajectory. The fact that SCM finds no effect while TWFE finds a positive (though insignificant) effect suggests the TWFE estimate may reflect pre-existing differences between Montana and the control states rather than a causal effect of the reform.

The event study analysis reinforces concerns about identification. Pre-treatment coefficients show substantial variation, with Montana's permit rate fluctuating relative to control states throughout the pre-period. The magnitude of pre-treatment coefficients (ranging from 0 to 21) indicates that Montana and control states did not follow parallel trajectories before the policy change. This pattern strongly cautions against causal interpretation of either estimate.

Several factors likely contribute to the imprecision of these estimates:

\textbf{Limited post-treatment period.} The analysis includes only 22 months of post-treatment data. Building permits represent intentions to construct, not completed construction, and there are typically lags between policy changes and permit applications. \cite{mayersomerville2000} demonstrate that the adjustment process in housing construction involves substantial delays, as developers must acquire sites, secure financing, obtain permits, and coordinate construction. These dynamics suggest effects may require longer time horizons to materialize.

\textbf{Legal uncertainty.} The preliminary injunction against Montana's reforms created substantial uncertainty from December 2023 through March 2025. Developers may have postponed projects during this period, waiting for legal clarity before investing in project planning.

\textbf{Housing market conditions.} The post-treatment period coincided with elevated interest rates and broader housing market cooling. Mortgage rates exceeded 7\% through much of 2024, dampening construction activity nationwide. \cite{gyourkosaiz2004} show that construction costs and labor availability can bind even when regulatory constraints are relaxed, implying that zoning reform alone may not increase construction if other supply-side constraints are binding. These headwinds may have offset any positive effects from zoning liberalization.

\textbf{Small sample size.} With only 6 states in the analysis (Montana plus 5 controls), statistical power is limited. The clustered standard errors may also be conservative given the small number of clusters.

\subsection{Comparison to Other Studies}

The modest short-run effects documented here are broadly consistent with findings from other zoning reform evaluations. Studies of California's ADU reforms have found meaningful but gradual increases in ADU construction over time, with effects becoming more pronounced several years after implementation.

Research on Minneapolis's 2040 plan has similarly found limited short-run effects on permit activity, though longer-run effects remain to be evaluated. These findings suggest that zoning reforms may take considerable time to translate into measurable increases in housing construction.

\subsection{Policy Implications}

The null finding in this analysis should not be interpreted as evidence that Montana's zoning reforms are ineffective. Rather, it indicates that effects have not yet manifested in ways that are statistically distinguishable from noise in the available data.

Several implications follow for policymakers considering similar reforms. First, patience is warranted in evaluating zoning liberalization initiatives. Housing supply responses to regulatory changes are likely to unfold over years rather than months, as developers must identify sites, secure financing, obtain permits, and complete construction. Evaluation frameworks that assess policy impacts within the first year or two may substantially underestimate long-run effects. Studies of California's ADU reforms found that permit applications increased gradually over a five-year period following implementation \cite{garcia2017}, suggesting that Montana's reforms may similarly require extended time horizons for full assessment.

Second, legal stability matters substantially for policy effectiveness. The preliminary injunction against Montana's reforms likely delayed their effects by creating uncertainty for developers considering investments. States pursuing zoning reform would benefit from careful attention to constitutional considerations and stakeholder engagement during the legislative process to minimize the risk of legal challenge. The robustness check using March 2025 as the treatment date suggests meaningfully larger effects once legal uncertainty was resolved, providing some evidence that litigation can substantially delay policy impacts.

Third, complementary policies may be needed to realize the full potential of zoning reform. Zoning liberalization removes regulatory barriers but does not directly incentivize construction or address other constraints such as construction costs, labor availability, or financing challenges. Tax incentives for ADU construction, low-interest loan programs, pre-approved design templates, and streamlined permitting processes may all help translate regulatory permission into actual housing units. The experience of California cities that combined zoning reform with incentive programs suggests that such complementary approaches can accelerate ADU production \cite{chapple2020}.

\subsection{Limitations}

Several limitations of this analysis should be acknowledged. The most significant data limitation is that the Building Permits Survey captures building permits at the state level but does not distinguish ADUs from other types of construction. The BPS categorizes structures by number of units (single-family, two-unit, three-to-four units, five-plus units), but does not separately identify accessory dwelling units, which are typically classified as single-family structures regardless of whether they represent additions to existing lots. A more granular analysis using local permit data from Montana municipalities could better identify the specific effects of ADU and duplex legalization, though such data collection would be substantially more resource-intensive.

The choice of control states involves judgment, and different selections could yield different results. The analysis assumes that Mountain West states---Wyoming, Idaho, North Dakota, South Dakota, and Nebraska---provide valid counterfactuals for Montana in the absence of zoning reform. Alternative approaches such as synthetic control methods could potentially improve the match between Montana and its counterfactual \cite{abadie2010}, though the small number of potential donor states limits the applicability of such methods in this context.

The short post-treatment period represents perhaps the most fundamental limitation. With less than two years of post-treatment data (and only eight months following the resolution of legal uncertainty), the analysis cannot speak to longer-run effects that may be more policy-relevant. Housing construction involves substantial lead times, and the full effects of zoning reform may take five to ten years to manifest fully in permit data and completed housing units.

Finally, the analysis does not capture potential general equilibrium effects. If Montana's reforms affect housing prices or migration patterns in ways that spill over to control states, the difference-in-differences estimator would not fully capture the policy's effects. Similarly, the analysis focuses on construction activity rather than housing prices or affordability outcomes, which are ultimately of greater policy interest but require additional data and longer time horizons to assess.

%==============================================================================
\section{Conclusion}
%==============================================================================

This paper provides the first empirical evaluation of Montana's 2023 zoning reforms using both difference-in-differences and synthetic control methods. The analysis produces conflicting estimates: a positive but insignificant TWFE estimate of +3.2 permits per 100,000 per month versus a synthetic control estimate of essentially zero ($-0.7$ permits/100K, p = 0.958). Event study diagnostics reveal problematic pre-trends, and the conflicting estimates across methodologies suggest that credible identification is not achieved with available data.

The null finding should not be interpreted as evidence that Montana's reforms are ineffective. Rather, the analysis illustrates fundamental challenges in evaluating single-treated-unit policy reforms using state-level data. Several design limitations prevent strong conclusions. First, the Building Permits Survey does not identify ADUs separately, creating measurement error relative to the policy margin. Second, state-level aggregation dilutes treatment exposure, since SB 323 (duplex legalization) applied only to cities exceeding 5,000 population. Third, with a single treated state, inference depends critically on control group selection and pre-treatment fit, neither of which is fully satisfactory.

Future research should exploit within-state variation using place-level permit data. Montana cities above and below the 5,000 population threshold provide a natural comparison group for triple-difference designs that would dramatically improve identification. Such analyses would require accessing the Building Permits Survey's place-level files or collecting administrative data directly from Montana municipalities.

Despite these limitations, the analysis makes methodological contributions by demonstrating the importance of synthetic control methods and permutation inference for single-treated-unit designs. The substantial divergence between TWFE and SCM estimates (a difference of nearly 4 permits/100K) illustrates how standard approaches can produce misleading results when identification assumptions are not satisfied. Researchers evaluating other statewide zoning reforms should consider these issues carefully before drawing causal conclusions from state-level comparisons.

\newpage
\bibliographystyle{plainnat}
\begin{thebibliography}{99}

\bibitem{garcia2017} Garcia, D. (2017). ADU Update: Early Lessons and Impacts of California's State and Local Policy Changes. Terner Center for Housing Innovation, UC Berkeley.

\bibitem{glaeser2003} Glaeser, E. L., \& Gyourko, J. (2003). The impact of building restrictions on housing affordability. \textit{Federal Reserve Bank of New York Economic Policy Review}, 9(2), 21-39.

\bibitem{glaeser2005} Glaeser, E. L., Gyourko, J., \& Saks, R. E. (2005). Why is Manhattan so expensive? Regulation and the rise in housing prices. \textit{The Journal of Law and Economics}, 48(2), 331-369.

\bibitem{hsieh2019} Hsieh, C. T., \& Moretti, E. (2019). Housing constraints and spatial misallocation. \textit{American Economic Journal: Macroeconomics}, 11(2), 1-39.

\bibitem{kahlenberg2019} Kahlenberg, R. D. (2019). How Minneapolis Confronted Its History of Housing Segregation. \textit{The Century Foundation}.

\bibitem{manville2022} Manville, M., Lens, M., \& Monkkonen, P. (2022). Zoning and affordability: A reply to critics. \textit{Urban Studies}, 59(4), 683-698.

\bibitem{saiz2010} Saiz, A. (2010). The geographic determinants of housing supply. \textit{The Quarterly Journal of Economics}, 125(3), 1253-1296.

\bibitem{wegmann2014} Wegmann, J., \& Chapple, K. (2014). Hidden density in single-family neighborhoods: Backyard cottages as an equitable smart growth strategy. \textit{Journal of Urbanism}, 7(3), 307-329.

\bibitem{abadie2010} Abadie, A., Diamond, A., \& Hainmueller, J. (2010). Synthetic control methods for comparative case studies: Estimating the effect of California's tobacco control program. \textit{Journal of the American Statistical Association}, 105(490), 493-505.

\bibitem{angrist2009} Angrist, J. D., \& Pischke, J. S. (2009). \textit{Mostly Harmless Econometrics: An Empiricist's Companion}. Princeton University Press.

\bibitem{roth2022} Roth, J., Sant'Anna, P. H., Bilinski, A., \& Poe, J. (2023). What's trending in difference-in-differences? A synthesis of the recent econometrics literature. \textit{Journal of Econometrics}, 235(2), 2218-2244.

\bibitem{callaway2021} Callaway, B., \& Sant'Anna, P. H. C. (2021). Difference-in-differences with multiple time periods. \textit{Journal of Econometrics}, 225(2), 200-230.

\bibitem{goodmanbacon2021} Goodman-Bacon, A. (2021). Difference-in-differences with variation in treatment timing. \textit{Journal of Econometrics}, 225(2), 254-277.

\bibitem{sunabraham2021} Sun, L., \& Abraham, S. (2021). Estimating dynamic treatment effects in event studies with heterogeneous treatment effects. \textit{Journal of Econometrics}, 225(2), 175-199.

\bibitem{dechaisemartin2020} de Chaisemartin, C., \& d'Haultf{\oe}uille, X. (2020). Two-way fixed effects estimators with heterogeneous treatment effects. \textit{American Economic Review}, 110(9), 2964-2996.

\bibitem{conleytaber2011} Conley, T. G., \& Taber, C. R. (2011). Inference with ``difference in differences'' with a small number of policy changes. \textit{The Review of Economics and Statistics}, 93(1), 113-125.

\bibitem{fermanpinto2019} Ferman, B., \& Pinto, C. (2019). Inference in differences-in-differences with few treated groups and heteroskedasticity. \textit{Journal of Econometrics}, 218(2), 343-365.

\bibitem{cameronetal2008} Cameron, A. C., Gelbach, J. B., \& Miller, D. L. (2008). Bootstrap-based improvements for inference with clustered errors. \textit{The Review of Economics and Statistics}, 90(3), 414-427.

\bibitem{arkhangelsky2021} Arkhangelsky, D., Athey, S., Hirshberg, D. A., Imbens, G. W., \& Wager, S. (2021). Synthetic difference-in-differences. \textit{American Economic Review}, 111(12), 4088-4118.

\bibitem{greenaway2021} Greenaway-McGrevy, R., \& Phillips, P. C. B. (2021). The impact of upzoning on housing construction in Auckland. \textit{Journal of Urban Economics}, 121, 103316.

\bibitem{mayersomerville2000} Mayer, C. J., \& Somerville, C. T. (2000). Land use regulation and new construction. \textit{Regional Science and Urban Economics}, 30(6), 639-662.

\bibitem{gyourkosaiz2004} Gyourko, J., \& Saiz, A. (2004). Reinvestment in the housing stock: The role of construction costs and the supply side. \textit{Journal of Urban Economics}, 55(2), 238-256.

\bibitem{abadie2015} Abadie, A., Diamond, A., \& Hainmueller, J. (2015). Comparative politics and the synthetic control method. \textit{American Journal of Political Science}, 59(2), 495-510.

\bibitem{xu2017} Xu, Y. (2017). Generalized synthetic control method: Causal inference with interactive fixed effects models. \textit{Political Analysis}, 25(1), 57-76.

\bibitem{whitaker2021} Whitaker, S. D. (2021). Did the COVID-19 pandemic cause an urban exodus? \textit{Federal Reserve Bank of Cleveland District Data Briefs}.

\bibitem{chapple2020} Chapple, K., Garcia, D., \& Valchuis, E. (2020). Reaching California's ADU Potential: Progress to Date and the Need for ADU Finance. Terner Center for Housing Innovation, UC Berkeley.

\bibitem{glaeser2008} Glaeser, E. L. (2008). \textit{Cities, Agglomeration, and Spatial Equilibrium}. Oxford University Press.

\bibitem{gyourko2018} Gyourko, J., \& Molloy, R. (2015). Regulation and housing supply. In \textit{Handbook of Regional and Urban Economics} (Vol. 5, pp. 1289-1337). Elsevier.

\bibitem{quigley2005} Quigley, J. M., \& Rosenthal, L. A. (2005). The effects of land use regulation on the price of housing: What do we know? What can we learn? \textit{Cityscape}, 8(1), 69-137.

\bibitem{been2014} Been, V., Ellen, I. G., \& O'Regan, K. (2019). Supply skepticism: Housing supply and affordability. \textit{Housing Policy Debate}, 29(1), 25-40.

\end{thebibliography}

\newpage
\appendix
\section{Additional Figures}

\begin{figure}[H]
\centering
\includegraphics[width=0.95\textwidth]{figures/fig2_permits_by_type.png}
\caption{Building Permits by Type: Single-Family vs. Multi-Family}
\label{fig:bytype}
\caption*{\footnotesize Notes: Left panel shows single-family permits per 100,000 population; right panel shows multi-family permits. The solid blue line shows Montana; the dashed red line shows the control state average.}
\end{figure}

\begin{figure}[H]
\centering
\includegraphics[width=0.95\textwidth]{figures/fig5_placebo_distribution.png}
\caption{Placebo Test Distribution: Montana vs. All Donor States}
\label{fig:placebo}
\caption*{\footnotesize Notes: Each bar shows the average post-treatment effect for a state using synthetic control with remaining states as donors. Montana (red) ranks 20th of 48 states. Gray bars are placebo states. States with poor pre-fit (top 10\% of pre-RMSPE) excluded.}
\end{figure}

\end{document}
