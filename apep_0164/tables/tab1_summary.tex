\begin{table}[htbp]
\centering
\caption{Summary Statistics: Postpartum Women (Pre-Treatment, 2017--2019)}
\label{tab:summary}
\begin{tabular}{lcc}
\toprule
 & Treated States & Control States \\
\midrule
N & 97,592 & 4,552 \\
Medicaid (\%) & 29.4 & 29.2 \\
Uninsured (\%) & 11.9 & 10.2 \\
Employer Ins (\%) & 53.8 & 58.0 \\
Age & 30.1 & 29.3 \\
Married (\%) & 64.9 & 68.4 \\
White NH (\%) & 52.2 & 73.4 \\
Black NH (\%) & 14.6 & 9.8 \\
Hispanic (\%) & 22.7 & 9.9 \\
BA+ (\%) & 35.2 & 33.3 \\
Below 200\% FPL (\%) & 42.5 & 45.3 \\
\midrule
States (clusters) & 47 & 4 \\
\bottomrule
\end{tabular}
\begin{tablenotes}[flushleft]
\small
\item \textit{Notes:} Sample is women aged 18--44 who gave birth in the past 12 months.
Pre-treatment period is 2017--2019 (before PHE and policy adoption).
Statistics are weighted using ACS person weights.
Treated states adopted the 12-month postpartum extension by 2024.
Control states: AR, WI (never adopted), ID, IA (adopt 2025). Total clusters: 51.
\end{tablenotes}
\end{table}

