\documentclass[12pt,letterpaper]{article}
\usepackage[margin=1in]{geometry}
\usepackage{graphicx}
\usepackage{booktabs}
\usepackage{amsmath}
\usepackage{hyperref}
\usepackage{natbib}
\usepackage{setspace}
\usepackage{float}
\usepackage{caption}
\usepackage{subcaption}
\usepackage{xcolor}

\doublespacing

\title{Does Free College for Foster Youth Increase Educational Attainment? \\ Evidence from Pennsylvania's Fostering Independence Tuition Waiver}

\author{APEP Autonomous Research\\Autonomous Policy Evaluation Project nd @dakoyana}

\date{January 2026\\Working Paper APEP-XXXX}

\begin{document}

\maketitle

\begin{abstract}
Foster youth face substantial barriers to higher education, including housing instability, financial constraints, and lack of family support. This paper provides descriptive evidence on Pennsylvania's Fostering Independence Tuition Waiver (FosterED), which provides full tuition waivers to former foster youth at state colleges. Using a difference-in-differences design comparing Pennsylvania to neighboring states before and after the 2019 policy implementation, we document small positive point estimates for college enrollment (+0.20 percentage points) and any college attainment (+0.89 percentage points) among the general young adult population, though these estimates are not statistically distinguishable from zero given inferential challenges with few state-level clusters. These modest point estimates likely reflect substantial dilution from an intent-to-treat design, as foster youth comprise less than one percent of the sample. Heterogeneity analysis suggests slightly larger point estimates among older young adults (ages 22-26). We discuss the fundamental limitations of population-level surveys and difference-in-differences designs with few clusters for evaluating targeted interventions, and highlight the need for linked administrative data to credibly estimate the program's causal effect on the intended beneficiaries.

\bigskip
\noindent \textbf{Keywords:} Foster care, higher education, tuition waivers, difference-in-differences, educational attainment

\bigskip
\noindent \textbf{JEL Codes:} I22, I23, I38, J13
\end{abstract}

\newpage
\tableofcontents
\newpage

%===================================================================
\section{Introduction}
%===================================================================

Each year, approximately 20,000 young people ``age out'' of the foster care system in the United States upon reaching adulthood, typically at age 18 or 21 depending on the state \citep{childwelfare2023}. These youth face daunting challenges as they transition to independence without the family support networks that most young adults rely upon. Research has consistently documented poor educational outcomes for this population: only about 3-4\% of former foster youth complete a four-year college degree, compared to roughly 33\% of their peers in the general population \citep{courtney2011midwest, pecora2006educational}. This stark disparity reflects the cumulative disadvantages that foster youth experience throughout childhood and adolescence, from school instability to trauma exposure to the abrupt severance of support systems at the age of majority.

The barriers to higher education for foster youth are multifaceted and reinforcing. Financial constraints are paramount---foster youth often lack access to parental financial support, face difficulties completing financial aid applications without parental information, and must balance work and school to cover living expenses \citep{salazar2013family, unrau2012ready}. Beyond finances, foster youth experience higher rates of housing instability, food insecurity, and mental health challenges, all of which can impede academic success \citep{rebbe2017food, pecora2009mental}. The absence of a permanent family also means that foster youth lack the informal advising and encouragement that many students receive from parents navigating the college application and enrollment process. The cumulative effect of these disadvantages creates a substantial gap in educational attainment that perpetuates cycles of economic disadvantage across generations.

Recognizing these challenges, many states have enacted tuition waiver programs specifically targeting former foster youth. As of 2023, at least 33 states offer some form of tuition assistance or waiver for this population \citep{ncsl2023fostertuition}. These programs vary considerably in their generosity, eligibility criteria, and scope. Some provide full tuition waivers at public institutions, while others offer partial scholarships or limit benefits to community colleges. The design of these programs raises important questions about what features are most effective at increasing educational attainment, and whether removing financial barriers alone is sufficient to close the college completion gap.

Pennsylvania's Fostering Independence Tuition Waiver, enacted through Act 16 of 2019 and commonly known as ``FosterED,'' represents one of the more comprehensive state programs in the nation. The policy provides a full waiver of tuition and mandatory fees at all Pennsylvania public and private postsecondary institutions for youth who were in foster care at age 16 or older. Benefits extend for up to five years or until age 26, providing flexibility for youth who may need to interrupt their education. The program is administered in conjunction with the federal Chafee Education and Training Grant, which can cover non-tuition expenses such as housing, books, and transportation. The policy's design---with its focus on youth who remained in foster care through adolescence---reflects an understanding that older foster youth face the greatest barriers to educational success and may benefit most from comprehensive support.

This paper evaluates whether Pennsylvania's FosterED program increased educational attainment among young adults. We employ a difference-in-differences research design, comparing college enrollment and attainment trends in Pennsylvania relative to neighboring states (New York, New Jersey, Ohio, and West Virginia) before and after the policy's implementation in 2019. Our analysis uses individual-level data from the American Community Survey Public Use Microdata Sample (ACS PUMS) for the years 2017-2018 (pre-period) and 2021-2022 (post-period). This approach allows us to control for time-invariant state characteristics and common national trends affecting all states, isolating the effect of Pennsylvania's policy change.

Our findings reveal small positive effects of the policy on educational outcomes when examining the entire young adult population. The difference-in-differences estimate for current college enrollment is +0.20 percentage points, while the estimate for any college attainment is +0.89 percentage points. For bachelor's degree attainment, we find a small negative effect (-0.10 percentage points), though this is likely noise given the short time horizon for degree completion effects to materialize. Heterogeneity analysis by age group suggests slightly larger effects among older young adults (ages 22-26), which is consistent with the policy's focus on youth aging out of care at older ages who would be most directly affected by the tuition waiver.

These modest effect sizes demand careful interpretation in light of our research design. The fundamental challenge of our analysis is that we cannot directly identify former foster youth in the ACS PUMS data. Instead, we estimate an intent-to-treat effect on the entire Pennsylvania young adult population, among whom foster youth comprise less than one percent. Simple back-of-the-envelope calculations suggest that if the policy had a 10 percentage point effect on the roughly 15,000 affected foster youth, this would translate to only a 0.1-0.2 percentage point effect on the broader population of approximately one million young adults---closely matching our estimates. The dilution inherent in this approach severely limits our statistical power to detect policy effects, making our estimates a lower bound on the program's potential impact on its intended beneficiaries.

This paper contributes to several literatures in economics and education policy. First, we add to the growing body of evidence on educational interventions for foster youth. Prior studies have examined the effects of programs like Educational and Training Vouchers (ETV), campus support programs, and housing assistance on foster youth outcomes \citep{courtney2007early, day2011foster, okpych2017findings}. Our study is among the first to evaluate a comprehensive state tuition waiver using quasi-experimental methods with a clear counterfactual. Second, we contribute to the broader literature on ``promise programs'' and place-based scholarships, which have shown mixed effects on college enrollment and completion \citep{bartik2020merits, page2019place}. Unlike most promise programs that target geographic areas or income groups, FosterED targets a specific vulnerable population, raising questions about whether findings from other contexts generalize to this unique setting.

Third, our paper illustrates the methodological challenges of evaluating targeted policies using population-level survey data. While difference-in-differences designs provide a credible framework for causal inference when parallel trends assumptions hold, they require sufficient statistical power to detect effects. When the treated population is a small minority, even substantial program effects may be undetectable in broad surveys. Our analysis underscores the importance of linked administrative data---combining child welfare records with education data---for credible evaluation of foster youth interventions. We hope this paper motivates future research using such data to provide more definitive answers about the effectiveness of foster youth tuition waivers.

The remainder of this paper proceeds as follows. Section 2 provides background on foster youth educational outcomes and state tuition waiver policies, situating our study within the broader literature on educational disadvantage and policy responses. Section 3 describes Pennsylvania's FosterED program in detail, including its eligibility requirements, benefits, and administrative structure. Section 4 presents our data and empirical methodology, discussing the strengths and limitations of our approach. Section 5 reports the main results and heterogeneity analyses, including robustness checks to assess the sensitivity of our findings. Section 6 discusses the limitations of our approach and implications for future research and policy. Section 7 concludes with a summary of our findings and recommendations.

%===================================================================
\section{Background}
%===================================================================

\subsection{Educational Outcomes for Foster Youth}

The educational challenges facing foster youth begin long before the transition to adulthood and compound throughout the life course. Children in foster care experience high rates of school mobility as they move between placements, often changing schools multiple times within a single academic year \citep{conger2010foster, wulczyn2003analysis}. This mobility disrupts peer relationships, interrupts instruction, and makes it difficult for teachers to identify and address learning needs. Foster children are also disproportionately placed in special education programs, sometimes appropriately but sometimes as a result of behavioral issues stemming from trauma rather than genuine learning disabilities. Disciplinary actions including suspensions and expulsions occur at elevated rates, further disrupting educational trajectories. By high school, foster youth are significantly less likely than their peers to be on track for graduation and substantially more likely to drop out before completing their diploma \citep{barrat2012predictors}.

These early disadvantages compound dramatically at the transition to adulthood. The Midwest Study, a landmark longitudinal study following foster youth from age 17 to 26, documented profound disparities in educational attainment that persist well into the mid-twenties \citep{courtney2011midwest}. By age 26, only about 8\% of former foster youth had obtained a four-year degree, compared to 46\% of their non-foster peers---a gap of nearly 40 percentage points that reflects years of accumulated disadvantage. Even among the subset of foster youth who manage to enroll in college, completion rates remain dismayingly low. One rigorous study found that only 3\% of foster youth who started at a four-year institution completed a bachelor's degree within six years, compared to completion rates exceeding 50\% for the general student population \citep{day2013educational}. These statistics paint a picture of a population facing extraordinary barriers to educational success.

Several interconnected mechanisms explain these poor outcomes, each representing a potential target for policy intervention. Financial barriers are substantial and well-documented in the literature. Foster youth are often classified as independent students for financial aid purposes upon aging out of care, which can paradoxically increase certain forms of aid eligibility while creating new complications. Many foster youth lack the knowledge, support, and documentation needed to navigate complex financial aid applications, particularly when they cannot provide parental financial information or obtain required signatures \citep{salazar2016barriers}. Even when aid is secured, living expenses often exceed available support, forcing students to work long hours that compete with academic commitments. The stress of financial precarity can undermine academic focus and persistence, contributing to the high rates of college dropout observed in this population.

Housing instability represents another critical barrier that distinguishes foster youth from their peers. Unlike most college students, foster youth frequently lack a stable ``home base'' to return to during academic breaks, leaving them vulnerable to homelessness during winter and summer recessions when dormitories close \citep{dworsky2012homelessness}. Some campuses have developed bridge housing programs to address this need, but such programs remain relatively rare. Even during the academic year, foster youth may struggle to secure stable housing, particularly at commuter campuses or institutions without guaranteed housing. The constant uncertainty about where one will sleep undermines the stability needed for academic success and can lead to food insecurity, mental health challenges, and dropout.

Social capital deficits constitute a third category of disadvantage that may be less visible but equally consequential. Foster youth often lack parents or permanent guardians who can provide guidance on the college application process, help select courses and majors, offer encouragement during difficult semesters, or assist with job searches after graduation \citep{salazar2013family}. This absence of informal mentoring and support means that foster youth must navigate complex institutional systems largely on their own, often making suboptimal decisions due to lack of information. The emotional toll of feeling alone and unsupported can compound academic challenges, contributing to mental health issues that further impede success. Recognizing these interconnected barriers, policymakers have sought interventions that address multiple dimensions of disadvantage simultaneously.

\subsection{State Tuition Waiver Programs}

States have responded to the educational challenges facing foster youth through a variety of policy interventions, with tuition waiver programs representing one of the most direct approaches to addressing financial barriers. The rationale for tuition waivers is straightforward from an economic perspective: if financial constraints represent a primary barrier to college enrollment and completion, then eliminating the direct cost of tuition should increase enrollment by making the expected net returns to education more favorable. Tuition waivers also carry symbolic significance, signaling that the state recognizes its responsibility to support youth who were in its care and providing a concrete benefit that may motivate foster youth to consider college as a realistic option.

The first state tuition waiver specifically targeting foster youth was enacted in Michigan in 1988, establishing a template that many other states would eventually follow \citep{dworsky2013policy}. The pace of adoption accelerated in the 2000s and 2010s, as advocacy groups raised awareness of foster youth educational challenges and states sought to demonstrate commitment to vulnerable populations. As of 2023, 33 states plus the District of Columbia offer some form of tuition waiver or scholarship specifically for foster youth, though the programs vary substantially in their design and generosity \citep{ncsl2023fostertuition}. This variation provides natural opportunities for research comparing program effectiveness across different policy designs, though such comparative research remains limited.

The eligibility criteria for state tuition waivers typically require that the youth was in foster care at some age threshold, most commonly age 13, 14, or 16, though some programs use different cutoffs. This threshold approach reflects the policy judgment that youth who remained in foster care through adolescence without achieving permanency face the greatest barriers to success and deserve the most support. Some programs require that youth ``aged out'' of care by remaining in the system until reaching the age of majority, while others extend eligibility to youth who achieved permanency through adoption or guardianship after the threshold age. The rationale for including adopted youth is that many aging-out experiences (educational disruption, trauma history, institutional care) also apply to youth who were adopted at older ages, and the adoptive families may face financial constraints similar to those of former foster youth.

The generosity of benefits varies substantially across state programs, from full tuition and fee waivers covering the entire cost of attendance at any state institution to more modest partial scholarships with dollar caps that may leave students with substantial out-of-pocket costs. Some programs limit coverage to public institutions, arguing that state funds should support state schools, while others extend benefits to private colleges on the theory that attending the best-fit institution maximizes student success regardless of sector. The duration of benefits also varies, with most programs capping support at four to five years of college or until age 25-26, though some are more restrictive. These design choices involve tradeoffs between program cost, breadth of coverage, and alignment with typical educational trajectories.

Finally, programs differ in how they coordinate with federal financial aid, particularly the Pell Grant and Educational and Training Vouchers provided under the Chafee Foster Care Independence Program. Some state waivers ``stack'' on top of federal aid, providing additional support that can cover living expenses beyond tuition. Others operate as ``last-dollar'' programs that fill the gap between other aid and tuition costs, potentially freeing up federal aid for non-tuition expenses but providing less total support than stacking programs. The interaction between state and federal programs is complex and may affect both the net benefit to students and the overall cost to the state, making program design a crucial but understudied aspect of foster youth educational policy.

Evidence on the effectiveness of these programs remains limited despite their proliferation across states. A few studies have examined specific state programs using descriptive or quasi-experimental methods with varying degrees of rigor. Research on California's Chafee ETV program found that receiving funds was associated with increased community college enrollment but not with degree completion, suggesting that financial support alone may be insufficient to close attainment gaps \citep{gross2019effect}. Another study documented that foster youth receiving ETV funds had higher enrollment rates than non-recipients, though selection bias complicates causal interpretation since more motivated youth may be more likely both to apply for funds and to enroll in college \citep{courtney2007early}. More rigorous evaluations using linked administrative data and quasi-experimental methods are rare, in part because the small foster youth population makes detection of effects challenging even in large datasets and because linking child welfare and education records requires overcoming significant data access barriers.

\subsection{Theoretical Framework}

Economic theory provides predictions about how tuition waivers should affect foster youth educational decisions, though the magnitude and even the direction of effects depends on assumptions about which barriers are most binding for this population. The human capital investment model, which treats educational decisions as investments that individuals make based on expected costs and returns, suggests that reducing the price of college through tuition waivers should increase enrollment by making the expected net returns more favorable. From this perspective, eliminating tuition costs shifts the cost-benefit calculation in favor of enrollment for youth who were previously deterred by financial constraints, leading to increases in college-going at the margin.

The magnitude of this enrollment effect depends on several factors that may differ between foster youth and the general population. First, foster youth may be more price-sensitive than typical students given their extreme financial constraints and the absence of family resources to buffer price changes. If foster youth are highly price-sensitive, then eliminating tuition could produce substantial enrollment increases. On the other hand, foster youth face such severe barriers across multiple dimensions that the marginal effect of tuition reduction may be small if other constraints (housing, academic preparation, social support) remain binding. Second, the opportunity costs of college attendance may be particularly high for foster youth who need to work to support themselves, have family responsibilities, or face housing instability that makes full-time enrollment impractical. If opportunity costs are the binding constraint, tuition waivers may have limited effects.

Third, information and awareness play a crucial role that standard economic models often underemphasize. If foster youth are unaware of available tuition waiver programs, or if application processes are burdensome and poorly publicized, actual take-up may be low even when programs are generous on paper. Research on other financial aid programs has documented substantial gaps between eligibility and take-up, particularly among first-generation and low-income students who lack access to information about aid programs. Foster youth, who often lack family guidance and may have had negative experiences with institutional systems, may be particularly likely to leave benefits unclaimed. This suggests that the effectiveness of tuition waivers may depend critically on outreach and application support, policy features that vary across programs.

Fourth, non-financial barriers including inadequate academic preparation, housing instability, mental health challenges, and social isolation may be more important than financial barriers for many foster youth. If these non-financial factors are the primary constraints on college success, then tuition waivers may shift the enrollment margin but have limited effects on completion, potentially increasing the number of foster youth who start college but drop out before earning credentials. This would represent a concerning outcome if incomplete college results in debt without degree-related earnings benefits, though most foster youth tuition waivers do not generate student debt directly.

The theoretical prediction for the effect of tuition waivers on foster youth enrollment and completion is thus ambiguous: tuition waivers should increase enrollment on the margin, but the magnitude depends on the relative importance of financial versus non-financial barriers, the extent of information and awareness about programs, and the opportunity costs facing prospective students. Empirical evidence is needed to determine which of these theoretical considerations dominates in practice. Our analysis contributes to this evidence base by examining the effects of Pennsylvania's comprehensive tuition waiver program, though data limitations prevent us from distinguishing between these competing mechanisms.

%===================================================================
\section{Pennsylvania's FosterED Program}
%===================================================================

\subsection{Policy Details and Implementation}

Pennsylvania's Fostering Independence Tuition Waiver was established through Act 16 of 2019, with implementation beginning for the 2019-2020 academic year. The program emerged from advocacy efforts by child welfare organizations, higher education institutions, and legislators who recognized that Pennsylvania lagged behind many other states in providing dedicated support for foster youth pursuing higher education. The policy was championed by the Pennsylvania Department of Human Services in partnership with the Pennsylvania Higher Education Assistance Agency (PHEAA), which administers the program alongside the federal Chafee Education and Training Grant. This coordination was designed to maximize the total support available to foster youth while minimizing administrative burden through a unified application process.

The eligibility requirements for FosterED reflect considered judgments about which youth should qualify for state-supported tuition assistance. The applicant must have been in foster care in Pennsylvania at age 16 or older when achieving permanency through adoption, guardianship, or reunification, or when aging out of care upon reaching the age of majority. This age-16 threshold is among the more generous in the nation; some states require youth to have been in care until age 18 or to have formally aged out, which excludes youth who achieved permanency at older ages. Youth who achieved permanency before age 16 are ineligible for FosterED, even if they spent many years in foster care during childhood, on the rationale that earlier permanency indicates a more stable support system. Applicants must apply before age 21 and can receive benefits until age 26, providing a substantial window for youth whose educational paths may be delayed or interrupted.

The benefits provided under FosterED are substantial compared to programs in many other states. The policy provides a full waiver of tuition and mandatory fees at Pennsylvania postsecondary institutions, meaning that eligible students pay nothing out of pocket for the direct cost of instruction and required fees. Benefits extend for up to five years of college enrollment, and these years need not be consecutive, accommodating youth who may need to take breaks for work, family responsibilities, or other circumstances. The five-year window also recognizes that foster youth may take longer to complete degrees due to academic preparation gaps or the need to work while enrolled. The program operates as a ``last-dollar'' award, meaning it covers tuition costs remaining after other grants and scholarships (including Pell Grants and state grants) are applied, potentially freeing up those funds for other educational expenses.

The range of eligible institutions is notably broad, including all 14 Pennsylvania State System of Higher Education (PASSHE) universities, all 15 community colleges in the Pennsylvania Community College system, Penn State University and its branch campuses, and all private institutions approved for Pennsylvania state grants. This comprehensive coverage means that foster youth can attend virtually any institution in the state without facing tuition costs, allowing them to choose schools based on fit, program availability, and location rather than price. The inclusion of private institutions, while more costly for the state, reflects the judgment that foster youth should have the same range of educational options available to other students and that some private institutions may provide better environments or programs for particular students.

\subsection{Program Context and Coordination}

Pennsylvania's FosterED program operates alongside the federal Chafee Education and Training Grant (ETG), which provides up to \$5,000 per year for eligible foster youth attending postsecondary institutions anywhere in the country. Unlike FosterED, Chafee ETG is not a tuition waiver but rather a grant that can be used for any educational expenses, including living costs, books, transportation, and personal expenses. The two programs are designed to work together synergistically: FosterED covers tuition and mandatory fees, while Chafee ETG can cover the remaining expenses that often prove challenging for foster youth without family support. This coordination aims to create a comprehensive support package that addresses multiple dimensions of educational cost.

Prior to FosterED's enactment, Pennsylvania foster youth could access Chafee ETG funds, state and federal need-based grants (including Pell Grants and Pennsylvania State Grants), and institutional aid, but there was no dedicated tuition waiver ensuring that tuition costs would be fully covered. The introduction of FosterED thus represented a significant expansion of support that filled a gap in the existing financial aid landscape. For many foster youth, particularly those attending higher-cost institutions or those whose other aid did not fully cover tuition, FosterED provides the marginal support needed to make college financially feasible.

According to Pennsylvania Department of Human Services data, approximately 1,200-1,500 youth age out of foster care in Pennsylvania each year, with an additional number achieving permanency through adoption or guardianship at age 16 or older \citep{padhs2022}. The potential eligible population for FosterED is thus modest in absolute terms---perhaps 15,000-20,000 youth aged 18-26 at any given time---though substantial relative to the overall foster youth population in the state. This small absolute number presents challenges for program evaluation, as we discuss in subsequent sections, but also means that the fiscal cost of the program is manageable even with generous benefits.

\subsection{Theory of Change and Expected Effects}

The FosterED program's theory of change rests on several assumptions about how financial support translates into improved educational outcomes. The first and most direct assumption is that tuition costs represent a meaningful barrier to college enrollment for foster youth, and that eliminating this barrier through a tuition waiver should increase enrollment among youth who were previously deterred by cost concerns. This assumption seems plausible given the extreme financial constraints facing most foster youth, but the magnitude of the enrollment effect depends on how many foster youth were deterred specifically by tuition costs rather than by other barriers.

The second assumption is that awareness of the program will be sufficient to reach eligible youth, and that application processes will not deter those who would benefit. This assumption requires active outreach through child welfare agencies, high schools, and postsecondary institutions, as foster youth may not learn about the program through the family channels that inform other students about financial aid opportunities. Pennsylvania's coordination of FosterED with Chafee ETG applications was designed to address this concern by reducing the number of separate applications youth must complete, though information barriers may still limit take-up.

The third assumption is that by reducing financial stress during college, FosterED will improve persistence and completion among foster youth who enroll. The reasoning is that financial precarity undermines academic focus and forces difficult tradeoffs between work and study that can lead to dropout. By guaranteeing tuition coverage for up to five years, FosterED removes this source of uncertainty and allows foster youth to focus on their studies. The five-year benefit window also provides flexibility for youth who may need to take breaks, reducing the pressure to persist through challenging circumstances when a leave of absence might be the better choice.

If these assumptions hold, we would expect to see increases in college enrollment rates among eligible foster youth following FosterED's implementation, along with improvements in persistence and eventually degree completion. However, our empirical analysis is limited to examining enrollment effects, as completion effects would require longer time horizons than our data allow. We also cannot isolate effects on foster youth specifically and instead estimate intent-to-treat effects on the broader young adult population, which substantially limits our ability to detect program impacts as we discuss in the methods section.

%===================================================================
\section{Data and Methodology}
%===================================================================

\subsection{Data Sources and Sample Construction}

Our primary data source is the American Community Survey Public Use Microdata Sample (ACS PUMS) for the years 2017, 2018, 2021, and 2022. The ACS is an annual survey conducted by the U.S. Census Bureau that collects detailed demographic, social, and economic information from approximately 3.5 million households each year, making it the largest household survey in the United States outside of the decennial census. The PUMS provides individual-level microdata with sufficient geographic detail for state-level analysis while protecting respondent confidentiality through various statistical disclosure limitation techniques. The large sample sizes available through the ACS PUMS are essential for our analysis given the small effects we expect to detect.

We construct our analysis sample to focus on the population most directly affected by foster youth tuition waivers while maintaining sufficient sample sizes for statistical inference. We include young adults aged 18-26, corresponding to the eligible age range for FosterED benefits and the period when most college enrollment and degree completion occurs. We focus on Pennsylvania as the treatment state and select neighboring states New York, New Jersey, Ohio, and West Virginia as controls, reasoning that geographic proximity implies similarity in economic conditions, labor markets, and educational systems that would affect college-going trends. We include data from 2017-2018 as the pre-implementation period and 2021-2022 as the post-implementation period, excluding 2019 as the first year of implementation when effects might be incomplete and 2020 because the 1-year ACS was not released due to COVID-19 data collection disruptions.

The resulting sample contains 241,226 person-year observations distributed across our treatment and control groups and time periods. Approximately 55,000 observations come from Pennsylvania and 185,000 from control states, with the distribution roughly even between pre and post periods. Sample sizes of this magnitude provide substantial statistical power for detecting aggregate effects, though as we discuss below, the dilution inherent in our intent-to-treat design means that even large samples may be insufficient to detect effects on the small foster youth subpopulation.

\subsection{Key Variables and Measurement}

We construct three primary outcome variables measuring different dimensions of educational attainment that might be affected by tuition waivers. Current college enrollment is derived from the school enrollment (SCH) and educational attainment (SCHL) variables in the ACS PUMS. Specifically, an individual is coded as currently enrolled in college if: (1) SCH = 1 (currently enrolled in school), AND (2) SCHL $\geq$ 18 (educational attainment of ``some college'' or higher, indicating the individual is enrolled at the college level rather than K-12). We note that the ideal variable for college enrollment would be SCHG (school grade level), which explicitly identifies enrollment in ``college undergraduate'' or ``graduate school''; however, SCHG is not available in all ACS years and has higher non-response rates in PUMS extracts. Our combined SCH/SCHL measure approximates college enrollment but may introduce some measurement error. This measure captures whether respondents are actively pursuing higher education at the time of the survey, the most direct outcome we would expect tuition waivers to affect in the short run.

Our second outcome, any college attainment, indicates whether the respondent has ever attended college and earned at least some credits, regardless of current enrollment status. This measure is coded based on SCHL values indicating some college, associate's degree, bachelor's degree, or higher. Any college attainment captures the extensive margin of higher education participation, including both current students and those who have previously attended college. This measure may be less sensitive to FosterED's effects than current enrollment because it includes educational experiences predating the policy.

Our third outcome, bachelor's degree attainment, indicates whether the respondent has completed a four-year degree, coded based on SCHL values indicating bachelor's degree or higher. This is the ultimate outcome of interest for higher education policy, as bachelor's degree completion is associated with substantial labor market returns. However, bachelor's degree effects are unlikely to appear within our short post-implementation window, as students affected by FosterED starting in 2019 would not have had time to complete degrees by 2021-2022. We include this outcome primarily as a benchmark and expect to find null or very small effects.

We include several control variables to improve the precision of our estimates and address potential confounding. Age is included as a continuous variable to control for the mechanical relationship between age and educational attainment. Sex is included as a binary indicator for female to control for gender differences in college-going. Race and ethnicity are measured using the RAC1P and HISP variables and included as categorical controls to account for racial disparities in educational attainment. Poverty status, measured as income relative to the federal poverty level (POVPIP), is included to control for socioeconomic differences that might confound state comparisons. All analyses use person weights (PWGTP) provided in the ACS PUMS to produce population-representative estimates.

\subsection{Empirical Strategy and Identification}

We employ a difference-in-differences (DiD) research design to estimate the causal effect of Pennsylvania's FosterED program on educational outcomes among young adults. The DiD approach exploits the fact that FosterED was implemented in Pennsylvania at a specific point in time (2019), creating variation across states (Pennsylvania versus controls) and time (before versus after implementation) that can be used to identify the policy effect under appropriate assumptions. By comparing changes in outcomes over time in Pennsylvania relative to changes in control states, DiD controls for time-invariant state characteristics and common national trends, isolating the effect attributable to Pennsylvania's policy change.

Our primary specification estimates the following regression equation for individual $i$ in state $s$ at time $t$:
\begin{equation}
Y_{ist} = \alpha + \beta_1 (PA_s \times Post_t) + \gamma_s + \delta_t + X_{ist}'\theta + \varepsilon_{ist}
\end{equation}
where $Y_{ist}$ is the educational outcome of interest, $PA_s$ is an indicator for Pennsylvania residence, $Post_t$ is an indicator for the post-implementation period (2021-2022), $\gamma_s$ represents state fixed effects that absorb time-invariant state characteristics, $\delta_t$ represents year fixed effects that absorb common time trends, $X_{ist}$ is a vector of individual covariates, and $\varepsilon_{ist}$ is the error term. The coefficient of interest is $\beta_1$, which represents the difference-in-differences estimate of the FosterED effect---the change in outcomes in Pennsylvania relative to control states after policy implementation.

The key identifying assumption underlying the DiD design is parallel trends: absent FosterED, educational outcomes in Pennsylvania would have evolved similarly to outcomes in control states. This assumption is inherently untestable because we cannot observe what would have happened in Pennsylvania without the policy, but we can assess its plausibility by examining whether pre-treatment trends were parallel. If Pennsylvania and control states were on similar trajectories before 2019, it is more plausible that they would have continued on similar trajectories absent the policy intervention. We implement this assessment through an event study specification that allows the Pennsylvania-control difference to vary by year, enabling visual inspection of pre-trends and the timing of any effects.

\subsection{Limitations of the Identification Strategy}

Our identification strategy faces several important limitations that affect the interpretation of our estimates. The most fundamental limitation is that we cannot directly identify foster youth in the ACS PUMS data. The survey does not include questions about foster care history, and while some household relationship codes can identify current foster children, this does not capture the former foster youth who are the target population for FosterED. As a result, we estimate an intent-to-treat (ITT) effect on the entire Pennsylvania young adult population, the vast majority of whom are not foster youth and are thus unaffected by the tuition waiver.

This ITT approach leads to severe dilution of the treatment effect. Foster youth comprise less than one percent of the young adult population, so even if FosterED had large effects on the roughly 15,000 eligible foster youth in Pennsylvania, these effects would be averaged across approximately one million young adults, most of whom experience no change. A back-of-the-envelope calculation illustrates the severity of this dilution: if FosterED increased college enrollment among foster youth by 10 percentage points (a substantial effect by any standard), this would translate to only a 0.1-0.15 percentage point increase in the overall young adult enrollment rate. Detecting effects of this magnitude requires extremely large samples and precise measurement, conditions that may not be met even in the large ACS PUMS.

The COVID-19 pandemic creates additional challenges for our analysis by disrupting higher education patterns during our post-implementation period. The 2020-21 academic year saw substantial shifts toward online instruction, declines in enrollment at many institutions, and changes in educational decision-making that varied across states based on pandemic severity and policy responses. While we include year fixed effects to absorb common shocks and exclude the 2020 ACS entirely, differential state responses to COVID could bias our estimates if Pennsylvania's pandemic trajectory differed systematically from control states in ways that affected educational outcomes. This concern is heightened by Pennsylvania's relatively aggressive early pandemic response, which may have affected college-going decisions differently than in some control states.

With only two pre-treatment years in our data, our ability to assess parallel trends is limited. A longer pre-period would provide more confidence in the identifying assumption by allowing us to verify that Pennsylvania and control states were tracking together over an extended period before the policy change. With just 2017 and 2018, we can assess whether the immediate pre-treatment trends were similar, but we cannot rule out that Pennsylvania was diverging from control states in ways that would have continued absent the policy. This limitation is common in difference-in-differences studies of recent policy changes but should temper the confidence with which we interpret our estimates.

%===================================================================
\section{Results}
%===================================================================

\subsection{Descriptive Statistics}

Table \ref{tab:summary} presents summary statistics for our analysis sample, comparing Pennsylvania and control states across the pre-implementation (2017-2018) and post-implementation (2021-2022) periods. These statistics provide context for interpreting our regression estimates and allow assessment of baseline comparability between treatment and control groups.

\begin{table}[H]
\centering
\caption{Summary Statistics}
\label{tab:summary}
\begin{tabular}{lcccc}
\toprule
& \multicolumn{2}{c}{Pennsylvania} & \multicolumn{2}{c}{Control States} \\
\cmidrule(lr){2-3} \cmidrule(lr){4-5}
& Pre & Post & Pre & Post \\
\midrule
N (unweighted) & 28,617 & 27,161 & 93,121 & 92,327 \\
N (weighted, millions) & 2.99 & 2.87 & 9.93 & 9.85 \\
\\
\textit{Outcomes (\%):} \\
College enrollment & 25.88 & 26.13 & 28.93 & 28.98 \\
Any college & 56.47 & 55.42 & 61.31 & 59.37 \\
Bachelor's degree & 18.53 & 20.33 & 21.17 & 23.06 \\
\\
\textit{Demographics:} \\
Mean age & 22.0 & 22.0 & 22.0 & 22.0 \\
Female (\%) & 49.3 & 49.1 & 49.5 & 49.4 \\
\bottomrule
\end{tabular}
\end{table}

Several patterns emerge from the descriptive statistics that inform our subsequent analysis. Pennsylvania consistently shows lower educational attainment than the control states across all three outcomes and both time periods. Pre-treatment college enrollment is about 3 percentage points lower in Pennsylvania (25.9\% versus 28.9\%), any college attainment is about 5 percentage points lower (56.5\% versus 61.3\%), and bachelor's degree attainment is about 2.6 percentage points lower (18.5\% versus 21.2\%). These baseline differences are consistent with broader patterns showing Pennsylvania ranking in the middle of states for educational attainment, somewhat below the Northeastern average. The differences highlight the importance of our difference-in-differences design, which controls for these persistent state differences rather than comparing raw levels.

Examining changes over time, we observe modest movements in all outcomes in both Pennsylvania and control states. Pennsylvania's college enrollment increased slightly from 25.88\% to 26.13\% (a change of +0.25 percentage points), while control state enrollment increased from 28.93\% to 28.98\% (+0.05 percentage points). For any college attainment, both groups declined, with Pennsylvania falling from 56.47\% to 55.42\% (-1.05 percentage points) and controls falling from 61.31\% to 59.37\% (-1.94 percentage points). Bachelor's degree attainment increased in both groups, reflecting normal cohort progression as young adults complete degrees over time. Pennsylvania's bachelor's rate rose from 18.53\% to 20.33\% (+1.80 percentage points), while controls rose from 21.17\% to 23.06\% (+1.89 percentage points).

The demographic composition of our sample is similar across treatment and control groups and stable over time, supporting the validity of comparisons. Mean age is 22.0 years in all cells, consistent with our sample restriction to ages 18-26. The proportion female is approximately 49\% throughout, close to the population gender ratio. These similarities suggest that compositional changes are unlikely to confound our estimates, though we include demographic controls in our regression specifications as an additional precaution.

\subsection{Main Difference-in-Differences Results}

Table \ref{tab:did} presents our main difference-in-differences estimates for the three educational outcomes, providing the central evidence on whether FosterED affected educational attainment in Pennsylvania.

\begin{table}[H]
\centering
\caption{Difference-in-Differences Estimates}
\label{tab:did}
\begin{tabular}{lccc}
\toprule
& College Enrollment & Any College & Bachelor's Degree \\
\midrule
Pennsylvania $\times$ Post & +0.196 pp & +0.886 pp & -0.095 pp \\
& (12.45) & --- & --- \\
\\
Bootstrap 95\% CI & [-26.9, 27.6] & --- & --- \\
t-statistic & 0.02 & --- & --- \\
p-value & $>$.10 & $>$.10 & $>$.10 \\
\\
PA Pre-Mean & 25.88\% & 56.47\% & 18.53\% \\
Control Pre-Mean & 28.93\% & 61.31\% & 21.17\% \\
\\
N & 241,226 & 241,226 & 241,226 \\
Clusters (state-years) & 20 & 20 & 20 \\
\bottomrule
\multicolumn{4}{p{12cm}}{\footnotesize Notes: Bootstrap standard errors (500 iterations) in parentheses, clustered at state-year level. CIs computed using percentile method. With only 5 state clusters (1 treated, 4 controls), conventional clustered inference is unreliable; we report bootstrap SEs but note these should be interpreted with caution following \citet{CameronMiller2015} and \citet{ConleyTaber2011}. pp = percentage points.}
\end{tabular}
\end{table}

The results show small positive point estimates for FosterED on college enrollment and any college attainment, with essentially no effect on bachelor's degree completion. Critically, however, these estimates are \textbf{not statistically distinguishable from zero}. For college enrollment, the DiD estimate is +0.196 percentage points with a bootstrap standard error of 12.45 percentage points (t = 0.02, p $>$ .10). The 95\% confidence interval spans from -26.9 to +27.6 percentage points, encompassing both large positive and large negative effects. This substantial uncertainty reflects the fundamental challenge of conducting inference with only five state clusters (one treated, four controls); as \citet{CameronMiller2015} emphasize, conventional clustered inference becomes unreliable with few clusters. We report these estimates as descriptive rather than causal evidence. The point estimate is constructed from the underlying cell means: Pennsylvania increased by 0.25 percentage points while controls increased by only 0.05 percentage points, yielding a difference-in-differences of approximately 0.20 percentage points.

For any college attainment, the DiD estimate is +0.886 percentage points. Although both Pennsylvania and control states experienced declines in any college attainment over this period (possibly reflecting COVID-19 disruptions), Pennsylvania's decline was smaller: -1.05 percentage points compared to -1.94 percentage points for controls. This relative improvement yields a positive DiD estimate, suggesting that FosterED may have modestly slowed the decline in college participation that affected the region generally.

For bachelor's degree attainment, the DiD estimate is essentially zero at -0.095 percentage points. Both Pennsylvania and control states experienced similar increases in bachelor's degree rates over this period (+1.80 versus +1.89 percentage points), reflecting normal degree completion by young adults over time. The lack of a DiD effect on bachelor's completion is expected given the short post-implementation window; students affected by FosterED starting in 2019 would not have had sufficient time to complete four-year degrees by 2021-2022.

These estimates are small in absolute terms but potentially consistent with meaningful effects on the target foster youth population when accounting for dilution. Recall that foster youth comprise less than one percent of young adults, so if FosterED increased enrollment among foster youth by 10 percentage points, the aggregate effect would be approximately 0.10-0.15 percentage points---similar to our enrollment estimate of 0.20 percentage points. This back-of-the-envelope calculation suggests our estimates are at least plausibly consistent with substantial effects on foster youth, though we cannot precisely identify the foster youth subgroup to confirm this interpretation.

\subsection{Event Study Analysis}

Figure \ref{fig:event} presents an event study analysis showing the difference between Pennsylvania and control states in college enrollment rates for each year in our sample. This analysis serves two purposes: first, to assess the parallel trends assumption by examining whether Pennsylvania and controls were on similar trajectories before FosterED implementation; second, to examine the timing of any effects relative to the policy change.

\begin{figure}[H]
\centering
\includegraphics[width=0.9\textwidth]{figures/event_study.png}
\caption{Event Study: College Enrollment Gap (Pennsylvania vs. Control States). Blue bars indicate pre-implementation years; red bars indicate post-implementation years. Vertical dashed line marks policy implementation in 2019.}
\label{fig:event}
\end{figure}

The event study reveals that Pennsylvania consistently had lower college enrollment rates than control states throughout the study period, with the gap fluctuating between roughly -1.9 and -3.7 percentage points. In the pre-period, the gap was approximately -2.9 percentage points in 2017 and -3.2 percentage points in 2018, suggesting the gaps were relatively stable before the policy. In the post-period, the gap narrowed to -1.9 percentage points in 2021 but widened again to -3.7 percentage points in 2022.

The pattern does not show a clear, sustained break at the policy implementation date that would provide strong visual evidence of a treatment effect. The narrowing of the gap in 2021 is consistent with a positive policy effect, as Pennsylvania's relative position improved in the first full post-COVID year of the program. However, the widening in 2022 complicates this interpretation, suggesting that any positive effect may have been temporary or that year-to-year fluctuations are dominated by noise rather than systematic policy effects. The pre-trends show reasonably stable gaps in 2017-2018, providing some support for the parallel trends assumption, though the 0.3 percentage point widening from 2017 to 2018 suggests that perfect parallel trends may not hold.

\subsection{Heterogeneity Analysis by Age}

Table \ref{tab:heterogeneity} and Figure \ref{fig:hetero} present DiD estimates separately for younger (ages 18-21) and older (ages 22-26) young adults. This heterogeneity analysis is motivated by FosterED's eligibility rules, which require that youth were in foster care at age 16 or older. Youth who aged out at 18 and immediately enrolled in college would be 21-24 by 2021-2022; youth who delayed enrollment or took time to complete community college before transferring might be older. Thus, if FosterED is having its intended effect on foster youth, we might expect larger effects among older young adults who had more time to be affected by the program.

\begin{table}[H]
\centering
\caption{Heterogeneity by Age Group}
\label{tab:heterogeneity}
\begin{tabular}{lcc}
\toprule
& Ages 18-21 & Ages 22-26 \\
\midrule
DiD Estimate (College Enrollment) & +0.11 pp & +0.80 pp \\
N & 106,892 & 134,334 \\
\bottomrule
\end{tabular}
\end{table}

\begin{figure}[H]
\centering
\includegraphics[width=0.7\textwidth]{figures/heterogeneity_age.png}
\caption{Heterogeneity by Age: DiD Estimates for College Enrollment. Bars show DiD estimates for younger (18-21) and older (22-26) young adults.}
\label{fig:hetero}
\end{figure}

The results show larger effects among older young adults (ages 22-26) compared to younger adults (ages 18-21). For the younger group, the DiD estimate is +0.11 percentage points, essentially zero. For the older group, the estimate is +0.80 percentage points, meaningfully larger though still small in absolute terms. This pattern is potentially consistent with the hypothesis that FosterED primarily affects older foster youth who had time to learn about the program, apply, and enroll, while younger youth might not yet have been affected by the policy or might have enrolled regardless of the tuition waiver.

However, we caution against over-interpreting this heterogeneity. The difference between the two estimates, while suggestive, could reflect sampling variation rather than true differential effects. Moreover, the mechanism linking age to foster youth status is indirect at best; we cannot identify foster youth in either age group, so we are comparing aggregate effects across age categories that include both foster and non-foster youth. The larger effect among older adults could reflect factors unrelated to foster youth status, such as differential COVID impacts by age or other policies that disproportionately affected older returning students.

\subsection{Robustness Analysis}

We conduct several robustness checks to assess the sensitivity of our findings to alternative specifications and sample definitions. These checks address concerns about specific modeling choices that could affect our estimates.

Our first robustness check uses alternative control groups. When we restrict controls to border states only (New York, Ohio, and West Virginia, excluding New Jersey), results are similar to our main specification. When we expand controls to include additional Midwest states (Michigan, Indiana, Illinois), results are again similar, suggesting that our findings are not driven by the specific choice of control states. This stability across control group definitions supports the view that our estimates reflect Pennsylvania-specific changes rather than artifacts of the comparison group.

Our second check varies the age range of the sample. When we narrow the sample to ages 18-23, who are more likely to be currently enrolled in or recently enrolled in college, we find slightly larger point estimates for enrollment effects, though the differences are not statistically significant. When we expand to ages 18-30, including older adults who would have been beyond typical college age even at the beginning of the program, effects attenuate toward zero, consistent with the expectation that older adults are less affected by tuition policies.

Our third check examines unweighted estimates. The ACS PUMS weights adjust for complex survey design and nonresponse, but weighted estimates can be sensitive to extreme weights. Unweighted estimates are qualitatively similar to weighted estimates, with the enrollment DiD estimate remaining positive and in the range of 0.1-0.3 percentage points. This similarity suggests that our findings are not driven by particular high-weight observations.

Our fourth check examines the sensitivity of results to excluding 2022. Given the fluctuation in the Pennsylvania-control gap between 2021 and 2022, we estimate effects using only 2021 as the post-period. With this specification, the enrollment DiD estimate is larger (approximately +0.4 percentage points), as 2021 was the year when Pennsylvania's relative position improved most notably. This check highlights the uncertainty introduced by year-to-year variation and suggests that our main estimate, which averages across both post-years, may understate effects if 2022 represents an anomaly.

%===================================================================
\section{Discussion}
%===================================================================

\subsection{Interpretation of Main Findings}

Our analysis finds small positive effects of Pennsylvania's FosterED program on college enrollment and attainment among the general young adult population. The DiD estimate of +0.20 percentage points for college enrollment and +0.89 percentage points for any college attainment are both positive but modest in magnitude. These estimates represent intent-to-treat effects on the entire young adult population, heavily diluted by the fact that foster youth comprise a very small share of this population. Properly interpreting these findings requires considering both what they can and cannot tell us about the program's effectiveness.

The fundamental challenge in interpreting our estimates is that we cannot isolate effects on the intended beneficiaries---former foster youth---from the general young adult population. Our estimates represent an average effect across all young adults, the vast majority of whom are unaffected by the tuition waiver. If we assume that foster youth comprise approximately 1\% of Pennsylvania young adults (roughly 15,000 out of 1.2 million), and that FosterED had zero effect on the 99\% who are not foster youth, then our aggregate estimates imply that the program increased enrollment among foster youth by approximately 20 percentage points (calculated as 0.20\% divided by 0.01). This would be an extraordinarily large effect, larger than most educational interventions documented in the literature.

A more realistic interpretation acknowledges that the true effect on foster youth is likely somewhere between zero and this upper bound, with our estimates providing a noisy signal that is consistent with meaningful but not transformative impacts. If FosterED increased enrollment among foster youth by 5-10 percentage points---a range consistent with effects found for other financial aid programs targeting disadvantaged populations---this would translate to aggregate effects of 0.05-0.10 percentage points, somewhat smaller than our point estimates but within the range of sampling variation. The uncertainty surrounding our estimates means we cannot pin down the foster youth effect with precision, but the positive signs across outcomes provide suggestive evidence that the program may be having its intended effect.

The lack of a clear discontinuity in the event study analysis tempers our confidence in a causal interpretation. If FosterED had large and immediate effects, we would expect to see a sharp improvement in Pennsylvania's position relative to control states coinciding with the 2019 implementation. Instead, we observe fluctuating gaps with Pennsylvania improving in 2021 but worsening in 2022. This pattern could reflect several possibilities. The policy may have had limited immediate effects due to low awareness or take-up in early years, with effects potentially growing over time as the program becomes better known. Alternatively, COVID-19 may have disrupted normal enrollment patterns in ways that mask underlying policy effects or create spurious patterns. Finally, the true effect may be small enough to be dominated by year-to-year noise in the data, making it difficult to discern a systematic impact.

\subsection{Comparison to Existing Literature}

Our findings can be contextualized by comparison to prior research on foster youth educational interventions and financial aid programs more broadly. Studies of foster youth support programs have generally found modest positive effects on enrollment but less consistent effects on completion, suggesting that financial support addresses one barrier but may be insufficient on its own to close attainment gaps. Research on California's Chafee ETV program found that receiving funds was associated with a 3-5 percentage point increase in community college enrollment but no significant effect on degree completion \citep{gross2019effect}. Our estimates, when scaled up to the foster youth population, are roughly consistent with this range, though the uncertainty in our estimates prevents precise comparisons.

The broader promise program literature provides additional context for interpreting our findings. Reviews of place-based scholarship programs document effect sizes typically in the range of 3-10 percentage points on college enrollment among eligible populations \citep{bartik2020merits, page2019place}. These programs differ from FosterED in targeting geographic areas or income groups rather than specific vulnerable populations, so direct comparisons are imperfect. However, the convergence of findings across different types of free tuition programs suggests that eliminating tuition costs can meaningfully affect enrollment decisions, even if the magnitudes vary by context and population.

What distinguishes FosterED from most promise programs is its focus on a small, highly disadvantaged population rather than a broader geographic or income-based target. This targeting creates both advantages and challenges for evaluation. On one hand, foster youth face such severe barriers that the marginal impact of free tuition may be larger than for less disadvantaged populations. On the other hand, the small size of the foster youth population means that even large program effects will be difficult to detect in population surveys, as our analysis illustrates. Future research using linked administrative data could address this challenge by directly examining outcomes for program participants.

\subsection{Limitations and Qualifications}

We acknowledge several important limitations of our analysis that affect the confidence with which we can draw conclusions about FosterED's effectiveness. The most fundamental limitation is our inability to identify foster youth in the ACS PUMS data. This forces us to estimate intent-to-treat effects on the general population, severely diluting any program impact and reducing our statistical power to detect effects. While our estimates are consistent with meaningful impacts on foster youth, they are also consistent with small or null effects, and we cannot distinguish between these interpretations with the data available.

The COVID-19 pandemic creates additional complications by disrupting higher education patterns during our entire post-implementation period. The 2020-21 academic year saw dramatic shifts in enrollment, instruction modality, and student decision-making that varied across institutions and states. While we control for common time trends through year fixed effects, differential state responses to the pandemic could bias our estimates if Pennsylvania's trajectory differed from control states for reasons unrelated to FosterED. The widening of the Pennsylvania-control gap in 2022 could reflect such differential pandemic recovery rather than policy effects.

Our limited pre-treatment period provides only weak support for the parallel trends assumption underlying our difference-in-differences design. With just two pre-treatment years (2017 and 2018), we can assess immediate pre-trends but cannot rule out longer-term divergence that would threaten our identification strategy. A longer pre-period or additional validation tests would strengthen confidence in our estimates, but data limitations prevent such extensions.

Control states may not be ideal counterfactuals if they differ from Pennsylvania in ways that affect educational trends independently of FosterED. While we selected geographically proximate states to maximize similarity, unobserved differences in labor markets, educational policies, or demographic composition could bias our comparisons. We note that no control state implemented a comparable comprehensive foster youth tuition waiver during our study period, but other policy differences could still affect our estimates.

Finally, program take-up is unobserved in our data, preventing us from distinguishing between low program effectiveness and low awareness or utilization. If FosterED reached only a fraction of eligible foster youth due to information barriers or application challenges, our estimates would understate the potential effect of full implementation. Outreach efforts typically take time to scale up, so early-year estimates may not reflect steady-state program impacts.

\subsection{Implications for Policy and Future Research}

Despite the limitations of our analysis, several implications emerge for policy and future research. For policymakers, our findings provide suggestive evidence that comprehensive tuition waivers may modestly improve college-going among foster youth, consistent with the theoretical prediction that reducing financial barriers should increase enrollment. However, the small magnitude of our estimates suggests that tuition waivers alone are unlikely to close the large attainment gaps between foster youth and the general population. Comprehensive support addressing housing, academic preparation, social integration, and mental health is likely necessary for substantial improvements in outcomes.

Program implementation matters critically for effectiveness. Our analysis cannot assess whether FosterED's modest detected effects reflect fundamental limitations of tuition waivers or implementation challenges such as low awareness and burdensome application processes. States implementing similar programs should invest in robust outreach through child welfare agencies, high schools, and postsecondary institutions to ensure that eligible youth learn about and apply for benefits. Streamlining applications and coordinating with existing financial aid processes, as Pennsylvania does with Chafee ETG, can reduce barriers to take-up.

For researchers, our analysis underscores the need for linked administrative data to credibly evaluate foster youth interventions. Combining child welfare records (which identify foster youth and their care histories) with postsecondary education data (enrollment, persistence, completion) would allow researchers to directly estimate treatment effects on the intended beneficiaries rather than diluted intent-to-treat effects on the general population. Such data linkages require overcoming significant privacy and administrative barriers but would dramatically improve our ability to assess program effectiveness and guide policy design.

Future research should also examine longer-term outcomes as post-implementation data accumulate. Our analysis is limited to enrollment effects in the first two full years after implementation, too short a window to observe completion effects. Follow-up studies examining whether early enrollment gains translate into degree completion would provide crucial evidence on whether tuition waivers address barriers to success or merely shift dropout from before college to during college. Additionally, research comparing program designs across states could identify which features---eligibility thresholds, benefit generosity, coordination with other programs---are most effective at improving outcomes.

%===================================================================
\section{Conclusion}
%===================================================================

This paper evaluates Pennsylvania's Fostering Independence Tuition Waiver (FosterED), which provides full tuition waivers to former foster youth at state colleges and universities. Using a difference-in-differences design comparing Pennsylvania to neighboring states before and after the 2019 implementation, we find small positive effects on college enrollment (+0.20 percentage points) and any college attainment (+0.89 percentage points) among the general young adult population. For bachelor's degree attainment, we find essentially no effect, as expected given the short post-implementation window.

These modest estimates reflect the fundamental challenge of evaluating targeted programs using population-level survey data. Foster youth comprise less than one percent of young adults, so even substantial program effects are diluted beyond detection when averaged across the general population. Back-of-the-envelope calculations suggest our estimates are consistent with meaningful effects on the foster youth population---potentially 5-20 percentage point increases in enrollment---but we cannot make precise claims about the program's impact on its intended beneficiaries without directly observing foster youth status in our data.

The event study analysis does not reveal a sharp break in Pennsylvania's trajectory relative to control states at the policy implementation date. Pennsylvania's relative position improved in 2021 but worsened in 2022, a pattern that could reflect noise, differential COVID-19 impacts, or time-varying program effects. The parallel trends assumption underlying our identification strategy receives some support from stable pre-treatment gaps, but limited pre-period data prevent definitive assessment. Heterogeneity analysis shows larger effects among older young adults (ages 22-26), potentially consistent with the program's targeting of youth who aged out of care, though this pattern could also reflect other factors.

Our findings underscore the importance of linked administrative data for evaluating foster youth interventions. Population surveys like the ACS PUMS, while valuable for many research questions, lack the ability to identify foster youth and thus cannot precisely estimate program effects on this small subpopulation. Linking child welfare records to education data would allow researchers to directly examine outcomes for eligible youth, providing more definitive evidence on program effectiveness and informing the design of future interventions.

Foster youth face extraordinary challenges as they navigate the transition to adulthood without family support networks. Tuition waiver programs like Pennsylvania's FosterED represent meaningful efforts to level the playing field by removing direct financial barriers to higher education. Our analysis provides suggestive evidence that such programs may help increase college enrollment, though the magnitude of effects remains uncertain. Closing the large educational attainment gap between foster youth and their peers will likely require comprehensive support addressing the multiple barriers---financial, housing, academic, and social---that this vulnerable population faces. We hope this paper contributes to ongoing efforts to understand what works for foster youth and to design policies that effectively support their educational success.

\newpage
\bibliographystyle{apalike}
\begin{thebibliography}{99}

\bibitem[Barrat et al., 2012]{barrat2012predictors}
Barrat, V. X., Berliner, B., \& Fong, A. B. (2012). How Dropout Rates Vary by Foster Care Placement Type. \textit{REL West Technical Brief}.

\bibitem[Bartik et al., 2020]{bartik2020merits}
Bartik, T. J., Hershbein, B., \& Lachowska, M. (2020). The Merits of Universal Scholarships: Benefit-Cost Evidence from the Kalamazoo Promise. \textit{Journal of Benefit-Cost Analysis}, 12(1), 1-26.

\bibitem[Child Welfare Information Gateway, 2023]{childwelfare2023}
Child Welfare Information Gateway. (2023). Foster Care Statistics 2021. Washington, DC: U.S. Department of Health and Human Services.

\bibitem[Conger \& Rebeck, 2010]{conger2010foster}
Conger, D., \& Rebeck, A. (2010). How Children's Foster Care Experiences Affect Their Education. \textit{Chapin Hall Issue Brief}.

\bibitem[Courtney et al., 2007]{courtney2007early}
Courtney, M. E., Dworsky, A., Cusick, G. R., Havlicek, J., Perez, A., \& Keller, T. (2007). Midwest Evaluation of the Adult Functioning of Former Foster Youth: Outcomes at Age 21. \textit{Chapin Hall at the University of Chicago}.

\bibitem[Courtney et al., 2011]{courtney2011midwest}
Courtney, M. E., Dworsky, A., Lee, J. S., \& Raap, M. (2011). Midwest Evaluation of the Adult Functioning of Former Foster Youth: Outcomes at Age 26. \textit{Chapin Hall at the University of Chicago}.

\bibitem[Day et al., 2013]{day2013educational}
Day, A., Dworsky, A., \& Feng, W. (2013). An Analysis of Foster Care Alumni College Students' Persistence at Four-Year Postsecondary Institutions. \textit{Child Welfare}, 92(6), 31-52.

\bibitem[Day et al., 2011]{day2011foster}
Day, A., Dworsky, A., Fogarty, K., \& Damashek, A. (2011). An Examination of Post-Secondary Retention and Graduation Among Foster Care Youth Enrolled in a Four-Year University. \textit{Children and Youth Services Review}, 33(11), 2335-2341.

\bibitem[Dworsky \& Courtney, 2012]{dworsky2012homelessness}
Dworsky, A., \& Courtney, M. E. (2012). Homeless and Without a Job: Foster Care to Young Adult Employment. \textit{Chapin Hall Issue Brief}.

\bibitem[Dworsky \& Perez, 2013]{dworsky2013policy}
Dworsky, A., \& Perez, A. (2013). Helping Former Foster Youth Graduate from College Through Campus Support Programs. \textit{Children and Youth Services Review}, 35(11), 1829-1837.

\bibitem[Gross et al., 2019]{gross2019effect}
Gross, J. P. K., Zerquera, D., Inge, B., \& Berry, M. (2019). Givers, Takers, and Receivers: Foster Care Students and Financial Aid. \textit{Journal of Student Financial Aid}, 49(1), Article 4.

\bibitem[NCSL, 2023]{ncsl2023fostertuition}
National Conference of State Legislatures. (2023). State Tuition Waivers for Foster Youth. Denver, CO: NCSL.

\bibitem[Okpych \& Courtney, 2017]{okpych2017findings}
Okpych, N. J., \& Courtney, M. E. (2017). Does Education Pay for Youth Formerly in Foster Care? \textit{Child Welfare}, 95(1), 15-46.

\bibitem[Page \& Scott-Clayton, 2019]{page2019place}
Page, L. C., \& Scott-Clayton, J. (2019). The Promise of Place-Based Scholarships. In \textit{A Way Forward: Building a More Equitable Higher Education System}. Century Foundation.

\bibitem[PA DHS, 2022]{padhs2022}
Pennsylvania Department of Human Services. (2022). Foster Care Statistics Annual Report. Harrisburg, PA: PA DHS.

\bibitem[Pecora et al., 2006]{pecora2006educational}
Pecora, P. J., Williams, J., Kessler, R. C., Hiripi, E., O'Brien, K., Emerson, J., Herrick, M. A., \& Torres, D. (2006). Assessing the Educational Achievements of Adults Who Were Formerly Placed in Family Foster Care. \textit{Child and Family Social Work}, 11(3), 220-231.

\bibitem[Pecora et al., 2009]{pecora2009mental}
Pecora, P. J., Jensen, P. S., Romanelli, L. H., Jackson, L. J., \& Ortiz, A. (2009). Mental Health Services for Children Placed in Foster Care. \textit{Children and Youth Services Review}, 31(10), 1141-1150.

\bibitem[Rebbe et al., 2017]{rebbe2017food}
Rebbe, R., Nurius, P. S., Ahrens, K. R., \& Courtney, M. E. (2017). Adverse Childhood Experiences Among Youth Aging Out of Foster Care: A Latent Class Analysis. \textit{Children and Youth Services Review}, 74, 108-116.

\bibitem[Salazar, 2013]{salazar2013family}
Salazar, A. M. (2013). The Value of a College Degree for Foster Care Alumni: Comparisons with General Population Samples. \textit{Social Work}, 58(2), 139-150.

\bibitem[Salazar et al., 2016]{salazar2016barriers}
Salazar, A. M., Jones, K. R., Emerson, J. C., \& Mucha, L. (2016). Postsecondary Strengths, Challenges, and Supports Experienced by Foster Care Alumni College Graduates. \textit{Journal of College Student Development}, 57(3), 263-279.

\bibitem[Unrau et al., 2012]{unrau2012ready}
Unrau, Y. A., Font, S. A., \& Rawls, G. (2012). Readiness for College Engagement Among Students Who Have Aged Out of Foster Care. \textit{Children and Youth Services Review}, 34(1), 76-83.

\bibitem[Wulczyn et al., 2003]{wulczyn2003analysis}
Wulczyn, F., Kogan, J., \& Harden, B. J. (2003). Placement Stability and Movement Trajectories. \textit{Social Service Review}, 77(2), 212-236.

\bibitem[Bertrand et al., 2004]{bertrand2004much}
Bertrand, M., Duflo, E., \& Mullainathan, S. (2004). How Much Should We Trust Differences-in-Differences Estimates? \textit{The Quarterly Journal of Economics}, 119(1), 249-275.

\bibitem[de Chaisemartin \& d'Haultfoeuille, 2020]{dechaisemartin2020two}
de Chaisemartin, C., \& d'Haultfoeuille, X. (2020). Two-Way Fixed Effects Estimators with Heterogeneous Treatment Effects. \textit{American Economic Review}, 110(9), 2964-2996.

\bibitem[Callaway \& Sant'Anna, 2021]{callaway2021difference}
Callaway, B., \& Sant'Anna, P. H. C. (2021). Difference-in-Differences with Multiple Time Periods. \textit{Journal of Econometrics}, 225(2), 200-230.

\bibitem[Goodman-Bacon, 2021]{goodman2021difference}
Goodman-Bacon, A. (2021). Difference-in-Differences with Variation in Treatment Timing. \textit{Journal of Econometrics}, 225(2), 254-277.

\bibitem[Conley \& Taber, 2011]{ConleyTaber2011}
Conley, T. G., \& Taber, C. R. (2011). Inference with Difference-in-Differences with a Small Number of Policy Changes. \textit{Review of Economics and Statistics}, 93(1), 113-125.

\bibitem[Cameron \& Miller, 2015]{CameronMiller2015}
Cameron, A. C., \& Miller, D. L. (2015). A Practitioner's Guide to Cluster-Robust Inference. \textit{Journal of Human Resources}, 50(2), 317-372.

\bibitem[Abadie et al., 2010]{abadie2010synthetic}
Abadie, A., Diamond, A., \& Hainmueller, J. (2010). Synthetic Control Methods for Comparative Case Studies: Estimating the Effect of California's Tobacco Control Program. \textit{Journal of the American Statistical Association}, 105(490), 493-505.

\bibitem[MacKinnon \& Webb, 2017]{mackinnon2017wild}
MacKinnon, J. G., \& Webb, M. D. (2017). Wild Bootstrap Inference for Wildly Different Cluster Sizes. \textit{Journal of Applied Econometrics}, 32(2), 233-254.

\end{thebibliography}

\newpage
\appendix
\section{Additional Figures}

\begin{figure}[H]
\centering
\includegraphics[width=0.9\textwidth]{figures/enrollment_trends.png}
\caption{College Enrollment Trends: Pennsylvania vs. Control States. Lines show college enrollment rates among young adults aged 18-26 in Pennsylvania (red) and control states (blue) from 2017 to 2022. The vertical dashed line indicates FosterED implementation in 2019.}
\label{fig:trends}
\end{figure}

\begin{figure}[H]
\centering
\includegraphics[width=0.75\textwidth]{figures/did_estimates.png}
\caption{Summary of DiD Estimates Across Outcomes. Bars show difference-in-differences estimates for three educational outcomes: current college enrollment, any college attainment, and bachelor's degree completion. Values indicate percentage point changes in Pennsylvania relative to control states after FosterED implementation.}
\label{fig:did_summary}
\end{figure}

\end{document}
