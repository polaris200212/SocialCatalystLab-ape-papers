\begin{table}
\centering
\caption{\label{tab:mdonly_results}Effect of Medicaid Unwinding on Provider Supply: No-NP/PA Panel}
\centering
\begin{threeparttable}
\begin{tabular}[t]{lrrrr}
\toprule
Specialty & Estimate & SE & Mean $\bar{Y}$ & N\\
\midrule
\textit{Pooled} & 0.1049 & (0.2122) & 0.397 & 509,328\\
Behavioral Health & 0.0729 & (0.3172) & 0.594 & 84,888\\
Dental & 0.4937 & (0.6861) & 0.713 & 84,888\\
OB-GYN & 0.1171 & (0.0818) & 0.148 & 84,888\\
Primary Care & -0.0394 & (0.2375) & 0.604 & 84,888\\
\addlinespace
Psychiatry & 0.0509 & (0.0727) & 0.191 & 84,888\\
Surgery & -0.0658 & (0.0618) & 0.130 & 84,888\\
\bottomrule
\end{tabular}
\begin{tablenotes}
\item \textit{Note: } 
\item Dependent variable: $\log$(active clinicians + 1). No-NP/PA classification excludes nurse practitioners and physician assistants but retains credential-based specialties (Behavioral Health and Dental) that have no NP/PA taxonomy mappings. Standard errors clustered at the state level. BH and Dental estimates are identical to the all-clinicians panel by construction. Both panels use the same balanced county $\times$ specialty $\times$ quarter structure (zero-filled), so N is identical; only provider counts differ.
\end{tablenotes}
\end{threeparttable}
\end{table}
