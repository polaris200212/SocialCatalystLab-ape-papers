\begin{table}
\centering
\caption{\label{tab:trends}Active Medicaid Clinicians by Specialty and Year (Mean Quarterly Count)}
\centering
\resizebox{\ifdim\width>\linewidth\linewidth\else\width\fi}{!}{
\begin{threeparttable}
\begin{tabular}[t]{lrrrrrrrr}
\toprule
specialty & 2018 & 2019 & 2020 & 2021 & 2022 & 2023 & 2024 & Change (\%)\\
\midrule
Behavioral Health & 7,544 & 7,768 & 7,322 & 9,046 & 9,354 & 9,461 & 8,351 & 10.7\\
Dental & 13,615 & 12,660 & 10,969 & 12,221 & 10,883 & 10,772 & 10,640 & -21.9\\
OB-GYN & 2,236 & 2,071 & 1,874 & 2,004 & 1,870 & 1,772 & 1,620 & -27.5\\
Primary Care & 22,698 & 21,306 & 20,708 & 22,620 & 22,786 & 22,782 & 21,675 & -4.5\\
Psychiatry & 3,411 & 3,267 & 3,138 & 3,220 & 3,185 & 3,150 & 3,023 & -11.4\\
\addlinespace
Surgery & 1,569 & 1,659 & 1,513 & 1,844 & 1,798 & 1,837 & 1,639 & 4.5\\
\bottomrule
\end{tabular}
\begin{tablenotes}
\item \textit{Note: } 
\item Mean quarterly count of active clinicians across Q1--Q3 of each year (Q4 excluded for comparability since 2024 data ends at Q3). All clinicians (MD/DO + NP/PA mapped to clinical specialty) with $\geq$4 Medicaid claims per quarter. Counts reflect unique providers per county-quarter summed across counties in each quarter, then averaged across quarters within the year.
\end{tablenotes}
\end{threeparttable}}
\end{table}
