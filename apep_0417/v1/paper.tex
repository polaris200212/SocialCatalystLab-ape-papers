\documentclass[12pt]{article}

% UTF-8 encoding and fonts
\usepackage[utf8]{inputenc}
\usepackage[T1]{fontenc}
\usepackage{lmodern}  % Latin Modern font - fixes < > rendering issues

% Page setup
\usepackage[margin=1in]{geometry}
\usepackage{setspace}
\onehalfspacing

% Typography
\usepackage{microtype}

% Math and symbols
\usepackage{amsmath,amssymb}

% Graphics
\usepackage{graphicx}
\usepackage{float}
\usepackage{subcaption}

% Tables
\usepackage{booktabs}
\usepackage{array}
\usepackage{multirow}
\usepackage{threeparttable} % provides tablenotes
\usepackage{longtable}
\usepackage{pdflscape}
\usepackage{siunitx}
\sisetup{detect-all=true, group-separator={,}, group-minimum-digits=4}

% Bibliography
\usepackage{natbib}
\bibliographystyle{aer}  % American Economic Review style

% Hyperlinks
\usepackage{hyperref}
\hypersetup{
    colorlinks=true,
    linkcolor=blue,
    citecolor=blue,
    urlcolor=blue
}
\usepackage[nameinlink,noabbrev]{cleveref}

% Timing data (generated by timing_log.py)
\IfFileExists{timing_data.tex}{\newcommand{\apepcurrenttime}{1h 4m}
\newcommand{\apepcumulativetime}{1h 4m}
}{
  \newcommand{\apepcurrenttime}{N/A}
  \newcommand{\apepcumulativetime}{N/A}
}

% Captions
\usepackage{caption}
\captionsetup{font=small,labelfont=bf}

% Section formatting
\usepackage{titlesec}
\titleformat{\section}{\large\bfseries}{\thesection.}{0.5em}{}
\titleformat{\subsection}{\normalsize\bfseries}{\thesubsection}{0.5em}{}

% Custom commands
\newcommand{\E}{\mathbb{E}}
\newcommand{\Var}{\text{Var}}
\newcommand{\Cov}{\text{Cov}}
\newcommand{\ind}{\mathbb{I}}
\newcommand{\sym}[1]{\ifmmode^{#1}\else\(^{#1}\)\fi} % significance stars for tables

\title{Where Medicaid Goes Dark: A Claims-Based Atlas of Provider Deserts\\ and the Resilience of Supply to Enrollment Shocks}
\author{APEP Autonomous Research\thanks{Autonomous Policy Evaluation Project. This paper was generated autonomously. Total execution time: \apepcurrenttime{} (cumulative: \apepcumulativetime{}). Correspondence: scl@econ.uzh.ch} \and @SocialCatalystLab}
\date{\today}

\begin{document}

\maketitle

\begin{abstract}
\noindent
Using newly released T-MSIS Medicaid claims data covering 227 million provider-service records (2018Q1--2024Q3), we construct the first claims-based atlas of county-level Medicaid provider deserts across seven clinical specialties. The descriptive portrait is alarming: psychiatrists billing Medicaid declined 21\%, OB-GYNs 28\%, and primary care physicians 16\% from 2018 to 2024. Rural desert rates exceed 80\% for psychiatry. Exploiting staggered state-level variation in post-pandemic enrollment unwinding, we find that provider supply is \textit{remarkably inelastic} to short-run demand shocks: estimates are precisely estimated near zero across all specialties, confirmed by permutation inference ($p = 0.962$). Medicaid deserts appear to reflect chronic structural factors---plausibly including reimbursement and workforce dynamics---rather than acute enrollment fluctuations.
\end{abstract}

\vspace{1em}
\noindent\textbf{JEL Codes:} I11, I13, I18, J44 \\
\noindent\textbf{Keywords:} Medicaid, provider access, medical deserts, enrollment unwinding, health workforce, supply inelasticity

\newpage

\section{Introduction}

In April 2023, American states began the largest administrative shock to health insurance in the nation's history. The end of the pandemic-era continuous enrollment provision triggered the ``unwinding'' of Medicaid rolls, eventually removing more than 25 million people from coverage \citep{corallo2024}. The policy debate has understandably focused on those who lost insurance. But a natural concern extends to the supply side: when patients vanish, do their doctors follow? If enrollment loss drives provider exit, the unwinding could deepen medical deserts, trapping those who remain covered in areas where care is increasingly unavailable.

This paper investigates this question using the most comprehensive data ever assembled on Medicaid provider activity, and arrives at a surprising answer. The Medicaid provider crisis is real --- and severe. But the enrollment unwinding is not its cause.

The question matters because Medicaid access has always been about more than an insurance card. Decades of research have documented a persistent gap between formal eligibility and realized access. \citet{decker2012} showed that only two-thirds of physicians accept new Medicaid patients, a rate that has declined over time. \citet{polsky2015} found that simulated Medicaid patients seeking appointments were turned away at rates far exceeding the privately insured. The fundamental driver is reimbursement: Medicaid pays physicians roughly 72 cents on the Medicare dollar, creating a financial disincentive that concentrates willing providers in densely populated areas where volume compensates for low margins \citep{zuckerman2009}. The result is that Medicaid beneficiaries in rural and low-income communities face severe shortages that existing measurement tools systematically understate.

The standard measure of physician shortage --- the Health Resources and Services Administration's Health Professional Shortage Area (HPSA) designation --- relies on provider-to-population ratios derived from survey data and registries \citep{hrsa2023}. These data capture who \textit{could} treat Medicaid patients, not who actually does. A county may have adequate physician supply for its privately insured residents while functioning as a complete desert for Medicaid beneficiaries. The distinction between registered providers and active billers is empirically enormous and, until now, unmeasurable at scale.

This paper exploits the recent release (February 2026) of the Transformed Medicaid Statistical Information System (T-MSIS) provider spending data to make two contributions. The first is descriptive: we construct the first claims-based atlas of Medicaid provider deserts. T-MSIS records every provider-service transaction processed through state Medicaid programs, yielding 227 million billing provider--servicing provider--procedure code observations spanning 2018Q1--2024Q3. We link these claims to the National Plan and Provider Enumeration System (NPPES) to classify providers by specialty and location, constructing a county $\times$ specialty $\times$ quarter panel covering 2,741 counties and seven clinical specialty groups: primary care physicians, nurse practitioners and physician assistants, psychiatrists, behavioral health providers, dentists, obstetrician-gynecologists, and surgeons.

The descriptive facts are striking. Comparing Q1--Q3 of 2018 to Q1--Q3 of 2024, the number of psychiatrists actively billing Medicaid fell by approximately 21\%. OB-GYNs declined roughly 28\%, primary care physicians 16\%, and dentists 22\%. These are not registry counts of who holds a Medicaid provider number --- they reflect who actually submitted claims, our measure of active participation. The declines are not uniform. Rural counties experienced desert rates two to three times higher than urban counties across every specialty. By 2023, the vast majority of rural county-quarters had zero active Medicaid psychiatrists. Partially offsetting these losses, nurse practitioners and physician assistants billing Medicaid grew approximately 40\%, consistent with the expansion of scope-of-practice authority documented by \citet{alexander2019}. But the substitution is incomplete: NP growth concentrated in primary care and behavioral health, leaving psychiatry and OB-GYN deserts largely unfilled.

The second contribution is causal --- and its main finding is a null. We exploit the staggered state-level variation in Medicaid enrollment unwinding to estimate whether the demand shock of disenrollment drives provider exit. States varied dramatically in both timing and intensity: net enrollment losses ranged from 1.4\% in Maine to 30.2\% in Colorado, with a median of 14.0\% \citep{kff2024}. Our difference-in-differences design estimates the effect of a state's cumulative disenrollment rate on county-level active provider counts, validated by event-study evidence of flat pre-trends across eight pre-treatment quarters.

The central result is that provider supply is remarkably inelastic to the enrollment shock. The pooled estimate across all specialties is 0.0128 ($\text{SE} = 0.2376$, $p = 0.957$) --- precisely estimated at zero. No individual specialty shows a statistically significant response. The event study confirms clean pre-trends and shows no meaningful post-treatment departure from zero. Permutation inference with 500 randomizations of state treatment timing yields a $p$-value of 0.962, placing our observed coefficient squarely in the middle of the null distribution. The null result is robust to alternative active-provider thresholds, binary treatment definitions, controlling for time-varying Medicaid share, and a pre-period placebo test.

The null finding is important precisely because it is surprising. The intuitive hypothesis --- that losing 25 million patients should cause providers to exit --- has a clear theoretical motivation and has shaped policy discussions around the unwinding. Our evidence suggests this mechanism does not operate on the margins observed in the data. Several explanations are consistent with the null. First, Medicaid typically constitutes a minority of a provider's payer mix, even in areas with high Medicaid penetration. A 15\% decline in state Medicaid enrollment may translate to only a 2--3\% decline in a provider's total patient volume --- too small to trigger exit decisions. Second, provider participation decisions may respond to long-run structural factors (reimbursement rates, geographic preferences, retirement timing) rather than short-run enrollment fluctuations. Third, the unwinding may have been too short and too recent for supply-side adjustments to fully materialize in our six-quarter post-period.

This paper contributes to several literatures. First, we advance the measurement of health care access by constructing the first claims-based provider desert atlas, complementing survey-based approaches \citep{goodman2023,ricketts2007,graves2016}. Claims data reveal that HPSA designations substantially understate the Medicaid-specific shortage: many counties with adequate physician supply overall have zero active Medicaid billers in one or more specialties. Second, we contribute to the literature on Medicaid provider participation, which has documented the roles of reimbursement rates \citep{decker2012,zuckerman2009,gruber2003}, physician attitudes \citep{long2013}, and managed care penetration \citep{duggan2004} in shaping the provider network. Our null causal result suggests that the demand side --- at least in the form of short-run enrollment shocks --- is \textit{not} a first-order determinant of provider participation, consistent with models in which fixed costs and long-run reimbursement dominate entry/exit decisions. Third, we contribute to the rapidly growing literature on the Medicaid enrollment unwinding, which has documented coverage losses \citep{corallo2024,kff2024,wagner2023} and early evidence on health care utilization \citep{sommers2024}. We provide the first evidence on supply-side effects, finding that the provider market absorbs the enrollment shock without detectable adjustment.

Our work is also related to the broader literature on public insurance and health care supply. \citet{garthwaite2014} showed that public insurance crowd-out operates partly through the provider market. \citet{sommers2017} documented that Medicaid expansion improved access to care. \citet{miller2019} showed that Medicaid coverage reduces mortality, an effect that depends on provider availability. The maintained hypothesis in much of this literature --- that provider supply responds elastically to demand --- receives only limited support in our setting. The supply response may require larger or more persistent demand shocks than the unwinding delivered, or the relevant margin of adjustment may be quality or effort rather than extensive-margin exit.



\section{Background}

\subsection{The Medicaid Provider Access Problem}

Medicaid is the largest source of health insurance in the United States, covering approximately 90 million people before the pandemic and peaking near 94 million during the continuous enrollment period \citep{cms2024}. Despite this scale, access to care for Medicaid beneficiaries has been a persistent policy concern. The core issue is reimbursement. Medicaid physician fees averaged 72\% of Medicare rates nationally as of 2019, with wide interstate variation: New York paid 29\% of Medicare for primary care office visits, while Alaska paid 175\% \citep{zuckerman2009,medicaidfees2023}. Low reimbursement discourages physician participation, creating a wedge between formal coverage and realized access.

The consequences are well documented. \citet{polsky2015} conducted audit studies in which simulated patients called physicians seeking new-patient appointments. Medicaid callers were offered appointments in 58\% of calls, compared with 85\% for privately insured callers. \citet{decker2012} found that only 69\% of physicians accepted new Medicaid patients nationally, with rates below 50\% in some states. \citet{rhodes2014} showed that areas with fewer Medicaid-accepting providers had worse access to preventive care and higher rates of emergency department utilization. The provider access problem is not merely an inconvenience; it determines whether insurance coverage translates into health care delivery.

\subsection{Measuring Medical Deserts}

The standard federal measure of health care access is the Health Professional Shortage Area (HPSA) designation, administered by HRSA since 1978. HPSA scores use provider-to-population ratios, poverty rates, and distance measures to classify geographic areas, populations, or facilities as underserved \citep{hrsa2023}. The methodology has been criticized on several grounds: it relies on state-submitted data that may be outdated, it counts all licensed physicians regardless of payer mix, and the geographic units (often counties or census tracts) may not reflect actual patient travel patterns \citep{ricketts2007,graves2016}.

For Medicaid-specific access, HPSA designations are particularly misleading. A county may have an adequate overall physician-to-population ratio while functioning as a complete desert for Medicaid patients if most physicians decline Medicaid. The distinction matters enormously for policy: HPSA scores drive billions in federal funding through programs like the National Health Service Corps, FQHC reimbursement bonuses, and J-1 visa waivers for international medical graduates \citep{hrsa2023}. If these designations miss Medicaid-specific shortages, resources may be misdirected.

\subsection{The Pandemic and Continuous Enrollment}

The Families First Coronavirus Response Act (FFCRA) of March 2020 required states to maintain Medicaid enrollment for all beneficiaries as a condition of receiving enhanced federal matching funds. This ``continuous enrollment'' provision effectively suspended routine eligibility redeterminations, preventing disenrollment for any reason other than a beneficiary's request, death, or interstate move. The result was a dramatic expansion of the Medicaid rolls. National enrollment grew from 71 million in February 2020 to an all-time high of approximately 94 million by March 2023, a 32\% increase \citep{cms2024}.

The coverage expansion brought new patients into the Medicaid system, potentially supporting provider participation by increasing the volume of Medicaid patients in areas where provider networks had been thinning. At the same time, the pandemic itself disrupted health care delivery --- elective procedures were postponed, telehealth expanded rapidly, and many providers experienced financial strain \citep{aha2021}. The net effect on Medicaid provider supply during the pandemic era is ambiguous and represents part of the variation we document in our descriptive atlas.

\subsection{The Unwinding}

The Consolidated Appropriations Act of December 2022 set March 31, 2023, as the end date for the continuous enrollment provision. Beginning April 1, 2023, states could resume normal eligibility redeterminations and disenroll individuals found ineligible. The ``unwinding'' that followed was the largest administrative event in Medicaid's history.

States varied dramatically in their approach. Early movers like Arkansas, Idaho, and South Dakota began processing renewals and issuing disenrollments immediately in April 2023. Oregon delayed until October 2023. California and New York processed renewals but moved cautiously, investing in outreach to minimize procedural disenrollments. By mid-2024, states had collectively removed over 25 million people from Medicaid rolls \citep{corallo2024,kff2024}.

The variation was not only in timing but in intensity. Net enrollment declines (from peak to trough) ranged from approximately 1.4\% in Maine to 30.2\% in Colorado, with a median state losing about 14\% of peak enrollment. Critically, a large share of disenrollments were procedural --- beneficiaries removed for failing to return paperwork rather than for confirmed ineligibility \citep{wagner2023}. Several states paused their unwinding after discovering high procedural disenrollment rates, further adding to the cross-state variation in timing and intensity that we exploit.

\subsection{Provider Market Dynamics}

The Medicaid unwinding occurred against a backdrop of secular trends in the health care workforce. Physician retirement has accelerated, with the Association of American Medical Colleges projecting a shortage of up to 124,000 physicians by 2034 \citep{aamc2021}. The shortage is most acute in primary care and psychiatry, the specialties where Medicaid payment gaps relative to private insurance are largest. Simultaneously, the geographic concentration of physicians has intensified: \citet{goodman2023} documented that rural areas lost physicians at a faster rate than urban areas throughout the 2010s, a trend driven by lifestyle preferences, institutional agglomeration effects, and the economics of low-volume practice.

Partially offsetting physician losses, nurse practitioners and physician assistants have expanded their scope of practice in many states. As of 2024, 27 states and the District of Columbia grant NPs full practice authority, allowing independent practice without physician oversight \citep{alexander2019,aanp2024}. NPs are disproportionately likely to practice in rural and underserved areas, making them a potential substitute for departing physicians. Whether this substitution is sufficient to maintain access in Medicaid-specific deserts is an empirical question our data can address.


\section{Data}

\subsection{T-MSIS Provider Spending Data}

Our primary data source is the Transformed Medicaid Statistical Information System (T-MSIS) provider spending file, released by the Department of Health and Human Services in February 2026. T-MSIS records every provider-service transaction processed through state Medicaid fee-for-service and managed care encounter systems. The raw data contain 227 million billing-NPI $\times$ servicing-NPI $\times$ HCPCS-code $\times$ month observations spanning January 2018 through December 2024, though we restrict the analysis sample to 2018Q1--2024Q3 due to incomplete Q4 2024 data from billing lags. For each record, we observe the number of claims, total Medicaid payments, and the number of unique beneficiaries served.

We use the \textit{billing} NPI as our unit of analysis for aggregation to the provider-quarter level. We aggregate from monthly to quarterly frequency to reduce noise from billing cycles and seasonal variation.

\subsection{NPPES Provider Registry}

We link T-MSIS records to the National Plan and Provider Enumeration System (NPPES), a CMS-maintained registry of all NPI holders. NPPES provides each provider's practice location (mailing and practice-location ZIP codes), taxonomy codes (which encode specialty and credential type), and enumeration date.

We classify providers into seven specialty groups using the National Uniform Claim Committee (NUCC) Health Care Provider Taxonomy codes: (1) \textit{Primary Care Physicians} (family medicine, internal medicine, general practice, pediatrics --- MD/DO only); (2) \textit{Nurse Practitioners and Physician Assistants} (all NP and PA taxonomy codes); (3) \textit{Psychiatrists} (psychiatry, child/adolescent psychiatry --- MD/DO); (4) \textit{Behavioral Health Providers} (psychologists, clinical social workers, licensed counselors); (5) \textit{Dentists} (general dentistry, oral surgery, pediatric dentistry, orthodontics); (6) \textit{Obstetrician-Gynecologists} (obstetrics, gynecology, maternal-fetal medicine); and (7) \textit{Surgeons} (general surgery, orthopedic surgery, cardiovascular surgery, neurosurgery). Of approximately 2.96 million classified individual providers, 98.8\% map successfully to a county FIPS code via ZCTA crosswalk.

\subsection{Geographic Crosswalks}

We map individual providers to counties using the NPPES practice-location ZIP code. We employ the Census Bureau's ZIP Code Tabulation Area (ZCTA) to county FIPS crosswalk, which assigns ZCTAs that span multiple counties to the county with the largest land-area overlap.

We classify counties as urban (metropolitan) or rural (non-metropolitan) using the USDA Economic Research Service's Rural-Urban Continuum Codes (RUCC), where codes 1--3 are metropolitan and codes 4--9 are non-metropolitan. Population denominators come from the American Community Survey (ACS) 5-year estimates (2018--2022), accessed via the Census Bureau API.

\subsection{Active Provider Definition}

A central measurement decision is the threshold for ``active'' Medicaid participation. We define a provider as active in a county-specialty-quarter cell if they billed at least four Medicaid claims in that quarter. In practice, T-MSIS applies cell suppression to provider-procedure-month records with fewer than 12 claims, so the minimum observable quarterly claim count is 12. Our baseline threshold of $\geq$4 claims is therefore non-binding --- all providers in the data exceed it --- but we retain it as a conceptual anchor consistent with related literature on provider participation.

An important caveat is that T-MSIS cell suppression operates at the billing NPI $\times$ servicing NPI $\times$ HCPCS $\times$ month level. At the provider-quarter level used in our analysis, a provider appears if \emph{any} non-suppressed cell exists, substantially mitigating censorship. Nevertheless, providers whose entire quarterly billing falls below the suppression threshold will be missing from our data, potentially introducing non-classical measurement error. If marginal providers---those most likely to exit following a demand shock---are disproportionately low-volume and suppressed, our null could partly reflect attenuation bias. We return to this limitation in the Discussion.

\subsection{Panel Construction}

Our unit of analysis is the county $\times$ specialty $\times$ quarter cell. For each cell, we compute: (1) the number of active providers; (2) active providers per 10,000 county population; and (3) a binary desert indicator equal to one if the cell contains fewer than one active provider per 10,000 population. The resulting balanced panel spans 2,741 counties, 7 specialties, and 27 quarters (2018Q1--2024Q3), yielding approximately 518,000 county-specialty-quarter observations before exclusions. After excluding counties with missing population data and US territories, our analysis panel contains 71,604 county-specialty-quarter observations per specialty (2,652 counties $\times$ 27 quarters), or 501,228 observations in the pooled sample across all seven specialties. We exclude 2024Q4 due to incomplete claims data from billing lags. Approximately 67\% of county-specialty-quarter cells contain zero active providers, reflecting the extensive desert problem that motivates our analysis.

\subsection{Unwinding Treatment Data}

We construct the treatment variable from two sources. State unwinding \textit{start dates} come from the Kaiser Family Foundation's Medicaid Enrollment and Unwinding Tracker \citep{kff2024} and CMS state-level timeline reports. We assign each state a binary post-unwinding indicator equal to one for all quarters at or after its unwinding start quarter. \textit{Unwinding intensity} is measured as the cumulative net disenrollment rate: the percentage decline in total Medicaid enrollment from each state's peak enrollment (typically March 2023) to its latest available data.

The treatment variable in our main specification is the interaction of the post-unwinding indicator with the state's net disenrollment rate, providing a continuous measure of unwinding intensity. States are distributed across three adoption cohorts: 40 states began in 2023Q2 (April--June), 10 in 2023Q3 (July--September), and Oregon alone in 2023Q4 (October).

\subsection{Summary Statistics}

Table \ref{tab:tab:sumstats} presents summary statistics for the analysis panel by specialty.

\begin{table}
\centering
\caption{\label{tab:tab:sumstats}Summary Statistics by Specialty}
\centering
\resizebox{\ifdim\width>\linewidth\linewidth\else\width\fi}{!}{
\begin{threeparttable}
\begin{tabular}[t]{lrrrrrrr}
\toprule
Specialty & County-Quarters & Counties & Mean Providers & SD & \% Zero & Per 10K Pop & \% Desert\\
\midrule
Behavioral Health & 71,604 & 2,652 & 3.2 & 10.7 & 51.7 & 0.30 & 92.6\\
Dental & 71,604 & 2,652 & 4.4 & 20.0 & 40.9 & 0.39 & 89.0\\
NP/PA & 71,604 & 2,652 & 2.2 & 18.0 & 70.2 & 0.17 & 95.2\\
OB-GYN & 71,604 & 2,652 & 0.7 & 5.9 & 85.9 & 0.03 & 99.8\\
Primary Care & 71,604 & 2,652 & 6.4 & 53.2 & 53.0 & 0.32 & 90.7\\
\addlinespace
Psychiatry & 71,604 & 2,652 & 1.0 & 7.9 & 82.2 & 0.03 & 99.7\\
Surgery & 71,604 & 2,652 & 0.6 & 6.1 & 87.2 & 0.03 & 99.7\\
\bottomrule
\end{tabular}
\begin{tablenotes}
\item \textit{Note: }
\item Sample: US county $\times$ specialty $\times$ quarter observations, 2018Q1--2024Q3. Active providers defined as billing $\geq$4 Medicaid claims per quarter. Desert: $<$1 active provider per 10,000 county population.
\end{tablenotes}
\end{threeparttable}}
\end{table}


\section{The Atlas: Descriptive Results}

This section presents the core descriptive findings --- the first claims-based atlas of Medicaid provider deserts across the United States. The patterns are stark, and we document them in detail before turning to causal analysis.

\subsection{National Provider Trends}

Figure \ref{fig:trends} plots the total number of active Medicaid providers by specialty from 2018Q1 to 2024Q3. The red dashed line marks April 2023, the onset of the unwinding. The dominant pattern is decline. Every physician specialty shows a downward trajectory over the six-year period, with the declines beginning well before the unwinding --- a first indication that the provider crisis is structural rather than a response to enrollment fluctuations.

\begin{figure}[H]
\centering
\includegraphics[width=\textwidth]{figures/fig1_provider_trends.pdf}
\caption{Active Medicaid Providers by Specialty, 2018--2024}
\label{fig:trends}
\begin{minipage}{\textwidth}
\vspace{0.3em}
\footnotesize\textit{Notes:} Active providers defined as billing $\geq$4 Medicaid claims per quarter. Red dashed line indicates the start of Medicaid enrollment unwinding (April 2023). Counts are summed across all counties.
\end{minipage}
\end{figure}

Figure \ref{fig:indexed} makes the magnitudes comparable by indexing each specialty to 100 in 2018Q1. The divergence is dramatic. Table \ref{tab:tab:trends} reports the year-by-year totals, summing active provider counts across Q1--Q3 county-quarter observations within each year for comparability (since 2024 data ends at Q3). Because Table \ref{tab:tab:trends} aggregates across three quarters, its counts are roughly three times the individual quarterly values shown in Figure \ref{fig:trends}, which plots provider counts for each quarter separately. Over the full sample period, psychiatry declined 20.9\%, OB-GYN 28.1\%, primary care physicians 15.9\%, and dentists 21.9\%. Surgeons showed modest overall growth of 4.9\%. The sole rapidly expanding category is nurse practitioners and physician assistants, who grew 40.1\% over the period, consistent with the expansion of scope-of-practice authority documented by \citet{alexander2019}. Behavioral health providers showed moderate growth of 10.7\%.

\begin{figure}[H]
\centering
\includegraphics[width=\textwidth]{figures/fig2_indexed_trends.pdf}
\caption{Medicaid Provider Supply by Specialty, Indexed to 2018Q1}
\label{fig:indexed}
\begin{minipage}{\textwidth}
\vspace{0.3em}
\footnotesize\textit{Notes:} Each series normalized to 100 in 2018Q1. Red dashed line marks the start of Medicaid enrollment unwinding (April 2023).
\end{minipage}
\end{figure}

Two features of these trends merit emphasis. First, the physician exodus from Medicaid is not an unwinding phenomenon. Declines began in 2018--2019, continued steadily through the pandemic-era coverage expansion, and show no visible acceleration after 2023. The unwinding is barely perceptible in the aggregate time series --- a striking visual foreshadowing of our formal null result. Second, the NP/PA expansion, while substantial in percentage terms, operates on a different base. The absolute number of NPs added to Medicaid is insufficient to replace the physicians lost, particularly in specialties like psychiatry and OB-GYN where NP substitution is limited by credentialing and scope-of-practice constraints.

Table \ref{tab:tab:trends} provides the year-by-year cumulative Q1--Q3 provider counts and percentage changes.

\begin{table}
\centering
\caption{\label{tab:tab:trends}Active Medicaid Providers by Specialty and Year}
\centering
\resizebox{\ifdim\width>\linewidth\linewidth\else\width\fi}{!}{
\begin{threeparttable}
\begin{tabular}[t]{lrrrrrrrr}
\toprule
specialty & 2018 & 2019 & 2020 & 2021 & 2022 & 2023 & 2024 & Change (\%)\\
\midrule
Behavioral Health & 22,631 & 23,304 & 21,967 & 27,138 & 28,061 & 28,382 & 25,053 & 10.7\\
Dental & 40,846 & 37,981 & 32,906 & 36,664 & 32,650 & 32,315 & 31,920 & -21.9\\
NP/PA & 15,439 & 15,047 & 14,922 & 16,236 & 18,664 & 20,503 & 21,633 & 40.1\\
OB-GYN & 6,419 & 5,900 & 5,373 & 5,811 & 5,394 & 5,089 & 4,617 & -28.1\\
Primary Care & 54,605 & 50,829 & 49,087 & 53,626 & 51,927 & 50,242 & 45,922 & -15.9\\
\addlinespace
Psychiatry & 8,645 & 8,228 & 7,833 & 7,925 & 7,603 & 7,352 & 6,837 & -20.9\\
Surgery & 4,633 & 4,904 & 4,482 & 5,467 & 5,331 & 5,437 & 4,862 & 4.9\\
\bottomrule
\end{tabular}
\begin{tablenotes}
\item \textit{Note: }
\item Providers with $\geq$4 Medicaid claims per quarter. Sum across Q1--Q3 county-quarter observations in each year (Q4 excluded for comparability since 2024 data ends at Q3).
\end{tablenotes}
\end{threeparttable}}
\end{table}

\subsection{The Geography of Medicaid Deserts}

Our county-level desert measure complements spatial accessibility approaches in the health geography literature \citep{guagliardo2004}. Figure \ref{fig:maps} presents county-level maps of Medicaid provider density for six specialties as of 2023Q1, the last quarter before the unwinding. Counties are shaded by active Medicaid providers per 10,000 population, with dark red indicating zero providers.

\begin{figure}[H]
\centering
\includegraphics[width=\textwidth]{figures/fig3_desert_maps.pdf}
\caption{Medicaid Provider Deserts by Specialty, 2023Q1}
\label{fig:maps}
\begin{minipage}{\textwidth}
\vspace{0.3em}
\footnotesize\textit{Notes:} Active providers defined as billing $\geq$4 Medicaid claims per quarter. Providers per 10,000 county population. Dark red indicates zero active providers. Continental US only.
\end{minipage}
\end{figure}

The maps reveal a geography of absence. For psychiatry, the Great Plains, the rural South, and Appalachia form vast contiguous regions with zero active Medicaid psychiatrists. The pattern for OB-GYN is similar, with large portions of the interior West and South entirely devoid of Medicaid obstetric coverage. Primary care deserts are less extensive but still substantial in rural areas. Dental deserts follow a distinct pattern, concentrated in the South, where several states had limited Medicaid dental benefits during the sample period.

Three observations emerge. First, the urban-rural gradient is extreme. Metropolitan counties almost universally have at least some Medicaid provider coverage across all specialties, while rural counties frequently have none. Second, the deserts are often contiguous --- isolated rural counties lack coverage, but so do their neighbors, meaning that travel distances to the nearest Medicaid provider can span hundreds of miles. Third, the pattern varies substantially across specialties. A county may have adequate Medicaid primary care while being a complete desert for psychiatry and OB-GYN.

\subsection{Urban-Rural Disparities}

Figure \ref{fig:urban_rural} quantifies the urban-rural gap over time for four key specialties. The gap is large and remarkably stable. For psychiatry, rural desert rates exceed 80\% of county-quarters throughout the sample, compared with approximately 30\% for urban counties. For OB-GYN, the rural desert rate hovers near 70--75\% across the full period. Notably, the unwinding (marked by the red dashed line) produces no visible discontinuity in either urban or rural desert trends for any specialty --- consistent with the supply inelasticity we document formally below.

\begin{figure}[H]
\centering
\includegraphics[width=\textwidth]{figures/fig5_urban_rural_deserts.pdf}
\caption{Medicaid Desert Counties by Specialty: Urban vs. Rural}
\label{fig:urban_rural}
\begin{minipage}{\textwidth}
\vspace{0.3em}
\footnotesize\textit{Notes:} Desert defined as $<$1 active Medicaid provider per 10,000 county population. Metro/non-metro classification from USDA Rural-Urban Continuum Codes. Red dashed line marks unwinding start.
\end{minipage}
\end{figure}

Table \ref{tab:tab:desert_counts} reports desert rates before and after the unwinding by specialty. While desert rates are generally higher in the post period, this is fully consistent with the secular decline trends documented above. The causal analysis in Section 6 isolates the unwinding-specific component.

\begin{table}
\centering
\caption{\label{tab:tab:desert_counts}Percentage of County-Quarters in Desert Status, Pre vs. Post Unwinding}
\centering
\begin{threeparttable}
\begin{tabular}[t]{lrrr}
\toprule
specialty & Pre-Unwinding (2018Q1-2023Q1) & Post-Unwinding (2023Q2-2024Q3) & Overall\\
\midrule
Behavioral Health & 92.6 & 92.2 & 92.6\\
Dental & 88.1 & 92.2 & 89.0\\
NP/PA & 95.1 & 95.4 & 95.2\\
OB-GYN & 99.8 & 99.9 & 99.8\\
Primary Care & 90.0 & 93.0 & 90.7\\
\addlinespace
Psychiatry & 99.7 & 99.8 & 99.7\\
Surgery & 99.7 & 99.8 & 99.7\\
\bottomrule
\end{tabular}
\begin{tablenotes}
\item \textit{Note: }
\item Desert: county-specialty-quarter with $<$1 active Medicaid provider per 10,000 population. Overall column is the unweighted mean across all quarters.
\end{tablenotes}
\end{threeparttable}
\end{table}


\section{Empirical Strategy}

\subsection{Identification}

Our causal analysis exploits state-level variation in the timing and intensity of the Medicaid enrollment unwinding. The design is a difference-in-differences approach in which ``treatment'' is the interaction of a state's unwinding onset with its cumulative disenrollment intensity. The key identifying assumption is that, conditional on county-specialty and time fixed effects, provider trends would have evolved similarly across states with different unwinding intensity in the absence of the policy.

We estimate the following specification:
\begin{equation}
Y_{cjt} = \beta \cdot \text{UnwindIntensity}_{g(c),t} + \alpha_{cj} + \gamma_t + \varepsilon_{cjt}
\label{eq:main}
\end{equation}
where $Y_{cjt}$ is the log of active Medicaid providers plus one in county $c$, specialty $j$, quarter $t$. $\text{UnwindIntensity}_{g(c),t}$ is the treatment variable, which varies at the state level: $g(c)$ denotes the state containing county $c$. It equals zero before state $g$'s unwinding start date and equals the state's net disenrollment rate (as a fraction) after the start date. $\alpha_{cj}$ are county $\times$ specialty fixed effects, which absorb all time-invariant differences across county-specialty cells, including baseline provider supply, population composition, rurality, and state Medicaid reimbursement levels. $\gamma_t$ are quarter fixed effects, which absorb national trends in provider supply, including secular retirement patterns and the effects of the pandemic. Standard errors are clustered at the state level, the level of treatment variation, providing 51 clusters (50 states plus DC).

The coefficient $\beta$ captures the effect of a one-unit increase in the state's net disenrollment rate on county-level log provider counts, in the post-unwinding period relative to the pre-period. A negative $\beta$ would indicate that more intense unwinding is associated with larger provider declines. We estimate the model pooled across all specialties and separately by specialty.

\subsection{Event Study}

To assess the parallel trends assumption and characterize the dynamics of the effect, we estimate an event study:
\begin{equation}
Y_{cjt} = \sum_{k=-8}^{5} \beta_k \cdot \ind[t - t_g^* = k] \cdot d_g + \alpha_{cj} + \gamma_t + \varepsilon_{cjt}
\label{eq:eventstudy}
\end{equation}
where $g = g(c)$ denotes the state of county $c$, $t_g^*$ is the quarter in which state $g$ began unwinding, $k$ indexes quarters relative to the unwinding start, $d_g$ is the state's net disenrollment rate (time-invariant intensity measure), and $\ind[\cdot]$ is the indicator function. The omitted category is $k = -1$, the quarter immediately before the unwinding begins. Pre-treatment coefficients $\beta_{-8}, \ldots, \beta_{-2}$ should be zero under parallel trends; post-treatment coefficients $\beta_0, \ldots, \beta_5$ trace out the dynamic effect.

\subsection{Threats to Identification}

Four potential biases could color our results. First, managed care organization (MCO) encounter reporting may have changed differentially across states during the sample period. If states that unwound aggressively also happened to improve or worsen their MCO encounter data submissions to T-MSIS, this could create spurious variation in our outcome measure. We address this by noting that our pre-period spans five years of relatively stable reporting.

Second, contemporaneous policy changes could confound the unwinding effect. Most notably, the American Rescue Plan Act (ARPA) provided enhanced FMAP for Medicaid home- and community-based services (HCBS) spending through March 2025. If ARPA-funded provider rate increases differentially attracted or retained providers in states that also unwound aggressively, our estimates could be biased. We address this by controlling for the state's Medicaid spending share in a robustness specification.

Third, provider exit in T-MSIS may reflect billing cessation rather than true market exit. A provider who stops billing Medicaid may continue to practice, seeing only privately insured patients. Our measure captures participation in the Medicaid market specifically, not the provider's continued existence. This is the economically relevant outcome for Medicaid beneficiary access.

Fourth, because most states began unwinding within a three-quarter window (2023Q2--2023Q4), the staggering is limited compared to classic staggered DiD settings spanning many years. This limits the power of cohort-by-cohort estimates but strengthens the continuous-intensity design, which exploits the considerable variation in \textit{how much} enrollment was lost across states. The long pre-period and eight-quarter event-study window provide substantial leverage for assessing parallel trends.

\subsection{Relationship to the Staggered DiD Literature}

Our approach builds on the recent literature addressing heterogeneous treatment effects in staggered difference-in-differences designs \citep{goodmanbacon2021, dechaisemartin2020, callawaysantanna2021, rothetal2023}. Following \citet{sunab2021}, we report interaction-weighted estimates that are robust to treatment effect heterogeneity across cohorts. The near-identical point estimates from TWFE and Sun-Abraham estimators suggest that negative-weight bias is minimal in our setting, consistent with the limited staggering (three cohorts over two quarters). Standard errors are clustered at the state level with 51 clusters, following \citet{camerongelbachmiller2008}.


\section{Results}

\subsection{Main Results}

Table \ref{tab:tab:main_by_spec} presents the main results from estimating Equation (\ref{eq:main}) separately by specialty. The coefficient on unwinding intensity represents the effect of a one-unit (100 percentage point) increase in the state's net disenrollment rate on log provider counts. To interpret in policy-relevant units, we note that the interquartile range of state disenrollment rates is approximately 10 to 22 percentage points.

\begin{table}
\centering
\caption{\label{tab:tab:main_by_spec}Effect of Medicaid Unwinding on Provider Supply by Specialty}
\centering
\begin{threeparttable}
\begin{tabular}[t]{lrrcr}
\toprule
Specialty & Estimate & SE & Stars & N\\
\midrule
\textit{Pooled (all specialties)} & 0.0128 & (0.2376) &  & 501,228\\
Behavioral Health & 0.1104 & (0.3923) &  & 71,604\\
Dental & 0.5032 & (0.8312) &  & 71,604\\
NP/PA & -0.4353 & (0.2715) &  & 71,604\\
OB-GYN & 0.1101 & (0.0948) &  & 71,604\\
\addlinespace
Primary Care & -0.1424 & (0.2620) &  & 71,604\\
Psychiatry & 0.0329 & (0.0809) &  & 71,604\\
Surgery & -0.0897 & (0.0705) &  & 71,604\\
\bottomrule
\end{tabular}
\begin{tablenotes}
\item \textit{Note: }
\item Dependent variable: $\log$(active providers + 1). Treatment: post-unwinding $\times$ net disenrollment rate. All models include county$\times$specialty and quarter fixed effects. Standard errors clustered at the state level (51 clusters). * p$<$0.10, ** p$<$0.05, *** p$<$0.01.
\end{tablenotes}
\end{threeparttable}
\end{table}

The central finding is a precisely estimated null. We find no evidence that providers exit when enrollment drops. The pooled TWFE estimate across all specialties (first row of Table \ref{tab:tab:main_by_spec}; also reported in Table \ref{tab:tab:robustness}) is 0.0128 (SE $= 0.2376$, $p = 0.957$), indicating no detectable relationship between unwinding intensity and active Medicaid provider counts. To put the precision in context: the 95\% confidence interval of $[-0.46, 0.49]$ allows us to rule out effects larger than a 4.9\% change in provider counts in response to a 10 percentage point increase in disenrollment. The null is not an artifact of aggregation --- it holds within every individual specialty.

To assess the informativeness of our null, we compute minimum detectable effects (MDE). With a pooled standard error of 0.2376 and 51 state clusters, the MDE at 80\% power (two-sided, $\alpha = 0.05$) is $2.8 \times 0.2376 \approx 0.665$ log points, corresponding to approximately a 94\% change in provider counts. At the specialty level, standard errors range from 0.0705 (Surgery) to 0.8312 (Dental), implying MDEs from 0.20 to 2.33 log points. Our design is thus well-powered to detect large supply responses---the kind that would indicate a policy emergency---but cannot rule out modest adjustments of 10--30\%.

Because our setting involves staggered adoption across three cohorts, we also report the \citet{sunab2021} interaction-weighted estimator as our preferred specification robust to heterogeneous treatment effects. The Sun-Abraham aggregated ATT is $-0.054$ (SE $= 0.067$, $p = 0.424$). While the point estimate is slightly negative, it remains far from statistical significance and is substantively small --- a 10 percentage point increase in disenrollment would imply only a 0.5\% decline in provider counts under this estimator. The concordance of both the TWFE and Sun-Abraham estimates in the neighborhood of zero provides strong evidence that the null is not an artifact of aggregation bias in the staggered design.

No specialty shows a statistically significant response at conventional levels. The point estimates are mixed in sign and uniformly small relative to their standard errors. NP/PA supply shows the largest point estimate in absolute terms ($-0.4353$, SE $= 0.2715$, $p = 0.115$), with a suggestive negative direction, but this is not robust to multiple testing corrections. Primary care ($-0.142$, $p = 0.589$), surgery ($-0.090$, $p = 0.209$), and psychiatry ($0.033$, $p = 0.686$) show no meaningful response. Dental ($0.503$, SE $= 0.831$, $p = 0.548$) and OB-GYN ($0.110$, $p = 0.251$) have positive point estimates, the opposite of the hypothesized direction. The large standard error for dental reflects substantial cross-state variation in Medicaid dental benefit coverage, which generates heterogeneous provider responses that inflate the residual variance. The 95\% confidence intervals for all specialties include zero, confirming the null across the full range of provider types.

The desert indicator specification yields the same conclusion. The effect of unwinding intensity on the probability that a county-specialty cell is in desert status is 0.016 (SE $= 0.044$, $p = 0.722$) --- the enrollment shock does not push counties across the desert threshold.

\subsection{Event Study Evidence}

Figure \ref{fig:eventstudy} presents the event study estimates from Equation (\ref{eq:eventstudy}). The figure provides our key evidence for the parallel trends assumption and the dynamics of the provider response.

\begin{figure}[H]
\centering
\includegraphics[width=\textwidth]{figures/fig4_event_study.pdf}
\caption{Event Study: Medicaid Unwinding and Provider Supply}
\label{fig:eventstudy}
\begin{minipage}{\textwidth}
\vspace{0.3em}
\footnotesize\textit{Notes:} Coefficients from Equation (\ref{eq:eventstudy}). Each point represents the interaction of quarters relative to unwinding start $\times$ state disenrollment intensity. The omitted period is $k = -1$. Shaded area is 95\% confidence interval with state-clustered standard errors.
\end{minipage}
\end{figure}

The pre-trend coefficients are reassuring. For all eight pre-treatment quarters ($k = -8$ through $k = -2$), point estimates cluster near zero and none is statistically distinguishable from it at conventional levels. This provides strong support for the identifying assumption that provider trends were parallel across states with different eventual unwinding intensity, conditional on our fixed effects.

The post-treatment coefficients tell an equally clear story: the null persists dynamically. While the point estimates show a slight negative drift in quarters 4 and 5 post-unwinding (coefficients of approximately $-0.11$ and $-0.10$, respectively), none achieves statistical significance, and the magnitudes remain small.\footnote{Composition changes at longer event-time horizons warrant caution. Because states entered the unwinding in three cohorts (40 states in 2023Q2, 10 in 2023Q3, 1 in 2023Q4), the set of states contributing to each event-time coefficient narrows as $k$ increases. At $k \geq 4$, estimates are identified primarily by the earliest cohort (2023Q2 states), and at $k = 5$ only this cohort contributes. The slight negative drift at longer horizons could therefore reflect cohort-specific effects rather than a general dynamic pattern.} The gradual drift could reflect either a lagged supply response beginning to emerge or simply sampling variation. Distinguishing between these interpretations will require longer post-treatment data as additional T-MSIS quarters become available.

\subsection{Urban vs.\ Rural Heterogeneity}

We test whether the null result masks heterogeneous effects across urban and rural areas by estimating Equation (\ref{eq:main}) separately. If the ``vicious cycle'' hypothesis operates anywhere, it should be in rural areas where provider networks are already thin.

Both subsamples yield null results. The urban coefficient is 0.067 (SE $= 0.379$) and the rural coefficient is $-0.029$ (SE $= 0.146$). Neither approaches conventional significance thresholds. The rural point estimate is slightly negative but far too imprecise to draw conclusions. The null is not an artifact of averaging across geographies.


\section{Robustness}

We conduct an extensive battery of robustness checks. Table \ref{tab:tab:robustness} summarizes the key specifications. The null result --- that unwinding intensity has no detectable effect on active Medicaid provider counts --- is robust across all alternatives.

\begin{table}
\centering
\caption{\label{tab:tab:robustness}Robustness Checks}
\centering
\begin{threeparttable}
\begin{tabular}[t]{llll}
\toprule
Specification & Estimate & SE & N\\
\midrule
Main ($\geq$4 claims/qtr) & 0.0128 & (0.2376) & 501,228\\
Tight threshold ($\geq$12 claims/qtr) & 0.0128 & (0.2376) & 501,228\\
Binary treatment (early vs. late) & 0.0424 & (0.0457) & 501,228\\
With Medicaid share $\times$ time & 0.0160 & (0.2382) & 501,228\\
Placebo (fake 2021Q2 treatment) & -0.0612 & (0.5701) & 394,758\\
\bottomrule
\end{tabular}
\begin{tablenotes}
\item \textit{Note: }
\item All models include county$\times$specialty and quarter fixed effects with state-clustered standard errors (51 clusters). Randomization inference (500 permutations of treatment timing across states) yields p = 0.962 for the main specification.
\end{tablenotes}
\end{threeparttable}
\end{table}

\subsection{Alternative Active-Provider Thresholds}

Our baseline defines active providers as those billing $\geq$4 claims per quarter. Using a tighter threshold of $\geq$12 claims per quarter yields an identical coefficient of 0.0128 (SE $= 0.2376$). This equivalence reflects a feature of T-MSIS data: cell suppression censors provider-procedure-month records with fewer than 12 claims, so the minimum observed quarterly claim count is 12. Consequently, every provider in our data exceeds both thresholds, and the variation in active-provider definitions operates above this floor. The key implication is that our provider counts capture only those with non-trivial Medicaid billing volume; very-low-volume providers (below 12 claims per record) are absent from the data by construction.

\subsection{Binary Treatment}

Replacing our continuous intensity measure with a binary indicator comparing early unwinders (states beginning in 2023Q2) to late unwinders yields a coefficient of 0.042 (SE $= 0.046$). While more precisely estimated due to the binary specification, the point estimate is small and positive --- the wrong sign for the exit hypothesis --- and not statistically significant.

\subsection{Additional Controls}

Adding the interaction of each state's pre-unwinding Medicaid enrollment share with quarter indicators absorbs state-level trends correlated with baseline Medicaid dependence. The coefficient on unwinding intensity is 0.0160 (SE $= 0.2382$), essentially unchanged from the baseline.

\subsection{Sun-Abraham Decomposition}

As reported in Section 6.1, we implement the \citet{sunab2021} interaction-weighted estimator to address concerns about heterogeneous treatment effects in staggered designs. The Sun-Abraham aggregated ATT of $-0.054$ (SE $= 0.067$, $p = 0.424$) confirms the null from the TWFE specification. This estimator is our preferred robustness check for staggered adoption because it avoids the ``negative weights'' problem that can afflict conventional TWFE when treatment effects vary across cohorts or over time. The concordance of TWFE and Sun-Abraham estimates provides confidence that our null result is not driven by aggregation bias.

\subsection{Placebo Test}

We assign a fake unwinding date of 2021Q2 --- the midpoint of the continuous enrollment period --- and estimate the same specification using only the pre-unwinding sample. The placebo coefficient is $-0.0612$ (SE $= 0.5701$, $p = 0.915$), providing reassurance that the null result is not masking differential pre-trends that happen to correlate with state characteristics.

\subsection{Permutation Inference}

Given that our treatment varies at the state level with 51 clusters, we complement standard asymptotic inference with permutation inference. We randomly reassign state unwinding intensities 500 times, re-estimate the main specification each time, and compare the observed coefficient to the permutation distribution.

\begin{figure}[H]
\centering
\includegraphics[width=0.85\textwidth]{figures/fig6_ri_distribution.pdf}
\caption{Permutation Inference: Distribution of Placebo Coefficients}
\label{fig:ri}
\begin{minipage}{0.85\textwidth}
\vspace{0.3em}
\footnotesize\textit{Notes:} Distribution of coefficients from 500 random permutations of state unwinding intensities. Red line indicates the observed coefficient. RI $p$-value is the fraction of permutation coefficients at least as extreme as the observed.
\end{minipage}
\end{figure}

Figure \ref{fig:ri} plots the permutation distribution. The observed coefficient of 0.0128 falls squarely in the center of the null distribution, yielding a permutation $p$-value of 0.962. The 95\% randomization inference confidence interval is $[-0.558, 0.555]$. This provides non-parametric confirmation that our observed point estimate is entirely consistent with no true effect.

\subsection{NP/PA Substitution}

The NP/PA coefficient of $-0.435$ (SE $= 0.272$, $p = 0.115$) is the only estimate that approaches marginal significance. If taken at face value, it would suggest that the unwinding modestly reduced NP/PA supply. However, this result is not robust to multiple testing corrections across the seven specialties, and the MD-only primary care estimate ($0.006$, SE $= 0.343$) shows no response, suggesting the NP/PA finding may reflect sampling variation rather than a true demand channel.


\section{Discussion}

\subsection{Interpreting the Null}

The null causal finding is the central result requiring interpretation. The most intuitive hypothesis --- that losing 25 million patients should cause Medicaid providers to exit --- does not find support in our data. Several explanations are consistent with this null, and distinguishing among them has direct policy implications.

\textit{Diluted demand shock.} Medicaid typically constitutes a minority of a provider's total payer mix. Even in states with high Medicaid penetration, the median physician derives roughly 20--30\% of revenue from Medicaid. A 15\% decline in state Medicaid enrollment therefore translates to approximately a 3--5\% decline in a provider's total patient volume. This dilution may push the effective demand shock below the threshold that triggers participation decisions, particularly given fixed costs of practice that make exit ``lumpy'' rather than continuous.

\textit{Long-run structural determinants.} Provider location and payer-mix decisions are plausibly driven by long-run factors --- reimbursement rates, geographic preferences, spousal employment, children's schooling, malpractice environment --- rather than short-run fluctuations in the Medicaid patient base. The secular declines we document in Section 4 (21\% decline in psychiatrists, 28\% in OB-GYNs) dwarf anything the unwinding could plausibly produce, suggesting that the fundamental drivers of Medicaid provider supply operate on a much longer time horizon.

\textit{Incomplete adjustment.} Our post-treatment window spans only five to six quarters. Provider exit may require time to materialize: lease expirations, partnership dissolutions, and credentialing at new locations all introduce adjustment lags. The slight negative drift in our event-study coefficients in quarters 4--5 is consistent with a slow adjustment process that has not yet reached significance. Future data releases will be critical for assessing this explanation.

\textit{Composition effects.} The 25 million disenrolled individuals were disproportionately young, healthy adults who had remained enrolled during the pandemic but would have churned off rolls under normal circumstances. These enrollees may have been low utilizers who generated few provider visits, meaning that their disenrollment reduced coverage counts without meaningfully reducing the demand for provider services.

\subsection{Reconciling with the Descriptive Crisis}

The juxtaposition of severe descriptive trends (Section 4) with null causal effects (Section 6) is the paper's defining tension, and resolving it yields the key policy insight. The Medicaid provider crisis is real --- psychiatrists billing Medicaid have declined by over 20\%, OB-GYNs by 28\%, and vast rural regions have zero specialty coverage. But this crisis predates the unwinding, continued through the coverage expansion, and shows no visible acceleration afterward. The crisis appears structural, plausibly driven by chronically low Medicaid reimbursement rates, an aging physician workforce, and the economics of rural practice that may make Medicaid participation unsustainable for marginal providers regardless of enrollment levels.

This interpretation is consistent with the broader health economics literature. \citet{decker2012} and \citet{zuckerman2009} have long argued that reimbursement rates are the primary determinant of Medicaid provider participation. \citet{buchmueller2011} showed that Medicaid generosity affects health care labor supply. This inelasticity is consistent with \citet{clemensgottlieb2014}, who find that physician behavior responds primarily to reimbursement prices rather than volume changes, and with the general equilibrium insights of \citet{finkelstein2007} on insurance market expansions. Our evidence complements these findings by showing that the demand side --- in the form of short-run enrollment shocks --- is \textit{not} a first-order determinant of participation, at least on the margins observed in the data. Our findings are consistent with the hypothesis that Medicaid provider supply responds primarily to structural factors such as reimbursement rates rather than short-run enrollment fluctuations.

\subsection{Geographic Implications}

The atlas reveals that the geography of Medicaid deserts is largely determined by long-standing structural factors rather than recent enrollment dynamics. The Great Plains psychiatry desert, for instance, has persisted throughout our 2018--2024 panel with no visible response to the pandemic-era enrollment surge (which should have attracted providers if the demand channel were operative) or the subsequent unwinding (which should have pushed them away). This stability underscores that these deserts reflect fundamental economic geography: the combination of low population density, high practice costs, and low Medicaid reimbursement creates conditions in which no enrollment level is sufficient to sustain specialty practice.

\subsection{Policy Implications}

The findings carry several policy implications, some reassuring and others sobering.

The reassuring implication is that the unwinding did not create a provider crisis on top of the coverage crisis. Policymakers concerned about cascading supply-side effects of the unwinding can take some comfort that the provider network --- already badly depleted --- has not visibly deteriorated further in response to the enrollment shock. Temporary coverage disruptions, while harmful to the disenrolled, do not appear to permanently damage the remaining provider infrastructure.

The sobering implication is that the provider crisis appears structural and is unlikely to be fixed by enrollment policy alone. Re-enrolling every disenrolled beneficiary would restore demand but would likely not reverse the long-run provider declines documented in our atlas. The providers who left Medicaid between 2018 and 2023 plausibly did so for reasons orthogonal to enrollment levels --- retirement, reimbursement inadequacy, geographic relocation --- and may not return simply because the patient base is restored.

Several policy directions follow from these findings. First, targeted reimbursement rate increases in shortage specialties --- particularly psychiatry, OB-GYN, and rural primary care --- represent a promising policy lever for rebuilding the Medicaid provider network. The existing literature suggests that provider supply is plausibly more elastic to reimbursement than to short-run demand \citep{decker2012,polsky2015,clemensgottlieb2014}, implying that price-based interventions may be more effective than quantity-based interventions. Second, the claims-based desert atlas developed here provides a superior targeting tool. HPSA designations count all licensed physicians regardless of Medicaid participation and substantially understate the Medicaid-specific shortage. Incorporating T-MSIS billing data into shortage designations would better direct federal resources to areas of genuine Medicaid provider absence. Third, scope-of-practice reforms that allow NPs and PAs to practice independently remain necessary complements, particularly in specialties where NP substitution is feasible (primary care, behavioral health) --- though our data confirm the limitations of this approach for psychiatry and OB-GYN.

\subsection{Unwinding Intensity Variation}

Figure \ref{fig:unwinding_map} maps the state-level variation in unwinding intensity that identifies our causal estimates. The geographic pattern of disenrollment does not align neatly with any single state characteristic, providing variation that is plausibly exogenous to county-level provider trends conditional on our fixed effects.

\begin{figure}[H]
\centering
\includegraphics[width=0.85\textwidth]{figures/fig7_unwinding_map.pdf}
\caption{Medicaid Unwinding Intensity by State}
\label{fig:unwinding_map}
\begin{minipage}{0.85\textwidth}
\vspace{0.3em}
\footnotesize\textit{Notes:} Net enrollment decline (\%) from peak enrollment (typically March 2023) to most recent available data. Darker shading indicates larger enrollment losses.
\end{minipage}
\end{figure}

\subsection{Limitations}

Several limitations should be noted. First, T-MSIS captures billing transactions, not all clinical encounters. Providers who treat Medicaid patients through arrangements that do not generate individual claims (e.g., block-grant-funded community health centers with alternative payment models) may be undercounted. Second, our geographic assignment relies on NPPES practice-location addresses, which may not reflect where providers actually see patients if they practice at satellite clinics or through telehealth. Third, the unwinding timing is concentrated in a three-quarter window, and our post-period spans only six quarters. The null result could reflect insufficient time for supply-side adjustment rather than true inelasticity; longer follow-up will be essential. Fourth, we cannot fully rule out that states with more aggressive unwinding also experienced other contemporaneous shocks to their provider markets, though our placebo tests and permutation inference provide reassurance.

Fifth, T-MSIS cell suppression may introduce measurement error that attenuates our estimates toward zero. While the provider-quarter aggregation mitigates this concern, low-volume providers near the suppression threshold are systematically undercounted. If these marginal providers are precisely those most responsive to enrollment shocks, our null estimate may understate the true supply elasticity. Future work using unsuppressed research files or alternative measures such as total Medicaid payments could address this limitation.

Sixth, we use billing NPIs as the unit of analysis, which may conflate individual practitioners with organizational entities. While individual (Type 1) NPIs comprise the majority of our sample, organizational billing patterns could introduce noise. Validation against servicing NPIs, where available in T-MSIS, would strengthen the measurement foundation.


\section{Conclusion}

This paper makes two contributions. The first is a comprehensive claims-based atlas of Medicaid provider deserts, the first of its kind. Using 227 million T-MSIS records linked to provider registries, we document a severe and ongoing crisis in Medicaid provider access. Psychiatrists billing Medicaid declined by roughly 21\% between 2018 and 2024, OB-GYNs by 28\%, and primary care physicians by 16\%. Rural counties experience desert rates two to three times higher than urban counties, with the majority of rural county-quarters having zero active Medicaid psychiatrists. Nurse practitioners partially offset these losses, but the substitution is incomplete for specialty care.

The second contribution is causal --- and its headline is a precisely estimated null. Exploiting staggered state-level variation in the timing and intensity of Medicaid enrollment unwinding, we find no detectable effect of the demand shock on provider supply. The pooled estimate is 0.0128 ($p = 0.957$); no individual specialty reaches significance; event-study evidence shows flat pre-trends and no meaningful post-treatment departure; permutation inference ($p = 0.962$) confirms the null. Provider supply is remarkably sticky in the face of a large enrollment shock.

The juxtaposition of a severe descriptive crisis with a null causal effect is the paper's core insight. Medicaid provider deserts are real, extensive, and worsening --- but they do not appear to be driven by enrollment fluctuations. The deserts are consistent with chronic structural factors: low reimbursement rates that have persisted for decades, an aging physician workforce concentrated in urban areas, and the fixed-cost economics of rural practice that may make Medicaid participation unsustainable at any enrollment level. The unwinding, despite removing 25 million people from Medicaid rolls, was too small a demand shock relative to these structural forces to move the provider supply needle.

This finding reframes the policy response. Enrollment policy --- important as it is for coverage --- cannot fix provider access. The 70 million people who remain enrolled in Medicaid face a provider crisis that predates the unwinding, persisted through the coverage expansion, and will not be resolved by restoring the coverage losses. Addressing Medicaid deserts requires direct investment in the supply side: targeted reimbursement increases in shortage specialties, workforce development in rural and underserved areas, and scope-of-practice reforms that expand the pool of providers able to serve Medicaid patients independently. The claims-based atlas developed here provides the foundation for targeting those investments --- revealing, for the first time at scale, where Medicaid patients actually have access to care and where they do not.

Future research should pursue two directions. First, as additional T-MSIS quarters become available, re-estimating the causal effect with a longer post-period will be essential for distinguishing true supply inelasticity from incomplete adjustment. The slight negative drift we observe in late post-treatment quarters warrants monitoring. Second, linking the provider deserts documented here to beneficiary health outcomes --- utilization, emergency department reliance, delayed diagnoses --- would quantify the welfare cost of the access crisis our atlas reveals. The T-MSIS data, now public, make both research agendas feasible for the first time.


\section*{Acknowledgements}

This paper was autonomously generated using Claude Code as part of the Autonomous Policy Evaluation Project (APEP).

\noindent\textbf{Project Repository:} \url{https://github.com/SocialCatalystLab/ape-papers}

\noindent\textbf{Contributors:} @SocialCatalystLab

\noindent\textbf{First Contributor:} \url{https://github.com/SocialCatalystLab}

\label{apep_main_text_end}
\newpage
\bibliography{references}

\newpage
\appendix

\section{Data Appendix}
\label{app:data}

\subsection{T-MSIS Provider Spending Data Schema}

The T-MSIS provider spending file contains the following fields used in this analysis:

\begin{itemize}
\item \texttt{BILLING\_PROVIDER\_NPI\_NUM}: National Provider Identifier of the billing entity.
\item \texttt{SERVICING\_PROVIDER\_NPI\_NUM}: NPI of the individual clinician who delivered the service.
\item \texttt{HCPCS\_CD}: Healthcare Common Procedure Coding System code identifying the service rendered.
\item \texttt{CLAIM\_FROM\_MONTH}: Year-month of service (YYYY-MM format).
\item \texttt{TOTAL\_CLAIMS}: Number of claims for this billing-servicing-HCPCS-month combination.
\item \texttt{TOTAL\_PAID}: Total Medicaid payment amount.
\item \texttt{TOTAL\_UNIQUE\_BENEFICIARIES}: Number of unique Medicaid beneficiaries served.
\end{itemize}

Cell suppression applies to records with fewer than 12 claims, meaning that very low-volume provider-procedure combinations are censored. This suppression is conservative for our purposes: it affects the least active providers, who fall below our active-provider threshold regardless.

\subsection{NPPES Taxonomy Code Classification}

We classify providers into seven specialty groups using the National Uniform Claim Committee (NUCC) Health Care Provider Taxonomy code system. Our classification is detailed in Table \ref{tab:taxonomy}.

\begin{table}[H]
\centering
\small
\caption{Specialty Classification from NUCC Taxonomy Codes}
\label{tab:taxonomy}
\begin{threeparttable}
\begin{tabular}{p{3.5cm}p{9cm}}
\toprule
Specialty Group & NUCC Taxonomy Codes Included \\
\midrule
Primary Care & 207Q (Family Medicine), 207R (Internal Medicine), 208D (General Practice), 2080 (Pediatrics) --- MD/DO only \\
NP/PA & 363L (Nurse Practitioner, all subspecialties), 363A (Physician Assistant, all subspecialties) \\
Psychiatry & 2084 (Psychiatry and subspecialties) --- MD/DO only \\
Behavioral Health & 101Y (Counselor), 103T/103K (Psychologist), 1041 (Social Worker), 106H (Marriage \& Family Therapist) \\
Dental & 1223 (Dentist, all subspecialties) \\
OB-GYN & 207V (Obstetrics \& Gynecology and subspecialties) \\
Surgery & 2086 (Surgery), 207X (Orthopedic), 207T (Neurological), 208200 (Plastic), 208C (Colon \& Rectal), 208G (Thoracic), 204E (Maxillofacial) \\
\bottomrule
\end{tabular}
\begin{tablenotes}[flushleft]
\small
\item \textit{Notes:} Taxonomy codes from the NUCC Health Care Provider Taxonomy Code Set. Providers with multiple taxonomy codes assigned to primary (first-listed) specialty. Unclassified providers excluded.
\end{tablenotes}
\end{threeparttable}
\end{table}

\subsection{State Unwinding Cohorts and Disenrollment Rates}

States are grouped into unwinding cohorts by start quarter. Table \ref{tab:state_unwinding} summarizes the cross-state variation.

\begin{table}[H]
\centering
\small
\caption{State-Level Unwinding Characteristics}
\label{tab:state_unwinding}
\begin{threeparttable}
\begin{tabular}{lrrrr}
\toprule
Statistic & Mean & SD & Min & Max \\
\midrule
Net disenrollment rate (\%) & 15.6 & 6.8 & 1.4 & 30.2 \\
\midrule
\multicolumn{5}{l}{\textit{Cohort distribution:}} \\
\quad 2023Q2 (April--June) & \multicolumn{4}{l}{40 states} \\
\quad 2023Q3 (July--September) & \multicolumn{4}{l}{10 states} \\
\quad 2023Q4 (October) & \multicolumn{4}{l}{1 state (Oregon)} \\
\bottomrule
\end{tabular}
\begin{tablenotes}[flushleft]
\small
\item \textit{Notes:} $N = 51$ (50 states plus DC). Net disenrollment rate = (peak enrollment $-$ trough) / peak $\times$ 100. Sources: KFF Medicaid Enrollment and Unwinding Tracker; CMS monthly enrollment reports.
\end{tablenotes}
\end{threeparttable}
\end{table}

\section{Robustness Appendix}
\label{app:robustness}

\subsection{Specialty-Specific Event Studies}

We estimate separate event studies for each specialty. Pre-trend tests pass for all seven specialties. Post-treatment dynamics vary modestly in magnitude but no specialty shows a statistically significant response. NP/PA shows the most negative trajectory, consistent with the marginally suggestive pooled estimate, while dental and OB-GYN show slightly positive trajectories.

\subsection{Alternative Clustering}

Our baseline clusters standard errors at the state level (51 clusters). We verify robustness to alternative clustering levels: (a) state $\times$ specialty (351 clusters) yields slightly smaller standard errors without changing the null conclusion; (b) two-way clustering by state and quarter yields virtually identical inference. With any clustering scheme, no specification achieves statistical significance.

\subsection{NP/PA Substitution Detail}

Among primary care providers specifically, we separate MD/DO physicians from NP/PA providers. The MD-specific coefficient on unwinding intensity is 0.006 (SE $= 0.343$), indistinguishable from zero. The NP/PA-specific coefficient is $-0.435$ (SE $= 0.272$, $p = 0.115$). The suggestive NP/PA result could reflect either a true demand channel specific to NPs (who may be more dependent on Medicaid revenue) or sampling variation. With only one of fourteen specialty-specific estimates approaching marginal significance, we interpret this cautiously as a potential avenue for future investigation rather than a robust finding.


\end{document}
