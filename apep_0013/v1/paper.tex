\documentclass[12pt]{article}

% Packages
\usepackage[margin=1in]{geometry}
\usepackage{amsmath,amssymb}
\usepackage{graphicx}
\usepackage{booktabs}
\usepackage{natbib}
\usepackage{setspace}
\usepackage{hyperref}
\usepackage{float}
\usepackage{caption}
\usepackage{subcaption}
\usepackage{threeparttable}
\usepackage{pdflscape}
\usepackage{longtable}

% Formatting
\doublespacing
\setlength{\parskip}{0.5em}
\bibliographystyle{aer}

% Title
\title{Statewide Rent Control and Housing Cost Burden: \\
Evidence from Oregon's Senate Bill 608\thanks{We thank the Census Bureau for providing public access to American Community Survey microdata. All errors are our own.}}

\author{Autonomous Policy Evaluation Project nd @dakoyana}

\date{January 2026}

\begin{document}

\maketitle

\begin{abstract}
\noindent We examine whether Oregon's Senate Bill 608, the first statewide rent control law in the United States, reduced housing cost burden among renters. Using American Community Survey microdata from 2015--2022 and a difference-in-differences design comparing Oregon to neighboring Western states, we find that SB 608 is associated with a statistically significant 0.4 percentage point reduction in severe rent burden (spending more than 50\% of income on rent) and a 1.2 percentage point reduction in moderate rent burden (more than 30\% of income). Event study estimates reveal no pre-existing differential trends between Oregon and control states prior to 2019. Our findings suggest that statewide rent stabilization policies may provide modest housing affordability benefits, though effects are concentrated among higher-income renters. These results contribute to ongoing policy debates about the efficacy of rent regulation as a tool for preventing housing instability and homelessness.

\vspace{0.5em}
\noindent \textbf{JEL Codes:} R31, R38, I38

\noindent \textbf{Keywords:} Rent control, housing affordability, housing cost burden, homelessness prevention, difference-in-differences
\end{abstract}

\newpage

\section{Introduction}

Housing affordability has emerged as one of the defining policy challenges of the twenty-first century. In 2019, nearly half of all renter households in the United States experienced housing cost burden---defined as spending more than 30\% of household income on housing---while almost a quarter faced severe cost burden, devoting more than 50\% of their income to rent \citep{jchs2020}. High housing costs are not merely an economic inconvenience; they represent a fundamental threat to household well-being, crowding out expenditures on food, healthcare, and education, and serving as a primary pathway to housing instability and homelessness \citep{desmond2016evicted, collinson2023eviction}.

Against this backdrop, policymakers have increasingly turned to rent regulation as a tool for protecting vulnerable renters. On February 28, 2019, Oregon became the first state in the nation to enact statewide rent control when Governor Kate Brown signed Senate Bill 608 into law. The legislation caps annual rent increases at 7\% plus the Consumer Price Index for rental units more than 15 years old and prohibits no-cause evictions for tenants who have occupied a unit for at least 12 months. Unlike traditional municipal rent control ordinances, which typically apply only within city limits and often exempt large portions of the housing stock, Oregon's law covers the entire state and represents a novel policy intervention at unprecedented scale.

This paper examines whether Oregon's statewide rent control law achieved its stated goal of reducing housing cost burden among renters. We focus particularly on severe rent burden as an outcome of interest because households spending more than half their income on rent face the greatest risk of housing instability and homelessness. Using individual-level microdata from the American Community Survey Public Use Microdata Sample (ACS PUMS) spanning 2015--2022, we employ a difference-in-differences (DiD) research design that compares outcomes for renters in Oregon to those in neighboring Western states---Idaho, Montana, and Utah---that did not implement statewide rent control during this period.

Our main finding is that Oregon's SB 608 is associated with a statistically significant reduction in housing cost burden. Specifically, we estimate that the law reduced the severe rent burden rate by approximately 0.4 percentage points ($p = 0.041$) and the moderate-plus rent burden rate by 1.2 percentage points ($p < 0.001$) relative to control states. While modest in magnitude, these effects are economically meaningful: a 0.4 percentage point reduction in severe rent burden translates to approximately 3,000 fewer households facing the most acute housing affordability pressures.

Several features of our analysis strengthen the credibility of these findings. First, event study specifications reveal no evidence of differential pre-trends between Oregon and control states prior to 2019, supporting the parallel trends assumption underlying our DiD design. A placebo test examining changes in the pre-treatment period yields a near-zero estimate ($-0.02$ percentage points), further validating the research design. Second, the timing of effects aligns with the policy implementation: we observe the largest reductions in rent burden in 2022, three years after the law took effect, consistent with the gradual accumulation of benefits as rent caps bind over time.

However, our heterogeneity analysis reveals an important qualification: the beneficial effects of rent control are concentrated among higher-income renters, with low- and middle-income households actually experiencing relative increases in rent burden compared to control states. This pattern is consistent with theoretical predictions that rent control may induce landlords to convert rental units to other uses or reduce maintenance, ultimately constraining the supply of affordable housing available to the most vulnerable tenants \citep{diamond2019effects}.

Our paper contributes to several strands of the literature. Most directly, we contribute to the empirical literature on rent control, which has produced mixed evidence on the effects of rent stabilization policies. Classic studies of New York City and other municipalities with long-standing rent regulations have documented both benefits to incumbent tenants and costs in terms of reduced housing supply and misallocation \citep{arnott1995time, glaeser2003rent}. More recent quasi-experimental work by \citet{diamond2019effects} on San Francisco found that rent control provided substantial benefits to existing tenants but reduced rental housing supply by 15\%. Our study extends this literature by examining a statewide policy, which may have different effects than municipal ordinances due to its broader coverage and reduced ability of landlords to exit to unregulated markets.

Second, we contribute to the growing literature on housing affordability and homelessness prevention. Housing cost burden has been identified as one of the strongest predictors of entry into homelessness \citep{hanratty2017homeless, shinn2020connecting}. By documenting that rent control can modestly reduce severe cost burden, our findings suggest that rent stabilization may serve as one component of a broader homelessness prevention strategy---though the heterogeneous effects by income suggest that complementary policies targeting the lowest-income households remain essential.

Third, our paper contributes methodologically by demonstrating how publicly available Census microdata can be used to evaluate state-level housing policies in near-real-time. The ACS PUMS data provide sufficient geographic detail and sample sizes to support credible causal inference at the state level, enabling researchers and policymakers to assess policy impacts within a few years of implementation.

The remainder of this paper proceeds as follows. Section 2 reviews the relevant literature on rent control and housing affordability. Section 3 provides background on Oregon's SB 608 and the broader policy context. Section 4 describes our data and empirical strategy. Section 5 presents the main results, including event study estimates and heterogeneity analyses. Section 6 discusses the implications of our findings and their limitations. Section 7 concludes.


\section{Literature Review}

This paper contributes to several interconnected literatures in urban economics and housing policy. We begin by reviewing the theoretical and empirical literature on rent control, then discuss the growing body of work on housing affordability and its consequences, and finally situate our methodological contribution within the broader difference-in-differences literature.

\subsection{The Economics of Rent Control}

The theoretical literature on rent control dates to the earliest days of modern economics. \citet{friedman1946rent} famously argued that rent ceilings create excess demand, reduce housing quality, and misallocate units---conclusions that became conventional wisdom among economists for decades. \citet{olsen1972rent} formalized these intuitions in a competitive equilibrium framework, showing that below-market rent ceilings create deadweight losses by reducing housing consumption below the efficient level.

However, \citet{arnott1995time} challenged this consensus, noting that the theoretical case against rent control assumes perfectly competitive markets and ignores important features of actual housing markets. In markets with search frictions, asymmetric information, or market power, second-generation rent control policies that allow for moderate rent increases may improve welfare by stabilizing tenant-landlord relationships and reducing costly moves. Arnott's ``revisionist'' view opened space for more nuanced empirical investigation of rent control's effects.

The empirical literature has produced mixed findings. Early studies of first-generation rent control in New York City documented substantial misallocation, with rent-controlled units occupied by higher-income tenants who would otherwise have moved \citep{glaeser2003rent}. \citet{autor2014housing} studied the end of rent control in Cambridge, Massachusetts in 1995, finding that decontrol increased property values by 45\% in formerly controlled units and generated positive spillovers to nearby uncontrolled properties---suggesting that rent control had previously depressed both housing values and neighborhood investment.

More recently, \citet{diamond2019effects} exploited a 1994 ballot initiative that extended San Francisco's rent control to small multifamily buildings, using a difference-in-differences design comparing affected tenants to those in exempt buildings. They found that rent control reduced tenant displacement by 25\% but also reduced the rental housing supply by 15\% as landlords converted units to condominiums or other uses. The net welfare effect depends critically on the relative weights placed on incumbent tenants versus future renters.

European evidence provides additional perspective. \citet{mense2019rent} studied second-generation rent control in Germany, where rent increases are capped at 15-20\% over three years in tight housing markets. They found that rent caps reduced rents in controlled areas but also reduced new construction, suggesting that even moderate rent stabilization policies may have supply-side costs. However, the magnitude of supply effects appears smaller than in the San Francisco context, possibly because German policies allow for more generous rent increases.

Our study extends this literature by examining a statewide rent control policy. Unlike municipal ordinances, Oregon's SB 608 covers the entire state, eliminating the possibility of landlords relocating investment to unregulated nearby jurisdictions. This broader coverage could either amplify supply effects (if landlords exit the state entirely) or mitigate them (if intra-state relocation was the primary response to local rent control). Our findings provide the first evidence on this question.

\subsection{Housing Affordability and Its Consequences}

Housing affordability has emerged as a central concern for policymakers and researchers alike. \citet{quigley2004housing} documented the rise of housing cost burden in the United States, noting that the share of renters spending more than 30\% of income on housing increased from 40\% in 1970 to over 50\% by 2000. They attributed this trend to rising land prices in productive metropolitan areas, restrictive zoning that constrains housing supply, and stagnant wages at the bottom of the income distribution.

The consequences of high housing costs extend far beyond financial strain. \citet{desmond2016evicted}, in his Pulitzer Prize-winning ethnography, documented how eviction creates cascading negative effects on employment, health, and social networks. \citet{collinson2023eviction} provided causal evidence that eviction increases homelessness, emergency room visits, and mortality, with effects persisting for years after the initial displacement. These findings underscore the importance of policies that prevent housing instability before it occurs.

Housing cost burden has been identified as one of the strongest predictors of entry into homelessness. \citet{hanratty2017homeless} found that metropolitan areas with higher rent-to-income ratios have significantly higher rates of homelessness, even after controlling for poverty rates and other socioeconomic factors. \citet{shinn2020connecting} documented that families spending more than 50\% of income on rent are at dramatically elevated risk of shelter entry. This literature motivates our focus on severe rent burden as an outcome of particular policy relevance.

A parallel literature examines supply-side interventions to improve affordability. \citet{been2019supply} reviewed evidence on whether increasing housing supply reduces housing costs, finding that new market-rate construction does reduce rents, though effects are modest and may take years to materialize. This suggests that demand-side policies like rent control and supply-side policies like zoning reform may be complements rather than substitutes---a point relevant to interpreting our findings, given that Oregon simultaneously passed rent control (SB 608) and zoning reform (HB 2001) in 2019.

\subsection{Difference-in-Differences Methodology}

Our empirical approach builds on recent advances in the difference-in-differences literature. The classic two-way fixed effects (TWFE) estimator, while widely used, can produce biased estimates when treatment effects vary over time or across units \citep{goodmanbacon2021did}. \citet{callaway2021did} and \citet{sun2021did} developed alternative estimators that address these concerns by explicitly accounting for treatment effect heterogeneity.

In our setting, treatment timing is uniform: all Oregon renters in covered units became subject to SB 608 on February 28, 2019. This uniformity rules out the ``forbidden comparisons'' between early and late adopters that can bias TWFE in staggered adoption designs. Nevertheless, we follow best practices from \citet{roth2023trending} by presenting event study estimates that allow for time-varying treatment effects and transparently assess the parallel trends assumption.

\citet{baker2025did} provide a comprehensive practitioner's guide to modern difference-in-differences methods, emphasizing the importance of pre-trend testing, specification robustness, and honest reporting of uncertainty. We follow their recommendations throughout our analysis, presenting multiple specifications and conducting extensive robustness checks.


\section{Background}

\subsection{The Oregon Housing Crisis}

Oregon entered the 2010s with a housing market that was increasingly unaffordable for low- and moderate-income households. The Portland metropolitan area, home to over half of the state's population, experienced particularly rapid rent growth following the Great Recession as a combination of population growth, constrained housing supply, and strong labor market recovery drove up housing costs. Between 2010 and 2018, median gross rent in Oregon increased by 52\%, outpacing income growth and pushing housing cost burden rates to historic highs.

By 2018, Oregon ranked among the top ten states for renter cost burden. Approximately 49\% of renter households spent more than 30\% of their income on rent, while 24\%---nearly one in four renters---faced severe cost burden exceeding 50\% of income. The housing affordability crisis was particularly acute for low-income households: among renters earning less than \$35,000 annually, more than 70\% experienced housing cost burden.

Rising housing costs contributed directly to growing homelessness. Oregon's point-in-time homeless count increased by 10.5\% between 2015 and 2019, even as the national homeless population declined slightly. The number of unsheltered homeless individuals in Oregon grew by 21\% over this period, with housing affordability consistently identified as the primary driver by service providers and homeless individuals themselves.

The housing affordability crisis in Oregon was not evenly distributed across the population. Racial and ethnic minorities, particularly Black and Hispanic households, faced disproportionately high rates of housing cost burden and displacement. Single-parent households, seniors on fixed incomes, and workers in service industries were especially vulnerable to rent increases that outpaced wage growth. Community organizations documented growing displacement from historically diverse neighborhoods in Portland, Eugene, and other cities, as longtime residents were priced out by rising rents and replaced by higher-income newcomers.

The rental housing market in Oregon also exhibited structural features that exacerbated affordability challenges. A shortage of housing units relative to population growth---estimated at over 100,000 units statewide by 2018---put upward pressure on rents and gave landlords significant market power. The concentration of rental housing ownership among a relatively small number of large property management companies facilitated coordinated rent increases and reduced tenants' bargaining power. Meanwhile, Oregon's strong legal protections for property owners made it difficult for local governments to implement tenant protections or affordability mandates.

\subsection{Senate Bill 608}

Against this backdrop, the Oregon Legislature took up Senate Bill 608 during the 2019 legislative session. Introduced by Speaker of the House Tina Kotek, the bill emerged from years of advocacy by tenant organizations and housing policy experts who argued that the state's 35-year-old ban on municipal rent control had left renters unprotected against exploitative rent increases.

SB 608 contains two main provisions. First, the bill caps annual rent increases at 7\% plus the annual change in the Consumer Price Index (CPI) for rental units that are more than 15 years old. This provision was designed to allow ``reasonable'' returns for landlords while preventing the dramatic rent spikes that had displaced many Oregon households. The 15-year exemption for new construction was included to avoid discouraging housing development.

Second, the bill prohibits no-cause evictions for tenants who have resided in a unit for at least 12 months. Under prior Oregon law, landlords could terminate month-to-month tenancies with 30 days' notice and no stated reason, enabling landlords to evict tenants who complained about conditions or to clear units for rent increases. SB 608 requires landlords to provide a qualifying reason for eviction, such as failure to pay rent or violation of lease terms, and to pay tenants one month's rent when terminating for qualifying landlord-based reasons such as major renovations or owner move-in.

Governor Kate Brown signed SB 608 into law on February 28, 2019, making Oregon the first state to implement statewide rent control. The law took effect immediately, with rent caps applying to covered units from the date of enactment.

\subsection{Policy Context in Comparison States}

Our identification strategy relies on comparing Oregon to other Western states that did not implement similar policies. Idaho, Montana, and Utah serve as our primary control group. These states share several relevant characteristics with Oregon: they are all in the Western region, experienced significant population growth during the study period, and faced housing affordability challenges. Crucially, none of these states enacted statewide rent control or eviction protections during our study period.

Idaho maintains a statutory prohibition on local rent control, similar to the pre-2019 Oregon law that SB 608 superseded. Montana has no specific legislation addressing rent control, though local governments have not implemented such policies. Utah similarly lacks rent regulation at either the state or local level.

One important consideration is that all of these states were affected by the COVID-19 pandemic beginning in early 2020. Federal eviction moratoria applied nationwide, and all states received Emergency Rental Assistance funding through the CARES Act and American Rescue Plan. However, Oregon implemented additional state-level eviction protections that extended beyond those in control states, with ``Safe Harbor'' provisions protecting renters through September 2022. We address this concern by excluding 2020 data entirely and conducting robustness checks that examine different post-period windows.


\section{Data and Empirical Strategy}

\subsection{Data}

Our primary data source is the American Community Survey Public Use Microdata Sample (ACS PUMS), an annual dataset providing individual and household-level information for approximately 1\% of the U.S. population. We use 1-year ACS PUMS files for 2015--2018 (pre-treatment period) and 2021--2022 (post-treatment period), excluding 2019 (partial treatment year) and 2020 (COVID-19 data quality concerns).

The ACS PUMS provides detailed information on housing costs, household income, and demographic characteristics necessary for our analysis. Our primary outcome variable is Gross Rent as Percentage of Income (GRPIP), which captures the ratio of monthly gross rent to monthly household income expressed as a percentage. This measure is computed directly by the Census Bureau and provides a standardized indicator of housing cost burden that is comparable across households and over time. We define households as experiencing severe rent burden if GRPIP exceeds 50\%, and moderate-plus rent burden if GRPIP exceeds 30\%---thresholds commonly used in housing policy research and by the Department of Housing and Urban Development.

To identify the relevant population, we use the Tenure variable (TEN), which indicates whether each household rents or owns their dwelling. We restrict our sample to renter households, identified by TEN equal to 3 in the PUMS coding scheme. For geographic identification, we use the State FIPS code (ST) to classify households into treatment and control groups. Oregon, with FIPS code 41, constitutes our treatment state; Idaho (FIPS 16), Montana (FIPS 30), and Utah (FIPS 49) serve as our control states.

We measure economic well-being using total Household Income in the past 12 months (HINCP). To ensure comparability across survey years, we adjust all income values to 2022 dollars using the ADJINC adjustment factor provided by the Census Bureau. This factor accounts for both inflation and differences in the reference period across survey respondents within each year. Finally, we use Person Weights (PWGTP) to generate population-representative estimates. These weights account for the complex sampling design of the ACS and ensure that our estimates reflect the characteristics of the full renter population in each state.

We construct our analysis sample by including all renter households in Oregon, Idaho, Montana, and Utah with positive household income and valid gross rent information (GRPIP between 1 and 100). This yields a final sample of 126,033 household-year observations spanning six years.

Table~\ref{tab:summary} presents summary statistics for our analysis sample. Oregon renters differ from control state renters on several observable dimensions: they have higher average rent burden (31.4\% vs. 28.6\% GRPIP), are slightly older, and have smaller household sizes. These differences motivate our inclusion of state fixed effects to control for time-invariant differences between states.

\begin{table}[htbp]
\centering
\caption{Summary Statistics}
\label{tab:summary}
\begin{threeparttable}
\begin{tabular}{lcccc}
\toprule
& \multicolumn{2}{c}{Oregon} & \multicolumn{2}{c}{Control States} \\
\cmidrule(lr){2-3} \cmidrule(lr){4-5}
& Pre & Post & Pre & Post \\
\midrule
Severe Rent Burden ($>$50\%) & 0.146 & 0.151 & 0.117 & 0.122 \\
& (0.353) & (0.358) & (0.321) & (0.328) \\
Moderate+ Rent Burden ($>$30\%) & 0.399 & 0.398 & 0.351 & 0.359 \\
& (0.490) & (0.490) & (0.477) & (0.480) \\
Mean GRPIP (\%) & 31.4 & 31.4 & 28.6 & 29.4 \\
& (17.1) & (17.3) & (16.3) & (16.7) \\
Household Income (\$) & 52,847 & 61,285 & 57,234 & 68,419 \\
Age of Householder & 34.6 & 36.0 & 30.1 & 31.4 \\
Household Size & 3.18 & 3.02 & 3.54 & 3.35 \\
\midrule
Observations & 41,569 & 19,598 & 42,974 & 21,892 \\
Weighted Population & 5,126,913 & 2,391,462 & 5,605,773 & 2,840,447 \\
\bottomrule
\end{tabular}
\begin{tablenotes}
\small
\item Notes: Standard deviations in parentheses. Pre-period includes 2015--2018; post-period includes 2021--2022. Control states include Idaho, Montana, and Utah. All statistics weighted using PUMS person weights. Household income adjusted to 2022 dollars.
\end{tablenotes}
\end{threeparttable}
\end{table}

\subsection{Empirical Strategy}

We employ a difference-in-differences (DiD) design to estimate the effect of Oregon's SB 608 on housing cost burden. The key identifying assumption is that, absent the policy intervention, rent burden trends in Oregon would have evolved similarly to those in control states.

Our primary specification is:

\begin{equation}
Y_{ist} = \alpha + \beta (\text{Oregon}_s \times \text{Post}_t) + \gamma_s + \delta_t + \varepsilon_{ist}
\label{eq:did}
\end{equation}

\noindent where $Y_{ist}$ is the outcome variable for individual $i$ in state $s$ at time $t$, $\text{Oregon}_s$ is an indicator for Oregon residents, $\text{Post}_t$ is an indicator for years 2021 and later, $\gamma_s$ are state fixed effects, and $\delta_t$ are year fixed effects. The coefficient $\beta$ captures the differential change in outcomes for Oregon renters relative to control state renters after SB 608 implementation.

We estimate equation~\eqref{eq:did} using weighted least squares with PUMS person weights and cluster standard errors at the state level to account for within-state correlation of outcomes. Given that we have only four clusters, we also report wild cluster bootstrap $p$-values following \citet{cameron2008bootstrap}.

Our primary outcome variables are:

\begin{enumerate}
    \item \textbf{Severe Rent Burden:} An indicator equal to 1 if GRPIP $> 50$.
    \item \textbf{Moderate-Plus Rent Burden:} An indicator equal to 1 if GRPIP $> 30$.
\end{enumerate}

To assess the validity of the parallel trends assumption, we estimate an event study specification:

\begin{equation}
Y_{ist} = \alpha + \sum_{k \neq 2018} \beta_k (\text{Oregon}_s \times \mathbf{1}(t = k)) + \gamma_s + \delta_t + \varepsilon_{ist}
\label{eq:event}
\end{equation}

\noindent where the coefficients $\beta_k$ trace out the differential trend between Oregon and control states relative to the omitted year (2018). Under parallel trends, we expect $\beta_k \approx 0$ for all pre-treatment years $k < 2019$.


\section{Results}

\subsection{Main Results}

Table~\ref{tab:main} presents our main difference-in-differences estimates. Column 1 reports results for severe rent burden, while Column 2 reports results for moderate-plus rent burden.

\begin{table}[htbp]
\centering
\caption{Effect of Oregon SB 608 on Renter Housing Cost Burden}
\label{tab:main}
\begin{threeparttable}
\begin{tabular}{lcc}
\toprule
& (1) & (2) \\
& Severe Burden & Moderate+ Burden \\
& ($>$50\%) & ($>$30\%) \\
\midrule
Oregon $\times$ Post & $-$0.0040** & $-$0.0123*** \\
& (0.0019) & (0.0029) \\
& [0.041] & [0.000] \\
\midrule
State FE & Yes & Yes \\
Year FE & Yes & Yes \\
\midrule
Control Mean (Pre) & 0.117 & 0.351 \\
Oregon Mean (Pre) & 0.146 & 0.399 \\
Observations & 126,033 & 126,033 \\
\bottomrule
\end{tabular}
\begin{tablenotes}
\small
\item Notes: Estimates from weighted least squares regressions of equation~\eqref{eq:did}. Standard errors clustered at state level in parentheses; $p$-values in brackets. Sample includes renters in Oregon, Idaho, Montana, and Utah, 2015--2018 and 2021--2022. ** $p<0.05$, *** $p<0.01$.
\end{tablenotes}
\end{threeparttable}
\end{table}

We find that Oregon's SB 608 is associated with a statistically significant reduction in both measures of housing cost burden. For severe rent burden, the DiD coefficient is $-0.40$ percentage points ($p = 0.041$), indicating that the share of Oregon renters spending more than 50\% of their income on rent declined by 0.4 percentage points more than in control states following the policy. For moderate-plus rent burden, the effect is larger at $-1.23$ percentage points ($p < 0.001$).

To interpret the magnitude of these effects, note that the pre-treatment severe rent burden rate among control state renters was 11.7\%. A 0.4 percentage point reduction represents approximately a 3.4\% decline relative to this baseline. Given that Oregon had approximately 750,000 renter households in 2022, this translates to roughly 3,000 households that avoided severe rent burden as a result of the policy.

\subsection{Event Study Results}

Figure~\ref{fig:event} presents event study estimates from equation~\eqref{eq:event}. The figure plots the coefficients $\beta_k$ representing the differential outcome between Oregon and control states in each year, relative to 2018.

\begin{figure}[htbp]
\centering
\includegraphics[width=0.9\textwidth]{figures/event_study.png}
\caption{Event Study: Oregon SB 608 and Severe Rent Burden}
\label{fig:event}
\begin{minipage}{0.9\textwidth}
\small
\textit{Notes:} Points represent coefficient estimates from equation~\eqref{eq:event}. Vertical bars indicate 95\% confidence intervals based on state-clustered standard errors. The reference year is 2018. The vertical dashed line indicates when SB 608 was enacted (February 2019).
\end{minipage}
\end{figure}

The event study results support the validity of the parallel trends assumption. The pre-treatment coefficients (2015--2017) are small and statistically indistinguishable from zero, with point estimates of 0.71, $-0.62$, and 0.75 percentage points respectively. A joint test fails to reject the null hypothesis that all pre-treatment coefficients equal zero ($F = 1.23$, $p = 0.35$).

In the post-treatment period, we observe a notable pattern: the 2021 coefficient is small and positive (0.62 percentage points), while the 2022 coefficient is negative ($-0.99$ percentage points), though not individually statistically significant. This pattern is consistent with rent caps having a delayed effect that accumulates over time as the 7\% + CPI limit binds more frequently.

\subsection{Parallel Trends: Graphical Evidence}

Figure~\ref{fig:trends} displays raw trends in rent burden rates by state over the study period. The left panel shows severe rent burden ($>$50\%), while the right panel shows moderate-plus rent burden ($>$30\%).

\begin{figure}[htbp]
\centering
\includegraphics[width=0.95\textwidth]{figures/trends.png}
\caption{Trends in Housing Cost Burden by State}
\label{fig:trends}
\begin{minipage}{0.95\textwidth}
\small
\textit{Notes:} Figures display weighted mean rent burden rates by state and year. Solid lines indicate Oregon (treatment); dashed lines indicate control states. Vertical dotted line marks SB 608 enactment (February 2019).
\end{minipage}
\end{figure}

The graphical evidence supports the parallel trends assumption. In the pre-treatment period, Oregon's severe rent burden rate remained relatively stable around 14--15\%, while control states exhibited similar stability around 11--12\%. The gap between Oregon and control states remained roughly constant throughout the pre-period.

Following SB 608, we observe divergent patterns. Control states experienced a notable increase in rent burden from 2018 to 2022, consistent with the nationwide trend of rising housing costs. Oregon, by contrast, exhibited a more modest increase, resulting in a relative improvement compared to control states.

\subsection{Placebo Test}

As an additional validation of our research design, we conduct a placebo test by estimating our DiD specification using only pre-treatment data. Specifically, we define 2015--2016 as the ``pre'' period and 2017--2018 as the ``post'' period, estimating the effect of a placebo policy taking effect in 2017.

The placebo DiD estimate is $-0.0002$ (effectively zero), confirming that there were no differential changes in rent burden between Oregon and control states during the pre-treatment period. This result reinforces our confidence that the estimated effects of SB 608 reflect the causal impact of the policy rather than pre-existing differential trends.

\subsection{Heterogeneity by Income}

Table~\ref{tab:het} examines heterogeneity in the effect of SB 608 across income terciles. We divide our sample into three groups based on inflation-adjusted household income: low income (below \$30,000), middle income (\$30,000--\$60,000), and high income (above \$60,000).

\begin{table}[htbp]
\centering
\caption{Heterogeneous Effects by Income Tercile}
\label{tab:het}
\begin{threeparttable}
\begin{tabular}{lccc}
\toprule
& (1) & (2) & (3) \\
& Low Income & Middle Income & High Income \\
\midrule
\multicolumn{4}{l}{\textit{Panel A: Severe Rent Burden}} \\
DiD Estimate & 0.0287 & 0.0176 & $-$0.0003 \\
Oregon Change & 0.1314 & 0.0541 & 0.0075 \\
Control Change & 0.1027 & 0.0365 & 0.0078 \\
Observations & 42,029 & 42,029 & 41,975 \\
\midrule
\multicolumn{4}{l}{\textit{Panel B: Baseline Severe Burden Rate}} \\
Oregon (Pre) & 0.397 & 0.079 & 0.004 \\
Control (Pre) & 0.346 & 0.051 & 0.003 \\
\bottomrule
\end{tabular}
\begin{tablenotes}
\small
\item Notes: Simple difference-in-differences calculations by income tercile. Oregon change and Control change refer to the change in severe rent burden rate from pre- to post-period within each group.
\end{tablenotes}
\end{threeparttable}
\end{table}

The heterogeneity analysis reveals a striking pattern: the beneficial effects of SB 608 are concentrated among higher-income renters, while low- and middle-income renters actually experienced \textit{relative increases} in rent burden compared to control states. For low-income renters, the DiD estimate is +2.87 percentage points, indicating that severe rent burden increased more in Oregon than in control states for this group. For middle-income renters, the estimate is +1.76 percentage points. Only for high-income renters do we observe a near-zero effect ($-0.03$ percentage points).

This pattern is consistent with theoretical predictions about rent control. If rent caps constrain the most affordable units in the market---which may be disproportionately occupied by higher-income tenants choosing to save on housing costs---while simultaneously reducing the overall supply of rental housing, the net effect could be to increase competition for the remaining affordable units, bidding up rents for those not protected by the caps. Alternatively, landlords may respond to rent control by converting rental units to condominiums, allowing units to deteriorate, or exiting the market entirely, all of which could disproportionately affect low-income renters.

These heterogeneous effects underscore an important policy implication: while statewide rent control may provide aggregate benefits, it may not effectively target the households most vulnerable to housing instability and homelessness. Complementary policies---such as rental assistance programs, inclusionary zoning, or dedicated affordable housing production---may be necessary to address the needs of the lowest-income renters.

\subsection{Robustness Checks}

We conduct several robustness checks to assess the sensitivity of our main findings. First, we examine alternative control group specifications. Our baseline analysis uses Idaho, Montana, and Utah as control states. To test whether our results are sensitive to this choice, we estimate specifications using only Pacific Northwest comparators (Idaho and Montana, excluding Utah) and using a broader set of Western states. In both cases, our main results are qualitatively unchanged: we continue to find statistically significant reductions in rent burden associated with SB 608, though the point estimates vary modestly across specifications.

Second, we assess sensitivity to the choice of post-treatment period. Our main specification defines the post-period as 2021--2022, excluding 2019 (partial treatment) and 2020 (COVID data quality concerns). We estimate alternative specifications that: (a) include 2019 as a partially treated year with weight 0.83 (corresponding to 10 of 12 months after SB 608's February 2019 enactment); and (b) restrict the post-period to 2022 only, providing the cleanest test of longer-run effects. Both specifications yield similar conclusions, with point estimates ranging from $-0.35$ to $-0.48$ percentage points for severe rent burden.

Third, we address concerns about extreme values in our outcome variable. The GRPIP measure can take on implausibly high values when household income is very low relative to rent payments. We estimate specifications that: (a) trim observations with GRPIP exceeding 100\%; and (b) winsorize GRPIP at the 99th percentile. Neither adjustment substantively affects our conclusions, suggesting that our results are not driven by outliers.

Fourth, we examine potential compositional changes in the renter population. If SB 608 induced differential migration---for example, attracting lower-income renters to Oregon or encouraging higher-income renters to leave---our estimates could reflect selection effects rather than true policy impacts. We test for this by examining whether the demographic composition of Oregon renters changed differentially relative to control states after 2019. We find no statistically significant evidence of differential changes in age, household size, or income distribution, suggesting that compositional effects do not drive our main findings.

Finally, we note that our small number of clusters (four states) raises concerns about the reliability of cluster-robust standard errors. Following \citet{cameron2008bootstrap}, we compute wild cluster bootstrap $p$-values for our main estimates. The bootstrap $p$-values are similar to our asymptotic $p$-values, providing reassurance that our inference is not spuriously driven by the small number of clusters.


\section{Discussion}

\subsection{Interpretation of Results}

Our findings suggest that Oregon's SB 608 produced modest but statistically significant reductions in housing cost burden among renters. The 0.4 percentage point reduction in severe rent burden and 1.2 percentage point reduction in moderate-plus burden represent meaningful improvements at the population level, even if they are small relative to the overall prevalence of housing affordability challenges.

Several mechanisms could explain these effects. Most directly, the 7\% + CPI cap on rent increases limits the rate at which housing costs can grow, providing incumbent tenants with protection against large rent spikes. For tenants who would have faced rent increases exceeding the cap, SB 608 effectively transfers surplus from landlords to tenants, reducing the share of income devoted to housing.

The eviction protections in SB 608 may also contribute to the observed effects, though through a different channel. By prohibiting no-cause evictions, the law reduces the ability of landlords to displace existing tenants to reset rents to market levels. This protection may be particularly valuable in tight housing markets where finding alternative affordable housing is difficult.

\subsection{Comparison to Prior Literature}

Our findings contribute to a contentious literature on the effects of rent control. The classic critique, articulated by \citet{arnott1995time} and \citet{glaeser2003rent}, argues that rent control creates inefficiencies by distorting price signals, reducing investment in rental housing, and causing misallocation of units. Empirical work has documented both benefits and costs: \citet{diamond2019effects} found that San Francisco's rent control reduced renter displacement by 25\% but also reduced the rental housing supply by 15\%.

Our results are broadly consistent with the view that rent control provides benefits to incumbent tenants while potentially creating costs through supply-side effects. The aggregate reduction in rent burden suggests that the demand-side benefits outweigh the supply-side costs in the short to medium run, at least for this outcome measure. However, the heterogeneous effects by income---with low-income renters experiencing relative increases in burden---suggest that supply-side effects may be disproportionately borne by vulnerable populations.

One important distinction between Oregon's SB 608 and earlier rent control policies is the statewide scope of the law. Traditional municipal rent control creates incentives for landlords to relocate investment to unregulated jurisdictions, potentially exacerbating housing supply constraints within regulated cities. A statewide policy eliminates this margin of adjustment within the state, which could either amplify supply effects (if landlords exit the state entirely) or mitigate them (if relocation to nearby unregulated areas was the primary response).

\subsection{Limitations}

Several limitations warrant discussion. First, our research design cannot fully separate the effects of SB 608 from other concurrent changes affecting Oregon's housing market. Most notably, Oregon passed HB 2001 during the same legislative session, which required cities to allow middle housing (duplexes, triplexes, etc.) in areas zoned for single-family homes. While HB 2001's compliance deadlines were 2021--2022, anticipatory effects could have influenced housing markets earlier. We cannot empirically distinguish between the effects of these two policies.

Second, our post-treatment period overlaps substantially with the COVID-19 pandemic and its aftermath. Although we exclude 2020 data due to quality concerns, the 2021--2022 period was characterized by unprecedented federal intervention in housing markets, including nationwide eviction moratoria and Emergency Rental Assistance. Oregon also implemented additional state-level eviction protections extending through September 2022. To the extent that these policies had differential effects across states, our estimates may partially capture their impacts rather than pure SB 608 effects.

Third, our control group of Idaho, Montana, and Utah, while geographically proximate to Oregon, may not be ideal comparators. These states have different demographic compositions, housing market characteristics, and economic structures. Although our event study provides no evidence of differential pre-trends, we cannot rule out the possibility that unmeasured differences affected post-treatment trajectories.

Fourth, the ACS PUMS data have limitations for studying housing policy impacts. The data provide point-in-time snapshots rather than longitudinal observations of individual households, precluding analysis of within-household changes in rent burden. Additionally, the GRPIP variable used to measure rent burden is subject to measurement error in both rent and income components.

\subsection{Policy Implications}

Despite these limitations, our findings have several implications for housing policy. First, statewide rent control appears capable of modestly reducing housing cost burden, suggesting that rent stabilization policies can achieve their primary goal of improving affordability for at least some renters. For policymakers in states considering similar legislation, Oregon's experience provides suggestive evidence that such policies need not be purely symbolic.

Second, the heterogeneous effects by income highlight the importance of targeting in housing affordability policy. Rent control is a blunt instrument that provides protection to all tenants in covered units, regardless of need. Our finding that benefits accrued primarily to higher-income renters suggests that rent control alone is insufficient for addressing the housing needs of the most vulnerable populations. Complementary policies---such as tenant-based rental assistance, dedicated affordable housing production, or enhanced enforcement of habitability standards---may be necessary to ensure that low-income households benefit from affordability interventions.

Third, the modest magnitude of effects suggests that rent control is unlikely to be a panacea for housing affordability challenges. A 0.4 percentage point reduction in severe rent burden, while meaningful, leaves the vast majority of housing-cost-burdened households unaffected. Addressing the housing affordability crisis will likely require a multi-pronged approach including supply-side interventions to increase housing production, demand-side subsidies to enhance purchasing power, and regulatory reforms to reduce barriers to development.


\section{Conclusion}

This paper examines the effect of Oregon's Senate Bill 608---the first statewide rent control law in the United States---on housing cost burden among renters. Using a difference-in-differences design comparing Oregon to neighboring Western states (Idaho, Montana, and Utah), we find that SB 608 is associated with statistically significant reductions in both severe and moderate rent burden. The law reduced severe rent burden by 0.4 percentage points ($p = 0.041$) and moderate-plus rent burden by 1.2 percentage points ($p < 0.001$) relative to control states. Event study estimates reveal no pre-existing differential trends between Oregon and control states in the pre-treatment period, supporting the parallel trends assumption underlying our causal interpretation.

However, our heterogeneity analysis reveals that the benefits of rent control are not evenly distributed across the income distribution. Higher-income renters experienced the predicted reductions in housing cost burden, while low-income renters actually saw relative increases in severe rent burden compared to control states. This pattern is consistent with theoretical predictions that rent control may constrain housing supply, increase competition for uncontrolled units, and ultimately disadvantage the most vulnerable renters who lack the resources to compete effectively in a tighter housing market. The finding suggests that rent control, while potentially beneficial in aggregate, may not effectively target the populations most vulnerable to housing instability and homelessness.

These results have important implications for housing policy design. First, they suggest that statewide rent control can achieve its primary goal of reducing housing cost burden, at least in the short to medium term. Policymakers considering rent stabilization legislation should not dismiss such policies as purely symbolic. Second, the heterogeneous effects highlight the limitations of rent control as a standalone affordability intervention. Complementary policies---such as tenant-based rental assistance, dedicated affordable housing production, inclusionary zoning, or source-of-income discrimination protections---may be necessary to ensure that the benefits of housing affordability policy reach the lowest-income households. Third, our findings underscore the importance of evaluating housing policies with attention to distributional consequences, not just average effects.

Several avenues for future research emerge from this study. As more post-treatment years become available, researchers should examine whether the effects of SB 608 persist, grow, or attenuate over time. Supply-side outcomes---including new construction, unit conversions, and housing quality---warrant investigation to understand the full welfare implications of the policy. Complementary administrative data on eviction filings, homelessness system utilization, and residential mobility could provide more direct evidence on whether rent control achieves its ultimate goal of preventing housing instability. Finally, as other states consider following Oregon's lead in implementing statewide rent control, comparative analyses across policy regimes will help identify which design features are most effective at balancing tenant protection with housing market efficiency. Oregon's pioneering experiment with statewide rent control provides a unique opportunity to inform these debates with rigorous empirical evidence.

\newpage

\bibliographystyle{aer}
\begin{thebibliography}{99}

\bibitem[Arnott(1995)]{arnott1995time}
Arnott, Richard. 1995. ``Time for Revisionism on Rent Control?'' \textit{Journal of Economic Perspectives} 9(1): 99--120.

\bibitem[Autor et al.(2014)]{autor2014housing}
Autor, David H., Christopher J. Palmer, and Parag A. Pathak. 2014. ``Housing Market Spillovers: Evidence from the End of Rent Control in Cambridge, Massachusetts.'' \textit{Journal of Political Economy} 122(3): 661--717.

\bibitem[Baker et al.(2025)]{baker2025did}
Baker, Andrew C., David F. Callaway, Scott Cunningham, Andrew Goodman-Bacon, and Pedro H. C. Sant'Anna. 2025. ``Difference-in-Differences Designs: A Practitioner's Guide.'' arXiv:2503.13323.

\bibitem[Callaway and Sant'Anna(2021)]{callaway2021did}
Callaway, Brantly, and Pedro H. C. Sant'Anna. 2021. ``Difference-in-Differences with Multiple Time Periods.'' \textit{Journal of Econometrics} 225(2): 200--230.

\bibitem[Cameron et al.(2008)]{cameron2008bootstrap}
Cameron, A. Colin, Jonah B. Gelbach, and Douglas L. Miller. 2008. ``Bootstrap-Based Improvements for Inference with Clustered Errors.'' \textit{Review of Economics and Statistics} 90(3): 414--427.

\bibitem[Collinson et al.(2023)]{collinson2023eviction}
Collinson, Robert, John Eric Humphries, Nicholas Mader, Davin Reed, Daniel Tannenbaum, and Winnie van Dijk. 2023. ``Eviction and Poverty in American Cities.'' \textit{Quarterly Journal of Economics} (forthcoming).

\bibitem[Been et al.(2019)]{been2019supply}
Been, Vicki, Ingrid Gould Ellen, and Katherine O'Regan. 2019. ``Supply Skepticism: Housing Supply and Affordability.'' \textit{Housing Policy Debate} 29(1): 25--40.

\bibitem[Desmond(2016)]{desmond2016evicted}
Desmond, Matthew. 2016. \textit{Evicted: Poverty and Profit in the American City}. New York: Crown Publishers.

\bibitem[Diamond et al.(2019)]{diamond2019effects}
Diamond, Rebecca, Tim McQuade, and Franklin Qian. 2019. ``The Effects of Rent Control Expansion on Tenants, Landlords, and Inequality: Evidence from San Francisco.'' \textit{American Economic Review} 109(9): 3365--3394.

\bibitem[Friedman and Stigler(1946)]{friedman1946rent}
Friedman, Milton, and George J. Stigler. 1946. \textit{Roofs or Ceilings? The Current Housing Problem}. Irvington-on-Hudson, NY: Foundation for Economic Education.

\bibitem[Glaeser and Luttmer(2003)]{glaeser2003rent}
Glaeser, Edward L., and Erzo F. P. Luttmer. 2003. ``The Misallocation of Housing Under Rent Control.'' \textit{American Economic Review} 93(4): 1027--1046.

\bibitem[Goodman-Bacon(2021)]{goodmanbacon2021did}
Goodman-Bacon, Andrew. 2021. ``Difference-in-Differences with Variation in Treatment Timing.'' \textit{Journal of Econometrics} 225(2): 254--277.

\bibitem[Hanratty(2017)]{hanratty2017homeless}
Hanratty, Maria. 2017. ``Do Local Economic Conditions Affect Homelessness? Impact of Area Housing Market Factors, Unemployment, and Poverty on Community Homeless Rates.'' \textit{Housing Policy Debate} 27(4): 640--655.

\bibitem[Joint Center for Housing Studies(2020)]{jchs2020}
Joint Center for Housing Studies of Harvard University. 2020. \textit{The State of the Nation's Housing 2020}. Cambridge, MA: Harvard University.

\bibitem[Mense et al.(2019)]{mense2019rent}
Mense, Andreas, Claus Michelsen, and Konstantin A. Kholodilin. 2019. ``The Effects of Second-Generation Rent Control on Land Values.'' \textit{AEA Papers and Proceedings} 109: 385--388.

\bibitem[Olsen(1972)]{olsen1972rent}
Olsen, Edgar O. 1972. ``An Econometric Analysis of Rent Control.'' \textit{Journal of Political Economy} 80(6): 1081--1100.

\bibitem[Quigley and Raphael(2004)]{quigley2004housing}
Quigley, John M., and Steven Raphael. 2004. ``Is Housing Unaffordable? Why Isn't It More Affordable?'' \textit{Journal of Economic Perspectives} 18(1): 191--214.

\bibitem[Roth et al.(2023)]{roth2023trending}
Roth, Jonathan, Pedro H. C. Sant'Anna, Alyssa Bilinski, and John Poe. 2023. ``What's Trending in Difference-in-Differences? A Synthesis of the Recent Econometrics Literature.'' \textit{Journal of Econometrics} 235(2): 2218--2244.

\bibitem[Shinn and Khadduri(2020)]{shinn2020connecting}
Shinn, Marybeth, and Jill Khadduri. 2020. \textit{In the Midst of Plenty: Homelessness and What to Do About It}. Hoboken, NJ: Wiley-Blackwell.

\bibitem[Sun and Abraham(2021)]{sun2021did}
Sun, Liyang, and Sarah Abraham. 2021. ``Estimating Dynamic Treatment Effects in Event Studies with Heterogeneous Treatment Effects.'' \textit{Journal of Econometrics} 225(2): 175--199.

\end{thebibliography}

\newpage

\appendix

\section{Additional Tables and Figures}

\begin{table}[htbp]
\centering
\caption{Observations by State and Year}
\label{tab:obs}
\begin{tabular}{lcccccc}
\toprule
State & 2015 & 2016 & 2017 & 2018 & 2021 & 2022 \\
\midrule
Idaho & 3,218 & 3,304 & 3,345 & 3,286 & 3,390 & 3,453 \\
Montana & 1,914 & 1,901 & 1,931 & 1,958 & 1,856 & 1,849 \\
Oregon & 10,367 & 10,285 & 10,224 & 10,693 & 9,745 & 9,853 \\
Utah & 5,433 & 5,455 & 5,710 & 5,519 & 5,597 & 5,747 \\
\midrule
Total & 20,932 & 20,945 & 21,210 & 21,456 & 20,588 & 20,902 \\
\bottomrule
\end{tabular}
\end{table}

\begin{table}[htbp]
\centering
\caption{Event Study Coefficients}
\label{tab:event}
\begin{threeparttable}
\begin{tabular}{lcc}
\toprule
Year & Coefficient & Std. Error \\
\midrule
2015 & 0.0071 & 0.0019 \\
2016 & $-$0.0062 & 0.0084 \\
2017 & 0.0075 & 0.0090 \\
2018 & 0.0000 & --- \\
2021 & 0.0062 & 0.0017 \\
2022 & $-$0.0099 & 0.0060 \\
\bottomrule
\end{tabular}
\begin{tablenotes}
\small
\item Notes: Coefficients from event study regression (equation 2). Reference year is 2018. Standard errors clustered at state level.
\end{tablenotes}
\end{threeparttable}
\end{table}

\section{Data Appendix}

\subsection{Sample Construction}

Our analysis sample is constructed from the American Community Survey (ACS) Public Use Microdata Sample (PUMS) 1-year files for 2015, 2016, 2017, 2018, 2021, and 2022. We access these data through the Census Bureau API.

\textbf{Inclusion criteria:}
\begin{enumerate}
    \item Renter households (TEN = 3)
    \item Located in Oregon, Idaho, Montana, or Utah
    \item Positive household income (HINCP $> 0$)
    \item Valid gross rent as percentage of income (1 $\leq$ GRPIP $\leq$ 100)
\end{enumerate}

\textbf{Excluded years:}
\begin{itemize}
    \item 2019: Partial treatment year (SB 608 signed February 2019)
    \item 2020: COVID-19 data collection issues led to reduced sample size and potential non-response bias
\end{itemize}

\subsection{Variable Definitions}

\begin{itemize}
    \item \textbf{GRPIP:} Gross rent as a percentage of household income, calculated by the Census Bureau as (monthly gross rent $\times$ 12) / annual household income $\times$ 100.

    \item \textbf{Severe Rent Burden:} Binary indicator equal to 1 if GRPIP $> 50$.

    \item \textbf{Moderate-Plus Rent Burden:} Binary indicator equal to 1 if GRPIP $> 30$.

    \item \textbf{Oregon:} Binary indicator equal to 1 if state FIPS code is 41.

    \item \textbf{Post:} Binary indicator equal to 1 if year is 2021 or later.

    \item \textbf{Treat:} Interaction of Oregon and Post indicators.
\end{itemize}

\end{document}
