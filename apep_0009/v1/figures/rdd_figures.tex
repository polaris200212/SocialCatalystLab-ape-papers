% RDD Figures for Colorado OAP Paper

% Include these in the main paper with % RDD Figures for Colorado OAP Paper

% Include these in the main paper with % RDD Figures for Colorado OAP Paper

% Include these in the main paper with % RDD Figures for Colorado OAP Paper

% Include these in the main paper with \input{figures/rdd_figures}

\begin{figure}[H]
\centering
\begin{tikzpicture}
\begin{axis}[
    width=0.9\textwidth,
    height=0.6\textwidth,
    xlabel={Age},
    ylabel={Labor Force Participation Rate},
    xmin=54.5, xmax=65.5,
    ymin=0.129,
    ymax=0.394,
    xtick={55,56,57,58,59,60,61,62,63,64,65},
    grid=major,
    grid style={dashed, gray!30},
    title={Labor Force Participation by Age},
    legend pos=north east,
]

% Vertical line at cutoff
\addplot[dashed, thick, red] coordinates {(60, 0.129) (60, 0.394)};

% Data points below cutoff
\addplot[only marks, mark=*, blue, mark size=3pt] coordinates {
    (55, 0.3708)
    (56, 0.3600)
    (57, 0.3740)
    (58, 0.3253)
    (59, 0.3201)
}};

% Data points at and above cutoff
\addplot[only marks, mark=*, red!70!black, mark size=3pt] coordinates {
    (60, 0.3035)
    (61, 0.2856)
    (62, 0.2389)
    (63, 0.1931)
    (64, 0.2009)
    (65, 0.1494)
};

\legend{Cutoff (Age 60), Below 60, At/Above 60}

\end{axis}
\end{tikzpicture}
\caption{Labor Force Participation by Age. Each point represents the weighted mean for that age bin in the low-income sample (personal income $<$ \$20,000). The vertical dashed line indicates the age-60 OAP eligibility threshold.}
\label{fig:rdd_lfp}
\end{figure}

\clearpage

\begin{figure}[H]
\centering
\begin{tikzpicture}
\begin{axis}[
    width=0.9\textwidth,
    height=0.6\textwidth,
    xlabel={Age},
    ylabel={Public Assistance Receipt Rate},
    xmin=54.5, xmax=65.5,
    ymin=0.003,
    ymax=0.058,
    xtick={55,56,57,58,59,60,61,62,63,64,65},
    grid=major,
    grid style={dashed, gray!30},
    title={Public Assistance Receipt by Age},
    legend pos=north east,
]

% Vertical line at cutoff
\addplot[dashed, thick, red] coordinates {(60, 0.003) (60, 0.058)};

% Data points below cutoff
\addplot[only marks, mark=*, blue, mark size=3pt] coordinates {
    (55, 0.0272)
    (56, 0.0298)
    (57, 0.0264)
    (58, 0.0364)
    (59, 0.0332)
}};

% Data points at and above cutoff
\addplot[only marks, mark=*, red!70!black, mark size=3pt] coordinates {
    (60, 0.0335)
    (61, 0.0379)
    (62, 0.0289)
    (63, 0.0247)
    (64, 0.0227)
    (65, 0.0312)
};

\legend{Cutoff (Age 60), Below 60, At/Above 60}

\end{axis}
\end{tikzpicture}
\caption{Public Assistance Receipt by Age. Each point represents the weighted mean for that age bin in the low-income sample (personal income $<$ \$20,000). The vertical dashed line indicates the age-60 OAP eligibility threshold.}
\label{fig:rdd_pap}
\end{figure}

\clearpage

\begin{figure}[H]
\centering
\begin{tikzpicture}
\begin{axis}[
    width=0.9\textwidth,
    height=0.6\textwidth,
    xlabel={Age},
    ylabel={Average Hours Worked},
    xmin=54.5, xmax=65.5,
    ymin=5.140,
    ymax=11.214,
    xtick={55,56,57,58,59,60,61,62,63,64,65},
    grid=major,
    grid style={dashed, gray!30},
    title={Average Hours Worked by Age},
    legend pos=north east,
]

% Vertical line at cutoff
\addplot[dashed, thick, red] coordinates {(60, 5.140) (60, 11.214)};

% Data points below cutoff
\addplot[only marks, mark=*, blue, mark size=3pt] coordinates {
    (55, 10.9953)
    (56, 11.0164)
    (57, 11.1937)
    (58, 10.0028)
    (59, 10.0782)
}};

% Data points at and above cutoff
\addplot[only marks, mark=*, red!70!black, mark size=3pt] coordinates {
    (60, 9.6557)
    (61, 9.1214)
    (62, 7.4808)
    (63, 5.9600)
    (64, 6.3272)
    (65, 5.1596)
};

\legend{Cutoff (Age 60), Below 60, At/Above 60}

\end{axis}
\end{tikzpicture}
\caption{Average Hours Worked by Age. Each point represents the weighted mean for that age bin in the low-income sample (personal income $<$ \$20,000). The vertical dashed line indicates the age-60 OAP eligibility threshold.}
\label{fig:rdd_hours}
\end{figure}

\clearpage



\begin{figure}[H]
\centering
\begin{tikzpicture}
\begin{axis}[
    width=0.9\textwidth,
    height=0.6\textwidth,
    xlabel={Age},
    ylabel={Labor Force Participation Rate},
    xmin=54.5, xmax=65.5,
    ymin=0.129,
    ymax=0.394,
    xtick={55,56,57,58,59,60,61,62,63,64,65},
    grid=major,
    grid style={dashed, gray!30},
    title={Labor Force Participation by Age},
    legend pos=north east,
]

% Vertical line at cutoff
\addplot[dashed, thick, red] coordinates {(60, 0.129) (60, 0.394)};

% Data points below cutoff
\addplot[only marks, mark=*, blue, mark size=3pt] coordinates {
    (55, 0.3708)
    (56, 0.3600)
    (57, 0.3740)
    (58, 0.3253)
    (59, 0.3201)
}};

% Data points at and above cutoff
\addplot[only marks, mark=*, red!70!black, mark size=3pt] coordinates {
    (60, 0.3035)
    (61, 0.2856)
    (62, 0.2389)
    (63, 0.1931)
    (64, 0.2009)
    (65, 0.1494)
};

\legend{Cutoff (Age 60), Below 60, At/Above 60}

\end{axis}
\end{tikzpicture}
\caption{Labor Force Participation by Age. Each point represents the weighted mean for that age bin in the low-income sample (personal income $<$ \$20,000). The vertical dashed line indicates the age-60 OAP eligibility threshold.}
\label{fig:rdd_lfp}
\end{figure}

\clearpage

\begin{figure}[H]
\centering
\begin{tikzpicture}
\begin{axis}[
    width=0.9\textwidth,
    height=0.6\textwidth,
    xlabel={Age},
    ylabel={Public Assistance Receipt Rate},
    xmin=54.5, xmax=65.5,
    ymin=0.003,
    ymax=0.058,
    xtick={55,56,57,58,59,60,61,62,63,64,65},
    grid=major,
    grid style={dashed, gray!30},
    title={Public Assistance Receipt by Age},
    legend pos=north east,
]

% Vertical line at cutoff
\addplot[dashed, thick, red] coordinates {(60, 0.003) (60, 0.058)};

% Data points below cutoff
\addplot[only marks, mark=*, blue, mark size=3pt] coordinates {
    (55, 0.0272)
    (56, 0.0298)
    (57, 0.0264)
    (58, 0.0364)
    (59, 0.0332)
}};

% Data points at and above cutoff
\addplot[only marks, mark=*, red!70!black, mark size=3pt] coordinates {
    (60, 0.0335)
    (61, 0.0379)
    (62, 0.0289)
    (63, 0.0247)
    (64, 0.0227)
    (65, 0.0312)
};

\legend{Cutoff (Age 60), Below 60, At/Above 60}

\end{axis}
\end{tikzpicture}
\caption{Public Assistance Receipt by Age. Each point represents the weighted mean for that age bin in the low-income sample (personal income $<$ \$20,000). The vertical dashed line indicates the age-60 OAP eligibility threshold.}
\label{fig:rdd_pap}
\end{figure}

\clearpage

\begin{figure}[H]
\centering
\begin{tikzpicture}
\begin{axis}[
    width=0.9\textwidth,
    height=0.6\textwidth,
    xlabel={Age},
    ylabel={Average Hours Worked},
    xmin=54.5, xmax=65.5,
    ymin=5.140,
    ymax=11.214,
    xtick={55,56,57,58,59,60,61,62,63,64,65},
    grid=major,
    grid style={dashed, gray!30},
    title={Average Hours Worked by Age},
    legend pos=north east,
]

% Vertical line at cutoff
\addplot[dashed, thick, red] coordinates {(60, 5.140) (60, 11.214)};

% Data points below cutoff
\addplot[only marks, mark=*, blue, mark size=3pt] coordinates {
    (55, 10.9953)
    (56, 11.0164)
    (57, 11.1937)
    (58, 10.0028)
    (59, 10.0782)
}};

% Data points at and above cutoff
\addplot[only marks, mark=*, red!70!black, mark size=3pt] coordinates {
    (60, 9.6557)
    (61, 9.1214)
    (62, 7.4808)
    (63, 5.9600)
    (64, 6.3272)
    (65, 5.1596)
};

\legend{Cutoff (Age 60), Below 60, At/Above 60}

\end{axis}
\end{tikzpicture}
\caption{Average Hours Worked by Age. Each point represents the weighted mean for that age bin in the low-income sample (personal income $<$ \$20,000). The vertical dashed line indicates the age-60 OAP eligibility threshold.}
\label{fig:rdd_hours}
\end{figure}

\clearpage



\begin{figure}[H]
\centering
\begin{tikzpicture}
\begin{axis}[
    width=0.9\textwidth,
    height=0.6\textwidth,
    xlabel={Age},
    ylabel={Labor Force Participation Rate},
    xmin=54.5, xmax=65.5,
    ymin=0.129,
    ymax=0.394,
    xtick={55,56,57,58,59,60,61,62,63,64,65},
    grid=major,
    grid style={dashed, gray!30},
    title={Labor Force Participation by Age},
    legend pos=north east,
]

% Vertical line at cutoff
\addplot[dashed, thick, red] coordinates {(60, 0.129) (60, 0.394)};

% Data points below cutoff
\addplot[only marks, mark=*, blue, mark size=3pt] coordinates {
    (55, 0.3708)
    (56, 0.3600)
    (57, 0.3740)
    (58, 0.3253)
    (59, 0.3201)
}};

% Data points at and above cutoff
\addplot[only marks, mark=*, red!70!black, mark size=3pt] coordinates {
    (60, 0.3035)
    (61, 0.2856)
    (62, 0.2389)
    (63, 0.1931)
    (64, 0.2009)
    (65, 0.1494)
};

\legend{Cutoff (Age 60), Below 60, At/Above 60}

\end{axis}
\end{tikzpicture}
\caption{Labor Force Participation by Age. Each point represents the weighted mean for that age bin in the low-income sample (personal income $<$ \$20,000). The vertical dashed line indicates the age-60 OAP eligibility threshold.}
\label{fig:rdd_lfp}
\end{figure}

\clearpage

\begin{figure}[H]
\centering
\begin{tikzpicture}
\begin{axis}[
    width=0.9\textwidth,
    height=0.6\textwidth,
    xlabel={Age},
    ylabel={Public Assistance Receipt Rate},
    xmin=54.5, xmax=65.5,
    ymin=0.003,
    ymax=0.058,
    xtick={55,56,57,58,59,60,61,62,63,64,65},
    grid=major,
    grid style={dashed, gray!30},
    title={Public Assistance Receipt by Age},
    legend pos=north east,
]

% Vertical line at cutoff
\addplot[dashed, thick, red] coordinates {(60, 0.003) (60, 0.058)};

% Data points below cutoff
\addplot[only marks, mark=*, blue, mark size=3pt] coordinates {
    (55, 0.0272)
    (56, 0.0298)
    (57, 0.0264)
    (58, 0.0364)
    (59, 0.0332)
}};

% Data points at and above cutoff
\addplot[only marks, mark=*, red!70!black, mark size=3pt] coordinates {
    (60, 0.0335)
    (61, 0.0379)
    (62, 0.0289)
    (63, 0.0247)
    (64, 0.0227)
    (65, 0.0312)
};

\legend{Cutoff (Age 60), Below 60, At/Above 60}

\end{axis}
\end{tikzpicture}
\caption{Public Assistance Receipt by Age. Each point represents the weighted mean for that age bin in the low-income sample (personal income $<$ \$20,000). The vertical dashed line indicates the age-60 OAP eligibility threshold.}
\label{fig:rdd_pap}
\end{figure}

\clearpage

\begin{figure}[H]
\centering
\begin{tikzpicture}
\begin{axis}[
    width=0.9\textwidth,
    height=0.6\textwidth,
    xlabel={Age},
    ylabel={Average Hours Worked},
    xmin=54.5, xmax=65.5,
    ymin=5.140,
    ymax=11.214,
    xtick={55,56,57,58,59,60,61,62,63,64,65},
    grid=major,
    grid style={dashed, gray!30},
    title={Average Hours Worked by Age},
    legend pos=north east,
]

% Vertical line at cutoff
\addplot[dashed, thick, red] coordinates {(60, 5.140) (60, 11.214)};

% Data points below cutoff
\addplot[only marks, mark=*, blue, mark size=3pt] coordinates {
    (55, 10.9953)
    (56, 11.0164)
    (57, 11.1937)
    (58, 10.0028)
    (59, 10.0782)
}};

% Data points at and above cutoff
\addplot[only marks, mark=*, red!70!black, mark size=3pt] coordinates {
    (60, 9.6557)
    (61, 9.1214)
    (62, 7.4808)
    (63, 5.9600)
    (64, 6.3272)
    (65, 5.1596)
};

\legend{Cutoff (Age 60), Below 60, At/Above 60}

\end{axis}
\end{tikzpicture}
\caption{Average Hours Worked by Age. Each point represents the weighted mean for that age bin in the low-income sample (personal income $<$ \$20,000). The vertical dashed line indicates the age-60 OAP eligibility threshold.}
\label{fig:rdd_hours}
\end{figure}

\clearpage



\begin{figure}[H]
\centering
\begin{tikzpicture}
\begin{axis}[
    width=0.9\textwidth,
    height=0.6\textwidth,
    xlabel={Age},
    ylabel={Labor Force Participation Rate},
    xmin=54.5, xmax=65.5,
    ymin=0.129,
    ymax=0.394,
    xtick={55,56,57,58,59,60,61,62,63,64,65},
    grid=major,
    grid style={dashed, gray!30},
    title={Labor Force Participation by Age},
    legend pos=north east,
]

% Vertical line at cutoff
\addplot[dashed, thick, red] coordinates {(60, 0.129) (60, 0.394)};

% Data points below cutoff
\addplot[only marks, mark=*, blue, mark size=3pt] coordinates {
    (55, 0.3708)
    (56, 0.3600)
    (57, 0.3740)
    (58, 0.3253)
    (59, 0.3201)
}};

% Data points at and above cutoff
\addplot[only marks, mark=*, red!70!black, mark size=3pt] coordinates {
    (60, 0.3035)
    (61, 0.2856)
    (62, 0.2389)
    (63, 0.1931)
    (64, 0.2009)
    (65, 0.1494)
};

\legend{Cutoff (Age 60), Below 60, At/Above 60}

\end{axis}
\end{tikzpicture}
\caption{Labor Force Participation by Age. Each point represents the weighted mean for that age bin in the low-income sample (personal income $<$ \$20,000). The vertical dashed line indicates the age-60 OAP eligibility threshold.}
\label{fig:rdd_lfp}
\end{figure}

\clearpage

\begin{figure}[H]
\centering
\begin{tikzpicture}
\begin{axis}[
    width=0.9\textwidth,
    height=0.6\textwidth,
    xlabel={Age},
    ylabel={Public Assistance Receipt Rate},
    xmin=54.5, xmax=65.5,
    ymin=0.003,
    ymax=0.058,
    xtick={55,56,57,58,59,60,61,62,63,64,65},
    grid=major,
    grid style={dashed, gray!30},
    title={Public Assistance Receipt by Age},
    legend pos=north east,
]

% Vertical line at cutoff
\addplot[dashed, thick, red] coordinates {(60, 0.003) (60, 0.058)};

% Data points below cutoff
\addplot[only marks, mark=*, blue, mark size=3pt] coordinates {
    (55, 0.0272)
    (56, 0.0298)
    (57, 0.0264)
    (58, 0.0364)
    (59, 0.0332)
}};

% Data points at and above cutoff
\addplot[only marks, mark=*, red!70!black, mark size=3pt] coordinates {
    (60, 0.0335)
    (61, 0.0379)
    (62, 0.0289)
    (63, 0.0247)
    (64, 0.0227)
    (65, 0.0312)
};

\legend{Cutoff (Age 60), Below 60, At/Above 60}

\end{axis}
\end{tikzpicture}
\caption{Public Assistance Receipt by Age. Each point represents the weighted mean for that age bin in the low-income sample (personal income $<$ \$20,000). The vertical dashed line indicates the age-60 OAP eligibility threshold.}
\label{fig:rdd_pap}
\end{figure}

\clearpage

\begin{figure}[H]
\centering
\begin{tikzpicture}
\begin{axis}[
    width=0.9\textwidth,
    height=0.6\textwidth,
    xlabel={Age},
    ylabel={Average Hours Worked},
    xmin=54.5, xmax=65.5,
    ymin=5.140,
    ymax=11.214,
    xtick={55,56,57,58,59,60,61,62,63,64,65},
    grid=major,
    grid style={dashed, gray!30},
    title={Average Hours Worked by Age},
    legend pos=north east,
]

% Vertical line at cutoff
\addplot[dashed, thick, red] coordinates {(60, 5.140) (60, 11.214)};

% Data points below cutoff
\addplot[only marks, mark=*, blue, mark size=3pt] coordinates {
    (55, 10.9953)
    (56, 11.0164)
    (57, 11.1937)
    (58, 10.0028)
    (59, 10.0782)
}};

% Data points at and above cutoff
\addplot[only marks, mark=*, red!70!black, mark size=3pt] coordinates {
    (60, 9.6557)
    (61, 9.1214)
    (62, 7.4808)
    (63, 5.9600)
    (64, 6.3272)
    (65, 5.1596)
};

\legend{Cutoff (Age 60), Below 60, At/Above 60}

\end{axis}
\end{tikzpicture}
\caption{Average Hours Worked by Age. Each point represents the weighted mean for that age bin in the low-income sample (personal income $<$ \$20,000). The vertical dashed line indicates the age-60 OAP eligibility threshold.}
\label{fig:rdd_hours}
\end{figure}

\clearpage

