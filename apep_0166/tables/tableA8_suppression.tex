\begin{table}[htbp]
\centering
\caption{CDC Suppression in Working-Age (25--64) Diabetes Mortality Data}
\label{tab:suppression}
\begin{tabular}{lc}
\hline\hline
Statistic & Value \\
\hline
Potential state-year cells (51 jurisdictions $\times$ 25 years) & 1,275 \\
Non-suppressed observations in analysis panel & 1,142 \\
Suppressed or missing state-year cells & 133 \\
Suppression rate (share of potential cells) & 10.4\% \\
Jurisdictions with any suppressed years & $\geq$10 \\
Vermont (excluded: fully suppressed post-treatment) & 1 \\
\hline
\multicolumn{2}{l}{\textit{Suppression Bounds (TWFE ATT)}} \\
Lower bound (impute deaths $= 0$) & -0.117 \\
Primary (drop suppressed) & -0.117 \\
Upper bound (impute deaths $= 9$) & -0.117 \\
\hline\hline
\end{tabular}
\begin{tablenotes}
\small
\item \textit{Notes:} CDC WONDER suppresses state-year cells with fewer than 10 deaths.
The 133 missing cells include both CDC-suppressed cells and the 2018--2019 gap
(approximately 100 cells from 50 states $\times$ 2 years, plus additional single-year
suppression in small states such as Alaska, Wyoming, North Dakota, and DC).
\item Primary specification drops suppressed cells. Bounds impute at extreme values.
\item Suppression is concentrated in small-population states and is uncorrelated with
treatment status (suppressed cells occur in both treated and control states).
\end{tablenotes}
\end{table}
