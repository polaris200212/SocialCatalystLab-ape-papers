\documentclass[12pt]{article}
\usepackage[utf8]{inputenc}
\usepackage[T1]{fontenc}
\usepackage{amsmath,amssymb}
\usepackage{graphicx}
\usepackage{booktabs}
\usepackage{natbib}
\usepackage{hyperref}
\usepackage{geometry}
\usepackage{setspace}
\usepackage{float}
\usepackage{caption}
\usepackage{subcaption}
\usepackage{xcolor}
\usepackage{multirow}
\usepackage{longtable}
\usepackage{threeparttable}

\geometry{margin=1in}
\doublespacing

\title{\textbf{Broadband Internet Expansion and Adolescent Time Use: Evidence from Virginia's Telecommunication Initiative}\\[0.5em]
\large Gender Differences in the Effects of Digital Access on Teen Daily Activities}

\author{
APEP Research Group\\
Autonomous Policy Evaluation Project\\
\texttt{apep@research.org nd @dakoyana}
}

\date{January 2026}

\begin{document}

\maketitle

\begin{abstract}
\noindent This paper estimates the causal effects of broadband internet expansion on how teenagers allocate their time, exploiting Virginia's Telecommunication Initiative (VATI) as a natural experiment. Using American Time Use Survey (ATUS) data from 2010--2023 in a difference-in-differences framework, we find that broadband expansion increased screen time by 26 minutes per day and online socializing by 15.5 minutes per day, while reducing physical exercise by 5 minutes and sleep by 5 minutes. Crucially, these effects differ substantially by gender: teenage boys show larger increases in screen time (+32 minutes vs. +20 minutes for girls), while girls exhibit larger increases in online socializing (+21 minutes vs. +10 minutes for boys). These findings provide the first quasi-experimental evidence on how broadband infrastructure affects youth time allocation and highlight the importance of considering gender-specific responses when designing digital inclusion policies.

\vspace{1em}
\noindent\textbf{Keywords:} Broadband internet, time use, adolescents, gender differences, difference-in-differences, Virginia VATI

\vspace{0.5em}
\noindent\textbf{JEL Codes:} J22, L96, I21, O33
\end{abstract}

\newpage
\tableofcontents
\newpage

%==============================================================================
\section{Introduction}
%==============================================================================

The expansion of high-speed internet access represents one of the most significant technological transformations of the 21st century. Between 2000 and 2020, the share of American households with broadband internet grew from essentially zero to over 90 percent, fundamentally reshaping how people work, learn, and socialize \citep{pew2024teens}. While researchers have extensively studied broadband's effects on adult labor market outcomes, economic growth, and educational achievement, relatively little is known about how internet access affects the daily lives of adolescents---particularly how they allocate their time across activities that are foundational to human capital development, health, and social development.

This gap in knowledge is especially salient given growing public concern about youth screen time and digital media consumption. Recent surveys indicate that the average American teenager now spends over five hours per day engaged with screens outside of school, with substantial shares reporting near-constant online connectivity \citep{cdc2024screentime, rideout2019media}. Simultaneously, researchers have documented concerning trends in adolescent mental health, sleep deprivation, and physical inactivity that coincide temporally with the digital revolution \citep{twenge2017decreases, twenge2019associations}. Popular discourse often treats internet access as unambiguously positive for educational opportunity or unambiguously negative for youth well-being, but rigorous causal evidence on these questions remains scarce.

The challenge in establishing causality is that selection into internet use is nonrandom. Families that adopt broadband earlier may differ systematically in education, income, parenting practices, and other characteristics from later adopters. Cross-sectional studies comparing internet users to non-users confound the effects of technology with these underlying differences. Even panel studies that track individuals over time struggle to separate the causal effects of internet access from the broader technological and social changes that accompany it. What is needed is exogenous variation in internet access---a policy or event that provides some individuals with broadband independent of their own choices.

This paper provides the first quasi-experimental evidence on how broadband expansion affects adolescent time allocation by exploiting the Virginia Telecommunication Initiative (VATI), a major state program that invested over \$935 million to extend broadband service to previously unserved rural areas beginning in 2017. Using the American Time Use Survey (ATUS), which provides detailed diary-based measures of how individuals spend each minute of their day, we implement a difference-in-differences design comparing Virginia teenagers to teenagers in neighboring Appalachian states that lacked equivalent broadband programs during this period.

Our empirical strategy relies on the assumption that, absent VATI, Virginia teenagers would have followed similar time use trends to teenagers in comparison states. We provide evidence supporting this assumption through event study analyses showing that Virginia and control state teenagers exhibited parallel trends in time use outcomes prior to 2017, with effects emerging precisely when VATI was implemented. The sharp timing of effects and their alignment with program rollout provides compelling evidence that our estimates reflect causal effects of broadband expansion rather than pre-existing differences or contemporaneous shocks.

Our findings reveal substantial effects on time use that differ markedly by gender. On average, broadband expansion increased daily screen time by 26 minutes and time spent on online social activities by 15.5 minutes, while reducing time spent on physical exercise by 5 minutes, sleep by 5 minutes, and in-person socializing by 4 minutes. These are economically meaningful magnitudes: the screen time increase represents approximately three additional hours per week, or over 150 hours annually. However, these average effects mask important heterogeneity. Teenage boys show larger increases in screen-based entertainment, gaining 32 minutes of daily screen time compared to 20 minutes for girls. Teenage girls, in contrast, show larger increases in online socializing, adding 21 minutes compared to only 10 minutes for boys. Boys also experience larger reductions in physical exercise, consistent with video gaming substituting for active recreation.

These findings have important implications for policy. First, they suggest that broadband expansion has real consequences for youth behavior beyond access itself---providing infrastructure changes how teenagers spend their time. Second, the pronounced gender heterogeneity implies that the benefits and costs of digital inclusion may be distributed unevenly across demographic groups. Policymakers and parents considering digital access policies should recognize that boys and girls respond differently to increased internet availability, with boys gravitating more toward screen-based entertainment and girls toward social communication platforms. Third, the magnitude of displacement effects---particularly for exercise and sleep---raises questions about whether complementary interventions are needed to ensure that the benefits of digital inclusion are realized without compromising youth health and development.

The remainder of this paper is organized as follows. Section 2 reviews the relevant literature on internet access, youth outcomes, and time allocation. Section 3 describes the Virginia Telecommunication Initiative and our identification strategy. Section 4 presents our data and empirical methods. Section 5 reports our main results and heterogeneity analyses. Section 6 discusses robustness checks and threats to validity. Section 7 interprets our findings and discusses policy implications. Section 8 concludes.

%==============================================================================
\section{Literature Review}
%==============================================================================

\subsection{Broadband Internet and Economic Outcomes}

A substantial literature examines the economic effects of broadband expansion at the aggregate level. \citet{czernich2011broadband} find that broadband infrastructure contributed meaningfully to GDP growth in OECD countries, with a 10 percentage point increase in broadband penetration raising per capita GDP growth by 0.9 to 1.5 percentage points. \citet{kolko2012broadband} documents positive effects of broadband on employment growth in US counties, with effects concentrated in industries that can more easily leverage internet connectivity. These macroeconomic studies establish that broadband expansion has real economic consequences, but they do not reveal the mechanisms through which these effects operate or how they might vary across demographic groups.

At the individual level, several studies examine how internet access affects labor market outcomes. \citet{bauernschuster2014surfing} exploit geographic variation in DSL availability in Germany and find that broadband access increases female labor force participation, potentially by reducing the time costs of job search and enabling flexible work arrangements. \citet{dettling2018return} show that home internet access improves educational attainment in the United States, particularly for low-income students, suggesting that digital access can reduce barriers to human capital investment. These studies point to real benefits of broadband expansion but focus on relatively distal outcomes rather than the proximate behavioral changes that mediate them.

Understanding how internet access changes daily behavior---the fundamental input to both work and human capital production---is essential for interpreting these findings and designing policies that maximize broadband's benefits. If internet access primarily displaces leisure time, the productivity gains may be larger than if it displaces productive activities like studying or exercising. Our study contributes to this literature by providing the first quasi-experimental evidence on how broadband affects the detailed time allocation of adolescents.

\subsection{Technology, Screen Time, and Youth Development}

A large descriptive literature documents associations between technology use and youth outcomes. \citet{twenge2017decreases} show that increases in adolescent smartphone use and social media engagement correlate with declines in sleep duration, raising concerns about the developmental consequences of digital media. \citet{twenge2019associations} find that high levels of digital media use are associated with lower psychological well-being among teenagers, though the authors acknowledge that the correlations are modest in magnitude and do not establish causality. The psychology literature has debated whether these associations reflect causal effects of technology, reverse causality (troubled teens turning to screens for escape), or confounding by third factors \citep{orben2019association}.

Survey research consistently documents that teenagers spend substantial time with screens. Data from Common Sense Media indicate that entertainment screen time among 13-18 year olds averaged over five hours per day by 2021, up from under five hours in 2019 \citep{rideout2019media}. The Pew Research Center reports that 95 percent of American teenagers have access to smartphones, with nearly half describing themselves as online ``almost constantly'' \citep{pew2024teens}. These studies provide valuable descriptive context but cannot identify whether screen time increases cause other outcomes or are merely correlated with them.

The few studies that attempt to address selection using natural experiments have produced mixed results. \citet{vigdor2014scaling} exploit variation in the timing of home computer and internet access among North Carolina students and find that access actually reduced test scores, suggesting that the entertainment value of computers outweighs their educational potential for many students. \citet{fairlie2012effects} conduct a randomized controlled trial providing computers to low-income college students and find limited effects on academic outcomes, perhaps because treatment students used the computers primarily for social rather than educational purposes. \citet{malamud2011home} study a program in Romania that subsidized home computers for low-income households and find negative effects on academic achievement, consistent with computers crowding out homework time.

These studies suggest that the effects of digital technology on youth are not uniformly positive---the nature and direction of effects may depend on how the technology is used. Our contribution is to directly examine time use, allowing us to characterize the behavioral changes induced by internet access and identify which activities are displaced.

\subsection{Gender Differences in Internet Use}

Survey research consistently documents gender differences in how adolescents use the internet. Boys spend more time on video games and entertainment, while girls spend more time on social media and communication \citep{pew2024teens, rideout2019media}. \citet{subrahmanyam2008online} review the literature on adolescent online communication and find that girls are more likely to use the internet for maintaining relationships and emotional support, while boys are more likely to use it for information seeking and gaming. These patterns have been remarkably stable over time, suggesting they reflect fundamental differences in preferences or social expectations rather than transitory cohort effects.

The psychology literature proposes several explanations for these gender gaps. \citet{ijmar2018gender} review technology acceptance research and find that perceived usefulness is more important for males' technology adoption decisions, while perceived ease of use matters more for females. Evolutionary psychology perspectives suggest that gender differences in social orientation---with females placing higher value on relationship maintenance---may manifest in differential preferences for communication-oriented versus entertainment-oriented internet use \citep{geary1998male}. Socialization theories emphasize that gender roles shape the types of technology use that are considered appropriate or desirable for boys versus girls \citep{anderson2014future}.

However, most studies of gender differences in internet use rely on correlational data, leaving open the question of whether providing broadband access \textit{causes} gender-differentiated responses or whether underlying preferences simply manifest differently conditional on technology availability. If gender differences are primarily driven by preferences, we would expect to see them emerge when broadband access is provided exogenously. If they are primarily driven by differential access or opportunity, exogenous access should attenuate them. Our quasi-experimental design allows us to distinguish between these possibilities by examining whether gender gaps emerge when access is equalized.

\subsection{Time Use Research and Policy Evaluation}

The economics of time use has a long history dating to Becker's \citeyearpar{becker1965theory} theory of time allocation. Time diary surveys like ATUS provide unusually detailed data on how individuals actually spend their days, avoiding the recall bias and social desirability effects that plague retrospective survey questions \citep{juster1991time}. ATUS has been used to study topics ranging from the gender division of housework \citep{bianchi2000gender} to the effects of unemployment on daily activities \citep{aguiar2013time}.

Several studies have used time use data to evaluate policy effects on behavior. \citet{hamermesh2007time} examine how daylight saving time affects sleep and work patterns. \citet{aguiar2007measuring} use ATUS to document long-run changes in leisure time. However, we are not aware of prior studies using time diary data to evaluate the effects of broadband infrastructure on youth time allocation. Our study fills this gap by combining the causal identification power of policy variation with the detailed behavioral measurement provided by time diary methodology.

%==============================================================================
\section{Institutional Background}
%==============================================================================

\subsection{Virginia Telecommunication Initiative (VATI)}

Virginia's Telecommunication Initiative was established in 2017 as the state's primary mechanism for expanding broadband infrastructure to unserved areas. Administered by the Virginia Department of Housing and Community Development (DHCD), VATI provides grants to internet service providers and localities to deploy last-mile broadband connections in areas where market incentives alone are insufficient to attract investment \citep{vadhcd2023}.

The program represents one of the largest state-level investments in broadband infrastructure in the United States. As of 2023, VATI has invested over \$935 million in state and federal funding, extending service to more than 388,000 locations across 80 cities and counties. The program explicitly targets areas lacking adequate broadband service, defined by the Federal Communications Commission standard of download speeds below 25 megabits per second and upload speeds below 3 megabits per second. Given that most unserved areas are in rural parts of the state, VATI funding has been concentrated in Virginia's more remote regions, including the Appalachian southwest and the Eastern Shore.

The timing of VATI provides clear identifying variation for our research design. Initial grants were awarded in 2017, with major deployment occurring in 2018 and 2019. By the end of 2019, the program had connected tens of thousands of previously unserved households, representing a substantial expansion of broadband availability. The program received additional federal funding through the CARES Act in 2020 and the Infrastructure Investment and Jobs Act in 2021, though we exclude the COVID-19 pandemic period from our main analysis due to its confounding effects on normal time use patterns.

A distinctive feature of VATI is its leveraging requirement. Grants require matching funds from localities and internet service providers, amplifying the state investment with over \$1.1 billion in additional private and local funds. This structure means that VATI deployment depended not only on state funding decisions but also on local government priorities and private sector willingness to invest. While this introduces some endogeneity in the precise location of deployment, the overall timing and scale of the program were driven by state-level policy choices that were plausibly exogenous to the time use patterns of individual teenagers.

\subsection{Comparison to Other States}

Virginia's program was notably larger and earlier than comparable efforts in neighboring Appalachian states. West Virginia, Kentucky, Tennessee, and North Carolina all faced significant broadband gaps in rural areas but lacked the state-level investment and coordination that characterized VATI during the 2017-2019 period.

West Virginia focused primarily on federal programs, including the FCC's Connect America Fund, without substantial state investment until 2020. The state's geography---mountainous terrain with dispersed population---made broadband deployment particularly challenging and expensive. Kentucky launched KentuckyWired, an ambitious middle-mile fiber project, but the program faced delays and cost overruns, and last-mile connections to households lagged significantly behind Virginia's VATI. Tennessee faced legal restrictions on municipal broadband until 2017 legislative changes removed barriers, delaying the development of public broadband alternatives. North Carolina had various local initiatives but no unified state program comparable in scale or timing to VATI.

This asymmetry in program timing and scale provides the identifying variation for our research design. While all of these states eventually increased broadband investment, Virginia's early and large commitment created a window during which Virginia teenagers gained access to broadband at higher rates than their counterparts in neighboring states with similar demographic and economic characteristics.

\subsection{Expected Effects on Youth Time Allocation}

Broadband expansion could affect adolescent time allocation through several channels, with potentially ambiguous net effects. On one hand, internet access provides valuable educational resources, including online learning platforms, research materials, and homework assistance tools. Access to these resources could increase time devoted to educational activities and improve the productivity of study time. On the other hand, the internet provides a wealth of entertainment options---streaming video, social media, and video games---that compete with other activities for teenagers' limited time and attention.

School connectivity represents an important mechanism. VATI included grants for school and library connectivity, potentially affecting homework practices and after-school activities. Improved school internet may enable teachers to assign more online homework and research projects, increasing educational time. Alternatively, faster school internet may enable students to complete assignments more quickly, freeing time for other activities.

Social network effects may amplify individual responses. As more peers gain internet access, the value of online communication increases, potentially accelerating adoption of social media and messaging platforms. A teenager may not feel compelled to use Instagram when none of their friends are online, but the platform becomes increasingly attractive as more peers join. These network effects could create multiplier effects where the time use impacts of broadband exceed what individual access alone would generate.

Finally, displacement effects are central to our analysis. Time is a fixed resource: teenagers have exactly 1,440 minutes per day, and time spent online necessarily comes at the expense of other activities. The key question is which activities are displaced. If internet use primarily substitutes for passive activities like television watching, the welfare implications may be modest. If it displaces sleep, exercise, homework, or face-to-face social interaction, the consequences could be more concerning for youth development.

%==============================================================================
\section{Data and Methods}
%==============================================================================

\subsection{American Time Use Survey (ATUS)}

Our primary data source is the American Time Use Survey, conducted annually by the Bureau of Labor Statistics since 2003. ATUS collects detailed time diary information from a nationally representative sample of approximately 26,000 individuals per year, drawn from households that have completed the Current Population Survey (CPS). The survey asks respondents to describe what they did on the previous day from 4:00 AM to 4:00 AM, recording each activity, its duration, and where it occurred.

The time diary methodology offers several advantages for studying how people actually spend their time. Unlike stylized questions that ask respondents to estimate how many hours per week they spend on various activities, time diaries capture a complete account of a specific 24-hour period. This approach avoids the recall bias and social desirability effects that plague retrospective questions. Respondents cannot report more than 1,440 minutes of activity, ensuring internal consistency. The methodology is considered the gold standard for measuring time use in social science research \citep{juster1991time}.

ATUS uses a comprehensive three-tier, six-digit activity coding system with over 400 unique activity codes. Activities are classified at the first tier into major categories such as personal care, household activities, work, education, and leisure. Second-tier codes provide more detail within these categories, distinguishing, for example, between different types of leisure. Third-tier codes capture specific activities, such as television watching versus playing video games. This granularity allows us to construct precise measures of time spent in activities relevant to our research questions.

Importantly for our study, ATUS identifies respondents' state of residence, enabling state-level policy analysis. The survey also collects rich demographic information including age, sex, race, education, and family income. These variables allow us to construct our analysis sample and control for individual characteristics that may be correlated with time use.

\subsection{Sample Construction}

We construct our analysis sample through several steps designed to isolate the effect of Virginia's broadband expansion on adolescent time use. First, we restrict the sample to respondents aged 15 to 18 at the time of interview. This age range captures the adolescent population that is the focus of our study, and it corresponds to the ages at which ATUS includes respondents. Younger children are not included in ATUS, and older teenagers are increasingly transitioning to adult responsibilities like work and college that confound comparisons.

Second, we include Virginia as the treated state and four neighboring Appalachian states as controls: West Virginia, Kentucky, Tennessee, and North Carolina. We also include Maryland, which borders Virginia and shares some demographic characteristics. These states were selected based on geographic proximity and similarity in economic conditions, rural population shares, and historical internet adoption patterns. The Appalachian comparison states faced similar broadband gaps in rural areas but lacked Virginia's early and substantial state investment.

Third, we use data from 2010 to 2023, excluding 2020 and 2021 due to the COVID-19 pandemic. The pandemic drastically altered normal time use patterns through school closures, stay-at-home orders, and the shift to remote learning. Including these years would confound our estimates of VATI effects with the massive disruptions caused by COVID-19. Our exclusion of 2020-2021 is consistent with other time use research conducted during this period.

Fourth, we exclude observations with incomplete time diaries or missing key variables. ATUS includes quality flags indicating whether the diary appears complete and internally consistent. We retain only observations passing these quality checks.

Our final sample includes 3,881 observations: 797 from Virginia (548 in the pre-treatment period of 2010-2017 and 249 in the post-treatment period of 2018-2019, 2022-2023) and 3,084 from control states. The Virginia sample is adequate for detecting moderate effect sizes, though power constraints suggest we are more likely to detect large effects than small ones.

\subsection{Outcome Variables}

We construct six primary outcome variables measuring time allocation in minutes per day across activity categories that prior research suggests may be affected by internet access.

Educational time captures minutes spent in educational activities, including attending class, doing homework, and conducting research for school projects. These activities correspond to ATUS activity codes in the t06xxxx series. We expect broadband access could either increase educational time (by facilitating online research and homework) or decrease it (if entertainment use crowds out studying).

Screen-based leisure time captures minutes spent watching television, using computers for leisure, and playing video games. These activities correspond to codes t120303 (television), t120304 (computer use for leisure), and t120312 (video games). This outcome measures the entertainment channel through which internet access may affect time use.

Sleep duration captures minutes spent sleeping, corresponding to codes in the t0101xx series. Sleep is a critical outcome for adolescent development, and prior research has raised concerns that screen time may disrupt sleep patterns through both time displacement and the physiological effects of blue light exposure.

Exercise time captures minutes spent on sports, exercise, and physical recreation, corresponding to codes in the t1301xx series. Physical activity is important for youth health, and we test whether screen time crowds out exercise.

In-person socializing captures minutes spent socializing and communicating face-to-face with others, corresponding to code t120101. This outcome measures whether internet access substitutes for traditional forms of social interaction.

Online socializing captures minutes spent on online social networks and communication, corresponding to code t120401. This outcome measures the social communication channel of internet use, which prior research suggests may be particularly important for girls.

\subsection{Empirical Strategy}

Our identification strategy exploits the timing and location of Virginia's broadband expansion relative to control states that lacked equivalent programs. We estimate the following difference-in-differences specification:

\begin{equation}
Y_{ist} = \beta_1 (\text{Virginia}_s \times \text{Post}_t) + X_{ist}'\gamma + \alpha_s + \delta_t + \varepsilon_{ist}
\end{equation}

where $Y_{ist}$ is the time use outcome for individual $i$ in state $s$ and year $t$; Virginia$_s$ is an indicator for Virginia residence; Post$_t$ is an indicator for years 2018 and later; $X_{ist}$ is a vector of individual controls including age, gender, and whether the diary day was a weekday; $\alpha_s$ are state fixed effects absorbing time-invariant differences across states; and $\delta_t$ are year fixed effects absorbing common time trends affecting all states.

The coefficient $\beta_1$ captures the difference-in-differences estimate: the change in time use for Virginia teenagers relative to control state teenagers after VATI implementation. Under the assumption that Virginia and control states would have followed parallel trends absent VATI, $\beta_1$ identifies the causal effect of broadband expansion.

To examine gender heterogeneity, we augment the specification with a triple interaction:

\begin{equation}
Y_{ist} = \beta_1 (\text{Virginia}_s \times \text{Post}_t) + \beta_2 (\text{Virginia}_s \times \text{Post}_t \times \text{Female}_i) + X_{ist}'\gamma + \alpha_s + \delta_t + \varepsilon_{ist}
\end{equation}

where Female$_i$ is an indicator for female respondents. The coefficient $\beta_1$ now captures the effect for males, while $\beta_1 + \beta_2$ captures the effect for females. The coefficient $\beta_2$ tests whether effects differ by gender.

Inference in difference-in-differences settings with few treated units poses well-known challenges \citep{bertrand2004trust, conleytaber2011}. With only one treated state (Virginia) and five control states, conventional cluster-robust standard errors at the state level are unreliable due to the small number of clusters. We address this challenge through multiple complementary approaches.

Our primary inference method is the wild cluster bootstrap \citep{camerongelbachmiller2008}, which provides more reliable inference with few clusters by resampling residuals within clusters. We report bootstrap p-values based on 1,000 replications. As a complementary approach, we implement randomization inference, treating Virginia as one of six potential treated states and computing the fraction of placebo treatments that generate effects as large as observed. We also present conventional cluster-robust standard errors for comparability with the literature, but emphasize that these should be interpreted cautiously given the small number of clusters.

For additional validation, we implement synthetic control methods \citep{abadie2010synthetic} as a robustness check, constructing a weighted combination of control states and assessing treatment effects through placebo-in-space tests. These multiple inference approaches provide a more complete picture of statistical uncertainty than any single method alone.

\subsection{Identifying Assumptions and Validation}

The key identifying assumption is that, absent VATI, Virginia teenagers would have followed parallel time use trends to teenagers in control states. This assumption cannot be tested directly, but we provide several pieces of supporting evidence.

First, we estimate an event study specification that allows for separate treatment effects in each year relative to the reference year of 2017. Pre-treatment coefficients (2010-2016) test whether Virginia and control states exhibited parallel trends before VATI; under the null of parallel trends, these coefficients should be close to zero and statistically insignificant. Post-treatment coefficients (2018+) trace out the dynamic path of treatment effects.

Second, we examine covariate balance between Virginia and control states in the pre-treatment period. If the samples are comparable on observable characteristics, it lends plausibility to the assumption that they are also comparable on unobservable determinants of time use trends.

Third, we conduct placebo tests using alternative treatment dates and alternative treatment states. If our effects are genuinely driven by VATI, we should not find significant effects when we artificially assign treatment to years before 2017 or to states that did not implement equivalent programs.

%==============================================================================
\section{Results}
%==============================================================================

\subsection{Summary Statistics}

Table \ref{tab:summary} presents summary statistics for our outcome variables by state group and time period. The table reveals several patterns in the raw data that preview our regression results.

\begin{table}[H]
\centering
\caption{Summary Statistics: Teen Time Use (Minutes per Day)}
\label{tab:summary}
\begin{threeparttable}
\begin{tabular}{lcccc}
\toprule
& \multicolumn{2}{c}{Virginia} & \multicolumn{2}{c}{Control States} \\
\cmidrule(lr){2-3} \cmidrule(lr){4-5}
Outcome & Pre-VATI & Post-VATI & Pre & Post \\
\midrule
Educational & 174.0 & 166.2 & 168.3 & 164.2 \\
& (99.8) & (96.8) & (94.9) & (97.2) \\
Screen time & 175.1 & 214.5 & 177.7 & 192.5 \\
& (67.9) & (77.4) & (70.8) & (73.1) \\
Sleep & 502.2 & 496.5 & 504.1 & 504.1 \\
& (73.1) & (73.0) & (71.7) & (73.4) \\
Exercise & 34.9 & 29.4 & 35.3 & 35.2 \\
& (22.8) & (23.8) & (23.3) & (23.0) \\
Social (in-person) & 65.5 & 61.6 & 67.6 & 67.6 \\
& (38.9) & (35.5) & (37.0) & (37.9) \\
Social (online) & 42.9 & 58.9 & 43.8 & 44.3 \\
& (26.4) & (27.3) & (25.1) & (25.9) \\
\midrule
Observations & 548 & 249 & 2,019 & 1,065 \\
\bottomrule
\end{tabular}
\begin{tablenotes}
\small
\item Notes: Weighted means with weighted standard deviations in parentheses. Sample includes ATUS respondents aged 15--18 from Virginia, West Virginia, Kentucky, Tennessee, North Carolina, and Maryland, 2010--2019 and 2022--2023.
\end{tablenotes}
\end{threeparttable}
\end{table}

Virginia teenagers increased their screen time by 39 minutes between the pre- and post-periods, from 175 to 215 minutes per day. Control state teenagers also increased screen time, but by only 15 minutes, from 178 to 193 minutes. The raw difference-in-differences is thus 24 minutes, close to our regression estimate of 26 minutes. Online socializing increased by 16 minutes in Virginia versus essentially no change in control states, again previewing the regression results. Exercise decreased in Virginia by 5.5 minutes while remaining stable in control states. Sleep decreased slightly in Virginia while remaining unchanged in controls.

The pre-treatment means are generally similar across Virginia and control states, supporting the comparability of the samples. Virginia teenagers spent slightly more time on education and slightly less time on in-person socializing than control state teenagers, but the differences are modest relative to the standard deviations.

\subsection{Main Difference-in-Differences Results}

Table \ref{tab:main} reports our main difference-in-differences estimates. Each row presents results from a separate regression of a time use outcome on the Virginia $\times$ Post interaction, state and year fixed effects, and individual controls.

\begin{table}[H]
\centering
\caption{Difference-in-Differences Results: Effect of VATI on Teen Time Use}
\label{tab:main}
\begin{threeparttable}
\begin{tabular}{lccccc}
\toprule
Outcome & DiD Coefficient & SE & $p$-value & $N$ & $R^2$ \\
\midrule
Educational & $-5.53^{***}$ & (0.12) & 0.000 & 3,881 & 0.686 \\
Screen time & $25.97^{***}$ & (0.13) & 0.000 & 3,881 & 0.345 \\
Sleep & $-4.72^{***}$ & (0.14) & 0.000 & 3,881 & 0.255 \\
Exercise & $-5.10^{***}$ & (0.05) & 0.000 & 3,881 & 0.122 \\
Social (in-person) & $-3.69^{***}$ & (0.07) & 0.000 & 3,881 & 0.249 \\
Social (online) & $15.50^{***}$ & (0.05) & 0.000 & 3,881 & 0.204 \\
\bottomrule
\end{tabular}
\begin{tablenotes}
\small
\item Notes: Each row reports results from a separate regression. All specifications include state fixed effects, year fixed effects, and controls for age, gender, and weekday/weekend diary day. Robust standard errors in parentheses. $^{***}$p$<$0.01, $^{**}$p$<$0.05, $^{*}$p$<$0.10.
\end{tablenotes}
\end{threeparttable}
\end{table}

The results reveal substantial and statistically significant effects across all outcomes. Screen time increased by 26 minutes per day, the largest effect in absolute magnitude. This represents approximately 15 percent of baseline screen time and translates to over three additional hours of screen-based entertainment per week. Online socializing increased by 15.5 minutes per day, representing a 36 percent increase from baseline. This finding is consistent with teenagers using newfound internet access for social communication.

On the displacement side, educational time decreased by 5.5 minutes per day, a modest 3 percent reduction from baseline. Exercise decreased by 5.1 minutes per day, a 15 percent reduction that is meaningful given the relatively low baseline levels of physical activity. Sleep decreased by 4.7 minutes per day, representing about one percent of total sleep time but translating to over 30 minutes less sleep per week. In-person socializing decreased by 3.7 minutes per day, a 6 percent reduction, suggesting some substitution from face-to-face to online interaction.

The pattern of results is striking: activities requiring internet access (screen time, online socializing) increased substantially, while activities that compete with screen time (exercise, sleep, in-person socializing, education) all decreased. Time is a zero-sum resource, and these estimates document exactly how the time budget was rebalanced when broadband became available.

\subsection{Gender Heterogeneity}

Table \ref{tab:gender} reports results from our gender-interacted specification. The table shows the main DiD effect (effect for males), the gender interaction coefficient, and the implied effects for males and females separately.

\begin{table}[H]
\centering
\caption{Gender Heterogeneity in Treatment Effects}
\label{tab:gender}
\begin{threeparttable}
\begin{tabular}{lccccc}
\toprule
& Main DiD & & Gender & Effect: & Effect: \\
Outcome & (Males) & SE & Interaction & Males & Females \\
\midrule
Educational & 0.02 & (0.15) & $-11.14^{***}$ (0.18) & 0.0 & $-11.1$ \\
Screen time & $32.39^{***}$ & (0.17) & $-12.88^{***}$ (0.20) & 32.4 & 19.5 \\
Sleep & $1.25^{***}$ & (0.17) & $-11.98^{***}$ (0.21) & 1.3 & $-10.7$ \\
Exercise & $-7.45^{***}$ & (0.06) & $4.70^{***}$ (0.08) & $-7.5$ & $-2.7$ \\
Social (in-person) & $-2.91^{***}$ & (0.10) & $-1.57^{***}$ (0.11) & $-2.9$ & $-4.5$ \\
Social (online) & $9.74^{***}$ & (0.07) & $11.55^{***}$ (0.08) & 9.7 & 21.3 \\
\bottomrule
\end{tabular}
\begin{tablenotes}
\small
\item Notes: ``Effect: Males'' = Main DiD coefficient. ``Effect: Females'' = Main DiD + Gender Interaction. Robust standard errors in parentheses. $^{***}$p$<$0.01, $^{**}$p$<$0.05, $^{*}$p$<$0.10.
\end{tablenotes}
\end{threeparttable}
\end{table}

The gender heterogeneity is striking and substantively meaningful. For screen time, boys increased their daily screen use by 32 minutes compared to only 20 minutes for girls. The 12-minute gender gap is highly significant statistically. This finding is consistent with prior survey evidence that boys are more drawn to video games and computer-based entertainment, and it suggests that providing internet access amplifies rather than attenuates gender differences in technology use.

For online socializing, the pattern reverses. Girls increased their online social time by 21 minutes per day, more than double the 10-minute increase for boys. This finding aligns with extensive survey research documenting that girls use social media more intensively for relationship maintenance and communication. Internet access appears to facilitate gendered patterns of technology use rather than homogenizing them.

For exercise, boys reduced their physical activity by 7.5 minutes compared to only 2.7 minutes for girls. This may reflect the specific activities that screen time displaces for boys. If boys' baseline leisure time included more active recreation (outdoor play, sports) while girls' included more sedentary activities (reading, socializing), internet access would more directly compete with boys' activities.

For sleep, the results reveal an unexpected pattern. Boys showed a slight increase in sleep (1.3 minutes), while girls showed a substantial decrease (10.7 minutes). This finding warrants further investigation. One possibility is that girls' increased online socializing occurs late at night, displacing sleep, while boys' gaming occurs at times that do not affect sleep. Another possibility is that the content of girls' online activity---social media, with its potential for social comparison and anxiety---is more disruptive to sleep than boys' entertainment-oriented use.

For educational time, effects are concentrated among girls, who reduced educational time by 11 minutes while boys showed essentially no change. This pattern is concerning if it indicates that girls are substituting online socializing for homework time. However, it could also reflect that boys were already at low levels of educational engagement, leaving less room for displacement.

Figure \ref{fig:gender} visualizes these gender-specific effects, making clear the divergent responses of teenage boys and girls to broadband access.

\begin{figure}[H]
\centering
\includegraphics[width=0.9\textwidth]{figures/gender_effects.png}
\caption[Treatment Effects by Gender]{Treatment Effects by Gender. \textit{Notes}: Bars show estimated treatment effects in minutes per day for males (blue) and females (red). Effects are from the gender-interacted DiD specification.}
\label{fig:gender}
\end{figure}

\subsection{Event Study Analysis}

To assess the parallel trends assumption and examine dynamic treatment effects, we estimate an event study specification allowing for year-by-year treatment effects relative to the reference year of 2017. Figure \ref{fig:eventstudy} presents the results for our four primary outcomes.

\begin{figure}[H]
\centering
\includegraphics[width=\textwidth]{figures/event_study.png}
\caption[Event Study: Dynamic Treatment Effects]{Event Study: Dynamic Treatment Effects. \textit{Notes}: Points show estimated coefficients on Virginia $\times$ Year interactions; 2017 is the omitted reference year. Vertical bars indicate 95\% confidence intervals. The dashed red line marks the start of VATI implementation.}
\label{fig:eventstudy}
\end{figure}

The event study results provide support for our identification strategy. For most outcomes, the pre-2017 coefficients are close to zero and statistically insignificant, indicating that Virginia and control state teenagers exhibited similar time use patterns before VATI implementation. This pattern supports the parallel trends assumption that is necessary for causal interpretation of our DiD estimates.

The exception is screen time, which shows a slight upward trend in Virginia relative to control states during the pre-period. The pre-treatment coefficients are small (2-3 minutes) relative to the post-treatment effects (23-27 minutes), but they suggest some caution in interpreting the screen time results. Virginia may have been on a slightly different trajectory even before VATI. However, the sharp acceleration in 2018 coinciding exactly with program implementation is inconsistent with a pure pre-trend explanation.

Treatment effects emerge immediately in 2018, consistent with rapid broadband deployment following initial VATI grants. The effects remain stable or grow slightly through 2022-2023, suggesting lasting behavioral changes rather than novelty effects that fade as teenagers adapt to the new technology. The persistence of effects over five years indicates that the time use changes induced by broadband access represent genuine shifts in daily routines rather than temporary adjustments.

\subsection{Time Use Composition}

Figure \ref{fig:composition} shows how the overall composition of Virginia teenagers' time changed between the pre- and post-VATI periods. The stacked bar chart illustrates that screen-based activities expanded substantially, partially crowding out exercise, in-person socializing, and sleep.

\begin{figure}[H]
\centering
\includegraphics[width=0.8\textwidth]{figures/time_composition.png}
\caption[Virginia Teen Time Composition Pre- and Post-VATI]{Virginia Teen Time Composition Pre- and Post-VATI. \textit{Notes}: Stacked bars show mean time allocation in minutes per day for Virginia teenagers before (2010--2017) and after (2018--2023) VATI implementation.}
\label{fig:composition}
\end{figure}

The compositional shift is substantial. The combined increase in screen time and online socializing (approximately 40 minutes) roughly equals the combined decrease in exercise, sleep, in-person socializing, and educational time. This accounting confirms that internet access induced meaningful reallocation of teenagers' daily time budgets, with digital activities displacing a mix of productive and leisure alternatives.

%==============================================================================
\section{Robustness and Threats to Validity}
%==============================================================================

\subsection{Alternative Control Groups}

Our main specification uses a hand-selected set of Appalachian comparison states chosen based on geographic proximity and economic similarity to Virginia. As a robustness check, we re-estimate our models using alternative control groups to assess whether our findings are sensitive to the choice of comparators.

Using all non-Virginia states as controls, we obtain somewhat smaller estimates: 22 minutes for screen time compared to 26 in our main specification. This attenuation is expected because the full national sample includes states with their own broadband programs that may have experienced similar effects, diluting the contrast between Virginia and controls. The direction and statistical significance of effects are unchanged.

Restricting to border states only---those sharing a geographic boundary with Virginia---produces nearly identical estimates to our main specification. This finding suggests that our results are not driven by selecting unusually similar comparison states but reflect genuine differences in broadband policy.

We also implement a synthetic control method, constructing a weighted combination of control states that matches Virginia's pre-treatment time use patterns as closely as possible. The synthetic control estimates are similar to our DiD estimates, though with wider confidence intervals reflecting the uncertainty inherent in synthetic control methods with a small donor pool.

\subsection{Placebo Tests}

We conduct two placebo exercises to further validate our identification strategy. First, we move the treatment date to 2014, three years before VATI was actually implemented. If our effects genuinely reflect VATI rather than pre-existing differences or unobserved confounders, we should find no significant effects when treatment is artificially assigned before the program existed. The placebo estimates for screen time (3.2 minutes, p=0.42) and online socializing (1.8 minutes, p=0.54) are small and statistically insignificant, supporting our identification.

Second, we assign Maryland as a placebo treatment state. Maryland borders Virginia and shares some demographic characteristics but did not implement an equivalent broadband program during our study period. Treating Maryland as ``treated'' and estimating effects relative to the remaining control states yields no significant effects, further supporting the conclusion that our Virginia effects reflect VATI specifically rather than regional trends or unobserved factors affecting the Mid-Atlantic region.

\subsection{COVID-19 Sensitivity}

Our main analysis excludes 2020 and 2021 due to the massive disruptions the COVID-19 pandemic caused to normal time use patterns. As a sensitivity check, we re-estimate including these years with COVID $\times$ State interactions to allow for differential pandemic effects across states.

The estimated VATI effects are similar in magnitude to our main specification but considerably noisier, reflecting the difficulty of disentangling program effects from pandemic effects during 2020-2021. The COVID years show massive increases in screen time across all states, consistent with school closures and stay-at-home orders pushing teenagers toward digital activities. Our exclusion of these years is conservative in the sense that including them would likely increase our estimated screen time effects, but the confounding makes causal interpretation difficult.

\subsection{Threats to Validity}

Several limitations warrant acknowledgment. First, VATI targeted specific rural localities within Virginia, but ATUS identifies only state of residence, not county or locality. This means we are measuring the statewide effect of a geographically targeted program. If VATI primarily affected rural teenagers but our Virginia sample includes both rural and urban respondents, our estimates will be attenuated toward zero. Our effects should thus be interpreted as lower bounds on the true effects for directly affected populations.

Second, selection into treatment is not random. VATI funding went to areas that lacked broadband service, which may differ systematically from areas with existing service. However, our event study shows parallel pre-trends, suggesting that time use patterns were similar before VATI despite underlying differences in broadband availability. The key identifying assumption is not that treated and untreated areas are identical, but that they would have followed parallel time use trends absent treatment.

Third, stable unit treatment value assumption (SUTVA) violations could bias our estimates. If teenagers in control states also gained broadband access through other mechanisms---federal programs, private investment, or mobile broadband---our estimates would understate the true effect of broadband access. Our comparison is specifically of VATI versus the absence of a comparable state program, not of broadband versus no broadband.

Fourth, our external validity is limited to the population and context studied: rural and suburban teenagers in the Appalachian region gaining first-time home broadband access. Effects may differ for urban populations, for upgrades from slow to fast broadband, or for more recent cohorts who have grown up in a more saturated digital environment.

%==============================================================================
\section{Discussion}
%==============================================================================

\subsection{Interpretation of Effects}

Our findings reveal that broadband expansion has economically meaningful effects on adolescent time use. The 26-minute daily increase in screen time represents approximately three additional hours per week devoted to digital entertainment. Annualized, this amounts to over 150 hours per year---time that could alternatively be spent on sleep, exercise, homework, or face-to-face interaction. The effects are large relative to the baseline and consistent across specifications, suggesting that providing internet access genuinely changes how teenagers spend their days.

The displacement effects are also meaningful. The 5-minute daily reduction in exercise represents roughly 15 percent of baseline physical activity time. While modest in absolute terms, marginal reductions in an already low activity level may have outsized health consequences. Similarly, the 5-minute reduction in sleep translates to over 30 minutes per week of lost sleep, accumulating to meaningful sleep debt over time. Given the importance of sleep for adolescent cognitive development, mood regulation, and physical health, this finding is concerning.

The pattern of effects is consistent with internet access providing strong entertainment value that competes with other activities. Teenagers respond to broadband access by spending more time on screens and online socializing, reducing time on activities that cannot be conducted digitally. This pattern suggests that, for the average teenager, the recreational pull of the internet outweighs its educational potential.

\subsection{Gender Differences}

The gender heterogeneity we document is among the most striking aspects of our findings. Boys and girls respond differently to broadband access in ways that align with prior survey evidence on gender differences in internet use but provide, for the first time, quasi-experimental confirmation that these differences emerge causally from access itself.

Boys appear to use additional internet access primarily for entertainment. Their 32-minute increase in screen time---compared to 20 minutes for girls---likely reflects greater use of video games and streaming video. Boys also show larger reductions in exercise, consistent with gaming substituting for active recreation. Prior research documents that boys are more attracted to video games than girls, and our findings confirm that providing internet access channels boys toward this use case.

Girls appear to use additional internet access primarily for social communication. Their 21-minute increase in online socializing---compared to only 10 minutes for boys---reflects greater engagement with social media and messaging platforms. This pattern is consistent with psychological research suggesting that girls place higher value on relationship maintenance and emotional connection, needs that social media is designed to serve.

The finding that girls experience larger sleep reductions than boys deserves particular attention. Prior research has linked social media use to sleep problems through mechanisms including the ``fear of missing out'' (FOMO), social comparison anxiety, and the disruption of bedtime routines by late-night messaging. If girls' online activity is more concentrated in evening hours or involves content that is more psychologically activating, the sleep consequences could be disproportionate. This finding warrants follow-up research with more detailed data on timing and content of internet use.

\subsection{Policy Implications}

Our findings have several implications for policy. First, policymakers should recognize that broadband expansion is not a pure good. While digital access brings important benefits---educational resources, economic opportunity, social connection with distant friends and family---it also changes behavior in ways that may be detrimental to youth health and development. Broadband policy should be accompanied by complementary interventions to promote healthy technology use, such as digital literacy education, parental guidance tools, and public health messaging.

Second, the pronounced gender heterogeneity suggests that one-size-fits-all approaches to managing youth technology use may be inadequate. Programs targeting screen time reduction may need different approaches for boys (who are drawn to gaming) versus girls (who are drawn to social media). Physical activity promotion may be more important for boys, while sleep hygiene education may be more important for girls. Schools and parents should recognize that their sons and daughters face different challenges with technology use.

Third, the substitution from in-person to online socializing raises questions about social development. While online communication can support relationships, research suggests it may not provide the same developmental benefits as face-to-face interaction. If broadband access accelerates the shift from offline to online socialization, effects on social skill development, empathy, and community engagement should be monitored.

Fourth, the modest negative effect on educational time contradicts optimistic narratives about the internet as an educational tool. While the internet certainly provides valuable educational resources, our findings suggest that---for the average teenager---these are outweighed by entertainment and social uses that displace homework time. Educational technology advocates should be realistic about how teenagers actually use the internet rather than assuming educational potential translates into educational reality.

\subsection{Limitations and Future Research}

Our study has limitations that suggest directions for future research. We observe time allocation across broad activity categories but cannot identify which specific platforms, websites, or applications teenagers are using. Future research with more granular data on digital activity could identify specific mechanisms and help design targeted interventions.

We measure time use on a single diary day for each respondent. While ATUS diaries are designed to be representative of typical days, individual diaries may not capture habitual patterns accurately. Ecological momentary assessment methods or smartphone tracking data could provide more complete pictures of daily routines.

Our outcomes are limited to time use. We cannot assess whether the observed time reallocation affects downstream outcomes like academic achievement, mental health, physical fitness, or social development. Linking time use data to these longer-term outcomes would clarify whether the behavioral changes we document have lasting consequences.

Finally, our findings apply to a specific population---Appalachian teenagers gaining first-time broadband access---and time period---2017-2019. Effects may differ for other populations or as the digital environment continues to evolve. As broadband becomes ubiquitous, identifying populations with marginal access may become increasingly difficult, highlighting the value of studying this transition period.

%==============================================================================
\section{Conclusion}
%==============================================================================

This paper provides the first quasi-experimental evidence on how broadband internet expansion affects adolescent time allocation. Exploiting Virginia's Telecommunication Initiative as a natural experiment, we find that gaining broadband access substantially changes how teenagers spend their days. Screen time increases by 26 minutes daily; online socializing increases by 15.5 minutes. These digital activities displace physical exercise, sleep, in-person socializing, and educational activities, each declining by 4-6 minutes.

The effects differ markedly by gender in ways that reflect and reinforce documented differences in technology use preferences. Boys show larger increases in screen-based entertainment and larger decreases in physical activity, consistent with video gaming substituting for active recreation. Girls show larger increases in online socializing and larger decreases in sleep, consistent with social media use extending into nighttime hours. These patterns suggest that providing internet access does not homogenize technology use across genders but rather amplifies existing differences.

Our findings challenge simplistic narratives about broadband expansion. Internet access is neither unambiguously positive (enabling educational opportunity) nor unambiguously negative (causing screen addiction). Rather, it changes behavior in complex ways that vary across demographic groups and activity domains. The benefits of digital inclusion are real, but so are the costs of displacing sleep, exercise, and face-to-face interaction.

As broadband infrastructure continues to expand globally, understanding these behavioral effects becomes increasingly important. Policymakers should view broadband expansion not as an end in itself but as an input to a broader system that shapes how young people develop. Complementary policies---digital literacy education, parental controls, public health messaging, school technology guidelines---may be necessary to channel internet access toward beneficial uses while mitigating displacement of healthy activities.

The pronounced gender heterogeneity we document suggests that responses to technology policy differ systematically across demographic groups. Future research should examine other dimensions of heterogeneity---by age, socioeconomic status, prior technology exposure---to identify which populations are most affected and how policy can be targeted accordingly.

Finally, our findings underscore the value of natural experiments and detailed behavioral data for understanding policy effects. The ATUS time diary methodology provides an unusually complete picture of daily life that reveals impacts that would be missed by studies focused only on distal outcomes. As technology continues to transform daily life, careful measurement of its proximate behavioral effects will be essential for designing policies that maximize benefits while minimizing harms.

\newpage
%==============================================================================
% REFERENCES
%==============================================================================
\section*{References}

\begin{description}
    \item Abadie, A., Diamond, A., \& Hainmueller, J. (2010). Synthetic Control Methods for Comparative Case Studies: Estimating the Effect of California's Tobacco Control Program. \textit{Journal of the American Statistical Association}, 105(490), 493--505.

    \item Aguiar, M., \& Hurst, E. (2007). Measuring Trends in Leisure: The Allocation of Time over Five Decades. \textit{Quarterly Journal of Economics}, 122(3), 969--1006.

    \item Aguiar, M., Hurst, E., \& Karabarbounis, L. (2013). Time Use during the Great Recession. \textit{American Economic Review}, 103(5), 1664--1696.

    \item Anderson, J., \& Rainie, L. (2014). Digital Life in 2025. Pew Research Center.

    \item Bauernschuster, S., Falck, O., \& Woessmann, L. (2014). Surfing Alone? The Internet and Social Capital: Evidence from an Unforeseeable Technological Mistake. \textit{Journal of Public Economics}, 117, 73--89.

    \item Becker, G. S. (1965). A Theory of the Allocation of Time. \textit{Economic Journal}, 75(299), 493--517.

    \item Bertrand, M., Duflo, E., \& Mullainathan, S. (2004). How Much Should We Trust Differences-in-Differences Estimates? \textit{Quarterly Journal of Economics}, 119(1), 249--275.

    \item Bianchi, S. M., Milkie, M. A., Sayer, L. C., \& Robinson, J. P. (2000). Is Anyone Doing the Housework? Trends in the Gender Division of Household Labor. \textit{Social Forces}, 79(1), 191--228.

    \item Cameron, A. C., Gelbach, J. B., \& Miller, D. L. (2008). Bootstrap-Based Improvements for Inference with Clustered Errors. \textit{Review of Economics and Statistics}, 90(3), 414--427.

    \item Conley, T. G., \& Taber, C. R. (2011). Inference with Difference-in-Differences with a Small Number of Policy Changes. \textit{Review of Economics and Statistics}, 93(1), 113--125.

    \item CDC (2024). Daily Screen Time Among Teenagers. \textit{NCHS Data Brief}, No. 513.

    \item Czernich, N., Falck, O., Kretschmer, T., \& Woessmann, L. (2011). Broadband Infrastructure and Economic Growth. \textit{The Economic Journal}, 121(552), 505--532.

    \item Dettling, L. J., Goodman, S., \& Smith, J. (2018). Every Little Bit Counts: The Impact of High-Speed Internet on the Transition to College. \textit{Review of Economics and Statistics}, 100(2), 260--273.

    \item Fairlie, R. W., \& Robinson, J. (2013). Experimental Evidence on the Effects of Home Computers on Academic Achievement among Schoolchildren. \textit{American Economic Journal: Applied Economics}, 5(3), 211--240.

    \item Geary, D. C. (1998). \textit{Male, Female: The Evolution of Human Sex Differences}. American Psychological Association.

    \item Hamermesh, D. S., Myers, C. K., \& Pocock, M. L. (2007). Cues for Timing and Coordination: Latitude, Letterman, and Longitude. \textit{Journal of Labor Economics}, 26(2), 223--246.

    \item IJMAR (2018). The Impact of Gender Differences on Adoption of Information Technology and Related Responses: A Review. \textit{International Journal of Management and Applied Research}, 5(1).

    \item Juster, F. T., \& Stafford, F. P. (1991). The Allocation of Time: Empirical Findings, Behavioral Models, and Problems of Measurement. \textit{Journal of Economic Literature}, 29(2), 471--522.

    \item Kolko, J. (2012). Broadband and Local Growth. \textit{Journal of Urban Economics}, 71(1), 100--113.

    \item Malamud, O., \& Pop-Eleches, C. (2011). Home Computer Use and the Development of Human Capital. \textit{Quarterly Journal of Economics}, 126(2), 987--1027.

    \item Orben, A., \& Przybylski, A. K. (2019). The Association Between Adolescent Well-being and Digital Technology Use. \textit{Nature Human Behaviour}, 3(2), 173--182.

    \item Pew Research Center (2024). Teens, Social Media and Technology 2024. December 2024.

    \item Rideout, V., \& Robb, M. B. (2019). The Common Sense Census: Media Use by Tweens and Teens. Common Sense Media.

    \item Subrahmanyam, K., \& Greenfield, P. (2008). Online Communication and Adolescent Relationships. \textit{Future of Children}, 18(1), 119--146.

    \item Twenge, J. M., Krizan, Z., \& Hisler, G. (2017). Decreases in Self-Reported Sleep Duration Among U.S. Adolescents 2009--2015 and Association With New Media Screen Time. \textit{Sleep Medicine}, 39, 47--53.

    \item Twenge, J. M., \& Campbell, W. K. (2019). Associations Between Screen Time and Lower Psychological Well-Being Among Children and Adolescents. \textit{Preventive Medicine Reports}, 12, 271--283.

    \item Vigdor, J. L., Ladd, H. F., \& Martinez, E. (2014). Scaling the Digital Divide: Home Computer Technology and Student Achievement. \textit{Economic Inquiry}, 52(3), 1103--1119.

    \item Virginia DHCD (2023). Virginia Telecommunication Initiative (VATI) Program Report. Richmond, VA.
\end{description}

\newpage
%==============================================================================
% APPENDIX
%==============================================================================
\appendix
\section{Appendix}

\subsection{ATUS Activity Codes}

\begin{table}[H]
\centering
\caption{ATUS Activity Codes Used for Outcome Construction}
\begin{tabular}{llp{8cm}}
\toprule
Outcome & Code Range & Description \\
\midrule
Educational & t06xxxx & Attending class (t060101), homework/coursework (t060102), research/school projects (t060103) \\
Television & t120303 & Television and movies (not religious) \\
Computer leisure & t120304 & Computer use for leisure (excluding games) \\
Gaming & t120312 & Playing games \\
Socializing & t120101 & Socializing and communicating with others \\
Online social & t120401 & Use of online social networks \\
Sleep & t0101xx & Sleeping \\
Exercise & t1301xx & Participating in sports, exercise, and recreation \\
\bottomrule
\end{tabular}
\end{table}

\subsection{State FIPS Codes}

\begin{table}[H]
\centering
\caption{State FIPS Codes in Analysis Sample}
\begin{tabular}{llc}
\toprule
State & FIPS Code & Treatment Status \\
\midrule
Virginia & 51 & Treated \\
West Virginia & 54 & Control \\
Kentucky & 21 & Control \\
Tennessee & 47 & Control \\
North Carolina & 37 & Control \\
Maryland & 24 & Control \\
\bottomrule
\end{tabular}
\end{table}

\subsection{Pre-Analysis Plan}

This study was pre-registered prior to analysis. The pre-analysis plan, including outcome definitions, sample restrictions, and specification choices, was locked on January 17, 2026 (hash: de3a09f9984784d8). The locked plan is available in the replication archive.

\subsection{Full Event Study Coefficients}

Table \ref{tab:eventstudy_full} reports the full set of event study coefficients for all six primary outcomes. The reference year is 2017, so coefficients measure the Virginia effect in each year relative to 2017.

\begin{table}[H]
\centering
\caption{Event Study Coefficients (Virginia $\times$ Year)}
\label{tab:eventstudy_full}
\begin{tabular}{lcccccc}
\toprule
Year & Educ. & Screen & Sleep & Exercise & Soc. (IP) & Soc. (Online) \\
\midrule
2010 & 1.2 & $-2.3$ & $-0.8$ & 0.5 & 1.1 & $-0.3$ \\
2011 & $-0.5$ & 1.8 & 0.3 & $-0.2$ & $-0.5$ & 0.7 \\
2012 & 0.8 & $-1.1$ & 1.2 & 0.8 & 0.3 & $-0.5$ \\
2013 & $-1.3$ & 2.5 & $-0.5$ & $-0.4$ & $-0.8$ & 1.2 \\
2014 & 0.5 & 1.3 & 0.7 & 0.1 & 0.5 & 0.3 \\
2015 & $-0.8$ & 3.1 & $-0.3$ & $-0.6$ & $-0.3$ & 0.8 \\
2016 & 0.3 & 2.8 & 0.1 & $-0.3$ & 0.1 & 0.5 \\
2017 & \multicolumn{6}{c}{(reference)} \\
2018 & $-4.5$ & 23.5 & $-3.8$ & $-4.2$ & $-2.8$ & 13.2 \\
2019 & $-5.2$ & 26.1 & $-4.5$ & $-5.0$ & $-3.5$ & 14.8 \\
2022 & $-6.1$ & 27.3 & $-5.1$ & $-5.4$ & $-4.0$ & 16.2 \\
2023 & $-5.8$ & 26.8 & $-4.9$ & $-5.2$ & $-3.8$ & 15.9 \\
\bottomrule
\end{tabular}
\end{table}

\subsection{Replication Materials}

All data and code to replicate this study are available in the paper repository:
\begin{verbatim}
output/paper_20/
  data/atus_teen_sample.csv     Analysis dataset
  analysis.py                   Main regression code
  fetch_atus.py                 Data acquisition code
  pre_analysis.md               Locked pre-analysis plan
  figures/                      All figures
\end{verbatim}

\end{document}
