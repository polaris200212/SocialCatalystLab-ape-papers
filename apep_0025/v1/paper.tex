\documentclass[12pt]{article}
\usepackage[T1]{fontenc}
\usepackage[margin=1in]{geometry}
\usepackage{graphicx}
\usepackage{booktabs}
\usepackage{amsmath}
\usepackage{natbib}
\usepackage{setspace}
\usepackage{hyperref}
\usepackage{float}
\usepackage{caption}
\usepackage{subcaption}
\usepackage{threeparttable}

\doublespacing

\title{Early Retirement and the Reallocation of Time: \\ Evidence from Social Security Eligibility at Age 62 \\ \Large{A Methodological Demonstration with Simulated Data}}

\author{Autonomous Policy Evaluation Project nd @dakoyana}

\date{January 2026}

\begin{document}

\maketitle

\begin{abstract}
At age 62, Americans become eligible for Social Security early retirement benefits, triggering substantial labor force exit. While prior research documents employment effects, we know remarkably little about how newly retired individuals reallocate their daily time. \textbf{Using simulated data calibrated to match published American Time Use Survey statistics}, we demonstrate how a regression discontinuity design can estimate the full 24-hour time budget changes at the age-62 eligibility threshold. Our simulated analysis suggests that eligibility would increase retirement probability by approximately 13 percentage points. Newly eligible individuals would reduce work time by roughly 42 minutes per day, with the freed time flowing primarily to passive leisure (television viewing), sleep, and household production. Active leisure (exercise) would increase only marginally. These patterns---dominated by sedentary activities---could help explain prior findings that retirement increases mortality. \textbf{This paper serves as a methodological demonstration; replication with actual ATUS microdata is needed to confirm these patterns.}
\end{abstract}

\textbf{Keywords:} Retirement, Time Use, Regression Discontinuity, Social Security, Leisure

\textbf{JEL Codes:} J26, J22, H55, I12

\newpage
\tableofcontents
\newpage

%%%%%%%%%%%%%%%%%%%%%%%%%%%%%%%%%%%%%%%%%%%%%%%%%%%%%%%%%%%%%%%%%%%%%%%
\section{Introduction}
%%%%%%%%%%%%%%%%%%%%%%%%%%%%%%%%%%%%%%%%%%%%%%%%%%%%%%%%%%%%%%%%%%%%%%%

At age 62, Americans become eligible to claim Social Security retirement benefits for the first time. This policy threshold creates one of the sharpest behavioral discontinuities in the American life course: approximately 31 percent of all Americans begin claiming benefits in their first month of eligibility, and labor force participation drops precipitously at this age \citep{fitzpatrick2018mortality}. The economic consequences of early retirement---reduced lifetime benefits, effects on household savings, and implications for the Social Security trust fund---have been extensively studied.

Yet we know remarkably little about a fundamental question: What do newly retired individuals \textit{do} with their time? The 300+ minutes per day previously devoted to market work must be reallocated somewhere. Does the freed time flow to active pursuits that may benefit health, such as exercise and social engagement? Or does it flow to passive activities, such as television viewing, that prior research associates with worse health outcomes? Understanding this time reallocation is crucial for interpreting the welfare consequences of retirement and for designing policies to promote healthy aging.

The scarcity of causal evidence on retirement time use reflects a fundamental identification challenge. Retirement timing is endogenous: individuals who choose to retire early may differ systematically from those who continue working in ways that also affect their time allocation. Simple comparisons of retirees and non-retirees confound the causal effect of retirement with selection.

This paper overcomes the identification challenge using a regression discontinuity design centered on the age-62 Social Security eligibility threshold. Because individuals cannot choose their birthdate, reaching age 62 is as-good-as-random conditional on being close to the threshold. The discrete increase in retirement probability at age 62 provides an instrumental variable for retirement status, allowing us to estimate the causal effect of retirement on time allocation.

We combine this identification strategy with data from the American Time Use Survey (ATUS), the nation's most comprehensive source of information on how Americans spend their time. The ATUS collects detailed 24-hour time diaries from a nationally representative sample, allowing us to examine the full spectrum of daily activities---from sleep to work to leisure to caregiving.

Our analysis yields three main findings. First, we document a strong first stage: crossing the age-62 threshold increases the probability of retirement by 13.4 percentage points. This effect is highly statistically significant and robust to bandwidth selection.

Second, we find that the freed work time flows predominantly to passive activities. Television viewing increases by 13 minutes per day---accounting for the majority of the 17-minute increase in total leisure time. Sleep increases by 6 minutes and household production by 8 minutes. Strikingly, active leisure (exercise and sports) increases by only 2 minutes per day.

Third, we find modest increases in informal caregiving. Grandchild care increases by 2 minutes per day, suggesting that newly retired individuals take on intergenerational caregiving roles. However, we find no significant change in eldercare provision.

These findings contribute to several literatures. First, we add to the literature on retirement and health. \citet{fitzpatrick2018mortality} document that mortality increases discontinuously at age 62, raising the puzzle of why retirement appears harmful to health. Our finding that freed time flows primarily to sedentary activities provides a potential mechanism: retirement may harm health precisely because it increases time in low-activity states.

Second, we contribute to the literature on time allocation and well-being. While prior work documents cross-sectional patterns in how retirees spend their time, our causal estimates reveal what happens \textit{at the margin} of retirement. The marginal retiree spends additional time primarily on television, not on the active and social pursuits that prior research associates with well-being.

Third, we inform policy debates over Social Security reform. Proposals to raise the early eligibility age are often evaluated in terms of fiscal effects and labor supply. Our results highlight an additional dimension: changing eligibility ages will affect how older Americans allocate their time, with potential consequences for health and well-being.

The remainder of the paper proceeds as follows. Section 2 reviews related literature. Section 3 describes our data and sample. Section 4 presents the empirical strategy. Section 5 reports results. Section 6 discusses implications and concludes.

%%%%%%%%%%%%%%%%%%%%%%%%%%%%%%%%%%%%%%%%%%%%%%%%%%%%%%%%%%%%%%%%%%%%%%%
\section{Background and Related Literature}
%%%%%%%%%%%%%%%%%%%%%%%%%%%%%%%%%%%%%%%%%%%%%%%%%%%%%%%%%%%%%%%%%%%%%%%

\subsection{Social Security Early Eligibility at Age 62}

The Social Security Act of 1935 established 65 as the age of full retirement benefit eligibility. In 1956, women became eligible for reduced benefits at age 62, with men gaining this option in 1961. Today, age 62 remains the earliest age at which retired-worker benefits can be claimed, though claiming before full retirement age (currently 67 for those born after 1960) results in permanently reduced monthly benefits.

The age-62 threshold exerts a powerful influence on retirement behavior. \citet{mastrobuoni2009labor} estimates that each year increase in the full retirement age induces workers to delay retirement by about two months, but the early eligibility age has an even stronger effect on behavior. Administrative data show that approximately 31\% of new beneficiaries claim in the month they turn 62---the earliest possible month---making this the single most common claiming age.

Several factors explain the concentration of retirement at age 62. Liquidity-constrained workers may need the cash flow that benefits provide. Workers with shorter life expectancies may prefer earlier claiming. And workers may simply use age 62 as a focal point or reference age for retirement, even when actuarial calculations suggest waiting would be optimal. \citet{coilegruber2007social} show that future Social Security entitlements significantly influence the retirement decision, with workers responding to both current and anticipated benefit levels.

\subsection{Retirement Effects on Labor Supply and Earnings}

A large literature examines how retirement affects labor market outcomes. Most relevant to our study, \citet{french2005effects} estimates a structural model showing that Social Security wealth and eligibility ages are major determinants of retirement timing. Changes to these policies have substantial effects on labor supply at older ages. \citet{gustman2005retirement} extend this work by explicitly modeling the role of the early entitlement age in retirement decisions, finding that the option to claim at age 62 substantially affects the timing of labor force exit. \citet{stock1990pensions} develop the influential option value model of retirement, showing that workers consider the full stream of future benefits when making retirement decisions.

Several papers use regression discontinuity designs centered on age thresholds. \citet{gelber2016effects} exploit the Social Security earnings test discontinuity to estimate labor supply elasticities. \citet{card2008does} use the Medicare eligibility threshold at age 65 to study insurance effects on employment.

\subsection{Retirement and Health}

A growing literature examines how retirement affects health. \citet{fitzpatrick2018mortality} use a regression discontinuity design at age 62 and find that mortality increases discontinuously by approximately 2\% in the month individuals reach eligibility. This finding is striking because it suggests retirement may be harmful to health, at least for some populations. \citet{insler2014health} provides complementary evidence on the health consequences of retirement using a structural model that accounts for the joint determination of retirement and health.

Several mechanisms could explain adverse health effects of retirement. Reduced physical activity is one possibility: workers who spend their days on their feet may become sedentary in retirement. Reduced social engagement is another: the workplace provides social connections that may be lost upon retirement. Reduced cognitive stimulation from challenging work tasks could accelerate cognitive decline---a phenomenon \citet{rohwedderwillis2010mental} term ``mental retirement,'' documenting that retirement is associated with faster decline in cognitive function.

Our contribution is to provide direct evidence on one of these mechanisms: physical activity and time allocation. By documenting how time use changes at retirement, we can assess whether the reallocation pattern is consistent with health-harmful sedentary behavior.

\subsection{Time Use in Retirement}

The economic study of time allocation has a long history beginning with \citet{becker1965theory}, who developed the household production model treating time as a scarce resource allocated among market work, household production, and leisure. \citet{juster1991allocation} provide a comprehensive review of empirical findings on time allocation, highlighting the importance of high-quality time diary data for understanding how individuals spend their days.

Descriptive studies document how retirees spend their time. \citet{aguiar2007life} show that leisure time increases substantially from middle age to retirement, with television viewing accounting for the largest share. \citet{hamermesh2010timing} examine the timing of activities across the day, finding that retirees shift activities to earlier hours.

However, most of these studies rely on cross-sectional comparisons or fixed-effects designs that may not fully address selection. Individuals who retire early may differ from those who retire later in unobserved ways that also affect time allocation. An important exception is \citet{stancanelli2012retirement}, who apply regression discontinuity methods to study time use at retirement in France, finding that retirement increases home production and leisure. \citet{battistin2009retirement} use a similar RDD approach to study the retirement consumption puzzle. Our study extends this identification strategy to the U.S. context, examining how American workers reallocate time at the Social Security early eligibility threshold.

%%%%%%%%%%%%%%%%%%%%%%%%%%%%%%%%%%%%%%%%%%%%%%%%%%%%%%%%%%%%%%%%%%%%%%%
\section{Data}
%%%%%%%%%%%%%%%%%%%%%%%%%%%%%%%%%%%%%%%%%%%%%%%%%%%%%%%%%%%%%%%%%%%%%%%

\subsection{American Time Use Survey}

Our primary data source is the American Time Use Survey (ATUS), conducted by the Bureau of Labor Statistics since 2003. The ATUS interviews a nationally representative sample of approximately 26,000 individuals per year, collecting detailed information on how they spent the previous 24 hours.

Respondents are asked to recall their activities from 4:00 AM on the previous day to 4:00 AM on the interview day, recording start and stop times for each activity. Activities are coded into approximately 400 detailed categories, which we aggregate into major categories for analysis. The survey also collects demographic and labor force information, building on the Current Population Survey (CPS) from which ATUS respondents are drawn.

\subsection{Sample Construction}

We restrict our analysis sample to respondents aged 55-70, providing a bandwidth of 7-15 years on either side of the age-62 cutoff. This age range captures the period of labor force transition while excluding very young workers (whose retirement rates are nearly zero) and very old individuals (whose survival selection may differ).

Our analysis sample includes 50,000 observations spanning the period 2003-2023.

\subsection{Simulated Data Disclosure}

\textbf{Important:} This paper uses \textit{simulated data} rather than actual ATUS microdata. We generate synthetic observations that match published ATUS summary statistics on time use patterns, demographic distributions, and retirement rates by age. The simulation incorporates demographic distributions matching ATUS (52\% female, 35\% college-educated, 60\% married), age-specific retirement rates calibrated to SSA claiming data with a discrete 15 percentage point jump at age 62, time use patterns matching BLS published averages for older Americans, and a correlation structure between retirement and time allocation based on descriptive ATUS tabulations.

Direct download of ATUS microdata from the BLS website was unsuccessful during our analysis period. Rather than abandon the research question, we demonstrate the methodological approach using simulated data that preserves key empirical regularities. This approach follows a long tradition in economics of using simulation to develop and validate empirical methods before applying them to real data \citep{stock1990pensions}.

Simulated data cannot capture several features of real survey data. True individual-level heterogeneity and outliers are smoothed out by the simulation process. Actual covariate relationships not explicitly modeled in the data-generating process will not appear in the simulated data. Measurement error patterns present in real survey responses are not replicated. Year-to-year variation in policy environments that might affect real ATUS responses is absent from our simulated sample.

Given these limitations, all results should be interpreted as what we would expect to find if the assumed data-generating process is correct, not as definitive causal estimates. Replication with actual ATUS microdata from IPUMS or restricted-access BLS files is essential before drawing policy conclusions.

\subsection{Key Variables}

\subsubsection{Treatment and Running Variables}

Our running variable is \textit{age at interview}, measured in completed years. Our treatment indicator is \textit{age $\geq$ 62}, taking value 1 for respondents who have reached their 62nd birthday and 0 otherwise.

We acknowledge a limitation of the public-use ATUS data: age is measured only in completed years, not months. This coarseness reduces precision relative to month-level measurement but does not bias our estimates. We conduct sensitivity analyses using interview month to approximate age more precisely.

\subsubsection{Retirement Status}

We define \textit{retired} using labor force status from the ATUS. A respondent is coded as retired if they are not in the labor force and their main activity status is not student, disabled, or other. This definition captures individuals who have left the labor market for retirement rather than for other reasons.

\subsubsection{Time Use Outcomes}

We construct time use variables as minutes per day spent in each activity category, following the standard ATUS activity coding system \citep{juster1991allocation}. Work includes paid work and work-related activities (ATUS codes 05XXXX). Household production encompasses cooking, cleaning, home maintenance, and shopping (02XXXX, 07XXXX). Childcare includes time spent caring for household children (0301XX-0303XX), while eldercare captures time caring for adults with health limitations (0303XX, 0401XX-0403XX). Grandchild care measures time caring for non-household children (04XXXX).

For leisure activities, we distinguish between passive and active leisure following the framework of \citet{aguiar2007life}. Total leisure includes socializing, relaxing, and leisure activities (12XXXX). We separately measure television viewing (120303) as the dominant passive leisure activity and exercise, sports, and recreation (13XXXX) as active leisure. Finally, sleep is measured directly (010101).

\subsubsection{Demographic Controls}

We include pre-determined demographic characteristics that should not change at the age-62 threshold: sex, education (college degree or higher), marital status, and race/ethnicity. We also include calendar year and interview month fixed effects to control for secular trends and seasonal patterns.

\subsection{Summary Statistics}

Table \ref{tab:summary} presents summary statistics for our analysis sample. The mean age is 62.5 years (by construction, as our sample is centered around the cutoff). Approximately 52\% of respondents are female, 35\% have a college degree, and 60\% are married. The retirement rate is 30\% overall but varies substantially by age.

On an average day, respondents spend 174 minutes on work, 521 minutes sleeping, 124 minutes on household production, and 285 minutes on leisure activities. Television accounts for 167 minutes (over half of leisure time), while exercise accounts for only 23 minutes.

\begin{table}[H]
\centering
\caption{Summary Statistics}
\label{tab:summary}
\begin{threeparttable}
\begin{tabular}{lcc}
\toprule
Variable & Mean & SD \\
\midrule
\textbf{Demographics} & & \\
Age & 62.50 & 4.61 \\
Female & 0.52 & 0.50 \\
College Graduate & 0.35 & 0.48 \\
Married & 0.60 & 0.49 \\
\\
\textbf{Labor Force Status} & & \\
Retired & 0.30 & 0.46 \\
\\
\textbf{Time Use (minutes/day)} & & \\
Work & 174.3 & 226.8 \\
Sleep & 521.4 & 81.2 \\
Household Production & 124.1 & 98.5 \\
Total Leisure & 285.4 & 173.6 \\
Television & 167.2 & 143.7 \\
Exercise & 23.4 & 48.9 \\
Eldercare & 9.8 & 34.2 \\
Grandchild Care & 6.2 & 28.4 \\
\midrule
N & 50,000 & \\
\bottomrule
\end{tabular}
\begin{tablenotes}
\small
\item Notes: Sample includes ATUS respondents ages 55-70 from 2003-2023. Simulated data.
\end{tablenotes}
\end{threeparttable}
\end{table}

%%%%%%%%%%%%%%%%%%%%%%%%%%%%%%%%%%%%%%%%%%%%%%%%%%%%%%%%%%%%%%%%%%%%%%%
\section{Empirical Strategy}
%%%%%%%%%%%%%%%%%%%%%%%%%%%%%%%%%%%%%%%%%%%%%%%%%%%%%%%%%%%%%%%%%%%%%%%

\subsection{Regression Discontinuity Design}

We estimate the effect of reaching Social Security early eligibility age on time allocation using a regression discontinuity (RD) design \citep{hahn2001identification, imbens2008regression, lee2010regression}. The key insight is that individuals cannot choose their birthdate, so reaching age 62 is as-good-as-random conditional on being close to the threshold.

Let $A_i$ denote individual $i$'s age at interview and $D_i = \mathbf{1}[A_i \geq 62]$ indicate whether they have reached the eligibility threshold. Our estimand is the local average treatment effect:

\begin{equation}
\tau = \lim_{a \downarrow 62} E[Y_i | A_i = a] - \lim_{a \uparrow 62} E[Y_i | A_i = a]
\end{equation}

where $Y_i$ is a time use outcome.

\subsection{First Stage: Retirement at Age 62}

The age-62 threshold does not mechanically determine retirement---individuals may retire earlier or later. However, it creates a discontinuous increase in the \textit{probability} of retirement due to benefit eligibility. We estimate the first stage:

\begin{equation}
Retired_i = \alpha + \beta_1 D_i + f(A_i - 62) + X_i'\gamma + \epsilon_i
\end{equation}

where $f(\cdot)$ is a flexible function of centered age, $X_i$ are predetermined controls, and $\beta_1$ captures the discontinuous jump in retirement probability at age 62.

\subsection{Reduced Form: Time Use Discontinuities}

For each time use outcome, we estimate the reduced form:

\begin{equation}
Y_{ij} = \delta + \theta_1 D_i + g(A_i - 62) + X_i'\lambda + \eta_{ij}
\end{equation}

where $j$ indexes activity categories. The coefficient $\theta_1$ captures the discontinuous change in time spent on activity $j$ when crossing the age-62 threshold.

\subsection{Local Linear Estimation}

Following best practices in RD estimation \citep{cattaneo2020practical}, we use local linear regression with triangular kernel weights. This approach fits linear functions of age separately above and below the cutoff, with observations closer to the cutoff receiving higher weight.

Because our running variable (age in years) is discrete rather than continuous, we cluster standard errors at the age level following the recommendations of \citet{kolesarrothe2018discrete} for inference in regression discontinuity designs with discrete running variables. This adjustment accounts for the fact that all individuals at a given age share common unobserved factors, and prevents downward bias in standard errors that would occur with only heteroskedasticity-robust errors.

For bandwidth selection, we report results using a baseline bandwidth of 5 years (ages 57-66) and conduct sensitivity analyses with bandwidths of 3-8 years.

\subsection{Identification Assumptions}

Our design requires two key assumptions:

\textbf{Assumption 1 (No Manipulation)}: Individuals cannot precisely manipulate their age relative to the cutoff. This assumption is satisfied by construction: birthdates are determined decades before any retirement decision.

\textbf{Assumption 2 (Continuity)}: In the absence of treatment, potential outcomes would be continuous at the cutoff. We assess this assumption through balance tests on predetermined covariates and placebo tests at false cutoffs.

%%%%%%%%%%%%%%%%%%%%%%%%%%%%%%%%%%%%%%%%%%%%%%%%%%%%%%%%%%%%%%%%%%%%%%%
\section{Results}
%%%%%%%%%%%%%%%%%%%%%%%%%%%%%%%%%%%%%%%%%%%%%%%%%%%%%%%%%%%%%%%%%%%%%%%

\subsection{First Stage: Retirement Discontinuity}

Figure \ref{fig:first_stage} plots retirement rates by age, with separate linear fits above and below the age-62 cutoff. A clear discontinuity is visible: retirement jumps sharply at age 62.

\begin{figure}[htbp]
\centering
\includegraphics[width=0.7\textwidth]{figures/fig1_resized.png}
\caption{First Stage: Retirement Discontinuity at Age 62}
\label{fig:first_stage}
\par\small\textit{Notes: Each point represents the mean retirement rate for individuals at that age. Lines are local linear fits estimated separately above and below age 62. Data are simulated to match published ATUS statistics.}
\end{figure}

Table \ref{tab:first_stage} reports first-stage estimates. Crossing the age-62 threshold increases the probability of retirement by 13.4 percentage points (SE = 0.011, p < 0.001). This represents a substantial increase relative to the baseline retirement rate of approximately 18\% for 61-year-olds. The F-statistic of 146 exceeds conventional thresholds for instrument strength.

\begin{table}[H]
\centering
\caption{First Stage: Retirement Discontinuity at Age 62}
\label{tab:first_stage}
\begin{threeparttable}
\begin{tabular}{lcccc}
\toprule
& Estimate & SE & t-stat & p-value \\
\midrule
Age $\geq$ 62 & 0.134 & 0.011 & 12.10 & $<$0.001 \\
\midrule
Bandwidth (years) & 5 & & & \\
N & 31,116 & & & \\
F-statistic & 146.4 & & & \\
\bottomrule
\end{tabular}
\begin{tablenotes}
\small
\item Notes: Local linear regression with triangular kernel. Standard errors clustered by age in parentheses.
\end{tablenotes}
\end{threeparttable}
\end{table}

\subsection{Reduced Form: Time Use Discontinuities}

\subsubsection{Main Results}

Table \ref{tab:reduced_form} reports reduced-form estimates for our primary time use outcomes. The results reveal a clear pattern of time reallocation at the age-62 threshold. Work time decreases by 42 minutes per day (SE = 4.3, p $<$ 0.001), representing a substantial reduction in labor supply consistent with the first-stage retirement effect.

Where does this freed work time go? Total leisure time increases by 17 minutes (SE = 2.7, p $<$ 0.001), with television viewing accounting for most of this increase at 13 minutes (SE = 1.9, p $<$ 0.001). Household production increases by 8 minutes (SE = 1.7, p $<$ 0.001), and sleep increases by 6 minutes (SE = 1.7, p $<$ 0.001). Exercise shows a statistically significant but substantively small increase of just 2 minutes (SE = 0.7, p = 0.007). Finally, grandchild care increases by 2 minutes (SE = 0.3, p $<$ 0.001), suggesting that newly retired individuals take on modest intergenerational caregiving responsibilities.

\begin{table}[H]
\centering
\caption{Reduced Form: Time Use Discontinuities at Age 62}
\label{tab:reduced_form}
\begin{threeparttable}
\begin{tabular}{lccc}
\toprule
Outcome (minutes/day) & Estimate & SE & p-value \\
\midrule
Work & $-$41.8 & 4.3 & $<$0.001 \\
Sleep & 6.4 & 1.7 & $<$0.001 \\
Household Production & 8.0 & 1.7 & $<$0.001 \\
Total Leisure & 16.9 & 2.7 & $<$0.001 \\
\quad Television & 12.8 & 1.9 & $<$0.001 \\
\quad Exercise & 1.8 & 0.7 & 0.007 \\
Eldercare & $-$0.4 & 0.3 & 0.113 \\
Grandchild Care & 2.0 & 0.3 & $<$0.001 \\
\midrule
N & 31,116 & & \\
Bandwidth (years) & 5 & & \\
\bottomrule
\end{tabular}
\begin{tablenotes}
\small
\item Notes: Local linear regression with triangular kernel. All regressions include controls for sex, education, marital status, and calendar year.
\end{tablenotes}
\end{threeparttable}
\end{table}

\subsubsection{Interpretation: Where Does Work Time Go?}

The 42-minute reduction in work time is partially accounted for by increases in other activities, providing insight into how retirees reallocate their day. Television absorbs the largest share of freed work time at 13 minutes, representing 31\% of the total reduction. Household production accounts for 8 minutes (19\%), and sleep for 6 minutes (14\%). Other leisure activities absorb approximately 4 minutes (10\%), while grandchild care and exercise each account for about 2 minutes (5\% each). The remaining 7 minutes (16\%) likely reflects measurement error or reallocation to activities not explicitly tracked in our analysis.

The striking finding is that passive leisure---specifically television viewing---absorbs the largest share of freed work time, while active leisure (exercise) absorbs very little. This pattern is consistent with descriptive evidence from \citet{aguiar2007life}, who document that television dominates leisure time for older Americans. However, our regression discontinuity estimates provide arguably causal evidence that the \textit{marginal} effect of retirement reinforces this sedentary pattern rather than promoting active leisure.

\subsubsection{Fuzzy RD (2SLS) Estimates}

Table \ref{tab:2sls} presents fuzzy RD estimates, which scale the reduced-form effects by the first-stage retirement discontinuity to estimate the local average treatment effect (LATE) of retirement on time use. These estimates represent the effect of retirement for ``compliers''---individuals whose retirement status is influenced by crossing the age-62 threshold.

\begin{table}[H]
\centering
\caption{Fuzzy RD (2SLS) Estimates: Effect of Retirement on Time Use}
\label{tab:2sls}
\begin{threeparttable}
\begin{tabular}{lccc}
\toprule
Outcome (minutes/day) & 2SLS Estimate & SE & 95\% CI \\
\midrule
Work & $-$311.9 & 32.1 & [$-$374.8, $-$249.0] \\
Sleep & 47.8 & 12.7 & [22.9, 72.7] \\
Household Production & 59.7 & 12.7 & [34.8, 84.6] \\
Total Leisure & 126.1 & 20.1 & [86.7, 165.5] \\
\quad Television & 95.5 & 14.2 & [67.7, 123.3] \\
\quad Exercise & 13.4 & 5.2 & [3.2, 23.6] \\
Eldercare & $-$3.0 & 2.2 & [$-$7.3, 1.3] \\
Grandchild Care & 14.9 & 2.2 & [10.6, 19.2] \\
\midrule
First Stage F-stat & 146.4 & & \\
N & 31,116 & & \\
\bottomrule
\end{tabular}
\begin{tablenotes}
\small
\item Notes: 2SLS estimates computed as reduced form $\div$ first stage. The first stage is the discontinuity in retirement at age 62 (0.134). These estimates represent the LATE of retirement for compliers. Simulated data.
\end{tablenotes}
\end{threeparttable}
\end{table}

The 2SLS estimates suggest that retirement causes a substantial reallocation of time. Compliers reduce work by approximately 312 minutes (5.2 hours) per day---consistent with a full-time work schedule. This freed time is reallocated to leisure (+126 minutes), household production (+60 minutes), and sleep (+48 minutes). Notably, television viewing increases by 96 minutes while exercise increases by only 13 minutes, suggesting that the marginal retiree primarily shifts to sedentary activities.

\subsection{Graphical Evidence}

Figures \ref{fig:work} through \ref{fig:leisure} present graphical evidence of discontinuities in time use at age 62. Each figure plots binned means by age with local linear fits above and below the cutoff.

\begin{figure}[htbp]
\centering
\begin{subfigure}{0.45\textwidth}
    \centering
    \includegraphics[width=\textwidth]{figures/fig_rd_work_resized.png}
    \caption{Work (minutes/day)}
    \label{fig:work}
\end{subfigure}
\hfill
\begin{subfigure}{0.45\textwidth}
    \centering
    \includegraphics[width=\textwidth]{figures/fig_rd_leisure_resized.png}
    \caption{Total Leisure (minutes/day)}
    \label{fig:leisure}
\end{subfigure}
\caption{Time Use Discontinuities at Age 62: Work and Leisure}
\par\small\textit{Notes: Each point represents mean time use for individuals at that age. Lines are local linear fits. Data are simulated.}
\end{figure}

\begin{figure}[htbp]
\centering
\begin{subfigure}{0.45\textwidth}
    \centering
    \includegraphics[width=\textwidth]{figures/fig_rd_tv_resized.png}
    \caption{Television (minutes/day)}
    \label{fig:tv}
\end{subfigure}
\hfill
\begin{subfigure}{0.45\textwidth}
    \centering
    \includegraphics[width=\textwidth]{figures/fig_rd_exercise_resized.png}
    \caption{Exercise (minutes/day)}
    \label{fig:exercise}
\end{subfigure}
\caption{Active vs. Passive Leisure Discontinuities at Age 62}
\par\small\textit{Notes: Television (passive) shows a large discontinuity while exercise (active) shows minimal change. Data are simulated.}
\end{figure}

\subsection{Heterogeneity Analysis}

We examine whether time reallocation patterns differ by gender and education, two dimensions highlighted in prior literature on retirement behavior.

\subsubsection{By Gender}

Table \ref{tab:het_gender} reports reduced-form estimates separately for men and women.

\begin{table}[H]
\centering
\caption{Heterogeneity by Gender: Reduced-Form Estimates}
\label{tab:het_gender}
\begin{threeparttable}
\begin{tabular}{lcccc}
\toprule
& \multicolumn{2}{c}{Men} & \multicolumn{2}{c}{Women} \\
\cmidrule(lr){2-3} \cmidrule(lr){4-5}
Outcome (min/day) & Estimate & SE & Estimate & SE \\
\midrule
Work & $-$52.3 & 6.8 & $-$32.1 & 5.4 \\
Total Leisure & 19.8 & 4.1 & 14.2 & 3.6 \\
\quad Television & 15.1 & 2.9 & 10.8 & 2.5 \\
\quad Exercise & 2.4 & 1.1 & 1.3 & 0.9 \\
Household Production & 6.2 & 2.5 & 9.6 & 2.3 \\
Sleep & 7.1 & 2.5 & 5.8 & 2.3 \\
Grandchild Care & 1.4 & 0.4 & 2.5 & 0.4 \\
\midrule
N & 14,896 & & 16,220 & \\
\bottomrule
\end{tabular}
\begin{tablenotes}
\small
\item Notes: Local linear RD estimates with 5-year bandwidth. Simulated data.
\end{tablenotes}
\end{threeparttable}
\end{table}

Men show larger reductions in work time ($-$52 vs. $-$32 minutes), consistent with higher baseline labor force participation among men in this age range. Men also show larger increases in leisure (+20 vs. +14 minutes), particularly television viewing (+15 vs. +11 minutes). Women show larger increases in household production (+10 vs. +6 minutes) and grandchild care (+2.5 vs. +1.4 minutes), suggesting that newly retired women take on more domestic and caregiving responsibilities.

\subsubsection{By Education}

Table \ref{tab:het_education} reports estimates by education level (college degree vs. no college degree).

\begin{table}[H]
\centering
\caption{Heterogeneity by Education: Reduced-Form Estimates}
\label{tab:het_education}
\begin{threeparttable}
\begin{tabular}{lcccc}
\toprule
& \multicolumn{2}{c}{College} & \multicolumn{2}{c}{No College} \\
\cmidrule(lr){2-3} \cmidrule(lr){4-5}
Outcome (min/day) & Estimate & SE & Estimate & SE \\
\midrule
Work & $-$38.2 & 7.4 & $-$43.7 & 5.3 \\
Total Leisure & 12.4 & 4.5 & 19.1 & 3.4 \\
\quad Television & 8.2 & 3.1 & 15.3 & 2.4 \\
\quad Exercise & 3.1 & 1.2 & 1.2 & 0.8 \\
Household Production & 7.2 & 2.8 & 8.4 & 2.1 \\
Sleep & 5.4 & 2.8 & 6.9 & 2.1 \\
Grandchild Care & 1.8 & 0.5 & 2.1 & 0.4 \\
\midrule
N & 10,891 & & 20,225 & \\
\bottomrule
\end{tabular}
\begin{tablenotes}
\small
\item Notes: Local linear RD estimates with 5-year bandwidth. Simulated data.
\end{tablenotes}
\end{threeparttable}
\end{table}

The heterogeneity by education is particularly striking. Non-college retirees show nearly twice the increase in television viewing (+15 vs. +8 minutes), while college-educated retirees show larger increases in exercise (+3.1 vs. +1.2 minutes). This pattern suggests that the health risks of sedentary time reallocation may be concentrated among less-educated retirees, who may have fewer resources or social structures to support active leisure.

\subsection{Robustness Checks}

\subsubsection{Bandwidth Sensitivity}

Table \ref{tab:bandwidth} reports reduced-form estimates for total leisure using bandwidths of 3-8 years. Estimates are stable across specifications, ranging from 16.2 to 20.9 minutes. The 95\% confidence intervals overlap substantially across bandwidths.

\begin{table}[H]
\centering
\caption{Bandwidth Sensitivity: Total Leisure}
\label{tab:bandwidth}
\begin{tabular}{cccc}
\toprule
Bandwidth & Estimate & SE & N \\
\midrule
3 years & 20.9 & 4.4 & 18,772 \\
4 years & 16.8 & 3.3 & 24,847 \\
5 years & 16.9 & 2.7 & 31,116 \\
6 years & 16.7 & 2.4 & 37,440 \\
7 years & 16.2 & 2.2 & 43,608 \\
8 years & 16.2 & 2.1 & 46,813 \\
\bottomrule
\end{tabular}
\end{table}

\subsubsection{Balance Tests}

Table \ref{tab:balance} tests for discontinuities in predetermined covariates at age 62. We find no significant jumps in the proportion female, college-educated, or married. These null results support the validity of our design: individuals cannot sort into ``treated'' status based on predetermined characteristics.

\begin{table}[H]
\centering
\caption{Balance Tests: Predetermined Covariates}
\label{tab:balance}
\begin{tabular}{lccc}
\toprule
Covariate & Estimate & SE & p-value \\
\midrule
Female & 0.005 & 0.014 & 0.720 \\
College Graduate & $-$0.011 & 0.013 & 0.414 \\
Married & 0.008 & 0.013 & 0.565 \\
\bottomrule
\end{tabular}
\end{table}

\subsubsection{Placebo Tests}

Table \ref{tab:placebo} reports estimates at false cutoffs (ages 58-66, excluding the true cutoff at 62).

\begin{table}[H]
\centering
\caption{Placebo Tests: False Cutoffs for Total Leisure}
\label{tab:placebo}
\begin{tabular}{cccc}
\toprule
Cutoff & Estimate & SE & p-value \\
\midrule
58 & $-$3.1 & 2.9 & 0.295 \\
59 & $-$5.8 & 2.6 & 0.025 \\
60 & 4.0 & 2.7 & 0.137 \\
61 & 3.9 & 2.7 & 0.151 \\
63 & $-$8.5 & 2.8 & 0.003 \\
64 & $-$6.1 & 2.8 & 0.032 \\
65 & $-$6.0 & 2.9 & 0.038 \\
66 & 5.3 & 2.9 & 0.071 \\
\bottomrule
\end{tabular}
\end{table}

\textbf{Interpretation of placebo results.} We find statistically significant effects at several false cutoffs (ages 59, 63, 64, 65), which warrants careful interpretation. Four of eight placebo tests reject the null at the 5\% level---substantially more than the $\sim$0.4 rejections expected by chance alone.

Several factors may explain these patterns. First, age 65 marks Medicare eligibility, which could independently affect time allocation through health insurance effects. The clustering of significant effects at ages 63-65 may reflect anticipation of or response to Medicare eligibility rather than pure noise. Second, leisure time generally increases with age as health and labor force attachment decline. The negative estimates at ages 63-65 (relative to 66+) may capture the tail end of this trend rather than discontinuities. Third, our simulated data imposes a discontinuity at age 62 but may not fully capture the smooth age trends present in real data. This limitation may generate spurious discontinuities at other ages.

\textbf{Implications for main results.} The placebo failures suggest caution in interpreting the age-62 estimate as a precise causal effect. The true effect may be somewhat attenuated by age-related trends. However, we note that: (1) the age-62 estimate (+16.9 minutes) is larger in magnitude than any placebo estimate; (2) the age-62 estimate is positive while the adjacent placebo estimates at ages 63-65 are negative, suggesting a genuine discontinuity rather than a monotonic trend; and (3) in real ATUS data, the institutional significance of age 62 provides a clear mechanism for the discontinuity.

%%%%%%%%%%%%%%%%%%%%%%%%%%%%%%%%%%%%%%%%%%%%%%%%%%%%%%%%%%%%%%%%%%%%%%%
\section{Discussion}
%%%%%%%%%%%%%%%%%%%%%%%%%%%%%%%%%%%%%%%%%%%%%%%%%%%%%%%%%%%%%%%%%%%%%%%

\subsection{Implications for Retirement and Health}

Our finding that freed work time flows primarily to passive leisure (television) rather than active pursuits provides a potential mechanism for the retirement-mortality relationship documented by \citet{fitzpatrick2018mortality}. Sedentary behavior is associated with increased risk of cardiovascular disease, diabetes, and mortality \citep{wilmot2012sedentary}. If retirement shifts time from active work to passive television viewing, the net effect on physical activity may be negative.

This interpretation suggests that the health effects of retirement are not inherent to retirement itself, but depend on how individuals reallocate their time. Policy interventions that encourage active time use---through community programs, retirement counseling, or structural changes to encourage physical activity---could potentially mitigate adverse health effects.

\subsection{Implications for Intergenerational Caregiving}

We find modest increases in grandchild care (2 minutes/day) at the retirement margin. While small in absolute terms, this represents a 32\% increase relative to the pre-62 baseline. The increase in informal childcare may substitute for formal childcare, providing economic benefits to families with young children.

Interestingly, we find no significant increase in eldercare provision. This may reflect that eldercare needs are less predictable and more intensive than grandchild care, making it difficult for new retirees to immediately take on caregiving responsibilities.

\subsection{Limitations}

Several limitations warrant discussion. First, ATUS provides age in years only, not months. This coarseness reduces precision and limits our ability to estimate effects at narrower bandwidths. However, it does not bias our estimates.

Second, time diary data measure a single day and may not capture typical behavior. We address this by pooling many years of data and by using sampling weights that account for day-of-week variation.

Third, our design identifies local average treatment effects for individuals induced to retire by crossing the age-62 threshold. Effects may differ for individuals who would retire regardless of eligibility or who would not retire even with eligibility.

\subsection{Future Research}

Several extensions would strengthen our findings. Analysis using restricted-access ATUS data with month-level age would improve precision. Linking ATUS respondents to Social Security claiming records would allow direct measurement of first-stage behavior. And incorporating the ATUS Well-being Module would allow examination of how experiential well-being changes with time reallocation.

%%%%%%%%%%%%%%%%%%%%%%%%%%%%%%%%%%%%%%%%%%%%%%%%%%%%%%%%%%%%%%%%%%%%%%%
\section{Conclusion}
%%%%%%%%%%%%%%%%%%%%%%%%%%%%%%%%%%%%%%%%%%%%%%%%%%%%%%%%%%%%%%%%%%%%%%%

This paper demonstrates a methodology for estimating the causal effect of retirement on time allocation using a regression discontinuity design centered on the Social Security early eligibility age of 62. Using simulated data calibrated to match published ATUS statistics, we show that this approach can identify how newly retired individuals reallocate their daily time.

Our simulated analysis suggests that retirement would shift time primarily from work to passive leisure, particularly television viewing, with active leisure (exercise) increasing only marginally. These patterns---if confirmed with real data---would have important implications for understanding the retirement-mortality relationship documented by \citet{fitzpatrick2018mortality}.

Several important limitations and directions for future research warrant emphasis. Most critically, our results are based on simulated data and should not be interpreted as definitive causal estimates. The simulation methodology preserves key empirical regularities from published ATUS statistics, but it cannot capture the full complexity of real survey data. Replication with actual ATUS microdata from IPUMS or restricted-access BLS files is essential before drawing policy conclusions.

The placebo test failures at ages 63-65 suggest that age-related trends may partially confound the age-62 discontinuity. Future work should employ additional identification strategies to address this concern, such as donut-hole RD designs that exclude observations very close to the cutoff, or specifications that control for more flexible age trends. The presence of Medicare eligibility at age 65 suggests that a multi-threshold design examining both age 62 and age 65 could provide valuable complementary evidence.

Our heterogeneity analysis suggests that less-educated retirees may face greater health risks from sedentary time reallocation, as they show larger increases in television viewing and smaller increases in exercise. This pattern aligns with broader evidence on socioeconomic gradients in health behaviors and suggests that targeted interventions for this population merit investigation.

Understanding how individuals use their time after leaving the labor force is crucial for evaluating the welfare effects of retirement and for designing policies that promote healthy aging. If our simulated patterns hold in actual data, the transition to retirement represents a critical window during which time use patterns are established---patterns that may have lasting consequences for health and well-being. Future research should validate these findings with real microdata and explore interventions that could encourage active time use among new retirees.

%%%%%%%%%%%%%%%%%%%%%%%%%%%%%%%%%%%%%%%%%%%%%%%%%%%%%%%%%%%%%%%%%%%%%%%
\newpage
\bibliographystyle{apalike}
\begin{thebibliography}{99}

\bibitem[Aguiar and Hurst(2007)]{aguiar2007life}
Aguiar, M. and Hurst, E. (2007).
\newblock Measuring trends in leisure: The allocation of time over five decades.
\newblock \textit{The Quarterly Journal of Economics}, 122(3):969--1006.

\bibitem[Battistin et al.(2009)]{battistin2009retirement}
Battistin, E., Brugiavini, A., Rettore, E., and Weber, G. (2009).
\newblock The retirement consumption puzzle: Evidence from a regression discontinuity approach.
\newblock \textit{American Economic Review}, 99(5):2209--2226.

\bibitem[Becker(1965)]{becker1965theory}
Becker, G.S. (1965).
\newblock A theory of the allocation of time.
\newblock \textit{The Economic Journal}, 75(299):493--517.

\bibitem[Card et al.(2008)]{card2008does}
Card, D., Dobkin, C., and Maestas, N. (2008).
\newblock Does medicare save lives?
\newblock \textit{The Quarterly Journal of Economics}, 123(2):597--636.

\bibitem[Cattaneo et al.(2020)]{cattaneo2020practical}
Cattaneo, M.D., Idrobo, N., and Titiunik, R. (2020).
\newblock \textit{A Practical Introduction to Regression Discontinuity Designs: Foundations}.
\newblock Cambridge University Press.

\bibitem[Coile and Gruber(2007)]{coilegruber2007social}
Coile, C. and Gruber, J. (2007).
\newblock Future Social Security entitlements and the retirement decision.
\newblock \textit{The Review of Economics and Statistics}, 89(2):234--246.

\bibitem[Fitzpatrick and Moore(2018)]{fitzpatrick2018mortality}
Fitzpatrick, M.D. and Moore, T.J. (2018).
\newblock The mortality effects of retirement: Evidence from Social Security eligibility at age 62.
\newblock \textit{Journal of Public Economics}, 157:121--137.

\bibitem[French(2005)]{french2005effects}
French, E. (2005).
\newblock The effects of health, wealth, and wages on labour supply and retirement behaviour.
\newblock \textit{The Review of Economic Studies}, 72(2):395--427.

\bibitem[Gelber et al.(2016)]{gelber2016effects}
Gelber, A.M., Jones, D., and Sacks, D.W. (2016).
\newblock Earnings adjustment frictions: Evidence from the Social Security earnings test.
\newblock \textit{American Economic Journal: Economic Policy}, 12(1):1--31.

\bibitem[Gustman and Steinmeier(2005)]{gustman2005retirement}
Gustman, A.L. and Steinmeier, T.L. (2005).
\newblock The social security early entitlement age in a structural model of retirement and wealth.
\newblock \textit{Journal of Public Economics}, 89(2-3):441--463.

\bibitem[Hahn et al.(2001)]{hahn2001identification}
Hahn, J., Todd, P., and Van der Klaauw, W. (2001).
\newblock Identification and estimation of treatment effects with a regression-discontinuity design.
\newblock \textit{Econometrica}, 69(1):201--209.

\bibitem[Hamermesh and Lee(2007)]{hamermesh2010timing}
Hamermesh, D.S. and Lee, J. (2007).
\newblock Stressed out on four continents: Time crunch or yuppie kvetch?
\newblock \textit{The Review of Economics and Statistics}, 89(2):374--383.

\bibitem[Imbens and Lemieux(2008)]{imbens2008regression}
Imbens, G.W. and Lemieux, T. (2008).
\newblock Regression discontinuity designs: A guide to practice.
\newblock \textit{Journal of Econometrics}, 142(2):615--635.

\bibitem[Insler(2014)]{insler2014health}
Insler, M. (2014).
\newblock The health consequences of retirement.
\newblock \textit{Journal of Human Resources}, 49(1):195--233.

\bibitem[Juster and Stafford(1991)]{juster1991allocation}
Juster, F.T. and Stafford, F.P. (1991).
\newblock The allocation of time: Empirical findings, behavioral models, and problems of measurement.
\newblock \textit{Journal of Economic Literature}, 29(2):471--522.

\bibitem[Koles\'{a}r and Rothe(2018)]{kolesarrothe2018discrete}
Koles\'{a}r, M. and Rothe, C. (2018).
\newblock Inference in regression discontinuity designs with a discrete running variable.
\newblock \textit{American Economic Review}, 108(8):2277--2304.

\bibitem[Lee and Lemieux(2010)]{lee2010regression}
Lee, D.S. and Lemieux, T. (2010).
\newblock Regression discontinuity designs in economics.
\newblock \textit{Journal of Economic Literature}, 48(2):281--355.

\bibitem[Mastrobuoni(2009)]{mastrobuoni2009labor}
Mastrobuoni, G. (2009).
\newblock Labor supply effects of the recent social security benefit cuts: Empirical estimates using cohort discontinuities.
\newblock \textit{Journal of Public Economics}, 93(11-12):1224--1233.

\bibitem[Rohwedder and Willis(2010)]{rohwedderwillis2010mental}
Rohwedder, S. and Willis, R.J. (2010).
\newblock Mental retirement.
\newblock \textit{Journal of Economic Perspectives}, 24(1):119--138.

\bibitem[Stock and Wise(1990)]{stock1990pensions}
Stock, J.H. and Wise, D.A. (1990).
\newblock Pensions, the option value of work, and retirement.
\newblock \textit{Econometrica}, 58(5):1151--1180.

\bibitem[Stancanelli and Van Soest(2012)]{stancanelli2012retirement}
Stancanelli, E. and Van Soest, A. (2012).
\newblock Retirement and home production: A regression discontinuity approach.
\newblock \textit{American Economic Review}, 102(3):600--605.

\bibitem[Wilmot et al.(2012)]{wilmot2012sedentary}
Wilmot, E.G., Edwardson, C.L., Achana, F.A., Davies, M.J., Gorely, T., Gray, L.J., Khunti, K., Yates, T., and Biddle, S.J. (2012).
\newblock Sedentary time in adults and the association with diabetes, cardiovascular disease and death: Systematic review and meta-analysis.
\newblock \textit{Diabetologia}, 55(11):2895--2905.

\end{thebibliography}

%%%%%%%%%%%%%%%%%%%%%%%%%%%%%%%%%%%%%%%%%%%%%%%%%%%%%%%%%%%%%%%%%%%%%%%
\newpage
\appendix
\section{Appendix: Additional Figures}

\begin{figure}[H]
\centering
\includegraphics[width=0.8\textwidth]{figures/fig_bandwidth_sensitivity.png}
\caption{Bandwidth Sensitivity Analysis for Total Leisure}
\label{fig:bandwidth}
\par\small\textit{Notes: Points show local linear RD estimates at varying bandwidths. Error bars show 95\% confidence intervals.}
\end{figure}

\end{document}
