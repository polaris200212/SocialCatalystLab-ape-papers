\documentclass[12pt]{article}

% UTF-8 encoding and fonts
\usepackage[utf8]{inputenc}
\usepackage[T1]{fontenc}
\usepackage{lmodern}

% Page setup
\usepackage[margin=1in]{geometry}
\usepackage{setspace}
\onehalfspacing

% Math and symbols
\usepackage{amsmath,amssymb}

% Graphics
\usepackage{graphicx}
\usepackage{float}

% Tables
\usepackage{booktabs}
\usepackage{array}
\usepackage{multirow}

% Bibliography
\usepackage{natbib}
\bibliographystyle{aer}

% Hyperlinks
\usepackage{hyperref}
\hypersetup{
    colorlinks=true,
    linkcolor=blue,
    citecolor=blue,
    urlcolor=blue
}

% Captions
\usepackage{caption}
\captionsetup{font=small,labelfont=bf}

% Section formatting
\usepackage{titlesec}
\titleformat{\section}{\large\bfseries}{\thesection.}{0.5em}{}
\titleformat{\subsection}{\normalsize\bfseries}{\thesubsection}{0.5em}{}

% Custom commands
\newcommand{\E}{\mathbb{E}}
\newcommand{\Var}{\text{Var}}
\newcommand{\Cov}{\text{Cov}}

\title{Self-Employment and Health Insurance Coverage in the Post-ACA Era:\\ Evidence from the American Community Survey}
\author{APEP Autonomous Research\thanks{Autonomous Policy Evaluation Project (APEP). This paper was produced autonomously by Claude Code. All data are from the U.S. Census Bureau American Community Survey Public Use Microdata Sample (2022). Replication materials: \url{https://github.com/SocialCatalystLab/auto-policy-evals}}}
\date{\today}

\begin{document}

\maketitle

\begin{abstract}
\noindent
Does self-employment reduce health insurance coverage compared to traditional wage employment in the post-Affordable Care Act era? Using data on 1.3 million workers from the 2022 American Community Survey, I estimate that self-employed workers are 6.1 percentage points less likely to have any health insurance coverage compared to observationally similar wage workers---a relative gap of 6.6\%. This coverage deficit operates through dramatically lower employer-sponsored insurance rates (27.2 pp, or 36\%), partially offset by higher direct-purchase coverage (+18.3 pp, which includes both ACA Marketplace and off-exchange individual plans) and Medicaid enrollment (+3.2 pp). The coverage gap is substantially smaller in Medicaid expansion states (6.4 pp vs. 10.1 pp). The gap is smallest for lowest-income workers (1.6 pp) who qualify for Medicaid and highest-income workers (4.6 pp) who can afford direct-purchase coverage, while middle-income workers face the largest penalty (9.5 pp). These patterns suggest that Medicaid expansion has been particularly effective in protecting low-income self-employed workers. However, a persistent coverage penalty remains for middle-income workers, indicating that the ACA's reforms have not fully equalized access.
\end{abstract}

\vspace{1em}
\noindent\textbf{JEL Codes:} I13, J32, L26 \\
\noindent\textbf{Keywords:} self-employment, health insurance, Affordable Care Act, Marketplace, gig economy

\newpage

\section{Introduction}

The rise of self-employment, independent contracting, and gig work has transformed the American labor market. By 2022, approximately 10\% of employed workers were self-employed, encompassing both traditional small business owners and the growing ranks of platform-based independent contractors. Unlike wage workers, these individuals lack access to employer-sponsored health insurance---historically the primary source of coverage for working-age Americans. The Affordable Care Act (ACA) of 2010 promised to address this gap through individual insurance Marketplaces and expanded Medicaid eligibility, potentially decoupling health insurance from employment. Yet nearly a decade after full ACA implementation, it remains unclear whether these reforms have equalized insurance access between self-employed and wage workers.

This paper provides a comprehensive post-ACA analysis of the self-employment health insurance gap using the 2022 American Community Survey (ACS). I analyze 1.3 million employed workers aged 25-64, comparing insurance coverage outcomes between self-employed and wage workers while controlling for a rich set of demographic, labor market, and geographic characteristics. The core finding is striking: even after adjusting for observable differences, self-employed workers are 6.1 percentage points less likely to have any health insurance coverage than comparable wage workers. This represents a 6.6\% relative reduction from the 92.0\% baseline coverage rate among wage workers.

The mechanisms driving this gap are clear. Self-employed workers are 27.2 percentage points less likely to have employer-sponsored insurance---a relative decline of 36\%---reflecting their structural exclusion from group coverage. However, this deficit is partially offset through alternative channels: self-employed workers are 18.3 percentage points more likely to purchase coverage directly (including through ACA Marketplaces and the off-exchange individual market) and 3.2 percentage points more likely to receive Medicaid. These compensating mechanisms are incomplete, leaving a residual 6.1 pp coverage gap. Importantly, the gap varies substantially across policy environments and income levels. In Medicaid expansion states, the self-employment coverage penalty is 6.4 percentage points, compared to 10.1 percentage points in non-expansion states. The income pattern is striking: the penalty is smallest for the lowest-income quintile (1.6 pp), where Medicaid provides coverage, largest for middle-income quintiles (7.9-9.5 pp), and moderate for the highest-income quintile (4.6 pp). These patterns suggest that Medicaid expansion has been particularly effective in protecting low-income self-employed workers, while middle-income workers remain the most exposed.

This paper contributes to several literatures. First, it updates the literature on self-employment and health insurance, which has largely focused on the pre-ACA period. Classic studies documented substantial coverage gaps and found that lack of insurance access deterred entrepreneurship, a phenomenon termed ``entrepreneurship lock.'' The ACA's reforms were explicitly designed to reduce this friction, yet rigorous post-ACA evidence has been limited. Second, this paper contributes to the growing literature evaluating ACA implementation, particularly the Marketplaces and Medicaid expansion. By examining a population with structurally different insurance access---the self-employed---I provide a novel test of whether the ACA has successfully provided viable alternatives to employer-sponsored coverage. Third, this analysis is directly relevant to contemporary policy debates about portable benefits, gig worker classification, and the future of employer-based insurance.

The remainder of this paper proceeds as follows. Section 2 provides institutional background on the ACA reforms and self-employment trends. Section 3 reviews the related literature. Section 4 describes the data and empirical strategy. Section 5 presents the main results and robustness checks. Section 6 concludes with policy implications.

\section{Institutional Background}

\subsection{Health Insurance and Self-Employment Before the ACA}

Before the Affordable Care Act, the U.S. health insurance system was heavily oriented around employer-sponsored coverage. For working-age adults, employers provided the primary path to affordable group insurance, with favorable tax treatment and risk pooling across employees. Self-employed individuals faced a fundamentally different market. They could purchase individual insurance directly, but this market was plagued by several problems: insurers could deny coverage or charge higher premiums based on pre-existing conditions, policies were often limited in scope, and prices were substantially higher than group rates. As a result, self-employed workers faced both higher costs and greater coverage uncertainty.

Research from this period consistently documented a self-employment health insurance penalty. Studies found that self-employed workers were significantly less likely to have coverage than comparable wage workers, with gaps ranging from 10 to 20 percentage points depending on the population and time period studied. This gap had real consequences: it deterred potential entrepreneurs from leaving wage employment (``entrepreneurship lock''), led to underinsurance and medical debt among the self-employed, and contributed to job lock more broadly as workers remained in suboptimal jobs for insurance benefits.

\subsection{The Affordable Care Act Reforms}

The ACA transformed the individual insurance market through several key provisions. First, it established Health Insurance Marketplaces (originally called Exchanges) where individuals could shop for and purchase private insurance plans. These Marketplaces standardized plan design and facilitated comparison shopping. Second, premium tax credits were made available to households with incomes between 100\% and 400\% of the federal poverty level, making coverage substantially more affordable for moderate-income households. Third, the law prohibited insurers from denying coverage or charging higher premiums based on health status, ending the pre-existing condition exclusions that had made individual coverage inaccessible for many. Fourth, the Medicaid expansion provision allowed states to extend eligibility to adults with incomes up to 138\% of poverty---though the Supreme Court's 2012 decision made this expansion optional, resulting in uneven adoption across states.

For the self-employed, these reforms were particularly consequential. The Marketplace provided a viable alternative to the problematic pre-ACA individual market, with guaranteed issue and community rating ensuring access regardless of health status. Premium subsidies made coverage affordable for the many self-employed workers with moderate incomes. And Medicaid expansion, where adopted, provided a coverage option for lower-income self-employed individuals who previously had few options.

By 2022---our study year---the ACA's main provisions had been in effect for eight years and Marketplace enrollment had stabilized. As of January 2022, 38 states plus D.C. had implemented Medicaid expansion with coverage effective for adults (following the Kaiser Family Foundation definition of implemented expansion). This mature post-implementation period allows us to assess whether the ACA's reforms achieved their goal of providing viable insurance alternatives to the self-employed.

\subsection{Self-Employment Trends}

Self-employment encompasses a heterogeneous population. Traditional self-employment includes small business owners, professionals in private practice (lawyers, doctors, accountants), and farmers. The incorporated self-employed operate formal businesses and may be able to provide themselves employer-sponsored coverage through their companies. The unincorporated self-employed---a larger group---includes freelancers, consultants, independent contractors, and gig workers. This latter category has grown substantially with the rise of platform-based work (rideshare drivers, delivery workers, online freelancers).

In our 2022 sample, approximately 10.4\% of employed workers aged 25-64 are self-employed. Of these, about 38\% are incorporated and 62\% are unincorporated. The self-employed differ from wage workers in several observable ways: they are older on average (47.2 vs. 43.9 years), more likely to be male (60.6\% vs. 51.3\%), more likely to be married (68.0\% vs. 59.9\%), and somewhat less likely to have a college degree (38.0\% vs. 43.7\%). These differences motivate our regression-adjusted estimates.

\section{Related Literature}

This paper connects to three strands of literature. First, a substantial body of work has examined the relationship between health insurance and self-employment, primarily using pre-ACA data. Holtz-Eakin, Penrod, and Rosen (1996) documented that health insurance considerations significantly affected the self-employment decision, with workers covered by a spouse's insurance being substantially more likely to start a business. Wellington (2001) found that self-employed men were 25\% less likely to have health insurance than wage workers. Fairlie, Kapur, and Gates (2011) estimated that the self-employed were 8.4 percentage points more likely to be uninsured than wage workers in the mid-2000s. These studies established the existence of a meaningful self-employment health insurance penalty prior to the ACA.

Several studies have examined how specific policy changes affected this relationship. Gruber and Madrian (1997) found that continuation-of-coverage laws (COBRA) increased self-employment entry. More recently, studies of Massachusetts health reform (which presaged the ACA) found that coverage mandates and subsidies reduced the self-employment insurance gap. However, comprehensive analysis of the national ACA's effects on self-employed coverage has been limited, partly due to data lags and the complexity of the multi-state Medicaid expansion.

Second, this paper contributes to the large literature evaluating ACA implementation. Studies have examined the law's effects on coverage, access, and health for the general population. The Medicaid expansion literature has found significant coverage gains in expansion states. Marketplace studies have examined enrollment patterns, plan choice, and the effects of subsidies. However, few studies have focused specifically on how the self-employed---a population with unique insurance needs---have fared under the new system.

Third, this paper speaks to contemporary policy debates about non-traditional work arrangements. The growth of the gig economy has renewed interest in ``portable benefits'' that follow workers across jobs and employment arrangements. Understanding whether the ACA provides a viable coverage pathway for the self-employed is directly relevant to these debates.

\section{Data and Methods}

\subsection{Data}

I use the American Community Survey (ACS) Public Use Microdata Sample for 2022, accessed via the Census Bureau API. The ACS is the largest U.S. household survey, sampling approximately 3.5 million addresses annually. It provides detailed information on demographic characteristics, employment, income, and health insurance coverage. Using a single year avoids complications from time-varying policy environments, particularly Medicaid expansion timing, which changed for several states between 2018 and 2022.

I restrict the sample to employed civilians aged 25-64 with valid class-of-worker codes who report working at least 10 hours per week, yielding 1,296,497 observations. The age restriction ensures a working-age population not yet eligible for Medicare (which begins at 65). The employment and hours restrictions focus on substantially employed workers for whom the self-employment versus wage work distinction is economically meaningful; workers with fewer than 10 hours per week are likely to have other primary activities.

\subsection{Variable Definitions}

\textbf{Treatment: Self-Employment.} I define self-employment using the ACS class-of-worker variable (COW). Self-employed workers are those reporting self-employment in an own incorporated business (COW=7) or self-employment in an own not-incorporated business (COW=6). Wage workers include private for-profit employees, nonprofit employees, and government employees (COW=1-5).

\textbf{Outcomes: Health Insurance Coverage.} The ACS asks respondents about coverage from multiple sources. I construct the following outcomes: (1) \textit{Any insurance}: covered by any type of health insurance; (2) \textit{Employer-sponsored}: covered through an employer or union; (3) \textit{Direct purchase}: coverage purchased directly from an insurance company (this category includes both ACA Marketplace plans and off-exchange individual market plans; the ACS does not distinguish between these sources); (4) \textit{Medicaid}: covered by Medicaid, Medical Assistance, or other government assistance plans for low-income individuals.

\textbf{Covariates.} I control for age (linear and quadratic), sex, race/ethnicity (White, Black, Hispanic, Asian, Other), educational attainment (less than high school, high school, some college, bachelor's, graduate degree), marital status, usual hours worked per week (linear and quadratic), household size, household income quintile, and state fixed effects. Since the data are from a single year (2022), Medicaid expansion status is time-invariant at the state level and is absorbed by state fixed effects in the main specification. I examine heterogeneity by expansion status in separate subgroup regressions that split the sample by state expansion status as of 2022.

\subsection{Empirical Strategy}

I estimate the association between self-employment and health insurance outcomes using ordinary least squares regression with robust standard errors:

\begin{equation}
Y_i = \alpha + \beta \cdot SelfEmployed_i + X_i'\gamma + \delta_s + \epsilon_i
\end{equation}

where $Y_i$ is a binary insurance outcome, $SelfEmployed_i$ indicates self-employment status, $X_i$ is a vector of individual and household covariates, and $\delta_s$ are state fixed effects. The coefficient $\beta$ captures the conditional difference in coverage between self-employed and wage workers with similar observable characteristics.

The key identifying assumption is selection on observables: conditional on the included covariates, self-employment status is independent of potential insurance outcomes. This assumption would be violated if unobserved factors (e.g., risk tolerance, health status, entrepreneurial ability) jointly affect both the decision to be self-employed and insurance coverage. I address this concern in several ways. First, I conduct formal sensitivity analysis using the methods of Cinelli and Hazlett (2020) to assess how strong unmeasured confounding would need to be to explain away the results. Second, I examine heterogeneity across subgroups where confounding patterns might differ. Third, I discuss the nature and likely direction of potential biases.

\section{Results}

\subsection{Descriptive Statistics}

Table 1 presents summary statistics by self-employment status. Self-employed workers differ from wage workers in several ways. They are older on average (47.2 vs. 43.9 years), more likely to be male (60.6\% vs. 51.3\%), and more likely to be married (68.0\% vs. 59.9\%). They are slightly less likely to have a college degree (38.0\% vs. 43.7\%) and work similar hours (40.7 vs. 41.0 per week).

\begin{table}[H]
\centering
\caption{Summary Statistics by Self-Employment Status}
\label{tab:summary}
\begin{tabular}{lcc}
\toprule
Variable & Wage Workers & Self-Employed \\
\midrule
Observations & 1,161,571 & 134,926 \\
\addlinespace
\textit{Demographics} \\
Age (mean) & 43.9 & 47.2 \\
Female (\%) & 48.7 & 39.4 \\
College graduate (\%) & 43.7 & 38.0 \\
Married (\%) & 59.9 & 68.0 \\
Hours worked (mean) & 41.0 & 40.7 \\
\addlinespace
\textit{Health Insurance (\%)} \\
Any insurance & 92.0 & 84.4 \\
Employer-sponsored & 76.3 & 44.3 \\
Direct purchase & 9.2 & 28.5 \\
Medicaid & 9.1 & 14.2 \\
\bottomrule
\multicolumn{3}{l}{\footnotesize Note: Sample is employed civilians age 25-64 with hours $\geq$ 10/week, ACS 2022.}
\end{tabular}
\end{table}

Unadjusted insurance rates reveal large differences. Among wage workers, 92.0\% have any health insurance, with 76.3\% covered through an employer. Among the self-employed, only 84.4\% have coverage---a 7.6 percentage point gap---and just 44.3\% have employer-sponsored coverage (a 32 pp gap). However, the self-employed partially compensate through other channels: 28.5\% have direct-purchase coverage (vs. 9.2\% of wage workers) and 14.2\% have Medicaid (vs. 9.1\%).

\subsection{Main Results}

Table 2 presents the regression-adjusted estimates. After controlling for demographics, labor market characteristics, income, and state fixed effects, self-employed workers are 6.1 percentage points less likely to have any health insurance coverage (95\% CI: -6.3 to -5.9 pp, p $<$ 0.001). This represents a 6.6\% relative reduction from the 92.0\% baseline among wage workers.

\begin{table}[H]
\centering
\caption{Effect of Self-Employment on Health Insurance Coverage}
\label{tab:main}
\begin{tabular}{lcccc}
\toprule
Outcome & Coefficient & SE & 95\% CI & Baseline \\
\midrule
Any insurance & -0.061*** & 0.001 & [-0.063, -0.059] & 92.0\% \\
Employer-sponsored & -0.272*** & 0.001 & [-0.274, -0.269] & 76.3\% \\
Direct purchase & 0.183*** & 0.001 & [0.180, 0.185] & 9.2\% \\
Medicaid & 0.032*** & 0.001 & [0.030, 0.033] & 9.1\% \\
\bottomrule
\multicolumn{5}{l}{\footnotesize Note: OLS estimates with robust (HC2) standard errors. N = 1,296,497.} \\
\multicolumn{5}{l}{\footnotesize Controls: age, age$^2$, sex, race, education, marital status, hours, hours$^2$,} \\
\multicolumn{5}{l}{\footnotesize household size, income quintile, state FE. *** p $<$ 0.001.}
\end{tabular}
\end{table}

The mechanisms are clear from the source-specific outcomes. Self-employed workers are 27.2 percentage points less likely to have employer-sponsored insurance---a 36\% relative decline. This reflects the fundamental structural difference: self-employed workers lack access to employer group plans. This deficit is partially offset by two channels. Self-employed workers are 18.3 percentage points more likely to have direct-purchase coverage, representing a 199\% relative increase over the 9.2\% baseline. The ACS direct-purchase category includes both ACA Marketplace plans and off-exchange individual market coverage; we cannot distinguish between these sources in the data. Additionally, self-employed workers are 3.2 percentage points more likely to have Medicaid coverage, a 35\% relative increase. However, these alternative sources do not fully compensate for the loss of employer coverage, leaving the 6.1 pp net coverage gap.

\subsection{Heterogeneity by Policy Environment}

If the ACA's reforms---Medicaid expansion and individual market regulations---have improved insurance access for the self-employed, we should see smaller coverage gaps where these policies are more generous. Table 3 presents estimates by Medicaid expansion status and income level.

\begin{table}[H]
\centering
\caption{Heterogeneity in Self-Employment Coverage Gap}
\label{tab:het}
\begin{tabular}{lccc}
\toprule
Subgroup & Effect & SE & N \\
\midrule
\textit{By Medicaid Expansion (as of 2022)} \\
Expansion states & -0.064 & 0.004 & 1,116,693 \\
Non-expansion states & -0.101 & 0.011 & 179,804 \\
\addlinespace
\textit{By Income Quintile} \\
Q1 (lowest) & -0.016 & 0.007 & 233,454 \\
Q2 & -0.079 & 0.008 & 300,556 \\
Q3 & -0.095 & 0.009 & 295,494 \\
Q4 & -0.085 & 0.008 & 252,830 \\
Q5 (highest) & -0.046 & 0.006 & 214,163 \\
\bottomrule
\multicolumn{4}{l}{\footnotesize Note: Each row is a separate OLS regression on the indicated subsample. Robust} \\
\multicolumn{4}{l}{\footnotesize (HC2) SEs. Controls: age, age$^2$, sex, race, education, marital status, hours,} \\
\multicolumn{4}{l}{\footnotesize hours$^2$, household size, state FE. Income quintile included in expansion panel} \\
\multicolumn{4}{l}{\footnotesize (top); omitted in income panel (bottom) since constant within each subsample.}
\end{tabular}
\end{table}

The results strongly support the importance of policy environment. In Medicaid expansion states, the self-employment coverage penalty is 6.4 percentage points, compared to 10.1 percentage points in non-expansion states---a 37\% smaller gap. This suggests that Medicaid expansion provides a valuable coverage alternative for lower-income self-employed workers who might otherwise remain uninsured.

The income gradient reveals an important pattern. For the lowest-income workers (Q1), self-employment is associated with only a 1.6 pp coverage penalty---the smallest across all quintiles. The penalty is largest in the middle-income quintiles: 7.9 pp in Q2 and 9.5 pp in Q3, then declines to 8.5 pp in Q4 and 4.6 pp in Q5. This inverted-U pattern has a clear interpretation: the lowest-income self-employed workers qualify for Medicaid (especially in expansion states), providing substantial protection. Middle-income workers in Q2-Q3 earn too much for Medicaid but lack employer coverage and may find individual market premiums burdensome even with subsidies. Higher-income workers can more easily afford direct-purchase coverage without subsidies. The small penalty for Q1 workers demonstrates the importance of Medicaid as a safety net for low-income self-employed workers.

\subsection{Sensitivity Analysis}

The key threat to interpreting these estimates as causal effects is unmeasured confounding. Workers who select into self-employment may differ from wage workers in ways that also affect insurance coverage. For example, more risk-tolerant individuals might both prefer self-employment and be more willing to go without insurance. Alternatively, individuals with worse health might avoid self-employment precisely because of insurance concerns, biasing the estimated penalty toward zero.

I assess sensitivity to unmeasured confounding using the methodology of Cinelli and Hazlett (2020). The robustness value (RV) indicates how much residual variation an unmeasured confounder would need to explain in both the treatment and outcome to reduce the estimated effect to zero. For the any-insurance outcome, the robustness value is 7.9\%. This means an unmeasured confounder would need to explain at least 7.9\% of the residual variance in both self-employment status and insurance coverage to fully account for the observed 6.1 pp estimate.

To calibrate this magnitude, I benchmark against observed confounders. Female gender explains 0.30\% of residual variance in self-employment and 0.20\% in insurance coverage. Marital status explains 0.08\% of residual variance in self-employment and 0.82\% in insurance. An unmeasured confounder three times as strong as gender would reduce the estimate to approximately -4.8 pp (still highly significant). An unmeasured confounder three times as strong as marital status would reduce it to approximately -4.3 pp. These calibrations suggest that while unmeasured confounding could attenuate the estimates, implausibly strong confounders would be needed to eliminate the effect entirely.

\section{Discussion and Conclusion}

This paper provides a comprehensive analysis of the self-employment health insurance gap in the post-ACA era. Using data on 1.3 million workers from the 2022 American Community Survey, I find that self-employed workers are 6.1 percentage points less likely to have health insurance coverage than comparable wage workers. This gap operates through dramatically lower employer-sponsored coverage (-27.2 pp), partially offset by higher direct-purchase and Medicaid enrollment. Importantly, the gap is substantially smaller in Medicaid expansion states (6.4 vs. 10.1 pp). The income pattern reveals that Medicaid effectively protects low-income self-employed workers (1.6 pp penalty) while middle-income workers face the largest penalty (9.5 pp), suggesting that the ACA's coverage alternatives have had differential success across the income distribution.

Several policy implications emerge. First, the persistence of a meaningful coverage gap suggests that the ACA, while beneficial, has not fully solved the insurance challenges facing the self-employed. Policy reforms that further expand Marketplace subsidies, reduce coverage costs, or create new portable benefits options could further narrow this gap. Second, the strong gradient by income and Medicaid expansion status highlights the importance of these policies for lower-income self-employed workers. States that have not expanded Medicaid leave their self-employed residents particularly exposed. Third, as non-traditional work arrangements continue to grow, these insurance access questions will become increasingly salient for a larger share of the workforce.

This analysis has limitations. The observational design cannot rule out all unmeasured confounding, though sensitivity analysis suggests results are robust to moderate confounding. The ACS measures insurance coverage at a point in time and cannot capture coverage stability or churn. The direct-purchase category combines ACA Marketplace and off-exchange individual market coverage, limiting our ability to isolate Marketplace effects specifically. Future research using administrative data on Marketplace enrollment could address these limitations.

Despite these caveats, the findings are clear: self-employment continues to carry a health insurance penalty in the post-ACA era. While the Marketplaces and Medicaid expansion have provided valuable coverage alternatives, the structural disconnect between self-employment and employer-sponsored insurance has not been fully bridged. As the labor market continues to evolve toward more flexible and independent work arrangements, ensuring adequate insurance access for non-traditional workers remains an important policy challenge.

\newpage

\section*{References}

\begin{itemize}
\item Cinelli, C. and Hazlett, C. (2020). Making sense of sensitivity: Extending omitted variable bias. \textit{Journal of the Royal Statistical Society: Series B}, 82(1), 39-67.

\item Fairlie, R.W., Kapur, K., and Gates, S. (2011). Is employer-based health insurance a barrier to entrepreneurship? \textit{Journal of Health Economics}, 30(1), 146-162.

\item Gruber, J. and Madrian, B.C. (1997). Employment separation and health insurance coverage. \textit{Journal of Public Economics}, 66(3), 349-382.

\item Holtz-Eakin, D., Penrod, J.R., and Rosen, H.S. (1996). Health insurance and the supply of entrepreneurs. \textit{Journal of Public Economics}, 62(1-2), 209-235.

\item Wellington, A.J. (2001). Health insurance coverage and entrepreneurship. \textit{Contemporary Economic Policy}, 19(4), 465-478.
\end{itemize}

\newpage

\section*{Appendix: Data and Replication}

\subsection*{Data Source}

All data are from the U.S. Census Bureau American Community Survey (ACS) Public Use Microdata Sample (PUMS), accessed via the Census API. The analysis uses the 2022 1-year ACS file.

API endpoint: \texttt{https://api.census.gov/data/2022/acs/acs1/pums}

\subsection*{Sample Construction}

\begin{enumerate}
\item Download person-level records for all states from 2022 ACS
\item Restrict to ages 25-64
\item Restrict to employed civilians with hours worked $\geq$ 10
\item Restrict to valid class-of-worker codes (COW = 1-7)
\item Drop observations with missing values on key variables
\end{enumerate}

Final sample: 1,296,497 observations.

\subsection*{Code Availability}

All analysis code is available at the APEP project repository:

\url{https://github.com/SocialCatalystLab/auto-policy-evals}

See the \texttt{papers/} directory for the published paper folder containing all R scripts, data cleaning code, and figure generation code. Data can be reproduced by querying the Census Bureau API as documented in the scripts.

\end{document}
