\begin{table}[htbp]
\centering
\caption{HonestDiD Sensitivity: Gender Gap Effect}
\label{tab:honestdid_gender}
\begin{tabular}{cccc}
\toprule
$M$ & Estimate & 95\% CI & Zero Excluded? \\
\midrule
0.0 & 0.0714 & [0.0431, 0.0996] & Yes \\
0.5 & 0.1492 & [$-1.58$, 1.88] & No \\
1.0 & 0.1492 & [$-3.25$, 3.55] & No \\
\bottomrule
\end{tabular}
\begin{minipage}{0.90\textwidth}
\footnotesize
\textit{Notes:} HonestDiD sensitivity analysis \citep{rambachan2023more} applied to the gender gap event study (female C-S ATT minus male C-S ATT at each event time). $M$ indicates the maximum magnitude of post-treatment parallel trends violations relative to the largest pre-treatment coefficient. The ``Estimate'' column reports the midpoint of the FLCI bounds, not the DDD coefficient from Table~\ref{tab:gender}; the difference arises because this analysis uses the Callaway-Sant'Anna gender gap event study rather than the TWFE DDD interaction. At $M = 0$ (exact parallel trends), the 95\% CI firmly excludes zero. For $M \geq 0.5$, bounds become uninformative because the gender-disaggregated event-study estimates have high sampling variance: with only 6 treated states split by gender, the largest pre-treatment gap coefficient is $|0.040|$, and the relative magnitudes approach compounds this noise rapidly. Rows for $M > 1$ are omitted as they are strictly wider. The permutation inference in Table~\ref{tab:alt_inference} provides complementary design-based evidence.
\end{minipage}
\end{table}
