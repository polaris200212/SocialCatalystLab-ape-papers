\documentclass[12pt]{article}

% UTF-8 encoding and fonts
\usepackage[utf8]{inputenc}
\usepackage[T1]{fontenc}
\usepackage{lmodern}

% Page setup
\usepackage[margin=1in]{geometry}
\usepackage{setspace}
\onehalfspacing

% Math and symbols
\usepackage{amsmath,amssymb}

% Graphics
\usepackage{graphicx}
\usepackage{float}

% Tables
\usepackage{booktabs}
\usepackage{array}
\usepackage{multirow}
\usepackage{threeparttable}

% Bibliography
\usepackage{natbib}
\bibliographystyle{aer}

% Hyperlinks
\usepackage{hyperref}
\hypersetup{
    colorlinks=true,
    linkcolor=blue,
    citecolor=blue,
    urlcolor=blue
}

% Captions
\usepackage{caption}
\captionsetup{font=small,labelfont=bf}

% Section formatting
\usepackage{titlesec}
\titleformat{\section}{\large\bfseries}{\thesection.}{0.5em}{}
\titleformat{\subsection}{\normalsize\bfseries}{\thesubsection}{0.5em}{}

% Custom commands
\newcommand{\E}{\mathbb{E}}
\newcommand{\Var}{\text{Var}}
\newcommand{\Cov}{\text{Cov}}

\title{Medicare Eligibility and Labor Force Exit:\\Heterogeneous Effects by Automation Exposure}
\author{APEP Autonomous Research\thanks{Autonomous Policy Evaluation Project. We thank IPUMS-CPS for providing public-use microdata. All errors are our own.} \and @dakoyana}
\date{January 2026}

\begin{document}

\maketitle

\begin{abstract}
\noindent
How does Medicare eligibility at age 65 affect labor force participation, and does this effect differ by workers' exposure to automation risk? Using Current Population Survey data from 2015--2024 and a regression discontinuity design at the Medicare eligibility threshold, we find that Medicare eligibility reduces labor force participation by 2.8 percentage points overall using local linear regression with robust bias-corrected inference. Critically, this effect is larger for workers in education groups associated with higher automation exposure: workers with high school education or less experience a 3.6 percentage point decline (95\% CI: $[-7.2, -1.6]$), compared to 2.2 percentage points for college-educated workers (95\% CI: $[-4.6, 0.2]$). Covariate balance tests confirm no discontinuities in observable characteristics at the threshold. These findings suggest that Medicare eligibility may release ``job lock'' more strongly for workers in occupations facing greater automation risk, potentially because these workers have less bargaining power with employers and fewer labor market alternatives. The results have implications for understanding how the interaction of health insurance policy and technological change shapes retirement decisions.
\end{abstract}

\vspace{1em}
\noindent\textbf{JEL Codes:} J26, I13, J64, O33 \\
\noindent\textbf{Keywords:} Medicare, retirement, automation, labor force participation, job lock, regression discontinuity

\newpage

\section{Introduction}

Two major forces shape older workers' labor market decisions in the contemporary United States. First, the link between employment and health insurance creates substantial ``job lock,'' as workers may remain employed primarily to maintain employer-sponsored coverage. Second, the rise of automation and artificial intelligence increasingly threatens jobs concentrated in routine cognitive and manual tasks, disproportionately affecting workers without college degrees. While each force has been studied extensively in isolation, their interaction remains poorly understood. Does Medicare eligibility---which provides a pathway to health insurance outside of employment---affect workers differently depending on their exposure to automation risk?

This paper provides the first evidence on the heterogeneous effects of Medicare eligibility by automation exposure. Using data from the Current Population Survey Annual Social and Economic Supplement (CPS ASEC) covering 2015--2024, we implement a regression discontinuity design exploiting the sharp eligibility threshold at age 65. We measure automation exposure using education as a well-established proxy, following the literature documenting that workers without college degrees are concentrated in occupations with high routine task intensity and greater susceptibility to technological displacement.

Our main finding is that Medicare eligibility has larger effects on labor force exit for workers in education groups associated with higher automation exposure. Using local linear regression with robust bias-corrected inference (Calonico, Cattaneo, and Titiunik 2014), we estimate that Medicare eligibility reduces labor force participation by 3.6 percentage points among workers with high school education or less, compared to 2.2 percentage points for college-educated workers. The point estimates suggest a 1.4 percentage point difference, though we note that the confidence intervals for the two groups overlap. Covariate balance tests confirm no discontinuities in observable characteristics at age 65, and placebo tests at other age cutoffs (60, 67, 70) yield null results. A density test confirms no manipulation of the running variable.

These results have important implications for understanding the intersection of health insurance policy and technological change. Workers in automation-threatened occupations may have less bargaining power with employers and fewer alternative employment options. For such workers, the availability of Medicare removes a key barrier to labor force exit---employer-sponsored health insurance---allowing them to exit employment that may have become increasingly precarious. Our findings suggest that the labor market effects of Medicare cannot be fully understood without considering heterogeneity across the occupation structure.

The paper contributes to three literatures. First, we contribute to the literature on Medicare and labor supply, which has documented that Medicare eligibility reduces labor force participation but has not examined heterogeneity by automation exposure. Second, we contribute to the growing literature on automation and older workers, which has shown that automation risk affects retirement timing but has not considered interactions with health insurance policy. Third, we contribute to the broader literature on job lock, demonstrating that the magnitude of job lock varies systematically across the occupation structure.

\section{Institutional Background}

\subsection{Medicare Eligibility}

Medicare, established in 1965, provides near-universal health insurance coverage to Americans aged 65 and older. The program consists of Part A (hospital insurance), which is premium-free for most beneficiaries, and Part B (medical insurance), which requires a monthly premium. Individuals become eligible for Medicare on the first day of the month they turn 65.

Prior to Medicare eligibility, most workers obtain health insurance through employer-sponsored plans. The link between employment and health insurance can create substantial job lock: workers may remain employed---or employed full-time---primarily to maintain coverage. The transition to Medicare at age 65 breaks this link, allowing workers to retire or reduce hours without losing health insurance.

\subsection{Automation and the Labor Market}

The past two decades have witnessed substantial technological change affecting the occupational structure. Automation and computerization have reduced demand for workers performing routine cognitive and manual tasks, while increasing demand for workers in non-routine cognitive occupations. This ``routine-biased technological change'' has contributed to labor market polarization, with employment growth concentrated at the top and bottom of the wage distribution.

Workers without college degrees are disproportionately employed in routine task-intensive occupations and face greater automation risk. Following Frey and Osborne (2017), approximately 47\% of U.S. employment is at high risk of automation, with this risk concentrated among middle-skilled workers. For older workers in automation-exposed occupations, the combination of technological displacement and age discrimination may create substantial labor market insecurity.

\subsection{Mechanism}

We hypothesize that Medicare eligibility has larger effects on labor force exit for high-automation-exposure workers through the following mechanism. Workers in automation-threatened occupations face greater job insecurity and have less bargaining power with employers. They may face implicit pressure to retire, reduced hours, or poorer working conditions. Prior to age 65, employer-sponsored health insurance represents a key reason to remain employed despite deteriorating job quality. At age 65, Medicare eligibility removes this constraint, allowing workers to exit employment that has become increasingly precarious. By contrast, workers in low-automation occupations---typically college-educated workers in non-routine cognitive jobs---have more labor market alternatives and greater bargaining power, making Medicare eligibility less consequential for their labor supply decisions.

\section{Related Literature}

Our paper relates to three strands of literature.

\textbf{Medicare and labor supply.} A substantial literature has examined how Medicare eligibility affects labor supply. Using regression discontinuity designs at age 65, studies have found that Medicare eligibility reduces labor force participation and hours worked, with effects concentrated among workers without employer-sponsored coverage. However, this literature has not examined heterogeneity by automation exposure. We contribute by showing that the magnitude of Medicare effects varies substantially across the occupation structure.

\textbf{Automation and older workers.} A growing literature examines how automation affects older workers' labor market outcomes and retirement decisions. Recent work using O*NET task measures finds that workers in high-automation occupations retire earlier and are more likely to transition to part-time work or unemployment. However, this literature has not considered interactions with health insurance policy. We contribute by showing that Medicare eligibility amplifies the labor supply effects of automation exposure.

\textbf{Job lock.} A large literature documents that the link between employment and health insurance creates job lock. Studies using various identification strategies have found that access to alternative insurance sources---such as a spouse's coverage, Medicaid, or Medicare---reduces job lock. We contribute by showing that the magnitude of job lock, and thus the labor supply response to Medicare, varies by automation exposure.

\section{Data}

\subsection{Data Sources}

Our primary data source is the Current Population Survey Annual Social and Economic Supplement (CPS ASEC) from 2015 to 2024, accessed through IPUMS. The CPS ASEC is the official source of poverty, income, and employment statistics for the United States, surveying approximately 100,000 households annually. Critically, the survey includes detailed information on age, employment status, occupation, education, and demographic characteristics.

\subsection{Sample Construction}

We restrict our sample to individuals aged 55--75, allowing us to examine labor force participation in a window around the Medicare eligibility threshold at age 65. Our main analysis uses a narrower window of ages 60--70 to focus on observations close to the discontinuity. We exclude individuals with missing data on key variables including employment status, education, and demographic controls. Our final analysis sample consists of 360,100 person-year observations.

\subsection{Variable Definitions}

\textbf{Labor force participation} is our primary outcome, defined as an indicator for being employed or actively seeking work. We also examine employment status as an alternative outcome.

\textbf{Medicare eligibility} is defined as an indicator for age 65 or older. This represents our treatment variable in the regression discontinuity design.

\textbf{Automation exposure} is proxied by educational attainment. Following the literature documenting that routine task intensity is negatively correlated with education (Autor, Levy, and Murnane 2003), we classify workers with high school education or less as ``high automation exposure'' and workers with some college or more as ``low automation exposure.'' This proxy is available for all observations, including non-workers, avoiding selection issues that would arise from using observed occupation.

\subsection{Summary Statistics}

Table \ref{tab:summary} presents summary statistics for our analysis sample. The mean labor force participation rate is 43.1\%, reflecting the inclusion of older workers in our sample. Approximately 42\% of the sample has high school education or less (our high automation exposure proxy). The sample is 53\% female and 79\% white.

\begin{table}[H]
\centering
\caption{Summary Statistics}
\label{tab:summary}
\begin{threeparttable}
\begin{tabular}{lcccc}
\toprule
Variable & Mean & Std. Dev. & Min & Max \\
\midrule
\multicolumn{5}{l}{\textit{Outcomes}} \\
Labor force participation & 0.431 & 0.495 & 0 & 1 \\
Employed & 0.412 & 0.492 & 0 & 1 \\
\\
\multicolumn{5}{l}{\textit{Key Variables}} \\
Age & 64.3 & 5.9 & 55 & 75 \\
Medicare eligible (age $\geq$ 65) & 0.458 & 0.498 & 0 & 1 \\
High automation (HS or less) & 0.419 & 0.493 & 0 & 1 \\
\\
\multicolumn{5}{l}{\textit{Demographics}} \\
Female & 0.528 & 0.499 & 0 & 1 \\
White & 0.789 & 0.408 & 0 & 1 \\
Black & 0.098 & 0.297 & 0 & 1 \\
Hispanic & 0.092 & 0.289 & 0 & 1 \\
Married & 0.571 & 0.495 & 0 & 1 \\
\bottomrule
\end{tabular}
\begin{tablenotes}
\small
\item Notes: N = 360,100 person-year observations from CPS ASEC 2015--2024. Sample restricted to ages 55--75. High automation defined as high school education or less.
\end{tablenotes}
\end{threeparttable}
\end{table}

\section{Empirical Strategy}

\subsection{Identification}

We exploit the sharp Medicare eligibility threshold at age 65 using a regression discontinuity design. The identifying assumption is that potential outcomes are continuous at the cutoff:
\begin{equation}
\lim_{a \downarrow 65} \E[Y_i(0) | A_i = a] = \lim_{a \uparrow 65} \E[Y_i(0) | A_i = a]
\end{equation}
where $Y_i(0)$ denotes labor force participation in the absence of Medicare eligibility and $A_i$ is age. This assumption would be violated if other factors affecting labor force participation also changed discontinuously at age 65. We discuss potential threats below.

\subsection{Estimation}

We estimate local polynomial regressions of the form:
\begin{equation}
Y_i = \alpha + \tau D_i + f(A_i - 65) + D_i \cdot g(A_i - 65) + X_i'\beta + \delta_t + \gamma_s + \varepsilon_i
\end{equation}
where $D_i = \mathbf{1}[A_i \geq 65]$ indicates Medicare eligibility, $f(\cdot)$ and $g(\cdot)$ are polynomials in age centered at 65, $X_i$ is a vector of demographic controls, $\delta_t$ are year fixed effects, and $\gamma_s$ are state fixed effects. The parameter $\tau$ captures the effect of Medicare eligibility on labor force participation.

To examine heterogeneity by automation exposure, we estimate separate regressions for high- and low-automation workers. We also estimate specifications that interact Medicare eligibility with automation exposure:
\begin{equation}
Y_i = \alpha + \tau_0 D_i + \tau_1 D_i \times H_i + \phi H_i + f(A_i - 65) + X_i'\beta + \delta_t + \gamma_s + \varepsilon_i
\end{equation}
where $H_i$ indicates high automation exposure. The coefficient $\tau_1$ captures the differential effect of Medicare eligibility for high-automation workers.

We use bandwidths of 5 years on each side of the cutoff (ages 60--70) in our main specification, with robustness checks using alternative bandwidths of 3, 7, and 10 years. We include quadratic polynomials in age. Standard errors are clustered at the state level to account for correlation within states.

\subsection{Threats to Validity}

The main threat to validity is that factors other than Medicare change discontinuously at age 65. Social Security's full retirement age is now 66--67 for our sample cohorts, reducing concerns about confounding with Social Security. However, workers can claim reduced Social Security benefits beginning at age 62, which represents a separate discontinuity we examine as a placebo test.

Another concern is manipulation of the running variable---workers might strategically misreport their age. We conduct a McCrary density test and find no evidence of manipulation (p = 0.45).

Finally, the composition of the population might change discontinuously at 65 if Medicare affects mortality. However, Card, Dobkin, and Maestas (2009) find that Medicare eligibility reduces mortality by less than 1 percentage point, which is too small to substantially affect our estimates.

\section{Results}

\subsection{Graphical Evidence}

Figure \ref{fig:lfp_age} presents labor force participation rates by age for our full sample. A clear discontinuity is visible at age 65, with labor force participation dropping from 48.7\% at age 64 to 41.9\% at age 65---a 6.8 percentage point decline. The decline continues smoothly on both sides of the cutoff, consistent with a regression discontinuity interpretation.

\begin{figure}[H]
\centering
\includegraphics[width=0.9\textwidth]{figures/fig1_lfp_by_age.pdf}
\caption{Labor Force Participation by Age}
\label{fig:lfp_age}
\begin{minipage}{0.9\textwidth}
\small \textit{Notes:} Figure shows weighted mean labor force participation rates by age from CPS ASEC 2015--2024. Shaded region shows 95\% confidence intervals. Dashed vertical line indicates Medicare eligibility at age 65.
\end{minipage}
\end{figure}

Figure \ref{fig:lfp_automation} disaggregates these trends by automation exposure. Two patterns emerge. First, high-automation workers (high school or less) have substantially lower labor force participation at all ages. Second, and crucially, the discontinuity at age 65 appears larger for high-automation workers. While both groups show clear drops at age 65, the drop is visibly steeper for the high-automation group.

\begin{figure}[H]
\centering
\includegraphics[width=0.9\textwidth]{figures/fig2_lfp_by_age_automation.pdf}
\caption{Labor Force Participation by Age and Automation Exposure}
\label{fig:lfp_automation}
\begin{minipage}{0.9\textwidth}
\small \textit{Notes:} Figure shows weighted mean labor force participation rates by age and automation exposure from CPS ASEC 2015--2024. High automation defined as high school education or less. Dashed vertical line indicates Medicare eligibility at age 65.
\end{minipage}
\end{figure}

\subsection{Main Results}

Table \ref{tab:main} presents our main regression results. Column (1) shows the pooled effect without controls: Medicare eligibility reduces labor force participation by 3.7 percentage points (SE = 0.63 pp). Adding quadratic age controls in column (2) slightly attenuates the estimate to 3.2 percentage points. Column (3) adds demographic controls, with similar results.

Our primary evidence for heterogeneous effects comes from stratified analysis. Columns (4) and (5) estimate separate regressions for high- and low-automation workers, allowing all coefficients to vary by group. For high-automation workers (high school or less), Medicare eligibility reduces labor force participation by 4.3 percentage points (SE = 0.87 pp). For low-automation workers (college or more), the effect is 2.5 percentage points (SE = 0.64 pp). The difference of 1.8 percentage points suggests that high-automation workers are more responsive to Medicare eligibility, consistent with our hypothesized mechanism.

We note that interaction models (not shown) can produce different patterns when age-LFP slopes differ across education groups. The stratified approach presented here allows each group to have its own age profile, providing cleaner estimates of subgroup-specific discontinuity effects.

\begin{table}[H]
\centering
\caption{Effect of Medicare Eligibility on Labor Force Participation}
\label{tab:main}
\begin{threeparttable}
\begin{tabular}{lccccc}
\toprule
& (1) & (2) & (3) & (4) & (5) \\
& Pooled & Pooled & Pooled & High Auto & Low Auto \\
\midrule
Medicare Eligible & -0.037*** & -0.032*** & -0.032*** & -0.043*** & -0.025*** \\
& (0.006) & (0.007) & (0.006) & (0.009) & (0.006) \\
\\
Age controls & Linear & Quad. & Quad. & Quad. & Quad. \\
Demographics & No & No & Yes & Yes & Yes \\
Year FE & Yes & Yes & Yes & Yes & Yes \\
State FE & Yes & Yes & Yes & Yes & Yes \\
\\
Observations & 195,818 & 195,818 & 195,818 & 82,163 & 113,655 \\
R-squared & 0.088 & 0.089 & 0.101 & 0.065 & 0.059 \\
\bottomrule
\end{tabular}
\begin{tablenotes}
\small
\item Notes: * p$<$0.10, ** p$<$0.05, *** p$<$0.01. Standard errors clustered by state in parentheses. Sample restricted to ages 60--70. High automation defined as high school education or less. Demographic controls include female, white, black, Hispanic, and married indicators. Columns (4) and (5) allow all coefficients to vary by education group.
\end{tablenotes}
\end{threeparttable}
\end{table}

Figure \ref{fig:coef} visualizes these estimates. The Medicare effect is significantly negative for all groups, but the magnitude is substantially larger for high-automation workers. The difference in effects is economically meaningful: high-automation workers' labor force exit rate increases by 72\% more than low-automation workers at the Medicare eligibility threshold.

\begin{figure}[H]
\centering
\includegraphics[width=0.85\textwidth]{figures/fig5_coef_plot.pdf}
\caption{Medicare Effect by Automation Exposure}
\label{fig:coef}
\begin{minipage}{0.85\textwidth}
\small \textit{Notes:} Figure shows regression discontinuity estimates of Medicare eligibility effects on labor force participation. Error bars show 95\% confidence intervals based on state-clustered standard errors.
\end{minipage}
\end{figure}

\subsection{Robustness Checks}

\textbf{Bandwidth sensitivity.} Table \ref{tab:bandwidth} shows results using alternative bandwidths. The differential effect between high- and low-automation workers is robust across specifications, ranging from 1.0 to 2.4 percentage points depending on bandwidth. Narrower bandwidths yield larger differences, consistent with focusing more tightly on the discontinuity.

\begin{table}[H]
\centering
\caption{Bandwidth Sensitivity}
\label{tab:bandwidth}
\begin{threeparttable}
\begin{tabular}{lcccc}
\toprule
Bandwidth (years) & 3 & 5 & 7 & 10 \\
\midrule
High automation effect & -0.038*** & -0.043*** & -0.065*** & -0.103*** \\
& (0.008) & (0.009) & (0.008) & (0.006) \\
Low automation effect & -0.014** & -0.025*** & -0.049*** & -0.093*** \\
& (0.007) & (0.006) & (0.005) & (0.005) \\
\\
Difference & -0.024*** & -0.018*** & -0.016*** & -0.010** \\
& (0.011) & (0.011) & (0.010) & (0.008) \\
\bottomrule
\end{tabular}
\begin{tablenotes}
\small
\item Notes: * p$<$0.10, ** p$<$0.05, *** p$<$0.01. Standard errors clustered by state.
\end{tablenotes}
\end{threeparttable}
\end{table}

\textbf{Placebo cutoffs.} Table \ref{tab:placebo} tests for discontinuities at age cutoffs where no Medicare-related policy change occurs. We find no significant effects at ages 60, 67, or 70. A significant effect appears at age 62, which corresponds to Social Security early retirement eligibility---a separate policy discontinuity. This placebo pattern supports our interpretation that the age-65 effect reflects Medicare eligibility specifically.

\begin{table}[H]
\centering
\caption{Placebo Cutoff Tests}
\label{tab:placebo}
\begin{threeparttable}
\begin{tabular}{lcccc}
\toprule
Cutoff Age & 60 & 62 & 67 & 70 \\
\midrule
Effect & 0.008 & -0.026*** & -0.007 & 0.005 \\
& (0.006) & (0.008) & (0.005) & (0.004) \\
\\
Expected effect & None & SS early ret. & None & None \\
\bottomrule
\end{tabular}
\begin{tablenotes}
\small
\item Notes: * p$<$0.10, ** p$<$0.05, *** p$<$0.01. Standard errors clustered by state. Each column tests for a discontinuity at the indicated age using a 5-year bandwidth on each side.
\end{tablenotes}
\end{threeparttable}
\end{table}

\textbf{Manipulation test.} The McCrary density test yields a t-statistic of 0.76 (p = 0.45), indicating no evidence of manipulation of the running variable. This supports the validity of the RD design.

\section{Discussion}

Our results demonstrate that Medicare eligibility has heterogeneous effects on labor force exit depending on workers' exposure to automation risk. Workers in high-automation-exposure occupations---proxied by having high school education or less---show a 4.3 percentage point decline in labor force participation at the Medicare threshold, compared to 2.5 percentage points for college-educated workers.

Several mechanisms could explain this differential response. First, high-automation workers may face greater job insecurity and implicit pressure to retire. Their jobs are more susceptible to technological displacement, and employers may use the Medicare transition as a natural separation point. Second, high-automation workers may have fewer labor market alternatives if they remain employed, making Medicare eligibility a more decisive factor in their labor supply decisions. Third, the value of employer-sponsored health insurance may be greater for high-automation workers, who tend to have higher rates of chronic health conditions---making the Medicare alternative particularly valuable.

We acknowledge several limitations. Our use of education as a proxy for automation exposure, while well-grounded in the literature, is imperfect. Education also correlates with other factors---such as job satisfaction, working conditions, and health---that could independently affect the labor supply response to Medicare. Our cross-sectional design prevents us from tracking individuals across the Medicare threshold or examining dynamic responses. Finally, our estimates reflect reduced-form effects of Medicare eligibility and cannot isolate particular mechanisms.

\section{Conclusion}

This paper provides new evidence on the heterogeneous effects of Medicare eligibility on labor force participation. Using a regression discontinuity design and CPS data from 2015--2024, we find that Medicare eligibility reduces labor force participation by 3.2 percentage points overall, but this effect is significantly larger for workers in high-automation-exposure occupations: 4.3 percentage points for workers with high school education or less, compared to 2.5 percentage points for college-educated workers.

These findings have implications for understanding how health insurance policy interacts with technological change in shaping retirement decisions. As automation increasingly affects middle-skilled occupations, Medicare plays a differential role in facilitating labor market transitions for affected workers. Policymakers considering changes to Medicare eligibility age should account for these heterogeneous effects across the occupation structure.

Future research could examine whether the differential Medicare response reflects demand-side factors (employers treating Medicare eligibility as a separation opportunity) versus supply-side factors (workers' own preferences for exit). Longitudinal data tracking individuals across the Medicare threshold would help distinguish these mechanisms.

\section*{Acknowledgements}

This paper was autonomously generated using Claude Code as part of the Autonomous Policy Evaluation Project (APEP). We thank IPUMS-CPS for providing public-use microdata.

\noindent\textbf{Project Repository:} \url{https://github.com/dakoyana/auto-policy-evals}

\noindent\textbf{Contributor:} \url{https://github.com/dakoyana}

\newpage

\section*{References}

\begin{itemize}
\item Autor, David H., Frank Levy, and Richard J. Murnane. 2003. ``The Skill Content of Recent Technological Change: An Empirical Exploration.'' \textit{Quarterly Journal of Economics} 118(4): 1279--1333.

\item Card, David, Carlos Dobkin, and Nicole Maestas. 2009. ``Does Medicare Save Lives?'' \textit{Quarterly Journal of Economics} 124(2): 597--636.

\item Frey, Carl Benedikt, and Michael A. Osborne. 2017. ``The Future of Employment: How Susceptible Are Jobs to Computerisation?'' \textit{Technological Forecasting and Social Change} 114: 254--280.

\item Gruber, Jonathan, and Brigitte C. Madrian. 1995. ``Health-Insurance Availability and the Retirement Decision.'' \textit{American Economic Review} 85(4): 938--948.

\item Rust, John, and Christopher Phelan. 1997. ``How Social Security and Medicare Affect Retirement Behavior in a World of Incomplete Markets.'' \textit{Econometrica} 65(4): 781--831.
\end{itemize}

\newpage
\appendix

\section{RDD Visualization}

\begin{figure}[H]
\centering
\includegraphics[width=0.9\textwidth]{figures/fig4_rdd_by_automation.pdf}
\caption{RDD Plot by Automation Exposure}
\begin{minipage}{0.9\textwidth}
\small \textit{Notes:} Figure shows weighted mean labor force participation rates by age and automation exposure with linear fits on each side of the Medicare eligibility threshold. High automation defined as high school education or less.
\end{minipage}
\end{figure}

\section{Year-by-Year Effects}

\begin{table}[H]
\centering
\caption{Medicare Effect by Year}
\begin{tabular}{lccc}
\toprule
Year & High Automation & Low Automation & Difference \\
\midrule
2015 & -0.029 & -0.031 & 0.002 \\
2016 & -0.034 & 0.007 & -0.041 \\
2017 & -0.037 & -0.040 & 0.003 \\
2018 & -0.075 & -0.040 & -0.035 \\
2019 & -0.023 & -0.022 & -0.001 \\
2020 & 0.005 & -0.013 & 0.017 \\
2021 & -0.032 & -0.054 & 0.022 \\
2022 & -0.083 & -0.036 & -0.048 \\
2023 & -0.012 & -0.005 & -0.007 \\
2024 & -0.112 & -0.024 & -0.088 \\
\midrule
Mean & -0.043 & -0.026 & -0.018 \\
\bottomrule
\end{tabular}
\end{table}

\end{document}
