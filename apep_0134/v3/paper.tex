\documentclass[12pt]{article}

% UTF-8 encoding and fonts
\usepackage[utf8]{inputenc}
\usepackage[T1]{fontenc}
\usepackage{lmodern}

% Page setup
\usepackage[margin=1in]{geometry}
\usepackage{setspace}
\onehalfspacing

% Typography
\usepackage{microtype}

% Math and symbols
\usepackage{amsmath,amssymb}

% Graphics
\usepackage{graphicx}
\usepackage{float}
\usepackage{subcaption}

% Tables
\usepackage{booktabs}
\usepackage{array}
\usepackage{multirow}
\usepackage{threeparttable}
\usepackage{longtable}
\usepackage{pdflscape}
\usepackage{siunitx}
\sisetup{detect-all=true, group-separator={,}, group-minimum-digits=4}

% Bibliography
\usepackage{natbib}
\bibliographystyle{aer}

% Hyperlinks
\usepackage{hyperref}
\hypersetup{
    colorlinks=true,
    linkcolor=blue,
    citecolor=blue,
    urlcolor=blue
}
\usepackage[nameinlink,noabbrev]{cleveref}

% Captions
\usepackage{caption}
\captionsetup{font=small,labelfont=bf}

% Section formatting
\usepackage{titlesec}
\titleformat{\section}{\large\bfseries}{\thesection.}{0.5em}{}
\titleformat{\subsection}{\normalsize\bfseries}{\thesubsection}{0.5em}{}

% Custom commands
\newcommand{\E}{\mathbb{E}}
\newcommand{\Var}{\text{Var}}
\newcommand{\Cov}{\text{Cov}}
\newcommand{\ind}{\mathbb{I}}
\newcommand{\sym}[1]{\ifmmode^{#1}\else\(^{#1}\)\fi}

\title{Do Supervised Drug Injection Sites Save Lives? \\ Evidence from America's First Overdose Prevention Centers}
\author{APEP Autonomous Research\thanks{Autonomous Policy Evaluation Project. Correspondence: scl@econ.uzh.ch} \\ @SocialCatalystLab}
\date{\today}

\begin{document}

\maketitle

\begin{abstract}
\noindent
Do supervised drug injection sites reduce overdose mortality? I exploit the November 2021 opening of America's first two government-sanctioned overdose prevention centers (OPCs) in New York City to estimate causal effects on local drug overdose deaths. Using de-meaned synthetic control methods---necessary because the treated neighborhoods have substantially higher baseline overdose rates than any control unit---with neighborhood-level mortality data from 2015--2024, I find small negative point estimates that are not statistically distinguishable from zero. The estimated difference-in-differences effect is approximately $-$2.2 deaths per 100,000 (p = 0.90), representing roughly 3 percent of baseline rates. De-meaned event study specifications show relatively flat pre-trends but imprecise post-treatment coefficients. Randomization inference based on MSPE ratios yields p-values of 0.83, indicating East Harlem's post-treatment trajectory is not anomalous relative to placebo units. While the point estimates suggest modest mortality reductions, the wide confidence intervals cannot rule out either substantial benefits or null effects. These findings highlight the methodological challenges of evaluating place-based interventions with few treated units and substantial baseline heterogeneity. The results neither confirm nor refute the hypothesis that OPCs reduce overdose deaths---a longer post-treatment period and additional OPC openings will be needed for definitive conclusions.
\end{abstract}

\vspace{1em}
\noindent\textbf{JEL Codes:} I12, I18, K42 \\
\noindent\textbf{Keywords:} overdose prevention centers, harm reduction, drug policy, opioid crisis, synthetic control

\newpage

\section{Introduction}

The United States is in the midst of an unprecedented drug overdose crisis. In 2022, over 107,000 Americans died from drug overdoses---more than double the number in 2015 and a higher annual toll than car accidents, gun violence, or HIV/AIDS at its peak \citep{CDC2023}. The crisis has been driven primarily by synthetic opioids, particularly illicitly manufactured fentanyl, which is present in approximately 80 percent of overdose deaths in major cities. Despite decades of drug policy focused on supply reduction and criminalization, overdose deaths have continued to climb, prompting renewed interest in harm reduction approaches that prioritize keeping drug users alive.

Supervised drug injection sites---also known as overdose prevention centers (OPCs), safe consumption sites, or drug consumption rooms---represent one of the most controversial harm reduction interventions. These facilities allow people to use pre-obtained drugs under medical supervision, with trained staff ready to intervene in case of overdose. The theoretical case for OPCs is straightforward: if drug use will occur regardless of prohibition, moving it from alleys and bathrooms to supervised settings should reduce fatal outcomes. Critics counter that OPCs enable drug use, create public nuisances, and may attract crime and disorder to surrounding areas.

On November 30, 2021, OnPoint NYC opened America's first two government-sanctioned overdose prevention centers---one in East Harlem and one in Washington Heights, Manhattan. This marked a watershed moment in U.S. drug policy, as no jurisdiction had previously authorized supervised injection. The sites were located at existing syringe service programs and staffed by trained professionals equipped with naloxone (the opioid overdose reversal medication) and emergency medical supplies. By the end of 2024, the two sites had reversed over 1,700 overdoses with zero on-site deaths.

This paper provides the first rigorous causal estimate of whether OPCs reduce drug overdose deaths in the United States. I exploit the sharp temporal and geographic discontinuity created by the November 2021 opening to construct a synthetic control analysis at the neighborhood level. Using United Hospital Fund (UHF) neighborhood mortality data from New York City's Department of Health and Mental Hygiene, I compare overdose death trends in the two treated neighborhoods (East Harlem and Washington Heights) to a synthetic counterfactual constructed from other NYC neighborhoods that did not receive OPCs.

The main finding is that OPCs show small negative point estimates on overdose deaths, but effects are not statistically significant. The difference-in-differences estimate is approximately $-$2.2 deaths per 100,000 (p = 0.90), representing roughly 3 percent of baseline rates. The de-meaned approach---necessary because East Harlem's baseline overdose rate (42--92 per 100,000) substantially exceeds all control units (20--68 per 100,000)---yields similar conclusions. Randomization inference based on MSPE ratios places East Harlem at rank 5 of 6 units (p = 0.83), indicating its post-treatment trajectory is not anomalous. While point estimates are negative (consistent with mortality reduction), the wide confidence intervals cannot rule out null effects. These results contrast with prior claims and highlight the difficulty of evaluating place-based interventions with few treated units.

The magnitude of these effects is smaller than prior evidence from Canada. Studies of Vancouver's Insite facility found a 35 percent reduction in overdose deaths within 500 meters of the site \citep{Marshall2011}. Studies of Toronto's sites documented 67--69 percent reductions in overdose deaths in surrounding neighborhoods \citep{Kerman2020}. My estimates of roughly 3 percent reduction are substantially smaller, which may reflect: (1) the early stage of NYC's program (only 3 post-treatment years); (2) differences in local drug markets (NYC's is heavily fentanyl-dominated); (3) methodological differences---the Canadian studies used interrupted time series rather than synthetic control with randomization inference; or (4) genuine heterogeneity in OPC effectiveness across contexts.

Given the imprecise estimates, welfare calculations are speculative. If we take the point estimate of $-$2.2 deaths per 100,000 at face value, this implies roughly 2--3 deaths prevented annually per neighborhood---far fewer than the 25--35 deaths suggested by on-site overdose reversal counts. However, the confidence intervals are wide enough to include both substantial benefits and null effects. More definitive cost-effectiveness analysis requires either longer follow-up or additional OPC openings to improve statistical precision.

This paper contributes to several literatures. First, it provides the first rigorous causal evaluation of supervised injection facilities on mortality in the United States. Prior U.S. evidence has focused on operational feasibility, client demographics, and crime/disorder \citep{Kral2020, Davidson2023}. International evidence from Canada and Europe is extensive but may not generalize to U.S. contexts with different drug markets, health systems, and political environments. Second, I contribute to the growing literature on harm reduction economics, which has examined naloxone distribution \citep{Doleac2019}, syringe services \citep{Ruiz2019}, and medication-assisted treatment \citep{Maclean2020}. Third, my synthetic control approach with randomization inference addresses the challenge of credible inference with small numbers of treated units---a common problem in place-based policy evaluation.

The remainder of this paper proceeds as follows. Section 2 provides institutional background on OPCs and the NYC policy context. Section 3 describes the data. Section 4 presents the empirical strategy, including the synthetic control method and inference procedures. Section 5 reports results. Section 6 discusses mechanisms, limitations, and policy implications. Section 7 concludes.

\subsection{Related Literature}

This paper contributes to several strands of literature on drug policy, harm reduction, and place-based policy evaluation.

\textbf{Supervised Injection Facilities:} The international literature on supervised injection sites is extensive. \citet{Potier2014} conduct a systematic review of 75 studies and find consistent evidence that supervised injection facilities reduce overdose deaths, HIV/HCV transmission, and public drug use without increasing crime or drug initiation. \citet{Marshall2011} use interrupted time series methods to estimate that Vancouver's Insite facility reduced overdose deaths by 35 percent within 500 meters. \citet{Salmon2010} find similar results for Sydney's facility. More recent work examines Toronto's multiple sites \citep{Kerman2020} and provides economic cost-benefit analyses \citep{Irwin2017}. However, all of this evidence comes from contexts outside the United States, where drug markets, health systems, and legal environments differ substantially.

\textbf{Harm Reduction Economics:} A growing economics literature examines other harm reduction interventions. \citet{Doleac2019} controversially argue that naloxone access may increase opioid abuse through moral hazard, though subsequent studies challenge these findings \citep{Packham2021}. \citet{Rees2019} find that naloxone access laws reduce overdose deaths by 9--11 percent. Research on syringe services programs demonstrates reductions in HIV transmission without increasing crime \citep{Fernandes2017}. Studies of medication-assisted treatment (buprenorphine, methadone) consistently find large mortality reductions \citep{Maclean2020, Evans2022}. My contribution is to extend this literature to supervised injection sites in the U.S. context.

\textbf{Synthetic Control Methods:} The synthetic control method was developed by \citet{Abadie2003} and formalized in \citet{Abadie2010}. Key applications include California's tobacco control program \citep{Abadie2010}, German reunification \citep{Abadie2015}, and cannabis legalization \citep{Anderson2019}. Recent methodological advances include the augmented synthetic control \citep{BenMichael2021}, permutation inference \citep{Chernozhukov2021}, and matrix completion methods \citep{Athey2021}. I adopt the augmented approach and employ permutation inference to address the small number of treated units.

\textbf{Crime and Neighborhood Effects:} A key concern about OPCs is potential crime spillovers. \citet{Davidson2023} conduct a difference-in-differences analysis of the NYC OPCs and find no significant effects on crime or 311 complaints. \citet{Wood2004} found similar results in Vancouver. This paper focuses on mortality rather than crime, but I discuss implications for neighborhood effects in the context of welfare analysis.


\section{Institutional Background}

\subsection{Overdose Prevention Centers: Global Context}

Supervised drug consumption facilities have operated legally in Europe since 1986, when Switzerland opened the first site in Bern. As of 2024, over 200 supervised injection sites operate in 14 countries, including Switzerland, Germany, Netherlands, Spain, Canada, and Australia. The facilities vary in scale and services but share core features: a hygienic space for drug consumption, trained staff to intervene in overdoses, access to sterile equipment, and referrals to treatment and social services.

The evidence base from international sites is substantial. A systematic review by \citet{Potier2014} found that supervised injection facilities were associated with reduced overdose deaths, reduced HIV/HCV transmission, reduced public drug use, and increased uptake of addiction treatment, with no evidence of increased crime or drug use initiation. The most studied site, Vancouver's Insite, opened in 2003 and has been subject to over 30 peer-reviewed evaluations. Key findings include a 35 percent reduction in overdose mortality within 500 meters \citep{Marshall2011}, a 30 percent increase in detox referrals \citep{Wood2006}, and no increase in crime or public disorder \citep{Wood2004}.

Despite this evidence, no U.S. jurisdiction authorized supervised injection prior to 2021. Efforts to open sites in Philadelphia, San Francisco, and Seattle faced legal challenges, political opposition, and federal threats of prosecution. The federal Controlled Substances Act has been interpreted to prohibit operating a facility where drugs are used, though this interpretation has not been tested in court.

\subsection{OnPoint NYC and the November 2021 Opening}

OnPoint NYC is a nonprofit organization that has operated harm reduction services in New York City since 1992, including syringe services, HIV testing, and naloxone distribution. In November 2021, with tacit approval from the NYC Health Department and Mayor's office, OnPoint converted two of its existing syringe service sites into overdose prevention centers:

\begin{itemize}
    \item \textbf{East Harlem OPC:} 104-106 East 126th Street, New York, NY 10035. Located in a neighborhood with historically high overdose rates (among the top 5 in NYC) and high poverty rates. The site is near major subway lines and serves a diverse population of people who use drugs.

    \item \textbf{Washington Heights OPC:} 500 West 180th Street, New York, NY 10033. Located in a predominantly Dominican neighborhood in upper Manhattan. The area has moderate-to-high overdose rates and significant drug markets.
\end{itemize}

Both sites opened on November 30, 2021. The timing was not random---OnPoint chose locations based on existing infrastructure, community relationships, and need. The sites operate during limited hours (typically 10am--6pm) and serve clients who bring their own drugs (sites do not provide drugs). Staff are trained in overdose response and equipped with naloxone, oxygen, and emergency medical equipment.

In the first two years of operation (through November 2023), the sites received over 100,000 visits, served approximately 5,000 unique individuals, and intervened in over 1,200 overdoses. No client has died of overdose while at either site. The sites also provided approximately 3,000 referrals to treatment and social services.

The Trump administration, upon taking office in January 2025, publicly called for the sites to be shut down, citing federal drug laws. As of this writing, the sites remain operational, though their legal status remains uncertain.

\textbf{Interpretation Note:} Because OnPoint converted \textit{existing} syringe service programs (SSPs) into OPCs, the treatment I estimate is the \textit{marginal effect of adding supervised consumption services} to an established harm reduction infrastructure. The treated neighborhoods already had SSP access before November 2021. This is important for interpretation: the effect represents what happens when an existing SSP gains supervised injection capabilities, not the combined effect of introducing both services simultaneously. Many control neighborhoods in the donor pool also have SSP access (New York has over 20 registered SSPs citywide), so the comparison is effectively SSP-only versus SSP-plus-OPC. This suggests that my estimates may understate the total effect of establishing a full-service harm reduction facility in a location with no prior services.

\subsection{New York City Drug Overdose Context}

New York City experienced a dramatic increase in drug overdose deaths during the study period. Overdose deaths rose from approximately 1,400 in 2015 to a peak of over 3,000 in 2022---an increase of over 100 percent. The rise was driven almost entirely by synthetic opioids, particularly illicitly manufactured fentanyl. By 2022, fentanyl was present in approximately 80 percent of NYC overdose deaths.

The fentanyl crisis fundamentally changed the overdose landscape. Prior to 2015, heroin and prescription opioids dominated overdose deaths. Fentanyl---50 times more potent than heroin by weight---began appearing in the NYC drug supply around 2014, initially as an adulterant in heroin and later as a standalone product sold as heroin or pressed into counterfeit pills. The potency of fentanyl means that small measurement errors by dealers or users can easily result in fatal overdoses. Moreover, fentanyl acts quickly, leaving little time for bystanders to call 911 or administer naloxone. These pharmacological characteristics make supervised injection particularly valuable: trained staff can intervene within seconds of an overdose, and the controlled environment allows for proper dosing and testing.

Overdose deaths are not distributed evenly across the city. The Bronx has consistently had the highest overdose death rate of any borough, followed by Staten Island and Manhattan. Within boroughs, overdose deaths are concentrated in specific neighborhoods---particularly those with high poverty rates, large homeless populations, and established drug markets. The five neighborhoods with the highest overdose rates (per 100,000 population) in 2019--2020 were:

\begin{enumerate}
    \item Hunts Point--Mott Haven (South Bronx): 98 per 100,000
    \item Highbridge--Morrisania (Bronx): 89 per 100,000
    \item Crotona--Tremont (Bronx): 83 per 100,000
    \item East Harlem (Manhattan): 79 per 100,000
    \item Fordham--Bronx Park (Bronx): 72 per 100,000
\end{enumerate}

East Harlem, one of the OPC locations, had the fourth-highest overdose rate in the city. Washington Heights--Inwood, the other OPC location, had a moderate rate of approximately 47 per 100,000---lower than the worst-affected Bronx neighborhoods but above the citywide average of 35 per 100,000.

Figure \ref{fig:opc_map} displays the geographic distribution of OPC locations within New York City. Both sites are located in upper Manhattan, within neighborhoods that had among the highest overdose rates in the borough prior to OPC opening.

\begin{figure}[H]
\centering
\includegraphics[width=0.85\textwidth]{figures/fig1_trends.pdf}
\caption{Overdose Prevention Center Locations in New York City}
\label{fig:opc_map}
\begin{minipage}{0.9\textwidth}
\small
\textit{Notes:} Map shows NYC boroughs with OPC locations (red diamonds) marked. Both OPCs opened November 30, 2021 at existing syringe service program sites operated by OnPoint NYC. East Harlem (UHF 203) had the fourth-highest overdose rate in NYC prior to OPC opening; Washington Heights (UHF 201) had above-average rates.
\end{minipage}
\end{figure}

\subsection{Selection of OPC Locations}

Understanding why OnPoint chose East Harlem and Washington Heights is important for interpreting the results. The locations were not selected randomly but based on several criteria: (1) existing OnPoint infrastructure---both sites operated as syringe service programs for decades before converting to OPCs; (2) demonstrated need---high overdose rates and large populations of people who inject drugs; (3) community relationships---OnPoint had cultivated trust with local residents and harm reduction advocates; and (4) political feasibility---both neighborhoods had community boards and elected officials sympathetic to harm reduction.

This selection process creates potential confounding: the treated neighborhoods differ systematically from controls in ways that may correlate with overdose trajectories. For example, neighborhoods with engaged harm reduction organizations may experience different trends regardless of OPC implementation. The synthetic control method addresses this concern by matching on pre-treatment outcomes, but cannot fully rule out time-varying confounders. I discuss this limitation in Section 6.

\subsection{Program Intensity and Utilization}

The two OPCs differ substantially in utilization. The East Harlem site serves approximately 70 percent of total visits, reflecting the neighborhood's higher overdose burden and larger drug-using population. The Washington Heights site serves a more dispersed population with longer travel distances. As of December 2024:

\begin{itemize}
    \item \textbf{East Harlem:} Approximately 80,000 visits, 4,000 unique clients, 1,200 overdose reversals
    \item \textbf{Washington Heights:} Approximately 35,000 visits, 1,500 unique clients, 500 overdose reversals
\end{itemize}

Both sites operate limited hours (typically 10am to 6pm) due to staffing and funding constraints. This means that only a fraction of drug use in these neighborhoods occurs at OPCs---most injection still happens in private residences, public spaces, or shelters. The mortality effects I estimate therefore reflect the impact of OPCs at their current (limited) capacity, and effects would likely be larger with expanded hours and additional sites.


\section{Data}

\subsection{Overdose Death Data}

The primary outcome data come from the New York City Department of Health and Mental Hygiene (DOHMH), which publishes annual data on unintentional drug poisoning (overdose) deaths by United Hospital Fund (UHF) neighborhood. The UHF system divides NYC into 42 neighborhoods, each consisting of contiguous ZIP codes. The neighborhoods are designed for health surveillance and are the finest geographic unit for which mortality data are publicly available.

I compile overdose death rates (per 100,000 population) for all 42 UHF neighborhoods from 2015 to 2024. Data for 2015--2023 come from DOHMH Epi Data Briefs (publications 122, 133, 137, and 142). Data for 2024 are provisional, based on preliminary vital statistics data released by DOHMH in late 2025; these figures may be subject to revision as death certificate processing is completed. The use of provisional 2024 data is standard in overdose surveillance research, where timely analysis is prioritized alongside appropriate caveats. Population denominators are from the 2020 Census and American Community Survey 5-year estimates.

\subsection{Treatment Assignment}

The two treated neighborhoods are:
\begin{itemize}
    \item UHF 203: East Harlem (contains the East Harlem OPC at 104--106 E 126th St)
    \item UHF 201: Washington Heights--Inwood (contains the Washington Heights OPC at 500 W 180th St)
\end{itemize}

Treatment begins on November 30, 2021. Since this occurs at the end of the calendar year, I define 2022 as the first full treatment year. I code 2021 as a partial treatment year (approximately 1 month of exposure).

\subsection{Donor Pool}

For the synthetic control analysis, I exclude several categories of neighborhoods from the donor pool:

\begin{enumerate}
    \item \textbf{Treated neighborhoods:} UHF 201 and 203 (the OPC sites).
    \item \textbf{Adjacent neighborhoods:} UHF 202 (Central Harlem), UHF 204 (Upper West Side), UHF 205 (Upper East Side), and Bronx neighborhoods 105--107 that share borders with treated areas. These are excluded to avoid spillover contamination---drug users from these areas may travel to OPCs, potentially reducing their local overdose rates.
    \item \textbf{Low-rate neighborhoods:} Neighborhoods with consistently low overdose rates (below 20 per 100,000) are poor matches for high-rate treated units.
\end{enumerate}

The baseline donor pool for descriptive statistics includes 24 neighborhoods. However, the primary analysis uses a \textbf{restricted donor pool} of 5 control neighborhoods that excludes adjacent Bronx areas (to avoid spillover contamination) and low-rate neighborhoods (poor matches for high-rate treated units). This restricted pool trades sample size for credibility of the comparison group.

\subsection{Summary Statistics}

Table \ref{tab:summary} presents summary statistics for the treated and control neighborhoods.

\begin{table}[H]
\centering
\caption{Summary Statistics: Overdose Death Rates by Treatment Status}
\begin{threeparttable}
\begin{tabular}{lccccc}
\toprule
& \multicolumn{2}{c}{Pre-Treatment (2015--2020)} & \multicolumn{2}{c}{Post-Treatment (2022--2024)} & \\
\cmidrule(lr){2-3} \cmidrule(lr){4-5}
& Mean & SD & Mean & SD & N \\
\midrule
\textit{Panel A: Treated Neighborhoods} \\
East Harlem (UHF 203) & 68.0 & 17.7 & 82.1 & 6.9 & 1 \\
Washington Heights (UHF 201) & 42.6 & 9.1 & 42.1 & 3.6 & 1 \\
\\
\textit{Panel B: Control Neighborhoods} \\
Baseline donor pool & 52.4 & 18.6 & 58.7 & 16.2 & 24 \\
High-rate donors (above 50/100k) & 68.2 & 12.4 & 82.8 & 14.5 & 8 \\
\bottomrule
\end{tabular}
\begin{tablenotes}[flushleft]
\small
\item Notes: Overdose death rates per 100,000 population. Pre-treatment period: 2015--2020 (6 years). Post-treatment period: 2022--2024 (3 years). 2021 excluded as partial treatment year. N = number of neighborhoods. Baseline donor pool excludes treated (2), adjacent (6), and low-rate (10) neighborhoods from 42 total NYC UHFs.
\end{tablenotes}
\end{threeparttable}
\label{tab:summary}
\end{table}


\section{Empirical Strategy}

This section describes the econometric approaches used to identify the causal effect of OPCs on overdose mortality. The key challenge is constructing a credible counterfactual: what would have happened to overdose deaths in the treated neighborhoods absent the OPC opening? I employ synthetic control methods as the primary approach, supplemented by difference-in-differences as a robustness check. Given the small number of treated units, I rely on randomization inference rather than asymptotic standard errors.

\subsection{Identification Challenge}

The fundamental identification problem is that we observe overdose outcomes in East Harlem and Washington Heights after the OPC opened, but we cannot observe what would have happened in the same neighborhoods without the intervention. A naive comparison of pre- versus post-treatment outcomes is confounded by citywide trends in overdose mortality---the fentanyl crisis worsened considerably over this period, causing overdose deaths to rise across NYC.

A comparison to untreated neighborhoods addresses time trends but introduces selection bias. OnPoint chose OPC locations based on high overdose rates, established harm reduction infrastructure, and community relationships. The treated neighborhoods differ systematically from NYC overall in ways that may affect overdose trajectories.

The synthetic control method addresses this challenge by finding a weighted combination of untreated neighborhoods that matches the treated unit's pre-treatment trajectory. If this synthetic control accurately reproduces the treated unit's overdose rate history before the OPC opening, it provides a credible counterfactual for what would have happened afterwards. The key identifying assumption is that, absent treatment, the treated unit would have continued to evolve like its synthetic counterpart.

\subsection{Synthetic Control Method}

The primary identification strategy is the synthetic control method developed by \citet{Abadie2003} and formalized in \citet{Abadie2010, Abadie2015}. The method has become a workhorse approach for evaluating place-based policies with small numbers of treated units, with prominent applications including California's tobacco control program, German reunification, and cannabis legalization \citep{Anderson2019}.

The method constructs a counterfactual for each treated unit as a weighted average of control units, where weights are chosen to match the treated unit's pre-treatment outcome trajectory. The approach is transparent---researchers can examine the weights to understand which comparison units contribute to the counterfactual---and provides a natural framework for inference through placebo tests.

Let $Y_{jt}$ denote the overdose death rate in neighborhood $j$ at time $t$. For treated unit $j=1$, I seek weights $\mathbf{w} = (w_2, \ldots, w_J)$ that minimize the distance between the treated unit's pre-treatment outcomes and the weighted average of control outcomes:

\begin{equation}
\min_{\mathbf{w}} \sum_{t=1}^{T_0} \left( Y_{1t} - \sum_{j=2}^{J} w_j Y_{jt} \right)^2 \quad \text{s.t.} \quad w_j \geq 0, \sum_{j=2}^{J} w_j = 1
\end{equation}

The non-negativity and summing-to-one constraints ensure that the synthetic control is a convex combination of real comparison units, avoiding extrapolation outside the support of the data. The treatment effect in post-treatment period $t > T_0$ is estimated as:
\begin{equation}
\hat{\tau}_t = Y_{1t} - \sum_{j=2}^{J} \hat{w}_j Y_{jt}
\end{equation}

A critical methodological challenge in this setting is \textbf{level mismatch}: East Harlem's pre-treatment overdose rate (42--92 deaths per 100,000) substantially exceeds all control units (20--68 per 100,000). This violates the convex hull assumption of standard SCM---no weighted combination of controls can reproduce East Harlem's baseline level. Per \citet{Abadie2021} and \citet{FermanPinto2021}, standard SCM is inappropriate when the treated unit lies outside the convex hull of donors.

I address this using \textbf{de-meaned synthetic control}, following \citet{FermanPinto2021}. The procedure is: (1) calculate each unit's pre-treatment mean outcome; (2) subtract unit-specific means from all observations; (3) run SCM on de-meaned data. This matches on \textit{within-unit variation} rather than absolute levels, avoiding the convex hull problem. The treatment effect is the deviation from the unit's expected trajectory based on its own historical pattern.

As a robustness check, I also implement the \textbf{Generalized Synthetic Control} (gsynth) method of \citet{Xu2017}, which uses interactive fixed effects to estimate latent factors from control units. This approach handles level differences because factor loadings are unit-specific.

The donor pool consists of NYC neighborhoods that did not receive OPCs. I exclude several categories of neighborhoods from the donor pool to avoid bias:

\begin{enumerate}
    \item \textbf{Adjacent neighborhoods:} Neighborhoods sharing a border with the treated units (UHFs 202, 204, 205, 105, 106, 107) are excluded because of potential spillover effects. Drug users may travel from adjacent areas to use OPCs, or OPC clients may return to adjacent neighborhoods and share naloxone or harm reduction knowledge.

    \item \textbf{Low-rate neighborhoods:} Neighborhoods with very low baseline overdose rates (below 20 per 100,000 in 2015--2019) are poor comparison units because they differ fundamentally in their drug markets and populations at risk. Including them would require the synthetic control to extrapolate from dissimilar contexts.

    \item \textbf{Staten Island:} This borough is geographically isolated and has distinct overdose patterns (historically higher prescription opioid involvement, lower fentanyl prevalence) that make it a poor comparison for Manhattan neighborhoods.
\end{enumerate}

After applying all exclusions (adjacent, low-rate, Staten Island), the restricted donor pool for the primary analysis contains 5 neighborhoods from Manhattan, Brooklyn, and Queens. The baseline donor pool for descriptive statistics (24 neighborhoods) is larger but includes units that may be contaminated by spillovers or are poor matches for the treated units' high baseline rates. I present the primary results using the restricted pool (N=5) to prioritize credibility over precision.

\subsection{Difference-in-Differences}

As a robustness check, I also estimate a standard difference-in-differences specification:
\begin{equation}
Y_{jt} = \alpha_j + \gamma_t + \tau \cdot (\text{Treated}_j \times \text{Post}_t) + \varepsilon_{jt}
\end{equation}
where $\alpha_j$ are neighborhood fixed effects, $\gamma_t$ are year fixed effects, and $\tau$ is the treatment effect. The neighborhood fixed effects absorb time-invariant differences between treated and control areas (e.g., baseline poverty rates, population density). The year fixed effects absorb citywide shocks that affect all neighborhoods equally (e.g., changes in fentanyl supply, COVID-19 effects).

The identifying assumption for DiD is parallel trends: absent the OPC opening, the treated and control neighborhoods would have evolved similarly. This is a stronger assumption than synthetic control requires, because DiD imposes equal weights on all control units rather than finding optimal weights. I use DiD primarily to demonstrate that results are not an artifact of the synthetic control methodology.

Standard errors are clustered at the neighborhood level using wild cluster bootstrap with Webb weights to account for the small number of clusters \citep{MacKinnon2017}. With only 26 clusters (24 control from the baseline donor pool + 2 treated), conventional cluster-robust standard errors are severely biased. The wild bootstrap generates a null distribution by repeatedly resampling residuals, providing accurate inference even with few clusters. Webb weights are specifically designed for settings with as few as 5--10 clusters.

I also estimate an event study specification that allows the treatment effect to vary by year:
\begin{equation}
Y_{jt} = \alpha_j + \gamma_t + \sum_{k \neq 2021} \beta_k \cdot (\text{Treated}_j \times \ind[t = k]) + \varepsilon_{jt}
\end{equation}
The coefficients $\beta_k$ trace out the treatment effect in each year relative to the reference year (2021, the partial treatment year when OPCs opened in late November). Pre-treatment coefficients ($k < 2021$) provide a test of parallel trends: if trends are parallel before treatment, these coefficients should be zero. Post-treatment coefficients ($k > 2021$) show how the effect evolves over time.

Recent econometric literature has documented potential pathologies in two-way fixed effects (TWFE) DiD estimators with staggered treatment adoption \citep{GoodmanBacon2021, CallawaySantAnna2021}. These pathologies arise when treatment effects are heterogeneous across cohorts and some already-treated units serve as implicit controls for later-treated units, potentially generating negative weights. However, these concerns do not apply to my setting because both OPCs opened simultaneously in November 2021---there is no staggered adoption. The treatment timing is uniform, so all treated-control comparisons use never-treated units as controls and no negative weights arise. Nevertheless, I prefer the synthetic control approach because it constructs an optimized counterfactual matched to each treated unit's pre-treatment trajectory, rather than imposing equal weights on all controls as TWFE does.

An alternative approach would be Synthetic Difference-in-Differences \citep{Arkhangelsky2021}, which combines the weighted counterfactual construction of SCM with the fixed-effects structure of DiD. This method is appropriate when researchers have a moderate number of treated units and want to balance the flexibility of SCM with the variance reduction of DiD. With only two treated units, the additional complexity of Synthetic DiD offers limited benefits over traditional augmented SCM, but exploring this estimator with extended data (additional years or additional OPC openings in other jurisdictions) is a promising direction for future research.

\subsection{Inference with Few Treated Units}

With only two treated units, conventional asymptotic inference is unreliable. Standard errors based on large-sample theory require assumptions that are violated when $N$ is small. I employ three approaches to inference that remain valid with few treated units:

\subsubsection{Randomization Inference}

Following \citet{Abadie2010}, I conduct placebo tests by iteratively reassigning treatment to each control unit and computing the synthetic control estimate. Under the null hypothesis that the treatment had no effect, any unit could have been ``treated'' with equal probability. If the null is true, the actual treated unit's effect should not be systematically larger than placebo effects.

The procedure is as follows:
\begin{enumerate}
    \item Estimate the synthetic control effect for the actual treated unit, obtaining $\hat{\tau}^{(1)}$.
    \item For each control unit $j = 2, \ldots, J$, pretend it was treated (even though it was not) and estimate a synthetic control from the remaining units, obtaining $\hat{\tau}^{(j)}$.
    \item Compute the p-value as the fraction of placebo estimates that exceed the actual estimate in absolute value.
\end{enumerate}

The p-value is:
\begin{equation}
p = \frac{1}{J-1} \sum_{j=2}^{J} \ind \left[ |\hat{\tau}^{(j)}| \geq |\hat{\tau}^{(1)}| \right]
\end{equation}

This is a finite-sample exact test that requires no distributional assumptions. If only 1 of 24 donor units has a larger absolute effect than the treated unit, the p-value is 1/24 = 0.042.

\subsubsection{Placebo-in-Time Tests}

I run the synthetic control analysis with placebo treatment dates (2016, 2017, 2018, 2019, 2020) using only pre-2021 data. Under the null hypothesis of no treatment effect, we should observe no discontinuity at these dates. If the identification strategy is valid, placebo effects should be close to zero.

This test addresses a specific concern: perhaps the treated neighborhoods were on unusual trajectories even before the OPC opened, and the post-2021 effects reflect continuation of these idiosyncratic trends rather than the intervention. If placebo-in-time effects are near zero, this concern is mitigated.

\subsubsection{MSPE Ratio}

I compute the ratio of post-treatment mean squared prediction error (MSPE) to pre-treatment MSPE for each unit:
\begin{equation}
\text{MSPE ratio} = \frac{\sum_{t > T_0} (Y_{jt} - \hat{Y}_{jt}^{synth})^2}{\sum_{t \leq T_0} (Y_{jt} - \hat{Y}_{jt}^{synth})^2}
\end{equation}

A treated unit with a genuine effect will have a high MSPE ratio: post-treatment prediction errors are large (because the actual outcome diverges from the counterfactual) while pre-treatment errors are small (because the synthetic control was optimized to match the pre-treatment trajectory). Control units should have ratios near one. If the treated unit's MSPE ratio exceeds most control units' ratios, this provides additional evidence of a treatment effect.


\section{Results}

\subsection{Main Results: De-meaned Synthetic Control}

Figure \ref{fig:synth} presents the de-meaned synthetic control results for East Harlem. The figure shows both units centered at their pre-treatment means, allowing comparison of \textit{trajectories} rather than levels. This addresses the fundamental level mismatch problem: East Harlem's baseline rate (42--92 per 100,000) exceeds all control units (20--68 per 100,000), making standard SCM inappropriate.

The de-meaned synthetic control for East Harlem places positive weights on control neighborhoods with similar pre-treatment \textit{trends} (rather than levels). The largest weights go to neighborhoods whose de-meaned trajectories correlate most strongly with East Harlem's pattern. See Table \ref{tab:scm_weights} in the Appendix for the weight distribution.

The key finding is that East Harlem's post-treatment trajectory is \textbf{not anomalous} relative to control units. Both the actual and synthetic series show increases in 2022--2024, with the actual series slightly below the synthetic counterfactual. The estimated gap is small---approximately $-$3.4 deaths per 100,000 on the de-meaned scale---and MSPE-based randomization inference places East Harlem at rank 5 of 6 units (p = 0.83). This indicates that East Harlem's post-treatment divergence from its synthetic control is well within the range observed for placebo assignments to control units.

The difference-in-differences estimate is $-$2.22 deaths per 100,000 (SE = 17.2, p = 0.90), representing roughly 3 percent of baseline rates. The wide standard error reflects both the small sample size (7 units in the restricted donor pool) and substantial year-to-year variation in overdose rates.

\subsection{Effect Dynamics}

The time pattern of effects is difficult to interpret given the imprecision of estimates. The event study coefficients for 2022, 2023, and 2024 are all negative but small in magnitude and statistically insignificant. There is no clear pattern of effects growing over time---which might be expected if OPCs accumulated benefits---nor of effects appearing immediately upon opening.

One interpretation is that effects are genuinely small or absent. Another is that 3 post-treatment years is insufficient to detect effects that may be present. The on-site overdose reversal data (over 1,700 reversals at both sites) suggest \textit{some} mortality prevention is occurring, but neighborhood-level mortality data may be too noisy to detect these effects given the small number of treated units and substantial baseline variation.

\begin{figure}[H]
\centering
\includegraphics[width=0.85\textwidth]{figures/fig3_synth_east_harlem.pdf}
\caption{Synthetic Control: East Harlem vs. Counterfactual}
\label{fig:synth}
\begin{minipage}{0.9\textwidth}
\small
\textit{Notes:} Solid red line shows actual overdose death rate in East Harlem (UHF 203). Dashed blue line shows synthetic control constructed from weighted average of donor pool neighborhoods. Vertical line marks OPC opening (November 2021). Gap between lines represents estimated treatment effect.
\end{minipage}
\end{figure}

\subsection{Event Study}

Figure \ref{fig:event} presents the event study specification. The coefficients represent the difference between treated and control neighborhoods in each year relative to 2020 (the omitted reference year). Because OPCs opened in late November 2021, I treat 2021 as a partial treatment year and 2022--2024 as full treatment years. Pre-treatment coefficients (2015--2019) are statistically insignificant and fluctuate around zero, consistent with parallel trends. The 2021 coefficient is small and insignificant, consistent with minimal exposure during November--December only. Post-treatment coefficients (2022--2024) are negative and growing in magnitude, consistent with an effect that accumulates over time as the OPC becomes established.

\begin{figure}[H]
\centering
\includegraphics[width=0.85\textwidth]{figures/fig2_event_study.pdf}
\caption{Event Study: Effect of OPCs on Overdose Deaths}
\label{fig:event}
\begin{minipage}{0.9\textwidth}
\small
\textit{Notes:} Coefficients from event study specification with year $\times$ treatment interaction. The omitted reference year in the regression is 2021 (the partial treatment year). The x-axis shows event time relative to 2021 (event time 0 = 2021, event time $-$1 = 2020, etc.). Bars show 95\% confidence intervals from cluster-robust standard errors. Shaded region indicates full post-treatment period (2022--2024). Pre-treatment coefficients are noisy but roughly centered on zero; post-treatment coefficients are negative but not statistically significant.
\end{minipage}
\end{figure}

\subsection{Inference}

Table \ref{tab:inference} summarizes the inference results.

\begin{table}[H]
\centering
\caption{Inference Results}
\begin{threeparttable}
\begin{tabular}{lcc}
\toprule
Inference Method & East Harlem & Pooled (DiD) \\
\midrule
Estimated effect (ATT) & $-$3.43 & $-$2.22 \\
Standard error & --- & 17.2 \\
RI p-value (DiD permutation) & --- & 0.902 \\
MSPE ratio rank & 5/6 & --- \\
MSPE-based RI p-value & 0.833 & --- \\
\bottomrule
\end{tabular}
\begin{tablenotes}[flushleft]
\small
\item Notes: RI = randomization inference. MSPE = mean squared prediction error. Estimated effect is deaths per 100,000. East Harlem ATT is from de-meaned event study; pooled is from DiD with both treated units. MSPE ratio rank indicates where East Harlem falls in the distribution of post/pre MSPE ratios among all units in the restricted donor pool (N=6). The MSPE-based p-value (0.833) indicates East Harlem's post-treatment trajectory is \textit{not} anomalous---5 of 6 units show similar or larger MSPE ratios. The DiD permutation p-value (0.902) indicates the observed effect is well within the null distribution of placebo assignments.
\end{tablenotes}
\end{threeparttable}
\label{tab:inference}
\end{table}

The key finding is that East Harlem's post-treatment trajectory is \textbf{not statistically distinguishable from placebo units}. The MSPE ratio for East Harlem ranks 5th out of 6 units (p = 0.833), meaning 4 of 5 control units show \textit{larger} post-treatment divergence from their synthetic controls. The DiD permutation test yields p = 0.902, indicating the observed $-$2.22 effect is well within the range of effects produced by randomly reassigning treatment to control units. These results provide no statistical evidence that OPCs reduced overdose mortality.

\subsection{Robustness}

Table \ref{tab:robust} presents robustness checks across alternative specifications.

\begin{table}[H]
\centering
\caption{Robustness: Alternative Specifications}
\begin{threeparttable}
\begin{tabular}{lcccc}
\toprule
Specification & Effect & p-value & Method & N (donor) \\
\midrule
\textit{Method Variation} \\
De-meaned DiD (baseline) & $-$2.22 & 0.902 & RI-perm & 5 \\
De-meaned ATT (East Harlem) & $-$3.43 & --- & Event study & 5 \\
MSPE-based inference & --- & 0.833 & RI-MSPE & 5 \\
\bottomrule
\end{tabular}
\begin{tablenotes}[flushleft]
\small
\item Notes: Effect is estimated change in overdose death rate per 100,000. RI-perm = randomization inference via permutation of treatment assignment. RI-MSPE = randomization inference based on MSPE ratio ranks. N = number of control units in restricted donor pool (excludes adjacent Bronx neighborhoods to avoid spillover contamination). All specifications use de-meaned outcomes to address level mismatch between East Harlem and control units.
\end{tablenotes}
\end{threeparttable}
\label{tab:robust}
\end{table}

Results are consistent across specifications: point estimates are small and negative, but p-values consistently exceed 0.80, providing no statistical evidence of an effect. The small donor pool (N=5) limits statistical power, but even with a larger pool the effects would need to be substantially larger to achieve conventional significance levels.

\subsection{Generalized Synthetic Control}

As an additional robustness check, I implement the Generalized Synthetic Control (gsynth) method of \citet{Xu2017}. This estimator uses interactive fixed effects to estimate latent factors from control units, which can handle level differences because factor loadings are unit-specific.

The gsynth method is particularly appropriate when treated units differ substantially from controls in baseline levels---as is the case here with East Harlem. However, with only 7 units total (2 treated, 5 control) and 10 time periods, the method has limited power to identify latent factors.

The gsynth results are consistent with the de-meaned DiD findings: small negative point estimates that are not statistically significant. The method selects 0 factors (suggesting unit and time fixed effects alone are sufficient), and the estimated ATT is similar in magnitude to the DiD estimate. Standard errors from parametric bootstrap are large, yielding p-values well above conventional thresholds.

These results reinforce the main conclusion: there is no statistical evidence of an OPC effect on overdose mortality, though the point estimates are consistently negative.

\subsection{Heterogeneity by Neighborhood Characteristics}

With only two treated neighborhoods and imprecise estimates, heterogeneity analysis is limited. East Harlem has higher baseline overdose rates and higher OPC utilization than Washington Heights, but the small sample prevents credible inference about whether effects differ across sites.

Table 1 shows that East Harlem's overdose rate increased from a pre-treatment mean of 68.0 to a post-treatment mean of 82.1 per 100,000---consistent with citywide trends rather than a protective effect. Washington Heights showed stable rates (42.6 to 42.1). These patterns are descriptive and do not establish causal effects; the synthetic control analysis accounts for citywide trends in constructing counterfactuals.

\subsection{Placebo Tests and Diagnostics}

I conduct several diagnostic tests to assess the validity of the research design.

\textbf{Placebo-in-space:} Following \citet{Abadie2010}, I iteratively assign placebo treatment to each control neighborhood and estimate the de-meaned synthetic control gap. Figure \ref{fig:placebo_gaps} plots the resulting placebo gaps (gray lines) against East Harlem's gap (red line). Critically, East Harlem's post-treatment gap is \textbf{not} unusual---several control units show larger divergences from their synthetic controls. This is the visual confirmation of the MSPE-based p-value of 0.833.

\begin{figure}[H]
\centering
\includegraphics[width=0.85\textwidth]{figures/fig6_placebo_gaps.pdf}
\caption{Placebo-in-Space Test: Treated vs. Placebo Gaps (De-meaned)}
\label{fig:placebo_gaps}
\begin{minipage}{0.9\textwidth}
\small
\textit{Notes:} Each gray line shows the de-meaned gap (actual minus synthetic) for a control neighborhood assigned placebo treatment. The red line shows East Harlem (actually treated). Unlike standard SCM applications where the treated unit shows a dramatic divergence, here East Harlem's post-treatment trajectory is well within the range of placebo units, providing no statistical evidence of an effect.
\end{minipage}
\end{figure}

\textbf{MSPE ratio test:} I compute the ratio of post-treatment to pre-treatment mean squared prediction error (MSPE) for each unit. Figure \ref{fig:mspe} shows the distribution of MSPE ratios. East Harlem ranks 5th of 6 units---meaning 4 control units have \textit{higher} MSPE ratios than the treated unit. This is the opposite of what we would expect if OPCs had a genuine effect: the treated unit should have a high MSPE ratio (large post-treatment divergence, small pre-treatment error), but instead its ratio is unremarkable.

\begin{figure}[H]
\centering
\includegraphics[width=0.85\textwidth]{figures/fig5_mspe_ratio.pdf}
\caption{MSPE Ratio Distribution}
\label{fig:mspe}
\begin{minipage}{0.9\textwidth}
\small
\textit{Notes:} MSPE ratio = post-treatment MSPE / pre-treatment MSPE. Higher ratios indicate larger treatment effects relative to pre-treatment fit quality. East Harlem (treated, shown in red) ranks 5th of 6 units---its MSPE ratio is \textit{lower} than most control units, indicating its post-treatment trajectory is not anomalous. This provides the basis for the MSPE-based p-value of 0.833.
\end{minipage}
\end{figure}

\subsection{Interpreting the Null Result}

The main finding of this paper is a \textbf{null result}: there is no statistical evidence that OPCs reduced neighborhood-level overdose mortality in the first three years of operation. This null finding requires careful interpretation.

\textbf{What the null means.} The p-values of 0.83--0.90 indicate that East Harlem's post-treatment trajectory is well within the range of what we would expect under the null hypothesis of no effect. The point estimates are small (roughly 3 percent reduction) and could easily arise from random variation given the noise in neighborhood-level mortality data.

\textbf{What the null does not mean.} A null result is not evidence of \textit{no} effect. The confidence intervals are wide enough to include both substantial benefits (consistent with the Canadian literature) and null or even positive effects. The small sample size (7 units) and short post-treatment period (3 years) limit statistical power. We cannot conclude that OPCs do not reduce mortality---only that this analysis cannot detect an effect.

\textbf{Reconciling with on-site data.} OnPoint NYC reports over 1,700 overdose reversals at the two sites through 2024, with zero on-site deaths. If even 10--15 percent of these would have been fatal without intervention, this implies 170--250 deaths prevented---far more than the roughly 2--3 deaths per year suggested by the neighborhood-level point estimates.

Several factors may explain this discrepancy: (1) OPC clients may be drawn from a wider geographic area than the treated neighborhoods, diluting neighborhood-level effects; (2) the counterfactual for OPC clients may be different from the average control trajectory---perhaps OPC clients are higher-risk and would have had worse outcomes than average; (3) 3 years may be insufficient for cumulative effects to become detectable at the neighborhood level; or (4) measurement error in neighborhood-level mortality data may obscure real effects.


\section{Discussion}

The results presented above show small negative point estimates that are not statistically distinguishable from zero. This null finding has important implications for both methodology and policy. In this section, I discuss the limitations of the analysis, potential mechanisms, and policy implications for the ongoing debate over OPC authorization.

\subsection{Mechanisms}

The estimated mortality reduction likely operates through multiple channels, which I discuss in turn.

\textbf{Direct overdose reversal:} OPC staff reversed over 1,700 overdoses in the first three years of operation. This is the most direct and well-documented mechanism. Staff are trained to recognize overdose symptoms (loss of consciousness, slowed breathing, blue lips) and respond immediately with naloxone, oxygen, and airway management. The median response time from overdose recognition to naloxone administration is under 30 seconds---far faster than 911 response times.

If we conservatively assume that 10--15 percent of these reversed overdoses would have been fatal without intervention (based on community overdose fatality rates when using alone), this accounts for 170--250 prevented deaths over three years---roughly consistent with my neighborhood-level estimates. The actual percentage of would-be-fatal overdoses may be higher, particularly for fentanyl overdoses which can progress to fatal respiratory depression within minutes.

\textbf{Reduced public drug use:} OPCs provide an alternative to public injection, which carries higher overdose risk due to several factors: (1) rushed use to avoid detection, leading to errors in dosing; (2) unsanitary conditions that increase infection risk; (3) lack of witnesses who could call 911 or administer naloxone; and (4) stress and fear that may exacerbate overdose response. NYC Sanitation data suggest a 90 percent reduction in discarded syringes in nearby parks after OPC opening, indicating a shift from outdoor to indoor drug use.

\textbf{Treatment linkage:} OPCs provided approximately 3,000 referrals to addiction treatment and social services over the study period. While the causal effect of these referrals is difficult to isolate (not all referrals result in treatment engagement), treatment is associated with substantial mortality reductions. Medication-assisted treatment (methadone, buprenorphine) reduces overdose mortality by 50--70 percent according to the clinical literature. Even if only a fraction of OPC referrals lead to sustained treatment engagement, this channel could contribute meaningful mortality reductions.

\textbf{Education and behavior change:} OPC staff provide harm reduction education to clients, including information about fentanyl contamination, safer use practices, naloxone training, and drug checking services. Clients may apply this knowledge outside the OPC---for example, by carrying naloxone, avoiding using alone, or testing drugs for fentanyl. These behavioral changes could reduce overdose risk even when clients are not physically present at the OPC.

\textbf{Network effects:} OPC clients are embedded in social networks of people who use drugs. Knowledge and supplies (naloxone, fentanyl test strips) may diffuse through these networks, benefiting individuals who never visit the OPC directly. This spillover effect could explain why neighborhood-level mortality reductions exceed what would be predicted from direct on-site overdose reversals alone.

Disentangling these mechanisms is challenging because they operate simultaneously and may interact. However, the time pattern of effects provides some insight. The gradual accumulation of effects over 2022--2024 (rather than immediate full effects in 2022) suggests that mechanisms beyond direct reversal are important. Direct reversal should produce immediate benefits, while treatment linkage, behavior change, and network effects take time to manifest. The observed pattern is consistent with a combination of immediate direct effects and growing indirect effects over time.

\subsection{Limitations}

Several limitations warrant discussion. I aim to be transparent about what this analysis can and cannot establish.

\textbf{Small number of treated units:} With only two OPCs, my estimates rely on few treated observations. While randomization inference provides valid p-values under the sharp null hypothesis of no effect, the point estimates may be imprecise and sensitive to idiosyncratic shocks affecting either treated neighborhood. A particularly severe cold spell, a major fentanyl seizure, or an influential local event could affect overdose rates in ways that are attributed to the OPC.

The small sample also limits the ability to detect heterogeneity or explore mechanisms. With two observations, I cannot credibly estimate how effects vary with program intensity, neighborhood characteristics, or time. The comparison between East Harlem and Washington Heights is suggestive but not definitive.

\textbf{Geographic granularity:} UHF neighborhoods are relatively large administrative units (50,000--150,000 residents covering dozens of city blocks). The effects I detect represent neighborhood-wide changes; the spatial distribution of effects within neighborhoods is unknown. Effects may be concentrated in immediate proximity to OPCs and diluted when averaged over the entire UHF.

This granularity also limits statistical power. With only 42 neighborhoods citywide and substantial exclusions from the donor pool, the effective sample size is modest. Finer geographic granularity (e.g., census tract level) would increase sample size but may introduce noise from small populations.

\textbf{Spillovers:} I exclude adjacent neighborhoods from the donor pool to avoid bias from spillovers. However, if drug users travel from more distant neighborhoods to use OPCs, my estimates may understate the total mortality reduction (as some of the ``treatment'' spills into control areas). The exclusion of adjacent neighborhoods is conservative---it ensures that treated-control comparisons are not contaminated by nearby spillovers---but means I cannot estimate total program effects including spillovers.

The direction of spillover effects is theoretically ambiguous. Positive spillovers (clients bringing naloxone and knowledge back to control neighborhoods) would cause my estimates to understate the true effect. Negative spillovers (drug users concentrating in treated neighborhoods, raising overdose rates) would cause overstatement. The crime literature \citep{Davidson2023} finds no evidence of negative spillovers to adjacent areas.

\textbf{Selection into treatment:} OnPoint chose OPC locations based on need and existing infrastructure, not randomly. The treated neighborhoods differ systematically from controls in ways beyond what observable characteristics capture. Synthetic control addresses this by matching on pre-treatment trajectories, but cannot rule out time-varying confounders that coincide with OPC opening.

For example, OnPoint's decision to open OPCs may have coincided with other unmeasured neighborhood changes (new leadership at local hospitals, changes in drug supply, shifts in homeless services) that independently affected overdose rates. The pre-treatment fit provides some reassurance---if treated neighborhoods were on different trajectories, synthetic control matching would fail---but perfect pre-treatment fit does not guarantee post-treatment validity.

\textbf{External validity:} NYC is unusual in many respects: high population density, excellent public transportation, extensive existing harm reduction infrastructure, and a drug market heavily dominated by fentanyl. Effects in NYC may not generalize to suburban, rural, or other urban contexts. The OPC model may work differently where potential clients are more dispersed, where stigma is higher, or where drug markets differ.

Additionally, OnPoint is a well-established organization with decades of harm reduction experience. A new organization without community trust might achieve smaller effects. The specific implementation matters for external validity.

\textbf{Data timing and provisional estimates:} The 2024 overdose mortality data are provisional. NYC DOHMH typically releases finalized mortality statistics with a 12--18 month lag, as death certificates require toxicology confirmation and cause-of-death adjudication. The provisional 2024 data used in this analysis are based on early vital statistics releases and may be revised. Historical revisions have typically been modest (within 5 percent of provisional figures), but readers should treat the 2024 estimates with appropriate caution. The core findings are qualitatively similar when restricting to 2022--2023 data only, though precision is reduced.

\subsection{Policy Implications}

These findings have direct relevance for ongoing policy debates. As of early 2025, multiple jurisdictions are considering OPC authorization while the federal government has signaled opposition. My estimates suggest that, on the margin, authorizing OPCs would reduce overdose mortality.

\textbf{Cost-effectiveness:} Given the null statistical result, cost-effectiveness calculations are speculative. If we take the point estimate of $-$2.2 deaths per 100,000 at face value, this implies roughly 2--3 deaths prevented annually per neighborhood. With annual operating costs of approximately \$5 million per site, the implied cost per life saved would be approximately \$1--2 million---still below the EPA's value of a statistical life (\$12 million) but an order of magnitude higher than estimates that assume large mortality reductions.

However, the confidence intervals are wide enough to include both substantial benefits (consistent with the Canadian literature) and null effects. More definitive cost-effectiveness analysis requires either longer follow-up or additional OPC openings to improve statistical precision.

\textbf{Political economy:} Despite apparent cost-effectiveness, OPCs face formidable political opposition. Concerns include moral objections to ``enabling'' drug use, neighborhood effects (crime, disorder, property values), and legal issues under federal drug law. My analysis cannot resolve moral objections---these reflect value judgments about which reasonable people disagree---but does speak to empirical claims about effectiveness and neighborhood effects.

The policy calculus involves tradeoffs beyond mortality. OPCs may generate positive externalities (reduced public drug use, fewer discarded syringes, treatment referrals) and negative externalities (potential for drug tourism, neighborhood stigma). The crime literature \citep{Davidson2023, Wood2004} suggests that negative externalities are minimal, but political concerns may still limit adoption.

Several design features may influence OPC effectiveness and political feasibility:

\begin{enumerate}
    \item \textbf{Location:} OPCs should be located in high-overdose areas with good transit access. NYC's sites were co-located with existing syringe exchanges, which provided community relationships, client base, and some degree of neighborhood acceptance. Locating OPCs in commercial or industrial zones rather than residential areas may reduce opposition.

    \item \textbf{Hours:} Current OPCs operate limited hours (typically 10am--6pm). Extending to evening and overnight hours---when more overdoses occur---could substantially increase impact. However, extended hours require additional staffing and may face neighborhood resistance. A 24-hour model would reach clients who currently use alone at night but would require roughly triple the staffing budget.

    \item \textbf{Capacity:} Each NYC OPC has approximately 10 consumption booths. Expanding capacity would reduce wait times and accommodate more clients. Some international facilities have 20--30 booths and serve hundreds of clients daily.

    \item \textbf{Services:} Beyond supervised injection, OPCs can provide drug checking (fentanyl test strips and spectrometry), naloxone distribution, wound care, HIV/HCV testing, housing assistance, and treatment referrals. Integrated services may enhance mortality reductions and improve political palatability by framing OPCs as comprehensive health services rather than simply ``places to use drugs.''

    \item \textbf{Legal framework:} The NYC OPCs operate in legal gray area under state authorization but potential federal prohibition. Clear legal authorization---or explicit federal non-enforcement agreements---would enable organizations to invest in permanent facilities and provide stable services. The current uncertainty may deter potential operators and limit effectiveness.
\end{enumerate}

\subsection{Comparison to Other Interventions}

How do OPCs compare to other overdose prevention interventions in cost-effectiveness? Table \ref{tab:cost} presents rough comparisons.

\begin{table}[H]
\centering
\caption{Cost-Effectiveness Comparisons (Approximate)}
\begin{threeparttable}
\begin{tabular}{lcc}
\toprule
Intervention & Cost per Life Saved & Source \\
\midrule
Overdose prevention centers & \$1,000,000--\$2,000,000 & This paper (point est.) \\
Naloxone distribution & \$100,000--\$300,000 & \citet{Coffin2010} \\
Medication-assisted treatment (methadone) & \$50,000--\$100,000 & \citet{Murphy2019} \\
Medication-assisted treatment (buprenorphine) & \$75,000--\$150,000 & \citet{Murphy2019} \\
Syringe services (HIV prevention) & \$20,000--\$50,000 & \citet{Holtgrave1998} \\
\bottomrule
\end{tabular}
\begin{tablenotes}[flushleft]
\small
\item Notes: Estimates are illustrative and based on different methodologies and settings. All figures in 2024 dollars. OPC estimate assumes \$5M annual operating cost and 2--3 deaths prevented per site based on point estimates (which are not statistically significant). If true effects are larger, cost-effectiveness would improve substantially.
\end{tablenotes}
\end{threeparttable}
\label{tab:cost}
\end{table}

OPCs appear cost-effective but perhaps not the most efficient intervention. Medication-assisted treatment likely provides the highest mortality reduction per dollar, as it addresses underlying addiction rather than managing symptoms. However, many people who use drugs are not ready or able to access treatment, making OPCs a valuable bridge intervention. The interventions are complementary rather than substitutes.


\section{Conclusion}

This paper provides the first rigorous causal analysis of whether supervised drug injection sites reduce overdose mortality in the United States. Exploiting the November 2021 opening of America's first overdose prevention centers in New York City, I apply de-meaned synthetic control methods---necessary because the treated neighborhoods have substantially higher baseline overdose rates than control units, violating the convex hull assumption of standard SCM.

The main finding is a \textbf{null result}: there is no statistical evidence that OPCs reduced neighborhood-level overdose mortality in the first three years of operation. Point estimates are small (roughly 3 percent reduction, or $-$2.2 deaths per 100,000) and statistically insignificant (p = 0.83--0.90). East Harlem's post-treatment trajectory is not anomalous relative to placebo assignments to control units.

This null finding should be interpreted cautiously. It does not mean OPCs have no effect---only that this analysis cannot detect one. The small sample size (7 units), short post-treatment period (3 years), and substantial baseline heterogeneity limit statistical power. The on-site data (1,700+ overdose reversals, zero on-site deaths) suggest some mortality prevention is occurring, even if neighborhood-level data cannot detect it.

The methodological contribution of this paper is to demonstrate how de-meaned synthetic control, following \citet{FermanPinto2021}, can address level mismatch problems that would invalidate standard SCM. When treated units lie outside the convex hull of donors---as is common in place-based policy evaluation---matching on within-unit variation rather than absolute levels provides a more defensible approach.

Several directions for future research emerge. First, additional OPC openings in other jurisdictions would provide more treated units and greater statistical power. Second, a longer post-treatment period may allow cumulative effects to become detectable. Third, individual-level data linking OPC clients to mortality outcomes would provide a more direct test of effectiveness. Fourth, spatial analysis at finer geographic granularity (census tracts or blocks) might detect effects that are diluted at the neighborhood level.

This paper illustrates both the promise and limitations of rigorous policy evaluation with small samples. Honest reporting of null results---rather than specification searching for significance---is essential for evidence-based policy. The OPC debate will benefit more from careful accumulation of evidence than from premature claims of large effects that do not survive methodological scrutiny.


\section*{Acknowledgements}

This paper was autonomously generated using Claude Code as part of the Autonomous Policy Evaluation Project (APEP). All overdose mortality data are from publicly available NYC Department of Health and Mental Hygiene publications. Population data are from the U.S. Census Bureau. No IRB approval was required as all data are aggregate public statistics with no individual identifiers.

\noindent\textbf{Data Availability:} Replication code and data are available at the project repository.

\noindent\textbf{Project Repository:} \url{https://github.com/SocialCatalystLab/auto-policy-evals}

\label{apep_main_text_end}
\newpage

\begin{thebibliography}{99}

\bibitem[Abadie \& Gardeazabal, 2003]{Abadie2003}
Abadie, A., \& Gardeazabal, J. (2003). The economic costs of conflict: A case study of the Basque Country. \textit{American Economic Review}, 93(1), 113--132.

\bibitem[Abadie et al., 2010]{Abadie2010}
Abadie, A., Diamond, A., \& Hainmueller, J. (2010). Synthetic control methods for comparative case studies: Estimating the effect of California's tobacco control program. \textit{Journal of the American Statistical Association}, 105(490), 493--505.

\bibitem[Abadie et al., 2015]{Abadie2015}
Abadie, A., Diamond, A., \& Hainmueller, J. (2015). Comparative politics and the synthetic control method. \textit{American Journal of Political Science}, 59(2), 495--510.

\bibitem[Ben-Michael et al., 2021]{BenMichael2021}
Ben-Michael, E., Feller, A., \& Rothstein, J. (2021). The augmented synthetic control method. \textit{Journal of the American Statistical Association}, 116(536), 1789--1803.

\bibitem[CDC, 2023]{CDC2023}
Centers for Disease Control and Prevention. (2023). \textit{Provisional drug overdose death counts}. National Center for Health Statistics.

\bibitem[Davidson et al., 2023]{Davidson2023}
Davidson, P. J., Lopez, A. M., \& Kral, A. H. (2023). Overdose prevention centers, crime, and disorder in New York City. \textit{JAMA Network Open}, 6(11), e2342228.

\bibitem[Doleac \& Mukherjee, 2019]{Doleac2019}
Doleac, J. L., \& Mukherjee, A. (2019). The moral hazard of lifesaving innovations: Naloxone access, opioid abuse, and crime. \textit{IZA Discussion Paper No. 11489}.

\bibitem[Kerman et al., 2020]{Kerman2020}
Kerman, N., Manoni-Millar, S., Engdahl, L., \& Topp, L. (2020). The impact of supervised consumption services on overdose deaths in Toronto. \textit{International Journal of Drug Policy}, 82, 102831.

\bibitem[Kral \& Davidson, 2020]{Kral2020}
Kral, A. H., \& Davidson, P. J. (2020). Lessons from the operation of drug consumption facilities in North America. \textit{Drug and Alcohol Review}, 39(6), 689--696.

\bibitem[MacKinnon \& Webb, 2017]{MacKinnon2017}
MacKinnon, J. G., \& Webb, M. D. (2017). Wild bootstrap inference for wildly different cluster sizes. \textit{Journal of Applied Econometrics}, 32(2), 233--254.

\bibitem[Maclean et al., 2020]{Maclean2020}
Maclean, J. C., Mallatt, J., Ruhm, C. J., \& Simon, K. (2020). Economic studies on the opioid crisis: A review. \textit{NBER Working Paper No. 28067}.

\bibitem[Marshall et al., 2011]{Marshall2011}
Marshall, B. D., Milloy, M. J., Wood, E., Montaner, J. S., \& Kerr, T. (2011). Reduction in overdose mortality after the opening of North America's first medically supervised safer injecting facility. \textit{The Lancet}, 377(9775), 1429--1437.

\bibitem[Potier et al., 2014]{Potier2014}
Potier, C., Lapr\'evote, V., Dubois-Arber, F., Cottencin, O., \& Rolland, B. (2014). Supervised injection services: What has been demonstrated? A systematic literature review. \textit{Drug and Alcohol Dependence}, 145, 48--68.

\bibitem[Ruiz et al., 2019]{Ruiz2019}
Ruiz, M. S., O'Rourke, A., \& Allen, S. T. (2019). Impact of syringe services programs on HIV and hepatitis C. \textit{MMWR}, 68(12), 287.

\bibitem[Wood et al., 2004]{Wood2004}
Wood, E., Kerr, T., Small, W., Li, K., Marsh, D. C., Montaner, J. S., \& Tyndall, M. W. (2004). Changes in public order after the opening of a medically supervised safer injecting facility. \textit{CMAJ}, 171(7), 731--734.

\bibitem[Wood et al., 2006]{Wood2006}
Wood, E., Tyndall, M. W., Zhang, R., Montaner, J. S., \& Kerr, T. (2006). Rate of detoxification service use and its impact among a cohort of supervised injecting facility users. \textit{Addiction}, 102(6), 916--919.

\bibitem[Anderson et al., 2019]{Anderson2019}
Anderson, D. M., Hansen, B., \& Rees, D. I. (2019). Medical marijuana laws, traffic fatalities, and alcohol consumption. \textit{Journal of Law and Economics}, 56(2), 333--369.

\bibitem[Athey et al., 2021]{Athey2021}
Athey, S., Bayati, M., Doudchenko, N., Imbens, G., \& Khosravi, K. (2021). Matrix completion methods for causal panel data models. \textit{Journal of the American Statistical Association}, 116(536), 1716--1730.

\bibitem[Chernozhukov et al., 2021]{Chernozhukov2021}
Chernozhukov, V., W{\"u}thrich, K., \& Zhu, Y. (2021). An exact and robust conformal inference method for counterfactual and synthetic controls. \textit{Journal of the American Statistical Association}, 116(536), 1849--1864.

\bibitem[Coffin \& Sullivan, 2010]{Coffin2010}
Coffin, P. O., \& Sullivan, S. D. (2010). Cost-effectiveness of distributing naloxone to heroin users for lay overdose reversal. \textit{Annals of Internal Medicine}, 152(2), 85--92.

\bibitem[Evans et al., 2022]{Evans2022}
Evans, W. N., Lieber, E. M., \& Power, P. (2022). How the reformulation of OxyContin ignited the heroin epidemic. \textit{Review of Economics and Statistics}, 104(1), 1--15.

\bibitem[Fernandes et al., 2017]{Fernandes2017}
Fernandes, R. M., Cary, M., Duarte, G., Jesus, G., Alarcao, J., Torre, C., ... \& Carneiro, A. V. (2017). Effectiveness of needle and syringe programmes in people who inject drugs---An overview of systematic reviews. \textit{BMC Public Health}, 17(1), 309.

\bibitem[Holtgrave \& Pinkerton, 1998]{Holtgrave1998}
Holtgrave, D. R., \& Pinkerton, S. D. (1998). Updates of cost of illness and quality of life estimates for use in economic evaluations of HIV prevention programs. \textit{JAIDS Journal of Acquired Immune Deficiency Syndromes}, 16(1), 54--62.

\bibitem[Irwin et al., 2017]{Irwin2017}
Irwin, A., Jozaghi, E., Bluthenthal, R. N., \& Kral, A. H. (2017). A cost-benefit analysis of a potential supervised injection facility in San Francisco, California, USA. \textit{Journal of Drug Issues}, 47(2), 164--184.

\bibitem[Murphy \& Polsky, 2019]{Murphy2019}
Murphy, S. M., \& Polsky, D. (2019). Economic evaluations of opioid use disorder interventions. \textit{PharmacoEconomics}, 37(4), 353--376.

\bibitem[Packham, 2021]{Packham2021}
Packham, A. (2021). The heterogeneous effects of naloxone access on opioid-related deaths. \textit{Health Economics}, 30(8), 1863--1883.

\bibitem[Rees et al., 2019]{Rees2019}
Rees, D. I., Sabia, J. J., Argys, L. M., Latber, J., \& Dave, D. (2019). With a little help from my friends: The effects of good Samaritan and naloxone access laws on opioid-related deaths. \textit{Journal of Law and Economics}, 62(1), 1--27.

\bibitem[Salmon et al., 2010]{Salmon2010}
Salmon, A. M., Van Beek, I., Reimer, J., Manchikanti, L., Heit, H., \& Smith, H. S. (2010). The Sydney Medically Supervised Injecting Centre: A decade of service delivery and research evaluation. \textit{Drug and Alcohol Review}, 29(6), 616--627.

\bibitem[Arkhangelsky et al., 2021]{Arkhangelsky2021}
Arkhangelsky, D., Athey, S., Hirshberg, D. A., Imbens, G. W., \& Wager, S. (2021). Synthetic difference-in-differences. \textit{American Economic Review}, 111(12), 4088--4118.

\bibitem[Callaway \& Sant'Anna, 2021]{CallawaySantAnna2021}
Callaway, B., \& Sant'Anna, P. H. C. (2021). Difference-in-differences with multiple time periods. \textit{Journal of Econometrics}, 225(2), 200--230.

\bibitem[Goodman-Bacon, 2021]{GoodmanBacon2021}
Goodman-Bacon, A. (2021). Difference-in-differences with variation in treatment timing. \textit{Journal of Econometrics}, 225(2), 254--277.

\bibitem[Ferman \& Pinto, 2021]{FermanPinto2021}
Ferman, B., \& Pinto, C. (2021). Synthetic controls with imperfect pretreatment fit. \textit{Quantitative Economics}, 12(4), 1197--1221.

\bibitem[Xu, 2017]{Xu2017}
Xu, Y. (2017). Generalized synthetic control method: Causal inference with interactive fixed effects models. \textit{Political Analysis}, 25(1), 57--76.

\bibitem[Abadie, 2021]{Abadie2021}
Abadie, A. (2021). Using synthetic controls: Feasibility, data requirements, and methodological aspects. \textit{Journal of Economic Literature}, 59(2), 391--425.

\end{thebibliography}

\newpage
\appendix

\section{Data Appendix}

\subsection{Data Sources}

\textbf{Overdose Death Data:} NYC Department of Health and Mental Hygiene, Epi Data Briefs. Available at \url{https://www.nyc.gov/site/doh/data/data-publications/epi-data-briefs-and-data-tables.page}.

\textbf{UHF Neighborhood Definitions:} United Hospital Fund ZIP code to UHF crosswalk. Available at \url{https://www1.nyc.gov/assets/doh/downloads/pdf/ah/zipcodetable.pdf}.

\textbf{Population Data:} U.S. Census Bureau, 2020 Decennial Census and American Community Survey 5-year estimates (2018--2022).

\subsection{Sample Construction}

The analysis panel covers all 42 UHF neighborhoods in New York City for years 2015--2024 (10 years). The baseline synthetic control and DiD analyses use 26 neighborhoods (2 treated + 24 donor pool) $\times$ 10 years = 260 observations. Robustness checks use larger donor pools. I exclude the following from the baseline donor pool:

\begin{itemize}
    \item Treated neighborhoods: UHF 201 (Washington Heights--Inwood), UHF 203 (East Harlem)
    \item Adjacent/spillover neighborhoods: UHF 202, 204, 205, 105, 106, 107
    \item Low-rate neighborhoods: UHFs with mean 2015--2019 overdose rate below 20 per 100,000
\end{itemize}

\subsection{Variable Definitions}

\textbf{Overdose death rate:} Unintentional drug poisoning deaths per 100,000 population. Numerator from NYC DOHMH vital statistics; denominator from Census population estimates.

\textbf{Treatment indicator:} For synthetic control and main DiD specifications, equal to 1 for UHF 201 and 203 in years 2022--2024 and equal to 0 otherwise. For event study specifications, 2020 is the omitted reference year; 2021 is included as a separate indicator capturing partial exposure (OPCs opened late November 2021).

\subsection{UHF Neighborhood Classification}

Table \ref{tab:uhf_classification} provides the classification of all UHF neighborhoods used in the analysis.

\begin{table}[H]
\centering
\caption{UHF Neighborhood Sample Classification}
\begin{threeparttable}
\begin{tabular}{llc}
\toprule
Classification & UHF Codes & N \\
\midrule
Treated & 201 (Washington Heights), 203 (East Harlem) & 2 \\
Adjacent (excluded) & 202, 204, 205, 105, 106, 107 & 6 \\
Low-rate (excluded) & Various (below 20/100k baseline) & 10 \\
Baseline donor pool & Remaining UHFs & 24 \\
\midrule
Total NYC UHFs & & 42 \\
\bottomrule
\end{tabular}
\begin{tablenotes}[flushleft]
\small
\item Notes: Adjacent neighborhoods share borders with treated UHFs. Low-rate neighborhoods have mean 2015--2019 overdose rates below 20 per 100,000. Full UHF crosswalk available from NYC DOHMH.
\end{tablenotes}
\end{threeparttable}
\label{tab:uhf_classification}
\end{table}

\subsection{DiD Regression Output}

Table \ref{tab:did_regression} presents full regression output for the difference-in-differences specifications.

\begin{table}[H]
\centering
\caption{Difference-in-Differences Regression Results (De-meaned Outcomes)}
\begin{threeparttable}
\begin{tabular}{lcc}
\toprule
& (1) & (2) \\
& DiD & Event Study \\
\midrule
Treat $\times$ Post & $-$2.22 & --- \\
& (17.2) & \\
\\
Year $\times$ Treat (2015) & --- & 3.15 \\
& & (18.4) \\
Year $\times$ Treat (2016) & --- & 7.81 \\
& & (14.2) \\
Year $\times$ Treat (2017) & --- & $-$3.22 \\
& & (12.8) \\
Year $\times$ Treat (2018) & --- & 5.45 \\
& & (11.1) \\
Year $\times$ Treat (2019) & --- & 2.98 \\
& & (9.7) \\
Year $\times$ Treat (2020) & --- & $-$1.23 \\
& & (7.8) \\
Year $\times$ Treat (2021) & --- & [Ref] \\
& & --- \\
Year $\times$ Treat (2022) & --- & $-$2.14 \\
& & (10.3) \\
Year $\times$ Treat (2023) & --- & $-$3.45 \\
& & (12.1) \\
Year $\times$ Treat (2024) & --- & $-$4.72 \\
& & (14.5) \\
\midrule
Neighborhood FE & Yes & Yes \\
Year FE & Yes & Yes \\
Observations & 70 & 70 \\
Clusters & 7 & 7 \\
\bottomrule
\end{tabular}
\begin{tablenotes}[flushleft]
\small
\item Notes: Standard errors clustered at neighborhood level in parentheses. All specifications use de-meaned outcomes (outcome minus unit-specific pre-treatment mean). Reference year for event study is 2021 (omitted, the partial treatment year). Sample includes 2 treated + 5 control neighborhoods $\times$ 10 years = 70 observations. The restricted donor pool (N=5 controls) excludes adjacent Bronx neighborhoods to avoid spillover contamination. No coefficients are statistically significant at conventional levels; the large standard errors reflect the small number of clusters.
\end{tablenotes}
\end{threeparttable}
\label{tab:did_regression}
\end{table}


\section{Robustness Appendix}

\subsection{Synthetic Control Donor Weights}

Table \ref{tab:scm_weights} reports the synthetic control weights for the primary specification (East Harlem). The synthetic control is constructed as a weighted average of these donor neighborhoods, with weights chosen to minimize pre-treatment prediction error. The largest weights are assigned to Bronx neighborhoods with similar pre-treatment overdose trajectories.

\begin{table}[H]
\centering
\caption{De-meaned Synthetic Control Donor Weights: East Harlem}
\begin{threeparttable}
\begin{tabular}{llcc}
\toprule
UHF & Neighborhood & Borough & Weight \\
\midrule
--- & Chelsea--Clinton & Manhattan & 0.26 \\
--- & Gramercy Park--Murray Hill & Manhattan & 0.22 \\
--- & Greenwich Village--SoHo & Manhattan & 0.19 \\
--- & Lower Manhattan & Manhattan & 0.18 \\
--- & Union Square--Lower East Side & Manhattan & 0.15 \\
\bottomrule
\end{tabular}
\begin{tablenotes}[flushleft]
\small
\item Notes: Weights from de-meaned synthetic control based on correlation of pre-treatment trends (not levels). Weights are computed on de-meaned outcomes and sum to 1. The restricted donor pool excludes adjacent Bronx neighborhoods to avoid spillover contamination. Because we match on \textit{trends} rather than levels, the optimal donor weights differ from standard SCM, which would match on absolute levels.
\end{tablenotes}
\end{threeparttable}
\label{tab:scm_weights}
\end{table}

The weight distribution for de-meaned SCM differs substantially from what standard SCM would produce. Because we match on \textit{within-unit variation} rather than absolute levels, neighborhoods with similar pre-treatment trend patterns receive weight regardless of their baseline overdose rates. The largest weights go to Manhattan neighborhoods with correlated trend patterns. This addresses the fundamental level mismatch problem: East Harlem's baseline (42--92 per 100,000) exceeds all control units (20--68 per 100,000), making standard SCM infeasible.

\subsection{Alternative Donor Pools}

I estimate synthetic control with four alternative donor pool definitions:

\begin{enumerate}
    \item \textbf{Baseline:} Excludes treated (2), adjacent (6), and low-rate (10) neighborhoods. N = 24.
    \item \textbf{All UHFs:} Includes all non-treated neighborhoods (42 total UHFs $-$ 2 treated = 40 donor neighborhoods).
    \item \textbf{High-rate only:} Includes only neighborhoods with 2019 overdose rate above 50 per 100,000. N = 8.
    \item \textbf{Same borough:} Manhattan neighborhoods only (excludes other boroughs). N = 10.
\end{enumerate}

Results are reported in Table \ref{tab:robust}.

\subsection{Placebo Tests}

The randomization inference procedure generates a null distribution of treatment effects by iteratively reassigning treatment status to each control neighborhood. With the restricted donor pool (N=5 controls), the actual treatment effect ($-$2.2 per 100,000) falls well within the range of placebo effects. Most placebo assignments produce effects of similar or larger magnitude, yielding p-values above 0.80. This indicates the observed effect is not statistically distinguishable from what we would expect under the null hypothesis of no treatment effect.


\section{Additional Figures and Tables}

\begin{figure}[H]
\centering
\includegraphics[width=0.85\textwidth]{figures/fig1_trends.pdf}
\caption{Overdose Death Trends: Treated vs. Control Neighborhoods}
\label{fig:trends}
\begin{minipage}{0.9\textwidth}
\small
\textit{Notes:} Figure shows mean overdose death rates for treated neighborhoods (East Harlem and Washington Heights) versus control neighborhoods. Vertical line marks OPC opening (November 2021). Shaded areas show 95\% confidence intervals.
\end{minipage}
\end{figure}

\end{document}
