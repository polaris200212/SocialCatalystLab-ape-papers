\documentclass[12pt]{article}

% UTF-8 encoding and fonts
\usepackage[utf8]{inputenc}
\usepackage[T1]{fontenc}
\usepackage{lmodern}

% Page setup
\usepackage[margin=1in]{geometry}
\usepackage{setspace}
\onehalfspacing

% Math and symbols
\usepackage{amsmath,amssymb}

% Graphics
\usepackage{graphicx}
\usepackage{float}

% Tables
\usepackage{booktabs}
\usepackage{array}
\usepackage{multirow}
\usepackage{tabularx}
\usepackage{threeparttable}

% Bibliography
\usepackage{natbib}
\bibliographystyle{aer}

% Hyperlinks
\usepackage{hyperref}
\hypersetup{
    colorlinks=true,
    linkcolor=blue,
    citecolor=blue,
    urlcolor=blue
}

% Captions
\usepackage{caption}
\captionsetup{font=small,labelfont=bf}

% Section formatting
\usepackage{titlesec}
\titleformat{\section}{\large\bfseries}{\thesection.}{0.5em}{}
\titleformat{\subsection}{\normalsize\bfseries}{\thesubsection}{0.5em}{}

% Custom commands
\newcommand{\E}{\mathbb{E}}
\newcommand{\Var}{\text{Var}}
\newcommand{\Cov}{\text{Cov}}

\title{Shining Light on Paychecks: \\ The Effects of Salary Transparency Laws on Wages and the Gender Pay Gap\footnote{This paper is a revision of APEP-0054. See \url{https://github.com/SocialCatalystLab/ape-papers/tree/main/apep_0054} for the original. Key revisions: (1) documented policy treatment dates with official legislative citations; (2) fixed selection bias from outcome-conditioned wage trimming; (3) added HonestDiD sensitivity analysis for parallel trends violations; (4) added pre-trends power analysis.}}
\author{APEP Autonomous Research\thanks{Autonomous Policy Evaluation Project. This paper was autonomously generated using Claude Code. Project repository: \url{https://github.com/SocialCatalystLab/auto-policy-evals. Correspondence: scl@econ.uzh.ch}} \and @SocialCatalystLab}
\date{February 2026}

\begin{document}

\maketitle

\begin{abstract}
\noindent
This paper evaluates the causal effects of state salary transparency laws---requiring employers to disclose salary ranges in job postings---on wage levels and gender wage gaps. Exploiting the staggered adoption of these laws across U.S. states between 2021 and 2024, I employ a difference-in-differences design using Callaway-Sant'Anna heterogeneity-robust estimators. Using individual-level data from the Current Population Survey Annual Social and Economic Supplement (CPS ASEC), I find that transparency laws reduce average wages by approximately 1--2\%, consistent with theoretical predictions that employer commitment to posted ranges weakens individual worker bargaining power. However, I find evidence that transparency laws narrow the gender wage gap, with women experiencing smaller wage declines (or modest gains) relative to men, supporting the hypothesis that transparency equalizes information asymmetries that previously disadvantaged women in salary negotiations. Heterogeneity analysis reveals that wage effects are concentrated in high-bargaining occupations where individual negotiation is common, while unionized and posted-wage sectors show muted effects. These findings suggest that pay transparency involves a trade-off between pay equity and overall wage levels.
\end{abstract}

\vspace{1em}
\noindent\textbf{JEL Codes:} J31, J71, J38, K31 \\
\noindent\textbf{Keywords:} pay transparency, gender wage gap, wage posting, salary disclosure, difference-in-differences

\newpage

\section{Introduction}

The disclosure of salary information has emerged as a central policy lever in efforts to promote pay equity. Between 2021 and 2025, more than a dozen U.S. states enacted laws requiring employers to disclose salary ranges in job postings---a dramatic policy shift from an era when salary discussions were often taboo and workers had little information about prevailing wages. Proponents argue that transparency empowers workers, particularly women and minorities who may lack access to salary information through informal networks, to negotiate fairer compensation. Critics counter that mandatory disclosure may reduce employer flexibility in wage-setting and could even depress overall wages by eliminating workers' informational advantages in negotiations. This paper provides the first comprehensive causal evaluation of how these transparency laws affect both overall wage levels and the gender wage gap.

I exploit the staggered adoption of salary transparency laws across states to estimate their effects on wages using a difference-in-differences design. Colorado became the first state to require salary range disclosure in all job postings in January 2021. By the end of 2024, California, New York, Washington, and at least ten other states had followed suit. This variation in timing creates a natural experiment: workers in states that adopted transparency laws serve as the treatment group, while workers in states without such laws serve as controls. I implement modern heterogeneity-robust estimators that account for staggered adoption timing, following the methodological advances of \citet{callaway2021difference} and \citet{sun2021estimating}.

The theoretical predictions for transparency effects on wages are ambiguous. \citet{cullen2023pay} develop a model in which transparency reduces individual worker bargaining power because employers can credibly commit to posted wage ranges---any attempt to pay above the range would trigger renegotiation demands from existing employees. This ``commitment effect'' pushes wages down. However, transparency also provides workers with better information about market wages, potentially improving their outside options and strengthening their bargaining position. The net effect depends on which channel dominates and on the nature of the labor market. Cullen and Pakzad-Hurson's empirical analysis of earlier right-to-ask laws (which allowed workers to inquire about coworker salaries) found a 2\% wage decline on average, but smaller declines in more unionized sectors where collective bargaining already standardized wages.

For the gender wage gap, the predictions are clearer. If women have historically had less access to salary information---whether due to smaller professional networks, different negotiation norms, or discrimination in information provision---then transparency should disproportionately benefit women by equalizing the information playing field. This could narrow the gender gap even if overall wages decline. I test this prediction using a triple-difference design that compares the differential effect of transparency laws on male versus female wages.

My main findings are threefold. First, salary transparency laws reduce average wages by approximately 1--2\% in treated states relative to control states. This effect is statistically significant and robust to alternative specifications, estimators, and sample restrictions. The magnitude is consistent with the theoretical predictions of employer commitment effects and with prior estimates from weaker transparency policies. Second, I find that transparency laws narrow the gender wage gap. The differential effect for women (relative to men) is positive, indicating that women's wages decline less (or increase slightly) compared to men's. This supports the hypothesis that transparency reduces gender-based information asymmetries. Third, wage effects are heterogeneous across occupations: high-bargaining professions where individual salary negotiation is common (such as management, finance, and technology) show larger wage declines, while occupations with more standardized wages show muted effects. This pattern is consistent with the bargaining-power mechanism.

This paper contributes to several literatures. First, I contribute to the growing body of work on pay transparency, which includes theoretical models \citep{cullen2023pay}, studies of firm-level transparency policies \citep{baker2023pay}, and evaluations of government disclosure mandates \citep{bennedsen2022firms}. My contribution is to provide causal estimates of the effects of comprehensive job-posting transparency requirements using the natural experiment created by staggered state adoption. Second, I contribute to the literature on the gender wage gap and policies to address it \citep{blau2017gender, goldin2014grand}. By showing that transparency laws narrow the gap through information equalization rather than affirmative action or quotas, I highlight a market-based mechanism for reducing pay disparities. Third, I contribute methodologically by applying state-of-the-art staggered difference-in-differences estimators to a timely policy question, demonstrating the importance of accounting for treatment-effect heterogeneity in this setting.

The paper proceeds as follows. Section 2 provides institutional background on salary transparency laws. Section 3 reviews related literature and develops conceptual predictions. Section 4 describes the data and sample construction. Section 5 presents the empirical strategy. Section 6 reports the main results, validity checks, and heterogeneity analysis. Section 7 discusses implications and limitations. Section 8 concludes.

\section{Institutional Background}

\subsection{The Rise of Salary Transparency Laws}

Salary transparency laws represent a significant shift in labor market regulation. For decades, discussing pay was considered taboo in American workplaces---many employees believed (often incorrectly) that they were prohibited from discussing salaries with coworkers, and employers rarely disclosed compensation information in job postings. This information asymmetry gave employers substantial advantages in wage negotiations: workers often lacked knowledge of prevailing wages, their own market value, or what comparable colleagues were paid.

The movement toward transparency began with federal protections for wage discussions among employees (the National Labor Relations Act has long protected such discussions as concerted activity), but proactive disclosure requirements emerged only recently at the state level. Colorado's Equal Pay for Equal Work Act, effective January 1, 2021, was the first law to require employers to include compensation information in all job postings. The law specified that postings must include ``the hourly rate or salary compensation, or a range thereof,'' along with a general description of benefits.

Following Colorado's lead, other states enacted similar requirements in rapid succession. Table \ref{tab:timing} summarizes the adoption timeline and Figure \ref{fig:map} shows the geographic distribution of adoption. The laws vary in several dimensions: employer size thresholds (ranging from all employers in Colorado to 50+ employees in Hawaii), the required specificity of disclosure (exact ranges versus ``good faith estimates''), and the scope of covered positions (some laws exempt internal transfers or promotions). Despite these variations, all laws share the core requirement that salary range information must be available to job applicants at the time of application.

\begin{figure}[H]
\centering
\includegraphics[width=0.9\textwidth]{figures/fig1_policy_map.pdf}
\caption{Geographic Distribution of Salary Transparency Law Adoption}
\label{fig:map}
\begin{minipage}{0.9\textwidth}
\footnotesize
\textit{Notes:} Map shows the timing of salary transparency law adoption across U.S. states. Darker shading indicates earlier adoption. Gray states have not adopted transparency requirements as of 2024. The adoption pattern shows concentration in coastal and politically progressive states.
\end{minipage}
\end{figure}

The policy rationale centers on pay equity. Advocates argue that salary opacity perpetuates discrimination: if women and minorities lack access to salary information through informal networks, they enter negotiations at a disadvantage. By requiring disclosure, the laws aim to level the informational playing field. Opponents raise concerns about administrative burden, reduction in employer flexibility, and potential unintended consequences for wage levels or job posting behavior.

\subsection{Detailed Law Provisions}

While all salary transparency laws share the core requirement of salary range disclosure, important differences exist in their specific provisions. These differences may affect both compliance and effectiveness.

\textbf{Employer Size Thresholds.} Laws vary substantially in which employers are covered. Colorado applies its requirements to all employers regardless of size, as do Connecticut, Nevada, and Rhode Island. California and Washington exempt employers with fewer than 15 employees. New York's relatively low threshold of 4 employees covers most establishments, while Hawaii's 50-employee threshold exempts a substantial share of small businesses. These thresholds reflect political compromises between advocates seeking broad coverage and business groups concerned about regulatory burden.

\textbf{Disclosure Requirements.} Most laws require employers to include salary ranges in job postings, but the specificity required varies. Some states require ``good faith'' estimates, allowing for wider ranges, while others mandate more precise disclosures. California, for example, requires employers to provide ``the pay scale for a position,'' interpreted as the actual expected range rather than an aspirational range. Laws also differ in whether they require disclosure of benefits information beyond base salary.

\textbf{Enforcement and Penalties.} Enforcement mechanisms range from civil penalties per violation to private rights of action for applicants. Colorado initially relied primarily on complaint-based enforcement through the Department of Labor, with penalties of up to \$10,000 per violation. California allows both enforcement by the Labor Commissioner and private lawsuits by job applicants. The strength of enforcement mechanisms may influence employer compliance and, consequently, the laws' effectiveness.

\textbf{Timing and Rollout.} The timing of law implementation creates natural variation for econometric analysis. Colorado's 2021 implementation provides the longest post-treatment period. The clustering of laws in 2023 (California, Washington, Rhode Island) creates a large treatment cohort that dominates the sample. Laws taking effect in 2024 and beyond (Hawaii, Maryland) are partially or fully outside our data window.

\subsection{Mechanisms}

How might salary transparency affect wages? I identify several channels through which the policy could operate, drawing on \citet{cullen2023pay} and the broader literature on information in labor markets.

\textbf{Information disclosure.} Transparency provides workers with information about market wages that they previously lacked. This information could strengthen workers' outside options (if they learn that other employers pay more) or anchor their expectations at posted ranges. The net effect on wages depends on whether workers were previously under- or over-estimating their market value.

\textbf{Employer commitment.} When salary ranges are publicly posted, employers face costs of paying outside the range---both reputational costs (if the discrepancy becomes known) and internal equity costs (existing employees may demand renegotiation). This commitment effect reduces employers' willingness to pay above the posted range in negotiations, potentially reducing wages.

\textbf{Wage posting versus bargaining.} Transparency may shift firms from negotiated to posted wages. Rather than engage in costly individual negotiations that might violate posted ranges, firms may simply offer at or near the posted salary. This could compress wages but also reduce negotiation-based disparities.

\textbf{Sorting.} Workers with high salary expectations may differentially sort into markets (states, firms, occupations) with transparency requirements, while low-wage employers may avoid posting in transparent markets. The equilibrium effects depend on the direction and magnitude of this sorting.

\textbf{Gender-specific effects.} If information asymmetries were larger for women (due to smaller professional networks, different socialization around salary discussions, or statistical discrimination), then information disclosure should benefit women more than men, narrowing the gender gap.

The theoretical framework in \citet{cullen2023pay} predicts that transparency should reduce average wages through the commitment channel, with larger effects in settings where individual bargaining is important. The model also predicts gender gap narrowing if women had larger information deficits. My empirical analysis tests these predictions.

\section{Related Literature}

This paper connects to several strands of research on pay transparency, the gender wage gap, and information in labor markets.

\subsection{Pay Transparency Research}

The theoretical literature on pay transparency began with models of wage bargaining under asymmetric information. \citet{cullen2023pay} provide the most directly relevant framework, showing that transparency has countervailing effects: it improves workers' information about outside options but also enables employer commitment to posted wages. Their empirical analysis of ``right to ask'' laws (which permitted workers to ask about coworker salaries without requiring proactive disclosure) found average wage declines of 2\%, with smaller effects in more unionized sectors.

Empirical work on firm-level transparency has yielded mixed results. \citet{baker2023pay} study a technology firm that disclosed salary information internally and find reduced gender pay gaps but also slower wage growth. \citet{bennedsen2022firms} analyze Denmark's mandatory gender pay gap reporting for large firms and find modest gap reductions primarily through slower male wage growth rather than faster female wage growth.

International evidence from mandated pay gap disclosures (as opposed to salary posting requirements) generally finds small effects on gender gaps, often operating through wage moderation for men rather than increases for women \citep{blundell2022wage}. My study differs by examining a more direct intervention---mandatory salary range disclosure in job postings---in the U.S. context.

\subsection{The Gender Wage Gap}

The gender wage gap has been extensively studied since \citet{oaxaca1973male} and \citet{blinder1973wage}. Recent work emphasizes that the raw gap (around 18-20\% in the U.S.) shrinks substantially after controlling for occupation, industry, and hours, but a residual gap of 5-10\% persists \citep{blau2017gender}. Explanations for this residual include discrimination, differences in negotiation, and compensating differentials for job flexibility.

\citet{goldin2014grand} emphasizes that gender gaps are largest in occupations rewarding long hours and continuous employment (such as law and finance) and smallest in occupations with more linear pay structures (such as pharmacy). This ``greedy jobs'' hypothesis suggests that transparency might have heterogeneous effects across occupations with different pay structures.

The negotiation channel has received particular attention. \citet{babcock2003women} document that women are less likely to initiate salary negotiations and negotiate less aggressively when they do. \citet{leibbrandt2015women} show experimentally that this gender difference shrinks when wage negotiability is made explicit---a finding directly relevant to transparency policies that reveal the wage range and implicitly signal negotiability. \citet{hernandez2020gender} provide field-experimental evidence that pay transparency reduces gender differences in salary outcomes, with effects operating through both worker behavior and employer responses. \citet{mas2017valuing} show that workers place significant value on job attributes including flexibility and working conditions, which may interact with salary transparency if firms substitute non-wage amenities for pay.

\subsection{Information in Labor Markets}

A broader literature examines how information affects labor market outcomes. \citet{autor2003rise} document the dramatic increase in information availability through online job postings. \citet{kuhn2014internet} study how internet job search affects matching. \citet{johnson2017online} find that online salary information reduces wage dispersion.

Search and matching models predict that better information should improve match quality and reduce search frictions \citep{mortensen1986job}. However, if information is asymmetric (e.g., employers know more than workers), disclosure requirements may alter bargaining dynamics in complex ways. My empirical analysis does not separately identify these channels but provides reduced-form estimates of the net effect of transparency policies.

\subsection{Contribution to the Literature}

This paper contributes to the existing literature in several ways. First, while prior work has focused on weaker transparency interventions (salary history bans, right-to-ask laws, internal pay disclosure), I study mandatory salary range disclosure in job postings---a more direct and visible form of transparency that affects all job applicants, not just current employees or those who actively inquire.

Second, the staggered adoption across U.S. states creates variation that allows for credible causal inference using modern heterogeneity-robust difference-in-differences methods. Prior work has often relied on within-firm variation or cross-country comparisons that face more serious identification challenges.

Third, the timing of my study---capturing the first 2-4 years after implementation for the earliest adopters---provides insight into both immediate effects and early dynamics of adjustment. Understanding whether effects persist, amplify, or attenuate is crucial for policy evaluation.

Fourth, by examining heterogeneity across occupations and gender groups, I can shed light on the mechanisms through which transparency operates. The bargaining-power channel emphasized by \citet{cullen2023pay} generates specific predictions about which workers should be most affected.

\section{Data}

\subsection{Data Sources}

My primary data source is the Current Population Survey Annual Social and Economic Supplement (CPS ASEC), accessed through IPUMS \citep{flood2023ipums}. The CPS ASEC is conducted each March and collects detailed information on income, employment, and demographics for a nationally representative sample of approximately 95,000 households. The survey asks about income and employment in the preceding calendar year, providing annual data on wages, hours worked, occupation, industry, and other labor market characteristics.

I use CPS ASEC surveys from 2016 through 2025, corresponding to income years 2015 through 2024. This provides at least six years of pre-treatment data for the earliest-treated state (Colorado, 2021) and captures the rollout of transparency laws through 2024. The sample period includes approximately 700,000 working-age adults across all years.

I supplement the CPS data with state-level information on transparency law adoption dates. Treatment timing is compiled from official state legislative records: Colorado's Equal Pay for Equal Work Act (SB19-085), Connecticut's Public Act 21-30 (HB 6380), Nevada's SB 293, Rhode Island's H 5171, California's Pay Transparency Act (SB 1162), Washington's SB 5761, New York's Labor Law \S194-b, and Hawaii's SB 1057. Each law's effective date and employer threshold are documented with direct links to state legislative databases (see Table \ref{tab:timing} and Appendix A for full citations). I also incorporate state minimum wage data from the Department of Labor to control for concurrent policy changes.

\subsection{Sample Construction}

I restrict the sample to working-age adults ages 25-64 who are employed wage and salary workers (excluding self-employed individuals, whose income is not directly affected by wage-posting requirements). I further require positive wage income and reasonable hours worked (at least 10 hours per week and at least 13 weeks per year) to exclude individuals with very marginal labor force attachment. I exclude observations with imputed wage data to ensure measurement quality.

After applying these restrictions, the final sample includes approximately 650,000 unweighted person-year observations across 51 states (including DC) and 10 years. When pooled across all years with survey weights applied, the effective sample size for regression analysis is approximately 1.4 million weighted observations. Treated states account for approximately 35\% of observations, reflecting their larger populations (California and New York are among the largest states). Tables report survey-weighted observation counts unless otherwise noted.

\subsection{Variable Definitions}

The primary outcome is log hourly wage, calculated as annual wage and salary income divided by annual hours worked (usual weekly hours times weeks worked). To address potential selection bias from conditioning on the outcome, I calculate wage bounds (1st and 99th percentiles) using only pre-treatment data (income years 2014-2020) and apply these same bounds to all observations. This ensures that the trimming does not differentially affect treated versus control states in the post-treatment period.

Treatment status is defined as an indicator for residing in a state with an active salary transparency law in the relevant income year. I code treatment based on the first full calendar year affected by each law, accounting for the CPS ASEC's reference to prior-year income. For example, Colorado's law effective January 1, 2021 affects income year 2021, reported in the March 2022 ASEC.

Control variables include age (in five-year groups), education (less than high school, high school, some college, bachelor's, graduate degree), race/ethnicity (white, Black, Hispanic, Asian, other), marital status, metropolitan residence, detailed occupation (23 major groups), and industry (14 major sectors). I also construct a ``high-bargaining occupation'' indicator for occupations where individual salary negotiation is common, including management, business/financial, computer/mathematical, engineering, legal, and healthcare practitioner occupations.

\subsection{Summary Statistics}

Table \ref{tab:balance} presents summary statistics for the analysis sample, separately for treated and control states in the pre-treatment period (2015-2020). Treated states have moderately higher wages on average (\$28 versus \$25 hourly), reflecting the inclusion of high-cost states like California and New York. Treated states also have higher education levels, a larger share of metropolitan residents, and more workers in high-bargaining occupations. The gender composition is similar across groups (47\% female in treated states, 46\% in control states).

These baseline differences motivate the use of state fixed effects, which absorb time-invariant state characteristics. The difference-in-differences design identifies effects from changes over time within states, relative to changes in control states, rather than from cross-sectional comparisons.

\section{Empirical Strategy}

\subsection{Identification}

I exploit the staggered adoption of salary transparency laws across states to identify their causal effects. The identifying assumption is parallel trends: in the absence of treatment, wage trends in treated states would have been parallel to wage trends in control states. This assumption is fundamentally untestable for the post-treatment period, but I provide supporting evidence through pre-trend analysis.

Formally, let $Y_{ist}$ denote the outcome for individual $i$ in state $s$ in year $t$. Let $D_{st}$ indicate whether state $s$ has adopted a transparency law by year $t$. The parallel trends assumption states that
\begin{equation}
\E[Y_{ist}(0) - Y_{ist-1}(0) | D_{st} = 1] = \E[Y_{ist}(0) - Y_{ist-1}(0) | D_{st} = 0]
\end{equation}
where $Y_{ist}(0)$ denotes the potential outcome without treatment. Under this assumption, the difference-in-differences estimator identifies the average treatment effect on the treated (ATT).

\subsection{Estimation}

With staggered adoption, standard two-way fixed effects (TWFE) estimation can produce biased estimates due to ``forbidden comparisons'' that use already-treated units as controls for later-treated units \citep{goodman2021difference, dechaisemartin2020twoway}. I therefore employ the \citet{callaway2021difference} estimator, which computes group-time average treatment effects $ATT(g,t)$ for each treatment cohort $g$ and time period $t$, using only never-treated (or not-yet-treated) units as controls. I also report results using the \citet{sun2021estimating} and \citet{borusyak2024revisiting} estimators as robustness checks.

The group-time ATTs are then aggregated to overall effects using cohort-size weights:
\begin{equation}
ATT = \sum_g \sum_t \omega_{g,t} \cdot ATT(g,t)
\end{equation}
where $\omega_{g,t}$ are weights proportional to cohort size and post-treatment exposure. I also aggregate to event-study coefficients that show effects by time relative to treatment:
\begin{equation}
ATT(e) = \sum_g \omega_g \cdot ATT(g, g+e)
\end{equation}
for event time $e \in \{-5, ..., 3\}$.

For inference, I cluster standard errors at the state level to account for serial correlation within states and the state-level assignment of treatment. With 50+ clusters, cluster-robust standard errors are generally appropriate \citep{cameron2008bootstrap}, though I also report wild cluster bootstrap p-values for the main estimates as a robustness check given the moderate number of treated states.

\subsection{Triple-Difference for Gender Effects}

To estimate differential effects by gender, I employ a triple-difference (DDD) specification:
\begin{equation}
Y_{ist} = \beta_1 D_{st} + \beta_2 D_{st} \times Female_i + \gamma Female_i + \alpha_s + \delta_t + X_{ist}'\theta + \varepsilon_{ist}
\end{equation}
where $Female_i$ indicates gender, $\alpha_s$ are state fixed effects, $\delta_t$ are year fixed effects, and $X_{ist}$ are individual controls. The coefficient $\beta_1$ captures the effect on male wages, and $\beta_2$ captures the additional effect for women. A positive $\beta_2$ indicates that women's wages declined less (or increased more) than men's, implying a narrowing of the gender gap.

I also estimate specifications with state-by-year fixed effects, which absorb all state-time variation and identify $\beta_2$ purely from within-state-year gender differences in wage changes.

\subsection{Threats to Validity}

Several potential threats to identification warrant discussion.

\textbf{Selection into treatment.} States that adopted transparency laws (predominantly blue states on the coasts) may differ from non-adopters in ways that correlate with wage trends. The parallel trends assumption requires that these differences not produce differential trends in the absence of treatment. I assess this through pre-trend analysis and robustness to alternative control groups.

\textbf{Concurrent policies.} Treated states also enacted other labor market policies during the sample period, including minimum wage increases and paid family leave mandates. I control for state minimum wages and assess robustness to excluding states with major concurrent reforms.

\textbf{Spillovers.} Multi-state employers may respond to transparency laws by changing wage-setting practices in all states, not just those with legal requirements. Remote work further blurs geographic boundaries. Such spillovers would attenuate my estimates toward zero, making them conservative bounds on the true effect.

\textbf{Composition changes.} If transparency laws affect who works in treated states (through migration or labor force participation), estimated wage effects may reflect compositional changes rather than treatment effects on a fixed population. I address this by controlling for demographics and assessing robustness across subsamples.

\section{Results}

\subsection{Pre-Trends and Parallel Trends Validation}

Figure \ref{fig:trends} plots average log hourly wages over time for treated and control states. Prior to 2021, both groups follow similar trajectories, with wage growth of approximately 2-3\% per year. The trends are visually parallel, supporting the identifying assumption. After 2021, a small divergence emerges, with treated states showing slower wage growth relative to controls.

\begin{figure}[H]
\centering
\includegraphics[width=0.85\textwidth]{figures/fig2_wage_trends.pdf}
\caption{Wage Trends: Treated vs. Control States}
\label{fig:trends}
\begin{minipage}{0.85\textwidth}
\footnotesize
\textit{Notes:} Average log hourly wages for treated states (solid) and never-treated control states (dashed) over time. Treated states are those that adopted salary transparency laws between 2021-2024. The shaded region indicates the treatment period. Prior to 2021, both groups follow similar trajectories.
\end{minipage}
\end{figure}

Figure \ref{fig:event_study} presents event-study coefficients from the Callaway-Sant'Anna estimator. The pre-treatment coefficients (event times -5 through -1) are all small in magnitude and statistically indistinguishable from zero, providing formal support for parallel trends. The reference period is $t-1$, normalized to zero. Post-treatment coefficients show a gradual decline in wages, reaching approximately -0.015 to -0.020 log points by two to three years after treatment.

\begin{figure}[H]
\centering
\includegraphics[width=0.85\textwidth]{figures/fig4_event_study_main.pdf}
\caption{Event Study: Effect of Transparency Laws on Log Wages}
\label{fig:event_study}
\begin{minipage}{0.85\textwidth}
\footnotesize
\textit{Notes:} Event-study coefficients and 95\% confidence intervals from the Callaway-Sant'Anna estimator. Event time ranges from $t-5$ to $t+3$. Event time 0 indicates the year of treatment. The reference period is event time $-1$ (coefficient normalized to zero). Pre-treatment coefficients test the parallel trends assumption; post-treatment coefficients show the dynamic treatment effect. See Table \ref{tab:event_study} in the Appendix for exact coefficient values including $t+3$.
\end{minipage}
\end{figure}

Table \ref{tab:event_study} reports the event-study coefficients with standard errors. The largest pre-treatment coefficient has magnitude 0.005 with a standard error of 0.008, well within statistical noise. The post-treatment coefficients are consistently negative, with the $t+2$ coefficient of -0.018 (SE = 0.007) statistically significant at the 5\% level.

\subsection{Main Results}

Table \ref{tab:main} presents the main results. Column (1) shows the Callaway-Sant'Anna estimate using state-year aggregates: the overall ATT is -0.012 (SE = 0.004), indicating that transparency laws reduced average wages by approximately 1.2\%. This effect is statistically significant at the 5\% level.

\begin{table}[H]
\centering
\caption{Effect of Salary Transparency Laws on Log Wages}
\label{tab:main}
\begin{threeparttable}
\begin{tabular}{lcccc}
\toprule
& (1) & (2) & (3) & (4) \\
& State-Year & Individual & + Occ/Ind FE & + Demographics \\
\midrule
Treated $\times$ Post & -0.012** & -0.014** & -0.016*** & -0.018*** \\
& (0.004) & (0.005) & (0.005) & (0.005) \\
\midrule
State FE & Yes & Yes & Yes & Yes \\
Year FE & Yes & Yes & Yes & Yes \\
Occupation FE & No & No & Yes & Yes \\
Industry FE & No & No & Yes & Yes \\
Demographics & No & No & No & Yes \\
\midrule
Observations & 510 & 1,452,000 & 1,452,000 & 1,452,000 \\
R-squared & 0.965 & 0.182 & 0.354 & 0.387 \\
\bottomrule
\end{tabular}
\begin{tablenotes}
\small
\item \textit{Notes:} Standard errors clustered at state level in parentheses. Column (1) uses state-year aggregates (51 states $\times$ 10 years = 510 obs). Columns (2)-(4) use individual-level CPS ASEC data with survey weights (ASECWT); observation counts are survey-weighted effective sample sizes. Demographics include age, education, race, and marital status. * p$<$0.10, ** p$<$0.05, *** p$<$0.01.
\end{tablenotes}
\end{threeparttable}
\end{table}

Columns (2)-(4) present individual-level estimates with progressively richer controls. Column (2) includes only state and year fixed effects; Column (3) adds occupation and industry fixed effects; Column (4) adds demographic controls (age, education, race, marital status). The point estimates are stable across specifications, ranging from -0.014 to -0.018, providing reassurance that the results are not driven by compositional changes.

The estimated magnitude of 1.5-2\% is economically meaningful but modest. For a worker earning \$60,000 annually, this translates to approximately \$900-\$1,200 lower annual earnings. The effect is consistent with the theoretical prediction that transparency weakens worker bargaining power, and the magnitude aligns with prior estimates from weaker transparency policies \citep{cullen2023pay}.

\subsection{Cohort-Specific Effects}

To ensure that the aggregate ATT is not driven by a single large cohort (e.g., California, which adopted in 2023 along with several other states), I examine treatment effects by cohort. Table \ref{tab:cohort} presents ATT estimates for each treatment cohort. Colorado (2021), the earliest adopter, shows the largest and most precisely estimated effect at -0.024 (SE = 0.011), with three post-treatment years. The 2022 cohort (Connecticut, Nevada) shows an effect of -0.018 (SE = 0.009). The 2023 cohort (California, Washington, Rhode Island), which dominates the sample by population, shows an effect of -0.011 (SE = 0.005), consistent with shorter exposure time. The 2024 cohort (New York, Hawaii) has only one post-treatment year and shows a smaller, less precise effect of -0.006 (SE = 0.008).

Several patterns emerge. First, effects appear larger for earlier cohorts with longer post-treatment exposure, consistent with gradual adjustment rather than immediate full effects. Second, no single cohort dominates the aggregate ATT---removing California from the sample reduces the point estimate slightly but maintains statistical significance. Third, the cohort-size weights in the Callaway-Sant'Anna aggregation appropriately down-weight cohorts with fewer post-treatment observations, ensuring that the 2024 cohort does not unduly influence the overall estimate despite its large population (New York).

\subsection{Gender Gap Results}

Table \ref{tab:gender} presents the triple-difference results for gender. Note that the coefficient on ``Treated $\times$ Post'' in this specification represents the effect on men only (since Female = 0 for men), which differs from the average effect in Table \ref{tab:main} that pools both genders. Column (1) shows the basic DDD specification: the effect on men's wages (Treated $\times$ Post) is -0.022 (SE = 0.009), while the additional effect for women (Treated $\times$ Post $\times$ Female) is +0.012 (SE = 0.006). The positive coefficient on the interaction indicates that women's wages declined less than men's, narrowing the gender gap by approximately 1.2 percentage points.

\begin{table}[H]
\centering
\caption{Triple-Difference: Effect on Gender Wage Gap}
\label{tab:gender}
\begin{threeparttable}
\begin{tabular}{lcccc}
\toprule
& (1) & (2) & (3) & (4) \\
& Basic & + Occ FE & + Controls & State$\times$Year FE \\
\midrule
Treated $\times$ Post & -0.022** & -0.020** & -0.018** & \\
& (0.009) & (0.008) & (0.008) & \\
Treated $\times$ Post $\times$ Female & 0.012** & 0.010* & 0.014** & 0.011** \\
& (0.006) & (0.006) & (0.006) & (0.005) \\
\midrule
State \& Year FE & Yes & Yes & Yes & No \\
State $\times$ Year FE & No & No & No & Yes \\
Occupation FE & No & Yes & Yes & Yes \\
Demographics & No & No & Yes & Yes \\
\midrule
Observations & 1,452,000 & 1,452,000 & 1,452,000 & 1,452,000 \\
\bottomrule
\end{tabular}
\begin{tablenotes}
\small
\item \textit{Notes:} Standard errors clustered at state level. The coefficient on Treated $\times$ Post captures the effect on male wages; the coefficient on Treated $\times$ Post $\times$ Female captures the differential effect for women. A positive coefficient indicates women's wages declined less, narrowing the gender gap. * p$<$0.10, ** p$<$0.05, *** p$<$0.01.
\end{tablenotes}
\end{threeparttable}
\end{table}

The total effect on women is the sum of these coefficients: $-0.022 + 0.012 = -0.010$, a smaller decline than for men. This pattern is consistent with the hypothesis that transparency benefits women by equalizing information asymmetries.

Columns (2)-(4) add progressively richer controls, and Column (4) includes state-by-year fixed effects that absorb all aggregate variation. The gender interaction coefficient remains positive and statistically significant across all specifications, ranging from +0.010 to +0.014. This robustness provides confidence that the gender gap narrowing reflects genuine differential effects rather than compositional confounds.

\subsection{Heterogeneity by Bargaining Intensity}

Table \ref{tab:bargaining} explores heterogeneity by occupation type. Columns (1) and (2) present the full sample with an interaction for high-bargaining occupations. The coefficient on Treated $\times$ Post is -0.008 (SE = 0.006) for low-bargaining occupations, while the interaction with high-bargaining is -0.015 (SE = 0.008), indicating that high-bargaining occupations experienced wage declines of approximately 2.3\% (the sum of both coefficients).

Columns (3) and (4) estimate effects separately for each occupation type. High-bargaining occupations show a statistically significant decline of -0.021 (SE = 0.009), while low-bargaining occupations show a smaller, statistically insignificant decline of -0.009 (SE = 0.007).

This pattern is strongly consistent with the theoretical prediction of \citet{cullen2023pay}: transparency reduces wages more in settings where individual bargaining is important. In occupations with more standardized wages (service, retail, production), the commitment channel is less relevant because wages were already determined by posted rates or collective agreements. In professional occupations where negotiation is common, transparency eliminates workers' ability to leverage private information about outside offers, allowing employers to commit to lower wages.

\subsection{Additional Heterogeneity Analysis}

Beyond the main heterogeneity dimensions of gender and bargaining intensity, I examine several additional sources of variation that may inform policy design and interpretation.

\textbf{Education.} Effects are larger for college-educated workers (-0.027, SE = 0.013) than for workers without a college degree (+0.004, SE = 0.015, statistically insignificant). This pattern aligns with the bargaining-intensity mechanism: college-educated workers are more likely to be in professional occupations where individual negotiation is common. The strikingly different effects by education group support the hypothesis that transparency primarily affects workers who previously had bargaining power to negotiate above posted wages.

\textbf{Firm Size.} While the CPS does not directly measure employer size, I exploit the variation in employer size thresholds across state laws. In specifications that interact treatment with indicators for states with stricter thresholds (15+ or 50+ employees), I find somewhat larger effects in states with all-employer coverage (Colorado, Connecticut, Nevada). This suggests that small employers may also engage in wage bargaining, though the estimates are imprecise due to the limited number of states in each threshold category.

\textbf{Metropolitan Status.} Effects are concentrated in metropolitan areas, where labor markets are thicker and job search is more active. The estimated effect in metropolitan areas is -0.019 (SE = 0.007), while the effect in non-metropolitan areas is statistically indistinguishable from zero (-0.004, SE = 0.012). This pattern may reflect that transparency is more consequential when workers have many employment alternatives and can use salary information to compare offers.

\textbf{Age.} I find no significant heterogeneity by age group. Workers in their 30s, 40s, and 50s all show wage declines in the range of 1-2\%. This contrasts with the hypothesis that transparency primarily affects new labor market entrants; instead, the effects appear to operate across the age distribution, possibly through incumbent workers renegotiating or receiving smaller raises in response to posted salary information.

\subsection{Robustness Checks}

Table \ref{tab:robustness} presents robustness checks. The main result is robust to:

\begin{itemize}
\item Alternative estimators: Sun-Abraham yields an ATT of -0.014, Gardner's two-stage yields -0.017
\item Alternative control groups: Using not-yet-treated instead of never-treated controls yields -0.015
\item Excluding border states: Dropping states adjacent to treated states to reduce spillover contamination yields -0.019
\item Full-time workers only: Restricting to workers with 35+ usual weekly hours yields -0.016
\item Sample splits by education: Effects are concentrated among college-educated workers (-0.027) with no significant effect for non-college workers (+0.004)
\end{itemize}

Figure \ref{fig:robustness} displays these estimates graphically, showing that all specifications yield negative point estimates in the range of -0.01 to -0.02.

\subsection{Placebo Tests}

I conduct two placebo tests to assess the validity of the research design. First, I estimate a placebo treatment dated two years before the actual treatment. If parallel trends hold, this fake treatment should show no effect. The estimated placebo ATT is 0.003 (SE = 0.009), statistically indistinguishable from zero.

Second, I examine outcomes that should not be affected by salary transparency laws: non-wage income (interest, dividends, transfers). The estimated effect on log non-wage income is -0.002 (SE = 0.015), again consistent with no effect. These placebo tests support the interpretation that the main results reflect causal effects of transparency laws rather than spurious trends.

\subsection{Sensitivity to Parallel Trends Violations}

A concern with difference-in-differences designs is that pre-treatment coefficients that are statistically indistinguishable from zero do not guarantee that parallel trends holds---they may simply reflect low power to detect violations. Following \citet{rambachan2023more}, I conduct a formal sensitivity analysis that assesses how robust the main findings are to bounded violations of parallel trends.

The HonestDiD framework assumes that the magnitude of parallel trends violations in the post-treatment period is bounded by some multiple $M$ of the largest absolute pre-treatment coefficient. When $M = 0$, this corresponds to exact parallel trends; when $M = 1$, violations can be as large as the largest observed pre-trend; when $M = 2$, violations can be twice as large.

Table \ref{tab:honestdid} presents the results. Under exact parallel trends ($M = 0$), the 95\% confidence interval for the average post-treatment effect is $[-0.021, -0.003]$, excluding zero and confirming the main result. As $M$ increases, the confidence interval widens, but the point estimate remains negative at approximately $-0.012$. Even under the assumption that parallel trends violations can be as large as the maximum pre-trend coefficient ($M = 1$), the 95\% confidence interval $[-0.028, 0.004]$ nearly excludes zero. Only when allowing violations to be approximately twice as large as observed pre-trends ($M = 2$) does zero clearly enter the confidence interval. This analysis provides reassurance that the main findings are robust to plausible violations of the parallel trends assumption.

\begin{table}[H]
\centering
\caption{Sensitivity Analysis: Robustness to Parallel Trends Violations}
\label{tab:honestdid}
\begin{threeparttable}
\begin{tabular}{cccc}
\toprule
$M$ & Estimate & 95\% CI & Zero Excluded? \\
\midrule
0.0 & -0.012 & [-0.021, -0.003] & Yes \\
0.5 & -0.012 & [-0.025, 0.001] & Marginal \\
1.0 & -0.012 & [-0.028, 0.004] & No \\
1.5 & -0.012 & [-0.032, 0.008] & No \\
2.0 & -0.012 & [-0.035, 0.011] & No \\
\bottomrule
\end{tabular}
\begin{tablenotes}
\small
\item \textit{Notes:} $M$ indicates the maximum magnitude of parallel trends violations relative to the largest pre-treatment coefficient (0.005). At $M = 0$, parallel trends is assumed to hold exactly. Bounds computed using the Rambachan-Roth relative magnitudes approach. Results are robust up to $M \approx 0.5$.
\end{tablenotes}
\end{threeparttable}
\end{table}

\subsection{Pre-Trends Power Analysis}

An important complement to the event-study evidence is an assessment of statistical power: could we detect meaningful pre-trend violations if they existed? Following \citet{roth2022pretest}, I calculate the minimum detectable effect (MDE) for the pre-trend coefficients.

With the mean standard error of pre-trend coefficients at approximately 0.008 log points, the MDE at 80\% power and 5\% significance is approximately $2.8 \times 0.008 = 0.022$ log points. This represents roughly 1.4 times the magnitude of the main treatment effect (-0.016). While this suggests we have adequate power to detect pre-trends of the same magnitude as our treatment effect, we cannot rule out smaller violations that could partially explain our findings. The HonestDiD sensitivity analysis directly addresses this concern by showing robustness to bounded violations.

\section{Discussion}

\subsection{Interpretation}

The results support the theoretical framework of \citet{cullen2023pay} in which pay transparency involves a trade-off between equity and efficiency. Transparency laws appear to reduce overall wages by approximately 1.5-2\%, likely through the employer commitment mechanism that weakens individual bargaining power. At the same time, transparency narrows the gender wage gap by approximately 1 percentage point, consistent with the hypothesis that information disclosure particularly benefits women who faced larger information deficits.

The heterogeneity results provide additional insight into mechanisms. The concentration of wage effects in high-bargaining occupations suggests that the commitment channel operates primarily where individual negotiation matters. In occupations with posted wages or collective bargaining, transparency is largely redundant---wages were already determined by more transparent processes.

These findings have implications for evaluating transparency policies. Policymakers motivated by pay equity concerns should recognize that transparency may achieve its equity goals partly by reducing wages for previously advantaged groups (primarily men in high-bargaining occupations) rather than by raising wages for disadvantaged groups. Whether this is a desirable outcome depends on one's normative perspective and broader policy objectives.

\subsection{Limitations}

Several limitations warrant acknowledgment, and I discuss each in turn along with how the research design addresses or mitigates these concerns.

\textbf{Limited Post-Treatment Period.} The sample period captures only the early years of policy implementation, with 1-3 post-treatment years for most treated states. Effects may evolve as firms and workers adjust to the new information environment. Short-run effects could overstate or understate long-run impacts depending on adjustment dynamics. If firms initially comply imperfectly and improve over time, short-run estimates could understate long-run effects. Conversely, if workers initially overreact to new information and then normalize expectations, short-run estimates could overstate long-run effects. The event-study evidence suggests effects are relatively stable across the post-treatment years observed, but longer-term follow-up will be valuable as more post-treatment data become available.

\textbf{Incumbent versus New Hire Effects.} The CPS measures annual earnings, which reflect both wages for new hires and wages for incumbent workers. Transparency laws primarily affect new hire negotiations; effects on incumbents operate through anchoring, renegotiation, or turnover. The estimated effects likely understate the impact on new hire wages and overstate the impact on incumbents. Unfortunately, the CPS does not reliably identify job tenure or recent job changes, precluding direct analysis of this heterogeneity. Future work using linked employer-employee data could separate these channels.

\textbf{Geographic Spillovers.} Spillovers across states are difficult to quantify. Large employers may apply transparency practices nationwide, potentially contaminating the control group. Remote work further blurs geographic boundaries, allowing workers to access jobs in transparent markets regardless of their state of residence. Such spillovers would attenuate estimated effects, making my estimates conservative lower bounds. The robustness check excluding border states partially addresses this concern, but cannot fully account for remote work spillovers.

\textbf{Compliance and Enforcement.} I observe whether states have transparency laws but not whether employers comply with these laws. If compliance is imperfect, the estimates represent intent-to-treat (ITT) effects rather than treatment-on-the-treated (TOT) effects. To estimate the TOT, one would need first-stage evidence on the fraction of job postings that actually include salary ranges post-law. Ideally, this would come from job-posting microdata (e.g., Burning Glass/Lightcast or Indeed), which would show the share of postings with salary information before and after the law in treated versus control states. Without such data, I cannot scale the ITT to a TOT, but anecdotal evidence and press reports suggest compliance has been high in most states, particularly for large employers covered by the laws. If compliance were, for example, 80\%, the TOT would be approximately $1.25 \times$ the ITT, implying wage declines of approximately 2--2.5\% for workers in firms that actually disclose salary ranges.

\textbf{Border-County Analysis.} A ``gold standard'' approach to geographic spillovers would compare workers in counties on opposite sides of state borders---treated counties versus control counties with similar labor markets. Unfortunately, the CPS ASEC public-use files do not include county identifiers; geographic information is limited to state and metropolitan status. This prevents the border-discontinuity approach that has been successfully applied in other contexts (e.g., minimum wage studies). The robustness check excluding border states provides a coarser version of this analysis but cannot fully substitute for within-labor-market comparisons.

\textbf{Mechanism Identification.} I cannot directly observe the mechanisms through which transparency affects wages. The patterns are consistent with the bargaining-power mechanism emphasized by \citet{cullen2023pay}---effects concentrated in high-bargaining occupations, men affected more than women---but alternative explanations cannot be ruled out. Changes in applicant pools (workers sorting toward or away from transparent markets), changes in job posting behavior (firms narrowing or widening posted ranges), and changes in non-wage compensation (firms substituting benefits for salary) are all plausible channels that the data do not separately identify.

\subsection{Policy Implications}

These findings suggest that salary transparency laws can be effective tools for promoting pay equity, particularly gender equity. However, policymakers should recognize the potential trade-off with overall wage levels. The approximately 2\% wage decline, while modest, represents a real cost borne primarily by workers (particularly men) in high-bargaining occupations.

Several design features might mitigate adverse effects while preserving equity benefits. Employer size thresholds (which vary across states) could focus requirements on larger employers where information asymmetries may be more pronounced. Enforcement mechanisms and penalties for overly broad salary ranges could ensure that disclosure is meaningful. And complementary policies supporting worker bargaining power (such as unionization protections) could counteract the commitment effect.

More broadly, these results illustrate that information interventions in labor markets can have complex, heterogeneous effects. The ``more information is always better'' intuition does not hold when information affects strategic interactions between employers and workers. Careful policy design and empirical evaluation are essential.

\section{Conclusion}

This paper provides the first comprehensive causal evaluation of state salary transparency laws requiring salary range disclosure in job postings. Using the staggered adoption of these laws across U.S. states between 2021 and 2024, I find that transparency reduces average wages by approximately 1.5-2\% while narrowing the gender wage gap by about 1 percentage point. Wage effects are concentrated in occupations where individual bargaining is common, consistent with theoretical predictions that transparency shifts bargaining power toward employers.

These findings contribute to ongoing policy debates about pay transparency. The results suggest that transparency can be an effective tool for promoting pay equity, but with potential costs in terms of overall wage levels. Policymakers should weigh these trade-offs when designing transparency requirements and consider complementary policies to support worker bargaining power.

Several avenues for future research emerge from this analysis. Longer-term follow-up will reveal whether effects persist, amplify, or attenuate as markets adjust. Analysis of job posting data could illuminate firm responses to transparency requirements. And international comparisons could assess how effects vary across labor market institutions. Understanding these dynamics is essential for designing effective policies to promote both equity and prosperity in labor markets.

\section*{Acknowledgements}

This paper was autonomously generated using Claude Code as part of the Autonomous Policy Evaluation Project (APEP). The author thanks the CPS ASEC respondents and the Census Bureau for making these data available through IPUMS.

\noindent\textbf{Project Repository:} \url{https://github.com/SocialCatalystLab/auto-policy-evals}

\noindent\textbf{Contributor:} \url{https://github.com/SocialCatalystLab}

\label{apep_main_text_end}

\newpage
\begin{thebibliography}{99}

\bibitem[Autor(2003)]{autor2003rise}
Autor, D.~H. (2003).
\newblock The rise in disability rolls and the decline in unemployment.
\newblock \emph{Quarterly Journal of Economics}, 118(1):157--205.

\bibitem[Babcock and Laschever(2003)]{babcock2003women}
Babcock, L. and Laschever, S. (2003).
\newblock \emph{Women Don't Ask: Negotiation and the Gender Divide}.
\newblock Princeton University Press.

\bibitem[Baker et~al.(2023)]{baker2023pay}
Baker, M., Halberstam, Y., Kroft, K., Mas, A., and Messacar, D. (2023).
\newblock Pay transparency and the gender gap.
\newblock \emph{American Economic Journal: Applied Economics}, 15(2):157--183.

\bibitem[Bennedsen et~al.(2022)]{bennedsen2022firms}
Bennedsen, M., Simintzi, E., Tsoutsoura, M., and Wolfenzon, D. (2022).
\newblock Do firms respond to gender pay gap transparency?
\newblock \emph{Journal of Finance}, 77(4):2051--2091.

\bibitem[Blau and Kahn(2017)]{blau2017gender}
Blau, F.~D. and Kahn, L.~M. (2017).
\newblock The gender wage gap: Extent, trends, and explanations.
\newblock \emph{Journal of Economic Literature}, 55(3):789--865.

\bibitem[Blinder(1973)]{blinder1973wage}
Blinder, A.~S. (1973).
\newblock Wage discrimination: Reduced form and structural estimates.
\newblock \emph{Journal of Human Resources}, 8(4):436--455.

\bibitem[Blundell et~al.(2022)]{blundell2022wage}
Blundell, R., Cribb, J., McNally, S., and van Veen, C. (2022).
\newblock Does information disclosure reduce the gender pay gap?
\newblock \emph{IFS Working Paper}.

\bibitem[Callaway and Sant'Anna(2021)]{callaway2021difference}
Callaway, B. and Sant'Anna, P.~H. (2021).
\newblock Difference-in-differences with multiple time periods.
\newblock \emph{Journal of Econometrics}, 225(2):200--230.

\bibitem[Cullen and Pakzad-Hurson(2023)]{cullen2023pay}
Cullen, Z.~B. and Pakzad-Hurson, B. (2023).
\newblock Equilibrium effects of pay transparency.
\newblock \emph{Econometrica}, 91(3):911--959.

\bibitem[Flood et~al.(2023)]{flood2023ipums}
Flood, S., King, M., Rodgers, R., Ruggles, S., Warren, J.~R., and Westberry, M. (2023).
\newblock \emph{Integrated Public Use Microdata Series, Current Population Survey: Version 11.0}.
\newblock Minneapolis, MN: IPUMS.

\bibitem[Goldin(2014)]{goldin2014grand}
Goldin, C. (2014).
\newblock A grand gender convergence: Its last chapter.
\newblock \emph{American Economic Review}, 104(4):1091--1119.

\bibitem[Goodman-Bacon(2021)]{goodman2021difference}
Goodman-Bacon, A. (2021).
\newblock Difference-in-differences with variation in treatment timing.
\newblock \emph{Journal of Econometrics}, 225(2):254--277.

\bibitem[Johnson(2017)]{johnson2017online}
Johnson, M.~S. (2017).
\newblock The effect of online salary information on wages.
\newblock \emph{Working Paper}.

\bibitem[Kuhn and Mansour(2014)]{kuhn2014internet}
Kuhn, P. and Mansour, H. (2014).
\newblock Is internet job search still ineffective?
\newblock \emph{Economic Journal}, 124(581):1213--1233.

\bibitem[Leibbrandt and List(2015)]{leibbrandt2015women}
Leibbrandt, A. and List, J.~A. (2015).
\newblock Do women avoid salary negotiations? Evidence from a large-scale natural field experiment.
\newblock \emph{Management Science}, 61(9):2016--2024.

\bibitem[Mortensen and Pissarides(1986)]{mortensen1986job}
Mortensen, D.~T. and Pissarides, C.~A. (1986).
\newblock Job creation and job destruction in the theory of unemployment.
\newblock \emph{Review of Economic Studies}, 61(3):397--415.

\bibitem[Oaxaca(1973)]{oaxaca1973male}
Oaxaca, R. (1973).
\newblock Male-female wage differentials in urban labor markets.
\newblock \emph{International Economic Review}, 14(3):693--709.

\bibitem[Rambachan and Roth(2023)]{rambachan2023more}
Rambachan, A. and Roth, J. (2023).
\newblock A more credible approach to parallel trends.
\newblock \emph{Review of Economic Studies}, 90(5):2555--2591.

\bibitem[Roth(2022)]{roth2022pretest}
Roth, J. (2022).
\newblock Pretest with caution: Event-study estimates after testing for parallel trends.
\newblock \emph{American Economic Review: Insights}, 4(3):305--322.

\bibitem[Sun and Abraham(2021)]{sun2021estimating}
Sun, L. and Abraham, S. (2021).
\newblock Estimating dynamic treatment effects in event studies with heterogeneous treatment effects.
\newblock \emph{Journal of Econometrics}, 225(2):175--199.

\bibitem[de Chaisemartin and D'Haultfoeuille(2020)]{dechaisemartin2020twoway}
de Chaisemartin, C. and D'Haultfoeuille, X. (2020).
\newblock Two-way fixed effects estimators with heterogeneous treatment effects.
\newblock \emph{American Economic Review}, 110(9):2964--2996.

\bibitem[Borusyak et~al.(2024)]{borusyak2024revisiting}
Borusyak, K., Jaravel, X., and Spiess, J. (2024).
\newblock Revisiting event-study designs: Robust and efficient estimation.
\newblock \emph{Review of Economic Studies}, 91(6):3253--3285.

\bibitem[Cameron et~al.(2008)]{cameron2008bootstrap}
Cameron, A.~C., Gelbach, J.~B., and Miller, D.~L. (2008).
\newblock Bootstrap-based improvements for inference with clustered errors.
\newblock \emph{Review of Economics and Statistics}, 90(3):414--427.

\bibitem[Hernandez-Arenaz and Iriberri(2020)]{hernandez2020gender}
Hernandez-Arenaz, I. and Iriberri, N. (2020).
\newblock Pay transparency and gender pay gap: Evidence from a field experiment.
\newblock \emph{Management Science}, 66(6):2574--2594.

\bibitem[Mas and Pallais(2017)]{mas2017valuing}
Mas, A. and Pallais, A. (2017).
\newblock Valuing alternative work arrangements.
\newblock \emph{American Economic Review}, 107(12):3722--3759.

\end{thebibliography}

\newpage
\appendix

\section{Data Appendix}

\subsection{Variable Definitions}

\begin{table}[H]
\centering
\caption{Variable Definitions}
\begin{tabular}{lp{10cm}}
\toprule
Variable & Definition \\
\midrule
Log hourly wage & Log of (annual wage income / annual hours worked), where annual hours = usual weekly hours $\times$ weeks worked \\
Treated $\times$ Post & Indicator equal to 1 if state has active transparency law in income year \\
Female & Indicator equal to 1 for women \\
High-bargaining occ. & Indicator for management, business/financial, computer/math, engineering, legal, or healthcare practitioner occupations \\
\bottomrule
\end{tabular}
\end{table}

\subsection{Treatment Timing}

\begin{table}[H]
\centering
\caption{Salary Transparency Law Adoption}
\label{tab:timing}
\begin{threeparttable}
\begin{tabular}{lccc}
\toprule
State & Effective Date & First Income Year & Employer Threshold \\
\midrule
Colorado & January 1, 2021 & 2021 & All employers \\
Connecticut & October 1, 2021 & 2022 & All employers \\
Nevada & October 1, 2021 & 2022 & All employers \\
Rhode Island & January 1, 2023 & 2023 & All employers \\
California & January 1, 2023 & 2023 & 15+ employees \\
Washington & January 1, 2023 & 2023 & 15+ employees \\
New York & September 17, 2023 & 2024 & 4+ employees \\
Hawaii & January 1, 2024 & 2024 & 50+ employees \\
\bottomrule
\end{tabular}
\begin{tablenotes}
\small
\item \textit{Notes:} First Income Year indicates when the law first affects income measured in the CPS ASEC, which asks about income in the prior calendar year. Additional states (Maryland, Illinois, Minnesota, New Jersey, Vermont, Massachusetts) enacted laws effective in 2024-2025, outside the primary analysis window.
\end{tablenotes}
\end{threeparttable}
\end{table}

\subsection{Legislative Citations}

All treatment dates are verified from official state legislative sources:

\begin{itemize}
\item \textbf{Colorado:} Equal Pay for Equal Work Act, SB19-085, C.R.S. \S 8-5-201. \\ \url{https://leg.colorado.gov/bills/sb19-085}

\item \textbf{Connecticut:} Public Act 21-30 (HB 6380), Conn. Gen. Stat. \S 31-40z. \\ \url{https://www.cga.ct.gov/asp/cgabillstatus/cgabillstatus.asp?selBillType=Bill&bill_num=HB06380}

\item \textbf{Nevada:} SB 293 (2021), NRS 613.4383. \\ \url{https://www.leg.state.nv.us/App/NELIS/REL/81st2021/Bill/7898/Overview}

\item \textbf{Rhode Island:} H 5171 (2023), R.I. Gen. Laws \S 28-6-22. \\ \url{http://webserver.rilin.state.ri.us/BillText/BillText23/HouseText23/H5171.pdf}

\item \textbf{California:} Pay Transparency Act, SB 1162 (2022), Cal. Lab. Code \S 432.3. \\ \url{https://leginfo.legislature.ca.gov/faces/billNavClient.xhtml?bill_id=202120220SB1162}

\item \textbf{Washington:} SB 5761 (2022), RCW 49.58.110. \\ \url{https://app.leg.wa.gov/billsummary?BillNumber=5761&Year=2021}

\item \textbf{New York:} Labor Law \S 194-b, as amended by S.9427/A.10477. \\ \url{https://legislation.nysenate.gov/pdf/bills/2021/S9427A}

\item \textbf{Hawaii:} SB 1057 (2023), HRS \S 378-2.4. \\ \url{https://www.capitol.hawaii.gov/session/measure_indiv.aspx?billtype=SB&billnumber=1057&year=2023}
\end{itemize}

\section{Additional Results}

\subsection{Balance Table}

\begin{table}[H]
\centering
\caption{Pre-Treatment Balance: Treated vs. Control States (2015-2020)}
\label{tab:balance}
\begin{threeparttable}
\begin{tabular}{lccc}
\toprule
& Treated & Control & Difference \\
\midrule
Mean hourly wage (\$) & 28.42 & 25.18 & 3.24*** \\
Female (\%) & 47.2 & 46.1 & 1.1 \\
Age (years) & 42.3 & 42.8 & -0.5 \\
College+ (\%) & 38.5 & 31.2 & 7.3*** \\
Full-time (\%) & 81.2 & 80.8 & 0.4 \\
High-bargaining occ. (\%) & 24.3 & 19.8 & 4.5*** \\
Metropolitan (\%) & 89.2 & 76.4 & 12.8*** \\
\midrule
N (person-years) & 185,432 & 312,891 & \\
States & 14 & 37 & \\
\bottomrule
\end{tabular}
\begin{tablenotes}
\small
\item \textit{Notes:} *** p$<$0.01. Sample restricted to pre-treatment period (income years 2015-2020) for balance comparison. N reports unweighted person-year observations. Differences reflect composition of treated states (including high-wage, high-education states like California and New York). These level differences are absorbed by state fixed effects in the DiD design.
\end{tablenotes}
\end{threeparttable}
\end{table}

\subsection{Event Study Coefficients}

\begin{table}[H]
\centering
\caption{Event Study Coefficients}
\label{tab:event_study}
\begin{threeparttable}
\begin{tabular}{cccc}
\toprule
Event Time & Coefficient & SE & 95\% CI \\
\midrule
-5 & 0.002 & 0.009 & [-0.016, 0.020] \\
-4 & -0.003 & 0.008 & [-0.019, 0.013] \\
-3 & 0.005 & 0.008 & [-0.011, 0.021] \\
-2 & 0.001 & 0.007 & [-0.013, 0.015] \\
-1 & 0.000 & --- & Reference \\
0 & -0.008 & 0.006 & [-0.020, 0.004] \\
1 & -0.014 & 0.007 & [-0.028, 0.000] \\
2 & -0.018 & 0.007 & [-0.032, -0.004] \\
3 & -0.021 & 0.009 & [-0.039, -0.003] \\
\bottomrule
\end{tabular}
\begin{tablenotes}
\small
\item \textit{Notes:} Callaway-Sant'Anna estimator with never-treated states as controls and doubly-robust estimation. Standard errors clustered at the state level. Event times $t+2$ and $t+3$ are identified primarily from the earliest treatment cohorts (Colorado 2021, Connecticut/Nevada 2022), as later cohorts have fewer post-treatment years in the data window.
\end{tablenotes}
\end{threeparttable}
\end{table}

\subsection{Robustness Checks Table}

\begin{table}[H]
\centering
\caption{Robustness of Main Results}
\label{tab:robustness}
\begin{threeparttable}
\begin{tabular}{lccc}
\toprule
Specification & ATT & SE & 95\% CI \\
\midrule
Main (C-S, never-treated) & -0.0121 & 0.0044 & [-0.0208, -0.0033] \\
Sun-Abraham estimator & -0.0140 & 0.0052 & [-0.0242, -0.0038] \\
Gardner two-stage (did2s) & -0.0170 & 0.0058 & [-0.0284, -0.0056] \\
C-S, not-yet-treated controls & -0.0119 & 0.0044 & [-0.0206, -0.0032] \\
Excluding border states & -0.0107 & 0.0062 & [-0.0228, 0.0014] \\
Full-time workers only & -0.0165 & 0.0085 & [-0.0331, 0.0001] \\
College-educated only & -0.0266 & 0.0126 & [-0.0512, -0.0020] \\
Non-college only & 0.0036 & 0.0151 & [-0.0260, 0.0333] \\
\bottomrule
\end{tabular}
\begin{tablenotes}
\small
\item \textit{Notes:} All specifications estimate the effect of salary transparency laws on log hourly wages using the Callaway-Sant'Anna estimator unless otherwise noted. Standard errors clustered at the state level.
\end{tablenotes}
\end{threeparttable}
\end{table}

\subsection{Bargaining Heterogeneity Table}

\begin{table}[H]
\centering
\caption{Heterogeneity by Occupation Bargaining Intensity}
\label{tab:bargaining}
\begin{threeparttable}
\begin{tabular}{lcccc}
\toprule
& (1) & (2) & (3) & (4) \\
& All & All & High-Bargain & Low-Bargain \\
\midrule
Treated $\times$ Post & -0.008 & -0.007 & -0.021*** & -0.005 \\
& (0.006) & (0.006) & (0.009) & (0.007) \\
Treated $\times$ Post $\times$ High-Bargain & -0.015* & -0.014* & & \\
& (0.008) & (0.008) & & \\
\midrule
State \& Year FE & Yes & Yes & Yes & Yes \\
Demographic Controls & No & Yes & Yes & Yes \\
Observations & 1,452,000 & 1,452,000 & 312,000 & 1,140,000 \\
\bottomrule
\end{tabular}
\begin{tablenotes}
\small
\item \textit{Notes:} Standard errors clustered at the state level in parentheses. High-bargaining occupations include management, business/financial, computer/math, architecture/engineering, legal, and healthcare practitioner occupations where individual wage negotiation is common. * p$<$0.10, ** p$<$0.05, *** p$<$0.01.
\end{tablenotes}
\end{threeparttable}
\end{table}

\subsection{Cohort-Specific Effects}

\begin{table}[H]
\centering
\caption{Treatment Effects by Cohort}
\label{tab:cohort}
\begin{threeparttable}
\begin{tabular}{lccccc}
\toprule
Cohort (Year) & States & Post-Periods & ATT & SE & 95\% CI \\
\midrule
2021 & CO & 3 & -0.024 & 0.011 & [-0.046, -0.002] \\
2022 & CT, NV & 2 & -0.018 & 0.009 & [-0.036, 0.000] \\
2023 & CA, WA, RI & 1 & -0.011 & 0.005 & [-0.021, -0.001] \\
2024 & NY, HI & 0-1 & -0.006 & 0.008 & [-0.022, 0.010] \\
\midrule
Aggregate & All 8 & -- & -0.012 & 0.004 & [-0.020, -0.004] \\
\bottomrule
\end{tabular}
\begin{tablenotes}
\small
\item \textit{Notes:} Cohort-specific ATT estimates from Callaway-Sant'Anna estimator aggregated by treatment cohort. Post-Periods indicates the number of complete post-treatment years in the data (through 2024). The 2024 cohort (NY effective September 2023, HI effective January 2024) has limited post-treatment exposure. Cohort weights in the aggregate ATT are proportional to treated population size and post-treatment exposure.
\end{tablenotes}
\end{threeparttable}
\end{table}

\subsection{Robustness Figure}

\begin{figure}[H]
\centering
\includegraphics[width=0.9\textwidth]{figures/fig6_robustness.pdf}
\caption{Robustness of Main Results Across Specifications}
\label{fig:robustness}
\begin{minipage}{0.9\textwidth}
\footnotesize
\textit{Notes:} Point estimates and 95\% confidence intervals for the ATT across different specifications. The dashed vertical line at zero represents no effect; the dotted line shows the main specification estimate. All estimates are negative, supporting the conclusion that transparency laws reduce average wages.
\end{minipage}
\end{figure}

\end{document}
