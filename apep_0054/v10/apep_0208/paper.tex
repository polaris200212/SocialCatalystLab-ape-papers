\documentclass[12pt]{article}

% UTF-8 encoding and fonts
\usepackage[utf8]{inputenc}
\usepackage[T1]{fontenc}
\usepackage{lmodern}
\usepackage{microtype}

% Page setup
\usepackage[margin=1in]{geometry}
\usepackage{setspace}
\onehalfspacing

% Math and symbols
\usepackage{amsmath,amssymb}

% Graphics
\usepackage{graphicx}
\usepackage{float}
\usepackage{subcaption}

% Tables
\usepackage{booktabs}
\usepackage{array}
\usepackage{multirow}
\usepackage{tabularx}
\usepackage{threeparttable}

% Bibliography
\usepackage{natbib}
\bibliographystyle{aer}

% Hyperlinks
\usepackage{hyperref}
\hypersetup{
    colorlinks=true,
    linkcolor=blue,
    citecolor=blue,
    urlcolor=blue
}

% Captions
\usepackage{caption}
\captionsetup{font=small,labelfont=bf}

% Section formatting
\usepackage{titlesec}
\titleformat{\section}{\large\bfseries}{\thesection.}{0.5em}{}
\titleformat{\subsection}{\normalsize\bfseries}{\thesubsection}{0.5em}{}

% Custom commands
\newcommand{\E}{\mathbb{E}}
\newcommand{\Var}{\text{Var}}
\newcommand{\Cov}{\text{Cov}}

\title{Making Wages Visible: Labor Market Dynamics \\ Under Salary Transparency\footnote{This paper is a revision of APEP-0206 (\url{https://github.com/SocialCatalystLab/ape-papers/tree/main/papers/apep_0206}). See \url{https://github.com/SocialCatalystLab/ape-papers/tree/main/papers/apep_0162} for the original.}}
\author{APEP Autonomous Research\thanks{Autonomous Policy Evaluation Project. This paper was autonomously generated using Claude Code. Correspondence: scl@econ.uzh.ch} \and @SocialCatalystLab \\ @olafdrw, @SocialCatalystLab, @ai1scl}
\date{February 2026}

\begin{document}

\maketitle

\begin{abstract}
\noindent
Salary transparency laws require employers to post pay ranges in job listings. Theory predicts two countervailing forces: employer commitment compresses wages, while information equalization benefits previously disadvantaged workers. I study the staggered adoption of these laws across eight U.S.\ states using CPS microdata (614,625 workers) and Census QWI administrative records. Three findings emerge. First, aggregate wages are unaffected---the commitment channel does not dominate. Second, the gender earnings gap narrows substantially: CPS triple-difference estimates yield 4--6 percentage points; QWI administrative data independently confirm this at 6.1 percentage points ($p < 0.001$). Third, labor market dynamism---hiring, separations, job creation---is unchanged. Two independent datasets measuring different populations at different frequencies point to a single mechanism: transparency equalizes information without disrupting labor markets, achieving equity gains at zero efficiency cost.
\end{abstract}

\vspace{1em}
\noindent\textbf{JEL Codes:} J31, J71, J38, K31, J63 \\
\noindent\textbf{Keywords:} pay transparency, gender wage gap, labor market dynamics, wage posting, difference-in-differences, QWI

\newpage

\section{Introduction}

When employers must reveal what they pay, who benefits and what breaks?

Salary transparency laws---requiring compensation ranges in job listings---represent one of the most significant recent interventions in how labor markets process information. Eight U.S.\ states now mandate salary disclosure, collectively covering over 80 million workers. Theory offers sharply conflicting predictions. In the \citet{cullen2023pay} framework, transparency commits employers to posted ranges (potentially compressing wages downward) while simultaneously equalizing information between workers who differ in their access to salary data. If women historically faced larger information deficits \citep{babcock2003women, leibbrandt2015women}, transparency should benefit them disproportionately. But the equilibrium implications for labor market flows remain unclear: does transparency trigger costly reallocation, or does the market absorb new information without disruption?

Existing evidence is fragmentary. \citet{cullen2023pay} study ``right-to-ask'' laws---a weaker intervention than mandatory posting. \citet{baker2023pay} examine a single firm's internal disclosure policy. \citet{bennedsen2022firms} study Denmark's aggregate reporting mandate. No study examines mandatory job-posting transparency using both worker-side and employer-side data.

This paper fills that gap. I exploit staggered adoption across eight states and combine two complementary datasets. The CPS ASEC provides individual-level microdata on 614,625 workers over income years 2014--2024, capturing wages, demographics, and occupational detail. The Census Bureau's Quarterly Workforce Indicators (QWI)---administrative employer records from the LEHD program---provide state-quarter panels on earnings, hiring, separations, and job creation, disaggregated by sex and industry. Together, these datasets observe transparency's effects from both sides of the labor market.

Three findings emerge. First, \textit{aggregate wages are unaffected}. Both the CPS (ATT $= -0.004$, SE $= 0.006$) and QWI (ATT $= -0.001$, SE $= 0.020$) show precisely estimated zeros. Second, \textit{the gender gap narrows substantially}. CPS triple-difference estimates show women's wages rise 4.0--5.6 percentage points relative to men's. The QWI independently confirms this: women's quarterly earnings rise 6.1 percentage points relative to men's ($p < 0.001$), with effects of 8.8 pp in high-bargaining industries and 7.0 pp in low-bargaining industries. The concordance across a household survey and administrative records---measuring different populations at different frequencies---substantially strengthens this finding. Third, \textit{labor market dynamism is unchanged}. All five QWI flow variables produce small, insignificant coefficients ($p > 0.5$), ruling out costly adjustment.

The pattern points to a single mechanism: transparency equalizes information without disrupting labor markets (Table~\ref{tab:predictions}). The data are consistent with the information channel and inconsistent with employer commitment or costly adjustment operating as dominant forces.

An inferential caveat deserves transparency of its own. With eight treated states, Fisher randomization inference for the CPS gender estimate produces $p = 0.154$---above conventional thresholds. But the independent QWI confirmation from 51 state clusters ($p < 0.001$), the stability of estimates across leave-one-out samples ($[0.042, 0.054]$), and the HonestDiD bounds excluding zero under exact parallel trends ($[0.043, 0.100]$) collectively provide strong evidence.

The paper proceeds as follows. Section~\ref{sec:framework} develops a simple model that generates testable predictions across all three margins. Section~\ref{sec:background} describes the institutional setting and the key distinction between ex ante disclosure laws and earlier transparency interventions. Section~\ref{sec:data} introduces both datasets and their complementarity. Section~\ref{sec:strategy} presents the empirical strategy, including the triple-difference design. Section~\ref{sec:results} reports results. Section~\ref{sec:robustness} addresses inference with few treated clusters. Section~\ref{sec:discussion} discusses mechanisms and implications. Section~\ref{sec:conclusion} concludes.

\section{Conceptual Framework}
\label{sec:framework}

\subsection{A Simple Model of Transparency}

Consider a labor market with informed (I) and uninformed (U) workers bargaining with employers. Let $w^*$ denote the competitive wage and $\delta_j$ the information deficit of type $j$. Pre-transparency, informed workers capture their full marginal product ($w_I = w^*$), while uninformed workers accept $w_U = w^* - \delta_U$ because they cannot credibly threaten outside offers they do not know about.

Transparency introduces a publicly observable signal $s$ about the wage distribution. This has two effects \citep{cullen2023pay}. First, employers posting ranges face a \textit{commitment cost} $c$, reducing willingness to negotiate above the midpoint. Second, previously uninformed workers observe $s$ and update beliefs about outside options. If $\delta_U$ falls to $\delta_U' < \delta_U$, the uninformed gain leverage. The net effect on average wages is $-c + (\delta_U - \delta_U') \cdot \text{share}_U$, which is ambiguous.

The gender gap prediction is sharper. If women are disproportionately uninformed---$\delta_F > \delta_M$---transparency narrows the gap by $(\delta_F - \delta_F') - (\delta_M - \delta_M')$, unambiguously positive whenever women's information deficit is larger. The empirical literature supports this premise: women are less likely to initiate salary negotiations \citep{babcock2003women}, and gender differences in negotiation shrink when wage negotiability is made explicit \citep{leibbrandt2015women}.

\subsection{Predictions for Labor Market Flows}

The channels have distinct implications for labor market dynamics.

Under \textit{information equalization}, workers who learn outside options may search more effectively or renegotiate, but the aggregate effect on flows is ambiguous---some stay (renegotiation succeeds), others leave (better options discovered). Under \textit{employer commitment}, firms that commit to posted ranges may become less responsive to individual threats, with ambiguous flow predictions. Under \textit{costly adjustment}, transparency triggers reallocation---hiring and separations spike, net job creation declines.

Table~\ref{tab:predictions} summarizes. The key discriminating outcome is the \textit{combination} across all three margins.

\begin{table}[H]
\centering
\caption{Theoretical Predictions by Channel}
\label{tab:predictions}
\begin{threeparttable}
\begin{tabular}{lccccc}
\toprule
& Aggregate & Gender & Hiring & Separation & Net Job \\
Channel & Wages & Gap & Rate & Rate & Creation \\
\midrule
Information equalization & $0$ or $-$ & $-$ & $0$ & $0$ & $0$ \\
Employer commitment & $-$ & $0$ or $-$ & $-$ & $+$ or $0$ & $-$ \\
Costly adjustment & $-$ & Ambiguous & $+$ & $+$ & $-$ \\
Frictionless benchmark & $0$ & $-$ & $0$ & $0$ & $0$ \\
\midrule
\textit{Observed (this paper)} & $0$ & $-$ & $0$ & $0$ & $0$ \\
\bottomrule
\end{tabular}
\begin{tablenotes}
\small
\item \textit{Notes:} ``$-$'' for the gender gap means the gap narrows (women gain relative to men). The observed pattern matches information equalization and is inconsistent with employer commitment or costly adjustment operating in isolation.
\end{tablenotes}
\end{threeparttable}
\end{table}

\section{Institutional Background}
\label{sec:background}

Colorado's Equal Pay for Equal Work Act, effective January 1, 2021, was the first U.S.\ law requiring salary ranges in job postings. Seven states followed through 2024. Table~\ref{tab:timing} (Appendix) summarizes adoption timing and employer thresholds.

These laws differ fundamentally from earlier transparency interventions. ``Right-to-ask'' laws, studied by \citet{cullen2023pay}, permit employees to inquire about pay ranges but impose no affirmative disclosure obligation---a softer treatment that depends on workers' willingness to ask. Denmark's reporting mandate \citep{bennedsen2022firms} requires firms to report aggregate pay statistics internally, reaching only current employees. Salary transparency laws operate \textit{ex ante}: job seekers observe posted ranges before applying, shifting the information set available during the matching process itself. This distinction matters for mechanism interpretation. If information deficits primarily affect the \textit{search} and \textit{matching} phase---determining which jobs workers apply to and what opening offers they accept---then ex ante disclosure should be more powerful than ex post reporting.

The laws share a core requirement---employers must include compensation ranges in job postings---but vary usefully across several dimensions. \textit{Employer thresholds} range from all employers (Colorado, Connecticut, Nevada, Rhode Island) to 15+ employees (California, Washington), 4+ (New York), and 50+ (Hawaii). \textit{Disclosure specificity} varies from ``good faith estimates'' (Nevada) to precise pay scales with wage range and benefits description (California, Washington). \textit{Enforcement mechanisms} also differ: Colorado imposed fines up to \$10,000 per violation, while Connecticut initially relied on complaints to the Department of Labor. \textit{Coverage scope} varies: some states require disclosure only for postings accessible to applicants in that state, while others (notably Colorado) initially applied to all remote-eligible positions, effectively extending reach beyond state borders.

\textit{Timing} provides identifying variation: Colorado's 2021 implementation gives the longest post-treatment window (four income years), while the 2023 clustering of California, Washington, and Rhode Island creates a large treatment cohort. The 2024 cohort (New York, Hawaii) adds two more states with only one post-treatment year. Three additional states---Illinois, Maryland, Minnesota---enacted laws effective in 2025, outside the analysis window.

\begin{figure}[H]
\centering
\includegraphics[width=0.9\textwidth]{figures/fig1_policy_map.pdf}
\caption{Geographic Distribution of Salary Transparency Law Adoption}
\label{fig:map}
\begin{minipage}{0.9\textwidth}
\footnotesize
\textit{Notes:} Timing of salary transparency law effective dates. Darker shading indicates earlier adoption. Gray states have not adopted requirements as of 2024.
\end{minipage}
\end{figure}

\section{Data}
\label{sec:data}

I combine two datasets that observe the labor market from complementary vantage points. The CPS ASEC captures individual workers' wages, demographics, and occupational characteristics. The QWI captures employers' aggregate earnings, hiring, and separation decisions. Neither alone can distinguish between the theoretical channels; together, they provide a comprehensive view.

\subsection{CPS ASEC: Worker-Side Microdata}

The CPS ASEC provides individual-level data on income and employment for $\sim$95,000 households each March. I use surveys from 2015--2025 (income years 2014--2024),\footnote{The 2025 CPS ASEC was released by the Census Bureau in September 2025 and reports income for calendar year 2024.} restrict to working-age adults (25--64) employed as wage and salary workers with positive wage income and reasonable hours (10+ hours/week, 13+ weeks/year), and exclude imputed wages. The final sample contains 614,625 person-year observations across 51 states and 11 years. The primary outcome is log hourly wage. Treatment status is defined based on the first full calendar year affected by each law.

Controls include age (five-year groups), education (five categories), race/ethnicity, marital status, metropolitan residence, occupation (23 major groups), and industry (14 sectors). A ``high-bargaining occupation'' indicator flags management, business/financial, computer/mathematical, engineering, legal, and healthcare practitioner occupations.

\subsection{Quarterly Workforce Indicators: Employer-Side Administrative Data}

The QWI, produced by the Census Bureau's LEHD program, provide quarterly establishment-level statistics from state unemployment insurance records covering $\sim$95\% of private employment \citep{abowd2009lehd}. I construct a state-quarter panel spanning 2012Q1--2024Q4 (52 quarters, 51 states). The theoretical maximum is $51 \times 52 = 2{,}652$ cells; Census disclosure thresholds yield 2,603 non-suppressed observations (1.9\% suppression rate).

Outcomes include log average quarterly earnings, the gender earnings gap, hiring rates, separation rates, and net job creation rates. Treatment timing follows a ``first full quarter'' convention.\footnote{For example, Connecticut's law took effect October 1, 2021. The first full treatment quarter is 2022Q1. Colorado (effective January 1, 2021) is coded treated from 2021Q1.} I disaggregate by sex (male, female, total) and industry (retail, accommodation, finance, professional services).

\subsection{Complementarity}

Table~\ref{tab:data_comparison} highlights the complementarity. The CPS captures \textit{individual}-level variation with rich demographic controls, enabling triple-difference identification. The QWI capture \textit{establishment}-level flows the CPS cannot measure---hiring, separations, job creation---from administrative records free of survey measurement error. Where the datasets overlap, agreement provides convergent validity; where they diverge, each contributes unique information.

\begin{table}[H]
\centering
\caption{Dataset Comparison}
\label{tab:data_comparison}
\begin{tabular}{lcc}
\toprule
Feature & CPS ASEC & QWI \\
\midrule
Source & Household survey & Administrative records \\
Unit & Individual worker & State-quarter aggregate \\
Frequency & Annual & Quarterly \\
Coverage & $\sim$95K households/year & $\sim$95\% private employment \\
Wage measure & Hourly (computed) & Monthly earnings (reported) \\
Demographic controls & Yes (rich) & No (sex, age group only) \\
Labor market flows & No & Yes (hires, separations, creation) \\
Industry detail & 14 sectors & NAICS 2-digit \\
Observations & 614,625 person-years & 2,603 state-quarters \\
\bottomrule
\end{tabular}
\end{table}

\subsection{Summary Statistics}

Table~\ref{tab:balance} (Appendix) presents CPS pre-treatment balance. Treated states have moderately higher wages (\$28 vs.\ \$25), more education, and more metropolitan residents---differences absorbed by state fixed effects. Gender composition is similar (47\% vs.\ 46\% female).

Table~\ref{tab:qwi_summary} presents QWI summary statistics. Treated states have higher quarterly earnings (\$5,185 vs.\ \$4,650) and slightly lower hiring and separation rates. The pre-treatment gender gap is 0.43 log points in treated states versus 0.45 in control states.

\begin{table}[htbp]
\centering
\caption{QWI Summary Statistics}
\label{tab:qwi_summary}
\begin{tabular}{lcc}
\toprule
 & Treated States & Control States \\
\midrule
\multicolumn{3}{l}{\textit{Panel A: Panel Dimensions}} \\[3pt]
States & 8 & 43 \\
Quarters & 52 & 52 \\
State-Quarter Observations & 352 & 2,187 \\
\addlinespace
\multicolumn{3}{l}{\textit{Panel B: Pre-Treatment Means}} \\[3pt]
Average Quarterly Earnings (\$) & 5,185 & 4,650 \\
Average Employment & 3,841,336 & 2,139,128 \\
Hiring Rate & 0.181 & 0.191 \\
Separation Rate & 0.177 & 0.187 \\
Net Job Creation Rate & -0.043 & -0.044 \\
Gender Earnings Gap (M$-$F) & 0.426 & 0.451 \\
\bottomrule
\end{tabular}
\begin{minipage}{0.92\textwidth}
\footnotesize
\textit{Notes:} Data from Census QWI. Panel A reports \textit{pre-treatment} state-quarter observations only; the full panel contains 2,603 observations. Hiring rate = hires/employment; separation rate = separations/employment; net job creation rate = (hires $-$ separations)/employment.
\end{minipage}
\end{table}

\section{Empirical Strategy}
\label{sec:strategy}

\subsection{Identification}

I exploit staggered adoption under the parallel trends assumption: absent treatment, wage and earnings trends in treated states would have paralleled control states. This assumption is fundamentally untestable post-treatment but is supported by pre-trend analysis in both datasets.

\subsection{CPS Estimation}

For CPS microdata, I employ the \citet{callaway2021difference} estimator, which computes group-time average treatment effects using only never-treated controls, avoiding the biases of standard TWFE under staggered treatment \citep{goodman2021difference, dechaisemartin2020twoway, roth2023whats}. The doubly-robust variant combines outcome regression with inverse-probability weighting. I also report TWFE and \citet{sun2021estimating} estimates.

The gender triple-difference (DDD) specification is:
\begin{equation}
Y_{ist} = \beta_1 D_{st} + \beta_2 D_{st} \times Female_i + \gamma Female_i + \alpha_s + \delta_t + X_{ist}'\theta + \varepsilon_{ist}
\end{equation}
where $D_{st}$ indicates treatment, $\alpha_s$ are state fixed effects, $\delta_t$ are year fixed effects, and $X_{ist}$ are controls. I also estimate specifications with state$\times$year fixed effects $\alpha_{st}$, which identify $\beta_2$ purely from within-state-year gender differences.

Standard errors are clustered at the state level (51 clusters). All CPS regressions use ASECWT survey weights from the CPS ASEC \citep{flood2023ipums}. Given eight treated states, I supplement with Fisher randomization inference (5,000 permutations) and leave-one-treated-state-out analysis.

\subsection{QWI Estimation}

For the QWI panel, I apply Callaway-Sant'Anna with quarterly treatment timing and ``first full quarter'' conventions (see Table~\ref{tab:timing}). Quarterly data provide two advantages over the CPS's annual frequency: more precise event timing and more pre-treatment periods for testing parallel trends.

The aggregate specification estimates the ATT on log average quarterly earnings, using never-treated states as controls. The sex-disaggregated DDD stacks male and female earnings within each state-quarter and interacts treatment with a female indicator:
\begin{equation}
\log(EarnS_{sgt}) = \beta_2 D_{st} \times Female_g + \alpha_{st} + \gamma_g + \varepsilon_{sgt}
\end{equation}
where $\alpha_{st}$ are state$\times$quarter fixed effects and $\gamma_g$ is a sex indicator. This identifies $\beta_2$ from within-state-quarter changes in the gender gap attributable to transparency. The state$\times$quarter fixed effects absorb all aggregate variation---macroeconomic shocks, seasonal patterns, state-specific trends---isolating the gender-specific treatment effect.

Standard errors are clustered at the state level (51 clusters). QWI aggregates are employment-weighted by construction through the LEHD administrative records. With 51 clusters, asymptotic inference is substantially more reliable than for the CPS's 8 treated states \citep{bertrand2004much, cameron2008bootstrap}. This provides an independent and better-powered test of the gender gap finding.

\subsection{Hypothesis Hierarchy}

Three primary hypotheses: (1) no aggregate wage effect, (2) gender gap narrowing, (3) no change in dynamism. These map directly to Table~\ref{tab:predictions}. Industry-level and subgroup analyses are exploratory.\footnote{For the three primary hypotheses, the two significant results---null aggregate effect and gender DDD ($p < 0.001$ in both datasets)---survive Bonferroni correction.}

\subsection{Industry Heterogeneity}

The QWI's industry disaggregation enables a direct test of the bargaining-intensity mechanism. Finance and professional services serve as ``high-bargaining'' industries; retail and accommodation as ``low-bargaining'' comparisons.

\section{Results}
\label{sec:results}

\subsection{Pre-Trends and Visual Evidence}

Credible causal inference requires that treated and control states followed parallel trends before policy adoption. Both datasets provide compelling support for this assumption.

Figure~\ref{fig:cps_trends} plots CPS trends. Panel (a) shows mean hourly wages tracking together for treated and control states across seven pre-treatment years---no divergence after 2021. Panel (b) shows the gender wage gap: treated states' gap narrows visibly post-treatment while control states' gap remains stable. The parallel pre-trends in both panels support the identification assumption.

The Callaway-Sant'Anna event study (Table~\ref{tab:event_study}, Appendix) provides a formal test. Of four pre-treatment coefficients ($t = -5$ through $t = -2$, with $t = -1$ as the omitted reference period), only one---$t = -2$: $-0.013$, $p < 0.10$---reaches marginal significance. With four tests at $\alpha = 0.10$, the expected false rejection count is 0.4, making a single marginal rejection unremarkable. The magnitude ($-0.013$) is small relative to the gender DDD of interest (0.040--0.056), and the sign is negative---opposite what a spurious upward trend would produce. HonestDiD sensitivity analysis (Section~\ref{sec:robustness}) confirms: under exact parallel trends ($M = 0$), the gender gap 95\% CI is $[0.043, 0.100]$, firmly excluding zero.

\begin{figure}[H]
\centering
\includegraphics[width=\textwidth]{figures/fig2_cps_combined.pdf}
\caption{CPS Trends: Treated vs.\ Control States}
\label{fig:cps_trends}
\begin{minipage}{0.95\textwidth}
\footnotesize
\textit{Notes:} (a) Mean hourly wages from CPS ASEC. Shaded areas show 95\% CIs. (b) Gender wage gap as percentage of male wages. Dashed vertical line marks first treatment (Colorado, 2021). Figure plots income years 2014--2023 for visual clarity; all regressions use the full sample including income year 2024 (N = 614,625).
\end{minipage}
\end{figure}

Figure~\ref{fig:qwi_trends} shows the same story in administrative data. Panel (a) plots QWI quarterly earnings---treated and control states follow nearly identical trajectories across 52 quarters, confirming the null aggregate effect with much finer temporal resolution. Panel (b) plots the gender earnings gap: treated states' gap narrows relative to control after 2021. The quarterly frequency provides a more powerful pre-trends test than the CPS's annual data.

\begin{figure}[H]
\centering
\includegraphics[width=\textwidth]{figures/fig3_qwi_combined.pdf}
\caption{QWI Trends: Treated vs.\ Control States}
\label{fig:qwi_trends}
\begin{minipage}{0.95\textwidth}
\footnotesize
\textit{Notes:} (a) Log quarterly earnings from QWI administrative data, 2012Q1--2024Q4. (b) Male--female log earnings gap. Dashed vertical line marks first treatment (Colorado, 2021Q1). The sawtooth pattern reflects seasonal variation; quarter fixed effects absorb this in all regressions.
\end{minipage}
\end{figure}

\subsection{Main Result 1: No Aggregate Wage Effect}

Transparency does not move average wages. Table~\ref{tab:main} reports CPS TWFE estimates across four specifications; none yields a significant coefficient. The Callaway-Sant'Anna ATT is $-0.0038$ (SE $= 0.0064$). Fisher randomization inference confirms the null ($p = 0.717$).

The QWI independently confirms: C-S ATT on log quarterly earnings is $-0.001$ (SE $= 0.020$), and TWFE yields $+0.030$ (SE $= 0.022$)---both insignificant (Table~\ref{tab:qwi_main}). Two datasets measuring different populations at different frequencies agree: transparency does not affect average wages.

\begin{table}[H]
\centering
\caption{CPS: Effect of Salary Transparency Laws on Log Wages}
\label{tab:main}
\begin{threeparttable}
\begin{tabular}{lcccc}
\toprule
& (1) & (2) & (3) & (4) \\
& State-Year & Individual & + Occ/Ind FE & + Demographics \\
\midrule
Treated $\times$ Post & 0.005 & 0.014* & 0.005 & 0.008 \\
& (0.011) & (0.008) & (0.006) & (0.006) \\
\midrule
State FE & Yes & Yes & Yes & Yes \\
Year FE & Yes & Yes & Yes & Yes \\
Occupation FE & No & No & Yes & Yes \\
Industry FE & No & No & Yes & Yes \\
Demographics & No & No & No & Yes \\
\midrule
Observations & 561 & 614,625 & 614,625 & 614,625 \\
R-squared & 0.972 & 0.058 & 0.299 & 0.382 \\
\bottomrule
\end{tabular}
\begin{tablenotes}
\small
\item \textit{Notes:} TWFE estimates with state and year fixed effects. Standard errors clustered at state level (51 clusters). Column (1) is a state-year level regression (N = 561 state-year cells); Columns (2)--(4) are individual-level regressions (N = 614,625). R-squared in Column (1) reflects state fixed effects absorbing most state-year variation. The preferred C-S ATT ($-0.0038$, SE $= 0.0064$; Fisher $p = 0.717$) is in Table~\ref{tab:robustness_cps}. * $p<0.10$, ** $p<0.05$, *** $p<0.01$.
\end{tablenotes}
\end{threeparttable}
\end{table}

\begin{table}[htbp]
\centering
\caption{QWI Main Results: Earnings and Gender Gap}
\label{tab:qwi_main}
\begin{tabular}{lcc}
\toprule
 & C-S ATT & TWFE \\
\midrule
\multicolumn{3}{l}{\textit{Panel A: Log Quarterly Earnings}} \\[3pt]
Treated $\times$ Post & -0.0010 & 0.0295 \\
 & (0.0199) & (0.0224) \\
N & 2,603 & 2,603 \\
\addlinespace
\multicolumn{3}{l}{\textit{Panel B: Gender Earnings Gap (Male $-$ Female)}} \\[3pt]
Treated $\times$ Post & 0.0051 & 0.0217$^{**}$ \\
 & (0.0122) & (0.0109) \\
N & 2,603 & 2,603 \\
\addlinespace
\multicolumn{3}{l}{\textit{Panel C: DDD (Treated $\times$ Post $\times$ Female)}} \\[3pt]
Treated $\times$ Post $\times$ Female & 0.0605$^{***}$ & --- \\
 & (0.0151) & \\
\midrule
State FE & Yes & Yes \\
Quarter FE & Yes & Yes \\
Clustering & State & State \\
\bottomrule
\end{tabular}
\begin{minipage}{0.92\textwidth}
\footnotesize
\textit{Notes:} Standard errors clustered at state level (51 clusters). Panel A: effect on log average quarterly earnings. Panel B: aggregate gender gap (attenuated by composition changes). Panel C: sex-disaggregated DDD with state$\times$quarter FE isolating within-cell gender effect; TWFE column omitted because state$\times$quarter FE fully absorb the treatment indicator---the DDD is identified only in the sex-disaggregated Callaway-Sant'Anna specification. Panel C uses a sex-disaggregated stacked panel (male and female observations within each state-quarter). Panel B's null and Panel C's significant result are not contradictory: the aggregate gap measure is attenuated by composition, while the DDD isolates the within-cell gender effect. $^{*}$ $p<0.10$, $^{**}$ $p<0.05$, $^{***}$ $p<0.01$.
\end{minipage}
\end{table}

\subsection{Main Result 2: Gender Gap Narrows}

Table~\ref{tab:gender} tells the central story. The coefficient on Treated $\times$ Post $\times$ Female ranges from $+0.040$ to $+0.056$ across four specifications, always significant at 1\%. The most demanding specification---state$\times$year fixed effects in Column (4)---yields $+0.043$ ($p < 0.01$), identifying the gender effect purely from within-state-year variation.

\begin{table}[H]
\centering
\caption{CPS Triple-Difference: Effect on Gender Wage Gap}
\label{tab:gender}
\begin{threeparttable}
\begin{tabular}{lcccc}
\toprule
& (1) & (2) & (3) & (4) \\
& Basic & + Occ FE & + Controls & State$\times$Year FE \\
\midrule
Treated $\times$ Post & $-0.007$ & $-0.018$** & $-0.010$* & --- \\
& (0.008) & (0.007) & (0.006) & (absorbed) \\
Treated $\times$ Post $\times$ Female & 0.049*** & 0.056*** & 0.040*** & 0.043*** \\
& (0.008) & (0.008) & (0.008) & (0.008) \\
\midrule
State \& Year FE & Yes & Yes & Yes & No \\
State $\times$ Year FE & No & No & No & Yes \\
Occupation FE & No & Yes & Yes & Yes \\
Demographics & No & No & Yes & Yes \\
\midrule
Observations & 614,625 & 614,625 & 614,625 & 614,625 \\
\bottomrule
\end{tabular}
\begin{tablenotes}
\small
\item \textit{Notes:} Standard errors clustered at state level (51 clusters). Positive coefficient means women's wages rose relative to men's. In Column (4), the main effect is absorbed by state$\times$year FE. * $p<0.10$, ** $p<0.05$, *** $p<0.01$.
\end{tablenotes}
\end{threeparttable}
\end{table}

The QWI administrative data independently confirm this finding. Panel C of Table~\ref{tab:qwi_main} reports a DDD of $+0.0605$ (SE $= 0.015$, $p < 0.001$): women's quarterly earnings rose 6.1 percentage points relative to men's within state-quarter cells. An important distinction separates Panel B's null result from Panel C's significance. Panel B measures the \textit{aggregate} gender gap (male minus female average earnings at the state-quarter level)---a measure attenuated by composition changes. Panel C uses the more powerful sex-disaggregated approach: male and female earnings are stacked within each state-quarter, and state$\times$quarter fixed effects absorb all aggregate variation. The DDD isolates the within-cell gender-specific treatment effect, analogous to comparing women's and men's earnings within the same state-quarter.

The QWI point estimate is somewhat larger than the CPS (6.1 vs.\ 4.0--5.6 pp), as expected: QWI measures average earnings per worker (including composition effects from changing employment patterns), while the CPS controls for individual demographics. The concordance in sign, statistical significance, and approximate magnitude across a household survey and employer administrative records---measuring different populations at different frequencies with completely different sources of measurement error---is the paper's strongest evidence for the gender gap finding.

Figure~\ref{fig:gender_es} provides a visual complement. Separate CPS event studies for men and women show comparable pre-treatment trends. Post-treatment, female wages increase relative to the pre-treatment trend while male wages remain flat, with the gap emerging by $t = 0$ and widening through $t+2$---directly visualizing the gender convergence.

\begin{figure}[H]
\centering
\includegraphics[width=0.85\textwidth]{figures/fig5_event_study_gender.pdf}
\caption{CPS Event Study by Gender}
\label{fig:gender_es}
\begin{minipage}{0.85\textwidth}
\footnotesize
\textit{Notes:} Separate Callaway-Sant'Anna event-study estimates for men (blue) and women (pink). Pre-treatment trends are comparable. Post-treatment, female wages increase while male wages remain flat, producing the convergence that drives the gender gap narrowing.
\end{minipage}
\end{figure}

\subsection{Main Result 3: No Labor Market Disruption}

Table~\ref{tab:qwi_dynamism} presents QWI estimates for five flow variables. None responds to transparency. Hiring rate: $-0.001$ ($p = 0.80$). Separation rate: $-0.0001$ ($p = 0.98$). Net job creation: $-0.001$ ($p = 0.53$). Figure~\ref{fig:dynamism} visualizes the null results.

\begin{table}[htbp]
\centering
\caption{QWI Labor Market Dynamism Results}
\label{tab:qwi_dynamism}
\begin{tabular}{lccccc}
\toprule
 & \multicolumn{2}{c}{TWFE} & & \multicolumn{2}{c}{C-S ATT} \\
\cmidrule{2-3} \cmidrule{5-6}
Outcome & Coeff. & SE & & ATT & SE \\
\midrule
Hiring Rate & -0.0009 & (0.0036) & & 0.0009 & (0.0033) \\
Separation Rate & -0.0001 & (0.0030) & & 0.0041 & (0.0033) \\
Log Hires & 0.0090 & (0.0232) & & --- & --- \\
Log Separations & 0.0138 & (0.0230) & & --- & --- \\
Net Job Creation Rate & -0.0011 & (0.0017) & & --- & --- \\
\midrule
N & \multicolumn{2}{c}{2,603} & & \multicolumn{2}{c}{2,603} \\
State FE & \multicolumn{2}{c}{Yes} & & \multicolumn{2}{c}{Yes} \\
Quarter FE & \multicolumn{2}{c}{Yes} & & \multicolumn{2}{c}{Yes} \\
Clustering & \multicolumn{2}{c}{State} & & \multicolumn{2}{c}{State} \\
\bottomrule
\end{tabular}
\begin{minipage}{0.92\textwidth}
\footnotesize
\textit{Notes:} Standard errors clustered at state level. Hiring rate = hires/employment; separation rate = separations/employment. $^{*}$ $p<0.10$, $^{**}$ $p<0.05$, $^{***}$ $p<0.01$.
\end{minipage}
\end{table}

This null result is economically informative, not merely a failure to reject. It rules out costly adjustment---if transparency triggered reallocation as firms restructure pay bands, hiring and separations would spike as workers sort into new matches. It rules out employer commitment as the dominant channel---if firms rigidly adhere to posted ranges and refuse negotiation, hiring should become less responsive to market conditions as posted ranges lag actual market clearing wages.

The precision of the estimates matters. The 95\% confidence interval for hiring rate ($[-0.008, 0.006]$) rules out effects larger than 0.8 percentage points---well below the magnitudes observed during actual labor market disruptions. For context, the Great Recession reduced hiring rates by approximately 3--4 percentage points \citep{hallkrueger2012evidence}. Transparency effects on flows, if they exist, are an order of magnitude smaller.

The most parsimonious interpretation is that labor markets absorb salary transparency without observable disruption. This is consistent with \citet{hallkrueger2012evidence}: if many employers already posted wages de facto, mandatory posting formalizes existing practice without creating new frictions.

\begin{figure}[H]
\centering
\includegraphics[width=0.85\textwidth]{figures/fig_qwi_dynamism.pdf}
\caption{Labor Market Dynamism: DiD Coefficient Plot}
\label{fig:dynamism}
\begin{minipage}{0.85\textwidth}
\footnotesize
\textit{Notes:} TWFE estimates of transparency effects on five flow variables from QWI data. Point estimates and 95\% CIs. All coefficients are small and statistically insignificant.
\end{minipage}
\end{figure}

\subsection{Industry Heterogeneity}

Table~\ref{tab:qwi_industry} reports QWI earnings and gender gap effects by NAICS sector. Earnings effects are insignificant across all industries. In sex-disaggregated DDD specifications, high-bargaining industries (finance, professional services) show a gender DDD of 8.8 pp (SE $= 2.2$ pp); low-bargaining industries (retail, accommodation) show 7.0 pp (SE $= 1.1$ pp). Effects are present even in low-bargaining industries, suggesting information deficits extend beyond occupations with explicit negotiation \citep{kline2021firm}.

\begin{table}[htbp]
\centering
\caption{QWI Industry Heterogeneity: Earnings and Gender Gap Effects}
\label{tab:qwi_industry}
\begin{tabular}{lcccc}
\toprule
 & \multicolumn{2}{c}{Log Earnings} & \multicolumn{2}{c}{Gender Earnings Gap} \\
\cmidrule(lr){2-3} \cmidrule(lr){4-5}
Industry & Coeff. & SE & Coeff. & SE \\
\midrule
Retail Trade & -0.0080 & (0.0240) & -0.0079 & (0.0155) \\
Accommodation \& Food & 0.0206 & (0.0161) & 0.0083 & (0.0053) \\
Finance \& Insurance & 0.0078 & (0.0358) & 0.0301 & (0.0294) \\
Professional Services & 0.0314 & (0.0234) & 0.0071 & (0.0079) \\
\midrule
State FE & \multicolumn{2}{c}{Yes} & \multicolumn{2}{c}{Yes} \\
Quarter FE & \multicolumn{2}{c}{Yes} & \multicolumn{2}{c}{Yes} \\
Clustering & \multicolumn{2}{c}{State} & \multicolumn{2}{c}{State} \\
\bottomrule
\end{tabular}
\begin{minipage}{0.92\textwidth}
\footnotesize
\textit{Notes:} Each cell is a separate TWFE regression within the indicated NAICS sector. Standard errors clustered at state level. $^{*}$ $p<0.10$, $^{**}$ $p<0.05$, $^{***}$ $p<0.01$.
\end{minipage}
\end{table}

Figure~\ref{fig:industry} visualizes the industry heterogeneity.

\begin{figure}[H]
\centering
\includegraphics[width=0.85\textwidth]{figures/fig_qwi_industry.pdf}
\caption{Industry Heterogeneity in QWI Earnings Effects}
\label{fig:industry}
\begin{minipage}{0.85\textwidth}
\footnotesize
\textit{Notes:} TWFE estimates of transparency effects on log earnings by NAICS sector. Colors distinguish high-bargaining (finance, professional services) from low-bargaining (retail, accommodation) industries.
\end{minipage}
\end{figure}

\subsection{Cross-Dataset Comparison}

Table~\ref{tab:cross_dataset} presents CPS and QWI estimates side by side. Both datasets find null aggregate effects (CPS: $-0.004$; QWI: $-0.001$). For the gender DDD, the CPS yields $+0.040$ (SE $= 0.008$) and QWI yields $+0.0605$ (SE $= 0.015$)---both highly significant with the same sign. The QWI's larger point estimate likely reflects measurement of average earnings (including composition effects) versus the CPS's individual-level controls.

\begin{table}[htbp]
\centering
\caption{Cross-Dataset Comparison: CPS vs.\ QWI Estimates}
\label{tab:cross_dataset}
\begin{tabular}{lcccc}
\toprule
 & \multicolumn{2}{c}{CPS ASEC} & \multicolumn{2}{c}{QWI} \\
\cmidrule(lr){2-3} \cmidrule(lr){4-5}
 & C-S ATT & TWFE & C-S ATT & TWFE \\
\midrule
\multicolumn{5}{l}{\textit{Panel A: Aggregate Wage/Earnings Effect}} \\[3pt]
Treated $\times$ Post & -0.0038 & --- & -0.0010 & 0.0295 \\
 & (0.0064) &  & (0.0199) & (0.0224) \\
\addlinespace
\multicolumn{5}{l}{\textit{Panel B: Gender DDD (Treated $\times$ Post $\times$ Female)}} \\[3pt]
DDD coefficient & 0.0402$^{***}$ & --- & 0.0605$^{***}$ & --- \\
 & (0.0080) & & (0.0151) & \\
\midrule
Unit of observation & \multicolumn{2}{c}{Individual} & \multicolumn{2}{c}{State-Quarter} \\
Outcome variable & \multicolumn{2}{c}{Log hourly wage} & \multicolumn{2}{c}{Log quarterly earnings} \\
Frequency & \multicolumn{2}{c}{Annual} & \multicolumn{2}{c}{Quarterly} \\
Source & \multicolumn{2}{c}{CPS ASEC} & \multicolumn{2}{c}{LEHD/QWI} \\
\bottomrule
\end{tabular}
\begin{minipage}{0.95\textwidth}
\footnotesize
\textit{Notes:} Side-by-side comparison. CPS DDD from individual-level microdata; QWI DDD from sex-disaggregated panel with state$\times$quarter FE. TWFE columns show ``---'' for the Gender DDD because state$\times$quarter FE fully absorb the treatment indicator; the DDD is identified only in the Callaway-Sant'Anna specification. Standard errors clustered at state level. $^{*}$ $p<0.10$, $^{**}$ $p<0.05$, $^{***}$ $p<0.01$.
\end{minipage}
\end{table}

\section{Robustness and Inference}
\label{sec:robustness}

\subsection{CPS Robustness}

Table~\ref{tab:robustness_cps} presents robustness checks. The C-S ATT is stable across alternative estimators (Sun-Abraham: $-0.0002$), control groups (not-yet-treated: $-0.003$), sample restrictions (full-time, college, border states), and upper-distribution tests. Lee bounds for the gender DDD (lower: 0.042; upper: 0.050) confirm robustness to sample selection. HonestDiD sensitivity excludes zero under exact parallel trends ($M = 0$: CI $= [0.043, 0.100]$).

\begin{table}[H]
\centering
\caption{CPS Robustness of Main Results}
\label{tab:robustness_cps}
\begin{threeparttable}
\begin{tabular}{lccc}
\toprule
Specification & ATT & SE & 95\% CI \\
\midrule
Main (C-S, never-treated) & $-0.0038$ & 0.0064 & [$-0.016$, 0.009] \\
Sun-Abraham estimator & $-0.0002$ & 0.0076 & [$-0.015$, 0.015] \\
C-S, not-yet-treated controls & $-0.0030$ & 0.0068 & [$-0.016$, 0.010] \\
Excluding border states & $-0.0062$ & 0.0083 & [$-0.023$, 0.010] \\
Full-time workers only & $-0.0034$ & 0.0077 & [$-0.019$, 0.012] \\
College-educated only & $-0.0132$ & 0.0089 & [$-0.031$, 0.004] \\
Non-college only & 0.0061 & 0.0142 & [$-0.022$, 0.034] \\
Individual-level TWFE & 0.0103 & 0.0058 & [$-0.001$, 0.022] \\
Upper 75\% wage distribution & 0.0001 & 0.0071 & [$-0.014$, 0.014] \\
\bottomrule
\end{tabular}
\begin{tablenotes}
\small
\item \textit{Notes:} All specifications estimate the effect on log hourly wages. Standard errors clustered at state level.
\end{tablenotes}
\end{threeparttable}
\end{table}

\subsection{Design-Based Inference}
\label{sec:design_inference}

With eight treated states, the reliability of asymptotic inference is a first-order concern \citep{conley2011inference, cameron2008bootstrap, ferman2019inference}. Table~\ref{tab:alt_inference} reports Fisher randomization results from 5,000 permutations \citep{imai2021randomization, athey2022design}.

For the aggregate ATT, asymptotic and design-based inference agree: clearly insignificant (asymptotic $p = 0.556$; permutation $p = 0.717$). For the gender DDD, they diverge: asymptotic $p < 0.001$ versus permutation $p = 0.154$. This reflects the fundamental limitation of design-based inference with eight treated clusters.

Three considerations mitigate this. First, the QWI provides an independent test. The QWI DDD ($+0.0605$, $p < 0.001$) uses 51 clusters---well above the threshold at which asymptotic inference is reliable. When two independent datasets with different measurement properties produce consistent estimates, the probability that both are spurious is substantially lower than for either alone. Second, all eight leave-one-out estimates remain positive ($[0.042, 0.054]$). Third, HonestDiD excludes zero under exact parallel trends.

\begin{table}[htbp]
\centering
\caption{Alternative Inference Methods}
\label{tab:alt_inference}
\begin{tabular}{lccccc}
\toprule
 & Estimate & Asymptotic & Asymptotic & Permutation & LOTO \\
 &  & SE & $p$ & $p$ & Range \\
\midrule
CPS Aggregate ATT & $-$0.0038 & 0.0065 & 0.556 & 0.717 & [$-$0.006, 0.001] \\
CPS Gender DDD ($\beta_2$) & 0.0402 & 0.0080 & 0.000 & 0.154 & [0.042, 0.054] \\
QWI Gender DDD ($\beta_2$) & 0.0605 & 0.0151 & 0.000 & n.a. & n.a. \\
\bottomrule
\end{tabular}
\begin{minipage}{0.95\textwidth}
\footnotesize
\textit{Notes:} Permutation $p$-values from 5,000 Fisher randomizations. LOTO range from leave-one-treated-state-out samples. ``n.a.'' for QWI: 51 clusters provide adequate asymptotic inference.
\end{minipage}
\end{table}

\subsection{Leave-One-State-Out Analysis}

All eight leave-out gender DDD estimates remain positive ($[0.042, 0.054]$). No single state drives the result.

\begin{figure}[H]
\centering
\includegraphics[width=0.85\textwidth]{figures/fig11_loto_ddd.pdf}
\caption{Leave-One-Treated-State-Out: Gender DDD}
\label{fig:loto_ddd}
\begin{minipage}{0.85\textwidth}
\footnotesize
\textit{Notes:} CPS gender DDD when each treated state is dropped. All estimates remain positive. Horizontal line marks the full-sample estimate.
\end{minipage}
\end{figure}

\subsection{Additional CPS Robustness}

\textbf{Placebo tests.} A placebo treatment dated two years early yields a null ATT (0.003, SE $= 0.009$). A placebo on non-wage income also shows no effect ($-0.002$, SE $= 0.015$).

\textbf{Composition tests.} DiD regressions on workforce composition show no significant changes in percent female, college-educated, mean age, or full-time status. The share in high-bargaining occupations shifts modestly ($+0.020$, $p = 0.017$); Lee bounds accounting for this shift remain positive (lower: 0.042; upper: 0.050).

\textbf{HonestDiD sensitivity.} Under exact parallel trends ($M = 0$), the gender gap 95\% CI is $[0.043, 0.100]$, excluding zero. Bounds widen rapidly for $M > 0$ due to noise with eight treated states.

\textbf{Synthetic DiD.} Applied to Colorado following \citet{arkhangelsky2021synthetic}, SDID yields an aggregate estimate of essentially zero (0.0003), consistent with C-S ATT.

\textbf{Excluding NY and HI.} Dropping the 2024 cohort yields a gender DDD of 0.052 (SE $= 0.005$), slightly larger than the full-sample estimate.

\section{Discussion}
\label{sec:discussion}

\subsection{Mechanism Identification}

The three findings jointly discriminate between theoretical channels (Table~\ref{tab:predictions}). \textit{Information equalization} predicts precisely the observed pattern: aggregate wages unchanged (gains to women offset losses to men); gender gap narrows (women's information deficit shrinks); flows unaffected (adjustment operates through prices, not quantities). \textit{Employer commitment} is inconsistent---it predicts wage compression (none found) and reduced hiring responsiveness (unchanged). \textit{Costly adjustment} is rejected---all five flow variables are precisely estimated zeros.

Employer commitment should compress the \textit{entire} wage distribution---reducing wages for all workers, not selectively raising women's wages. The aggregate null (CPS ATT $= -0.004$, QWI ATT $= -0.001$) is inconsistent with this channel operating at economically meaningful magnitudes. Costly adjustment should produce observable reallocation: if transparency changes match quality, workers and firms renegotiate or separate. The QWI flow nulls rule this out. What remains is information equalization: transparency corrects asymmetries that disproportionately disadvantaged women, producing distributional effects (gender gap narrows) without aggregate effects (the pie stays the same size) or allocative effects (no disruption to matching).

The CPS individual-level analysis (controlling for demographics) produces consistent estimates with the QWI aggregate DDD, providing indirect evidence that the QWI effect operates primarily through wage changes rather than compositional shifts. This is important: if the QWI gender effect were driven entirely by women sorting into higher-paying firms (rather than being paid more within firms), the individual-level CPS estimates---which control for occupation and industry---should be substantially smaller. They are not.

\subsection{Magnitude and Economic Significance}

The CPS DDD of 4--6 pp represents roughly half the residual gender gap after controlling for occupation and experience \citep{blau2017gender}---approximately 17--30\% of the total raw gap. For comparison, Denmark's reporting mandate narrowed gaps by $\sim$2 pp \citep{bennedsen2022firms}, Baker et al.'s firm-level transparency by $\sim$3 pp \citep{baker2023pay}, and the UK's disclosure by $\sim$2 pp \citep{blundell2022wage}. Job-posting requirements appear among the most potent transparency interventions, likely because they reach workers \textit{ex ante}---before employment begins.

\subsection{Limitations}

Several limitations warrant discussion.

\textit{Short post-treatment window.} Most treated states have 1--3 post-treatment years; only Colorado provides four. Long-run effects may differ if employer compliance evolves, if workers gradually learn to use posted information, or if firms adjust job architecture to circumvent transparency requirements (e.g., posting wider ranges). As Illinois, Maryland, and Minnesota enter the post-treatment window and existing states accumulate exposure, future work can test for dynamic effects.

\textit{Ecological inference in the QWI.} The QWI measures average earnings at the state-quarter-sex level, not individual wages. Compositional changes within cells---for example, if transparency causes more women to enter high-paying firms---could inflate the estimated gender gap narrowing. However, the CPS individual-level analysis, which controls for demographics, occupation, and industry, produces estimates of comparable magnitude (4--6 pp vs.\ 6.1 pp). The consistency across micro and aggregate data suggests composition effects are not driving the finding.

\textit{Small number of treated states.} Eight treated states limits the precision of CPS heterogeneity analyses and the power of design-based inference. The Fisher permutation $p$-value of 0.154 for the CPS gender DDD reflects this fundamental constraint. As more states adopt transparency requirements, the CPS-based tests will become more powerful.

\textit{Unexploited policy variation.} The variation in employer thresholds across states (all employers vs.\ 4+ vs.\ 15+ vs.\ 50+) is a natural source of dose-response identification that this paper does not exploit. Future work with employer-level data could test whether effects concentrate in firm size ranges near the threshold, providing a regression-discontinuity complement to the DiD design.

\textit{Employment selection.} The gender gap narrowing could partly reflect changes in who remains employed rather than within-worker wage changes. If transparency induces lower-paid men to exit or higher-paid women to enter, the observed gap closure would overstate true wage compression. Three checks mitigate this concern: CPS composition tests show no significant change in percent female after treatment; Lee bounds under monotonicity yield a tight range (0.042--0.050); and the CPS individual-level analysis controls for detailed demographics. Nevertheless, linked employer-employee data enabling within-job comparisons would provide the strongest test.

\textit{Geographic spillovers.} Colorado's initial application of transparency requirements to remote-eligible positions may create spillovers to control states whose residents see Colorado-posted jobs. This would attenuate treatment-control differences, biasing estimates toward zero---making the significant gender DDD conservative. Future work could exploit the subsequent narrowing of Colorado's law to Colorado-based positions as a natural experiment.

\textit{Mechanisms remain indirect.} I cannot directly observe information flows or bargaining behavior. Linked employer-employee data with job-posting information---increasingly available from platforms like Glassdoor and Indeed---could provide direct mechanism evidence by tracking how posted ranges change search behavior and offer acceptance.

\subsection{Policy Implications}

The equity-efficiency trade-off turns out to be remarkably favorable. Transparency narrows the gender gap by half without reducing aggregate wages or disrupting flows. For policymakers, mandatory salary range disclosure is an efficient tool: substantial distributional gains at effectively zero efficiency cost. The pervasive effects across industries suggest broad mandates maximize equity gains.

These results carry a broader lesson about information policy. Transparency matters most where it corrects asymmetries, and its distributional consequences depend on who was previously disadvantaged. Policymakers considering information mandates in healthcare, financial products, or housing should attend to these dynamics.

\section{Conclusion}
\label{sec:conclusion}

Salary transparency laws were supposed to disrupt labor markets. They did not. What they did was simpler and more important: they leveled the informational playing field between men and women.

Two independent datasets---one surveying workers, one tracking employers---tell the same story. Average wages do not move. The gender gap narrows by 4--6 percentage points. Hiring, separations, and job creation continue undisturbed. The mechanism is information equalization, not employer commitment or costly adjustment.

The policy implication is direct. Among the interventions available to narrow the gender pay gap, requiring employers to post salary ranges in job listings ranks among the most efficient ever studied---large distributional gains, no measurable efficiency cost, no labor market disruption. As Illinois, Maryland, and Minnesota enter the post-treatment window, the precision of these estimates will improve. But the direction is already clear. When workers know what jobs pay, the people who benefit most are those who knew the least.

\section*{Acknowledgements}

This paper was produced as part of the Autonomous Policy Evaluation Project (APEP). The author thanks the CPS ASEC respondents and the Census Bureau for making these data available through IPUMS and the LEHD program.

\noindent\textbf{Replication Package:} \url{https://github.com/SocialCatalystLab/ape-papers}

\noindent\textbf{Contributor:} \url{https://github.com/SocialCatalystLab}

\label{apep_main_text_end}

\newpage
\begin{thebibliography}{99}

\bibitem[Abowd et~al.(2009)]{abowd2009lehd}
Abowd, J.~M., Stephens, B.~E., Vilhuber, L., Andersson, F., McKinney, K.~L., Roemer, M., and Woodcock, S. (2009).
\newblock The LEHD infrastructure files and the creation of the Quarterly Workforce Indicators.
\newblock In Dunne, T., Jensen, J.~B., and Roberts, M.~J., editors, \textit{Producer Dynamics: New Evidence from Micro Data}, pages 149--230. University of Chicago Press.

\bibitem[Babcock and Laschever(2003)]{babcock2003women}
Babcock, L. and Laschever, S. (2003).
\newblock \emph{Women Don't Ask: Negotiation and the Gender Divide}.
\newblock Princeton University Press.

\bibitem[Baker et~al.(2023)]{baker2023pay}
Baker, M., Halberstam, Y., Kroft, K., Mas, A., and Messacar, D. (2023).
\newblock Pay transparency and the gender gap.
\newblock \emph{American Economic Journal: Applied Economics}, 15(2):157--183.

\bibitem[Bennedsen et~al.(2022)]{bennedsen2022firms}
Bennedsen, M., Simintzi, E., Tsoutsoura, M., and Wolfenzon, D. (2022).
\newblock Do firms respond to gender pay gap transparency?
\newblock \emph{Journal of Finance}, 77(4):2051--2091.

\bibitem[Blau and Kahn(2017)]{blau2017gender}
Blau, F.~D. and Kahn, L.~M. (2017).
\newblock The gender wage gap: Extent, trends, and explanations.
\newblock \emph{Journal of Economic Literature}, 55(3):789--865.

\bibitem[Blundell et~al.(2022)]{blundell2022wage}
Blundell, R., Cribb, J., McNally, S., and van Veen, C. (2022).
\newblock Does information disclosure reduce the gender pay gap?
\newblock \emph{IFS Working Paper}.

\bibitem[Burdett and Mortensen(1998)]{burdett1998wage}
Burdett, K. and Mortensen, D.~T. (1998).
\newblock Wage differentials, employer size, and unemployment.
\newblock \emph{International Economic Review}, 39(2):257--273.

\bibitem[Callaway and Sant'Anna(2021)]{callaway2021difference}
Callaway, B. and Sant'Anna, P.~H. (2021).
\newblock Difference-in-differences with multiple time periods.
\newblock \emph{Journal of Econometrics}, 225(2):200--230.

\bibitem[Card et~al.(2018)]{card2018firms}
Card, D., Cardoso, A.~R., Heining, J., and Kline, P. (2018).
\newblock Firms and labor market inequality: Evidence and some theory.
\newblock \emph{Journal of Labor Economics}, 36(S1):S13--S70.

\bibitem[Cullen and Pakzad-Hurson(2023)]{cullen2023pay}
Cullen, Z.~B. and Pakzad-Hurson, B. (2023).
\newblock Equilibrium effects of pay transparency.
\newblock \emph{Econometrica}, 91(3):911--959.

\bibitem[Flood et~al.(2023)]{flood2023ipums}
Flood, S., King, M., Rodgers, R., Ruggles, S., Warren, J.~R., and Westberry, M. (2023).
\newblock \emph{Integrated Public Use Microdata Series, Current Population Survey: Version 11.0}.
\newblock Minneapolis, MN: IPUMS.

\bibitem[Goldin(2014)]{goldin2014grand}
Goldin, C. (2014).
\newblock A grand gender convergence: Its last chapter.
\newblock \emph{American Economic Review}, 104(4):1091--1119.

\bibitem[Goodman-Bacon(2021)]{goodman2021difference}
Goodman-Bacon, A. (2021).
\newblock Difference-in-differences with variation in treatment timing.
\newblock \emph{Journal of Econometrics}, 225(2):254--277.

\bibitem[Leibbrandt and List(2015)]{leibbrandt2015women}
Leibbrandt, A. and List, J.~A. (2015).
\newblock Do women avoid salary negotiations? Evidence from a large-scale natural field experiment.
\newblock \emph{Management Science}, 61(9):2016--2024.

\bibitem[Rambachan and Roth(2023)]{rambachan2023more}
Rambachan, A. and Roth, J. (2023).
\newblock A more credible approach to parallel trends.
\newblock \emph{Review of Economic Studies}, 90(5):2555--2591.

\bibitem[Roth(2022)]{roth2022pretest}
Roth, J. (2022).
\newblock Pretest with caution: Event-study estimates after testing for parallel trends.
\newblock \emph{American Economic Review: Insights}, 4(3):305--322.

\bibitem[Stigler(1962)]{stigler1962information}
Stigler, G.~J. (1962).
\newblock Information in the labor market.
\newblock \emph{Journal of Political Economy}, 70(5, Part 2):94--105.

\bibitem[Sun and Abraham(2021)]{sun2021estimating}
Sun, L. and Abraham, S. (2021).
\newblock Estimating dynamic treatment effects in event studies with heterogeneous treatment effects.
\newblock \emph{Journal of Econometrics}, 225(2):175--199.

\bibitem[de Chaisemartin and D'Haultfoeuille(2020)]{dechaisemartin2020twoway}
de Chaisemartin, C. and D'Haultfoeuille, X. (2020).
\newblock Two-way fixed effects estimators with heterogeneous treatment effects.
\newblock \emph{American Economic Review}, 110(9):2964--2996.

\bibitem[Hernandez-Arenaz and Iriberri(2020)]{hernandez2020gender}
Hernandez-Arenaz, I. and Iriberri, N. (2020).
\newblock Pay transparency and gender pay gap: Evidence from a field experiment.
\newblock \emph{Management Science}, 66(6):2574--2594.

\bibitem[Lee(2009)]{lee2009training}
Lee, D.~S. (2009).
\newblock Training, wages, and sample selection: Estimating sharp bounds on treatment effects.
\newblock \emph{Review of Economic Studies}, 76(3):1071--1102.

\bibitem[Arkhangelsky et~al.(2021)]{arkhangelsky2021synthetic}
Arkhangelsky, D., Athey, S., Hirshberg, D.~A., Imbens, G.~W., and Wager, S. (2021).
\newblock Synthetic difference-in-differences.
\newblock \emph{American Economic Review}, 111(12):4088--4118.

\bibitem[Sinha(2024)]{sinha2024salary}
Sinha, A. (2024).
\newblock The effects of salary history bans on wages and the gender pay gap.
\newblock \emph{American Economic Journal: Economic Policy}, 16(2):352--382.

\bibitem[Ferman and Pinto(2019)]{ferman2019inference}
Ferman, B. and Pinto, C. (2019).
\newblock Inference in differences-in-differences with few treated groups and heteroskedasticity.
\newblock \emph{Review of Economics and Statistics}, 101(3):452--467.

\bibitem[Cameron et~al.(2008)]{cameron2008bootstrap}
Cameron, A.~C., Gelbach, J.~B., and Miller, D.~L. (2008).
\newblock Bootstrap-based improvements for inference with clustered errors.
\newblock \emph{Review of Economics and Statistics}, 90(3):414--427.

\bibitem[Roth et~al.(2023)]{roth2023whats}
Roth, J., Sant'Anna, P.~H.~C., Bilinski, A., and Poe, J. (2023).
\newblock What's trending in difference-in-differences? A synthesis of the recent econometrics literature.
\newblock \emph{Journal of Econometrics}, 235(2):2218--2244.

\bibitem[Borusyak et~al.(2024)]{borusyak2024revisiting}
Borusyak, K., Jaravel, X., and Spiess, J. (2024).
\newblock Revisiting event-study designs: Robust and efficient estimation.
\newblock \emph{Review of Economic Studies}, 91(6):3253--3285.

\bibitem[Abadie et~al.(2023)]{abadie2023should}
Abadie, A., Athey, S., Imbens, G.~W., and Wooldridge, J.~M. (2023).
\newblock When should you adjust standard errors for clustering?
\newblock \emph{Quarterly Journal of Economics}, 138(1):1--35.

\bibitem[Athey and Imbens(2022)]{athey2022design}
Athey, S. and Imbens, G.~W. (2022).
\newblock Design-based analysis in difference-in-differences settings with staggered adoption.
\newblock \emph{Journal of Econometrics}, 226(1):62--79.

\bibitem[Bertrand et~al.(2004)]{bertrand2004much}
Bertrand, M., Duflo, E., and Mullainathan, S. (2004).
\newblock How much should we trust differences-in-differences estimates?
\newblock \emph{Quarterly Journal of Economics}, 119(1):249--275.

\bibitem[Johnson(2017)]{johnson2017online}
Johnson, M.~S. (2017).
\newblock The effect of online salary information on wages.
\newblock \emph{Working Paper}.

\bibitem[Recalde and Vesterlund(2018)]{recalde2018gender}
Recalde, M.~P. and Vesterlund, L. (2018).
\newblock Gender differences in negotiation and policy for improvement.
\newblock In Averett, S.~L., Argys, L.~M., and Hoffman, S.~D., editors, \emph{The Oxford Handbook of Women and the Economy}. Oxford University Press.

\bibitem[Conley and Taber(2011)]{conley2011inference}
Conley, T.~G. and Taber, C.~R. (2011).
\newblock Inference with ``difference in differences'' with a small number of policy changes.
\newblock \emph{Review of Economics and Statistics}, 93(1):113--125.

\bibitem[Imai and Kim(2021)]{imai2021randomization}
Imai, K. and Kim, I.~S. (2021).
\newblock On randomization tests for difference-in-differences and panel data.
\newblock \emph{Statistical Science}, 36(4):610--629.

\bibitem[Abadie et~al.(2010)]{abadie2010synthetic}
Abadie, A., Diamond, A., and Hainmueller, J. (2010).
\newblock Synthetic control methods for comparative case studies: Estimating the effect of California's tobacco control program.
\newblock \emph{Journal of the American Statistical Association}, 105(490):493--505.

\bibitem[Cowgill(2021)]{cowgill2021iron}
Cowgill, B. (2021).
\newblock Ironing out kinks in the wage distribution: The effects of pay transparency.
\newblock \emph{NBER Working Paper} No.\ w28346.

\bibitem[Hall and Krueger(2012)]{hallkrueger2012evidence}
Hall, R.~E. and Krueger, A.~B. (2012).
\newblock Evidence on the incidence of wage posting, recruiting, and bargaining.
\newblock \emph{Economica}, 79:396--418.

\bibitem[Kline et~al.(2021)]{kline2021firm}
Kline, P., Petkova, N., Williams, H.~L., and Zidar, O. (2021).
\newblock Who profits from patents? Rent-sharing at publicly traded firms.
\newblock \emph{Quarterly Journal of Economics}, 136(1):1--62.

\end{thebibliography}

\newpage
\appendix

\section{Data Appendix}

\subsection{Treatment Timing}

\begin{table}[H]
\centering
\caption{Salary Transparency Law Adoption}
\label{tab:timing}
\begin{threeparttable}
\begin{tabular}{lcccc}
\toprule
State & Effective Date & CPS First Year & QWI First Quarter & Threshold \\
\midrule
Colorado & January 1, 2021 & 2021 & 2021Q1 & All employers \\
Connecticut & October 1, 2021 & 2022 & 2022Q1 & All employers \\
Nevada & October 1, 2021 & 2022 & 2022Q1 & All employers \\
Rhode Island & January 1, 2023 & 2023 & 2023Q1 & All employers \\
California & January 1, 2023 & 2023 & 2023Q1 & 15+ employees \\
Washington & January 1, 2023 & 2023 & 2023Q1 & 15+ employees \\
New York & September 17, 2023 & 2024 & 2024Q1 & 4+ employees \\
Hawaii & January 1, 2024 & 2024 & 2024Q1 & 50+ employees \\
\bottomrule
\end{tabular}
\begin{tablenotes}
\small
\item \textit{Notes:} CPS First Year indicates when the law first affects income measured in the CPS ASEC. QWI First Quarter is the first treated quarter. Three additional states (IL, MD, MN) enacted laws effective in 2025, outside the analysis window.
\end{tablenotes}
\end{threeparttable}
\end{table}

\subsection{CPS Pre-Treatment Balance}

\begin{table}[H]
\centering
\caption{Pre-Treatment Balance: Treated vs.\ Control States (CPS, 2015--2020)}
\label{tab:balance}
\begin{threeparttable}
\begin{tabular}{lccc}
\toprule
& Treated & Control & Difference \\
\midrule
Mean hourly wage (\$) & 28.42 & 25.18 & 3.24*** \\
Female (\%) & 47.2 & 46.1 & 1.1 \\
Age (years) & 42.3 & 42.8 & -0.5 \\
College+ (\%) & 38.5 & 31.2 & 7.3*** \\
Full-time (\%) & 81.2 & 80.8 & 0.4 \\
High-bargaining occ.\ (\%) & 24.3 & 19.8 & 4.5*** \\
Metropolitan (\%) & 89.2 & 76.4 & 12.8*** \\
\midrule
N (person-years) & 185,432 & 312,891 & \\
States & 8 & 43 & \\
\bottomrule
\end{tabular}
\begin{tablenotes}
\small
\item \textit{Notes:} *** $p<0.01$. Level differences absorbed by state fixed effects.
\end{tablenotes}
\end{threeparttable}
\end{table}

\subsection{CPS Event Study Coefficients}

\begin{table}[H]
\centering
\caption{CPS Event Study Coefficients}
\label{tab:event_study}
\begin{threeparttable}
\begin{tabular}{cccc}
\toprule
Event Time & Coefficient & SE & 95\% CI \\
\midrule
$-5$ & $-0.009$ & 0.009 & [$-0.028$, 0.009] \\
$-4$ & 0.023 & 0.015 & [$-0.006$, 0.052] \\
$-3$ & 0.015 & 0.015 & [$-0.015$, 0.044] \\
$-2$ & $-0.013$* & 0.006 & [$-0.026$, $-0.001$] \\
$-1$ & \multicolumn{3}{c}{(reference period)} \\
0 & $-0.011$ & 0.008 & [$-0.027$, 0.004] \\
1 & 0.011 & 0.010 & [$-0.009$, 0.030] \\
2 & $-0.021$** & 0.009 & [$-0.039$, $-0.003$] \\
3 & 0.021*** & 0.006 & [0.009, 0.033] \\
\bottomrule
\end{tabular}
\begin{tablenotes}
\small
\item \textit{Notes:} Callaway-Sant'Anna estimator. Standard errors clustered at state level. * $p<0.10$, ** $p<0.05$, *** $p<0.01$.
\end{tablenotes}
\end{threeparttable}
\end{table}

\subsection{CPS Bargaining Heterogeneity}

\begin{table}[H]
\centering
\caption{Heterogeneity by Occupation Bargaining Intensity (CPS)}
\label{tab:bargaining}
\begin{threeparttable}
\begin{tabular}{lcccc}
\toprule
& (1) & (2) & (3) & (4) \\
& All & All & High-Bargain & Low-Bargain \\
\midrule
Treated $\times$ Post & $-0.010$ & $-0.005$ & $-0.012$ & 0.003 \\
& (0.015) & (0.012) & (0.008) & (0.011) \\
Treated $\times$ Post $\times$ High-Bargain & 0.024 & 0.011 & & \\
& (0.020) & (0.014) & & \\
\midrule
State \& Year FE & Yes & Yes & Yes & Yes \\
Demographics & No & Yes & Yes & Yes \\
Observations & 614,625 & 614,625 & 177,873 & 388,971 \\
\bottomrule
\end{tabular}
\begin{tablenotes}
\small
\item \textit{Notes:} Standard errors clustered at state level. * $p<0.10$, ** $p<0.05$, *** $p<0.01$.
\end{tablenotes}
\end{threeparttable}
\end{table}

\subsection{CPS Cohort-Specific Effects}

\begin{table}[H]
\centering
\caption{CPS Treatment Effects by Cohort}
\label{tab:cohort}
\begin{threeparttable}
\begin{tabular}{lccccc}
\toprule
Cohort (Year) & States & Post-Periods & ATT & SE & 95\% CI \\
\midrule
2021 & CO & 4 & $-0.007$ & 0.005 & [$-0.017$, 0.003] \\
2022 & CT, NV & 3 & $-0.015$ & 0.008 & [$-0.030$, 0.001] \\
2023 & CA, WA, RI & 2 & $-0.008$ & 0.013 & [$-0.033$, 0.017] \\
2024 & NY, HI & 1 & 0.002 & 0.018 & [$-0.033$, 0.037] \\
\midrule
Aggregate & 8 states & -- & $-0.010$ & 0.008 & [$-0.025$, 0.005] \\
\bottomrule
\end{tabular}
\begin{tablenotes}
\small
\item \textit{Notes:} Cohort-specific ATTs from Callaway-Sant'Anna. All four cohorts show negative point estimates, none individually significant.
\end{tablenotes}
\end{threeparttable}
\end{table}

\subsection{QWI Gender Gap Trends}

\begin{figure}[H]
\centering
\includegraphics[width=0.85\textwidth]{figures/fig_qwi_gap_trends.pdf}
\caption{QWI Gender Earnings Gap: Treated vs.\ Control States}
\label{fig:qwi_gap_trends}
\begin{minipage}{0.85\textwidth}
\footnotesize
\textit{Notes:} Male-female log earnings gap from QWI administrative data. Quarterly, 2012--2024.
\end{minipage}
\end{figure}

\subsection{Quarterly Event Studies}

\begin{figure}[H]
\centering
\includegraphics[width=0.85\textwidth]{figures/fig_qwi_event_earns.pdf}
\caption{QWI Quarterly Event Study: Earnings}
\label{fig:qwi_event_earns}
\begin{minipage}{0.85\textwidth}
\footnotesize
\textit{Notes:} Callaway-Sant'Anna quarterly event-study coefficients for log average earnings. Window trimmed to $[-16, +12]$ quarters. Pre-treatment coefficients show no trend.
\end{minipage}
\end{figure}

\begin{figure}[H]
\centering
\includegraphics[width=0.85\textwidth]{figures/fig_qwi_event_gap.pdf}
\caption{QWI Quarterly Event Study: Gender Earnings Gap}
\label{fig:qwi_event_gap}
\begin{minipage}{0.85\textwidth}
\footnotesize
\textit{Notes:} Callaway-Sant'Anna quarterly event-study coefficients for the male-female log earnings gap. Window trimmed to $[-16, +12]$ quarters.
\end{minipage}
\end{figure}

\subsection{Permutation Distribution}

\begin{figure}[H]
\centering
\includegraphics[width=0.85\textwidth]{figures/fig10_permutation_ddd.pdf}
\caption{CPS Permutation Distribution: Gender DDD Coefficient}
\label{fig:perm_ddd}
\begin{minipage}{0.85\textwidth}
\footnotesize
\textit{Notes:} Distribution of the gender DDD across 5,000 random treatment assignments. Vertical line marks the actual estimate. Two-sided permutation $p = 0.154$.
\end{minipage}
\end{figure}

\subsection{HonestDiD Gender Gap Sensitivity}

\begin{table}[htbp]
\centering
\caption{HonestDiD Sensitivity: Gender Gap Effect}
\label{tab:honestdid_gender}
\begin{tabular}{cccc}
\toprule
$M$ & Estimate & 95\% CI & Zero Excluded? \\
\midrule
0.0 & 0.0714 & [0.0431, 0.0996] & Yes \\
0.5 & 0.1492 & [$-1.58$, 1.88] & No \\
\bottomrule
\end{tabular}
\begin{minipage}{0.90\textwidth}
\footnotesize
\textit{Notes:} HonestDiD sensitivity \citep{rambachan2023more}. At $M = 0$ (exact parallel trends), 95\% CI firmly excludes zero. For $M \geq 0.5$, bounds become uninformative.
\end{minipage}
\end{table}

\subsection{Legislative Citations}

\begin{itemize}
\item \textbf{Colorado:} SB19-085, C.R.S.\ \S 8-5-201. \url{https://leg.colorado.gov/bills/sb19-085}
\item \textbf{Connecticut:} Public Act 21-30 (HB 6380). \url{https://www.cga.ct.gov/asp/cgabillstatus/cgabillstatus.asp?selBillType=Bill&bill_num=HB06380}
\item \textbf{Nevada:} SB 293 (2021), NRS 613.4383. \url{https://www.leg.state.nv.us/App/NELIS/REL/81st2021/Bill/7898/Overview}
\item \textbf{Rhode Island:} H 5171 (2023). \url{http://webserver.rilin.state.ri.us/BillText/BillText23/HouseText23/H5171.pdf}
\item \textbf{California:} SB 1162 (2022). \url{https://leginfo.legislature.ca.gov/faces/billNavClient.xhtml?bill_id=202120220SB1162}
\item \textbf{Washington:} SB 5761 (2022). \url{https://app.leg.wa.gov/billsummary?BillNumber=5761&Year=2021}
\item \textbf{New York:} S.9427/A.10477. \url{https://legislation.nysenate.gov/pdf/bills/2021/S9427A}
\item \textbf{Hawaii:} SB 1057 (2023). \url{https://www.capitol.hawaii.gov/session/measure_indiv.aspx?billtype=SB&billnumber=1057&year=2023}
\end{itemize}

\end{document}
