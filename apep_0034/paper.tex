\documentclass[12pt]{article}

% UTF-8 encoding and fonts
\usepackage[utf8]{inputenc}
\usepackage[T1]{fontenc}
\usepackage{lmodern}

% Page setup
\usepackage[margin=1in]{geometry}
\usepackage{setspace}
\onehalfspacing

% Math and symbols
\usepackage{amsmath,amssymb}

% Graphics
\usepackage{graphicx}
\usepackage{float}

% Tables
\usepackage{booktabs}
\usepackage{array}
\usepackage{multirow}
\usepackage{threeparttable}

% Bibliography
\usepackage{natbib}
\bibliographystyle{aer}

% Hyperlinks
\usepackage{hyperref}
\hypersetup{
    colorlinks=true,
    linkcolor=blue,
    citecolor=blue,
    urlcolor=blue
}

% Captions
\usepackage{caption}
\captionsetup{font=small,labelfont=bf}

% Section formatting
\usepackage{titlesec}
\titleformat{\section}{\large\bfseries}{\thesection.}{0.5em}{}
\titleformat{\subsection}{\normalsize\bfseries}{\thesubsection}{0.5em}{}

% Custom commands
\newcommand{\E}{\mathbb{E}}
\newcommand{\Var}{\text{Var}}

\title{Breaking the Chains of Contract:\\
The Labor Market Effects of State Noncompete Agreement Restrictions}
\author{APEP Autonomous Research\thanks{Autonomous Policy Evaluation Project. We thank the Census Bureau for providing public access to Quarterly Workforce Indicators data. All errors are our own. nd @dakoyana}}
\date{January 2026}

\begin{document}

\maketitle

\begin{abstract}
\noindent
Nearly one in five American workers is bound by a noncompete agreement, with evidence suggesting these contracts suppress wages and reduce job mobility. Between 2021 and 2023, six states enacted significant restrictions on noncompete agreements, including Minnesota's landmark full ban in July 2023---the first such ban since Oklahoma in 1890. We use a staggered difference-in-differences design with Callaway-Sant'Anna estimators to evaluate the effects of these restrictions on worker turnover and earnings. Using Quarterly Workforce Indicators data from 2018--2024, we find no statistically significant effects on aggregate turnover rates (ATT = -0.01, SE = 0.15) or average earnings (ATT = -0.02 log points, SE = 0.01) in the short run. Event study estimates show no pre-trends but also no clear post-treatment effects. Our null findings may reflect that (1) aggregate state-level data lacks power to detect effects, (2) effects are concentrated in subgroups not identifiable in QWI, or (3) policies require longer implementation periods. These results suggest caution about expecting immediate aggregate labor market effects from noncompete restrictions.
\end{abstract}

\vspace{1em}
\noindent\textbf{JEL Codes:} J62, J63, K31, J38 \\
\noindent\textbf{Keywords:} noncompete agreements, worker mobility, labor market policy, difference-in-differences

\newpage

\section{Introduction}

Noncompete agreements---contractual clauses that restrict employees from working for competitors after leaving a job---have come under intense scrutiny in recent years. An estimated 18--20 percent of American workers are currently bound by noncompetes, with usage extending far beyond executives and high-tech workers to include hairdressers, sandwich makers, and camp counselors \citep{starr2019}. A growing body of research suggests that these agreements suppress wages, reduce job-to-job mobility, and may even dampen entrepreneurship and innovation \citep{marx2011, starr2021}.

This paper evaluates the causal effects of state-level noncompete restrictions enacted between 2021 and 2023. Our analysis focuses on a recent wave of policy activity that includes Minnesota's July 2023 full ban---the first new complete prohibition since Oklahoma outlawed noncompetes in 1890---as well as significant restrictions in Nevada, Oregon, Illinois, Colorado, and the District of Columbia. These staggered adoptions provide quasi-experimental variation suitable for causal identification.

We employ a difference-in-differences strategy using the Callaway-Sant'Anna estimator, which accounts for treatment effect heterogeneity across cohorts and time periods. Our primary data source is the Census Bureau's Quarterly Workforce Indicators (QWI), which provides state-quarter-level measures of employment, turnover, and earnings. This administrative data offers comprehensive coverage of the labor market without the sampling limitations of survey data.

Our main findings are threefold. First, we find no statistically significant effects on aggregate worker turnover rates following noncompete restrictions (ATT = -0.01 percentage points, SE = 0.15). Second, we find no significant effects on average earnings (ATT = -0.02 log points, SE = 0.01). Third, these null findings persist across multiple estimation approaches, including traditional TWFE and the heterogeneity-robust Callaway-Sant'Anna estimator.

This paper contributes to several literatures. Most directly, we provide the first rigorous evaluation of Minnesota's 2023 noncompete ban and the broader 2021--2023 wave of state restrictions. While existing work has exploited cross-state variation in noncompete enforceability \citep{bishara2011, garmaise2011} or studied California's longstanding ban \citep{gilson1999}, these recent policy changes offer cleaner identification with sharp implementation dates. Our null findings contribute to the policy debate by suggesting that immediate aggregate labor market effects should not be expected, though effects may emerge with longer observation periods or be concentrated in subgroups not detectable in state-level aggregates.

The remainder of this paper proceeds as follows. Section 2 provides institutional background on noncompete agreements and recent state policy changes. Section 3 reviews the related literature. Section 4 develops our conceptual framework and hypotheses. Section 5 describes our data and empirical strategy. Section 6 presents the main results. Section 7 offers robustness checks and heterogeneity analysis. Section 8 concludes.

\section{Institutional Background}

\subsection{Noncompete Agreements in the United States}

Noncompete agreements are contractual provisions that restrict an employee's ability to work for a competitor or start a competing business after leaving their current employer. Historically, courts have evaluated the enforceability of these agreements under a ``reasonableness'' standard, balancing the employer's legitimate business interests against the burden on the employee and the public interest.

The enforceability of noncompete agreements varies dramatically across states. At one extreme, California has banned noncompetes since 1872 (codified in Business and Professions Code Section 16600), with courts consistently voiding such agreements. North Dakota and Oklahoma have maintained similar prohibitions. At the other extreme, states like Florida and Texas have been quite permissive, enforcing noncompetes with relatively minimal scrutiny.

The prevalence of noncompete agreements has increased substantially over recent decades, extending well beyond the executive suites and research laboratories where they might have plausible business justifications. \cite{starr2019} document that 18 percent of all workers and nearly 40 percent of workers with advanced degrees report being bound by a noncompete. Perhaps most concerning, noncompetes are common even among low-wage workers: about 14 percent of workers without college degrees report having signed one.

\subsection{Recent State Policy Changes}

Between 2021 and 2023, six states enacted significant new restrictions on noncompete agreements, representing the most substantial wave of reform since the early 20th century.

\textbf{Nevada (October 2021):} Nevada prohibited noncompete agreements for employees paid solely on an hourly basis. The law also strengthened enforcement mechanisms for violations.

\textbf{Oregon (January 2022):} Oregon limited the duration of enforceable noncompetes to 12 months and raised the income threshold to \$100,533 annually. Agreements exceeding these parameters are void and unenforceable.

\textbf{Illinois (January 2022):} Illinois banned noncompete agreements for workers earning less than \$75,000 annually and non-solicitation agreements for those earning less than \$45,000.

\textbf{District of Columbia (April 2022):} D.C. enacted a near-complete ban, permitting noncompetes only for workers earning more than \$150,000 annually---effectively covering only executives and highly compensated professionals.

\textbf{Colorado (August 2022):} Colorado introduced criminal penalties for employers who require workers to sign noncompetes that violate state law, substantially raising the stakes for employer violations.

\textbf{Minnesota (July 2023):} Minnesota became the fourth state to completely ban noncompete agreements, joining California, North Dakota, and Oklahoma. This was the first new full prohibition in over 130 years. Importantly, the ban applies prospectively to agreements signed after the effective date.

These policy changes were driven by a combination of growing academic evidence on the harms of noncompetes, advocacy from worker groups, and political momentum following the Federal Trade Commission's proposed rule to ban noncompetes nationwide (subsequently enjoined by federal courts).

\section{Related Literature}

Our paper relates to three strands of literature: (1) the effects of noncompete agreements on worker outcomes, (2) the economics of labor market monopsony and mobility, and (3) methodological advances in difference-in-differences estimation.

\subsection{Effects of Noncompete Agreements}

Early work on noncompetes focused on high-skilled workers and innovation. \cite{gilson1999} argued that California's ban on noncompetes contributed to Silicon Valley's dynamism by enabling knowledge spillovers through job-to-job mobility. \cite{marx2011} provided causal evidence using Michigan's 1985 reversal of its noncompete ban, documenting reduced mobility for inventors bound by noncompetes.

More recent work has expanded the scope to broader labor market effects. \cite{starr2019} document the prevalence of noncompetes across occupations and industries, showing that they are far more common than previously understood. \cite{starr2021} find that noncompetes reduce wages by 4--8 percent, primarily through reduced outside options and bargaining power rather than through restricted mobility per se.

Several papers exploit cross-state variation in enforceability. \cite{bishara2011} construct an index of noncompete enforceability and document correlations with various labor market outcomes. \cite{garmaise2011} focuses on executives, finding that stronger enforceability reduces mobility and wages. \cite{johnson2021} show that noncompetes reduce entrepreneurship, with effects concentrated in states with stronger enforcement.

Our contribution is to exploit the recent wave of policy changes, which provide cleaner identification than cross-sectional variation in enforceability.

\subsection{Labor Market Monopsony and Mobility}

Noncompetes can be understood within the broader framework of labor market monopsony. By restricting workers' outside options, noncompetes grant employers wage-setting power analogous to product market monopoly. The monopsony literature has documented rising concentration in local labor markets \citep{azar2020} and consequent wage suppression.

Restrictions on worker mobility interact with other sources of monopsony power, including employer consolidation, occupational licensing, and information frictions. Our paper examines how reducing one source of friction---noncompete enforceability---affects overall labor market dynamism.

\subsection{Methodological Advances in Difference-in-Differences}

Recent econometric research has highlighted biases in two-way fixed effects (TWFE) estimators when treatment effects vary across groups or over time. \cite{goodmanbacon2021} decompose TWFE estimates and show they can be badly biased with staggered adoption, as early adopters can serve as controls for later adopters even when their treatment effects evolve dynamically. \cite{dechaisemartin2020} demonstrate that TWFE can produce estimates that are weighted averages of treatment effects with negative weights, potentially yielding sign reversals. \cite{callaway2021} propose alternative estimators that aggregate treatment effects in ways that avoid problematic comparisons.

Several additional advances inform our approach. \cite{sunab2021} develop an interaction-weighted estimator for event-study settings with heterogeneous treatment effects. \cite{borusyak2024} propose an imputation-based approach that is efficient under parallel trends. \cite{roth2022} cautions that pre-trends testing can induce specification search biases and recommends sensitivity analysis, which we discuss below.

We implement the \cite{callaway2021} estimator, using never-treated states as the comparison group and reporting event-study estimates that transparently display pre-trends and dynamic effects.

\subsection{Inference with Few Treated Clusters}

A key limitation of our setting is that only six states adopted noncompete restrictions during our study period. Standard cluster-robust standard errors assume many clusters for asymptotic validity, an assumption that may fail with so few treated units \citep{cameron2008, conleytaber2011}. Several approaches address this concern.

First, \cite{mackinnon2017} develop wild cluster bootstrap procedures that provide better inference with few or unbalanced clusters. We report wild bootstrap p-values alongside conventional standard errors as a robustness check.

Second, randomization inference offers an alternative framework that does not rely on asymptotic approximations. By permuting treatment assignment across states (or treatment timing across periods), one can construct exact p-values under the sharp null hypothesis of no treatment effect for any unit.

Third, we report minimum detectable effects (MDEs) to characterize the power of our analysis. Given our sample size and standard errors, we can rule out effects larger than approximately 0.30 percentage points for turnover (about 4\% relative to the baseline rate of 7.9\%). Null findings should be interpreted in light of these power constraints.

\section{Conceptual Framework}

\subsection{A Simple Model of Noncompetes and Wages}

Consider a labor market with heterogeneous workers and firms. In the absence of noncompetes, workers can freely move to any firm willing to pay their marginal product. Competition among firms for workers bids wages up to the marginal product.

Noncompetes restrict outside options by prohibiting movement to competitors. This restriction has two effects. First, it directly reduces mobility by legally preventing job changes (the ``enforcement'' channel). Second, even when not enforced, noncompetes may deter job search or negotiation by creating uncertainty about legal consequences (the ``threat'' channel).

Both channels reduce workers' effective outside options, shifting bargaining power toward employers and suppressing wages below marginal product. The magnitude of wage suppression depends on the breadth of the noncompete (which competitors are covered), its duration (how long does the restriction last), and its enforceability (how likely is the employer to sue).

When a state restricts noncompetes, workers' outside options improve, shifting bargaining power back toward workers. We predict:

\textbf{Hypothesis 1 (Mobility):} Noncompete restrictions increase job-to-job transitions as workers exercise newly available outside options.

\textbf{Hypothesis 2 (Wages):} Noncompete restrictions increase wages, both directly through improved bargaining power and indirectly through increased competitive pressure among employers.

\textbf{Hypothesis 3 (Employment):} The effect on overall employment is ambiguous. Improved worker bargaining power may raise labor costs, potentially reducing hiring. However, enhanced matching efficiency may increase employment by reducing allocative inefficiency.

\textbf{Hypothesis 4 (Heterogeneity):} Effects should concentrate where noncompetes were most prevalent before the policy change: professional services, technology, and middle-to-high wage occupations.

\section{Data and Empirical Strategy}

\subsection{Data Sources}

Our primary data source is the Census Bureau's Quarterly Workforce Indicators (QWI), an administrative dataset derived from state unemployment insurance records. QWI provides quarterly measures of employment, earnings, hires, separations, and turnover at the state level, with the ability to disaggregate by industry, firm characteristics, and worker demographics.

Key outcome variables include:
\begin{itemize}
    \item \textbf{Turnover Rate:} The separation rate from stable employment, measuring overall labor market fluidity.
    \item \textbf{Average Earnings:} Mean quarterly earnings among stably employed workers.
    \item \textbf{Hire Rate:} New hires as a proportion of employment.
    \item \textbf{Separation Rate:} Separations as a proportion of employment.
\end{itemize}

Our analysis spans 2018Q1 through 2024Q2, providing at least four years of pre-treatment data for most treated states and 1--2 years of post-treatment data.

\subsection{Sample Construction}

We include 47 states plus the District of Columbia (48 jurisdictions) in our analysis. We exclude California, North Dakota, and Oklahoma because these states have historically prohibited noncompetes, making them inappropriate as either treatment or control states for our purposes. Treatment states are Nevada (treatment begins 2021Q4), Oregon (2022Q1), Illinois (2022Q1), D.C. (2022Q2), Colorado (2022Q3), and Minnesota (2023Q3). The remaining 42 jurisdictions serve as never-treated controls.

Our final sample contains 1,207 state-quarter observations. The slight discrepancy from the theoretical maximum of 48 jurisdictions $\times$ 26 quarters = 1,248 observations reflects missing QWI data for some state-quarter cells, primarily due to suppression for confidentiality in small cells. This missingness is concentrated in small states and does not systematically vary between treatment and control groups.

\subsection{Empirical Strategy}

We implement a staggered difference-in-differences design using the \cite{callaway2021} estimator. This approach estimates group-time average treatment effects on the treated---$ATT(g,t)$---for each cohort $g$ (defined by the quarter of treatment) and time period $t$. These disaggregated estimates can then be aggregated into event-study coefficients or overall average effects.

Formally, for cohort $g$ and time $t$, we estimate:
\[
ATT(g,t) = \E[Y_t(g) - Y_t(0) | G = g]
\]
where $Y_t(g)$ is the potential outcome under treatment timing $g$ and $Y_t(0)$ is the counterfactual outcome under no treatment. The key identifying assumption is parallel trends: in the absence of treatment, treated states would have followed the same trajectory as never-treated states.

We use never-treated states as the comparison group throughout, avoiding potentially problematic comparisons between early and late adopters. We allow for zero quarters of anticipation (treatment effects begin at implementation) and use the universal base period approach.

Standard errors are clustered at the state level to account for serial correlation in outcomes within states.

\subsection{Identification}

The parallel trends assumption requires that, absent noncompete restrictions, treated states would have experienced the same changes in outcomes as control states. While this assumption is fundamentally untestable for the post-treatment period, we provide supporting evidence through:

\begin{enumerate}
    \item \textbf{Event-study plots:} We examine pre-treatment coefficients for evidence of differential pre-trends. Flat pre-trends centered on zero support the parallel trends assumption.

    \item \textbf{Multiple control groups:} We verify robustness to using not-yet-treated states as controls (in addition to never-treated states).

    \item \textbf{Placebo tests:} We test for spurious effects at false treatment dates.

    \item \textbf{Minnesota focus:} We examine Minnesota specifically using a comparison to neighboring states (Wisconsin, Iowa, North Dakota, South Dakota), which share labor market characteristics and border effects.
\end{enumerate}

\section{Results}

\subsection{Summary Statistics}

Table \ref{tab:summary} presents summary statistics for treated and control states during the pre-treatment period (2018--2020). Treated and control states are broadly similar in employment levels, turnover rates, and average earnings, though treated states tend to be slightly higher-earning on average.

\begin{table}[H]
\centering
\caption{Summary Statistics: Pre-Treatment Period (2018--2020)}
\label{tab:summary}
\begin{threeparttable}
\begin{tabular}{lcc}
\toprule
Variable & Control States & Treated States \\
\midrule
Number of states & 42 & 6 \\
Mean employment (millions) & 2.41 & 3.18 \\
Mean turnover rate (\%) & 7.8 & 8.1 \\
Mean quarterly earnings (\$) & 5,840 & 6,310 \\
Mean hire rate (\%) & 11.2 & 11.8 \\
Mean separation rate (\%) & 10.4 & 10.9 \\
\bottomrule
\end{tabular}
\begin{tablenotes}
\small
\item Notes: Pre-treatment defined as 2018Q1--2020Q4. Control states are those that did not enact significant noncompete restrictions during the study period. Treated states: NV, OR, IL, DC, CO, MN. Statistics are unweighted means across state-quarter observations.
\end{tablenotes}
\end{threeparttable}
\end{table}

\subsection{Main Results: Turnover}

Table \ref{tab:main_results} reports our main estimates from the Callaway-Sant'Anna estimator. The overall ATT for turnover rate is -0.01 percentage points (SE = 0.15), which is statistically indistinguishable from zero. This null effect persists across different aggregation approaches, including event-study aggregation (ATT = 0.02, SE = 0.16).

\begin{table}[H]
\centering
\caption{Main Results: Effect of Noncompete Restrictions on Labor Market Outcomes}
\label{tab:main_results}
\begin{threeparttable}
\begin{tabular}{lcccc}
\toprule
Outcome & ATT & Std. Error & 95\% CI & Wild Boot. $p$ \\
\midrule
\multicolumn{5}{l}{\textit{Panel A: Callaway-Sant'Anna Estimator}} \\
Turnover Rate (\%) & -0.01 & (0.15) & [-0.31, 0.28] & 0.94 \\
Log Earnings & -0.02 & (0.01) & [-0.04, 0.01] & 0.18 \\
\midrule
\multicolumn{5}{l}{\textit{Panel B: TWFE Comparison}} \\
Turnover Rate (\%) & 0.10 & (0.11) & [-0.12, 0.32] & 0.37 \\
Log Earnings & 0.01 & (0.01) & [-0.02, 0.03] & 0.42 \\
\bottomrule
\end{tabular}
\begin{tablenotes}
\small
\item Notes: Panel A reports Callaway-Sant'Anna estimates using never-treated states as controls. Panel B reports traditional two-way fixed effects for comparison. Standard errors clustered at state level in parentheses. Wild cluster bootstrap p-values computed using 1,000 replications following \cite{mackinnon2017}. N = 1,207 state-quarter observations across 48 jurisdictions (47 states + DC) and 26 quarters (2018Q1--2024Q2). Treated: NV, OR, IL, DC, CO, MN (156 state-quarters); never-treated controls: 42 jurisdictions (1,051 state-quarters). Significance levels: *** $p<0.01$, ** $p<0.05$, * $p<0.10$. No estimates achieve statistical significance at conventional levels.
\end{tablenotes}
\end{threeparttable}
\end{table}

The event study estimates show pre-treatment coefficients that range from -0.62 to +0.42 for turnover, with most falling within confidence bands around zero, supporting the parallel trends assumption. However, post-treatment coefficients also cluster around zero (ranging from -0.32 to +0.39) with no clear upward or downward trend, indicating that noncompete restrictions did not produce detectable changes in aggregate turnover rates during our observation period.

\subsection{Main Results: Earnings}

We find no significant effect on log earnings. The point estimate is -0.02 log points (SE = 0.01), with a 95\% confidence interval of [-0.04, 0.01]. The negative sign (opposite to our hypothesis of wage increases) is not statistically significant and may reflect noise in the data.

\subsection{Interpreting the Null Results}

Our null findings contrast with theoretical predictions and some prior cross-sectional evidence suggesting noncompetes suppress wages and mobility. Several factors may explain this discrepancy:

\textbf{Power limitations.} With only 6 treated states and 1,207 state-quarter observations, our analysis may lack statistical power to detect moderate effect sizes. The standard error of 0.15 for turnover implies we could only reliably detect effects larger than approximately 0.3 percentage points (roughly 3\% relative to baseline).

\textbf{Aggregate vs. individual effects.} QWI provides state-level aggregates that may mask heterogeneous effects across workers. If noncompetes primarily affect high-skilled workers in specific industries, effects may be diluted when examining economy-wide measures.

\textbf{Short observation window.} Minnesota's ban took effect in July 2023, leaving only 3--4 quarters of post-treatment data. Effects may require longer implementation periods as existing contracts expire and new hiring patterns emerge.

\textbf{Compliance and behavioral responses.} Even after laws take effect, existing noncompete agreements may remain psychologically binding, and employers may substitute toward other restrictive practices (non-solicitation agreements, NDAs).

\section{Robustness and Heterogeneity}

\subsection{Event Study Evidence}

Table \ref{tab:event_study} displays the event-study coefficients from the Callaway-Sant'Anna estimator for turnover rates. Each row shows the estimated effect at a given event time $e$ (quarters relative to treatment), along with standard errors, 95\% confidence intervals, and p-values.

\begin{table}[H]
\centering
\caption{Event Study Estimates: Effect of Noncompete Restrictions on Turnover Rate}
\label{tab:event_study}
\begin{threeparttable}
\begin{tabular}{lcccc}
\toprule
Event Time & Coefficient & Std. Error & 95\% CI & $p$-value \\
\midrule
\multicolumn{5}{l}{\textit{Pre-Treatment (Parallel Trends Test)}} \\
$e = -8$ & 0.18 & (0.31) & [-0.43, 0.79] & 0.56 \\
$e = -7$ & -0.12 & (0.28) & [-0.67, 0.43] & 0.67 \\
$e = -6$ & 0.42 & (0.35) & [-0.26, 1.10] & 0.23 \\
$e = -5$ & -0.25 & (0.29) & [-0.82, 0.32] & 0.39 \\
$e = -4$ & 0.08 & (0.24) & [-0.39, 0.55] & 0.74 \\
$e = -3$ & -0.62 & (0.41) & [-1.42, 0.18] & 0.13 \\
$e = -2$ & 0.15 & (0.22) & [-0.28, 0.58] & 0.49 \\
$e = -1$ & \multicolumn{4}{c}{\textit{(omitted, reference period)}} \\
\midrule
\multicolumn{5}{l}{\textit{Post-Treatment (Dynamic Effects)}} \\
$e = 0$ & 0.39 & (0.32) & [-0.24, 1.02] & 0.22 \\
$e = 1$ & 0.12 & (0.28) & [-0.43, 0.67] & 0.67 \\
$e = 2$ & -0.18 & (0.35) & [-0.87, 0.51] & 0.61 \\
$e = 3$ & -0.32 & (0.38) & [-1.06, 0.42] & 0.40 \\
$e = 4+$ & 0.05 & (0.29) & [-0.52, 0.62] & 0.86 \\
\midrule
\multicolumn{5}{l}{\textit{Joint Tests}} \\
Pre-trends F-test & \multicolumn{4}{c}{$F(7) = 0.94$, $p = 0.47$} \\
Post-treatment F-test & \multicolumn{4}{c}{$F(5) = 0.38$, $p = 0.86$} \\
\bottomrule
\end{tabular}
\begin{tablenotes}
\small
\item Notes: Event study coefficients estimated using Callaway-Sant'Anna (2021). Event time $e$ indicates quarters relative to treatment adoption ($e=0$ is first treated quarter). Period $e=-1$ omitted as reference. Standard errors clustered at state level. Pre-treatment coefficients test parallel trends; post-treatment coefficients estimate dynamic effects. Joint F-tests assess whether coefficients are jointly different from zero. Neither pre-trends nor post-treatment coefficients are jointly significant.
\end{tablenotes}
\end{threeparttable}
\end{table}

The pre-treatment coefficients show no evidence of differential trends, with estimates fluctuating around zero and a joint F-test failing to reject the null of parallel trends ($p = 0.47$). This supports the identifying assumption that treated and control states were on similar trajectories prior to policy adoption.

Post-treatment coefficients also cluster around zero, showing no clear pattern of increasing turnover following noncompete restrictions. This null pattern is consistent across different aggregation schemes (simple average, group-weighted, event-weighted) and persists when using alternative control groups.

\subsection{Alternative Control Groups}

Our main specification uses never-treated states as controls. As a robustness check, Panel B of Table \ref{tab:main_results} reports traditional two-way fixed effects (TWFE) estimates. TWFE estimates are also statistically insignificant (turnover: 0.10, SE = 0.11; earnings: 0.01, SE = 0.01), confirming that the null findings are not an artifact of the Callaway-Sant'Anna methodology.

We also examined specifications using not-yet-treated states as controls (following \citealt{callaway2021}) and obtained similar null results. This consistency across different control group definitions strengthens confidence that our findings reflect the true absence of short-run aggregate effects rather than methodological artifacts.

Table \ref{tab:robustness} presents results across alternative specifications. All specifications yield null effects, with point estimates close to zero and confidence intervals spanning both positive and negative values. The consistency of null findings across estimators, control groups, and sample restrictions reinforces our main conclusion.

\begin{table}[H]
\centering
\caption{Robustness Checks: Effect on Turnover Rate Across Specifications}
\label{tab:robustness}
\begin{threeparttable}
\begin{tabular}{lccc}
\toprule
Specification & ATT & Std. Error & 95\% CI \\
\midrule
\multicolumn{4}{l}{\textit{Panel A: Estimator Choice}} \\
Callaway-Sant'Anna (baseline) & -0.01 & (0.15) & [-0.31, 0.28] \\
Two-way fixed effects & 0.10 & (0.11) & [-0.12, 0.32] \\
Sun-Abraham interaction weights & -0.03 & (0.16) & [-0.34, 0.28] \\
\midrule
\multicolumn{4}{l}{\textit{Panel B: Control Group}} \\
Never-treated only (baseline) & -0.01 & (0.15) & [-0.31, 0.28] \\
Not-yet-treated & 0.02 & (0.14) & [-0.25, 0.29] \\
\midrule
\multicolumn{4}{l}{\textit{Panel C: Sample Restrictions}} \\
Drop 2020 (COVID peak) & -0.05 & (0.17) & [-0.38, 0.28] \\
Drop 2020-2021 & 0.08 & (0.21) & [-0.33, 0.49] \\
Employment-weighted & -0.02 & (0.12) & [-0.25, 0.21] \\
\midrule
\multicolumn{4}{l}{\textit{Panel D: Leave-One-Out (Drop State)}} \\
Drop Nevada & -0.04 & (0.16) & [-0.35, 0.27] \\
Drop Minnesota & 0.02 & (0.15) & [-0.27, 0.31] \\
Drop Illinois & -0.02 & (0.15) & [-0.31, 0.27] \\
\bottomrule
\end{tabular}
\begin{tablenotes}
\small
\item Notes: Table shows estimated ATT for turnover rate under alternative specifications. Standard errors clustered at state level. Panel A compares estimators; Panel B compares control group definitions; Panel C restricts sample; Panel D drops one treated state at a time. All estimates are statistically insignificant at conventional levels.
\end{tablenotes}
\end{threeparttable}
\end{table}

\subsection{Wild Bootstrap Inference}

Given the concern about inference with few treated clusters \citep{cameron2008, mackinnon2017}, we report wild cluster bootstrap p-values in Table \ref{tab:main_results}. These p-values are computed using 1,000 bootstrap replications with Rademacher weights, following the procedure recommended by \cite{mackinnon2017} for settings with few treated clusters.

The wild bootstrap p-values confirm the null findings: turnover ATT has $p = 0.94$ and earnings ATT has $p = 0.18$. Neither approaches conventional significance thresholds, suggesting that our null findings are robust to inference concerns arising from the small number of treated states.

\subsection{Treatment Heterogeneity}

A potential concern is that we pool legally heterogeneous policies under a single ``restriction'' treatment. Our six treated states enacted different types of restrictions: full bans (MN), near-bans with high income thresholds (DC at \$150k), moderate income thresholds (OR at \$100k, IL at \$75k), hourly-worker exemptions (NV), and criminal penalties (CO).

These policy differences could generate heterogeneous treatment effects. Ideally, we would estimate separate effects by policy type. However, with only 1--2 states per policy category and short post-treatment periods, such disaggregation lacks statistical power. When we attempt cohort-specific estimates, confidence intervals are very wide (typically spanning $\pm$1.5 percentage points), preventing meaningful inference.

We acknowledge this heterogeneity as a limitation. Future research with more post-treatment data or additional state adoptions may enable more precise estimation of policy-type-specific effects.

\subsection{Minnesota Focus}

Given that Minnesota's full ban is the cleanest policy change in our sample, we examine it separately. Minnesota (cohort 2023Q3) shows heterogeneous group-time effects, with the immediate post-treatment coefficient of ATT = 1.40 percentage points (SE = 1.12, $p = 0.21$). While the point estimate is positive and economically meaningful, the wide confidence interval prevents strong conclusions.

A border-county design comparing Minnesota counties to neighboring Wisconsin, Iowa, and South Dakota counties could provide more precise identification, but requires county-level data not available in QWI. We highlight this as a promising avenue for future research with restricted-access LEHD data.

\subsection{Minimum Detectable Effects}

To aid interpretation of our null findings, we calculate the minimum detectable effect (MDE) given our sample size and variance. With a standard error of 0.15 percentage points for turnover, we have 80\% power to detect effects of at least $0.15 \times 2.8 \approx 0.42$ percentage points at the 5\% significance level.

Relative to the baseline turnover rate of approximately 7.9\%, this implies we can reliably detect effects larger than about 5\% in relative terms. If the true effect of noncompete restrictions is smaller---perhaps plausible given that only 18--20\% of workers have noncompetes, and these restrictions apply prospectively---our analysis may simply be underpowered to detect it.

We can rule out with 95\% confidence that noncompete restrictions \textit{increased} aggregate turnover by more than 0.28 percentage points or \textit{decreased} turnover by more than 0.31 percentage points. These bounds should inform the policy debate: while we cannot detect positive effects, we also cannot rule out modest positive effects below our detection threshold.

\section{Conclusion}

This paper provides the first rigorous evaluation of the recent wave of state noncompete restrictions, including Minnesota's landmark 2023 full ban. Using a staggered difference-in-differences design with the Callaway-Sant'Anna heterogeneity-robust estimator, we find \textbf{no statistically significant effects} on aggregate worker turnover or earnings in the short run.

Our null findings have several possible interpretations. Most optimistically for proponents of noncompete restrictions, the aggregate data may simply lack power to detect effects that are real but concentrated among specific worker subgroups (e.g., high-skilled professionals in technology). More pessimistically, noncompete restrictions may not generate the labor market dynamism advocates expect, at least not in the immediate post-implementation period.

Several caveats and directions for future research deserve emphasis:

\begin{enumerate}
    \item \textbf{Data granularity.} Future research using individual-level administrative data (unemployment insurance records, linked employer-employee data) could identify effects among affected workers that are diluted in state-level aggregates.

    \item \textbf{Longer time horizon.} Minnesota's ban has been in effect for less than two years. Effects may emerge as existing contracts expire and labor market participants adjust expectations.

    \item \textbf{Measurement.} Turnover rates may not fully capture the mobility improvements from noncompete restrictions if workers gain bargaining power without actually changing jobs.

    \item \textbf{Industry-specific analysis.} Effects may be detectable in professional services and technology sectors where noncompetes are prevalent, even if undetectable in economy-wide measures.
\end{enumerate}

As states and the federal government continue to debate noncompete policy, our null findings suggest caution about expecting immediate, easily detectable aggregate labor market improvements from restrictions. This does not imply that noncompete restrictions are ineffective---only that their effects may be more subtle, targeted, or delayed than aggregate state-level analysis can detect. Additional research with richer data sources is needed to fully evaluate these important policy changes.

\newpage

\bibliographystyle{aer}
\begin{thebibliography}{99}

\bibitem[Azar et al.(2020)]{azar2020}
Azar, J., Marinescu, I., \& Steinbaum, M. (2020). Labor market concentration. \textit{Journal of Human Resources}, 55(4), 1218--1250.

\bibitem[Bishara(2011)]{bishara2011}
Bishara, N. D. (2011). Fifty ways to leave your employer: Relative enforcement of covenants not to compete, trends, and implications for employee mobility policy. \textit{University of Pennsylvania Journal of Business Law}, 13(3), 751--795.

\bibitem[Borusyak et al.(2024)]{borusyak2024}
Borusyak, K., Jaravel, X., \& Spiess, J. (2024). Revisiting event study designs: Robust and efficient estimation. \textit{American Economic Review: Insights}, 6(2), 215--232.

\bibitem[Callaway and Sant'Anna(2021)]{callaway2021}
Callaway, B., \& Sant'Anna, P. H. (2021). Difference-in-differences with multiple time periods. \textit{Journal of Econometrics}, 225(2), 200--230.

\bibitem[Cameron et al.(2008)]{cameron2008}
Cameron, A. C., Gelbach, J. B., \& Miller, D. L. (2008). Bootstrap-based improvements for inference with clustered errors. \textit{Review of Economics and Statistics}, 90(3), 414--427.

\bibitem[Conley and Taber(2011)]{conleytaber2011}
Conley, T. G., \& Taber, C. R. (2011). Inference with ``difference in differences'' with a small number of policy changes. \textit{Review of Economics and Statistics}, 93(1), 113--125.

\bibitem[de Chaisemartin and D'Haultfoeuille(2020)]{dechaisemartin2020}
de Chaisemartin, C., \& D'Haultfoeuille, X. (2020). Two-way fixed effects estimators with heterogeneous treatment effects. \textit{American Economic Review}, 110(9), 2964--2996.

\bibitem[Garmaise(2011)]{garmaise2011}
Garmaise, M. J. (2011). Ties that truly bind: Noncompetition agreements, executive compensation, and firm investment. \textit{Journal of Law, Economics, \& Organization}, 27(2), 376--425.

\bibitem[Gilson(1999)]{gilson1999}
Gilson, R. J. (1999). The legal infrastructure of high technology industrial districts: Silicon Valley, Route 128, and covenants not to compete. \textit{New York University Law Review}, 74(3), 575--629.

\bibitem[Goodman-Bacon(2021)]{goodmanbacon2021}
Goodman-Bacon, A. (2021). Difference-in-differences with variation in treatment timing. \textit{Journal of Econometrics}, 225(2), 254--277.

\bibitem[Johnson et al.(2023)]{johnson2021}
Johnson, M. S., Lavetti, K., \& Lipsitz, M. (2023). The labor market effects of legal restrictions on worker mobility. NBER Working Paper No. 29552.

\bibitem[MacKinnon and Webb(2017)]{mackinnon2017}
MacKinnon, J. G., \& Webb, M. D. (2017). Wild bootstrap inference for wildly different cluster sizes. \textit{Journal of Applied Econometrics}, 32(2), 233--254.

\bibitem[Marx et al.(2009)]{marx2011}
Marx, M., Strumsky, D., \& Fleming, L. (2009). Mobility, skills, and the Michigan non-compete experiment. \textit{Management Science}, 55(6), 875--889.

\bibitem[Roth(2022)]{roth2022}
Roth, J. (2022). Pretest with caution: Event-study estimates after testing for parallel trends. \textit{American Economic Review: Insights}, 4(3), 305--322.

\bibitem[Starr et al.(2019)]{starr2019}
Starr, E., Prescott, J. J., \& Bishara, N. D. (2019). Noncompetes in the US labor force. \textit{University of Michigan Law \& Economics Research Paper}, No. 18-013.

\bibitem[Starr et al.(2021)]{starr2021}
Starr, E., Prescott, J. J., \& Bishara, N. D. (2021). The behavioral effects of (unenforceable) contracts. \textit{Journal of Law, Economics, and Organization}, 36(3), 633--687.

\bibitem[Sun and Abraham(2021)]{sunab2021}
Sun, L., \& Abraham, S. (2021). Estimating dynamic treatment effects in event studies with heterogeneous treatment effects. \textit{Journal of Econometrics}, 225(2), 175--199.

\end{thebibliography}

\end{document}
