\begin{table}[htbp]
\centering
\caption{Cohort-Specific Treatment Effects (Callaway-Sant'Anna)}
\label{tab:cohort_att}
\begin{tabular}{ccccc}
\hline\hline
Treatment Cohort & ATT & SE & 95\% CI & $p$-value \\
\hline
2020 & -1.079 & (0.802) & [-2.650, 0.492] & 0.178 \\
2021 & 1.582 & (0.988) & [-0.354, 3.518] & 0.109 \\
2022 & -0.096 & (0.776) & [-1.618, 1.425] & 0.901 \\
2023 & -2.313** & (1.148) & [-4.563, -0.063] & 0.044 \\
\hline\hline
\end{tabular}
\begin{tablenotes}
\small
\item \textit{Notes:} $^{***}p<0.01$; $^{**}p<0.05$; $^{*}p<0.1$.
\item Group-specific ATTs from Callaway-Sant'Anna (2021) group aggregation.
\item Each row shows the average treatment effect for states that adopted copay caps
in the specified year. The 2023 cohort has only one post-treatment year of data,
making its estimate less reliable than cohorts with longer exposure.
\item The 2023 cohort's significant estimate ($p = 0.044$) should be interpreted with
caution: it reflects a single post-treatment year for three states, and the
result does not survive Bonferroni correction for testing across four cohorts
(adjusted threshold $p < 0.0125$). The aggregate ATT across all cohorts is not
statistically significant.
\end{tablenotes}
\end{table}
