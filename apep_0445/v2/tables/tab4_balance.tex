\begin{table}[htbp]
\centering
\caption{Covariate Balance at the 20\% Poverty Threshold}
\label{tab:balance}
\small
\begin{tabular}{lcccc}
\toprule
Covariate & RDD Estimate & Robust SE & $p$-value & N \\
\midrule
Population & -66.380 & (72.248) & 0.262 & 12,120 \\
\% Bachelor's degree & -1.560 & (0.405) & 0.000 & 10,732 \\
\% White & -4.095 & (1.225) & 0.000 & 9,910 \\
Median home value (\$) & $-$7,746 & (6,100) & 0.123 & 10,982 \\
Unemployment rate & 0.775 & (0.242) & 0.000 & 7,712 \\
Pre-period total employment & 181.301 & (249.836) & 0.308 & 8,197 \\
Pre-period info employment & 8.127 & (12.446) & 0.356 & 10,331 \\
\bottomrule
\end{tabular}
\begin{tablenotes}
\small
\item \textit{Notes:} Each row reports the RDD estimate for a pre-determined covariate at the 20\% poverty threshold, using \texttt{rdrobust} with MSE-optimal bandwidth and triangular kernel. $p$-values are from robust bias-corrected inference (which adjusts for boundary bias, so the $p$-value may differ from simple $t$-ratio). N varies across rows because \texttt{rdrobust} selects a separate MSE-optimal bandwidth for each covariate. Median home value is in US dollars. All covariates are from ACS 2011--2015 (pre-treatment).
\end{tablenotes}
\end{table}
