\begin{table}[htbp]
\centering
\caption{Summary Statistics: Census Tracts Near the 20\% Poverty Threshold}
\label{tab:summary}
\small
\begin{tabular}{lrrr}
\toprule
& Below 20\% & Above 20\% & Full Sample \\
\midrule
\textit{Panel A: Demographics} & & & \\
Number of tracts & 9,085 & 6,550 & 15,635 \\
Poverty rate (\%) & 15.4 & 23.6 & 18.9 \\
Total population & 3894 & 3754 & 3835 \\
\% Bachelor's degree & 16.0 & 12.3 & 14.5 \\
\% White & 79.4 & 66.1 & 73.8 \\
Median home value (\$) & 173,428 & 144,462 & 161,342 \\
Unemployment rate (\%) & 7.2 & 9.7 & 8.2 \\
\midrule
\textit{Panel B: Employment} & & & \\
Pre-period total employment & 1745.6 & 1701.8 & 1727.3 \\
Pre-period info employment & 31.50 & 26.71 & 29.49 \\
Post-period total employment & 1773.1 & 1707.9 & 1745.8 \\
Post-period info employment & 31.84 & 24.66 & 28.83 \\
$\Delta$ Total employment & 27.5 & 6.1 & 18.5 \\
$\Delta$ Info employment & 0.34 & -2.05 & -0.66 \\
\midrule
\textit{Panel C: OZ Status} & & & \\
OZ designated (\%, approx.) & 0.0 & 0.5 & 0.2 \\
\bottomrule
\end{tabular}
\begin{tablenotes}
\small
\item \textit{Notes:} Sample includes census tracts within the MSE-optimal bandwidth for the change in total employment outcome (7.9 percentage points from \citet{cattaneo2020}). Bandwidths vary by outcome in the main RDD estimates. Pre-period is the average of 2015--2017; post-period is the average of 2019--2023. Employment data from Census LEHD/LODES. Poverty rate and demographics from ACS 2011--2015. OZ designation is approximated by designating the top 25\% of poverty-eligible tracts within each state (see Appendix~\ref{app:data}); by construction, the approximated designation rate is zero for tracts below the 20\% poverty threshold.
\end{tablenotes}
\end{table}
