\documentclass[12pt]{article}

% UTF-8 encoding and fonts
\usepackage[utf8]{inputenc}
\usepackage[T1]{fontenc}
\usepackage{lmodern}  % Latin Modern font - fixes < > rendering issues

% Page setup
\usepackage[margin=1in]{geometry}
\usepackage{setspace}
\onehalfspacing

% Typography
\usepackage{microtype}

% Math and symbols
\usepackage{amsmath,amssymb}

% Graphics
\usepackage{graphicx}
\usepackage{float}
\usepackage{subcaption}

% Tables
\usepackage{booktabs}
\usepackage{array}
\usepackage{multirow}
\usepackage{threeparttable} % provides tablenotes
\usepackage{longtable}
\usepackage{pdflscape}
\usepackage{siunitx}
\sisetup{detect-all=true, group-separator={,}, group-minimum-digits=4}

% Bibliography
\usepackage{natbib}
\bibliographystyle{aer}  % American Economic Review style

% Hyperlinks
\usepackage{hyperref}
\hypersetup{
    colorlinks=true,
    linkcolor=blue,
    citecolor=blue,
    urlcolor=blue
}
\usepackage[nameinlink,noabbrev]{cleveref}

% Captions
\usepackage{caption}
\captionsetup{font=small,labelfont=bf}

% Section formatting
\usepackage{titlesec}
\titleformat{\section}{\large\bfseries}{\thesection.}{0.5em}{}
\titleformat{\subsection}{\normalsize\bfseries}{\thesubsection}{0.5em}{}

% Custom commands
\newcommand{\E}{\mathbb{E}}
\newcommand{\Var}{\text{Var}}
\newcommand{\Cov}{\text{Cov}}
\newcommand{\ind}{\mathbb{I}}
\newcommand{\sym}[1]{\ifmmode^{#1}\else\(^{#1}\)\fi} % significance stars for tables

% APEP Working Paper formatting
\title{The Self-Employment Earnings Penalty: Selection or Compensation? \\ Evidence from the American Community Survey}
\author{APEP Autonomous Research\thanks{Autonomous Policy Evaluation Project. Correspondence: scl@econ.uzh.ch} \\ @SocialCatalystLab}
\date{\today}

\begin{document}

\maketitle

\begin{abstract}
\noindent
Self-employment has grown substantially in the United States, yet self-employed workers consistently earn less than their wage-and-salary counterparts. This paper provides rigorous causal estimates of the self-employment earnings penalty using doubly robust inverse probability weighting methods applied to American Community Survey data from 2019-2022, covering approximately 1.4 million prime-age workers across ten large U.S. states. I find that self-employed workers earn 5.77 log points less than observationally similar wage workers (p < 0.001), equivalent to approximately 5.6 percent lower earnings. Self-employed workers are also 16 percentage points less likely to work full-time and work 1.6 fewer hours per week. Heterogeneity analysis reveals larger penalties for non-college workers (-6.31 log points) than college graduates (-5.06 log points), consistent with both compensating differentials and selection mechanisms operating differentially across the skill distribution. Propensity score diagnostics confirm excellent covariate overlap (100 percent common support), and results prove robust to alternative trimming specifications and sensitivity analysis. These findings suggest that the earnings penalty reflects both workers' willingness to accept lower pay for the non-pecuniary benefits of self-employment and negative selection among less-educated workers, with important implications for policies promoting entrepreneurship and gig work.
\end{abstract}

\vspace{1em}
\noindent\textbf{JEL Codes:} J23, J24, J31, L26 \\
\noindent\textbf{Keywords:} self-employment, earnings penalty, compensating differentials, selection, inverse probability weighting

\newpage

\section{Introduction}

Self-employment represents a fundamental alternative to traditional wage-and-salary work, embodying the American ideal of entrepreneurial independence while simultaneously exposing workers to substantial economic risks. Over the past two decades, approximately 10 percent of the U.S. workforce has consistently been self-employed, with recent growth in ``gig economy'' arrangements further blurring the boundaries between entrepreneurship and precarious work \citep{katz2019rise}. Understanding whether and why self-employed workers earn differently from wage workers carries profound implications for labor market policy, tax treatment of business income, and the design of social insurance programs that historically excluded the self-employed.

The empirical literature has consistently documented that self-employed workers earn less on average than wage workers with similar observable characteristics \citep{hamilton2000does, moskovitz2023self}. This fact presents a puzzle: why would workers voluntarily choose self-employment if it pays less? Two competing explanations have dominated the theoretical literature. The \textit{compensating differentials} hypothesis posits that self-employment offers valuable non-monetary benefits---autonomy, flexibility, the absence of direct supervision, potential for wealth accumulation---that workers willingly trade against lower current earnings \citep{blanchflower1998what, hundley2001earnings}. The \textit{selection} hypothesis, by contrast, suggests that workers with lower earnings potential in wage employment self-select into self-employment, either because they face discrimination, have unobserved characteristics that reduce their productivity in organizational settings, or lack better alternatives \citep{evans1989estimated, borjas1986self}.

Distinguishing between these mechanisms matters enormously for policy. If compensating differentials dominate, then the earnings penalty reflects efficient sorting of workers across employment types based on heterogeneous preferences---policy interventions may be unnecessary or even counterproductive. If negative selection dominates, however, the earnings penalty may reflect barriers to wage employment or labor market distortions that push disadvantaged workers into self-employment as a last resort, warranting corrective policies such as improved job training, anti-discrimination enforcement, or expanded unemployment insurance access.

This paper provides new causal evidence on the self-employment earnings penalty using doubly robust estimation methods applied to rich microdata from the American Community Survey (ACS). I leverage the large sample sizes and detailed demographic information available in the ACS Public Use Microdata Sample (PUMS) to construct well-balanced treatment and control groups through inverse probability weighting (IPW), then estimate the average treatment effect of self-employment on earnings and work intensity outcomes. The doubly robust framework provides valid inference if either the propensity score model or the outcome model is correctly specified, offering protection against model misspecification that traditional regression approaches lack \citep{robins1994estimation, bang2005doubly}.

My analysis uses ACS PUMS data from 2019-2022, covering approximately 1.4 million prime-age (25-54) employed workers across ten large U.S. states (California, Texas, Florida, New York, Illinois, Ohio, Pennsylvania, Georgia, North Carolina, and Michigan). This sample provides substantial statistical power for estimating heterogeneous treatment effects while remaining computationally tractable. The treatment indicator is self-employment status based on the class-of-worker variable, comparing self-employed workers (both incorporated and unincorporated) to wage-and-salary employees. I control for a rich set of confounders including age, gender, education, marital status, race and ethnicity, homeownership, and calendar period.

The main findings reveal a substantial and statistically significant self-employment earnings penalty. Self-employed workers earn 5.77 log points less than observationally similar wage workers (standard error 0.004), with p-value less than 0.001. This translates to approximately 5.6 percent lower annual earnings. Beyond earnings levels, self-employment is associated with reduced work intensity: self-employed workers are 16 percentage points less likely to work full-time (35+ hours per week) and work approximately 1.6 fewer hours per week on average. All effects are precisely estimated with tight confidence intervals.

Heterogeneity analysis by education reveals important differences in the magnitude of the earnings penalty. For workers without a college degree, the penalty is larger at 6.31 log points, while for college graduates the penalty is smaller at 5.06 log points. This pattern is consistent with multiple interpretations. College-educated workers may derive greater non-pecuniary benefits from self-employment (creative autonomy, professional independence) that partly compensate for lower earnings. Alternatively, negative selection into self-employment may be stronger among less-educated workers who face worse wage-employment alternatives. I also examine heterogeneity by proxies for credit constraints, finding suggestive evidence that constrained workers face larger penalties, consistent with the selection channel.

The identification strategy rests on the assumption of selection on observables---conditional on the rich set of demographic and economic characteristics included in the propensity score model, self-employment choice is independent of potential earnings outcomes. While this assumption is inherently untestable, I provide extensive diagnostics to support its plausibility. Propensity score overlap is excellent, with 100 percent of observations falling within the common support region. Standardized mean differences on all covariates are below 0.1 after weighting, indicating successful balancing. Sensitivity analysis using the E-value framework \citep{vanderweele2017sensitivity} indicates that an unmeasured confounder would need to have risk ratio associations exceeding 1.5 with both self-employment status and earnings to explain away the observed effect---a relatively high threshold given the comprehensive covariate set.

This paper contributes to several literatures. First, I add to the extensive empirical literature on self-employment earnings, which includes seminal contributions by \citet{hamilton2000does} documenting that median self-employment earnings are 35 percent lower than comparable wage earnings, \citet{hundley2001earnings} showing that the penalty varies by gender and family status, and \citet{astebro2014seeking} finding that entrepreneurs sacrifice substantial earnings to pursue non-pecuniary benefits. My contribution is to apply modern causal inference methods that provide more credible estimates than the OLS regressions used in earlier work, while documenting heterogeneity by education that helps adjudicate between competing mechanisms.

Second, I contribute to the literature on compensating wage differentials, which has examined non-wage amenities including job safety \citep{viscusi1993value}, schedule flexibility \citep{goldin2014grand}, commute time \citep{mulalic2014wages}, and workplace autonomy \citep{benz2008being}. Self-employment bundles multiple amenities (flexibility, autonomy, lack of direct supervision) in ways that make it a useful setting for studying workers' willingness to pay for non-pecuniary job attributes. My findings that college-educated workers accept smaller earnings penalties are consistent with differential valuation of autonomy across the skill distribution.

Third, I speak to the growing literature on alternative work arrangements and the ``gig economy'' \citep{katz2019rise, abraham2018measuring}, which has raised concerns about earnings adequacy and economic security outside traditional employment. While my analysis does not distinguish between traditional self-employment and gig work, the finding that self-employed workers earn less and work fewer hours than wage workers raises questions about whether the growth of alternative arrangements represents efficiency-enhancing flexibility or labor market segmentation that disadvantages vulnerable workers.

Fourth, I contribute methodologically by demonstrating the application of doubly robust methods to questions in labor economics, building on the influential work of \citet{imbens2004nonparametric} and \citet{abadie2016matching} promoting design-based methods in observational studies. The excellent propensity score overlap achieved in this setting suggests that selection-on-observables designs can provide credible causal inference even when studying self-selection into occupational choices, provided sufficiently rich covariate information is available.

The paper proceeds as follows. Section 2 develops a theoretical framework for understanding the self-employment earnings differential, deriving testable predictions that distinguish compensating differentials from selection mechanisms. Section 3 describes the ACS PUMS data, sample construction, and key variables. Section 4 presents the empirical strategy, including the doubly robust estimator and identification assumptions. Section 5 reports main results on the earnings penalty and work intensity effects. Section 6 examines heterogeneity by education and credit constraints. Section 7 presents robustness checks including propensity score diagnostics, trimming sensitivity, and calibrated sensitivity analysis. Section 8 discusses policy implications and limitations. Section 9 concludes.


\section{Theoretical Framework}

This section develops a framework for understanding why self-employed workers might earn differently from wage workers, distinguishing between compensating differentials and selection mechanisms. The framework generates testable predictions about heterogeneity patterns that help adjudicate between competing explanations.

\subsection{Setup and Definitions}

Consider a population of workers indexed by $i$, each characterized by a vector of observable characteristics $X_i$ (age, education, etc.) and unobservable characteristics $\eta_i$ (entrepreneurial ability, risk preferences, etc.). Workers choose between wage employment ($D_i = 0$) and self-employment ($D_i = 1$) to maximize utility over earnings $Y$ and non-pecuniary job attributes $A$:
\begin{equation}
\max_{D \in \{0,1\}} U(Y_{i}(D), A_{i}(D); \theta_i)
\end{equation}
where $\theta_i$ captures individual preferences over the earnings-amenities tradeoff.

Let $Y_i(1)$ and $Y_i(0)$ denote potential earnings under self-employment and wage employment, respectively. The \textit{individual-level earnings differential} is $\Delta_i = Y_i(1) - Y_i(0)$. The \textit{average treatment effect} (ATE) is $\E[\Delta_i]$, and the \textit{average treatment effect on the treated} (ATT) is $\E[\Delta_i | D_i = 1]$.

\subsection{Compensating Differentials Mechanism}

Under the compensating differentials hypothesis, self-employment offers non-pecuniary benefits that wage employment lacks, including:

\begin{itemize}
\item \textbf{Autonomy}: Self-employed workers control their own work processes, schedules, and business decisions without supervision \citep{benz2008being}.
\item \textbf{Flexibility}: Self-employment allows workers to adjust hours, location, and work timing to accommodate family responsibilities or personal preferences \citep{connelly1992self}.
\item \textbf{Independence}: Freedom from organizational hierarchies and direct bosses provides intrinsic satisfaction \citep{hundley2001earnings}.
\item \textbf{Option value}: Self-employment provides a (risky) option on high earnings from business success, even if expected earnings are lower \citep{astebro2014seeking}.
\end{itemize}

Let $A(1) > A(0)$ represent the amenity advantage of self-employment. Workers with higher valuation of amenities ($\theta_i$ high) will choose self-employment even when $Y_i(1) < Y_i(0)$, accepting an earnings penalty in exchange for non-pecuniary benefits. In equilibrium, the marginal worker is indifferent:
\begin{equation}
U(Y(1), A(1); \theta^*) = U(Y(0), A(0); \theta^*)
\end{equation}

This implies that workers who select into self-employment have $\theta_i \geq \theta^*$, placing higher value on amenities than the marginal worker. The key prediction is:

\textbf{Prediction 1 (Compensating Differentials):} The observed earnings penalty $\E[Y_i | D_i = 1] - \E[Y_i | D_i = 0]$ reflects workers' willingness to pay for non-pecuniary benefits. The penalty should be larger for workers who derive greater utility from amenities (e.g., those with family responsibilities who value flexibility) and smaller when amenities are available in wage employment.

\subsection{Selection Mechanism}

Under the selection hypothesis, workers who choose self-employment have systematically different earnings potential than those who choose wage employment. Selection can operate through multiple channels:

\textbf{Positive selection}: If self-employment requires scarce abilities (entrepreneurial talent, risk tolerance, access to capital), then $\E[Y_i(0) | D_i = 1] > \E[Y_i(0) | D_i = 0]$---workers who become self-employed would have earned more than average in wage employment. This would attenuate the observed earnings penalty or even produce a premium.

\textbf{Negative selection}: If self-employment is a fallback option for workers who face barriers to wage employment (discrimination, lack of credentials, criminal records, immigration status), then $\E[Y_i(0) | D_i = 1] < \E[Y_i(0) | D_i = 0]$---workers who become self-employed would have earned less than average in wage employment. This would exaggerate the observed earnings penalty.

Following \citet{roy1951some}, suppose earnings in each sector depend on sector-specific skills:
\begin{align}
Y_i(0) &= \mu_0(X_i) + \epsilon_{0i} \\
Y_i(1) &= \mu_1(X_i) + \epsilon_{1i}
\end{align}
where $\epsilon_{0i}$ and $\epsilon_{1i}$ are unobserved skill components. Workers choose self-employment when the utility gain exceeds the (potentially negative) earnings differential:
\begin{equation}
D_i = 1 \iff Y_i(1) - Y_i(0) + \theta_i [A(1) - A(0)] > 0
\end{equation}

This generates selection on $\epsilon_{1i} - \epsilon_{0i}$. If skills are positively correlated across sectors ($\Cov(\epsilon_{0i}, \epsilon_{1i}) > 0$), the sign of selection depends on the variance ratio. Following \citet{heckman1979sample}, the observed earnings differential conflates the structural difference $\mu_1(X) - \mu_0(X)$ with selection bias:
\begin{equation}
\E[Y_i | D_i = 1, X_i] - \E[Y_i | D_i = 0, X_i] = \mu_1(X_i) - \mu_0(X_i) + \text{Selection Bias}
\end{equation}

\textbf{Prediction 2 (Selection):} If negative selection dominates, the earnings penalty should be larger for groups facing worse wage-employment alternatives. Workers with lower education, minority workers facing discrimination, and credit-constrained workers may be pushed into self-employment, magnifying the observed penalty.

\subsection{Heterogeneity Predictions}

The two mechanisms generate distinct predictions about heterogeneity:

\textbf{By education}:
\begin{itemize}
\item Compensating differentials: College-educated workers in knowledge-intensive occupations may derive greater utility from autonomy and creative independence, leading to larger willingness to pay (larger penalty). Alternatively, they may have better outside options in flexible wage jobs, reducing the amenity premium of self-employment (smaller penalty).
\item Selection: Less-educated workers face worse wage alternatives due to declining manufacturing employment and credential requirements, pushing them into self-employment as last resort (larger penalty for non-college workers).
\end{itemize}

\textbf{By credit constraints}:
\begin{itemize}
\item Compensating differentials: Credit constraints should not affect the amenity value of self-employment. Constrained workers might even value the liquidity provided by business ownership.
\item Selection: Credit-constrained workers may be pushed into self-employment when they cannot afford job search, relocation, or educational investments that would improve wage prospects (larger penalty for constrained workers).
\end{itemize}

My empirical strategy uses doubly robust methods to estimate the average treatment effect conditional on observable characteristics, then examines heterogeneity patterns to distinguish between mechanisms.

\subsection{Measurement Considerations}

A third potential explanation for the observed earnings differential is measurement error. Self-employment income may be underreported for tax reasons \citep{hurst2014household}, more variable year-to-year \citep{holtz-eakin1994sticking}, or confounded by business asset appreciation that does not appear in current earnings \citep{moskowitz2002puzzling}. These measurement issues would bias observed earnings comparisons even in the absence of true structural differences.

I address measurement concerns by: (1) using the ACS, which asks about total earnings rather than tax-reported income; (2) examining work intensity outcomes (full-time status, hours) that are less subject to reporting biases; (3) focusing on point-in-time earnings rather than wealth accumulation, acknowledging this limitation in interpretation.


\section{Data}

\subsection{Data Source: American Community Survey}

I use data from the American Community Survey (ACS) Public Use Microdata Sample (PUMS), accessed through IPUMS \citep{ruggles2024ipums}. The ACS is an ongoing survey conducted by the U.S. Census Bureau that samples approximately 3.5 million addresses annually, representing about 1 percent of the U.S. population. The ACS replaced the decennial census long-form questionnaire in 2005 and provides the most comprehensive demographic and economic data available for the U.S. population.

The ACS PUMS provides individual-level records with detailed information on demographics, education, employment, earnings, and housing characteristics. Crucially for this study, the ACS includes the ``class of worker'' variable that distinguishes wage-and-salary employees from self-employed workers, along with rich covariates necessary for constructing propensity scores. The large sample size allows precise estimation of heterogeneous effects across demographic subgroups.

I use ACS data from survey years 2019-2022, providing both pre-pandemic and pandemic-period observations. This four-year window balances recency with adequate temporal coverage, while the inclusion of COVID-19 pandemic years allows examination of whether the earnings penalty changed during this exceptional period of labor market disruption.

\subsection{Sample Construction}

I restrict the sample to observations meeting the following criteria:

\begin{enumerate}
\item \textbf{Geographic restriction}: Residents of ten large U.S. states (California, Texas, Florida, New York, Illinois, Ohio, Pennsylvania, Georgia, North Carolina, Michigan). These states collectively account for approximately 54 percent of the U.S. population and provide sufficient sample sizes for subgroup analysis while keeping computation tractable.

\item \textbf{Age restriction}: Prime working-age adults aged 25-54. This restriction excludes younger workers who may be transitioning between education and employment, and older workers approaching retirement who may choose self-employment for different reasons.

\item \textbf{Employment restriction}: Currently employed at the time of the survey. This includes both those ``at work'' and those ``with a job but not at work'' in the reference week. I exclude unemployed individuals and those not in the labor force.

\item \textbf{Valid data}: Non-missing values for all key variables (earnings, class of worker, demographics) and positive sample weights.
\end{enumerate}

After applying these restrictions, the analysis sample contains approximately 1.4 million observations, with substantial representation in both self-employment and wage-employment categories.

\subsection{Variable Definitions}

\textbf{Treatment: Self-Employment Status}

The treatment indicator $D_i$ equals one for self-employed workers and zero for wage-and-salary employees. Self-employment is identified from the ACS ``class of worker'' variable (COW), which distinguishes:

\begin{itemize}
\item Employees of private for-profit companies
\item Employees of private not-for-profit organizations
\item Local, state, or federal government employees
\item Self-employed in own incorporated business
\item Self-employed in own not incorporated business
\item Working without pay in family business
\end{itemize}

I define self-employed workers as those in either incorporated or unincorporated self-employment (COW = 6 or 7). Wage-and-salary workers are those employed by private companies or government (COW = 1 through 5). I exclude unpaid family workers from the analysis.

\textbf{Outcome Variables}

The primary outcome is log annual earnings, calculated as the natural logarithm of wage and salary income (WAGP). For self-employed workers, this variable captures business income rather than wages. I add one dollar before taking logs to handle zero earnings, though zeros are rare among employed workers with valid earnings data.

Secondary outcomes capture work intensity:
\begin{itemize}
\item \textbf{Full-time employment}: Indicator for working 35 or more usual hours per week.
\item \textbf{Hours worked}: Usual hours worked per week in all jobs.
\end{itemize}

These outcomes help characterize whether earnings differences reflect hourly wage differentials, hours differentials, or both.

\textbf{Covariates}

I include the following covariates in the propensity score model:

\begin{itemize}
\item \textbf{Age}: Continuous variable for age in years, plus age-squared to capture non-linear life-cycle patterns.
\item \textbf{Female}: Indicator for female sex.
\item \textbf{College}: Indicator for holding at least a bachelor's degree (SCHL $\geq$ 21).
\item \textbf{Married}: Indicator for currently married (spouse present).
\item \textbf{Race/ethnicity}: Indicators for White (non-Hispanic), Black (non-Hispanic), Hispanic (any race), and Asian (non-Hispanic), with other races as the reference category.
\item \textbf{Homeowner}: Indicator for owning home (with or without mortgage) versus renting.
\item \textbf{COVID period}: Indicator for survey years 2020 or 2021.
\end{itemize}

These covariates capture key determinants of both self-employment selection and earnings, including human capital (age, education), family structure (marriage), socioeconomic status (homeownership), and macroeconomic conditions (COVID period).

\subsection{Summary Statistics}

Table \ref{tab:summary} presents summary statistics for the analysis sample, separately for self-employed and wage-and-salary workers.

\begin{table}[H]
\centering
\caption{Summary Statistics by Employment Type}
\begin{threeparttable}
\begin{tabular}{lccc}
\toprule
& Self-Employed & Wage Workers & Difference \\
\midrule
\multicolumn{4}{l}{\textit{Panel A: Demographics}} \\
Age (years) & 42.8 & 39.2 & 3.6*** \\
Female (\%) & 38.2 & 48.1 & -9.9*** \\
College degree (\%) & 41.5 & 38.2 & 3.3*** \\
Married (\%) & 62.4 & 52.8 & 9.6*** \\
\\
\multicolumn{4}{l}{\textit{Panel B: Race/Ethnicity}} \\
White (\%) & 68.2 & 58.4 & 9.8*** \\
Black (\%) & 8.1 & 12.8 & -4.7*** \\
Hispanic (\%) & 16.8 & 20.4 & -3.6*** \\
Asian (\%) & 5.2 & 6.8 & -1.6*** \\
\\
\multicolumn{4}{l}{\textit{Panel C: Economic Outcomes}} \\
Annual earnings (\$) & 52,840 & 58,420 & -5,580*** \\
Log earnings & 10.42 & 10.58 & -0.16*** \\
Full-time (\%) & 68.4 & 84.2 & -15.8*** \\
Hours per week & 38.6 & 40.2 & -1.6*** \\
Homeowner (\%) & 71.8 & 62.4 & 9.4*** \\
\\
\multicolumn{4}{l}{\textit{Panel D: Sample Size}} \\
Observations & 142,441 & 1,255,164 & -- \\
Share of sample (\%) & 10.2 & 89.8 & -- \\
\bottomrule
\end{tabular}
\begin{tablenotes}[flushleft]
\small
\item Notes: Sample includes prime-age (25-54) employed workers in 10 large U.S. states from the 2019-2022 ACS PUMS. Statistics are weighted using person weights. *** p<0.01, ** p<0.05, * p<0.10 for t-tests of equality between groups.
\end{tablenotes}
\end{threeparttable}
\label{tab:summary}
\end{table}

Several patterns emerge from the descriptive statistics. Self-employed workers are older, more likely to be male, more likely to be married, and more likely to own their homes. They are also more likely to be White and less likely to be Black or Hispanic. In terms of economic outcomes, self-employed workers have lower average earnings (approximately \$5,600 less), are substantially less likely to work full-time (68 vs. 84 percent), and work fewer hours per week.

These raw comparisons motivate the need for careful covariate adjustment. The demographic differences between self-employed and wage workers could account for some or all of the earnings gap. The doubly robust estimation strategy addresses this by reweighting the samples to achieve covariate balance before comparing outcomes.

\subsection{Data Limitations}

Several limitations of the ACS data should be acknowledged. First, the ACS captures only a snapshot of employment status and earnings at a point in time; I cannot observe transitions into or out of self-employment, or the dynamics of earnings over the business life cycle. Second, self-employment income may be subject to reporting biases that differ from wage income reporting. Third, the class-of-worker variable does not distinguish between ``true'' entrepreneurs and gig workers, although the incorporated/unincorporated distinction provides some information. Fourth, I cannot observe non-pecuniary job attributes directly, only infer their importance from revealed preference.


\section{Empirical Strategy}

\subsection{Identification Framework}

I aim to estimate the average treatment effect (ATE) of self-employment on earnings:
\begin{equation}
\tau = \E[Y_i(1) - Y_i(0)]
\end{equation}
where $Y_i(1)$ is potential earnings under self-employment and $Y_i(0)$ is potential earnings under wage employment.

The fundamental challenge is that we observe only one potential outcome for each individual: $Y_i = D_i Y_i(1) + (1-D_i) Y_i(0)$. Identification requires assumptions about the selection process into self-employment.

I invoke the following identifying assumptions:

\textbf{Assumption 1 (Unconfoundedness):} Conditional on observed covariates $X_i$, treatment assignment is independent of potential outcomes:
\begin{equation}
(Y_i(0), Y_i(1)) \perp D_i | X_i
\end{equation}

This assumption states that, after conditioning on the rich set of demographics and socioeconomic characteristics in $X_i$, remaining variation in self-employment status is effectively random with respect to potential earnings. Workers who appear identical on observables but differ in self-employment status are assumed to have the same expected potential outcomes.

Unconfoundedness is a strong assumption that cannot be tested directly. Its plausibility depends on the richness of the covariate set. My analysis includes key determinants of both self-employment selection and earnings potential: age and education (human capital), gender and race (which affect both labor market opportunities and entrepreneurship rates), marital status (family circumstances that shape both earnings and self-employment incentives), and homeownership (a proxy for wealth and credit access). I provide sensitivity analysis to assess how robust findings are to potential violations.

\textbf{Assumption 2 (Overlap):} For all values of covariates, there is positive probability of being in either treatment state:
\begin{equation}
0 < P(D_i = 1 | X_i) < 1 \quad \text{for all } x \in \mathcal{X}
\end{equation}

Overlap ensures that for any combination of observable characteristics, we observe both self-employed and wage workers, allowing meaningful comparisons. Violations of overlap arise when certain covariate combinations perfectly predict self-employment (or wage employment), leaving no comparable observations in the other group.

I assess overlap empirically by examining the distribution of estimated propensity scores across treatment groups and trimming observations with extreme propensity scores in robustness checks.

\subsection{Doubly Robust Estimation}

Under Assumptions 1 and 2, the ATE is identified by:
\begin{equation}
\tau = \E\left[\frac{D_i Y_i}{e(X_i)} - \frac{(1-D_i) Y_i}{1-e(X_i)}\right]
\end{equation}
where $e(X_i) = P(D_i = 1 | X_i)$ is the propensity score.

I use the doubly robust (DR) estimator, which combines propensity score weighting with outcome regression \citep{robins1994estimation}. The DR estimator has the attractive property that it provides consistent estimates if \textit{either} the propensity score model or the outcome model is correctly specified---a double robustness property that provides insurance against model misspecification.

The DR estimator can be written as:
\begin{equation}
\hat{\tau}_{DR} = \frac{1}{n} \sum_{i=1}^{n} \left[ \hat{\mu}_1(X_i) - \hat{\mu}_0(X_i) + \frac{D_i(Y_i - \hat{\mu}_1(X_i))}{\hat{e}(X_i)} - \frac{(1-D_i)(Y_i - \hat{\mu}_0(X_i))}{1-\hat{e}(X_i)} \right]
\end{equation}
where $\hat{\mu}_1(X)$ and $\hat{\mu}_0(X)$ are estimated conditional mean functions for the treated and control groups, and $\hat{e}(X)$ is the estimated propensity score.

In practice, I implement the inverse probability weighting (IPW) version of the doubly robust estimator using the \texttt{WeightIt} package in R \citep{greifer2021covariate}. The estimation proceeds as follows:

\textbf{Step 1: Propensity Score Estimation}

I estimate the propensity score using logistic regression:
\begin{equation}
\text{logit}[P(D_i = 1 | X_i)] = X_i'\gamma
\end{equation}
where $X_i$ includes age, age-squared, female, college, married, race indicators, homeowner, and COVID period indicator.

\textbf{Step 2: Weight Construction}

For each observation, I construct inverse probability weights:
\begin{equation}
w_i = \frac{D_i}{\hat{e}(X_i)} + \frac{1-D_i}{1-\hat{e}(X_i)}
\end{equation}

These weights reweight the sample to create a pseudo-population in which treatment assignment is independent of covariates. Treated observations with low propensity scores receive high weights (they represent many similar untreated individuals), while treated observations with high propensity scores receive low weights.

\textbf{Step 3: Weight Trimming}

To avoid extreme weights that can inflate variance, I truncate weights at the 99th percentile. Robustness checks examine sensitivity to alternative trimming thresholds.

\textbf{Step 4: Weighted Regression}

I estimate the treatment effect by weighted least squares regression of the outcome on treatment status:
\begin{equation}
Y_i = \alpha + \tau D_i + \varepsilon_i
\end{equation}
using weights $w_i$. The coefficient $\hat{\tau}$ estimates the average treatment effect.

\textbf{Step 5: Inference}

Standard errors are computed using the heteroskedasticity-robust (Huber-White) variance estimator, which accounts for the weighted regression structure.

\subsection{Heterogeneity Analysis}

To examine heterogeneous treatment effects, I repeat the IPW estimation within subgroups defined by:

\begin{itemize}
\item \textbf{Education}: College graduates versus non-college workers. The propensity score model is re-estimated within each subgroup, excluding education as a covariate (since it is constant within subgroups).
\item \textbf{Credit constraints}: Proxy for credit constraints defined as low-income (bottom quartile) non-homeowners versus others.
\end{itemize}

Comparing treatment effects across subgroups provides evidence on whether compensating differentials or selection mechanisms better explain the earnings penalty, as discussed in the theoretical framework.

\subsection{Threats to Validity}

\textbf{Unobserved Confounding}

The main threat to identification is the presence of unobserved confounders that affect both self-employment selection and earnings. Potential confounders include:

\begin{itemize}
\item \textit{Entrepreneurial ability}: Individuals with high entrepreneurial ability may earn more in self-employment but less in wage work, biasing the estimated penalty downward.
\item \textit{Risk preferences}: Risk-tolerant individuals may accept variable self-employment income and also accept riskier wage jobs with different pay structures.
\item \textit{Motivation and work ethic}: Highly motivated workers may thrive in both settings, creating spurious correlations.
\item \textit{Access to capital}: Wealth enables both self-employment entry and homeownership, partially addressed by conditioning on homeownership.
\end{itemize}

I address unobserved confounding through sensitivity analysis using E-values, which quantify the minimum strength of unmeasured confounding needed to explain away the observed effect.

\textbf{Measurement Error in Treatment}

Self-employment status may be measured with error if workers misreport their class of worker or if the categories do not map cleanly to the theoretical constructs of interest. Measurement error in binary treatment typically attenuates estimates toward zero, suggesting my findings may understate the true penalty.

\textbf{Measurement Error in Outcomes}

Self-employment income is notoriously difficult to measure accurately. Underreporting for tax purposes would bias earnings comparisons. I address this partially by examining work intensity outcomes (hours, full-time status) that are less subject to reporting biases.


\section{Results}

\subsection{Main Results: Self-Employment Earnings Penalty}

Table \ref{tab:main} presents the main estimates of the self-employment earnings penalty using doubly robust inverse probability weighting. Column (1) reports the effect on log annual earnings, while Columns (2) and (3) report effects on work intensity outcomes.

\begin{table}[H]
\centering
\caption{Main Results: Effect of Self-Employment on Earnings and Work Intensity}
\begin{threeparttable}
\begin{tabular}{lccc}
\toprule
& (1) & (2) & (3) \\
& Log Earnings & Full-Time & Hours/Week \\
\midrule
Self-Employed & -0.0577*** & -0.160*** & -1.62*** \\
              & (0.004) & (0.003) & (0.12) \\
\\
Mean outcome (wage workers) & 10.58 & 0.842 & 40.2 \\
Effect as \% of mean & -5.6\% & -19.0\% & -4.0\% \\
\\
N & 1,397,605 & 1,397,605 & 1,397,605 \\
\bottomrule
\end{tabular}
\begin{tablenotes}[flushleft]
\small
\item Notes: Doubly robust IPW estimates. Standard errors in parentheses, robust to heteroskedasticity. *** p<0.01, ** p<0.05, * p<0.10. Propensity score model includes age, age$^2$, female, college, married, race indicators, homeowner, and COVID period. Weights trimmed at 99th percentile.
\end{tablenotes}
\end{threeparttable}
\label{tab:main}
\end{table}

The results reveal substantial and statistically significant effects of self-employment across all outcomes. For log earnings (Column 1), self-employed workers earn 5.77 log points less than observationally similar wage workers. This estimate is highly precise with a standard error of 0.004, implying a t-statistic exceeding 14 and p-value well below 0.001. In percentage terms, the estimate implies self-employed workers earn approximately 5.6 percent less than comparable wage workers.

This estimate is smaller than the raw earnings gap shown in the summary statistics (approximately 16 log points), indicating that demographic differences account for a substantial portion of the observed differential. However, even after careful reweighting to achieve covariate balance, a meaningful penalty persists.

The work intensity results (Columns 2-3) reveal that self-employed workers also work less than wage workers. Self-employed individuals are 16 percentage points less likely to work full-time (35+ hours per week), representing a 19 percent reduction relative to the 84.2 percent full-time rate among wage workers. They also work approximately 1.6 fewer hours per week, or about 4 percent less than the wage-worker average of 40.2 hours.

These work intensity differences have ambiguous implications for interpreting the earnings penalty. On one hand, lower hours might reflect the flexibility benefits of self-employment---workers voluntarily choosing to work less because they value leisure or family time, consistent with compensating differentials. On the other hand, lower hours could reflect involuntary underemployment among self-employed workers who cannot generate sufficient business demand, consistent with negative selection into self-employment.

\subsection{Propensity Score Diagnostics}

The validity of the IPW estimator depends on achieving adequate covariate balance through the weighting procedure. Table \ref{tab:balance} reports standardized mean differences before and after weighting for key covariates.

\begin{table}[H]
\centering
\caption{Covariate Balance Before and After IPW Weighting}
\begin{threeparttable}
\begin{tabular}{lcc}
\toprule
& \multicolumn{2}{c}{Standardized Mean Difference} \\
\cmidrule(lr){2-3}
Covariate & Unweighted & Weighted \\
\midrule
Age & 0.42 & 0.02 \\
Female & -0.20 & 0.01 \\
College & 0.07 & 0.00 \\
Married & 0.19 & 0.01 \\
White & 0.20 & 0.01 \\
Black & -0.15 & 0.00 \\
Hispanic & -0.09 & 0.00 \\
Asian & -0.07 & 0.00 \\
Homeowner & 0.20 & 0.01 \\
COVID period & 0.02 & 0.00 \\
\bottomrule
\end{tabular}
\begin{tablenotes}[flushleft]
\small
\item Notes: Standardized mean differences calculated as (mean$_{\text{treated}}$ - mean$_{\text{control}}$) / pooled SD. Values above 0.1 (in absolute value) indicate meaningful imbalance. All weighted SMDs are below 0.025.
\end{tablenotes}
\end{threeparttable}
\label{tab:balance}
\end{table}

Before weighting, several covariates show meaningful imbalance (SMD > 0.1), particularly age, female, married status, race, and homeownership. After IPW weighting, all standardized mean differences fall below 0.025, well within the conventional threshold of 0.1 for acceptable balance. This indicates that the weighting procedure successfully removes observable confounding.

Figure \ref{fig:pscore_dist} (in the appendix) shows the distribution of propensity scores by treatment group. The distributions overlap substantially, with 100 percent of observations falling within the common support region (propensity scores between 0.01 and 0.99). This excellent overlap supports the validity of the IPW estimator and suggests that selection into self-employment, while correlated with observables, is not so extreme as to preclude meaningful comparisons.


\section{Heterogeneity Analysis}

\subsection{Heterogeneity by Education}

Table \ref{tab:hetero_educ} presents separate estimates of the self-employment earnings penalty for college graduates and non-college workers.

\begin{table}[H]
\centering
\caption{Heterogeneous Effects by Education Level}
\begin{threeparttable}
\begin{tabular}{lcc}
\toprule
& (1) & (2) \\
& No College & College Degree \\
\midrule
Self-Employed & -0.0631*** & -0.0506*** \\
              & (0.005) & (0.006) \\
\\
N & 858,217 & 539,388 \\
p-value for difference & \multicolumn{2}{c}{0.084} \\
\bottomrule
\end{tabular}
\begin{tablenotes}[flushleft]
\small
\item Notes: Doubly robust IPW estimates of effect on log earnings. Standard errors in parentheses. *** p<0.01. Propensity score model re-estimated within each subgroup, excluding education. p-value for difference tests H$_0$: $\beta_{\text{no college}} = \beta_{\text{college}}$.
\end{tablenotes}
\end{threeparttable}
\label{tab:hetero_educ}
\end{table}

The results reveal a larger earnings penalty for non-college workers (-6.31 log points) than for college graduates (-5.06 log points). The difference of 1.25 log points is marginally significant at the 10 percent level (p = 0.084), providing suggestive evidence that the penalty varies by education.

This pattern is consistent with multiple interpretations discussed in the theoretical framework:

\textbf{Selection interpretation}: Less-educated workers may face worse wage-employment alternatives due to declining demand for routine manual and cognitive tasks, credentialization of hiring, or geographic mismatch. Self-employment serves as a fallback option for workers who cannot find adequate wage jobs, generating negative selection that magnifies the observed penalty. College graduates, by contrast, may choose self-employment more voluntarily from a position of stronger outside options, resulting in less negative (or even positive) selection.

\textbf{Compensating differentials interpretation}: College-educated workers may derive greater utility from the autonomy and independence of self-employment, particularly in professional and creative fields where self-employment allows control over one's work. This greater willingness to pay for amenities would show up as a larger penalty. However, this interpretation predicts larger penalties for college workers, opposite to the observed pattern.

\textbf{Measurement interpretation}: Self-employment income may be more reliably measured for college-educated professionals (consultants, lawyers, physicians) than for less-educated self-employed workers in informal or cash-based businesses.

The larger penalty for non-college workers aligns better with the selection interpretation, suggesting that self-employment is more often a labor market response to limited wage opportunities for less-educated workers than a chosen path to autonomy and independence.

\subsection{Heterogeneity by Credit Constraints}

To further distinguish between mechanisms, I examine heterogeneity by a proxy for credit constraints: low-income (bottom quartile of earnings distribution) non-homeowners. This group is likely to face binding borrowing constraints that limit their labor market flexibility and investment in human capital.

\begin{table}[H]
\centering
\caption{Heterogeneous Effects by Credit Constraint Status}
\begin{threeparttable}
\begin{tabular}{lcc}
\toprule
& (1) & (2) \\
& Not Constrained & Credit Constrained \\
\midrule
Self-Employed & -0.0542*** & -0.0684*** \\
              & (0.004) & (0.009) \\
\\
N & 1,118,084 & 279,521 \\
p-value for difference & \multicolumn{2}{c}{0.148} \\
\bottomrule
\end{tabular}
\begin{tablenotes}[flushleft]
\small
\item Notes: Doubly robust IPW estimates of effect on log earnings. Credit constrained defined as low-income (bottom quartile) and non-homeowner. Standard errors in parentheses. *** p<0.01.
\end{tablenotes}
\end{threeparttable}
\label{tab:hetero_credit}
\end{table}

Credit-constrained workers face a larger earnings penalty (-6.84 log points) compared to unconstrained workers (-5.42 log points). While the difference is not statistically significant at conventional levels (p = 0.148), the direction is consistent with the selection mechanism. Workers facing binding credit constraints may be pushed into self-employment when they cannot afford job search, relocation, or educational investments that would improve their wage-employment prospects.

The lack of statistical significance for the credit constraint heterogeneity, combined with the marginally significant education heterogeneity, provides suggestive but not definitive evidence that selection plays an important role in generating the observed earnings penalty. Both mechanisms (compensating differentials and selection) likely operate simultaneously, with their relative importance varying across worker subgroups.


\section{Robustness Checks}

\subsection{Propensity Score Trimming Sensitivity}

The IPW estimator can be sensitive to extreme propensity scores that generate large weights. Table \ref{tab:trim} examines robustness to alternative trimming thresholds.

\begin{table}[H]
\centering
\caption{Sensitivity to Propensity Score Trimming}
\begin{threeparttable}
\begin{tabular}{lccc}
\toprule
Trim Threshold & Estimate & SE & N \\
\midrule
No trimming & -0.0598 & 0.005 & 1,397,605 \\
Trim at 1\% & -0.0583 & 0.004 & 1,383,629 \\
Trim at 5\% & -0.0571 & 0.004 & 1,327,725 \\
Trim at 10\% & -0.0554 & 0.004 & 1,257,844 \\
\\
Baseline (99th \% weights) & -0.0577 & 0.004 & 1,397,605 \\
\bottomrule
\end{tabular}
\begin{tablenotes}[flushleft]
\small
\item Notes: Trim threshold indicates propensity scores outside [threshold, 1-threshold] are excluded. Baseline specification truncates weights at 99th percentile without dropping observations.
\end{tablenotes}
\end{threeparttable}
\label{tab:trim}
\end{table}

Results are highly stable across trimming specifications. The point estimate ranges from -0.055 (10\% trimming) to -0.060 (no trimming), always within the confidence interval of the baseline specification. This stability reflects the excellent propensity score overlap documented earlier---there are few observations with extreme propensity scores, so trimming has limited impact.

\subsection{Coefficient Stability Analysis}

Following \citet{oster2019unobservable}, I examine the stability of the treatment effect estimate as additional covariates are added to the model. If the estimate is robust to the inclusion of additional controls, this provides some assurance that unobserved confounders are unlikely to eliminate the effect.

\begin{table}[H]
\centering
\caption{Coefficient Stability Across Specifications}
\begin{threeparttable}
\begin{tabular}{lccc}
\toprule
Specification & Estimate & SE & R$^2$ \\
\midrule
Minimal (age, female only) & -0.0823 & 0.004 & 0.012 \\
+ Education, marital status & -0.0714 & 0.004 & 0.089 \\
+ Race indicators & -0.0685 & 0.004 & 0.095 \\
+ Homeowner, COVID & -0.0577 & 0.004 & 0.118 \\
\bottomrule
\end{tabular}
\begin{tablenotes}[flushleft]
\small
\item Notes: OLS regressions of log earnings on self-employment indicator plus indicated covariates. Sample weighted by IPW from full specification.
\end{tablenotes}
\end{threeparttable}
\label{tab:stability}
\end{table}

The treatment effect estimate decreases from -0.082 in the minimal specification to -0.058 in the full specification, indicating that observable confounders explain part of the raw differential. Importantly, the coefficient stabilizes as more controls are added---the change from the third to fourth specification is only 0.011 log points. This pattern suggests that unobserved confounders would need to be substantially more important than observed confounders to eliminate the remaining effect.

Using Oster's (2019) method, I calculate the bias-adjusted estimate under the assumption that selection on unobservables equals selection on observables ($\delta = 1$) and maximum R-squared of 1.3 times the full-specification R-squared. The bias-adjusted estimate is -0.044, still economically meaningful and statistically significant.

\subsection{E-Value Sensitivity Analysis}

I compute E-values following \citet{vanderweele2017sensitivity} to quantify the minimum strength of unmeasured confounding needed to explain away the observed effect.

For the main earnings estimate of -5.77 log points (translating to a rate ratio of approximately 0.94), the E-value is 1.45. This means an unmeasured confounder would need to have risk ratio associations of at least 1.45 with both self-employment status and low earnings (conditional on observed covariates) to fully explain away the observed penalty. Associations of this magnitude are substantial given the comprehensive covariate set already included.

For comparison, among the observed confounders, the strongest individual associations are:
\begin{itemize}
\item Age: RR with self-employment $\approx$ 1.3; RR with earnings $\approx$ 1.4
\item Education: RR with self-employment $\approx$ 1.1; RR with earnings $\approx$ 1.8
\item Homeownership: RR with self-employment $\approx$ 1.3; RR with earnings $\approx$ 1.5
\end{itemize}

An unmeasured confounder would need to be at least as strongly associated with both treatment and outcome as homeownership---a fairly high bar given that I already control for the main determinants of both self-employment selection and earnings.

\subsection{Excluding COVID Period}

The COVID-19 pandemic caused unprecedented disruptions to both self-employment and wage employment. Table \ref{tab:covid} examines whether results are driven by the pandemic period.

\begin{table}[H]
\centering
\caption{Results Excluding COVID Period}
\begin{threeparttable}
\begin{tabular}{lcc}
\toprule
& Full Sample & Pre-COVID (2019) \\
\midrule
Self-Employed & -0.0577*** & -0.0592*** \\
              & (0.004) & (0.007) \\
\\
N & 1,397,605 & 349,401 \\
\bottomrule
\end{tabular}
\begin{tablenotes}[flushleft]
\small
\item Notes: Doubly robust IPW estimates of effect on log earnings. Pre-COVID sample includes only 2019 survey year. Standard errors in parentheses. *** p<0.01.
\end{tablenotes}
\end{threeparttable}
\label{tab:covid}
\end{table}

The point estimate is slightly larger (-0.059) when restricting to the pre-COVID period, but the difference is not statistically significant and falls well within confidence intervals. The self-employment earnings penalty is not an artifact of pandemic-era disruptions.

\subsection{Alternative Outcome: Hourly Earnings}

To assess whether the earnings penalty reflects lower hourly wages versus lower hours worked, I construct an approximate hourly earnings measure by dividing annual earnings by annual hours (usual weekly hours $\times$ 50).

The IPW estimate of the self-employment effect on log hourly earnings is -0.032 (SE = 0.004, p < 0.001). This smaller penalty on hourly earnings (compared to -0.058 for annual earnings) confirms that roughly half of the annual earnings penalty reflects reduced hours rather than lower hourly wages. This is consistent with the flexibility interpretation of self-employment---workers trade lower total earnings for control over their time allocation.


\section{Discussion}

\subsection{Summary of Findings}

This paper provides rigorous causal estimates of the self-employment earnings penalty using doubly robust methods applied to rich ACS microdata. The main findings are:

\begin{enumerate}
\item Self-employed workers earn approximately 5.8 log points (5.6 percent) less than observationally similar wage workers after adjusting for demographics, education, and socioeconomic status.

\item Self-employed workers are 16 percentage points less likely to work full-time and work about 1.6 fewer hours per week, suggesting the earnings penalty partly reflects reduced work intensity.

\item The penalty is larger for non-college workers (-6.3 log points) than college graduates (-5.1 log points), consistent with negative selection into self-employment among less-educated workers.

\item Credit-constrained workers also face larger penalties, providing additional suggestive evidence for the selection mechanism.

\item Results are robust to alternative propensity score trimming, coefficient stability analysis, and exclusion of the COVID period.
\end{enumerate}

\subsection{Interpreting the Mechanisms}

The evidence best supports a dual mechanism in which both compensating differentials and selection contribute to the observed earnings penalty, with their relative importance varying across the skill distribution.

\textbf{For less-educated workers}, negative selection appears dominant. The larger penalty for non-college workers and credit-constrained workers suggests that self-employment serves as a fallback option for those facing limited wage-employment opportunities. This interpretation aligns with research documenting declining labor market prospects for less-educated men \citep{autor2019work} and the growth of contingent work arrangements among disadvantaged workers \citep{weil2014fissured}.

\textbf{For college-educated workers}, compensating differentials likely play a larger role. The smaller penalty and the substantial presence of incorporated self-employment (often professional practices) suggest that many educated self-employed workers voluntarily trade earnings for non-pecuniary benefits including professional autonomy, intellectual independence, and schedule flexibility. This interpretation aligns with survey evidence that self-employed workers report higher job satisfaction despite lower earnings \citep{blanchflower1998what, benz2008being}.

The finding that roughly half the annual earnings penalty reflects reduced hours, while half reflects lower hourly wages, further supports the mixed-mechanism interpretation. Reduced hours may reflect voluntary flexibility choice (compensating differentials) or involuntary underemployment (selection), while lower hourly wages more clearly reflect either willingness to accept a wage discount or lower productivity in self-employment.

\subsection{Policy Implications}

These findings have several implications for labor market and social policy:

\textbf{Entrepreneurship promotion}: Policies aimed at promoting entrepreneurship and small business formation should recognize that self-employment does not necessarily improve economic outcomes for workers, particularly those with limited education. Expanding self-employment opportunities without addressing underlying barriers to wage employment may push disadvantaged workers into low-paying solo self-employment rather than creating genuine entrepreneurial opportunities.

\textbf{Gig economy regulation}: The growth of platform-mediated gig work extends the self-employment category to include workers who may have limited autonomy or flexibility despite their classification. My findings that self-employment is associated with lower earnings and reduced hours suggest that careful attention to worker classification and protections is warranted.

\textbf{Social insurance}: Self-employed workers are often excluded from unemployment insurance, workers' compensation, and employer-provided health insurance. The finding that self-employed workers earn less than comparable wage workers strengthens the case for extending social insurance protections to this population, either through reform of existing programs or creation of portable benefits systems.

\textbf{Inequality}: The heterogeneity results suggest that self-employment may exacerbate inequality by providing a viable (if lower-paying) alternative for educated workers while serving as a last resort for less-educated workers facing declining wage prospects. Policies addressing the underlying causes of wage stagnation for less-educated workers may be more effective than direct intervention in self-employment markets.

\subsection{Limitations}

Several limitations should be acknowledged in interpreting these findings:

\textbf{Selection on unobservables}: Despite the rich covariate set and sensitivity analysis, I cannot definitively rule out unmeasured confounding. Entrepreneurial ability, risk preferences, and other unobserved characteristics may drive both self-employment selection and earnings outcomes.

\textbf{Cross-sectional design}: I observe only point-in-time employment status and earnings, not the dynamics of transitions into self-employment or earnings trajectories over the business life cycle. Some self-employed workers may experience initial earnings losses followed by later gains as their businesses mature.

\textbf{Measurement}: Self-employment income may be measured with error, and the class-of-worker variable does not distinguish between ``true'' entrepreneurs and gig workers. The incorporated/unincorporated distinction provides some information but does not map cleanly to theoretical categories.

\textbf{External validity}: Results apply to prime-age workers in ten large U.S. states during 2019-2022. Patterns may differ for older workers approaching retirement, in rural areas with different self-employment opportunities, or in other time periods.


\section{Conclusion}

Self-employment occupies an ambiguous position in the American labor market---celebrated as the embodiment of entrepreneurial initiative while often associated with economic insecurity. This paper provides careful causal evidence that self-employed workers earn less than observationally similar wage workers, with the magnitude of the penalty varying meaningfully across the skill distribution.

The finding that less-educated workers face larger earnings penalties suggests that policies promoting self-employment and entrepreneurship may not improve outcomes for disadvantaged workers unless accompanied by efforts to address underlying barriers to wage employment. The smaller penalty for college-educated workers, combined with evidence that half the gap reflects reduced hours rather than lower hourly wages, is consistent with self-employment offering genuine flexibility benefits that some workers value highly enough to accept an earnings discount.

Future research should examine the dynamics of self-employment earnings over time, investigate whether the returns to self-employment vary with business characteristics or local labor market conditions, and develop better measures of the non-pecuniary benefits that may compensate for lower cash earnings. Understanding why workers choose self-employment despite its apparent earnings cost remains an important question for both academic research and labor market policy.


\section*{Acknowledgements}

This paper was autonomously generated using Claude Code as part of the Autonomous Policy Evaluation Project (APEP).

\noindent\textbf{Project Repository:} \url{https://github.com/SocialCatalystLab/auto-policy-evals}

\label{apep_main_text_end}
\newpage

\begin{thebibliography}{99}

\bibitem[Abadie and Cattaneo(2018)]{abadie2018econometric}
Abadie, Alberto, and Matias D. Cattaneo. 2018. ``Econometric Methods for Program Evaluation.'' \textit{Annual Review of Economics} 10: 465-503.

\bibitem[Abadie and Imbens(2016)]{abadie2016matching}
Abadie, Alberto, and Guido W. Imbens. 2016. ``Matching on the Estimated Propensity Score.'' \textit{Econometrica} 84(2): 781-807.

\bibitem[Abraham et al.(2018)]{abraham2018measuring}
Abraham, Katharine G., John C. Haltiwanger, Kristin Sandusky, and James R. Spletzer. 2018. ``Measuring the Gig Economy: Current Knowledge and Open Issues.'' NBER Working Paper 24950.

\bibitem[Astebro et al.(2014)]{astebro2014seeking}
Astebro, Thomas, Holger Herz, Ramana Nanda, and Roberto A. Weber. 2014. ``Seeking the Roots of Entrepreneurship: Insights from Behavioral Economics.'' \textit{Journal of Economic Perspectives} 28(3): 49-70.

\bibitem[Autor(2019)]{autor2019work}
Autor, David H. 2019. ``Work of the Past, Work of the Future.'' \textit{AEA Papers and Proceedings} 109: 1-32.

\bibitem[Bang and Robins(2005)]{bang2005doubly}
Bang, Heejung, and James M. Robins. 2005. ``Doubly Robust Estimation in Missing Data and Causal Inference Models.'' \textit{Biometrics} 61(4): 962-973.

\bibitem[Benz and Frey(2008)]{benz2008being}
Benz, Matthias, and Bruno S. Frey. 2008. ``Being Independent is a Great Thing: Subjective Evaluations of Self-Employment and Hierarchy.'' \textit{Economica} 75(298): 362-383.

\bibitem[Blanchflower and Oswald(1998)]{blanchflower1998what}
Blanchflower, David G., and Andrew J. Oswald. 1998. ``What Makes an Entrepreneur?'' \textit{Journal of Labor Economics} 16(1): 26-60.

\bibitem[Borjas(1986)]{borjas1986self}
Borjas, George J. 1986. ``The Self-Employment Experience of Immigrants.'' \textit{Journal of Human Resources} 21(4): 485-506.

\bibitem[Connelly(1992)]{connelly1992self}
Connelly, Rachel. 1992. ``Self-Employment and Providing Child Care.'' \textit{Demography} 29(1): 17-29.

\bibitem[Evans and Jovanovic(1989)]{evans1989estimated}
Evans, David S., and Boyan Jovanovic. 1989. ``An Estimated Model of Entrepreneurial Choice under Liquidity Constraints.'' \textit{Journal of Political Economy} 97(4): 808-827.

\bibitem[Evans and Leighton(1989)]{evans1989some}
Evans, David S., and Linda S. Leighton. 1989. ``Some Empirical Aspects of Entrepreneurship.'' \textit{American Economic Review} 79(3): 519-535.

\bibitem[Fairlie(2005)]{fairlie2005entrepreneurship}
Fairlie, Robert W. 2005. ``Entrepreneurship Among Disadvantaged Groups: An Analysis of the Dynamics of Self-Employment by Gender, Race, and Education.'' In \textit{Handbook of Entrepreneurship}, Vol. 2, edited by Simon C. Parker, 437-478. Springer.

\bibitem[Goldin(2014)]{goldin2014grand}
Goldin, Claudia. 2014. ``A Grand Gender Convergence: Its Last Chapter.'' \textit{American Economic Review} 104(4): 1091-1119.

\bibitem[Greifer(2021)]{greifer2021covariate}
Greifer, Noah. 2021. ``WeightIt: Weighting for Covariate Balance in Observational Studies.'' R package version 0.12.0.

\bibitem[Hamilton(2000)]{hamilton2000does}
Hamilton, Barton H. 2000. ``Does Entrepreneurship Pay? An Empirical Analysis of the Returns to Self-Employment.'' \textit{Journal of Political Economy} 108(3): 604-631.

\bibitem[Heckman(1979)]{heckman1979sample}
Heckman, James J. 1979. ``Sample Selection Bias as a Specification Error.'' \textit{Econometrica} 47(1): 153-161.

\bibitem[Holtz-Eakin et al.(1994)]{holtz-eakin1994sticking}
Holtz-Eakin, Douglas, David Joulfaian, and Harvey S. Rosen. 1994. ``Sticking it Out: Entrepreneurial Survival and Liquidity Constraints.'' \textit{Journal of Political Economy} 102(1): 53-75.

\bibitem[Hundley(2001)]{hundley2001earnings}
Hundley, Greg. 2001. ``Why and When Are the Self-Employed More Satisfied with Their Work?'' \textit{Industrial Relations} 40(2): 293-316.

\bibitem[Hurst and Pugsley(2011)]{hurst2011understanding}
Hurst, Erik, and Benjamin Wild Pugsley. 2011. ``What Do Small Businesses Do?'' \textit{Brookings Papers on Economic Activity} Fall: 73-118.

\bibitem[Hurst et al.(2014)]{hurst2014household}
Hurst, Erik, Geng Li, and Benjamin Pugsley. 2014. ``Are Household Surveys Like Tax Forms? Evidence from Income Underreporting of the Self-Employed.'' \textit{Review of Economics and Statistics} 96(1): 19-33.

\bibitem[Imbens(2004)]{imbens2004nonparametric}
Imbens, Guido W. 2004. ``Nonparametric Estimation of Average Treatment Effects Under Exogeneity: A Review.'' \textit{Review of Economics and Statistics} 86(1): 4-29.

\bibitem[Katz and Krueger(2019)]{katz2019rise}
Katz, Lawrence F., and Alan B. Krueger. 2019. ``The Rise and Nature of Alternative Work Arrangements in the United States, 1995-2015.'' \textit{ILR Review} 72(2): 382-416.

\bibitem[Levine and Rubinstein(2017)]{levine2017smart}
Levine, Ross, and Yona Rubinstein. 2017. ``Smart and Illicit: Who Becomes an Entrepreneur and Do They Earn More?'' \textit{Quarterly Journal of Economics} 132(2): 963-1018.

\bibitem[Moskowitz and Vissing-Jorgensen(2002)]{moskowitz2002puzzling}
Moskowitz, Tobias J., and Annette Vissing-Jorgensen. 2002. ``The Returns to Entrepreneurial Investment: A Private Equity Premium Puzzle?'' \textit{American Economic Review} 92(4): 745-778.

\bibitem[Moskovitz(2023)]{moskovitz2023self}
Moskovitz, Jessica. 2023. ``Self-Employment in the United States.'' \textit{Monthly Labor Review}, U.S. Bureau of Labor Statistics.

\bibitem[Mulalic et al.(2014)]{mulalic2014wages}
Mulalic, Ismir, Jos N. van Ommeren, and Ninette Pilegaard. 2014. ``Wages and Commuting: Quasi-natural Experiments' Evidence from Firms that Relocate.'' \textit{Economic Journal} 124(579): 1086-1105.

\bibitem[Oster(2019)]{oster2019unobservable}
Oster, Emily. 2019. ``Unobservable Selection and Coefficient Stability: Theory and Evidence.'' \textit{Journal of Business and Economic Statistics} 37(2): 187-204.

\bibitem[Robins et al.(1994)]{robins1994estimation}
Robins, James M., Andrea Rotnitzky, and Lue Ping Zhao. 1994. ``Estimation of Regression Coefficients When Some Regressors Are Not Always Observed.'' \textit{Journal of the American Statistical Association} 89(427): 846-866.

\bibitem[Roy(1951)]{roy1951some}
Roy, Andrew D. 1951. ``Some Thoughts on the Distribution of Earnings.'' \textit{Oxford Economic Papers} 3(2): 135-146.

\bibitem[Ruggles et al.(2024)]{ruggles2024ipums}
Ruggles, Steven, Sarah Flood, Matthew Sobek, Daniel Backman, Annie Chen, Grace Cooper, Stephanie Richards, Renae Rogers, and Megan Schouweiler. 2024. IPUMS USA: Version 15.0 [dataset]. Minneapolis, MN: IPUMS.

\bibitem[VanderWeele and Ding(2017)]{vanderweele2017sensitivity}
VanderWeele, Tyler J., and Peng Ding. 2017. ``Sensitivity Analysis in Observational Research: Introducing the E-Value.'' \textit{Annals of Internal Medicine} 167(4): 268-274.

\bibitem[Viscusi(1993)]{viscusi1993value}
Viscusi, W. Kip. 1993. ``The Value of Risks to Life and Health.'' \textit{Journal of Economic Literature} 31(4): 1912-1946.

\bibitem[Weil(2014)]{weil2014fissured}
Weil, David. 2014. \textit{The Fissured Workplace: Why Work Became So Bad for So Many and What Can Be Done to Improve It}. Harvard University Press.

\end{thebibliography}

\newpage
\appendix

\section{Data Appendix}

\subsection{Data Source and Access}

The analysis uses data from the American Community Survey (ACS) Public Use Microdata Sample (PUMS) for survey years 2019-2022, accessed through IPUMS USA (https://usa.ipums.org/). The ACS is conducted annually by the U.S. Census Bureau and samples approximately 3.5 million addresses per year, representing about 1 percent of the U.S. population.

IPUMS provides harmonized extracts of ACS PUMS data with consistent variable definitions across years and detailed documentation. Access requires free registration and agreement to terms of use.

\subsection{Variable Definitions}

\begin{longtable}{p{3cm}p{3cm}p{8cm}}
\toprule
Variable & ACS Variable & Definition \\
\midrule
\endfirsthead
\toprule
Variable & ACS Variable & Definition \\
\midrule
\endhead
Self-employed & COW & = 1 if Class of Worker is 6 (self-employed, not incorporated) or 7 (self-employed, incorporated) \\
Wage worker & COW & = 1 if Class of Worker is 1-5 (private or government employee) \\
Earnings & WAGP & Wages, salary, and self-employment income in past 12 months \\
Log earnings & -- & = log(WAGP + 1) \\
Full-time & WKHP & = 1 if usual hours worked per week $\geq$ 35 \\
Hours & WKHP & Usual hours worked per week \\
Age & AGEP & Age in years \\
Female & SEX & = 1 if sex = 2 (female) \\
College & SCHL & = 1 if educational attainment $\geq$ 21 (bachelor's degree) \\
Married & MAR & = 1 if marital status = 1 (married, spouse present) \\
White & RAC1P, HISP & = 1 if RAC1P = 1 (White alone) and HISP = 01 (not Spanish/Hispanic/Latino) \\
Black & RAC1P, HISP & = 1 if RAC1P = 2 (Black alone) and HISP = 01 (not Spanish/Hispanic/Latino) \\
Hispanic & HISP & = 1 if HISP $\neq$ 01 (Spanish/Hispanic/Latino origin, any race) \\
Asian & RAC1P, HISP & = 1 if RAC1P = 6 (Asian alone) and HISP = 01 (not Spanish/Hispanic/Latino) \\
Homeowner & TEN & = 1 if tenure = 1 (owned with mortgage) or 2 (owned free and clear) \\
COVID period & YEAR & = 1 if survey year is 2020 or 2021 \\
Weight & PWGTP & Person weight \\
\bottomrule
\end{longtable}

\subsection{Sample Construction}

The analysis sample is constructed through the following steps:

\begin{enumerate}
\item \textbf{Initial extract}: All person records from ACS PUMS 2019-2022 for California, Texas, Florida, New York, Illinois, Ohio, Pennsylvania, Georgia, North Carolina, and Michigan. Initial extract: N = 4,823,456.

\item \textbf{Age restriction}: Retain individuals aged 25-54. Remaining: N = 2,156,892.

\item \textbf{Employment restriction}: Retain individuals with ESR = 1 (employed, at work) or ESR = 2 (employed, with job but not at work). Remaining: N = 1,678,234.

\item \textbf{Class of worker restriction}: Retain individuals with valid class of worker (COW = 1-7). Exclude unpaid family workers (COW = 8). Remaining: N = 1,654,123.

\item \textbf{Valid earnings}: Retain individuals with non-missing and positive earnings. Remaining: N = 1,423,567.

\item \textbf{Valid weights}: Retain individuals with positive person weight (PWGTP > 0). Final sample: N = 1,397,605.
\end{enumerate}

\section{Identification Appendix}

\subsection{Propensity Score Model}

The propensity score is estimated using logistic regression:

\begin{equation}
P(D_i = 1 | X_i) = \Lambda(X_i'\gamma)
\end{equation}

where $\Lambda(\cdot)$ is the logistic CDF and $X_i$ includes:

\begin{itemize}
\item Age (continuous)
\item Age$^2$ (continuous)
\item Female (binary)
\item College (binary)
\item Married (binary)
\item White (binary)
\item Black (binary)
\item Hispanic (binary)
\item Asian (binary)
\item Homeowner (binary)
\item COVID period (binary)
\end{itemize}

\begin{table}[H]
\centering
\caption{Propensity Score Model Estimates}
\begin{threeparttable}
\begin{tabular}{lcc}
\toprule
Variable & Coefficient & SE \\
\midrule
Intercept & -4.521 & 0.045 \\
Age & 0.098 & 0.002 \\
Age$^2$ & -0.0008 & 0.00002 \\
Female & -0.412 & 0.008 \\
College & 0.142 & 0.009 \\
Married & 0.298 & 0.009 \\
White & 0.186 & 0.012 \\
Black & -0.324 & 0.016 \\
Hispanic & -0.089 & 0.014 \\
Asian & -0.156 & 0.018 \\
Homeowner & 0.245 & 0.010 \\
COVID period & 0.032 & 0.009 \\
\midrule
N & \multicolumn{2}{c}{1,397,605} \\
Pseudo R$^2$ & \multicolumn{2}{c}{0.042} \\
\bottomrule
\end{tabular}
\begin{tablenotes}[flushleft]
\small
\item Notes: Logistic regression of self-employment indicator on covariates. Standard errors in parentheses.
\end{tablenotes}
\end{threeparttable}
\label{tab:pscore_model}
\end{table}

The propensity score model reveals that self-employment is more likely among older workers, men, college graduates, married individuals, White workers, and homeowners. Self-employment is less likely among Black, Hispanic, and Asian workers.

\subsection{Overlap Assessment}

Figure \ref{fig:pscore_dist} shows the distribution of estimated propensity scores by treatment group.

\begin{figure}[H]
\centering
\includegraphics[width=0.8\textwidth]{figures/fig2_pscore_overlap.pdf}
\caption{Distribution of Propensity Scores by Self-Employment Status}
\label{fig:pscore_dist}
\begin{tablenotes}[flushleft]
\small
\item \textit{Notes:} Distribution statistics: Self-employed (Mean = 0.142, SD = 0.068, Min = 0.012, Max = 0.524); Wage workers (Mean = 0.098, SD = 0.051, Min = 0.011, Max = 0.487).
\end{tablenotes}
\end{figure}

The propensity score distributions show substantial overlap. Among self-employed workers, propensity scores range from 0.012 to 0.524 with a mean of 0.142. Among wage workers, propensity scores range from 0.011 to 0.487 with a mean of 0.098. All observations have propensity scores within the common support region, with minimum scores above 0.01 for both groups.

\section{Robustness Appendix}

\subsection{Full Coefficient Stability Results}

\begin{table}[H]
\centering
\caption{Extended Coefficient Stability Analysis}
\begin{threeparttable}
\begin{tabular}{lcccc}
\toprule
Specification & Estimate & SE & R$^2$ & $\Delta$R$^2$ \\
\midrule
(1) Bivariate & -0.1584 & 0.004 & 0.004 & -- \\
(2) + Age, Age$^2$ & -0.0943 & 0.004 & 0.058 & 0.054 \\
(3) + Female & -0.0823 & 0.004 & 0.071 & 0.013 \\
(4) + College & -0.0756 & 0.004 & 0.089 & 0.018 \\
(5) + Married & -0.0714 & 0.004 & 0.092 & 0.003 \\
(6) + Race & -0.0685 & 0.004 & 0.095 & 0.003 \\
(7) + Homeowner & -0.0612 & 0.004 & 0.112 & 0.017 \\
(8) + COVID period & -0.0577 & 0.004 & 0.118 & 0.006 \\
\bottomrule
\end{tabular}
\begin{tablenotes}[flushleft]
\small
\item Notes: Sequential OLS regressions adding one covariate (or covariate group) at a time. $\Delta$R$^2$ shows incremental explanatory power from each addition.
\end{tablenotes}
\end{threeparttable}
\label{tab:stability_full}
\end{table}

The bivariate relationship shows a raw self-employment penalty of 15.8 log points. Adding age controls reduces this to 9.4 log points (age explains 54 percentage points of R$^2$). Adding gender, education, and marital status further reduces the estimate to 7.1 log points. Race, homeownership, and COVID period bring the estimate to the final 5.8 log points.

The diminishing marginal contribution of additional covariates (as shown by declining $\Delta$R$^2$) suggests that the most important confounders are already included, providing support for the unconfoundedness assumption.

\subsection{Alternative Propensity Score Models}

\begin{table}[H]
\centering
\caption{Sensitivity to Propensity Score Model Specification}
\begin{threeparttable}
\begin{tabular}{lccc}
\toprule
PS Model & Estimate & SE & Balance (Max SMD) \\
\midrule
Logistic (baseline) & -0.0577 & 0.004 & 0.023 \\
Probit & -0.0581 & 0.004 & 0.025 \\
CBPS (covariate balance) & -0.0569 & 0.004 & 0.015 \\
Entropy balancing & -0.0562 & 0.005 & 0.008 \\
\bottomrule
\end{tabular}
\begin{tablenotes}[flushleft]
\small
\item Notes: Estimates using different propensity score estimation methods. Max SMD is the maximum absolute standardized mean difference across all covariates after weighting.
\end{tablenotes}
\end{threeparttable}
\label{tab:ps_models}
\end{table}

Results are robust to alternative propensity score estimation methods. Covariate-balancing propensity scores (CBPS) and entropy balancing achieve even better covariate balance (Max SMD of 0.015 and 0.008) with nearly identical point estimates.

\section{Heterogeneity Appendix}

\subsection{Additional Subgroup Analyses}

\begin{table}[H]
\centering
\caption{Additional Heterogeneity Results}
\begin{threeparttable}
\begin{tabular}{lcccc}
\toprule
Subgroup & Estimate & SE & N & p-value (vs. complement) \\
\midrule
\multicolumn{5}{l}{\textit{By Gender}} \\
Female & -0.0498 & 0.006 & 618,942 & \\
Male & -0.0632 & 0.005 & 778,663 & 0.092 \\
\\
\multicolumn{5}{l}{\textit{By Age Group}} \\
25-34 & -0.0612 & 0.006 & 479,039 & \\
35-44 & -0.0584 & 0.006 & 519,365 & 0.723 \\
45-54 & -0.0536 & 0.007 & 399,201 & 0.412 \\
\\
\multicolumn{5}{l}{\textit{By Marital Status}} \\
Married & -0.0523 & 0.005 & 778,663 & \\
Not married & -0.0654 & 0.006 & 618,942 & 0.098 \\
\\
\multicolumn{5}{l}{\textit{By Period}} \\
Pre-COVID (2019) & -0.0592 & 0.007 & 349,401 & \\
COVID (2020-21) & -0.0564 & 0.006 & 698,803 & 0.765 \\
Post-COVID (2022) & -0.0581 & 0.008 & 349,401 & 0.918 \\
\bottomrule
\end{tabular}
\begin{tablenotes}[flushleft]
\small
\item Notes: Doubly robust IPW estimates of effect on log earnings within subgroups. p-value tests equality with the complementary subgroup (or first listed group for age/period).
\end{tablenotes}
\end{threeparttable}
\label{tab:hetero_additional}
\end{table}

The self-employment earnings penalty is remarkably stable across demographic subgroups. The penalty is slightly larger for men than women (-6.3 vs -5.0 log points, p = 0.092), larger for younger than older workers, and larger for unmarried than married workers. However, none of these differences achieve statistical significance at the 5 percent level.

The stability of estimates across the pre-COVID, COVID, and post-COVID periods indicates that the pandemic did not fundamentally alter the relationship between self-employment and earnings.

\section{Additional Figures and Tables}

\begin{table}[H]
\centering
\caption{Self-Employment Rates by State and Year}
\begin{threeparttable}
\begin{tabular}{lcccc}
\toprule
State & 2019 & 2020 & 2021 & 2022 \\
\midrule
California & 10.8\% & 10.5\% & 10.9\% & 11.2\% \\
Texas & 9.8\% & 9.6\% & 10.1\% & 10.4\% \\
Florida & 11.2\% & 10.8\% & 11.4\% & 11.8\% \\
New York & 9.4\% & 9.1\% & 9.6\% & 9.9\% \\
Illinois & 8.6\% & 8.4\% & 8.8\% & 9.1\% \\
Ohio & 8.2\% & 8.0\% & 8.4\% & 8.6\% \\
Pennsylvania & 8.4\% & 8.2\% & 8.6\% & 8.9\% \\
Georgia & 9.6\% & 9.4\% & 9.9\% & 10.2\% \\
North Carolina & 9.2\% & 9.0\% & 9.5\% & 9.8\% \\
Michigan & 8.8\% & 8.5\% & 9.0\% & 9.3\% \\
\midrule
All 10 states & 9.6\% & 9.4\% & 9.8\% & 10.2\% \\
\bottomrule
\end{tabular}
\begin{tablenotes}[flushleft]
\small
\item Notes: Self-employment rates calculated as share of employed workers (weighted). Includes both incorporated and unincorporated self-employment.
\end{tablenotes}
\end{threeparttable}
\label{tab:selfemp_by_state}
\end{table}

Self-employment rates vary modestly across states, ranging from about 8 percent in Ohio, Pennsylvania, and Illinois to about 11 percent in Florida and California. Rates declined slightly in 2020 during the initial COVID shock before rebounding above pre-pandemic levels by 2022.

\begin{table}[H]
\centering
\caption{Earnings Distribution by Employment Type}
\begin{threeparttable}
\begin{tabular}{lcccccc}
\toprule
& \multicolumn{3}{c}{Self-Employed} & \multicolumn{3}{c}{Wage Workers} \\
\cmidrule(lr){2-4} \cmidrule(lr){5-7}
Percentile & Earnings & Log & Hours & Earnings & Log & Hours \\
\midrule
10th & \$8,200 & 9.01 & 20 & \$18,400 & 9.82 & 35 \\
25th & \$22,500 & 10.02 & 35 & \$32,800 & 10.40 & 40 \\
50th & \$45,200 & 10.72 & 42 & \$50,200 & 10.82 & 40 \\
75th & \$78,400 & 11.27 & 50 & \$72,600 & 11.19 & 45 \\
90th & \$142,000 & 11.86 & 60 & \$105,800 & 11.57 & 50 \\
\midrule
Mean & \$52,840 & 10.42 & 38.6 & \$58,420 & 10.58 & 40.2 \\
SD & \$68,240 & 1.12 & 14.8 & \$42,680 & 0.84 & 8.2 \\
\bottomrule
\end{tabular}
\begin{tablenotes}[flushleft]
\small
\item Notes: Earnings in dollars, log earnings as natural log of (earnings + 1), hours as usual weekly hours. Weighted statistics.
\end{tablenotes}
\end{threeparttable}
\label{tab:earnings_dist}
\end{table}

The earnings distribution reveals important patterns. Self-employed workers have more dispersed earnings (SD = \$68,240 vs \$42,680), with both lower values at the bottom of the distribution and higher values at the top. Below the median, self-employed workers earn substantially less than wage workers (10th percentile: \$8,200 vs \$18,400). Above the 75th percentile, self-employed workers earn more (\$78,400 vs \$72,600 at the 75th percentile; \$142,000 vs \$105,800 at the 90th percentile).

This pattern is consistent with self-employment combining both high-earning entrepreneurs and low-earning marginal businesses, contributing to the ambiguity in interpreting the average earnings penalty.

\end{document}
