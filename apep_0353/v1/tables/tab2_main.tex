\begin{table}[htbp]
\centering
\caption{Effect of Labor Market Tightness on HCBS Provider Supply}
\label{tab:main}
\begin{threeparttable}
\footnotesize
\begin{tabular}{lcccccccc}
\toprule
 & \multicolumn{4}{c}{OLS} & \multicolumn{4}{c}{IV (Bartik)} \\
\cmidrule(lr){2-5} \cmidrule(lr){6-9}
 & ln(Prov) & ln(Claims) & ln(Paid) & ln(Benes) & ln(Prov) & ln(Claims) & ln(Paid) & ln(Benes) \\
 & (1) & (2) & (3) & (4) & (5) & (6) & (7) & (8) \\
\midrule
Employment/Pop & -0.012 & 0.150 & -0.106 & -0.128 & 0.045 & -1.971 & 0.925 & -4.760*** \\
 & (0.242) & (0.347) & (0.643) & (0.282) & (0.522) & (2.277) & (1.911) & (1.767) \\
[0.5em]
\midrule
First-stage F & & & & & 1301.1 & 1301.1 & 1301.1 & 1301.1 \\
Observations & 81,293 & 68,646 & 68,646 & 68,646 & 81,293 & 68,646 & 68,646 & 68,646 \\
County FE & Yes & Yes & Yes & Yes & Yes & Yes & Yes & Yes \\
State $\times$ Quarter FE & Yes & Yes & Yes & Yes & Yes & Yes & Yes & Yes \\
Clustering & State & State & State & State & State & State & State & State \\
\bottomrule
\end{tabular}
\begin{tablenotes}[flushleft]
\small
\item \textit{Notes:} Each column reports a separate regression. The dependent variable is indicated in the column header. Employment/Population is the county-quarter ratio of total private employment (BLS QCEW) to population (ACS). Columns 5--8 instrument employment/population with a Bartik shift-share instrument using 2018 county industry shares and national industry employment growth, excluding healthcare (NAICS 62). Standard errors clustered at the state level in parentheses. *** p$<$0.01, ** p$<$0.05, * p$<$0.1.
\end{tablenotes}
\end{threeparttable}
\end{table}

