\begin{table}[htbp]
\centering
\caption{Summary Statistics}
\label{tab:summary}
\begin{threeparttable}
\begin{tabular}{lrrrrrr}
\toprule
Variable & Mean & SD & P25 & Median & P75 & N \\
\midrule
HCBS Providers & 16.83 & 51.55 & 1.00 & 5.00 & 13.00 & 81,300 \\
HCBS Claims & 66,412.20 & 337,002.11 & 1,695.00 & 7,111.00 & 29,081.00 & 68,665 \\
HCBS Spending ($) & 7,725,982.53 & 42,288,596.50 & 155,540.89 & 820,933.45 & 3,446,323.46 & 68,665 \\
HCBS Beneficiaries & 13,830.62 & 61,986.65 & 510.00 & 2,124.00 & 7,325.00 & 68,665 \\
Non-HCBS Providers & 80.93 & 319.10 & 8.00 & 20.00 & 55.00 & 81,300 \\
Employment/Pop Ratio & 0.28 & 0.14 & 0.19 & 0.26 & 0.34 & 81,300 \\
Bartik IV & 0.00 & 0.03 & -0.01 & 0.01 & 0.02 & 81,300 \\
Population & 110,208.37 & 344,087.97 & 12,281.00 & 28,057.00 & 73,604.00 & 81,300 \\
Poverty Rate & 0.14 & 0.06 & 0.10 & 0.13 & 0.17 & 81,300 \\
Elderly Share & 0.20 & 0.05 & 0.17 & 0.19 & 0.22 & 81,300 \\
Uninsured Rate & 0.09 & 0.05 & 0.06 & 0.08 & 0.11 & 81,300 \\
\bottomrule
\end{tabular}
\begin{tablenotes}[flushleft]
\small
\item \textit{Notes:} Unit of observation is county $\times$ quarter. Sample: 3,011 counties, 2018Q1--2024Q3 (27 quarters). Summary statistics N may differ slightly from regression N due to singleton fixed-effect observations. HCBS includes T-codes (personal care, habilitation), H-codes (behavioral health), and S-codes (temporary/state services) from T-MSIS. Employment data from Census QWI. Demographics from ACS 5-year estimates.
\end{tablenotes}
\end{threeparttable}
\end{table}

