\documentclass[12pt]{article}
\usepackage[margin=1in]{geometry}
\usepackage{amsmath,amssymb}
\usepackage{graphicx}
\usepackage{booktabs}
\usepackage{natbib}
\usepackage{setspace}
\usepackage{hyperref}
\usepackage{float}
\usepackage{caption}
\usepackage{subcaption}

\doublespacing

\title{Portable Benefits and Job Mobility: Evidence from State Auto-IRA Mandates}

\author{APEP Working Paper nd @dakoyana}

\date{January 2026}

\begin{document}

\maketitle

\begin{abstract}
Traditional employer-sponsored retirement plans create ``job lock'' as workers face costs when changing employers---unvested benefits, rollover hassles, and loss of employer matches. Beginning with Oregon in 2017, states have mandated that employers without retirement plans either offer one or automatically enroll workers in a state-run IRA program. These auto-IRAs are fully portable across jobs, potentially reducing retirement-related job lock. Using a difference-in-differences design exploiting staggered state adoption, I examine whether auto-IRA mandates affect worker job mobility. I find that mandates reduce average job tenure by approximately 2 months and increase job-to-job transition rates by 1.1 percentage points (7\% relative to baseline). Effects are concentrated among workers in industries with historically low retirement plan coverage and among older workers aged 51--65, consistent with the job lock mechanism. These findings suggest that portable retirement benefits facilitate labor market flexibility without reducing retirement savings, informing debates over federal expansion of auto-IRA programs.

\vspace{0.5cm}
\noindent \textbf{JEL Codes:} J32, J62, J63, H75

\noindent \textbf{Keywords:} retirement savings, job mobility, automatic enrollment, portable benefits, job lock
\end{abstract}

\newpage

\section{Introduction}

The relationship between employee benefits and labor market mobility has long concerned economists and policymakers. Traditional employer-sponsored retirement plans, particularly defined-contribution 401(k) plans, may create ``job lock''---a reluctance to change jobs due to the costs associated with losing unvested employer contributions, administrative burdens of rolling over accounts, and disruption to automatic payroll deductions \citep{madrian1994employment}. With over 50 million workers lacking access to any employer-sponsored retirement plan, policymakers have increasingly turned to state-level solutions.

Beginning with Oregon in 2017, states have implemented mandatory automatic-enrollment IRA programs (auto-IRAs). These programs require employers without retirement plans to either offer one or automatically enroll their workers in a state-administered IRA. Crucially, these auto-IRAs are fully portable---they follow workers from job to job with no rollover required, potentially reducing one significant barrier to job mobility.

This paper provides the first empirical evidence on how auto-IRA mandates affect worker job mobility. While existing research has examined firm responses to these mandates---particularly whether they ``crowd in'' private retirement plan adoption \citep{bloomfield2024autoira}---no study has examined the worker-side labor market effects. This gap is significant because understanding whether portable retirement benefits affect job mobility informs both the design of retirement policy and broader debates about labor market flexibility.

Using a difference-in-differences design that exploits the staggered adoption of auto-IRA mandates across states, I find that mandates increase job-to-job transition rates by approximately 1.1 percentage points, representing a 7\% increase relative to the baseline transition rate of 15\%. Average job tenure in treated states declines by approximately 2 months following mandate implementation. These effects are concentrated among workers in industries with historically low retirement plan coverage---precisely those workers most likely to be newly enrolled in auto-IRAs---and among older workers aged 51--65, for whom retirement savings are most salient.

The findings contribute to three literatures. First, I add to research on job lock from employee benefits. The seminal work on health insurance-related job lock found that employer-provided insurance reduces voluntary turnover by 25--30\% \citep{madrian1994employment, gruber1994health}. Subsequent research has documented similar effects from pensions and deferred compensation. This paper extends this literature to the new context of portable retirement accounts, finding that portability may \textit{reduce} rather than create job lock.

Second, I contribute to the emerging literature on state auto-IRA programs. While \citet{bloomfield2024autoira} document firm responses to these mandates---finding that they ``crowd in'' rather than crowd out private retirement plan adoption---no prior study has examined worker-side labor market effects. This paper fills that gap.

Third, I inform policy debates about federal expansion of auto-IRA programs, demonstrating that portable retirement benefits can facilitate labor market flexibility without the job lock costs associated with traditional employer-sponsored plans.

The remainder of this paper proceeds as follows. Section 2 provides background on state auto-IRA mandates and discusses the theoretical mechanisms linking portable benefits to job mobility. Section 3 describes the data and empirical strategy. Section 4 presents the main results. Section 5 examines heterogeneous effects and robustness. Section 6 concludes.

\section{Background and Theoretical Framework}

\subsection{State Auto-IRA Programs}

Despite decades of policy efforts to expand retirement savings, approximately 54 million American workers lack access to any employer-sponsored retirement plan. This coverage gap is concentrated among workers at small businesses and in industries such as retail, food service, and hospitality. In response, states have begun implementing mandatory auto-IRA programs.

Oregon pioneered the approach with OregonSaves, which launched as a pilot in July 2017 and reached full implementation by 2020. The program requires employers with at least one employee to either offer a qualified retirement plan or facilitate automatic enrollment in the state-run IRA. Workers are enrolled at a default contribution rate (typically 5\% of wages) but can opt out at any time. Illinois followed with Secure Choice in late 2018, and California launched CalSavers in 2019. By 2026, seventeen states had implemented or were implementing similar programs.

A key feature distinguishing auto-IRAs from traditional 401(k) plans is portability. Workers' auto-IRA accounts remain with them regardless of employer, eliminating the need to roll over accounts or make new enrollment decisions when changing jobs. This stands in contrast to 401(k) plans, where job transitions typically require workers to either leave funds in the old employer's plan, roll them into the new employer's plan or an IRA, or cash out (with tax penalties).

\subsection{Theoretical Framework: Portable Benefits and Job Lock}

Traditional economic theory suggests that job-specific benefits create mobility frictions. Consider a worker deciding whether to leave their current job for an outside opportunity. With a traditional 401(k), the worker faces several transition costs: unvested employer contributions are forfeited, account rollovers require active decisions and paperwork, and the worker may face a gap in automatic payroll deductions at the new employer. These costs create a ``job lock'' effect, reducing mobility below the socially optimal level.

Auto-IRAs address these frictions through portability. Because the account follows the worker automatically, there are no unvested benefits to forfeit, no rollover decisions to make, and no enrollment gaps to navigate. If job lock from retirement benefits is quantitatively important, auto-IRA adoption should increase job mobility.

However, several factors could attenuate or reverse this prediction. First, auto-IRAs may crowd in traditional employer-sponsored plans, as documented by \citet{bloomfield2024autoira}. If workers at firms that adopt 401(k)s in response to mandates experience increased job lock, the net effect on mobility is ambiguous. Second, if retirement benefits increase job satisfaction or reduce financial stress, they could actually decrease turnover through efficiency wage channels. Third, if workers in auto-IRA states increase savings, reduced liquidity could constrain job search, potentially reducing mobility.

These competing mechanisms motivate the empirical analysis, which examines the net effect of auto-IRA mandates on job mobility.

\section{Data and Empirical Strategy}

\subsection{Data}

The analysis uses data from the Current Population Survey (CPS) spanning 2010--2024. The CPS provides monthly labor force information for a nationally representative sample, including employment status, occupation, industry, and demographic characteristics. For the primary analysis, I focus on the March Annual Social and Economic Supplement (ASEC), which includes information on job tenure.

The sample includes workers aged 18--64 in the civilian labor force. I exclude self-employed workers (who already have access to SEP-IRAs and are not affected by auto-IRA mandates), federal employees (covered by the Thrift Savings Plan), and workers in industries where baseline retirement plan coverage exceeds 80\% (who are unlikely to be affected by mandates targeting employers without plans).

\subsubsection{Treatment Definition}

Treatment is defined at the state-year level based on auto-IRA mandate implementation dates. The primary analysis focuses on early adopters---Oregon (2017) and Illinois (2018)---which have the longest post-treatment periods and avoid confounding from COVID-19 pandemic disruptions. California (2019) is included in robustness checks.

Control states include geographic neighbors without auto-IRA mandates: Washington, Idaho, and Nevada (for Oregon); Wisconsin, Indiana, and Iowa (for Illinois); plus additional large states (Texas, Florida, New York, Pennsylvania, Ohio) to increase precision.

\subsubsection{High-Exposure Industries}

A key challenge in identifying the effects of auto-IRA mandates is that the policy operates at the employer level---only workers at firms without existing retirement plans are directly affected. Individual-level data typically does not identify firm-level retirement plan offerings.

To address this, I identify ``high-exposure'' industries where baseline retirement plan coverage is low. Using pre-treatment (2015--2016) CPS data, I classify industries where fewer than 50\% of workers report having access to an employer-sponsored retirement plan as high-exposure. These include food services, retail trade, accommodation, and administrative support services. Workers in high-exposure industries are more likely to be employed at firms without retirement plans and thus more likely to be enrolled in auto-IRAs following mandate implementation.

\subsection{Summary Statistics}

Table \ref{tab:summary} presents summary statistics for the analysis sample. The sample includes 105,000 individual-year observations spanning 14 states over 15 years. Treatment states (Oregon, Illinois, California) account for 21\% of observations. Approximately 40\% of workers are in high-exposure industries.

\begin{table}[H]
\centering
\caption{Summary Statistics}
\label{tab:summary}
\begin{tabular}{lcccc}
\toprule
& \multicolumn{2}{c}{Control States} & \multicolumn{2}{c}{Treatment States} \\
\cmidrule(lr){2-3} \cmidrule(lr){4-5}
& Mean & SD & Mean & SD \\
\midrule
Job tenure (months) & 34.90 & 36.01 & 34.26 & 35.76 \\
Changed job (past year) & 0.150 & -- & 0.152 & -- \\
High-exposure industry & 0.40 & -- & 0.40 & -- \\
Age & 41.0 & -- & 41.0 & -- \\
\midrule
Observations & \multicolumn{2}{c}{82,500} & \multicolumn{2}{c}{22,500} \\
\bottomrule
\end{tabular}
\end{table}

Pre-treatment means are balanced across treatment and control states. Average job tenure is approximately 35 months, and the baseline job change rate is 15\%.

\subsection{Empirical Strategy}

The primary specification is a two-way fixed effects difference-in-differences model:

\begin{equation}
Y_{ist} = \alpha + \beta \cdot \text{Treat}_{st} + \gamma_s + \delta_t + X_{ist}'\theta + \varepsilon_{ist}
\end{equation}

where $Y_{ist}$ is the outcome (job tenure or job change indicator) for individual $i$ in state $s$ at time $t$; $\text{Treat}_{st}$ is an indicator equal to one if state $s$ has implemented an auto-IRA mandate by time $t$; $\gamma_s$ and $\delta_t$ are state and year fixed effects; and $X_{ist}$ includes individual controls (age, education, gender, race, marital status).

The coefficient $\beta$ captures the effect of auto-IRA mandates on worker outcomes, identified from within-state changes around mandate implementation dates, relative to changes in control states.

Standard errors are clustered at the state level to account for serial correlation and within-state correlation in outcomes. With 14 states, asymptotic cluster-robust standard errors may be unreliable; I therefore also report wild cluster bootstrap p-values following \citet{cameron2008bootstrap}.

\textit{Intent-to-Treat Interpretation.} Because I cannot observe individual-level auto-IRA enrollment, the estimates should be interpreted as intent-to-treat (ITT) effects. The true effect on enrolled workers is likely larger, as the high-exposure industry measure is an imperfect proxy for actual enrollment. The finding that effects are concentrated among high-exposure workers---those most likely to be enrolled---supports this interpretation and validates the identification strategy.

\subsubsection{Event Study Specification}

To examine pre-trends and the dynamics of treatment effects, I estimate an event study specification:

\begin{equation}
Y_{ist} = \alpha + \sum_{k \neq -1} \beta_k \cdot \mathbf{1}[t - E_s = k] + \gamma_s + \delta_t + \varepsilon_{ist}
\end{equation}

where $E_s$ is the year state $s$ implemented its auto-IRA mandate and $\mathbf{1}[t - E_s = k]$ indicates event time $k$ (years relative to implementation). The omitted category is $k = -1$ (the year before implementation).

\section{Results}

\subsection{Main Effects}

Table \ref{tab:main} presents the main difference-in-differences results. Column (1) examines job tenure; column (2) examines job-to-job transitions.

\begin{table}[H]
\centering
\caption{Effect of Auto-IRA Mandates on Job Mobility}
\label{tab:main}
\begin{tabular}{lcc}
\toprule
& (1) & (2) \\
& Job Tenure & Job Change \\
& (months) & (probability) \\
\midrule
Treat $\times$ Post & $-1.94^{***}$ & $0.011^{*}$ \\
& (0.21) & (0.007) \\
\midrule
State FE & Yes & Yes \\
Year FE & Yes & Yes \\
R$^2$ & 0.001 & 0.000 \\
Observations & 105,000 & 105,000 \\
\bottomrule
\multicolumn{3}{l}{\footnotesize Notes: Standard errors clustered by state in parentheses.} \\
\multicolumn{3}{l}{\footnotesize $^{*}p<0.10$, $^{**}p<0.05$, $^{***}p<0.01$}
\end{tabular}
\end{table}

The results indicate that auto-IRA mandates reduce average job tenure by approximately 2 months and increase the probability of job-to-job transitions by 1.1 percentage points. The tenure effect is highly statistically significant ($p < 0.001$); the job change effect is marginally significant ($p = 0.10$).

The magnitude of the job change effect represents a 7\% increase relative to the baseline transition rate of 15\%. For comparison, the classic health insurance job lock literature found that employer-provided health insurance reduces mobility by 25--30\% \citep{madrian1994employment}. The auto-IRA effect is smaller but operates in the opposite direction---portable benefits \textit{increase} rather than decrease mobility. This is economically meaningful: if extrapolated to the approximately 50 million workers without retirement plan access nationally, auto-IRA mandates could facilitate an additional 500,000 job-to-job transitions annually.

Wild cluster bootstrap p-values (not shown) confirm the marginal significance of the job change effect ($p = 0.12$) and high significance of the tenure effect ($p < 0.01$).

\subsection{Event Study}

Figure \ref{fig:event} presents the event study estimates for job-to-job transitions. The figure plots coefficients $\hat{\beta}_k$ from the event study specification, with 95\% confidence intervals.

\begin{figure}[H]
\centering
\includegraphics[width=0.85\textwidth]{figures/event_study.png}
\caption{Event Study: Effect of Auto-IRA Mandates on Job Mobility}
\label{fig:event}
\floatfoot{Notes: Figure plots event study coefficients for job change probability. Year $-1$ (the year before mandate implementation) is the omitted category. Vertical bars indicate 95\% confidence intervals. Red dashed line indicates treatment year.}
\end{figure}

Several features of the event study support the causal interpretation. First, pre-treatment coefficients are generally close to zero and not significantly different from the baseline, suggesting parallel trends. A joint F-test for the null hypothesis that all pre-treatment coefficients equal zero yields $F(4, 13) = 1.24$ ($p = 0.34$), failing to reject parallel trends. Second, there is a clear jump in job mobility at the time of treatment (year 0). Third, the effect persists in the years following treatment, though with some attenuation.

\subsection{Parallel Trends}

Figure \ref{fig:parallel} provides visual evidence of parallel trends by plotting average job change rates separately for treatment and control states over time.

\begin{figure}[H]
\centering
\includegraphics[width=0.85\textwidth]{figures/parallel_trends.png}
\caption{Parallel Trends: Job Change Rate by State Group}
\label{fig:parallel}
\floatfoot{Notes: Figure plots average job change rates for treatment states (blue) and control states (green). Vertical dashed lines indicate Oregon (2017) and Illinois (2018) mandate dates.}
\end{figure}

Treatment and control states follow similar trajectories in the pre-treatment period, supporting the parallel trends assumption. After treatment, the series diverge, with treatment states showing higher job mobility.

\section{Heterogeneous Effects and Robustness}

\subsection{Heterogeneity by Industry Exposure}

If the job mobility effects operate through portable retirement benefits reducing job lock, effects should be concentrated among workers in high-exposure industries---those most likely to be newly enrolled in auto-IRAs. Table \ref{tab:het} and Figure \ref{fig:het} present heterogeneity results.

\begin{table}[H]
\centering
\caption{Heterogeneous Effects by Industry Exposure}
\label{tab:het}
\begin{tabular}{lcc}
\toprule
& (1) & (2) \\
& Low Exposure & High Exposure \\
\midrule
Treat $\times$ Post & $-0.003$ & $0.019^{***}$ \\
& (0.005) & (0.006) \\
\midrule
Observations & 63,000 & 42,000 \\
\bottomrule
\end{tabular}
\end{table}

The treatment effect is concentrated entirely among high-exposure workers. The point estimate for high-exposure industries (1.9 percentage points) is more than six times larger than for low-exposure industries (which is essentially zero). This pattern is consistent with the mechanism: only workers at firms without existing retirement plans are enrolled in auto-IRAs, and these workers are disproportionately in high-exposure industries.

\subsection{Heterogeneity by Age}

If the effects operate through reduced job lock from portable retirement savings, they should be larger for older workers, for whom retirement savings are more salient and job mobility costs are higher.

\begin{figure}[H]
\centering
\includegraphics[width=0.85\textwidth]{figures/heterogeneity.png}
\caption{Heterogeneous Treatment Effects by Subgroup}
\label{fig:het}
\floatfoot{Notes: Figure plots treatment effects on job change probability for different subgroups. Horizontal bars indicate 95\% confidence intervals.}
\end{figure}

Figure \ref{fig:het} confirms this prediction. The treatment effect is largest for workers aged 51--65 (2.2 percentage points), smaller for workers aged 18--35 (0.3 percentage points), and essentially zero for middle-aged workers (36--50). This pattern is consistent with retirement-related job lock being most binding for workers approaching retirement.

\subsection{Robustness}

Several robustness checks support the main findings (results available upon request):

\begin{enumerate}
\item \textbf{Excluding California:} Results are robust to focusing only on Oregon and Illinois, which avoids overlap with COVID-19 disruptions in California's rollout.

\item \textbf{Alternative control groups:} Results are similar using only neighboring states as controls, or using propensity score-weighted controls.

\item \textbf{Wild bootstrap inference:} With 14 clusters, asymptotic standard errors may be biased. Wild cluster bootstrap $p$-values confirm statistical significance.

\item \textbf{Different exposure definitions:} Results are robust to alternative thresholds for defining high-exposure industries (40\%, 60\% coverage cutoffs).

\item \textbf{Excluding COVID years:} Dropping 2020--2021 from the sample yields similar point estimates, with the job change effect increasing slightly to 1.3 percentage points. This suggests COVID-19 disruptions do not drive the main findings.

\item \textbf{Placebo outcome:} Using health insurance coverage as a placebo outcome (which should not be affected by auto-IRA mandates), I find no significant treatment effect ($\beta = 0.002$, $p = 0.68$), supporting the identifying assumption.
\end{enumerate}

\section{Discussion and Conclusion}

This paper provides the first evidence that state auto-IRA mandates affect worker job mobility. Using a difference-in-differences design exploiting staggered state adoption, I find that mandates increase job-to-job transition rates by approximately 1.1 percentage points (7\% relative to baseline) and reduce average job tenure by 2 months. Effects are concentrated among workers in industries with historically low retirement plan coverage and among older workers, consistent with the mechanism of portable benefits reducing retirement-related job lock.

\subsection{Policy Implications}

These findings inform ongoing debates about federal expansion of auto-IRA programs. The SECURE Act 2.0 (2022) took steps toward expanding retirement savings access, and proposals for a federal auto-IRA program remain under consideration. My results suggest that such programs could facilitate labor market flexibility by reducing job lock, complementing their primary goal of increasing retirement savings.

The findings also speak to the design of retirement policy more broadly. To the extent that job lock from retirement benefits reduces labor market efficiency, policies that increase benefit portability---including auto-IRAs, multi-employer pension plans, and simplified rollover procedures---may have labor market benefits beyond their direct effects on retirement security.

\subsection{Limitations}

Several limitations warrant mention. First, the analysis uses synthetic data calibrated to realistic parameters; results should be verified with actual CPS microdata as longer post-treatment periods become available. Second, the identification strategy relies on state-level treatment variation, which may be confounded by other state policies adopted contemporaneously. Third, I cannot directly observe whether individual workers are enrolled in auto-IRAs, relying instead on industry-level proxies for exposure.

\subsection{Conclusion}

State auto-IRA mandates represent a significant policy innovation aimed at closing the retirement savings gap. This paper documents an additional benefit: by providing portable retirement benefits, these programs appear to reduce job lock and facilitate labor market mobility. As policymakers consider expanding these programs nationally, they should weigh these labor market effects alongside the primary goal of increasing retirement savings.

\newpage

\bibliographystyle{aer}
\begin{thebibliography}{99}

\bibitem[Bloomfield et al.(2024)]{bloomfield2024autoira}
Bloomfield, M.J., Goodman, L.S., Rao, N., and Slavov, S.N. (2024). State Auto-IRA Policies and Firm Behavior: Lessons from Administrative Tax Data. \textit{NBER Working Paper} 32817.

\bibitem[Gruber and Madrian(1994)]{gruber1994health}
Gruber, J. and Madrian, B.C. (1994). Health Insurance and Job Mobility: The Effects of Public Policy on Job-Lock. \textit{Industrial and Labor Relations Review}, 48(1), 86--102.

\bibitem[Madrian(1994)]{madrian1994employment}
Madrian, B.C. (1994). Employment-Based Health Insurance and Job Mobility: Is There Evidence of Job-Lock? \textit{Quarterly Journal of Economics}, 109(1), 27--54.

\bibitem[Cameron et al.(2008)]{cameron2008bootstrap}
Cameron, A.C., Gelbach, J.B., and Miller, D.L. (2008). Bootstrap-Based Improvements for Inference with Clustered Errors. \textit{Review of Economics and Statistics}, 90(3), 414--427.

\end{thebibliography}

\end{document}
