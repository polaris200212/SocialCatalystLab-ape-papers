\begin{table}[htbp]
\centering
\caption{Summary Statistics by MGNREGA Phase}
\label{tab:summary}
\small
\begin{tabular}{lccc}
\toprule
 & Phase I & Phase II & Phase III \\
\midrule
\multicolumn{4}{l}{\textit{Panel A: Baseline Characteristics (Census 2001)}} \\
Districts & 200 & 130 & 310 \\
Population (thousands) & 1439 & 1822 & 1990 \\
 & (1017) & (1329) & (2576) \\
SC/ST share & 0.434 & 0.317 & 0.240 \\
 & (0.231) & (0.188) & (0.204) \\
Agricultural labor share & 0.191 & 0.152 & 0.083 \\
 & (0.106) & (0.100) & (0.074) \\
Illiteracy rate & 0.563 & 0.497 & 0.390 \\
 & (0.093) & (0.088) & (0.103) \\
Backwardness index & 0.696 & 0.565 & 0.348 \\
\midrule
\multicolumn{4}{l}{\textit{Panel B: Nighttime Luminosity (log(light + 1))}} \\
Pre-treatment (1994--2006) & 8.817 & 9.494 & 9.476 \\
 & (2.021) & (1.980) & (2.092) \\
Post-treatment (2007--2023) & 9.502 & 9.847 & 9.773 \\
 & (1.336) & (1.352) & (1.555) \\
\bottomrule
\end{tabular}
\begin{minipage}{0.95\textwidth}
\vspace{0.3em}
\footnotesize \textit{Notes:} Standard deviations in parentheses. Phase I districts (200) were assigned in February 2006, Phase II (+130) in April 2007, and Phase III (remaining) in April 2008. Assignment based on proxy backwardness index constructed from Census 2001 data (SC/ST share, agricultural labor share, illiteracy rate). Pre/post split defined by Phase I timing (2007); note that Phase II and III districts were not yet treated in 2007--2008. DMSP data calibrated to VIIRS using 2012--2013 overlap.
\end{minipage}
\end{table}
