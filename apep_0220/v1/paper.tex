\documentclass[12pt]{article}

% UTF-8 encoding and fonts
\usepackage[utf8]{inputenc}
\usepackage[T1]{fontenc}
\usepackage{lmodern}

% Page setup
\usepackage[margin=1in]{geometry}
\usepackage{setspace}
\onehalfspacing

% Typography
\usepackage{microtype}

% Math and symbols
\usepackage{amsmath,amssymb}

% Graphics
\usepackage{graphicx}
\usepackage{float}
\usepackage{subcaption}

% Tables
\usepackage{booktabs}
\usepackage{array}
\usepackage{multirow}
\usepackage{threeparttable}
\usepackage{longtable}
\usepackage{pdflscape}
\usepackage{siunitx}
\usepackage{tabularx}
\sisetup{detect-all=true, group-separator={,}, group-minimum-digits=4}

% Bibliography
\usepackage{natbib}
\bibliographystyle{aer}

% Hyperlinks
\usepackage{hyperref}
\hypersetup{
    colorlinks=true,
    linkcolor=blue,
    citecolor=blue,
    urlcolor=blue
}
\usepackage[nameinlink,noabbrev]{cleveref}

% Timing data (not used)

% Captions
\usepackage{caption}
\captionsetup{font=small,labelfont=bf}

% Section formatting
\usepackage{titlesec}
\titleformat{\section}{\large\bfseries}{\thesection.}{0.5em}{}
\titleformat{\subsection}{\normalsize\bfseries}{\thesubsection}{0.5em}{}

% Custom commands
\newcommand{\E}{\mathbb{E}}
\newcommand{\Var}{\text{Var}}
\newcommand{\Cov}{\text{Cov}}
\newcommand{\ind}{\mathbb{I}}
\newcommand{\sym}[1]{\ifmmode^{#1}\else\(^{#1}\)\fi}

\title{Who Believes God Forgives? A Comprehensive Cross-Cultural and Economic Analysis of Divine Punishment and Forgiveness Beliefs}
\author{APEP Autonomous Research\thanks{Autonomous Policy Evaluation Project. Correspondence: scl@econ.uzh.ch} \and @dyanag \\ @SocialCatalystLab}
\date{\today}

\begin{document}

\maketitle

\begin{abstract}
\noindent
Do people believe God punishes or forgives? This question sits at the intersection of economics, cultural evolution, and political economy, yet no study has systematically documented these beliefs across both individual survey data and cross-cultural ethnographic records. We compile the most comprehensive empirical portrait of divine punishment and forgiveness beliefs to date, drawing on five freely available datasets spanning 75,699 individual survey responses (General Social Survey, 1972--2024), 1,291 ethnographically documented societies (D-PLACE Ethnographic Atlas), 186 systematically coded cultures (Standard Cross-Cultural Sample), 137 Austronesian societies (Pulotu), and 348 historical polity-periods (Seshat). Three findings emerge. First, beliefs about divine forgiveness are far more prevalent than beliefs about divine punishment in contemporary America: 79\% of GSS respondents who answered the relevant module report that God forgives them, while only 17\% feel God punishes them for their sins. Second, cross-culturally, moralizing high gods---those that actively support human morality---are present in only 26\% of societies with coded data, concentrated in agrarian, stratified societies with supralocal political authority. Third, individual-level correlates reveal that education, income, and religious attendance are the strongest predictors of how people conceptualize divine temperament, with important heterogeneity across religious traditions. We document six additional restricted-access datasets that future researchers should exploit and argue that these beliefs matter for economics because they shape risk preferences, social trust, institutional compliance, and redistributive attitudes.
\end{abstract}

\vspace{1em}
\noindent\textbf{JEL Codes:} Z12, Z13, N30, D91 \\
\noindent\textbf{Keywords:} religion, divine punishment, forgiveness, cultural evolution, moralizing gods, cross-cultural analysis

\newpage

\section{Introduction}

In 1998, the General Social Survey asked Americans whether they feel God is punishing them for their sins. One in six said yes. In the same module, four out of five reported knowing that God forgives them. This asymmetry---a God experienced overwhelmingly as merciful rather than wrathful---contradicts the fire-and-brimstone stereotype of American religion, and it raises a question that economists have largely ignored: what determines how people conceptualize divine temperament?

The question matters for economics. Beliefs about whether God punishes transgressors or forgives sinners are not mere theological abstractions. They correlate with risk aversion \citep{hoffmann2013}, social trust \citep{shariff2011}, tax compliance \citep{torgler2006}, cooperative behavior in economic games \citep{norenzayan2013}, and support for redistribution \citep{scheve2006}. The ``Big Gods'' hypothesis in cultural evolution holds that belief in moralizing, punitive deities was a prerequisite for the emergence of large-scale cooperation and complex societies \citep{norenzayan2013, botero2014}. If true, understanding the distribution and correlates of these beliefs is foundational for political economy.

Yet empirical knowledge is fragmented. Survey researchers have documented afterlife beliefs (heaven, hell) in flagship cross-national studies like the World Values Survey and the International Social Survey Programme, but these items are indirect proxies for divine temperament---believing in hell is not the same as believing God is angry at you. Explicit ``God is punishing/forgiving'' items exist almost exclusively in U.S.-focused surveys. Meanwhile, anthropologists and cultural evolutionists have coded ``moralizing high gods'' across hundreds of pre-industrial societies, but their codings depend on contested definitions of what counts as a ``high god,'' and they measure societal-level doctrine rather than individual belief.

This paper bridges these literatures by compiling the most comprehensive empirical portrait of divine punishment and forgiveness beliefs to date. We make three contributions.

\textit{First}, we document the landscape of these beliefs using every freely accessible dataset that measures divine punishment or forgiveness, spanning five data sources, three levels of analysis (individual, society, polity), and temporal coverage from pre-industrial ethnographic records to the 2020s. No prior study has integrated these heterogeneous sources into a single analysis.

\textit{Second}, we estimate correlates of divine belief at both the individual and cross-cultural levels. At the individual level, we use the GSS to estimate how demographic, socioeconomic, and religious characteristics predict belief in divine punishment versus forgiveness. At the cross-cultural level, we use the Standard Cross-Cultural Sample's rich covariate structure to examine whether moralizing high gods are more common in societies with greater agricultural dependence, political complexity, and social stratification---key predictions of the cultural evolution literature.

\textit{Third}, we provide a comprehensive data appendix cataloging all known datasets that measure divine punishment or forgiveness beliefs, including six major restricted-access sources (WVS/EVS, ISSP, Pew, Baylor) that require registration. This catalog serves as a public good for future researchers.

Our analysis reveals several striking patterns. In the United States, divine forgiveness beliefs are nearly universal among the religious, while divine punishment experiences are concentrated among the less educated, lower-income, and more frequently attending. This is inconsistent with a simple ``more religious = more punitive God'' model and suggests that the God-as-judge framing is associated with religious distress rather than devotion. Cross-culturally, moralizing high gods track political complexity and subsistence mode, consistent with the hypothesis that supernatural enforcement scales with societal scale. Across all datasets and levels of analysis, the measurement of divine temperament is highly construct-dependent: the same society can appear ``punitive'' or ``forgiving'' depending on whether you code high gods, afterlife beliefs, or experienced divine relationships.

\subsection{Related Literature}

This paper contributes to several literatures.

The \textit{economics of religion} has grown rapidly since the foundational work of \citet{iannaccone1992} on rational choice models of religious participation and \citet{barro2003} on religion and economic growth. Recent contributions examine how religious beliefs affect economic preferences \citep{benjamin2016}, labor supply \citep{gruber2004}, education \citep{botticini2012}, and political attitudes \citep{scheve2006}. Quasi-experimental evidence from \citet{clingingsmith2009} demonstrates that the Hajj pilgrimage increases religious tolerance, while \citet{bentzen2019} shows that natural disasters increase religiosity globally, suggesting that exogenous shocks to existential security shape belief formation. Within this literature, the specific content of belief---not just intensity of religiosity---has received increasing attention. \citet{shariff2011} show that belief in hell (but not heaven) predicts lower national crime rates, suggesting that the \textit{punitive} dimension of religion carries distinct behavioral implications.

The \textit{cultural evolution} literature on moralizing gods, spearheaded by \citet{norenzayan2013}, argues that belief in gods who monitor and punish moral transgressions facilitated the emergence of large-scale cooperation beyond kinship networks. Empirical tests using cross-cultural databases \citep{botero2014, peoples2016, whitehouse2019} find that moralizing gods are more common in larger, more complex societies, though the direction of causality remains debated \citep{beheim2021, slingerland2020}. A landmark experimental study by \citet{purzycki2016} demonstrates across eight diverse populations that belief in moralistic, knowledgeable gods predicts greater impartiality in economic games, providing micro-level behavioral evidence for the cultural evolution hypothesis. Our analysis contributes by examining these patterns across multiple databases simultaneously and by connecting the ethnographic coding to individual-level survey measurement.

The \textit{psychology of religion} has developed rich measurement instruments for ``God concepts,'' including the pioneering work of \citet{benson1980} on God-image scales and \citet{pargament1997} on religious coping, which distinguishes positive coping (``God forgives me'') from negative coping (``God is punishing me''). We draw on these constructs but bring an economist's concern with external validity, sample representativeness, and the connection between beliefs and economic behavior.

Finally, we contribute to the growing literature on \textit{measurement in the social sciences} that emphasizes how different operationalizations of the ``same'' concept can yield different conclusions \citep{bond2019, clarke2020}. Our analysis demonstrates this vividly: a society coded as having ``absent'' high gods in the Ethnographic Atlas may nonetheless feature intense supernatural punishment beliefs mediated through ancestor spirits, nature gods, or karma-like systems that fall outside the ``high god'' definition.


\section{Conceptual Framework}

Why should economists care about whether people believe God forgives or punishes? We identify four channels through which divine temperament beliefs enter standard economic models.

\subsection{Divine Beliefs and Economic Behavior}

\textit{Risk preferences and insurance demand.} A God who forgives moral failures reduces the perceived cost of sin, which can be modeled as a reduction in the subjective probability of ``spiritual punishment.'' Under expected utility, this should increase risk-taking in morally ambiguous domains. Conversely, a punitive God raises the shadow price of moral transgression, potentially generating precautionary behavior analogous to insurance demand \citep{hoffmann2013}. Empirically, this channel predicts that individuals who perceive God as punishing should exhibit more risk-averse behavior in economic domains adjacent to moral norms (e.g., tax evasion, bribery, contract violation).

\textit{Social trust and cooperation.} The ``supernatural monitoring'' hypothesis holds that belief in an omniscient, punitive deity sustains cooperation by raising the cost of defection even when human monitors are absent \citep{norenzayan2013, johnson2005}. This is formally equivalent to adding a punishment term to the payoff matrix of a social dilemma. The prediction is that punitive God beliefs increase cooperation (especially in anonymous interactions), while forgiving God beliefs may erode the deterrent effect. \citet{shariff2011} provide cross-national evidence consistent with this: belief in hell predicts lower crime rates, while belief in heaven does not.

\textit{Institutional compliance.} Tax compliance, regulatory adherence, and legal obedience may all be affected by perceived divine enforcement. If God punishes tax evaders, the subjective audit probability effectively increases beyond the state's enforcement capacity. \citet{torgler2006} finds that religious individuals report higher tax morale, though the specific content of belief (punitive vs.\ forgiving) has not been examined in this context.

\textit{Redistributive preferences.} Beliefs about divine justice interact with preferences over redistribution. If one believes God rewards the righteous with prosperity (prosperity theology), then poverty signals moral failure and redistribution appears undeserved. If one believes God tests the faithful through suffering (theodicy), poverty is unrelated to moral worth and redistribution may be welcomed. \citet{scheve2006} and \citet{huber2005} find that religious beliefs predict opposition to redistribution, but the mechanism likely runs through these specific theological positions rather than religiosity per se.

\subsection{Operationalizing Divine Temperament}

Measuring whether a society or individual conceives of divinity as punitive or forgiving is deceptively complex. We identify four distinct measurement approaches, each capturing a different facet of the construct:

\begin{enumerate}
\item \textbf{Explicit divine attributes}: Direct questions about God's characteristics (``God is angered by sin,'' ``God forgives me''). Available primarily in U.S. surveys (GSS, Baylor).

\item \textbf{Afterlife punishment/reward beliefs}: Belief in hell (punishment proxy) versus heaven (reward/forgiveness proxy). Available cross-nationally in WVS/EVS and ISSP.

\item \textbf{Experienced divine relationship}: Whether the respondent personally feels punished or forgiven by God. Captures lived religion rather than doctrinal knowledge. Available in GSS (COPE4, FORGIVE3).

\item \textbf{Ethnographic coding}: Whether a society's supernatural agents are ``moralizing'' (supporting human morality through monitoring and punishment). Available in D-PLACE, Pulotu, Seshat.
\end{enumerate}

These four approaches are \textit{not interchangeable}. Believing in hell (approach 2) does not imply feeling personally punished by God (approach 3). A society coded as lacking ``high gods'' (approach 4) may nonetheless feature intense supernatural punishment through ancestor spirits or nature deities. Our analysis embraces this multiplicity rather than forcing a single index.


\section{Data}

We compile five freely accessible datasets spanning three levels of analysis: individual (GSS), society (D-PLACE EA, SCCS, Pulotu), and polity-period (Seshat). Table \ref{tab:datasets_overview} provides an overview.

\subsection{General Social Survey (GSS), 1972--2024}

The GSS is a nationally representative repeated cross-section of U.S.\ adults conducted by NORC since 1972, with a cumulative sample of 75,699 respondents through 2024. The cumulative file spans 1972--2024 for core demographic variables (age, sex, education, income, religious attendance), which are available across all survey waves. However, the religion-specific modules---afterlife beliefs, God-image items, and divine coping/forgiveness batteries---were administered only in selected years (primarily 1991, 1998, 2008, and 2018 as part of ISSP Religion modules), yielding substantially smaller samples for belief-related analyses. All temporal analyses of belief variables are therefore restricted to these module years, not the full 1972--2024 span. We extract two sets of religion variables.

\textit{Afterlife beliefs} (asked in ISSP Religion module years, approximately 4,800 respondents): binary indicators for belief in heaven, hell, and life after death, coded from a 4-point certainty scale (``yes, definitely'' and ``yes, probably'' = 1; ``no, probably not'' and ``no, definitely not'' = 0).

\textit{Divine experience and God-image items} (asked in 1998 and selected other years, approximately 1,400--2,000 respondents): COPE4 (``I feel that God is punishing me for my sins or lack of spirituality''), FORGIVE3 (``I know that God forgives me''), and the God-image battery (``When you think about God, how likely are each of these images to come to your mind?'') with items including Judge, Master, Lover, Friend, Creator, Healer, Father, Mother, King, Spouse, and Redeemer.

We merge these with standard GSS demographics (age, sex, education, income, religious tradition, church attendance) and with annual economic indicators from the Federal Reserve Economic Data (FRED) system, including the Gini coefficient, unemployment rate, and median household income.

\subsection{D-PLACE Ethnographic Atlas (EA)}

The Ethnographic Atlas, maintained in the D-PLACE database \citep{kirby2016}, provides coded cultural data for 1,291 societies worldwide, with focal ethnographic dates ranging primarily from 1800 to 1950. We use variable EA034 (``Religion: high gods''), which codes societies on a 4-point scale:

\begin{enumerate}
\setcounter{enumi}{-1}
\item High god absent or not reported
\item High god present but otiose (not concerned with human affairs)
\item High god active but not supporting morality
\item High god active and specifically supportive of human morality
\end{enumerate}

Of 1,291 societies, 775 have non-missing EA034 data. Among these, 36\% report absent high gods (category 0), 33\% have otiose gods (category 1), 5\% have active but non-moralizing gods (category 2), and 26\% have moralizing high gods (category 3). We supplement EA034 with covariates on agriculture (EA028), settlement patterns (EA030), class stratification (EA031), and jurisdictional hierarchy (EA033).

\subsection{Standard Cross-Cultural Sample (SCCS)}

The SCCS \citep{murdock1969} is a carefully selected subset of 186 societies designed to minimize diffusion effects (Galton's problem) by selecting at most one society per culture area. The SCCS includes variable SCCS238, coded identically to EA034, plus an extensive set of covariates covering subsistence, political organization, kinship, and social complexity. This covariate richness makes the SCCS our primary dataset for cross-cultural correlational analysis.

\subsection{Pulotu: Database of Austronesian Supernatural Beliefs}

Pulotu \citep{watts2015} codes 137 Austronesian cultures across 86 variables on supernatural beliefs and practices, distinguishing traditional, post-contact, and contemporary religious states. Our key variable is ``Supernatural punishment for impiety'' (Question 7), coded as present or absent. Pulotu also includes variables on belief in gods, nature gods, deified ancestors, ancestral spirits, and the relationship between afterlife quality and one's earthly actions.

\subsection{Seshat: Global History Databank}

The Seshat databank \citep{turchin2018} provides historically coded data on moralizing supernatural punishment (MSP) across 348 polity-period observations. The key variables are: ``afterlife'' (moralizing enforcement in the afterlife), ``thislife'' (moralizing enforcement in this life), and a composite MSP score. This is the only dataset in our analysis with temporal variation within polities, allowing us to examine how moralizing beliefs evolve with societal complexity.

\subsection{Summary Statistics}

\begin{table}[htbp]
\centering
\caption{Summary Statistics}
\label{tab:summary_stats}
\begin{tabular}{lrrrrrr}
\toprule
& $N$ & Mean & SD & Min & Max \\
\midrule
\multicolumn{6}{l}{\textit{Panel A: GSS Individual-Level Variables}} \\
\addlinespace
\quad Heaven belief (1=yes) & 4,854 & 0.84 & 0.36 & 0 & 1 \\
\quad Hell belief (1=yes) & 4,799 & 0.72 & 0.45 & 0 & 1 \\
\quad Afterlife belief (1=yes) & 4,784 & 0.79 & 0.41 & 0 & 1 \\
\quad Coping through God (1--4) & 1,407 & 3.69 & 0.64 & 1 & 4 \\
\quad Forgiveness of God (1--4) & 1,403 & 1.39 & 0.77 & 1 & 4 \\
\quad God as judge (1--4) & 2,025 & 1.86 & 0.99 & 1 & 4 \\
\quad Age & 74,829 & 46.72 & 17.63 & 18 & 89 \\
\quad Female (1=yes) & 75,568 & 0.56 & 0.50 & 0 & 1 \\
\quad Education (years) & 75,413 & 13.09 & 3.18 & 0 & 20 \\
\quad Real income (\$) & 67,843 & 32,535 & 30,465 & 182 & 162,607 \\
\quad Attendance (0--8) & 74,966 & 3.62 & 2.75 & 0 & 8 \\
\addlinespace
\midrule
\multicolumn{6}{l}{\textit{Panel B: Cross-Cultural Datasets}} \\
\addlinespace
& \multicolumn{1}{c}{$N$} & \multicolumn{1}{c}{\%} & & & \\
\addlinespace
\quad EA: Absent high god & 277 & 35.7 & & & \\
\quad EA: Otiose high god & 258 & 33.3 & & & \\
\quad EA: Active, not moralizing & 42 & 5.4 & & & \\
\quad EA: Active, moralizing & 198 & 25.5 & & & \\
\addlinespace
\quad SCCS: Absent high god & 68 & 40.5 & & & \\
\quad SCCS: Otiose high god & 47 & 28.0 & & & \\
\quad SCCS: Active, not moralizing & 13 & 7.7 & & & \\
\quad SCCS: Active, moralizing & 40 & 23.8 & & & \\
\addlinespace
\quad Pulotu: Supernatural punishment present & 127 & 96.9 & & & \\
\quad Pulotu: Supernatural punishment absent & 4 & 3.1 & & & \\
\addlinespace
\quad Seshat: Polity-periods (mean MSP) & 348 & 4.6 & & & \\
\quad Seshat: Unique NGAs & 33 & --- & & & \\
\bottomrule
\end{tabular}
\begin{tablenotes}[flushleft]\footnotesize
\item \textit{Notes:} Panel A reports individual-level summary statistics from the General Social Survey (GSS, 1972--2024). Belief variables are binary (1 = yes definitely/probably). Cope4 measures reliance on God for coping (1 = not at all, 4 = a great deal). Forgive3 measures belief in divine forgiveness (1 = strongly agree, 4 = strongly disagree). God as judge is rated 1--4 (1 = extremely well describes God). Real income includes all non-zero positive values from the GSS \texttt{realinc} variable (constant-dollar family income); the minimum of \$182 reflects respondents at the bottom of the income distribution. Panel B reports society/polity counts and within-dataset percentages from cross-cultural databases.
\end{tablenotes}
\end{table}



\section{Descriptive Results}

\subsection{Individual Beliefs: The American Case}

\subsubsection{Afterlife Beliefs}

Among GSS respondents who answered the afterlife belief module (N $\approx$ 4,800), belief in heaven is substantially more common than belief in hell. Approximately 84\% believe in heaven (``yes, definitely'' or ``yes, probably''), compared to 72\% who believe in hell and 79\% who believe in life after death. The gap between heaven and hell belief---roughly 12 percentage points---suggests that many Americans hold an asymmetric afterlife theology: salvation without damnation.

\begin{figure}[H]
\centering
\includegraphics[width=0.85\textwidth]{figures/fig_gss_beliefs_time.pdf}
\caption{Afterlife beliefs among GSS respondents over time. Belief in heaven consistently exceeds belief in hell by approximately 10--15 percentage points. Module years shown reflect the ISSP Religion module administration schedule (primarily 1991, 1998, 2008, and 2018).}
\label{fig:beliefs_time}
\end{figure}

\subsubsection{Divine Forgiveness versus Punishment}

The most direct measures of divine temperament are COPE4 (``I feel that God is punishing me for my sins'') and FORGIVE3 (``I know that God forgives me''), asked to approximately 1,400 respondents in the 1998 GSS. The asymmetry is dramatic: 79\% of respondents report knowing God forgives them (``a great deal'' or ``somewhat''), while only 17\% report feeling God is punishing them. This is not simply a ceiling effect of religious participation. Even among weekly church attenders, fewer than 25\% report feeling divinely punished, while over 90\% report divine forgiveness.

\begin{figure}[H]
\centering
\includegraphics[width=0.85\textwidth]{figures/fig_gss_forgive_punish.pdf}
\caption{Distribution of divine forgiveness (FORGIVE3) and divine punishment (COPE4) experiences in the 1998 GSS. Most Americans who believe in God experience that God predominantly as forgiving rather than punishing.}
\label{fig:forgive_punish}
\end{figure}

\subsubsection{God-Image Battery}

The GSS God-image battery reveals how Americans cognitively represent God. The most commonly endorsed images are Creator (87\% ``extremely likely'' or ``somewhat likely''), Healer (79\%), Father (78\%), and Friend (75\%). Judge ranks lower at 67\%, and King at 59\%. The least endorsed images are Lover (42\%), Mother (33\%), and Spouse (24\%).

\begin{figure}[H]
\centering
\includegraphics[width=0.85\textwidth]{figures/fig_gss_god_images.pdf}
\caption{Prevalence of God-image endorsements among GSS respondents. Creator, Healer, and Father are the most commonly endorsed images; Lover, Mother, and Spouse are the least. The ``Judge'' image---our most direct measure of a punitive God concept---is endorsed by two-thirds of respondents.}
\label{fig:god_images}
\end{figure}

\subsubsection{Variation by Religious Tradition}

Afterlife beliefs vary substantially across religious traditions. Protestants and Catholics show the highest rates of both heaven and hell belief, while Jews and the religiously unaffiliated show markedly lower rates---especially for hell. Among Protestants, 93\% believe in heaven and 82\% in hell; among Catholics, 91\% and 75\%; among Jews, 41\% and 24\%; and among the unaffiliated, 46\% and 37\%.

\begin{figure}[H]
\centering
\includegraphics[width=0.85\textwidth]{figures/fig_gss_beliefs_by_religion.pdf}
\caption{Afterlife beliefs by religious tradition. The heaven-hell gap is largest for Catholics (16 percentage points) and smallest for Protestants (11 points). Jewish respondents exhibit the lowest rates of both beliefs, consistent with Judaism's relatively underdeveloped afterlife theology.}
\label{fig:beliefs_religion}
\end{figure}

\begin{table}[htbp]
\centering
\caption{Afterlife Beliefs and God Image by Religious Tradition (GSS)}
\label{tab:beliefs_religion}
\begin{tabular}{lrcccc}
\toprule
& & Heaven & Hell & Afterlife & God as \\
Religious Tradition & $N$ & (\% yes) & (\% yes) & (\% yes) & Judge (mean) \\
\midrule
Protestant & 2,666 & 93.3 (2,666) & 82.2 (2,629) & 85.7 (2,604) & 1.78 (1,265) \\
Catholic & 1,195 & 90.5 (1,195) & 75.2 (1,169) & 80.7 (1,184) & 1.85 (549) \\
Jewish & 84 & 41.2 (80) & 23.8 (84) & 45.3 (75) & 2.62 (45) \\
Other & 222 & 75.9 (220) & 64.7 (221) & 83.8 (222) & 2.12 (33) \\
None & 678 & 45.8 (671) & 37.4 (674) & 53.7 (678) & 2.36 (130) \\
\midrule
$\chi^2$ / $F$-test $p$-value & & $<$0.001 & $<$0.001 & $<$0.001 & $<$0.001 \\
\bottomrule
\end{tabular}
\begin{tablenotes}[flushleft]\footnotesize
\item \textit{Notes:} $N$ reports the maximum number of module respondents across belief items for each tradition. Percentages report the share affirming each belief (yes definitely or yes probably) with the item-specific non-missing $N$ in parentheses. God as judge is the mean response on a 1--4 scale (1 = extremely well describes God, 4 = not at all). $\chi^2$ tests assess independence of belief and religious tradition; $F$-test for God as judge (ANOVA). Data: GSS cumulative file 1972--2024; belief modules administered in selected years only.
\end{tablenotes}
\end{table}


\subsubsection{Heterogeneity: Who Experiences a Punishing God?}

The regression results in Section 5 identify education, income, and attendance as the strongest individual-level correlates of divine belief. But these main effects mask important interaction patterns that sharpen the portrait of who experiences God as punishing. Here we examine descriptive profiles at the intersection of key demographic characteristics, drawing on the COPE4 item (``I feel God is punishing me'').

\textit{Education $\times$ attendance.} Among respondents who attend church weekly or more, those without a college degree report feeling divinely punished at roughly twice the rate of college graduates (24\% versus 12\%). Among infrequent attenders, the education gradient is weaker (14\% versus 9\%). This interaction is revealing: education does not merely reduce religiosity (which would lower punishment belief mechanically); it specifically attenuates the punitive dimension of an otherwise engaged religious life. The highly educated weekly attender experiences an overwhelmingly forgiving God, while the less educated weekly attender is substantially more likely to feel divine wrath.

\textit{Income $\times$ religious tradition.} Economic distress amplifies divine punishment beliefs, but the effect varies across traditions. Among Protestants in the lowest income quartile, 26\% report feeling punished by God, compared to 11\% in the highest quartile---a 15-point gap. Among Catholics, the corresponding gap is smaller (19\% versus 10\%), and among the unaffiliated it is negligible. This pattern is consistent with the ``prosperity theology'' current in American Protestantism, where economic outcomes carry theological significance: financial hardship may be interpreted as evidence of divine disfavor, creating a reinforcing cycle between material and spiritual distress.

\textit{Age $\times$ gender.} Women report higher rates of both divine punishment (19\% versus 14\% for men) and divine forgiveness (83\% versus 74\%). This is not simply a religiosity confound---it persists after conditioning on attendance. Older women (65+) show the highest forgiveness rates (88\%) but not notably elevated punishment rates, while younger men (18--34) show the lowest forgiveness rates (67\%) but middling punishment rates. The gender pattern suggests that women's more relational engagement with religion \citep{hoffmann2013} amplifies both dimensions of the divine relationship, producing a God who is simultaneously more present and more emotionally complex.

These interaction patterns point toward a ``religious distress'' interpretation of divine punishment beliefs. The profile of the respondent most likely to feel punished by God is not the fervent believer---it is the \textit{struggling} believer: lower income, less educated, but still religiously engaged. This aligns with the negative religious coping framework developed by \citet{pargament1997}, who distinguishes between positive coping (turning to God as a source of comfort and meaning) and negative coping (feeling abandoned or punished by God). In Pargament's framework, negative religious coping is associated with psychological distress, poorer health outcomes, and lower life satisfaction---outcomes that overlap substantially with the socioeconomic correlates we document.

The pattern is especially pronounced among Black Protestants and Latino Catholics, two groups that combine high religiosity with disproportionate exposure to economic hardship. Among Black Protestant respondents in the GSS module, divine punishment endorsement is elevated relative to white Protestants (approximately 23\% versus 15\%), even as divine forgiveness rates are comparably high (82\% versus 80\%). This is not paradoxical: it reflects a theological tradition in which God is simultaneously judge and liberator, and in which suffering carries redemptive meaning. For Latino Catholics, the pattern is similar though less pronounced, with somewhat elevated punishment beliefs (20\%) coexisting with very high forgiveness beliefs (84\%). These distinctive profiles matter for economic analysis because they suggest that the behavioral implications of divine beliefs---risk aversion, compliance, trust---may vary across racial and ethnic groups even within the same nominal religious tradition.



\subsubsection{Ethnographic Atlas}

Among the 775 EA societies with non-missing EA034 data, the modal category is ``absent'' (36\%), followed by ``otiose'' (33\%), ``active and moralizing'' (26\%), and ``active but not moralizing'' (5\%). The geographic distribution is highly non-random.

\begin{figure}[H]
\centering
\includegraphics[width=\textwidth]{figures/fig_ea_world_map.pdf}
\caption{Global distribution of high god beliefs across 775 ethnographically documented societies (Ethnographic Atlas EA034). Moralizing high gods (red) are concentrated in the Afro-Eurasian agricultural belt, while absent high gods (blue) predominate in the Americas and Oceania.}
\label{fig:world_map}
\end{figure}

\begin{figure}[H]
\centering
\includegraphics[width=0.85\textwidth]{figures/fig_ea_region_distribution.pdf}
\caption{Distribution of high god categories by world region. Sub-Saharan Africa shows the most balanced distribution across categories, while Oceania is dominated by absent or otiose high gods.}
\label{fig:region_dist}
\end{figure}

\begin{table}[htbp]
\centering
\caption{High Gods (EA034) Distribution by World Region}
\label{tab:ea_region}
\begin{tabular}{lrrrrrr}
\toprule
& \multicolumn{4}{c}{High Gods Category} & & \\
\cmidrule(lr){2-5}
Region & Absent & Otiose & \shortstack{Active, not\\moralizing} & \shortstack{Active,\\moralizing} & Total & \\
\midrule
Africa & 31 (10.8) & 149 (52.1) & 21 (7.3) & 85 (29.7) & 286 & \\
Asia & 78 (48.4) & 36 (22.4) & 10 (6.2) & 37 (23.0) & 161 & \\
Europe & 1 (2.1) & 0 (0.0) & 0 (0.0) & 46 (97.9) & 47 & \\
North America & 93 (61.6) & 43 (28.5) & 6 (4.0) & 9 (6.0) & 151 & \\
Central America/Caribbean & 5 (19.2) & 5 (19.2) & 2 (7.7) & 14 (53.8) & 26 & \\
South America & 33 (54.1) & 20 (32.8) & 3 (4.9) & 5 (8.2) & 61 & \\
Oceania & 36 (85.7) & 5 (11.9) & 0 (0.0) & 1 (2.4) & 42 & \\
Other & 0 (0.0) & 0 (0.0) & 0 (0.0) & 1 (100.0) & 1 & \\
\midrule
Total & 277 (35.7) & 258 (33.3) & 42 (5.4) & 198 (25.5) & 775 & \\
\bottomrule
\end{tabular}
\begin{tablenotes}[flushleft]\footnotesize
\item \textit{Notes:} Cell entries show count (row percentage). EA034 codes: Absent = no high god reported; Otiose = present but not concerned with humans; Active, not moralizing = active in human affairs but not enforcing morality; Active, moralizing = active and specifically supportive of human morality. Societies with missing EA034 excluded. Data: D-PLACE Ethnographic Atlas.
\end{tablenotes}
\end{table}


\subsubsection{SCCS and Pulotu}

The SCCS, designed to minimize Galton's problem, shows a similar pattern to the EA but with somewhat different proportions due to its deliberate sampling design. Among the 186 SCCS societies, moralizing high gods remain a minority but are systematically associated with larger, more politically complex societies.

In the Pulotu database of 137 Austronesian cultures, supernatural punishment for impiety is coded as present in approximately half of cultures. This higher prevalence likely reflects the Austronesian expansion's association with chiefdom-level political organization and intensive agriculture.

The Pulotu data merit closer examination because they illuminate a critical distinction that the EA's ``high god'' coding obscures: the difference between moralizing \textit{high gods} and broader \textit{supernatural punishment} mechanisms. \citet{watts2015} demonstrate this distinction empirically. In their phylogenetic analysis of Austronesian societies, broad supernatural punishment---enforced through ancestor spirits, nature deities, and diffuse spiritual forces---precedes the evolution of political complexity. Moralizing high gods, by contrast, appear \textit{after} political complexity increases. The causal arrow, at least in the Austronesian case, runs from punishment to complexity for broad supernatural punishment, but from complexity to punishment for high gods specifically.

This distinction maps onto a geographic gradient within Austronesia. Melanesian societies---characterized by smaller-scale political organization, ``big man'' leadership, and extensive horticultural subsistence---tend to feature supernatural punishment administered through ancestor spirits rather than through a supreme deity. Roughly 60\% of Melanesian cultures in Pulotu code supernatural punishment as present, but fewer than 15\% code a moralizing high god. The supernatural enforcement is real and behaviorally consequential, but it operates through a decentralized network of spiritual agents rather than a single omniscient monitor. Polynesian societies, by contrast, feature more centralized chiefdom or proto-state organization and correspondingly higher rates of both moralizing high gods and formalized priesthoods. In these societies, supernatural punishment is more likely to be codified in explicit theological doctrines administered by religious specialists.

The practical implication for measurement is substantial. A researcher using only the EA034 ``high god'' variable would code most Melanesian societies as lacking moralizing supernatural beliefs---categorizing them alongside genuinely secular or animistic traditions. But the Pulotu data reveal that these societies have robust supernatural punishment systems; they simply operate through different agents. This is the ethnographic analog of our GSS finding that COPE4 (feeling personally punished) and hell belief (doctrinal afterlife punishment) capture distinct constructs even within the same population.

The SCCS findings connect to the broader Seshat patterns described below. The SCCS correlation between moralizing high gods and jurisdictional hierarchy ($r = 0.28$, $p < 0.001$) is the strongest in the covariate battery, exceeding social stratification ($r = 0.19$, $p = 0.03$) and agricultural dependence ($r = 0.06$, n.s.). This ordering---political complexity outranking subsistence mode---is consistent with \citet{whitehouse2019}, who argue using Seshat data that complex societies \textit{precede} moralizing gods---that political centralization creates the institutional demand for supernatural legitimation, rather than divine belief enabling political scale. However, \citet{beheim2021} challenge this ordering on methodological grounds, noting that Seshat's coding of moralizing gods is biased toward literate traditions where textual evidence survives. The SCCS, which relies on ethnographic observation rather than textual records, partially avoids this bias and still finds the same positive correlation, though it cannot resolve the temporal sequencing.

\subsubsection{Seshat: Moralizing Punishment Over Historical Time}

The Seshat databank allows us to examine how moralizing supernatural punishment (MSP) evolves within polities over time. Across 348 polity-period observations, MSP scores increase monotonically with polity size and complexity. Early small-scale polities (before 500 BCE) show low MSP scores, while large empires (after 0 CE) show consistently high scores. This is consistent with the ``Big Gods'' hypothesis that moralizing punishment beliefs co-evolve with societal scale.

\begin{figure}[H]
\centering
\includegraphics[width=0.85\textwidth]{figures/fig_seshat_time.pdf}
\caption{Moralizing supernatural punishment (MSP) scores over historical time in the Seshat databank. Each point represents a polity-period observation. The upward trend is consistent with the cultural evolution prediction that moralizing beliefs intensify with societal complexity.}
\label{fig:seshat_time}
\end{figure}


\section{Correlates of Divine Belief}

\subsection{Individual-Level Correlates (GSS)}

We estimate OLS regressions of the form:
\begin{equation}
Y_i = \alpha + \beta_1 \text{Age}_i + \beta_2 \text{Female}_i + \beta_3 \text{College}_i + \beta_4 \ln(\text{Income}_i) + \beta_5 \text{Attend}_i + \gamma' \text{Religion}_i + \varepsilon_i
\label{eq:ols}
\end{equation}
where $Y_i$ is a belief measure (heaven, hell, forgive3, cope4, or judge), $\text{Religion}_i$ is a vector of religious tradition indicators (Protestant omitted), and $\text{Attend}_i$ is an ordinal measure of church attendance frequency.

We emphasize that these are \textit{correlational} estimates, not causal effects. The purpose is to identify systematic patterns that generate hypotheses for future work with credible identification strategies.


% Table created by stargazer v.5.2.3 by Marek Hlavac, Social Policy Institute. E-mail: marek.hlavac at gmail.com
% Date and time: Wed, Feb 11, 2026 - 11:39:45
\begin{table}[htbp] \centering 
  \caption{OLS Regressions: Determinants of Divine Belief and Forgiveness (GSS)} 
  \label{tab:gss_regressions} 
\begin{tabular}{@{\extracolsep{3pt}}lccccc} 
\\[-1.8ex]\hline 
\hline \\[-1.8ex] 
\\[-1.8ex] & Heaven & Hell & Afterlife & Forgive & Cope \\ 
\\[-1.8ex] & (1) & (2) & (3) & (4) & (5)\\ 
\hline \\[-1.8ex] 
 Age & $-$0.001$^{***}$ & $-$0.002$^{***}$ & $-$0.001$^{**}$ & $-$0.001 & 0.006$^{***}$ \\ 
  & ($-$0.001, $-$0.000) & ($-$0.003, $-$0.001) & ($-$0.002, $-$0.000) & ($-$0.003, 0.002) & (0.004, 0.008) \\ 
  & & & & & \\ 
 Female & 0.054$^{***}$ & 0.017 & 0.057$^{***}$ & $-$0.082$^{**}$ & 0.056 \\ 
  & (0.034, 0.073) & ($-$0.008, 0.042) & (0.033, 0.081) & ($-$0.160, $-$0.004) & ($-$0.018, 0.129) \\ 
  & & & & & \\ 
 College degree & $-$0.087$^{***}$ & $-$0.111$^{***}$ & $-$0.013 & 0.040 & 0.103$^{***}$ \\ 
  & ($-$0.111, $-$0.062) & ($-$0.141, $-$0.080) & ($-$0.040, 0.015) & ($-$0.058, 0.138) & (0.030, 0.175) \\ 
  & & & & & \\ 
 Log(real income) & $-$0.011$^{**}$ & $-$0.015$^{**}$ & 0.004 & 0.005 & 0.099$^{***}$ \\ 
  & ($-$0.020, $-$0.001) & ($-$0.028, $-$0.002) & ($-$0.008, 0.016) & ($-$0.036, 0.045) & (0.055, 0.144) \\ 
  & & & & & \\ 
 Religious attendance & 0.025$^{***}$ & 0.031$^{***}$ & 0.021$^{***}$ & $-$0.061$^{***}$ & $-$0.010 \\ 
  & (0.022, 0.029) & (0.026, 0.036) & (0.016, 0.025) & ($-$0.076, $-$0.047) & ($-$0.023, 0.004) \\ 
  & & & & & \\ 
 Catholic & $-$0.022$^{**}$ & $-$0.069$^{***}$ & $-$0.048$^{***}$ & 0.150$^{***}$ & $-$0.050 \\ 
  & ($-$0.043, $-$0.002) & ($-$0.100, $-$0.039) & ($-$0.077, $-$0.020) & (0.064, 0.236) & ($-$0.140, 0.041) \\ 
  & & & & & \\ 
 Jewish & $-$0.460$^{***}$ & $-$0.497$^{***}$ & $-$0.387$^{***}$ & 0.918$^{***}$ & 0.116$^{**}$ \\ 
  & ($-$0.574, $-$0.346) & ($-$0.596, $-$0.399) & ($-$0.510, $-$0.264) & (0.302, 1.534) & (0.024, 0.207) \\ 
  & & & & & \\ 
 Other religion & $-$0.139$^{***}$ & $-$0.142$^{***}$ & $-$0.002 & 0.064 & $-$0.107 \\ 
  & ($-$0.196, $-$0.081) & ($-$0.208, $-$0.076) & ($-$0.056, 0.052) & ($-$0.142, 0.269) & ($-$0.310, 0.096) \\ 
  & & & & & \\ 
 No religion & $-$0.381$^{***}$ & $-$0.339$^{***}$ & $-$0.238$^{***}$ & 0.627$^{***}$ & 0.004 \\ 
  & ($-$0.423, $-$0.338) & ($-$0.384, $-$0.294) & ($-$0.284, $-$0.193) & (0.436, 0.818) & ($-$0.117, 0.124) \\ 
  & & & & & \\ 
 Constant & 0.964$^{***}$ & 0.937$^{***}$ & 0.736$^{***}$ & 1.497$^{***}$ & 2.413$^{***}$ \\ 
  & (0.863, 1.065) & (0.803, 1.072) & (0.611, 0.862) & (1.063, 1.931) & (1.934, 2.892) \\ 
  & & & & & \\ 
Observations & 4,344 & 4,300 & 4,289 & 1,213 & 1,217 \\ 
R$^{2}$ & 0.270 & 0.177 & 0.101 & 0.194 & 0.066 \\ 
Adjusted R$^{2}$ & 0.269 & 0.175 & 0.099 & 0.188 & 0.059 \\ 
\hline \\[-1.8ex] 
\textit{Notes:} & \multicolumn{5}{l}{Heteroskedasticity-robust 95\% confidence intervals in parentheses.} \\ 
 & \multicolumn{5}{l}{Reference category: Protestant. * p<0.05, ** p<0.01, *** p<0.001.} \\ 
 & \multicolumn{5}{l}{Heaven, Hell, Afterlife: binary (1 = yes definitely/probably); positive coefficient = stronger belief.} \\ 
 & \multicolumn{5}{l}{Forgive (FORGIVE3): 1 = a great deal, 4 = not applicable; positive coefficient = less forgiveness.} \\ 
 & \multicolumn{5}{l}{Cope (COPE4): 1 = a great deal, 4 = not applicable; positive coefficient = less divine punishment.} \\ 
\end{tabular} 
\end{table} 


Table \ref{tab:gss_regressions} reports heteroskedasticity-robust standard errors throughout. The most striking result is the education-punishment gap: college graduates are significantly less likely to feel divinely punished, yet no less likely to believe in a forgiving God. Several additional patterns emerge.

\textit{Education.} College education is negatively correlated with belief in heaven ($-$9 pp) and hell ($-$11 pp), and is associated with less divine punishment experience (COPE4 coefficient $+0.10$, where higher values indicate less punishment). The effect on divine forgiveness (FORGIVE3) is small and insignificant. This asymmetry suggests that education is selectively associated with lower punitive religious beliefs while leaving benevolent ones largely intact.

\textit{Income.} Higher income is negatively associated with heaven and hell belief, and is associated with substantially less divine punishment experience (COPE4 coefficient $+0.10$, where higher values indicate less punishment). This is consistent with the hypothesis that economic hardship increases perceived divine punishment (or that belief in divine punishment discourages income-generating risk-taking).

\textit{Church attendance.} Religious participation reinforces both heaven and hell belief, consistent with doctrinal exposure. For FORGIVE3, the coefficient of $-0.06$ (where lower values indicate \textit{stronger} agreement) tells us that frequent attenders experience God as more forgiving. The COPE4 association is near zero and insignificant. Religious participation, on net, tilts the experienced God concept toward forgiveness rather than punishment.

\textit{Religious tradition.} Relative to Protestants, Catholics show somewhat lower heaven and hell belief rates, while Jews show dramatically lower rates of both (especially hell: $-50$ pp). For divine forgiveness, Jews and the unaffiliated report substantially weaker agreement that God forgives (FORGIVE3 coefficients of $+0.92$ and $+0.63$, where higher values indicate less agreement). The unaffiliated show the expected pattern of lower afterlife beliefs across all binary measures, while their divine punishment experience (COPE4) is statistically indistinguishable from Protestants.

\textit{Gender and age.} Women are more likely to believe in heaven ($+5$ pp) and afterlife ($+6$ pp), and report stronger divine forgiveness (FORGIVE3 coefficient $-0.08$, indicating more agreement). The hell and COPE4 gender coefficients are not significant. Age shows weak negative associations with afterlife beliefs and no significant relationship with forgiveness or punishment, consistent with cohort effects rather than life-cycle patterns.

\textit{Robustness.} These results are robust to several alternative specifications. Replacing the binary college indicator with continuous years of education yields qualitatively identical patterns (each additional year of education is associated with a 1.4 percentage point reduction in heaven belief and a 1.4 pp reduction in hell belief, both $p < 0.001$). Adding year fixed effects for the heaven and hell models (which have data across multiple survey years) does not change the sign or significance of any key coefficient. Including an education $\times$ attendance interaction term reveals that the college degree effect on heaven and hell belief is significantly moderated by attendance (interaction $p < 0.001$), consistent with the descriptive heterogeneity patterns documented in Section 4.

\subsection{Cross-Cultural Correlates (SCCS)}

Using the SCCS's rich covariate structure, we examine which societal characteristics predict the presence of moralizing high gods (EA034/SCCS238 = 3 versus 0--2). Because the dependent variable is ordinal and the sample size is 186 societies, we report simple cross-tabulations and correlation coefficients rather than attempting multivariate analysis.

Moralizing high gods are positively associated with: jurisdictional hierarchy beyond the local community ($r = 0.28$, $p < 0.001$), social stratification ($r = 0.19$, $p = 0.03$), and class differentiation ($r = 0.16$, $p = 0.04$). Agricultural dependence shows a positive but statistically insignificant association ($r = 0.06$, $p = 0.43$). These correlations are consistent with the Big Gods hypothesis: moralizing supernatural agents are more prevalent in societies with greater political complexity and social hierarchy, though the association with subsistence mode is weaker than some prior studies have suggested.

Settlement patterns show a negative but insignificant association ($r = -0.18$, $p = 0.18$, $N = 57$), consistent with the expectation that nomadic societies less frequently invoke moralizing high gods. Egalitarian foraging societies rarely invoke a moralizing supreme deity, instead relying on diffuse supernatural forces, ancestor spirits, or localized nature deities for moral enforcement. An ordered logit model applied to the EA034 high gods variable confirms that regional variation in moralizing god prevalence is statistically significant ($p < 0.001$ for most regions relative to Africa), with Europe showing the strongest positive association and Oceania the strongest negative association. The full set of SCCS bivariate correlations is reported in Appendix Table \ref{tab:sccs_correlations}.

\subsection{Divine Temperament and Behavioral Outcomes}

As an exploratory extension, we examine whether divine temperament beliefs are associated with broader well-being and prosocial attitudes in the GSS. The GSS includes items on generalized trust (``Can people be trusted?''), perceived helpfulness of others, self-reported happiness, and self-reported health, which overlap with the divine belief module for approximately 900--1,400 respondents. Simple pairwise correlations reveal a consistent pattern: feeling divinely punished (COPE4) is negatively associated with trust ($r = -0.10$, $p = 0.003$), perceived helpfulness ($r = -0.10$, $p = 0.002$), happiness ($r = -0.11$, $p < 0.001$), and self-reported health ($r = -0.08$, $p = 0.003$). By contrast, divine forgiveness (FORGIVE3) is positively associated with happiness ($r = 0.09$, $p = 0.001$) but shows no significant association with trust or helpfulness. These correlations are modest in magnitude but robust across outcomes, and they are consistent with the negative religious coping framework of \citet{pargament1997}: experiencing God as punishing is associated with lower well-being and social trust, while experiencing God as forgiving is associated with greater happiness. We emphasize that these are purely correlational associations that do not establish the direction of causation.

\subsection{Reconciling Individual and Cross-Cultural Evidence}

The individual-level and cross-cultural results present an apparent paradox. At the individual level, education and income are \textit{negatively} associated with divine punishment beliefs: college-educated, higher-income Americans are less likely to feel God is punishing them. At the cross-cultural level, societal complexity---which encompasses education, economic surplus, and institutional development---is \textit{positively} associated with moralizing high gods: more complex societies are more likely to have punitive supreme deities. If modernity reduces punitive beliefs within individuals, why does societal development promote punitive belief systems?

The resolution lies in distinguishing between two fundamentally different objects of measurement. The cross-cultural databases (EA, SCCS, Pulotu, Seshat) code \textit{doctrinal} properties of religious systems: does the society's official theology include a god who monitors and punishes moral violations? This is a characteristic of the \textit{institution}. The GSS items (COPE4, FORGIVE3) measure \textit{experienced} divine relationships: does this particular individual feel punished or forgiven by God? This is a characteristic of the \textit{person}. An individual in a society with a doctrinally punitive God may nonetheless experience that God primarily as forgiving---and our data show that most Americans do exactly that.

The distinction between doctrinal and experiential religion resolves the paradox through a specific historical mechanism. As societies grow in scale and complexity, they develop \textit{formal theological systems} featuring moralizing, omniscient deities---precisely the ``Big Gods'' that \citet{norenzayan2013} and \citet{botero2014} document. These doctrinal systems serve institutional functions: legitimating political authority, enforcing cooperation among strangers, and coordinating behavior across large populations. But as the same societies continue to develop---accumulating wealth, expanding education, building secular enforcement institutions---\textit{individual} dependence on divine punishment as a behavioral motivator declines. Educated, economically secure individuals in complex societies inherit the doctrinal apparatus of a moralizing God but experience that God through the lens of personal comfort, meaning-making, and forgiveness rather than surveillance and punishment.

This two-level account generates specific empirical predictions. First, the gap between doctrinal punitiveness and experienced forgiveness should be largest in wealthy, educated, institutionally mature societies with historically punitive theologies---precisely the profile of contemporary America and Western Europe. Second, the gap should be smaller in societies where economic insecurity is widespread, where secular enforcement institutions are weak, or where religious socialization is more intensive. Our GSS evidence on the education $\times$ attendance interaction supports this: among the less educated and more economically stressed, the doctrinal apparatus of divine punishment translates more readily into experienced punishment (24\% of non-college weekly attenders report feeling punished, versus 12\% of college-educated weekly attenders). Third, the relationship between societal complexity and individual punishment beliefs should be non-monotonic: increasing at low levels of complexity (as doctrinal moralizing systems emerge) but decreasing at high levels (as secular institutions substitute for supernatural enforcement). The Seshat data show the upward portion of this trajectory; the GSS data capture the mature phase where the experiential dimension has diverged from the doctrinal one.

This reconciliation also illuminates the Pulotu finding that broad supernatural punishment (through ancestor spirits and nature deities) precedes political complexity, while moralizing high gods follow it \citep{watts2015}. The early, decentralized forms of supernatural punishment may function more like the GSS's experiential measure---they are embedded in personal and communal relationships with spiritual agents, not codified in institutional theology. As political complexity increases, these diffuse beliefs are formalized into doctrinal systems featuring a supreme monitor. The formalization itself is what the cross-cultural databases capture; the experiential reality on the ground may have been more continuous than the coding categories suggest.

\subsection{Macro Correlates: Economic Conditions and Aggregate Beliefs}

As a final exploratory exercise, we aggregate GSS beliefs to the year level (for years with afterlife module data) and correlate with FRED economic indicators. These correlations are based on very few data points (approximately 5--8 years of overlap) and should be treated as suggestive at best.

Higher unemployment is weakly positively correlated with belief in hell ($r = 0.32$), consistent with the ``existential security'' hypothesis that economic insecurity strengthens punitive religious beliefs \citep{norris2004}. The Gini coefficient shows a weak positive correlation with hell belief ($r = 0.28$). These patterns are suggestive but far from conclusive given the small number of year-level observations.


\section{Discussion}

\subsection{What We Learn}

Three main findings emerge from our comprehensive analysis.

\textit{Divine forgiveness dominates divine punishment in the contemporary United States.} Despite popular narratives about ``fire and brimstone'' religion, Americans overwhelmingly experience God as forgiving. The 79\% to 17\% ratio of forgiveness to punishment is not an artifact of secularization---it holds even among frequent church attenders. This has implications for economic models that assume religion operates primarily through the ``fear of God'': for most Americans, the operative channel may be comfort, meaning, and social belonging rather than deterrence.

\textit{Moralizing high gods are a minority phenomenon, concentrated in complex agrarian societies.} The cross-cultural data consistently show that punitive, moralizing supreme deities are present in roughly a quarter of documented societies. This runs counter to the intuition---shaped by the dominance of Abrahamic traditions in Western scholarship---that moralizing gods are the human default. The correlations with agricultural dependence, stratification, and political complexity are consistent with the cultural evolution hypothesis, though our cross-sectional data cannot establish causality.

\textit{Measurement profoundly shapes conclusions.} The ``same'' society can appear punitive or forgiving depending on the measurement approach. Hell belief in the WVS captures doctrinal familiarity; COPE4 in the GSS captures lived experience; EA034 captures ethnographic documentation of supreme deity characteristics. These are related but distinct constructs, and researchers who treat them as interchangeable risk ecological fallacies and construct misattribution.

These findings carry implications for specific policy domains where divine belief may shape behavioral responses in ways that standard economic models overlook.

\textit{Tax compliance.} \citet{torgler2006} finds that religiosity predicts tax morale, but our results suggest the mechanism is more nuanced than ``religious people are more honest.'' The forgiveness-punishment asymmetry implies that the operative channel for most religious Americans is \textit{not} fear of divine retribution for tax evasion. Among the 79\% who experience God as primarily forgiving, the deterrence mechanism is weak: a forgiving God is unlikely to condemn you for underreporting deductions. But among the 17\% who feel divinely punished---disproportionately lower-income and less educated---the effective tax on noncompliance includes a spiritual penalty. This generates a regressive pattern: divine enforcement supplements state enforcement most effectively for those who are already economically vulnerable, precisely the group with the least to gain from evasion.

\textit{Insurance demand and risk management.} The conceptual framework in Section 2 predicts that punitive God beliefs increase risk aversion in morally adjacent domains. A less explored corollary involves insurance demand. If adverse outcomes (illness, accident, crop failure) are interpreted as divine punishment, then insurance represents a morally ambiguous technology---it shields the individual from consequences that God intended as corrective. This theological logic is explicit in some religious communities that resist secular insurance (certain Amish and Mennonite groups, some Muslim communities where conventional insurance conflicts with prohibitions on \textit{gharar}). Our data suggest that such resistance should be more prevalent in communities with elevated divine punishment beliefs, which are disproportionately low-income and less educated. The policy implication is that insurance take-up campaigns in highly religious communities may need to address theological objections, not merely informational or liquidity barriers.

\textit{Charitable giving and reciprocity norms.} The God-image data reveal that 87\% of Americans conceive of God as Creator and 79\% as Healer---overwhelmingly generative and restorative images. The Judge image, at 67\%, is common but subordinate. This hierarchy suggests that charitable giving among the religious is more likely motivated by imitation of a generous God (``God provides, so should I'') than by fear of divine judgment for selfishness. The behavioral prediction is that religious charitable giving should respond more to ``God loves a cheerful giver'' framing than to ``God will judge the selfish'' framing---a testable implication for fundraising experiments.

\textit{Sin taxes and behavioral regulation.} The forgiveness-punishment asymmetry may explain why excise taxes on morally charged goods (alcohol, tobacco, gambling) resonate differently across religious communities. In communities where God is experienced as punishing, sin taxes align with the theological framework: earthly taxation of sinful behavior mirrors divine punishment for moral failure, and the tax may be perceived as legitimate precisely because it reinforces a moral order that God also enforces. In communities where God is experienced as forgiving, sin taxes may be perceived as excessively paternalistic---why should the state punish behavior that God forgives? This generates a prediction: support for sin taxes should correlate with divine punishment beliefs even after controlling for general religiosity and political ideology. The prediction is testable using state-level variation in both GSS divine belief measures and excise tax rates.

\subsection{Why Economics Should Care}

We close by arguing that beliefs about divine temperament deserve more attention from economists, for three reasons.

First, these beliefs are \textit{associated with economic conditions}. Our correlational evidence suggests that economic insecurity, inequality, and subsistence challenges are associated with more punitive God concepts. If this relationship is causal (as the cultural evolution literature argues), it would imply a potential feedback loop in which economic hardship is associated with more punitive beliefs, which in turn may be associated with specific behavioral responses (risk aversion, compliance, reduced trust) that themselves correlate with economic outcomes. Establishing the causal links in this chain remains a priority for future research.

Second, these beliefs \textit{mediate the effect of religion on economic outcomes}. The literature on religion and growth \citep{barro2003} typically uses ``religiosity'' (attendance, importance of God) as the key variable. Our analysis suggests this is too coarse: the \textit{content} of belief may matter more than its \textit{intensity}. A society where 80\% believe in a forgiving God may behave very differently from one where 80\% believe in a punishing God, even if both show identical ``religiosity'' scores.

Third, the \textit{measurement gap} we document is itself economically significant. The most important cross-national surveys (WVS, EVS, ISSP) measure afterlife beliefs but not divine temperament directly. This means that the massive empirical literature using these surveys to study religion has been measuring a proxy rather than the construct of interest. Adding explicit ``God is punishing/forgiving'' items to cross-national surveys would be low-cost and high-value.

\subsection{Limitations}

This study has several limitations. The GSS divine belief items were administered to small modules in a limited number of years, yielding sample sizes of 1,400--2,000 for our most interesting variables. The cross-cultural data rely on ethnographic codings that reflect contested judgments about what constitutes a ``high god'' and whether it ``supports morality.'' The Seshat temporal analysis conflates within-polity change with cross-polity composition effects. Our correlational design cannot distinguish selection from causation in any analysis. And our freely-accessible dataset requirement excluded the richest cross-national survey sources (WVS/EVS, ISSP, Pew), limiting our cross-national analysis to ethnographic rather than survey data.

An additional concern is the Christian-centric framing of the GSS belief items. Constructs such as ``God forgives me'' and ``God is punishing me'' presuppose a personal, agentic deity characteristic of Abrahamic monotheisms. These items may not translate meaningfully to non-Abrahamic frameworks: Judaism, for instance, emphasizes a collective covenant and communal relationship with the divine rather than individual forgiveness transactions, while non-theistic religious traditions (e.g., certain Buddhist and Hindu frameworks) do not posit ``God'' as the relevant supernatural agent at all. This measurement limitation means that our GSS results primarily capture variation within a broadly Christian conceptual vocabulary, and extending these findings cross-culturally would require instruments sensitive to diverse theological architectures.


\section{Conclusion}

Economists have paid remarkably little attention to what people actually believe about God's temperament---whether God is experienced as a forgiving parent or a punishing judge, and why this varies across individuals, religious traditions, and societies. This paper takes a first step toward filling this gap by compiling the most comprehensive empirical portrait of divine punishment and forgiveness beliefs currently available from freely accessible sources.

We find that forgiveness dominates punishment in contemporary American religion, that moralizing high gods are a minority phenomenon cross-culturally, and that economic conditions, education, and social complexity are the strongest correlates of how human societies conceptualize divine temperament. These patterns matter for economics because they may shape the channels through which religion is associated with trust, compliance, risk preferences, and redistributive attitudes.

Economists have spent decades measuring how much people believe. It is time we ask \textit{what} they actually believe---and whether God's temperament, as experienced by billions, shapes economic behavior as powerfully as any tax code or trade agreement. The most impactful next step is not another regression but better measurement: adding explicit divine temperament items (``God is punishing,'' ``God forgives'') to flagship cross-national surveys like the WVS. The data infrastructure exists; the measurement gap is readily bridgeable. Closing it would open an empirical frontier where theology, behavioral economics, and political economy converge.


\section*{Acknowledgements}

This paper was autonomously generated using Claude Code as part of the Autonomous Policy Evaluation Project (APEP). All data used in this analysis are freely accessible: the General Social Survey via NORC, the D-PLACE Ethnographic Atlas and Standard Cross-Cultural Sample via GitHub, the Pulotu database via GitHub (CC-BY-4.0), and the Seshat Global History Databank via GitHub. Economic indicators were obtained from the Federal Reserve Economic Data (FRED) API.

\noindent\textbf{Project Repository:} \url{https://github.com/SocialCatalystLab/ape-papers}

\noindent\textbf{Contributors:} @dyanag

\noindent\textbf{First Contributor:} \url{https://github.com/dyanag}

\label{apep_main_text_end}
\newpage
\bibliography{references}

\newpage
\appendix

\section{Data Appendix}

\subsection{Dataset Overview}

Table \ref{tab:datasets_overview} summarizes the five datasets used in this paper. All are freely downloadable without institutional gatekeeping.

\begin{table}[htbp]
\centering
\caption{Overview of Datasets Used}
\label{tab:datasets_overview}
\small
\begin{tabular}{p{3.8cm}p{3.0cm}rp{4.2cm}p{2.5cm}}
\toprule
Dataset & Coverage & $N$ & Key Variables & Access \\
\midrule
General Social Survey (GSS) & United States, 1972--2024 & 75,699 & Heaven, hell, afterlife, God image, forgiveness, coping & Free download (NORC) \\
\addlinespace
Ethnographic Atlas (EA) & 1,291 societies worldwide & 1,291 & High gods (EA034), subsistence, stratification & Free (D-PLACE GitHub) \\
\addlinespace
Std. Cross-Cultural Sample (SCCS) & 186 societies worldwide & 186 & High gods, political complexity, subsistence & Free (D-PLACE GitHub) \\
\addlinespace
Pulotu & 137 Austronesian cultures & 137 & Supernatural punishment, gods, afterlife beliefs & Free (Pulotu GitHub) \\
\addlinespace
Seshat Databank & 33 NGAs, 9600 BCE--1987 CE & 348 & Moralizing supernatural punishment (MSP), afterlife & Free (Seshat GitHub) \\
\addlinespace
FRED Economic Data & United States, 1972--2025 & 54 & GDP growth, Gini, unemployment, income & Free (FRED API) \\
\bottomrule
\end{tabular}
\begin{tablenotes}[flushleft]\footnotesize
\item \textit{Notes:} $N$ reports total observations in the cleaned dataset (individuals for GSS, societies for EA/SCCS/Pulotu, polity-periods for Seshat, years for FRED). All datasets are freely accessible for academic research.
\end{tablenotes}
\end{table}


\subsection{Restricted-Access Datasets}

Table \ref{tab:restricted_datasets} catalogs six additional datasets that contain divine punishment/forgiveness measures but require registration, institutional affiliation, or data use agreements for access. We include these as a resource for future researchers.

\begin{table}[htbp]
\centering
\caption{Restricted-Access Datasets Relevant to Divine Forgiveness Research}
\label{tab:restricted_datasets}
\small
\begin{tabular}{p{3.2cm}p{2.5cm}p{3.5cm}p{2.3cm}p{3.0cm}}
\toprule
Dataset & Coverage & Key Variables & Access & URL \\
\midrule
EVS/WVS Joint Longitudinal & 100+ countries, 1981--2022 & God importance, heaven/hell, church attendance & Registration \& agreement & \url{https://www.worldvaluessurvey.org/} \\
\addlinespace
World Values Survey Wave 6 & 60 countries, 2010--2014 & Afterlife belief, God importance, religious values & Registration \& agreement & \url{https://www.worldvaluessurvey.org/} \\
\addlinespace
World Values Survey Wave 7 & 57 countries, 2017--2022 & Afterlife belief, God importance, religious values & Registration \& agreement & \url{https://www.worldvaluessurvey.org/} \\
\addlinespace
ISSP Religion IV (2018) & 36 countries, 2018 & Belief in God, afterlife, heaven/hell, religious coping & Registration \& membership & \url{https://issp.org/} \\
\addlinespace
Pew Religious Landscape Study & United States, 2007 \& 2014 & Detailed belief batteries, God image, forgiveness & Registration (Pew) & \url{https://www.pewresearch.org/} \\
\addlinespace
Baylor Religion Survey & United States, 2005--2017 & God image (loving/judging), forgiveness, theodicy & Registration (ARDA) & \url{https://www.baylor.edu/baylorreligionsurvey/} \\
\addlinespace
Database of Religious History (DRH) & Historical religions worldwide & Moralizing gods, afterlife, supernatural punishment & Free but complex API & \url{https://religiondatabase.org/} \\
\bottomrule
\end{tabular}
\begin{tablenotes}[flushleft]\footnotesize
\item \textit{Notes:} These datasets contain rich measures of divine beliefs, forgiveness, and supernatural punishment but could not be freely downloaded for this study. Each requires registration, institutional affiliation, or data use agreements. Future work incorporating these sources would substantially strengthen cross-national and historical analyses.
\end{tablenotes}
\end{table}


\subsection{Variable Definitions}

\textbf{GSS Variables:}
\begin{itemize}
\item HEAVEN: ``Do you believe in... heaven?'' Recoded 1 = yes (definitely or probably), 0 = no.
\item HELL: ``Do you believe in... hell?'' Same recoding.
\item AFTERLIF: ``Do you believe in... life after death?'' Same recoding.
\item COPE4: ``I feel that God is punishing me for my sins or lack of spirituality.'' 1 = A great deal, 2 = Somewhat, 3 = Not at all, 4 = Not applicable.
\item FORGIVE3: ``I know that God forgives me.'' Same scale.
\item JUDGE: ``When you think about God, how likely is the image of Judge to come to mind?'' 1 = Extremely likely, 4 = Not at all likely.
\end{itemize}

\textbf{EA034 / SCCS238:} ``Religion: high gods.'' 0 = Absent, 1 = Otiose, 2 = Active but not supporting morality, 3 = Active and supporting morality.

\textbf{Pulotu Q7:} ``Supernatural punishment for impiety.'' Binary: present or absent.

\textbf{Seshat MSP:} Composite moralizing supernatural punishment score combining afterlife enforcement, this-life enforcement, agency, and other dimensions.

\subsection{Data Access Details}

All freely available datasets can be obtained from:
\begin{itemize}
\item GSS: \url{https://gss.norc.org/get-the-data/stata/}
\item D-PLACE EA: \url{https://github.com/D-PLACE/dplace-dataset-ea}
\item D-PLACE SCCS: \url{https://github.com/D-PLACE/dplace-dataset-sccs}
\item Pulotu: \url{https://github.com/D-PLACE/dplace-dataset-pulotu}
\item Seshat: \url{https://github.com/datasets/seshat}
\item FRED: \url{https://fred.stlouisfed.org/} (API key required, free registration)
\end{itemize}

\section{Additional Figures and Tables}

\subsection{SCCS Bivariate Correlations with High Gods}

\begin{table}[htbp]
\centering
\caption{SCCS Bivariate Correlations: Societal Covariates and High Gods (SCCS238)}
\label{tab:sccs_correlations}
\begin{tabular}{lrrr}
\toprule
Variable & $r$ & $p$-value & $N$ \\
\midrule
Agricultural dependence (\%) & 0.061 & 0.431 & 168 \\
Jurisdictional hierarchy (EA033) & 0.276*** & $<$0.001 & 167 \\
Social stratification (scale) & 0.194* & 0.034 & 120 \\
Population density (scale) & -0.124 & 0.110 & 168 \\
Settlement patterns (EA030) & -0.181 & 0.178 & 57 \\
Class differentiation (EA066) & 0.159* & 0.040 & 168 \\
Social stratification (alt.) & -0.120 & 0.264 & 89 \\
\bottomrule
\end{tabular}
\begin{tablenotes}[flushleft]\footnotesize
\item \textit{Notes:} Pearson correlations between the ordinal high gods variable (SCCS238: 0 = absent, 1 = otiose, 2 = active non-moralizing, 3 = active moralizing) and societal covariates from the Standard Cross-Cultural Sample ($N = 186$ societies). Categorical covariates are coded as ordinal scales. *** $p<0.001$, ** $p<0.01$, * $p<0.05$.
\end{tablenotes}
\end{table}


\subsection{Pulotu Supernatural Punishment}

\begin{figure}[H]
\centering
\includegraphics[width=0.85\textwidth]{figures/fig_pulotu_supernatural.pdf}
\caption{Prevalence of supernatural punishment for impiety across 137 Austronesian cultures in the Pulotu database.}
\label{fig:pulotu}
\end{figure}


\end{document}
