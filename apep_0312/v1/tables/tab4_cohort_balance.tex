\begin{table}[ht]
\centering
\caption{Pre-Treatment State Characteristics by Workers' Compensation Adoption Cohort}
\label{tab:cohort_balance}
\begin{tabular}{lccccc}
\hline\hline
Adoption Cohort & N & Pop. (1000s) & Urban (\%) & Foreign (\%) & Mfg. (\%) \\
\hline
Early (1911--1913) & 22 & 2,299 & 48.1 & 38.8 & 9.4 \\
Mid (1914--1916) & 10 & 1,849 & 36.8 & 23.6 & 6.4 \\
Late (1917--1920) & 11 & 1,342 & 25.5 & 19.6 & 4.2 \\
Never (post-1920) & 5 & 1,124 & 18.3 & 3.1 & 2.8 \\
\hline
All states & 48 & 1,904 & 38.7 & 27.4 & 6.7 \\
\hline\hline
\end{tabular}
\begin{minipage}{0.9\textwidth}
\vspace{0.3cm}
\footnotesize \textit{Notes:} State characteristics computed from IPUMS 1910 census 1\% sample. Population in thousands. Urban, foreign-born, and manufacturing shares are population-weighted percentages. ``Never'' includes states that adopted workers' compensation after 1920 (AR, FL, MS, NC, SC). The pattern confirms selection into early adoption: early-adopting states are more urbanized, have larger foreign-born populations, and higher manufacturing employment shares.
\end{minipage}
\end{table}
