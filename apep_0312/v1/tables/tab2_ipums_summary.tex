\begin{table}[ht]
\centering
\caption{Summary Statistics: IPUMS Census Analysis Sample}
\label{tab:ipums_summary}
\begin{tabular}{lcc}
\hline\hline
 & 1910 & 1920 \\
\hline
Observations & 180,376 & 212,497 \\
Mean age & 35.4 & 36.7 \\
White (\%) & 92.4 & 92.2 \\
Foreign-born (\%) & 44.6 & 40.8 \\
Literate (\%) & 93.4 & 94.7 \\
Married (\%) & 60.2 & 64.7 \\
Urban (\%) & 68.4 & 73.4 \\
Dangerous occupation (\%) & 21.9 & 22.9 \\
Occ. income score & 27.1 & 27.6 \\
\hline\hline
\end{tabular}
\begin{minipage}{0.85\textwidth}
\vspace{0.3cm}
\footnotesize \textit{Notes:} Sample restricted to males aged 18--65 in the labor force with non-farm occupations. Data from IPUMS USA 1\% samples (1910: us1910k; 1920: us1920a). Dangerous occupations include mining, manufacturing operatives, construction, railroad, and lumbering, classified by OCC1950 codes. Occupational income score (OCCSCORE) measures median 1950 income for each occupation category (in hundreds of dollars).
\end{minipage}
\end{table}
