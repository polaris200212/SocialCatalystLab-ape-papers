\begin{table}[ht]
\centering
\caption{Heterogeneity: Effects by Industry Subsample}
\label{tab:heterogeneity}
\begin{tabular}{lccc}
\hline\hline
 & (1) & (2) & (3) \\
 & Mining & Manufacturing & Construction/ \\
 & & & Transportation \\
\hline
ATT (Dangerous occ.) & 0.1143\sym{*} & 0.0561\sym{***} & 0.0240 \\
 & (0.0610) & (0.0190) & (0.0270) \\
95\% CI & [$-$0.005, 0.234] & [0.019, 0.093] & [$-$0.029, 0.077] \\
$t$-statistic & 1.87 & 2.95 & 0.89 \\
\hline
Sample restriction & Mining & Mfg. workers & Constr./Transp. \\
\hline\hline
\end{tabular}
\begin{minipage}{0.9\textwidth}
\vspace{0.3cm}
\footnotesize \textit{Notes:} Each column re-estimates the DR model on a subsample restricted to workers in the indicated industry group. The outcome is the dangerous occupation indicator within each subsample. 95\% CIs computed as $\hat{\tau} \pm 1.96 \times \text{SE}$. Standard errors in parentheses (analytical). \sym{*} $p<0.10$, \sym{**} $p<0.05$, \sym{***} $p<0.01$. These subsample ATTs measure the effect on dangerous occupation classification within each industry, not the effect on industry employment shares.
\end{minipage}
\end{table}
