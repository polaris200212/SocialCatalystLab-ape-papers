\begin{table}[ht]
\centering
\caption{Main Results: Doubly Robust Estimates of Workers' Compensation Effects}
\label{tab:main_results}
\begin{tabular}{lcc}
\hline\hline
 & (1) & (2) \\
 & Dangerous Occ. & Occ. Income \\
 & (IPUMS) & Score (IPUMS) \\
\hline
Workers' comp ATT & 0.0533\sym{***} & 0.6310\sym{***} \\
 & (0.0091) & (0.2385) \\
95\% CI & [0.035, 0.071] & [0.163, 1.099] \\
$t$-statistic & 5.87 & 2.65 \\
\hline
Estimator & DRDID & DRDID \\
Data source & Census 1910, 1920 & Census 1910, 1920 \\
Panel structure & Rep. cross-sect. & Rep. cross-sect. \\
Covariates & Individual & Individual \\
Mean dep.\ var.\ (pre) & 0.219 & 21.4 \\
N & 388,962 & 388,962 \\
Treated states & 43 & 43 \\
Control states & 5 & 5 \\
\hline\hline
\end{tabular}
\begin{minipage}{0.9\textwidth}
\vspace{0.3cm}
\footnotesize \textit{Notes:} Both columns report the ATT from the Sant'Anna and Zhao (2020) improved doubly robust estimator for repeated cross-sections, comparing 1910 to 1920. Treatment is binary: states adopting workers' compensation by 1920 (43 states) versus states adopting after 1920 (5 states: AR, FL, MS, NC, SC). Individual covariates include age, age squared, race, nativity, literacy, marital status, and urban residence. 95\% CIs computed as $\hat{\tau} \pm 1.96 \times \text{SE}$. N reflects the estimation sample after dropping observations with missing covariates (approximately 3,900 observations, or 1\% of the full sample). Standard errors in parentheses (analytical, influence-function based). \sym{*} $p<0.10$, \sym{**} $p<0.05$, \sym{***} $p<0.01$.
\end{minipage}
\end{table}
