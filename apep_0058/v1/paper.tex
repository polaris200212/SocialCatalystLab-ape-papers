\documentclass[12pt]{article}

% Packages
\usepackage[margin=1in]{geometry}
\usepackage{amsmath,amssymb}
\usepackage{graphicx}
\usepackage{booktabs}
\usepackage{natbib}
\usepackage{setspace}
\usepackage{hyperref}
\usepackage{float}
\usepackage{caption}
\usepackage{subcaption}
\usepackage[title]{appendix}
\usepackage{threeparttable}

% Custom environments for notes under tables and figures
\newenvironment{tablenotes}{\par\vspace{0.5em}\footnotesize}{\par}
\newenvironment{figurenotes}{\par\vspace{0.5em}\footnotesize}{\par}

% Formatting
\doublespacing
\setlength{\parindent}{0.5in}
\setlength{\parskip}{0pt}

% Title
\title{\textbf{Dental Therapy Authorization and Oral Health Access:\\Evidence from Staggered State Adoption}}

\author{
APEP Working Paper 74\\[1em]
\small{Autonomous Policy Evaluation Project}
}

\date{January 2026}

\begin{document}

\maketitle

\begin{abstract}
\noindent Dental therapists are mid-level providers authorized to perform preventive and basic restorative dental procedures. Since 2009, thirteen U.S. states have authorized dental therapy practice, motivated by concerns about dental access in underserved communities. This paper provides quasi-experimental evidence on the population-level effects of dental therapy authorization on oral health access. Using a difference-in-differences design with staggered adoption and Callaway-Sant'Anna estimators, I compare dental visit rates in nine authorizing states with identified treatment effects to not-yet-treated states using BRFSS data from 2012--2020. Contrary to policy expectations, I find that dental therapy authorization is associated with a 1.3 percentage point \textit{decrease} in the proportion of adults visiting a dentist (SE = 0.6 pp, $p = 0.041$). Pre-trends tests do not reject the parallel trends assumption ($p = 0.12$). This counterintuitive finding may reflect reverse causation (states with declining dental access are more likely to adopt dental therapy) or compositional changes in the dental workforce. The results caution against expecting population-level improvements from dental therapy authorization alone and highlight the importance of understanding why states adopt such policies.

\vspace{1em}
\noindent\textbf{JEL Codes:} I11, I18, J44

\noindent\textbf{Keywords:} Dental therapy, oral health, access to care, difference-in-differences, healthcare workforce
\end{abstract}

\newpage

\section{Introduction}

Dental care access remains a persistent challenge in the United States. Over one-third of adults did not visit a dentist in the past year, with access particularly limited in rural areas and among low-income populations \citep{cdc2020oral}. The geographic maldistribution of dentists leaves many communities with inadequate provider supply, contributing to preventable dental disease and costly emergency department visits for dental conditions \citep{nasem2011oral}.

In response to these access challenges, policymakers have increasingly turned to workforce innovations. Dental therapists---mid-level providers trained to perform preventive services and basic restorative procedures---represent one such innovation. Modeled on international precedents (over 50 countries utilize dental therapists), the dental therapy model aims to extend practice capacity by allowing dentists to delegate routine procedures while focusing on complex cases. Proponents argue this can expand access, particularly in underserved areas where recruiting dentists is difficult \citep{nashp2019dental}.

Since Minnesota became the first state to authorize dental therapy in 2009, twelve additional states have followed, creating substantial geographic and temporal variation in policy adoption. This paper exploits this variation to estimate the causal effect of dental therapy authorization on population-level dental access.

To my knowledge, this is the first quasi-experimental study applying modern difference-in-differences methods to evaluate dental therapy's population-level effects. Prior evidence consists primarily of descriptive studies, stakeholder surveys, and case studies from Minnesota \citep{glasrud2017dental}. While such studies suggest high-quality care delivery and positive provider experiences, they cannot identify causal effects on access at scale.

Using data from the Behavioral Risk Factor Surveillance System (BRFSS) covering 2012--2020 and the Callaway-Sant'Anna (2021) estimator for staggered difference-in-differences, I find---contrary to policy expectations---that dental therapy authorization is associated with a statistically significant \textit{decrease} in adult dental visit rates. The estimated average treatment effect on the treated (ATT) is $-1.3$ percentage points (SE = 0.6 pp, $p = 0.041$). Pre-trends tests support the parallel trends assumption ($p = 0.12$).

This counterintuitive finding admits several interpretations. Most plausibly, it reflects selection into treatment: states may adopt dental therapy authorization precisely because they are experiencing declining dental access or anticipate future declines. If authorization is a policy response to deteriorating access rather than a cause of improved access, the observed negative association would be expected. Alternative explanations include compositional changes in the dental workforce or statistical artifacts from the small number of treated states.

The findings counsel caution about expecting rapid, measurable improvements in population-level oral health access from dental therapy authorization. Understanding the political economy of why states adopt such policies may be as important as evaluating their effects.

\section{Background}

\subsection{The Dental Access Problem}

Dental disease is among the most prevalent chronic conditions in the United States. Untreated tooth decay affects over 25\% of adults, with rates substantially higher among low-income populations \citep{dye2015trends}. Unlike most medical conditions, dental care is often excluded from health insurance coverage, creating financial barriers to access. Geographic barriers compound financial ones: dental health professional shortage areas (HPSAs) cover large portions of rural America \citep{hrsa2020shortage}.

The consequences of inadequate dental access extend beyond oral health. Untreated dental conditions lead to approximately 2 million emergency department visits annually, with associated costs exceeding \$2 billion \citep{wall2014dental}. Poor oral health is associated with systemic conditions including cardiovascular disease and adverse pregnancy outcomes \citep{garcia2017oral}.

\subsection{Dental Therapists as a Policy Response}

Dental therapists are mid-level providers who perform a subset of dental procedures under dentist supervision. Their scope of practice typically includes:
\begin{itemize}
    \item Preventive services (cleanings, sealants, fluoride application)
    \item Basic restorative procedures (fillings, simple extractions)
    \item Patient education and triage
\end{itemize}

The model originated in New Zealand in 1921 and has since expanded to over 50 countries. Proponents argue that training dental therapists requires less time and expense than training dentists (typically 2--3 years versus 8+ years), enabling faster workforce expansion.

\subsection{State Adoption of Dental Therapy}

Minnesota became the first state to authorize dental therapy through legislation in 2009, followed by Maine (2014), Vermont (2016), and others. As of 2024, thirteen states have authorized dental therapy (a fourteenth, Alaska, authorizes dental health aide therapists on tribal lands under a separate framework).

Table \ref{tab:adoption} summarizes the timing of state authorizations.

\begin{table}[H]
\centering
\caption{Dental Therapy Authorization Timing}
\label{tab:adoption}
\begin{tabular}{llccc}
\toprule
State & Auth. Year & Est. Cohort & Pre-periods & Post-periods \\
\midrule
Minnesota & 2009 & --- & 0 & 5 \\
Maine & 2014 & 2014 & 1 & 4 \\
Vermont & 2016 & 2016 & 2 & 3 \\
Arizona & 2018 & 2018 & 3 & 2 \\
Michigan & 2018 & 2018 & 3 & 2 \\
New Mexico & 2018 & 2018 & 3 & 2 \\
Idaho & 2019 & 2020 & 4 & 1 \\
Nevada & 2019 & 2020 & 4 & 1 \\
Oregon & 2020 & 2020 & 4 & 1 \\
Washington & 2020 & 2020 & 4 & 1 \\
Connecticut & 2021 & --- & 5 & 0 \\
Colorado & 2022 & --- & 5 & 0 \\
Wisconsin & 2024 & --- & 5 & 0 \\
\bottomrule
\end{tabular}
\begin{tablenotes}
\small
\item \textit{Notes:} ``Auth. Year'' = legislative authorization year. ``Est. Cohort'' = treatment cohort for estimation (first post-authorization observation period). Minnesota (2009) is dropped entirely---it is always-treated in the observed window. CT, CO, WI (post-2020) are not-yet-treated in the window and serve as comparison units. Pre/post periods refer to bi-annual BRFSS data availability (2012, 2014, 2016, 2018, 2020).
\end{tablenotes}
\end{table}

\section{Data}

\subsection{Outcome Data: BRFSS/NOHSS}

The primary outcome data come from the National Oral Health Surveillance System (NOHSS), which provides state-level aggregates derived from the CDC's Behavioral Risk Factor Surveillance System (BRFSS). BRFSS is an annual telephone survey of health behaviors covering all 50 states, with state-level sample sizes exceeding 5,000 respondents.

The key outcome variable is the proportion of adults aged 18+ who visited a dentist or dental clinic in the past year. This measure is available biennially (even years) from 2012--2020, providing five time periods for analysis.

\subsection{Treatment Data}

Treatment timing comes from legislative records and the Oral Health Workforce Research Center. I record the year of legislative authorization. For estimation, I define the \textit{treatment cohort} as the first BRFSS observation period at or after authorization. This distinction matters for odd-year adopters: Idaho and Nevada authorized in 2019, but since the outcome data is biennial (even years only), their treatment cohort for estimation is 2020. Table~\ref{tab:adoption} shows both the legislative year and the estimation cohort year.

\subsection{Sample and Summary Statistics}

The analysis sample includes \textbf{245 state-year observations}: 49 states $\times$ 5 time periods (2012, 2014, 2016, 2018, 2020). Minnesota is excluded entirely because, having authorized dental therapy in 2009, it is ``always-treated'' in the observed window and cannot serve as either a treated unit with identified effects or a valid control.

Of the remaining 49 states, 12 authorized dental therapy at some point. For difference-in-differences estimation:

\begin{itemize}
    \item \textbf{9 treated states} with identified ATTs: ME, VT, AZ, MI, NM, ID, NV, OR, WA (authorized 2014--2020)
    \item \textbf{40 comparison states}: 37 never-authorized plus 3 not-yet-treated within the data window (CT, CO, WI authorized 2021--2024)
\end{itemize}

Table \ref{tab:summary} presents summary statistics.

\begin{table}[H]
\centering
\caption{Summary Statistics by Treatment Status (Estimation Sample)}
\label{tab:summary}
\begin{tabular}{lcc}
\toprule
& Comparison States & Treated States \\
\midrule
N (state-years) & 200 & 45 \\
N (states) & 40 & 9 \\
Mean dental visit rate & 0.651 & 0.673 \\
SD dental visit rate & 0.052 & 0.044 \\
Mean sample size per state-year & 9,067 & 9,385 \\
\bottomrule
\end{tabular}
\begin{tablenotes}
\small
\item \textit{Notes:} Estimation sample excludes Minnesota (always-treated in observed window). Comparison states include 37 never-authorized plus CT, CO, WI (not-yet-treated through 2020). Total N = 245 state-years (49 states $\times$ 5 periods).
\end{tablenotes}
\end{table}

Treated states have slightly higher mean dental visit rates over the sample period (67.3\% vs. 65.1\%), consistent with selection into treatment by states with stronger oral health infrastructure or political will for dental access policy.

\section{Empirical Strategy}

\subsection{Difference-in-Differences Framework}

I estimate the causal effect of dental therapy authorization on dental visit rates using a difference-in-differences design with staggered treatment timing. The identifying assumption is that, absent treatment, dental visit rates would have followed parallel trends in treated and control states.

\subsection{Callaway-Sant'Anna Estimator}

Standard two-way fixed effects (TWFE) estimators are biased under treatment effect heterogeneity with staggered adoption \citep{goodman-bacon2021}. I therefore employ the Callaway-Sant'Anna (2021) estimator, which:

\begin{enumerate}
    \item Estimates group-time average treatment effects $ATT(g,t)$ for each cohort $g$ at each time $t$
    \item Aggregates to overall ATT using appropriate weights
    \item Avoids using already-treated units as controls
\end{enumerate}

The primary specification uses not-yet-treated states as the comparison group, which includes all states untreated at each time period. Robustness checks restrict to states untreated through 2020 (the 37 never-authorized states plus CT, CO, WI which were authorized after the data window).

\subsection{Treatment of Minnesota and Odd-Year Adopters}

Minnesota, the earliest adopter (2009), is ``always-treated'' in the 2012--2020 data window---all five observations are post-treatment. This poses a fundamental design problem: Minnesota cannot contribute an identified ATT (no pre-period for the DiD contrast), and it cannot serve as a valid control (it is not ``never-treated'' or ``not-yet-treated''). I therefore \textit{exclude Minnesota entirely from the estimation sample}. The analysis uses 49 states $\times$ 5 periods = 245 state-year observations.

For states authorizing in odd years (Idaho and Nevada in 2019), I define the treatment cohort as the first post-authorization observation period (2020). Table~\ref{tab:adoption} shows legislative authorization years; the estimation uses the mapped observation-period timing. This approach is conservative: any effects appearing in 2020 reflect at most one year of potential implementation.

\subsection{Identification Threats}

Several threats to identification warrant discussion:

\textbf{Selection into treatment.} States authorizing dental therapy may differ systematically from non-adopters. Critically, states may adopt precisely \textit{because} they anticipate declining dental access---in which case the negative association reflects reverse causation rather than a causal effect of authorization.

\textbf{Concurrent policies.} Other policies (e.g., Medicaid dental expansions) may coincide with dental therapy authorization.

\textbf{Implementation lags.} Authorization is necessary but not sufficient for dental therapist practice. Years may elapse before practitioners enter the workforce.

\section{Results}

\subsection{Pre-Trends}

Figure \ref{fig:raw_trends} plots mean dental visit rates by treatment status over time. Both groups exhibit roughly parallel trends, with adopter states maintaining slightly higher levels throughout.

\begin{figure}[H]
\centering
\includegraphics[width=0.9\textwidth]{figures/raw_trends.png}
\caption{Dental Visit Rates by Treatment Status}
\label{fig:raw_trends}
\begin{figurenotes}
\textit{Notes:} Figure plots mean dental visit rates (proportion of adults visiting dentist in past year) for dental therapy adopter states versus never-adopter states. Data from BRFSS/NOHSS, even years 2012--2020.
\end{figurenotes}
\end{figure}

\subsection{Event Study}

Figure \ref{fig:event_study} presents event study estimates using the Callaway-Sant'Anna estimator.

\begin{figure}[H]
\centering
\includegraphics[width=0.95\textwidth]{figures/event_study.png}
\caption{Event Study: Effect of Dental Therapy Authorization on Dental Visits}
\label{fig:event_study}
\begin{figurenotes}
\textit{Notes:} Figure plots event study coefficients from Callaway-Sant'Anna (2021) estimator. Comparison group: not-yet-treated states. Shaded region shows 95\% confidence intervals. Event time measured in years relative to estimation cohort year (see Table~\ref{tab:adoption}). Coefficients are weighted averages across cohorts with data at each event time; later cohorts (2018, 2020) contribute only to earlier event times due to the 2012--2020 data window. Reference period is $t=-2$ years (omitted); $t=-4$ is the only pre-treatment coefficient used in the parallel trends test.
\end{figurenotes}
\end{figure}

The pre-treatment coefficient at $t=-4$ years is $-1.2$ pp and statistically insignificant, supporting the parallel trends assumption. A joint test fails to reject the null of zero pre-treatment effects ($\chi^2(1) = 2.41$, $p = 0.12$).

Post-treatment coefficients are consistently negative: near zero at $t=0$ (first post-authorization observation), then increasingly negative at $t=+2$ years and $t=+4$ years. The coefficient at $t=+2$ years is statistically significant. This pattern suggests effects that worsen over time, inconsistent with gradual implementation leading to improvement.

\subsection{Overall Average Treatment Effect}

Table \ref{tab:main_results} presents the main results.

\begin{table}[H]
\centering
\caption{Effect of Dental Therapy Authorization on Dental Visit Rates}
\label{tab:main_results}
\begin{tabular}{lcccc}
\toprule
Specification & ATT & SE & $p$-value & 95\% CI \\
\midrule
Not-yet-treated comparison & $-$0.013 & 0.006 & 0.041 & [$-$0.025, $-$0.001] \\
Untreated through 2020 & $-$0.011 & 0.006 & 0.080 & [$-$0.023, 0.001] \\
\bottomrule
\end{tabular}
\begin{tablenotes}
\small
\item \textit{Notes:} Table reports overall ATT from Callaway-Sant'Anna (2021) estimator. N = 245 state-years (49 states $\times$ 5 periods; Minnesota excluded). 9 treated states across 4 treatment cohorts: 2014 (ME), 2016 (VT), 2018 (AZ, MI, NM), 2020 (ID, NV, OR, WA). Primary specification uses not-yet-treated comparison; robustness uses states untreated through 2020. Standard errors clustered at state level.
\end{tablenotes}
\end{table}

The estimated overall ATT is $-1.3$ percentage points using not-yet-treated comparison ($p = 0.041$) and $-1.1$ percentage points using states untreated through 2020 ($p = 0.080$). The 95\% confidence interval for the primary specification excludes zero and any positive effect.

\section{Discussion}

\subsection{Interpretation of Negative Effects}

The finding that dental therapy authorization is associated with \textit{decreased} dental visit rates is counterintuitive and warrants careful interpretation.

\textbf{Reverse causation.} The most plausible explanation is that states adopt dental therapy authorization in response to declining or stagnating dental access. If policymakers observe deteriorating access and respond by authorizing a new provider type, the negative association would be expected even if authorization has no causal effect (or a small positive effect offset by the underlying decline). This interpretation is consistent with the policy narrative: dental therapy is promoted as a solution to access problems.

\textbf{Implementation failure.} Even after authorization, it takes years to establish training programs, graduate practitioners, and integrate them into dental practices. If the post-treatment window is too short to capture implementation, we would observe the pre-existing decline without the benefits.

\textbf{Compositional effects.} Dental therapists may substitute for rather than supplement dentist labor. If dentists reduce their hours or retire earlier when dental therapists become available, net provider supply could be unchanged or reduced.

\textbf{Statistical artifact.} With only 9 treated states (and limited pre-treatment data for early adopters), results may be driven by idiosyncratic factors in a few states.

\subsection{Limitations}

Several limitations warrant caution.

First, the data are biennial, providing limited temporal resolution. The outcome measures ``past year'' dental visits, creating a recall window that may span authorization dates. For the 2020 cohort (ID, NV, OR, WA), the 2020 observation is also their first post-authorization period, meaning their ``post-treatment'' measurement partially reflects pre-authorization behavior.

Second, early adopters (Maine) have only one pre-treatment period (2012), limiting pre-trends assessment for this cohort. Minnesota is excluded entirely due to having no pre-treatment data.

Third, the 2012--2020 window may be too short to observe implementation effects, particularly for later adopters. Dental therapists require training and integration into practices; effects may take years to materialize at the population level.

Fourth, we cannot distinguish among the alternative interpretations offered above.

\subsection{Policy Implications}

These findings counsel against expecting rapid, measurable improvements in population-level dental access from dental therapy authorization alone. The negative association likely reflects the circumstances under which states adopt rather than the effects of the policy itself.

More fundamentally, the results highlight the importance of understanding policy endogeneity. Workforce policies are not randomly assigned; they are adopted in response to perceived problems. Evaluating such policies requires accounting for why they were adopted, not just measuring outcomes afterward.

\section{Conclusion}

This paper provides quasi-experimental evidence on the population-level effects of state dental therapy authorization. Using staggered difference-in-differences methods and BRFSS data covering 2012--2020, I find that authorization is associated with a 1.3 percentage point decrease in adult dental visit rates---a statistically significant negative effect contrary to policy expectations.

This counterintuitive finding most likely reflects reverse causation: states adopt dental therapy authorization precisely because they face declining dental access. The policy is a symptom of the problem, not necessarily a failed solution.

Future research should examine longer time horizons as dental therapist workforces mature, employ methods that can address policy endogeneity (such as synthetic control or instrumental variables), and utilize more granular data to identify effects on targeted populations. Understanding the political economy of dental therapy adoption may be as important as evaluating its effects.

\newpage
\bibliographystyle{aer}
\begin{thebibliography}{99}

\bibitem[Callaway and Sant'Anna(2021)]{callaway2021}
Callaway, Brantly, and Pedro H.C. Sant'Anna. 2021. ``Difference-in-Differences with Multiple Time Periods.'' \textit{Journal of Econometrics} 225(2): 200--230.

\bibitem[CDC(2020)]{cdc2020oral}
Centers for Disease Control and Prevention. 2020. ``Oral Health Surveillance Report 2019.'' Atlanta, GA: CDC.

\bibitem[Dye et al.(2015)]{dye2015trends}
Dye, Bruce A., Gina Thornton-Evans, Xianfen Li, and Timothy J. Iafolla. 2015. ``Dental Caries and Tooth Loss in Adults in the United States, 2011--2012.'' \textit{NCHS Data Brief} No. 197.

\bibitem[Garcia et al.(2017)]{garcia2017oral}
Garcia, Raul I., Michelle M. Henshaw, and Elizabeth A. Krall. 2017. ``Relationship Between Periodontal Disease and Systemic Health.'' \textit{Periodontology 2000} 44(1): 106--116.

\bibitem[Glasrud et al.(2017)]{glasrud2017dental}
Glasrud, Paul, Simona Iosud, and Tryfon Tsoukalas. 2017. ``Evaluation of Minnesota's Dental Therapist Workforce.'' Minneapolis: Minnesota Department of Health.

\bibitem[Goodman-Bacon(2021)]{goodman-bacon2021}
Goodman-Bacon, Andrew. 2021. ``Difference-in-Differences with Variation in Treatment Timing.'' \textit{Journal of Econometrics} 225(2): 254--277.

\bibitem[HRSA(2020)]{hrsa2020shortage}
Health Resources and Services Administration. 2020. ``Designated Health Professional Shortage Areas Statistics.'' Rockville, MD: HRSA.

\bibitem[NASEM(2011)]{nasem2011oral}
National Academies of Sciences, Engineering, and Medicine. 2011. \textit{Improving Access to Oral Health Care for Vulnerable and Underserved Populations}. Washington, DC: National Academies Press.

\bibitem[NASHP(2019)]{nashp2019dental}
National Academy for State Health Policy. 2019. ``Dental Therapists: Improving Access to Oral Health.'' Washington, DC: NASHP.

\bibitem[Wall et al.(2014)]{wall2014dental}
Wall, Thomas, Kamyar Nasseh, and Marko Vujicic. 2014. ``Emergency Department Use for Dental Conditions Continues to Increase.'' \textit{Health Policy Institute Research Brief}. American Dental Association.

\end{thebibliography}

\newpage
\begin{appendices}

\section{State-Level Trends}

Figure \ref{fig:state_trends} presents dental visit trends for each adopting state, with vertical lines indicating treatment timing.

\begin{figure}[H]
\centering
\includegraphics[width=0.95\textwidth]{figures/trends_by_state.png}
\caption{Dental Visit Rates by Adopting State}
\label{fig:state_trends}
\begin{figurenotes}
\textit{Notes:} Figure plots dental visit rates for each dental therapy adopter state. Dashed vertical lines indicate treatment year. Open circles indicate pre-treatment periods; filled circles indicate post-treatment periods.
\end{figurenotes}
\end{figure}

\section{Robustness Checks}

The main results are robust to using states untreated through 2020 as the comparison group (ATT = $-$1.1 pp, SE = 0.6 pp, $p = 0.08$). This comparison group includes the 37 never-authorized states plus CT, CO, and WI (authorized 2021--2024). Results are qualitatively similar though slightly attenuated, consistent with later adopters having different baseline characteristics.

\section{Data Sources}

\textbf{Outcome data:} National Oral Health Surveillance System (NOHSS) Adult Indicators, derived from BRFSS. Available at \url{https://healthdata.gov/dataset/NOHSS-Adult-Indicators}.

\textbf{Treatment data:} Oral Health Workforce Research Center, state legislative records.

\end{appendices}

\end{document}
