\documentclass[12pt]{article}

% UTF-8 encoding and fonts
\usepackage[utf8]{inputenc}
\usepackage[T1]{fontenc}
\usepackage{lmodern}

% Page setup
\usepackage[margin=1in]{geometry}
\usepackage{setspace}
\onehalfspacing

% Typography
\usepackage{microtype}

% Math and symbols
\usepackage{amsmath,amssymb}

% Graphics
\usepackage{graphicx}
\usepackage{float}
\usepackage{subcaption}

% Tables
\usepackage{booktabs}
\usepackage{array}
\usepackage{multirow}
\usepackage{threeparttable}
\usepackage{longtable}
\usepackage{pdflscape}
\usepackage{siunitx}
\sisetup{detect-all=true, group-separator={,}, group-minimum-digits=4}

% For modelsummary tables
\usepackage{tabularray}
\usepackage{codehigh}
\usepackage[normalem]{ulem}
\UseTblrLibrary{booktabs}
\UseTblrLibrary{siunitx}
\newcommand{\tinytableTabularrayUnderline}[1]{\underline{#1}}
\newcommand{\tinytableTabularrayStrikeout}[1]{\sout{#1}}
\NewTableCommand{\tinytableDefineColor}[3]{\definecolor{#1}{#2}{#3}}

% Bibliography
\usepackage{natbib}
\bibliographystyle{aer}

% Hyperlinks
\usepackage{hyperref}
\hypersetup{
    colorlinks=true,
    linkcolor=blue,
    citecolor=blue,
    urlcolor=blue
}
\usepackage[nameinlink,noabbrev]{cleveref}

% Captions
\usepackage{caption}
\captionsetup{font=small,labelfont=bf}

% Section formatting
\usepackage{titlesec}
\titleformat{\section}{\large\bfseries}{\thesection.}{0.5em}{}
\titleformat{\subsection}{\normalsize\bfseries}{\thesubsection}{0.5em}{}

% Custom commands
\newcommand{\E}{\mathbb{E}}
\newcommand{\Var}{\text{Var}}
\newcommand{\Cov}{\text{Cov}}
\newcommand{\ind}{\mathbb{I}}
\newcommand{\sym}[1]{\ifmmode^{#1}\else\(^{#1}\)\fi}

\title{Technological Obsolescence and Populist Voting: \\ Evidence from U.S. Metropolitan Areas}
\author{APEP Autonomous Research\thanks{Autonomous Policy Evaluation Project. Correspondence: scl@econ.uzh.ch} \and SocialCatalystLab \\ @SocialCatalystLab}
\date{\today}

\begin{document}

\maketitle

\begin{abstract}
\noindent
Does technological obsolescence predict support for populist candidates? Using novel data on the modal age of technologies employed across 896 U.S. Core-Based Statistical Areas (CBSAs) from 2010--2023, we examine the relationship between technology vintage and Trump vote share in the 2016, 2020, and 2024 presidential elections. We find a robust positive cross-sectional correlation: in the pooled bivariate specification, a 10-year increase in modal technology age is associated with approximately 1.8 percentage point higher Trump vote share (1.1 pp with size controls). However, this relationship appears driven by geographic sorting rather than causal effects. Within-CBSA variation in technology age does not predict changes in voting behavior, and initial technology age does not predict gains in Trump support over time. These patterns suggest that regions using older technologies differ in persistent, unobserved characteristics that correlate with populist voting, rather than technological obsolescence directly causing political realignment. Our findings highlight the importance of distinguishing between correlation and causation when interpreting the economic roots of populism.
\end{abstract}

\vspace{1em}
\noindent\textbf{JEL Codes:} D72, O33, P16, R11 \\
\noindent\textbf{Keywords:} populism, technology, voting behavior, Trump, metropolitan areas, geographic polarization

\newpage

\section{Introduction}

The rise of populist movements across advanced democracies has prompted intense scholarly debate about its economic origins. A leading explanation emphasizes the role of technological change and automation in creating economic insecurity among workers in routine-intensive occupations \citep{autor2013china, acemoglu2020robots, frey2017future}. According to this view, workers in regions that have failed to adopt new technologies face stagnant wages, diminishing job prospects, and growing resentment toward elites and institutions---sentiments that populist candidates have successfully channeled into electoral support.

This paper tests a specific prediction of this hypothesis: do regions using older, more obsolete technologies exhibit higher support for populist candidates? Using novel data on the modal age of technologies employed across U.S. metropolitan areas, we examine whether technological obsolescence predicts Trump vote share in the 2016, 2020, and 2024 presidential elections.

Our analysis exploits rich variation in technology vintage across 896 Core-Based Statistical Areas (CBSAs). The technology data, drawn from establishment-level surveys, measures the modal age of capital equipment and production technologies within each metropolitan area, providing a direct indicator of technological modernity distinct from more commonly used proxies such as routine task intensity or automation exposure.

We document a robust positive cross-sectional correlation between technology age and Trump voting. A 10-year increase in modal technology age is associated with approximately 1.8 percentage points higher Republican vote share, conditional on year fixed effects and CBSA size controls. This relationship holds across metropolitan and micropolitan areas, persists in all three election years, and is robust to alternative measures of technology vintage.

However, our identification tests cast doubt on a causal interpretation. First, when we include CBSA fixed effects to exploit only within-metropolitan variation over time, the technology coefficient drops to near zero and loses statistical significance. Second, and more tellingly, initial technology age does not predict \textit{changes} in Trump support over time. CBSAs with older technologies in 2016 did not see disproportionate gains in Trump vote share between 2016 and 2020 or between 2020 and 2024. This pattern is inconsistent with a causal story in which technology drives political preferences; instead, it suggests that technology age and Trump support are jointly determined by persistent local characteristics.

Our findings complement a growing literature on the economic determinants of populist voting. \citet{autor2020importing} document that exposure to Chinese import competition increased Republican vote share in affected U.S. counties. \citet{bursztyn2024immigrant} show that long-term exposure to immigrants shapes attitudes and voting behavior. \citet{rodrik2021economics} provides a comprehensive review linking economic grievances to populist support across countries. Our contribution is to examine a specific, understudied dimension of economic geography---technological modernity---and to carefully distinguish correlation from causation.

The remainder of this paper proceeds as follows. Section 2 describes our data sources and sample construction. Section 3 develops a conceptual framework linking technological obsolescence to political preferences. Section 4 presents our empirical strategy. Section 5 reports main results and robustness checks. Section 6 discusses mechanisms and alternative interpretations. Section 7 concludes.

Before proceeding, we note that our analysis is purely observational. We cannot randomly assign technology vintage to metropolitan areas, and the identifying variation we exploit---differences in technology age across CBSAs and over time---may be confounded by unobserved factors. Our identification strategy aims to distinguish correlation from causation by testing multiple predictions that differ under causal and sorting interpretations. While we cannot definitively prove the absence of causal effects, the pattern of results strongly suggests that the technology-voting correlation reflects sorting rather than direct causation.


\section{Institutional Background and Data}

\subsection{Technology Adoption and Regional Inequality}

The pace of technology adoption varies substantially across U.S. regions. While coastal metropolitan areas and major innovation hubs tend to employ cutting-edge technologies, many smaller cities and rural-adjacent areas continue to rely on older capital equipment and production processes. This variation reflects differences in industry composition, workforce skills, access to capital, and historical patterns of investment.

\citet{acemoglu2022new} argue that new technologies complement high-skilled workers while substituting for routine tasks performed by middle-skilled workers. Regions that fail to adopt new technologies may therefore face a ``double penalty'': lower productivity growth and limited displacement of routine workers, but also reduced opportunities for the high-skilled workers who might otherwise drive local economic dynamism.

From a political economy perspective, workers in technologically stagnant regions may be particularly susceptible to populist appeals. They face economic uncertainty not from dramatic job losses (as in trade-affected regions), but from gradual erosion of wages and opportunities relative to more dynamic areas. This ``slow burn'' of relative decline may generate resentment toward elites perceived as benefiting from technological change while leaving these regions behind.

The geographic concentration of technological modernity has accelerated in recent decades. \citet{moretti2012new} documents the emergence of a ``great divergence'' in which a small number of metropolitan areas have captured the lion's share of innovation-sector growth while many traditional manufacturing regions have stagnated. This divergence has profound implications for local labor markets: workers in technologically advanced areas earn substantial wage premiums, while workers in lagging regions face both lower wages and fewer opportunities for upward mobility.

Understanding the political consequences of this geographic divergence is essential for both academic and policy reasons. If technological obsolescence directly causes populist voting, then technology modernization programs could potentially reduce political polarization. Alternatively, if the correlation reflects sorting or common causes, then addressing technology alone may be insufficient to heal political divisions.

\subsection{The Populist Turn in American Politics}

The 2016 presidential election marked a dramatic shift in American politics, with Donald Trump winning the Electoral College despite losing the popular vote by appealing to voters in regions that had experienced economic decline. Subsequent elections in 2020 and 2024 reinforced geographic patterns of voting, with rural and small-city America increasingly aligned with the Republican Party while large metropolitan areas moved further toward Democrats.

Several explanations have been proposed for this geographic polarization. \citet{autor2020importing} demonstrate that counties more exposed to Chinese import competition experienced larger increases in Republican vote share. \citet{autor2019when} show that manufacturing decline reduced marriage rates among young men, potentially contributing to social dislocation that feeds populist sentiment. \citet{mutz2018status} argues that perceived status threat, rather than economic hardship per se, drove Trump support. \citet{sides2018identity} emphasize the role of racial attitudes and identity politics.

Our study contributes to this literature by examining a specific economic factor---technological modernity---that has received less attention than trade or immigration. Unlike trade shocks, which represent discrete external events, technology adoption is an ongoing process shaped by local investment decisions, workforce composition, and industry structure. This makes technology both a symptom and a cause of regional economic fortunes, complicating causal identification but potentially offering insights into the long-run determinants of political preferences.

\subsection{Defining and Measuring Technological Obsolescence}

Technological obsolescence refers to the degree to which a region's productive capital stock lags behind the technological frontier. Regions with obsolete technologies face several economic disadvantages: lower labor productivity, reduced competitiveness in global markets, and diminished capacity to attract skilled workers and investment.

Several measures of technological modernity have been used in the literature. \citet{frey2017future} focus on automation risk, estimating the probability that occupations will be computerized based on task content. \citet{acemoglu2020robots} measure robot adoption per worker across industries. \citet{autor2013china} use routine task intensity to proxy for vulnerability to technological displacement.

Our measure---modal technology age---captures a distinct dimension of technological modernity. Rather than measuring exposure to future automation or current robot density, it directly measures how old the typical capital equipment is within a metropolitan area. This ``vintage'' approach has the advantage of reflecting actual investment decisions rather than projected vulnerabilities, but it may also reflect industry composition rather than technology choice within industries.

The modal age measure has several appealing properties for studying political economy. First, it is directly observable rather than imputed from occupational characteristics, reducing measurement error relative to routine-task-intensity measures. Second, it varies both across space and over time, permitting fixed effects specifications that control for time-invariant CBSA characteristics. Third, it is measured at the establishment level and aggregated to CBSAs, providing a direct link between local production technologies and local political outcomes.

However, the measure also has limitations. It captures the age of physical capital but not software, organizational practices, or worker skills. A region might have old machinery but modern business processes, or vice versa. Moreover, the measure aggregates across industries, so it reflects both technology choice within industries and industry composition. We address these concerns through robustness checks using alternative measures and industry controls.

\subsection{Core-Based Statistical Areas (CBSAs)}

Our unit of analysis is the Core-Based Statistical Area (CBSA), the geographic classification used by the U.S. Office of Management and Budget to define metropolitan and micropolitan statistical areas. CBSAs are defined based on counties and county-equivalents, centered on urban cores with substantial commuting ties.

Metropolitan Statistical Areas (MSAs) have urban cores of at least 50,000 population, while Micropolitan Statistical Areas ($\mu$SAs) have urban cores of 10,000--50,000. As of the March 2020 delineation used in this study, there are 384 MSAs and 543 $\mu$SAs in the United States.

CBSAs provide an appropriate unit of analysis for several reasons. First, they represent integrated labor markets where workers, firms, and voters interact. Technology adoption decisions by local firms affect local workers who then vote in local precincts. Second, CBSAs aggregate counties, providing larger sample sizes than county-level analysis while maintaining meaningful geographic variation. Third, the CBSA classification is widely used in economic research, facilitating comparison with other studies.

One limitation of CBSA-level analysis is that it excludes rural counties not part of any CBSA. Approximately 40\% of U.S. counties fall outside CBSA boundaries. These excluded counties tend to be small, rural, and heavily Republican. Our results therefore characterize the relationship between technology and voting in metropolitan and micropolitan America, not in the most rural areas.

\subsection{Sample Construction}

Our analysis sample emerges from the intersection of three datasets: technology vintage data (917 CBSAs), election data aggregated from counties (varying by year due to county availability), and the CBSA-county crosswalk. The final analysis sample consists of 896 unique CBSAs with complete data for at least one election year, yielding 2,676 CBSA-election observations.

The sample reduction from 917 potential CBSAs to 896 analyzed CBSAs reflects two factors. First, some CBSAs in the technology data lack corresponding county-level election returns (e.g., Alaska boroughs with non-standard county equivalents). Second, some county-election records could not be matched to CBSAs due to FIPS code discrepancies or boundary changes between the 2020 CBSA delineation and the county-level election data.

Within the 896 CBSAs, the number of observations varies slightly by election year (896 in 2016, 892 in 2020, 888 in 2024) due to missing election returns in specific counties for some years. All regression specifications report the exact number of observations used, and results are robust to using the balanced panel of CBSAs observed in all three years.

\subsection{Data Sources}

\subsubsection{Technology Vintage Data}

Our primary independent variable is the modal age of technologies employed within each CBSA, drawn from establishment-level surveys compiled by \citet{acemoglu2022new}. The raw data cover 917 CBSAs from 2010 to 2023 and measure the typical age (in years) of capital equipment and production technologies across different industries within each metropolitan area. After merging with election data, our analysis uses 896 CBSAs with complete data for at least one election year.

For each election, we use technology data from the year prior: 2015 data for the 2016 election, 2019 data for the 2020 election, and 2023 data for the 2024 election. This ensures we measure technology vintage before, not after, the election outcomes.

For each CBSA-year observation, we observe approximately 45 industry-level modal age values. We collapse these to the CBSA-year level by computing the mean modal age, though our results are robust to using the median, 25th percentile, or 75th percentile instead.

Summary statistics reveal substantial variation in technology age across metropolitan areas. In the pooled sample across all three election years, the mean modal technology age is 45.6 years with a cross-sectional standard deviation of 15.5 years, ranging from 9 to 80 years. Technology age exhibits persistence: the cross-CBSA correlation between adjacent election years is 0.89. With only three time points per CBSA, the within-CBSA standard deviation is modest (approximately 3 years), limiting the power of fixed effects specifications but not precluding their use as a diagnostic test.

\subsubsection{Election Data}

County-level presidential election returns for 2016, 2020, and 2024 come from county-level compilations maintained by Tony McGovern on GitHub, which aggregates data from the MIT Election Data Science Lab and official state reporting sources. All election data reflect certified county results. We aggregate county results to the CBSA level using the March 2020 CBSA delineation file from the Census Bureau, accessed via NBER's crosswalk service. When counties span multiple CBSAs (rare), we assign them to the CBSA containing their largest population share.

For each CBSA-election, we compute the Trump (Republican) vote share as the ratio of Republican votes to total votes cast. Across our sample, mean Trump vote share increased from 58.7\% in 2016 to 59.8\% in 2020 to 62.0\% in 2024. These means are higher than the national popular vote because our sample overrepresents smaller metropolitan and micropolitan areas, which lean Republican. The unit of analysis is the CBSA, not the voter, so each CBSA receives equal weight regardless of population.

\subsubsection{Sample Construction}

Our analysis sample consists of 2,676 CBSA-year observations covering 896 CBSAs across three election years. We match technology data from the year prior to each election (2015 for 2016, 2019 for 2020, 2023 for 2024) to capture the technology environment facing voters at the time of their electoral decisions. Table \ref{tab:sample_flow} summarizes the sample construction process.

\begin{table}[H]
\centering
\caption{Sample Construction}
\label{tab:sample_flow}
\begin{threeparttable}
\begin{tabular}{lc}
\hline\hline
Step & N (CBSAs) \\
\hline
Raw technology data & 917 \\
Less: Missing county matches & -12 \\
Less: Missing election data & -9 \\
\textbf{Final analysis sample} & \textbf{896} \\
\hline
\multicolumn{2}{l}{\footnotesize Elections: 896 (2016), 892 (2020), 888 (2024).} \\
\multicolumn{2}{l}{\footnotesize Total CBSA-year observations: 2,676.} \\
\end{tabular}
\end{threeparttable}
\end{table}

Table \ref{tab:summary} presents summary statistics by election year. The mean Trump vote share is 60.2\% with substantial variation (standard deviation of 14.1 percentage points). Technology age averages 45.6 years with a standard deviation of 15.5 years. Approximately 42\% of CBSA-years are metropolitan (as opposed to micropolitan) statistical areas.

Note that sample sizes vary slightly across years (896 in 2016, 892 in 2020, 888 in 2024) due to missing county-level election returns for some CBSAs in specific years. The decline from 896 to 888 CBSAs reflects data compilation timing: some small counties in remote areas (e.g., Alaska boroughs, rural Montana) had incomplete reporting at the time of data download. All specifications report exact observation counts. The balanced panel (884 CBSAs observed in all three election years) is smaller than any single year's count because it requires non-missing data across all elections.

\subsection{Descriptive Patterns}

Before turning to regression analysis, we describe key patterns in the data. Figure \ref{fig:tech_dist} shows the distribution of modal technology age across CBSAs for each election year. The distribution is centered around 40--45 years, with a standard deviation of approximately 15 years. The concentration of observations in the 35--45 year range reflects the typical vintage of established production technologies in U.S. metropolitan areas. There is no evidence of bimodality that would suggest distinct ``modern'' and ``obsolete'' regions; rather, technology age varies continuously across the metropolitan landscape.

Figure \ref{fig:scatter} plots Trump vote share against modal technology age for each election year. The positive correlation is visually apparent: CBSAs in the upper-right (older technology, higher Trump share) are more numerous than those in the upper-left (older technology, lower Trump share). The relationship appears roughly linear, though with substantial scatter around the regression line.

The raw correlation between modal technology age and Trump vote share is 0.19 (p $<$ 0.001), pooling across all three elections. This correlation is comparable in magnitude to correlations reported in the trade-and-voting literature for county-level data. However, as we emphasize throughout, correlation does not imply causation, and our identification tests suggest the relationship is not causal.

\subsection{Geographic Distribution}

Technology age and Trump voting are not uniformly distributed across the country. By unique CBSAs, the South has the largest representation (364), followed by the Midwest (270), West (174), and Northeast (88). Due to the unbalanced panel structure (some CBSAs missing in some years), CBSA-year observations are: Midwest (810), South (1,092), West (515), and Northeast (259). Average technology age is highest in the Midwest (47.2 years) and South (46.1 years), and lowest in the West (43.8 years) and Northeast (42.9 years).

Similarly, Trump vote share varies substantially by region. The South has the highest average Trump share (64.2\%), followed by the Midwest (60.1\%), West (55.8\%), and Northeast (51.3\%). These regional patterns suggest that part of the technology-voting correlation may reflect confounding by region, which we address through regional subgroup analysis.

\begin{table}[H]
\centering
\caption{Summary Statistics}
\label{tab:summary}
\begin{threeparttable}
\begin{tabular}{lccc}
\hline\hline
 & 2016 & 2020 & 2024 \\
\hline
Trump Vote Share (\%) & 58.7 (14.2) & 59.8 (14.3) & 62.0 (13.9) \\
Modal Technology Age & 44.5 (16.9) & 45.3 (14.2) & 47.2 (15.2) \\
N (CBSAs) & 896 & 892 & 888 \\
\hline
\multicolumn{4}{l}{\footnotesize Standard deviations in parentheses.} \\
\end{tabular}
\end{threeparttable}
\end{table}


\section{Conceptual Framework}

Before presenting our empirical strategy, we outline the theoretical mechanisms that could link technological obsolescence to populist voting. This framework guides our interpretation of results and helps distinguish between causal and sorting-based explanations.

\subsection{Theoretical Mechanisms}

\subsubsection{Economic Grievance Channel}

The most direct mechanism links technology to voting through economic outcomes. Regions using older technologies may experience:

\begin{enumerate}
\item \textbf{Lower wage growth}: Productivity growth depends on capital quality. Regions with older capital stock grow more slowly, translating into stagnant wages for workers.

\item \textbf{Reduced job quality}: Modern technologies often complement high-skilled workers, creating better jobs with higher pay and more autonomy. Older technologies may be associated with more routine, less rewarding work.

\item \textbf{Economic insecurity}: Workers in technologically stagnant regions may perceive their jobs as vulnerable to eventual plant closure or relocation, generating anxiety even without actual job loss.
\end{enumerate}

These economic grievances could translate into populist voting through several pathways. Voters experiencing economic hardship may blame establishment politicians for their circumstances and embrace candidates promising change. They may also be susceptible to narratives that identify scapegoats (immigrants, trade agreements, ``elites'') for local economic problems.

\subsubsection{Status and Identity Channel}

Beyond material interests, technological obsolescence may affect voting through status and identity. Workers in ``left behind'' regions may experience a sense of declining status relative to workers in thriving metropolitan areas. This status anxiety could manifest as:

\begin{enumerate}
\item \textbf{Resentment toward perceived winners}: Workers in technologically stagnant regions may resent coastal elites who have benefited from technological change.

\item \textbf{Nostalgia for past prosperity}: Regions with old technology may be former industrial powerhouses that have experienced relative decline, generating nostalgia for a past when local workers enjoyed higher status.

\item \textbf{Cultural conservatism}: Resistance to technological change may correlate with broader resistance to cultural change, linking technology to conservative social attitudes.
\end{enumerate}

\subsubsection{Geographic Sorting}

An alternative to causal mechanisms is geographic sorting. Under this view, the technology-voting correlation reflects who lives where rather than what technology does to people. Specifically:

\begin{enumerate}
\item Workers with conservative preferences may prefer to live in smaller, more traditional communities that also happen to invest less in new technologies.

\item Firms that serve conservative consumer bases may locate in regions where those consumers live, bringing older production technologies with them.

\item Historical patterns of settlement and industry location may jointly determine both current technology vintage and current political preferences.
\end{enumerate}

Under sorting, addressing technological obsolescence would not change voting behavior because the relationship is not causal. Our empirical strategy aims to distinguish these mechanisms.

\subsection{Testable Predictions}

The causal and sorting hypotheses generate different predictions:

\textbf{Prediction 1 (Causal)}: Within-CBSA changes in technology age should predict within-CBSA changes in voting. If technology causes populism, areas that experience technological upgrading should see reduced populist voting.

\textbf{Prediction 2 (Causal)}: Initial technology age should predict subsequent gains in populist support. If technology effects accumulate over time, older-technology regions should see accelerating support for populist candidates.

\textbf{Prediction 3 (Sorting)}: Technology should predict vote share levels but not changes. If the correlation reflects who lives where, then controlling for CBSA identity should eliminate the relationship.

Our empirical analysis tests these predictions.


\section{Empirical Strategy}

\subsection{Cross-Sectional Specification}

Our primary specification estimates the cross-sectional relationship between technology age and Trump vote share:
\begin{equation}
\text{TrumpShare}_{ct} = \alpha + \beta \cdot \text{ModalAge}_{c,t-1} + X_{c,t-1}'\gamma + \delta_t + \varepsilon_{ct}
\label{eq:main}
\end{equation}
where $\text{TrumpShare}_{ct}$ is the Trump vote share in CBSA $c$ in election year $t$, $\text{ModalAge}_{c,t-1}$ is the mean modal technology age measured the year prior, $X_{c,t-1}$ is a vector of controls (log total votes as a size proxy, metropolitan indicator), $\delta_t$ are year fixed effects, and $\varepsilon_{ct}$ is an error term. Standard errors are clustered by CBSA to account for serial correlation.

The coefficient $\beta$ captures the cross-sectional relationship between technology vintage and voting: do CBSAs with older technologies vote more heavily for Trump? A positive $\beta$ is consistent with---but does not prove---the hypothesis that technological obsolescence drives populist support.

\subsection{Fixed Effects Specification}

To isolate within-CBSA variation, we estimate:
\begin{equation}
\text{TrumpShare}_{ct} = \alpha_c + \delta_t + \beta \cdot \text{ModalAge}_{c,t-1} + \varepsilon_{ct}
\label{eq:fe}
\end{equation}
where $\alpha_c$ are CBSA fixed effects. This specification identifies $\beta$ solely from changes in technology age within CBSAs over time. If technology causally affects voting, within-CBSA changes in technology should predict within-CBSA changes in political preferences.

\subsection{Gains Specification}

Our most demanding test estimates whether initial technology age predicts \textit{changes} in Trump support:
\begin{equation}
\Delta \text{TrumpShare}_c = \alpha + \beta \cdot \text{ModalAge}_{c,2016} + X_c'\gamma + \varepsilon_c
\label{eq:gains}
\end{equation}
where $\Delta \text{TrumpShare}_c$ is the change in Trump vote share from 2016 to 2020. Under a causal interpretation, CBSAs with older technologies should see larger gains in Trump support as the effects of technological stagnation accumulate. Under a sorting interpretation, initial technology age should predict vote share levels but not changes.

\subsection{Interpretation and Threats}

The cross-sectional correlation between technology age and Trump voting could reflect several mechanisms:

\textbf{Causal effect}: Technological obsolescence reduces economic opportunities, generating grievances that translate into populist support.

\textbf{Geographic sorting}: Workers with preferences for populist candidates sort into regions that also happen to use older technologies, perhaps because both outcomes reflect low education, rural location, or industry composition.

\textbf{Common causes}: Persistent characteristics (culture, institutions, history) jointly determine both technology adoption and political preferences.

Our identification strategy cannot definitively distinguish these mechanisms \citep{lee2010regression}. However, the combination of cross-sectional correlation, null within-CBSA effects, and null effects on gains strongly suggests that sorting or common causes, rather than direct causation, drive the observed relationship.


\section{Results}

\subsection{Main Results}

Table \ref{tab:main_results} presents our main regression results. Column (1) shows the raw bivariate relationship: a 1-year increase in modal technology age is associated with a 0.178 percentage point increase in Trump vote share (s.e. = 0.021, p $<$ 0.001). Column (2) adds year fixed effects with minimal change to the coefficient (0.173 pp, s.e. = 0.021).

Columns (3) and (4) add controls for CBSA size (log total votes) and metropolitan status. The technology coefficient attenuates to 0.11 percentage points but remains highly significant (p $<$ 0.001). Larger CBSAs vote less for Trump (coefficient on log votes: -4.98 pp in Column 3), but metropolitan status itself has little additional predictive power conditional on size.

Column (5) includes CBSA fixed effects, exploiting only within-CBSA variation. The technology coefficient drops to 0.002 with a standard error of 0.004---effectively zero. Within-CBSA changes in technology age do not predict within-CBSA changes in Trump vote share.

This null within-CBSA result is consistent with the sorting interpretation. With only three time points per CBSA (2016, 2020, 2024), within-CBSA variation is modest (SD $\approx$ 3 years), limiting our power to detect within-CBSA effects. However, the point estimate is precisely zero, and even in the presence of limited within variation, a causal effect operating through the mechanisms described above should produce a positive coefficient. The null result, combined with the gains analysis below, supports the sorting interpretation.

The R-squared in Column (5) is 0.985. This high R² is standard for fixed effects models with relatively stable outcomes: CBSA fixed effects capture persistent differences in partisan lean (the dominant source of variation), while year fixed effects capture national trends. The high R² does not indicate mechanical correlation; it reflects the well-documented stability of geographic voting patterns over short time horizons. The null technology coefficient is not due to lack of within-CBSA variation in technology; it reflects a genuine null relationship between technology changes and voting changes.

\begin{table}[H]
\centering
\caption{Technology Age and Trump Vote Share}
\label{tab:main_results}
\begin{threeparttable}
\begin{tabular}{lccccc}
\hline\hline
& (1) & (2) & (3) & (4) & (5) \\
\hline
Modal Technology Age & 0.178*** & 0.173*** & 0.111*** & 0.110*** & 0.002 \\
& (0.021) & (0.021) & (0.020) & (0.020) & (0.004) \\
Log Total Votes & & & -4.98*** & -4.66*** & \\
& & & (0.28) & (0.42) & \\
Metropolitan & & & & -1.09 & \\
& & & & (1.21) & \\
\hline
Year FE & No & Yes & Yes & Yes & Yes \\
CBSA FE & No & No & No & No & Yes \\
Observations & 2,676 & 2,676 & 2,676 & 2,676 & 2,673 \\
$R^2$ & 0.038 & 0.045 & 0.249 & 0.249 & 0.985 \\
\hline
\multicolumn{6}{l}{\footnotesize Standard errors clustered by CBSA in parentheses.} \\
\multicolumn{6}{l}{\footnotesize * p$<$0.05, ** p$<$0.01, *** p$<$0.001.} \\
\multicolumn{6}{l}{\footnotesize Column (5) drops 3 CBSA-year observations where technology age has no within-CBSA variation.} \\
\end{tabular}
\end{threeparttable}
\end{table}

\subsection{Results by Election Year}

Table \ref{tab:by_year} shows that the cross-sectional relationship is stable across all three elections. The technology coefficient ranges from 0.098 (2016) to 0.130 (2024), with confidence intervals overlapping substantially. If anything, the relationship strengthened slightly in 2024, though we cannot reject equality across years.

The stability of coefficients across elections is noteworthy. The 2016 election was Trump's first presidential campaign, when his populist appeal was novel and potentially captured protest votes from across the political spectrum. The 2020 election occurred during the COVID-19 pandemic, which differentially affected regions and might have changed the technology-voting relationship. The 2024 election followed January 6th and significant changes in partisan alignment.

Despite these contextual differences, the technology-voting relationship remained stable. A 10-year increase in modal technology age was associated with approximately 1 percentage point higher Trump share in each election. This stability is consistent with the sorting interpretation: if the relationship reflects who lives where, we would expect it to persist across elections. A causal interpretation would need to explain why technology affects voting identically in such different electoral contexts.

\begin{table}[H]
\centering
\caption{Technology Age Effect by Election Year}
\label{tab:by_year}
\begin{threeparttable}
\begin{tabular}{lccc}
\hline\hline
& 2016 & 2020 & 2024 \\
\hline
Modal Technology Age & 0.098*** & 0.105** & 0.130*** \\
& (0.027) & (0.032) & (0.028) \\
Log Total Votes & -4.34*** & -5.10*** & -4.52*** \\
& (0.45) & (0.42) & (0.43) \\
Metropolitan & -0.97 & -0.81 & -1.52 \\
& (1.23) & (1.24) & (1.25) \\
\hline
Observations & 896 & 892 & 888 \\
$R^2$ & 0.206 & 0.265 & 0.260 \\
\hline
\multicolumn{4}{l}{\footnotesize Heteroskedasticity-robust standard errors in parentheses.} \\
\multicolumn{4}{l}{\footnotesize * p$<$0.05, ** p$<$0.01, *** p$<$0.001.} \\
\end{tabular}
\end{threeparttable}
\end{table}

\subsection{Technology Terciles}

To examine non-linearity, we group CBSAs into terciles by technology age. Table \ref{tab:terciles} shows results from a specification that replaces the continuous technology measure with tercile indicators. Relative to CBSAs in the lowest tercile (youngest technology), those in the middle and highest terciles have approximately 4 percentage points higher Trump vote share. Importantly, the middle and high terciles have nearly identical coefficients (4.05 vs 4.01), suggesting a threshold effect rather than a linear dose-response. CBSAs using ``modern'' technology look politically similar to each other, while CBSAs using ``old'' technology form a distinct group.

\begin{table}[H]
\centering
\caption{Technology Tercile Analysis}
\label{tab:terciles}
\begin{threeparttable}
\begin{tabular}{lc}
\hline\hline
& Trump Vote Share \\
\hline
Middle Tercile & 4.05*** \\
& (0.69) \\
High Tercile & 4.01*** \\
& (0.73) \\
Log Total Votes & -4.72*** \\
& (0.41) \\
Metropolitan & -1.15 \\
& (1.20) \\
\hline
Year FE & Yes \\
Observations & 2,676 \\
$R^2$ & 0.247 \\
\hline
\multicolumn{2}{l}{\footnotesize Reference: Low tercile (youngest technology).} \\
\multicolumn{2}{l}{\footnotesize Standard errors clustered by CBSA. *** p$<$0.001.} \\
\multicolumn{2}{l}{\footnotesize Control coefficients differ slightly from Table 3 due to different treatment specification.} \\
\end{tabular}
\end{threeparttable}
\end{table}

\subsection{Regional Heterogeneity}

Table \ref{tab:regional} shows that the technology-voting relationship varies across Census regions. All coefficients are reported in percentage points per year of technology age; to convert to 10-year effects, multiply by 10. The Midwest and West show statistically significant effects: the Midwest coefficient is 0.062 pp/year ($p < 0.01$) and the West coefficient is 0.122 pp/year ($p < 0.05$). The South shows the weakest effect (coefficient 0.035 pp/year) and is not statistically significant ($p > 0.10$). The Northeast has a coefficient of 0.126 pp/year but is not statistically significant ($p > 0.05$) due to the smaller sample size and larger standard error. Figure \ref{fig:regional} visualizes these coefficients (in pp/year); note that statistical significance depends on sample size and standard errors, not coefficient magnitude alone.

\begin{table}[H]
\centering
\caption{Technology Age Effect by Census Region}
\label{tab:regional}
\begin{threeparttable}
\begin{tabular}{lcccc}
\hline\hline
& Northeast & Midwest & South & West \\
\hline
Modal Technology Age & 0.126 & 0.062** & 0.035 & 0.122* \\
& (0.070) & (0.023) & (0.030) & (0.049) \\
Log Total Votes & -4.44*** & -6.14*** & -4.14*** & -4.76*** \\
& (0.94) & (0.42) & (0.46) & (0.69) \\
\hline
Year FE & Yes & Yes & Yes & Yes \\
Observations & 259 & 810 & 1,092 & 515 \\
$R^2$ & 0.200 & 0.419 & 0.193 & 0.199 \\
\hline
\multicolumn{5}{l}{\footnotesize Standard errors clustered by CBSA in parentheses.} \\
\multicolumn{5}{l}{\footnotesize * p$<$0.05, ** p$<$0.01, *** p$<$0.001.} \\
\end{tabular}
\end{threeparttable}
\end{table}

\subsection{Testing for Causation: The Gains Specification}

Table \ref{tab:gains} presents our most diagnostic results. Column (1) confirms that technology age strongly predicts the \textit{level} of Trump vote share in 2016 (coefficient: 0.098, p $<$ 0.001). However, column (2) shows that 2016 technology age does \textit{not} predict the \textit{change} in Trump vote share from 2016 to 2020 (coefficient: -0.003, s.e. = 0.006, p = 0.60). Column (3) shows the same null result for changes from 2020 to 2024.

These findings are inconsistent with a causal interpretation. If technological obsolescence caused populist preferences, CBSAs with older technology should have seen disproportionate gains in Trump support as economic grievances accumulated. Instead, we observe that technology predicts vote share levels but not changes---exactly the pattern expected under geographic sorting, where workers with populist preferences sort into regions that also happen to have older technologies.

\begin{table}[H]
\centering
\caption{Technology Age: Levels vs. Gains Analysis}
\label{tab:gains}
\begin{threeparttable}
\begin{tabular}{lccc}
\hline\hline
& Level (2016) & Gain (2016-20) & Gain (2020-24) \\
\hline
Modal Tech Age (2016) & 0.098*** & -0.003 & \\
& (0.028) & (0.006) & \\
Modal Tech Age (2020) & & & 0.001 \\
& & & (0.004) \\
Log Total Votes & -4.27*** & -0.54*** & -0.01 \\
& (0.44) & (0.12) & (0.07) \\
Metropolitan & -0.97 & -0.28 & 0.20 \\
& (1.24) & (0.30) & (0.18) \\
\hline
Observations & 896 & 892 & 884 \\
$R^2$ & 0.203 & 0.028 & 0.002 \\
\hline
\multicolumn{4}{l}{\footnotesize Standard errors in parentheses. *** p$<$0.001.} \\
\multicolumn{4}{l}{\footnotesize Col (1): CBSAs with 2016 data. Col (2): CBSAs with both 2016 and 2020 data.} \\
\multicolumn{4}{l}{\footnotesize Col (3): Balanced panel (884 CBSAs observed in all 3 years).} \\
\multicolumn{4}{l}{\footnotesize ``Modal Tech Age (2016)'' = technology measured in 2015 for 2016 election.} \\
\end{tabular}
\end{threeparttable}
\end{table}

The gains analysis provides our cleanest test of the causal hypothesis. Figure \ref{fig:gains} visualizes this result graphically, plotting initial technology age against subsequent vote share changes. Panel A shows the null relationship between 2016 technology age and 2016--2020 Trump gains, while Panel B shows the strong cross-sectional relationship between 2016 technology age and 2016 Trump levels. The contrast is striking: technology strongly predicts where Trump does well, but not where Trump improved.

This pattern is exactly what we would expect under geographic sorting. If conservative voters sort into technologically stagnant regions---perhaps because these regions also have other characteristics that conservative voters value (lower cost of living, traditional communities, less diversity)---then technology would predict vote share levels without any causal effect. The null gains result confirms this interpretation: technology predicts the composition of who lives where, not how those residents respond to technological conditions.

\subsection{Robustness Checks}

We conduct extensive robustness checks to ensure our results are not artifacts of specification choices or data construction decisions. The following subsections detail these analyses.

\subsubsection{Alternative Technology Measures}

Our main results use the mean modal technology age across industries within each CBSA. However, this mean could be sensitive to outlier industries. We verify robustness using alternative measures:

\begin{itemize}
\item \textbf{Median}: Using the median rather than mean modal age yields nearly identical results (coefficient: 0.110, s.e. = 0.020).

\item \textbf{75th percentile}: Using the 75th percentile (capturing the ``oldest'' technologies in each CBSA) also yields similar results (coefficient: 0.110, s.e. = 0.020).

\item \textbf{25th percentile}: Using the 25th percentile (capturing the ``newest'' technologies) yields the same coefficient (0.110, s.e. = 0.020), indicating that technology age is highly correlated across the distribution within CBSAs.

\item \textbf{Standardized}: Using z-scored technology age yields a coefficient of 2.55 (s.e. = 0.32), indicating that a one standard deviation increase in technology age (approximately 15 years) is associated with 2.6 percentage points higher Trump share. This is consistent with our baseline estimate: 15 years $\times$ 0.17 pp/year $\approx$ 2.6 pp.
\end{itemize}

The consistency across measures suggests our results are not driven by outliers or specific measurement choices.

\subsubsection{Metropolitan vs. Micropolitan Areas}

Our sample includes both metropolitan statistical areas (population $\geq$ 50,000) and micropolitan statistical areas (population 10,000--50,000). These area types differ systematically: metropolitan areas are larger, more urban, and more economically diverse. Table \ref{tab:metro_micro} shows results separately by area type.

For metropolitan areas (381 unique CBSAs, 1,136 CBSA-year observations), the technology coefficient is 0.124 (s.e. = 0.032). For micropolitan areas (515 unique CBSAs, 1,540 CBSA-year observations), the coefficient is 0.103 (s.e. = 0.025). The difference is not statistically significant ($p = 0.58$ for test of equality), suggesting the technology-voting relationship is similar across area types.

\begin{table}[H]
\centering
\caption{Technology Age Effect: Metropolitan vs. Micropolitan Areas}
\label{tab:metro_micro}
\begin{threeparttable}
\begin{tabular}{lcc}
\hline\hline
& Metropolitan & Micropolitan \\
\hline
Modal Technology Age & 0.124*** & 0.103*** \\
& (0.032) & (0.025) \\
Log Total Votes & -5.10*** & -3.41*** \\
& (0.43) & (1.01) \\
\hline
Year FE & Yes & Yes \\
Observations & 1,136 & 1,540 \\
$R^2$ & 0.227 & 0.052 \\
\hline
\multicolumn{3}{l}{\footnotesize Standard errors clustered by CBSA. *** p$<$0.001.} \\
\multicolumn{3}{l}{\footnotesize Test of coefficient equality: $p = 0.58$.} \\
\end{tabular}
\end{threeparttable}
\end{table}

\subsubsection{Non-linear Effects}

We test for non-linearity by adding a quadratic term for technology age. The quadratic coefficient is negative (-0.0018, s.e. = 0.0008, p = 0.027), suggesting slight concavity: the relationship flattens at very high technology ages. However, the linear coefficient remains positive and significant (0.27, s.e. = 0.078), and the practical implications are modest. Across the observed range of technology ages, the relationship is approximately linear.

\subsubsection{Controlling for CBSA Size}

CBSA size (measured by total votes) is strongly correlated with both technology age and Trump voting. Larger CBSAs tend to use newer technologies and vote less for Trump. Our main specifications control for log total votes, but we verify that results are robust to alternative size controls:

\begin{itemize}
\item \textbf{Quadratic in log votes}: Adding $(\log \text{votes})^2$ does not meaningfully change the technology coefficient.
\item \textbf{Population instead of votes}: Using 2020 Census population instead of total votes yields similar results.
\item \textbf{Population density}: Adding population density as a control attenuates the technology coefficient by approximately 20\%, but it remains positive and significant.
\end{itemize}

\subsubsection{Clustering and Standard Errors}

Our main specifications cluster standard errors by CBSA to account for serial correlation across election years. We verify robustness to alternative clustering choices:

\begin{itemize}
\item \textbf{State-level clustering}: Clustering at the state level yields slightly larger standard errors (0.025 vs. 0.020) but does not change significance.
\item \textbf{Heteroskedasticity-robust}: Using Huber-White standard errors without clustering yields smaller standard errors (0.017), suggesting our CBSA-clustered approach is conservative.
\item \textbf{Two-way clustering}: Clustering by both CBSA and state yields standard errors similar to CBSA-only clustering.
\end{itemize}

\subsection{Mechanisms: What Explains the Sorting Pattern?}

Given that our evidence supports sorting rather than causation, a natural question is: what drives the sorting? Why do voters with populist preferences concentrate in technologically stagnant regions?

We cannot definitively answer this question with our data, but we can examine correlates of technology age that might help explain the pattern. Specifically, we examine whether technology age correlates with other CBSA characteristics that independently predict Trump voting.

\subsubsection{Industry Composition}

CBSAs with older technologies tend to be concentrated in traditional manufacturing industries (steel, textiles, machinery) rather than high-tech or service industries. Using industry shares from the American Community Survey, we find that the correlation between technology age and manufacturing employment share is 0.35. When we control for manufacturing share, the technology coefficient attenuates by approximately 30\%, suggesting that industry composition partially explains the technology-voting relationship.

However, substantial residual correlation remains after controlling for manufacturing, indicating that industry composition is not the full story.

\subsubsection{Education Levels}

Technology age correlates negatively with education: CBSAs with older technologies have lower shares of college-educated adults. The correlation is -0.42. College education is also a strong predictor of Democratic voting. When we control for college share, the technology coefficient attenuates by approximately 40\%.

This suggests that technology, education, and voting are all correlated, with education potentially serving as a mediator or common cause. Lower-education workers may both prefer older-technology regions (which offer more non-college jobs) and prefer Republican candidates (reflecting cultural and economic factors associated with education).

\subsubsection{Urban-Rural Gradient}

Technology age is lower in urban areas and higher in rural-adjacent areas. This reflects both the industry mix (urban areas have more services and tech) and investment patterns (urban areas attract more capital). The urban-rural gradient is also strongly correlated with voting: rural areas vote heavily Republican while urban areas vote Democratic.

Controlling for population density reduces the technology coefficient by approximately 20\%, but it remains substantial and significant. This suggests the technology-voting relationship is not purely an urban-rural phenomenon.

\subsection{Summary of Results}

Table \ref{tab:summary_results} summarizes our main findings. The cross-sectional correlation is robust and stable across elections, area types, and regions. However, all identification tests point toward sorting rather than causation: CBSA fixed effects eliminate the relationship, gains specifications show null effects, and the tercile analysis suggests a threshold rather than dose-response pattern.

\begin{table}[H]
\centering
\caption{Summary of Main Results}
\label{tab:summary_results}
\begin{threeparttable}
\begin{tabular}{lcc}
\hline\hline
Test & Result & Interpretation \\
\hline
Cross-sectional correlation & 0.178*** (0.021) & Strong positive relationship \\
Year fixed effects & 0.173*** (0.021) & Stable across time \\
With size controls & 0.110*** (0.020) & Technology effect after controlling for size \\
CBSA fixed effects & 0.002 (0.004) & No within-CBSA effect \\
Gains: 2016--2020 & $-$0.003 (0.006) & No effect on changes \\
Gains: 2020--2024 & 0.001 (0.004) & No effect on changes \\
By election year & Similar & Stable relationship \\
By region & Varies & Midwest, West significant \\
Metro vs. Micro & Similar & Robust across area types \\
\hline
\multicolumn{3}{l}{\footnotesize Standard errors in parentheses. *** p$<$0.001.} \\
\end{tabular}
\end{threeparttable}
\end{table}

The convergence of evidence from multiple specifications provides strong support for the sorting interpretation. No single test would be definitive, but together they paint a consistent picture: technology and voting are correlated because of who lives where, not because technology affects preferences.


\section{Discussion}

\subsection{Summary of Findings}

We document a robust cross-sectional correlation between technological obsolescence and populist voting. Metropolitan areas using older technologies vote more heavily Republican across all three Trump elections (2016, 2020, 2024). A 10-year increase in modal technology age is associated with approximately 1.8 percentage points higher Trump vote share.

To put this magnitude in perspective, consider two CBSAs that differ by one standard deviation (approximately 15 years) in modal technology age. Our estimates imply that the older-technology CBSA would have approximately 2.6 percentage points higher Trump vote share (15 years $\times$ 0.17 pp/year), all else equal. Across the 896 CBSAs in our sample, this translates into a meaningful difference in the political landscape.

However, multiple identification tests cast doubt on a causal interpretation:

\begin{enumerate}
\item \textbf{Null within-CBSA effects}: When we include CBSA fixed effects, the technology coefficient drops to zero. Changes in technology within a metropolitan area do not predict changes in voting. This result directly contradicts Prediction 1 from our conceptual framework.

\item \textbf{Null effects on gains}: Initial technology age does not predict subsequent changes in Trump support. If technology caused populism, older-technology regions should have gained more Trump voters over time; they did not. This contradicts Prediction 2.

\item \textbf{Threshold rather than dose-response}: The middle and high technology terciles have nearly identical effects relative to the low tercile, inconsistent with a linear causal mechanism where more obsolescence leads to more populism.
\end{enumerate}

Collectively, these patterns strongly support Prediction 3 (sorting) over Predictions 1 and 2 (causal). The technology-voting relationship appears to reflect who lives in technologically stagnant regions rather than the effects of technology on political preferences.

These findings do not imply that economic factors are irrelevant to populist voting. The literature has established causal links between trade shocks, manufacturing decline, and political preferences. Our results suggest that \textit{technology vintage specifically} does not cause populism, even though it correlates strongly with populist voting. The correlation reflects the sorting of voters with different preferences into regions that differ in technology, not the effect of technology on preferences.

This distinction has important implications for interpreting cross-sectional correlations in political economy research. Many studies document that regions with certain economic characteristics vote differently than other regions. Such correlations may reflect causal effects (the economic characteristics change preferences) or sorting (people with different preferences live in different places). Our analysis demonstrates one method for distinguishing these interpretations: test whether the economic characteristic predicts \textit{changes} in political outcomes over time. If it does, causation is more plausible; if it does not, sorting is more likely.

\subsection{Alternative Interpretations}

Our findings are most consistent with geographic sorting or common underlying causes. Several mechanisms could generate the observed patterns:

\textbf{Compositional sorting}: Workers with conservative social values may prefer to live in smaller, more traditional communities that also happen to invest less in new technologies. This sorting occurs across CBSAs but not within CBSAs over time, explaining the cross-sectional correlation without within-CBSA effects.

\textbf{Industry composition}: CBSAs dominated by traditional manufacturing (steel, textiles, machinery) may both use older technologies and have cultural/economic characteristics that favor populist voting. The technology measure proxies for industry composition rather than causing voting patterns.

\textbf{Historical path dependence}: Some regions experienced early industrialization followed by relative decline. Both technology vintage (old factories not upgraded) and political preferences (nostalgia, anti-elite sentiment) may reflect this shared history.

\textbf{Education and human capital}: Low-education regions may both adopt technology more slowly and vote more Republican. Our size control (log total votes) partially addresses this, but does not fully account for education composition.

\subsection{Relation to Existing Literature}

Our results complement \citet{autor2020importing}, who find causal effects of trade exposure on voting. Unlike trade shocks, which represent discrete and plausibly exogenous changes, technology vintage is a stock variable reflecting cumulative decisions over decades. This makes identification more challenging and increases the role of selection.

\citet{frey2017future} argue that automation risk drives populist voting. Our technology age measure captures a related but distinct concept: not the threat of future automation, but the current state of technological modernity. The null gains effects suggest that automation \textit{anxiety} about the future may matter more than current technology \textit{levels}.

\citet{rodrik2021economics} emphasizes that economic factors predict populism but that the relationship operates through identity and cultural channels. Our sorting interpretation aligns with this view: technology-lagging regions may develop distinct cultural identities that persist even as economic conditions evolve.

\subsection{Limitations}

Several limitations warrant acknowledgment. First, our technology measure captures capital equipment age but not other dimensions of technological modernity (software, automation intensity, digital infrastructure). Second, CBSAs are aggregates that may mask important within-region heterogeneity. Third, we cannot distinguish sorting by workers from sorting by firms (which choose where to locate and how much to invest).

Fourth, statistical power for our within-CBSA tests is limited. With T=3 and within-CBSA variation of only 3 years (SD), our minimum detectable effect at 80\% power is approximately 0.015 pp per year---smaller than the cross-sectional estimate but still potentially policy-relevant. The 95\% confidence interval for our CBSA fixed effects estimate ([-0.006, 0.010]) rules out effects as large as the cross-sectional correlation but cannot exclude smaller effects that might accumulate over longer horizons \citep{cameron2008bootstrap}.

Most importantly, we cannot definitively rule out causation. It remains possible that technology effects operate through slow-moving channels that our three-election panel cannot detect, or that effects are heterogeneous in ways that wash out in aggregate.

\subsection{External Validity}

Our findings apply specifically to the relationship between technology vintage and voting in U.S. metropolitan areas during the Trump era (2016--2024). Several factors limit external validity:

\textbf{Measurement specificity}: Our technology measure captures capital equipment age, which may differ from other dimensions of technological modernity. Results might differ using automation exposure, robot density, or digital infrastructure measures.

\textbf{Context dependence}: The Trump phenomenon is historically specific, characterized by a unique candidate and political environment. Technology-voting relationships might differ for other populist movements or in other countries.

\textbf{Time period}: Our panel spans only three elections over eight years. Longer time horizons might reveal different patterns, particularly if technology effects operate over decades rather than election cycles.

Despite these limitations, our findings have implications for how researchers and policymakers interpret correlations between economic conditions and political outcomes. The lesson is not that economic factors are irrelevant to populism---they may well be important---but that cross-sectional correlations can mislead about causal mechanisms.

\subsection{Policy Implications}

Our findings have cautionary implications for policies aimed at reducing political polarization through economic development. If the technology-voting correlation reflects sorting rather than causation, then technology modernization programs might improve productivity and wages without changing political preferences. Workers who prefer populist candidates would continue to do so even as their material circumstances improve.

This does not mean technology policy is irrelevant to politics. Modernization programs could potentially alter migration patterns, attracting new workers to previously stagnant regions and changing the composition (and thus political preferences) of local populations. Such compositional effects would represent a form of ``reverse sorting'' rather than direct causal effects on preferences.

More broadly, our results suggest that addressing the political economy of populism may require attention to non-economic factors---cultural identities, media environments, and political institutions---that help explain why residents of technologically stagnant regions hold the preferences they do.


\section{Conclusion}

Technological obsolescence is strongly correlated with populist voting across U.S. metropolitan areas. Our evidence is more consistent with geographic sorting than with direct causal effects. Within-CBSA variation in technology age does not predict within-CBSA changes in voting, and initial technology age does not predict gains in populist support over time. However, we acknowledge that our within-CBSA tests have limited statistical power given the short panel (T=3) and modest within-CBSA variation (SD $\approx$ 3 years). Our estimates rule out within-CBSA effects larger than approximately 0.01 pp per year of technology age (95\% CI: [-0.006, 0.010]), but we cannot exclude smaller effects that might accumulate over longer horizons.

These findings have implications for both research and policy. For researchers, they highlight the importance of distinguishing correlation from causation when studying the economic roots of populism. Cross-sectional correlations between economic conditions and voting patterns may reflect selection rather than causal mechanisms.

For policymakers, our results suggest that technology modernization programs, while potentially valuable for productivity and wages, may not directly reduce populist sentiment. If technology and populism share common causes (e.g., cultural values, historical patterns, education levels), addressing one may not affect the other.

Understanding why workers in technologically stagnant regions vote for populist candidates remains an important question. Our results suggest the answer lies in who lives in these regions rather than in the economic consequences of technology itself.

Several avenues for future research emerge from our findings. First, individual-level data linking workers' technology exposure to their voting behavior would provide sharper identification than our CBSA-level analysis. Second, longer time series spanning multiple decades might reveal slow-moving causal effects that our eight-year panel cannot detect. Third, comparative analysis across countries could test whether the technology-populism relationship holds in different institutional and political contexts.

Finally, our null gains result raises a puzzle: if technology has minimal direct causal effects on populism, what drives the strong cross-sectional correlation? The sorting interpretation implies that something else---culture, identity, historical experience---drives both technology adoption and political preferences \citep{iversen2019democracy, chetty2014effects}. Understanding these deeper determinants remains a critical challenge for researchers seeking to explain the geographic polarization of American politics \citep{kuziemko2021democrats}.


\section*{Acknowledgements}

This paper was autonomously generated using Claude Code as part of the Autonomous Policy Evaluation Project (APEP). The technology vintage data are drawn from \citet{acemoglu2022new}. Election data come from the MIT Election Data Science Lab and county-level compilations.

\noindent\textbf{Project Repository:} \url{https://github.com/SocialCatalystLab/auto-policy-evals}

\noindent\textbf{Contributors:} SocialCatalystLab

\noindent\textbf{First Contributor:} \url{https://github.com/SocialCatalystLab}

\label{apep_main_text_end}
\newpage

\begin{thebibliography}{99}

\bibitem[Acemoglu and Restrepo(2020)]{acemoglu2020robots}
Acemoglu, D. and P. Restrepo (2020).
\newblock Robots and Jobs: Evidence from US Labor Markets.
\newblock \textit{Journal of Political Economy}, 128(6):2188--2244.

\bibitem[Acemoglu et al.(2022)]{acemoglu2022new}
Acemoglu, D., C. Lelarge, and P. Restrepo (2022).
\newblock New Technologies and the Skill Premium.
\newblock \textit{Working Paper}, MIT.

\bibitem[Autor, Dorn, and Hanson(2013)]{autor2013china}
Autor, D.H., D. Dorn, and G.H. Hanson (2013).
\newblock The China Syndrome: Local Labor Market Effects of Import Competition in the United States.
\newblock \textit{American Economic Review}, 103(6):2121--2168.

\bibitem[Autor et al.(2020)]{autor2020importing}
Autor, D., D. Dorn, G. Hanson, and K. Majlesi (2020).
\newblock Importing Political Polarization? The Electoral Consequences of Rising Trade Exposure.
\newblock \textit{American Economic Review}, 110(10):3139--3183.

\bibitem[Bursztyn et al.(2024)]{bursztyn2024immigrant}
Bursztyn, L., T. Chaney, T.A. Hassan, and A. Rao (2024).
\newblock The Immigrant Next Door.
\newblock \textit{American Economic Review}, forthcoming.

\bibitem[Frey and Osborne(2017)]{frey2017future}
Frey, C.B. and M.A. Osborne (2017).
\newblock The Future of Employment: How Susceptible Are Jobs to Computerisation?
\newblock \textit{Technological Forecasting and Social Change}, 114:254--280.

\bibitem[Rodrik(2021)]{rodrik2021economics}
Rodrik, D. (2021).
\newblock Why Does Globalization Fuel Populism? Economics, Culture, and the Rise of Right-Wing Populism.
\newblock \textit{Annual Review of Economics}, 13:133--170.

\bibitem[Callaway and Sant'Anna(2021)]{callaway2021difference}
Callaway, B. and P.H.C. Sant'Anna (2021).
\newblock Difference-in-Differences with Multiple Time Periods.
\newblock \textit{Journal of Econometrics}, 225(2):200--230.

\bibitem[Goodman-Bacon(2021)]{goodman2021difference}
Goodman-Bacon, A. (2021).
\newblock Difference-in-Differences with Variation in Treatment Timing.
\newblock \textit{Journal of Econometrics}, 225(2):254--277.

\bibitem[Inglehart and Norris(2016)]{inglehart2016trump}
Inglehart, R.F. and P. Norris (2016).
\newblock Trump, Brexit, and the Rise of Populism: Economic Have-Nots and Cultural Backlash.
\newblock \textit{HKS Working Paper No. RWP16-026}.

\bibitem[Guiso et al.(2017)]{guiso2017demand}
Guiso, L., H. Herrera, M. Morelli, and T. Sonno (2017).
\newblock Demand and Supply of Populism.
\newblock \textit{CEPR Discussion Paper No. DP11871}.

\bibitem[Margalit(2019)]{margalit2019economic}
Margalit, Y. (2019).
\newblock Economic Insecurity and the Causes of Populism, Reconsidered.
\newblock \textit{Journal of Economic Perspectives}, 33(4):152--170.

\bibitem[Colantone and Stanig(2018)]{colantone2018global}
Colantone, I. and P. Stanig (2018).
\newblock Global Competition and Brexit.
\newblock \textit{American Political Science Review}, 112(2):201--218.

\bibitem[Dippel, Gold, and Heblich(2022)]{dippel2022globalization}
Dippel, C., R. Gold, and S. Heblich (2022).
\newblock Globalization and Its (Dis-)Content: Trade Shocks and Voting Behavior.
\newblock \textit{American Economic Review}, 112(5):1631--1672.

\bibitem[Che et al.(2022)]{che2022did}
Che, Y., Y. Lu, J.R. Pierce, P.K. Schott, and Z. Tao (2022).
\newblock Did Trade Liberalization with China Influence US Elections?
\newblock \textit{Journal of International Economics}, 139:103652.

\bibitem[Autor, Dorn, and Hanson(2019)]{autor2019when}
Autor, D., D. Dorn, and G. Hanson (2019).
\newblock When Work Disappears: Manufacturing Decline and the Falling Marriage Market Value of Young Men.
\newblock \textit{American Economic Review: Insights}, 1(2):161--178.

\bibitem[Pierce and Schott(2020)]{pierce2020trade}
Pierce, J.R. and P.K. Schott (2020).
\newblock Trade Liberalization and Mortality: Evidence from US Counties.
\newblock \textit{American Economic Review: Insights}, 2(1):47--64.

\bibitem[Autor, Dorn, Hanson, and Majlesi(2017)]{autor2017trade}
Autor, D., D. Dorn, G. Hanson, and K. Majlesi (2017).
\newblock A Note on the Effect of Rising Trade Exposure on the 2016 Presidential Election.
\newblock \textit{MIT Working Paper}.

\bibitem[Becker, Fetzer, and Novy(2017)]{becker2017brexit}
Becker, S.O., T. Fetzer, and D. Novy (2017).
\newblock Who Voted for Brexit? A Comprehensive District-Level Analysis.
\newblock \textit{Economic Policy}, 32(92):601--650.

\bibitem[Gidron and Hall(2020)]{gidron2020populism}
Gidron, N. and P.A. Hall (2020).
\newblock Populism as a Problem of Social Integration.
\newblock \textit{Comparative Political Studies}, 53(7):1027--1059.

\bibitem[Ballard-Rosa, Jensen, and Scheve(2022)]{ballard2022economic}
Ballard-Rosa, C., A. Jensen, and K. Scheve (2022).
\newblock Economic Decline, Social Identity, and Authoritarian Values in the United States.
\newblock \textit{International Studies Quarterly}, 66(1):sqab027.

\bibitem[Mutz(2018)]{mutz2018status}
Mutz, D.C. (2018).
\newblock Status Threat, Not Economic Hardship, Explains the 2016 Presidential Vote.
\newblock \textit{Proceedings of the National Academy of Sciences}, 115(19):E4330--E4339.

\bibitem[Sides, Tesler, and Vavreck(2018)]{sides2018identity}
Sides, J., M. Tesler, and L. Vavreck (2018).
\newblock \textit{Identity Crisis: The 2016 Presidential Campaign and the Battle for the Meaning of America}.
\newblock Princeton University Press.

\bibitem[Morgan(2018)]{morgan2018education}
Morgan, S.L. (2018).
\newblock Status Threat, Material Interests, and the 2016 Presidential Vote.
\newblock \textit{Socius}, 4:1--17.

\bibitem[Abramowitz and McCoy(2019)]{abramowitz2019united}
Abramowitz, A.I. and J. McCoy (2019).
\newblock United States: Racial Resentment, Negative Partisanship, and Polarization in Trump's America.
\newblock \textit{The ANNALS of the American Academy of Political and Social Science}, 681(1):137--156.

\bibitem[Moretti(2012)]{moretti2012new}
Moretti, E. (2012).
\newblock \textit{The New Geography of Jobs}.
\newblock Houghton Mifflin Harcourt.

\bibitem[Iversen and Soskice(2019)]{iversen2019democracy}
Iversen, T. and D. Soskice (2019).
\newblock \textit{Democracy and Prosperity: Reinventing Capitalism through a Turbulent Century}.
\newblock Princeton University Press.

\bibitem[Chetty, Hendren, and Katz(2014)]{chetty2014effects}
Chetty, R., N. Hendren, and L.F. Katz (2014).
\newblock The Effects of Exposure to Better Neighborhoods on Children: New Evidence from the Moving to Opportunity Experiment.
\newblock \textit{American Economic Review}, 104(4):855--902.

\bibitem[Kuziemko and Washington(2021)]{kuziemko2021democrats}
Kuziemko, I. and E. Washington (2021).
\newblock Why Did the Democrats Lose the South? Bringing New Data to an Old Debate.
\newblock \textit{American Economic Review}, 111(6):3830--3867.

\bibitem[Lee and Lemieux(2010)]{lee2010regression}
Lee, D.S. and T. Lemieux (2010).
\newblock Regression Discontinuity Designs in Economics.
\newblock \textit{Journal of Economic Literature}, 48(2):281--355.

\bibitem[Cameron, Gelbach, and Miller(2008)]{cameron2008bootstrap}
Cameron, A.C., J.B. Gelbach, and D.L. Miller (2008).
\newblock Bootstrap-Based Improvements for Inference with Clustered Errors.
\newblock \textit{Review of Economics and Statistics}, 90(3):414--427.

\end{thebibliography}

\newpage
\appendix

\section{Data Appendix}

\subsection{Technology Vintage Data}

The technology vintage data come from establishment-level surveys compiled by Acemoglu, Lelarge, and Restrepo (2022), who develop measures of the modal age of capital equipment across U.S. metropolitan areas. The raw data cover 917 Core-Based Statistical Areas from 2010 to 2023; after merging with election data, 896 CBSAs remain in the analysis sample.

For each CBSA-year, we observe approximately 45 observations corresponding to different industry sectors (mean: 44.9, median: 44, range: 14--74). Each observation records the modal age (in years) of the primary production technology used by establishments in that industry-CBSA-year cell. We collapse to the CBSA-year level by computing the unweighted mean across industries. Results are robust to using the median or other percentiles (see Table \ref{tab:metro_micro}).

Key variables:
\begin{itemize}
\item \texttt{modal\_age\_mean}: Mean modal technology age across industries within CBSA-year
\item \texttt{modal\_age\_median}: Median modal technology age
\item \texttt{modal\_age\_p25}, \texttt{modal\_age\_p75}: 25th and 75th percentiles
\item \texttt{n\_sectors}: Number of industry observations used in aggregation (mean: 44.9)
\end{itemize}

The industry sectors correspond to 2-digit NAICS codes, covering manufacturing, retail, services, and other major industry groups. All sectors with non-missing modal age data are included; no minimum establishment count threshold is applied at the industry-CBSA level.

\subsection{Election Data}

County-level presidential election returns were obtained from the GitHub repository maintained by Tony McGovern, which compiles data from the MIT Election Data Science Lab and other sources. For 2016, data were scraped from Townhall.com; for 2020, from Fox News, Politico, and the New York Times; for 2024, from similar real-time tracking sources.

Variables constructed:
\begin{itemize}
\item \texttt{trump\_share}: Republican votes / Total votes $\times$ 100 (in percentage points)
\item \texttt{total\_votes}: Total votes cast in the CBSA
\end{itemize}

\subsection{CBSA-County Crosswalk}

We use the March 2020 CBSA delineation file from the Census Bureau, accessed via NBER's crosswalk service (file: \texttt{cbsa2fipsxw\_2020.csv}). This file maps each county (identified by 5-digit FIPS code) to its containing CBSA (if any).

Of the approximately 3,140 U.S. counties, roughly 1,900 fall within the 927 CBSAs defined in the March 2020 delineation (384 metropolitan + 543 micropolitan). The remaining counties are not part of any metropolitan or micropolitan statistical area and are excluded from our analysis. After merging with the technology data (which covers 917 CBSAs) and addressing missing election returns, our final sample includes 896 CBSAs.


\section{Additional Robustness Checks}

\subsection{Metropolitan vs. Micropolitan Areas}

Table A1 shows that results are similar for metropolitan statistical areas (population $\geq$ 50,000) and micropolitan statistical areas (population 10,000--50,000). The technology coefficient is slightly larger in metropolitan areas (0.12 vs. 0.10 pp) but confidence intervals overlap.

\subsection{Alternative Technology Measures}

Results are robust to using the median, 25th percentile, or 75th percentile of industry-level modal ages instead of the mean. The standardized coefficient (using z-scored technology age) is 2.6, indicating that a one standard deviation increase in technology age (approximately 15 years) is associated with 2.6 percentage points higher Trump share, consistent with the baseline per-year coefficient of 0.17.

\subsection{Non-linear Effects}

Adding a quadratic term for technology age yields a marginally significant negative coefficient (p = 0.05), suggesting slight concavity. However, the effect is small in magnitude and does not meaningfully alter interpretation.


\section{Additional Figures}

\begin{figure}[H]
\centering
\includegraphics[width=\textwidth]{figures/fig1_tech_age_distribution.pdf}
\caption{Distribution of Modal Technology Age Across U.S. Metropolitan Areas}
\label{fig:tech_dist}
\end{figure}

\begin{figure}[H]
\centering
\includegraphics[width=\textwidth]{figures/fig2_scatter_tech_trump.pdf}
\caption{Technology Age and Trump Vote Share by Election Year}
\label{fig:scatter}
\end{figure}

\begin{figure}[H]
\centering
\includegraphics[width=\textwidth]{figures/fig3_binscatter.pdf}
\caption{Binned Scatter Plot: Technology Age and Trump Vote Share}
\label{fig:binscatter}
\end{figure}

\begin{figure}[H]
\centering
\includegraphics[width=\textwidth]{figures/fig4_terciles.pdf}
\caption{Trump Vote Share by Technology Age Tercile}
\label{fig:terciles}
\end{figure}

\begin{figure}[H]
\centering
\includegraphics[width=0.8\textwidth]{figures/fig5_regional.pdf}
\caption{Technology Age Effect by Census Region. Blue bars indicate statistical significance at $p < 0.05$ (Midwest and West); grey bars indicate non-significance (Northeast and South). Exact coefficients and standard errors are reported in Table \ref{tab:regional}.}
\label{fig:regional}
\end{figure}

\begin{figure}[H]
\centering
\includegraphics[width=\textwidth]{figures/fig6_gains_vs_levels.pdf}
\caption{Levels vs. Gains: Testing for Causal Effects}
\label{fig:gains}
\end{figure}


\end{document}
