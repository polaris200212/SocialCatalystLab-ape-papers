\documentclass[12pt]{article}

% UTF-8 encoding and fonts
\usepackage[utf8]{inputenc}
\usepackage[T1]{fontenc}
\usepackage{lmodern}

% Page setup
\usepackage[margin=1in]{geometry}
\usepackage{setspace}
\onehalfspacing

% Typography
\usepackage{microtype}

% Math and symbols
\usepackage{amsmath,amssymb}

% Graphics
\usepackage{graphicx}
\usepackage{float}
\usepackage{subcaption}

% Tables
\usepackage{booktabs}
\usepackage{array}
\usepackage{multirow}
\usepackage{threeparttable}
\usepackage{longtable}
\usepackage{pdflscape}
\usepackage{siunitx}
\sisetup{detect-all=true, group-separator={,}, group-minimum-digits=4}

% Bibliography
\usepackage{natbib}
\bibliographystyle{aer}

% Hyperlinks
\usepackage{hyperref}
\hypersetup{
    colorlinks=true,
    linkcolor=blue,
    citecolor=blue,
    urlcolor=blue
}
\usepackage[nameinlink,noabbrev]{cleveref}

% Captions
\usepackage{caption}
\captionsetup{font=small,labelfont=bf}

% Section formatting
\usepackage{titlesec}
\titleformat{\section}{\large\bfseries}{\thesection.}{0.5em}{}
\titleformat{\subsection}{\normalsize\bfseries}{\thesubsection}{0.5em}{}

% Custom commands
\newcommand{\E}{\mathbb{E}}
\newcommand{\Var}{\text{Var}}
\newcommand{\Cov}{\text{Cov}}
\newcommand{\ind}{\mathbb{I}}
\newcommand{\sym}[1]{\ifmmode^{#1}\else\(^{#1}\)\fi}
\newcommand{\floatfoot}[1]{\par\vspace{0.3em}\small#1}

\title{Credit Markets, Social Networks, and America's Divided Geography:\\ The Anatomy of Economic Opportunity and Political Polarization}
\author{Autonomous Policy Evaluation Project\thanks{APEP Working Paper. Project repository: \url{https://github.com/SocialCatalystLab/auto-policy-evals. Correspondence: scl@econ.uzh.ch}. This paper was autonomously generated using Claude Code. Correspondence: scl@econ.uzh.ch}}
\date{January 2026}

\begin{document}

\maketitle

\begin{abstract}
\noindent
This paper provides a comprehensive descriptive portrait of how credit access and cross-class social networks cluster geographically across U.S. counties, and how these patterns correlate with political polarization. Combining county-level credit data from Opportunity Insights with social capital measures derived from 21 billion Facebook friendships and voting data from 2016--2024, we document three key findings. First, credit scores and economic connectedness are extraordinarily correlated ($r = 0.82$), revealing that places where people have better access to credit are also places where social networks bridge class boundaries. Second, after controlling for income, education, and demographics, counties with higher credit scores are \textit{less} Republican by 5.5 percentage points per standard deviation---a reversal of the raw correlation---while economic connectedness shows a modest negative association with Republican voting. Third, the shift toward Republicans from 2016--2024 was concentrated in counties with lower college attainment and higher delinquency rates. Our typology of American counties reveals a complex mosaic: ``Affluent Blue'' counties with high credit and high social capital cluster on the coasts, while ``Struggling Red'' counties dominate the interior. These patterns suggest that the same forces shaping economic opportunity---access to credit and bridging social capital---may also be shaping America's political divide.
\end{abstract}

\vspace{1em}
\noindent\textbf{JEL Codes:} D31, G51, P16, Z13 \\
\noindent\textbf{Keywords:} credit access, social capital, economic connectedness, political polarization, inequality

\newpage

\section{Introduction}

The United States appears increasingly divided---economically, socially, and politically. Over the past two decades, income inequality has risen, residential segregation by class has intensified, and political polarization has reached levels not seen since the Gilded Age. These trends have generated substantial scholarly interest, but most research examines them in isolation: economists study credit access and financial inclusion, sociologists analyze social networks and community ties, and political scientists investigate voting behavior and partisan sorting. This paper takes a different approach, examining how these dimensions of American life---credit, social networks, and politics---cluster together geographically and potentially reinforce one another.

We combine three remarkable data sources to paint a comprehensive portrait of America's economic and social geography. First, we use the Opportunity Insights Credit Access database, which provides county-level measures of credit scores, student loan balances, mortgage and auto loan balances, and delinquency rates for 2020. Second, we draw on the Social Capital II dataset from \citet{chetty2022social1,chetty2022social2}, which uses 21 billion Facebook friendships to measure ``economic connectedness''---the extent to which low-socioeconomic-status individuals form friendships with high-SES individuals. Third, we incorporate county-level presidential voting data from 2016, 2020, and 2024 to examine political patterns and changes over time.

Our analysis proceeds in two parts. Part I provides rich descriptive evidence on the geography of credit and social capital in America. We present ten main-text exhibits documenting spatial patterns: choropleth maps showing where credit scores are high versus low, where cross-class friendships are common versus rare, and where debt delinquency concentrates. A striking finding emerges: credit access and economic connectedness are extraordinarily correlated across counties ($r = 0.82$), suggesting that the same places that provide access to financial markets also foster bridging social capital. We also visualize the 100$\times$100 ``friendship matrix'' showing how Americans at each socioeconomic percentile form friendships with those at every other percentile---a vivid illustration of class-based homophily.

Part II examines whether these credit and social capital patterns correlate with political polarization. The raw correlations are weak: credit scores and economic connectedness show near-zero bivariate associations with Republican vote share. But controlling for income, education, and demographics dramatically changes the picture. Counties with one-standard-deviation higher credit scores are 5.5 percentage points \textit{less} Republican after controlling for observable characteristics---suggesting that conditional on socioeconomic fundamentals, greater credit access is associated with less support for Republicans. Economic connectedness shows a more modest negative relationship with Republican voting after controls.

We construct a county typology based on whether counties are above or below median on credit scores, economic connectedness, and Republican vote share. This yields eight distinct county types, from ``Affluent Blue'' (high credit, high social capital, Democratic) to ``Struggling Red'' (low credit, low social capital, Republican). Mapping these types reveals a geographic mosaic: Affluent Blue counties cluster along the coasts and in major metropolitan areas; Struggling Red counties dominate the interior, particularly in Appalachia and the rural South. The emergence of ``Rural Connected'' counties---low credit but high social capital, voting Republican---suggests that social networks alone cannot explain political outcomes.

Our findings contribute to several literatures. First, we extend research on the geography of opportunity \citep{chetty2014land,chetty2018opportunity} by showing that credit access patterns mirror social capital patterns. Second, we contribute to the growing literature on social capital and political outcomes \citep{putnam2000bowling,guillen2024social} by providing granular county-level evidence on how economic connectedness relates to political behavior. Third, we inform debates about the ``left behind'' regions that have shifted toward populist politics \citep{rodriguez2018revenge,cramer2016politics} by documenting how credit and social capital jointly characterize these areas.

Several important limitations deserve emphasis. This is a purely descriptive and correlational study. We make no causal claims. The patterns we document could reflect reverse causality (political sorting into counties), omitted variables (underlying factors that jointly determine credit access, social networks, and political preferences), or measurement issues. Furthermore, our county-level analysis is subject to ecological fallacy concerns: patterns across counties need not apply to individuals within counties. Nonetheless, the striking co-movement of credit, social capital, and political outcomes across space suggests these are interrelated phenomena worthy of further causal investigation.

\section{Data}

We combine data from multiple sources, all measured at the county level. Our final analysis sample includes 2,992 counties with complete data on credit access, social capital, voting, and covariates, representing approximately 95\% of the continental U.S. population.

\subsection{Credit Access Data}

The Opportunity Insights Credit Access data provides county-level measures of credit market participation derived from credit bureau records. For each county, we observe average credit scores, student loan balances, mortgage balances, auto loan balances, credit card balances, and debt delinquency rates, all measured in 2020. We use the ``pooled'' estimates that combine all demographic groups and parental income percentiles.

Credit scores range from 593 to 754 across counties, with a mean of 677 (SD = 29). The average county has mean student loan balances of \$10,000 and mortgage balances of \$70,000. The delinquency variable from Opportunity Insights measures the share of individuals with any delinquent debt on their credit record, with a county mean of 45\%. This includes any account 30+ days past due and is substantially higher than 90+ day ``serious delinquency'' measures commonly used in credit research.\footnote{The high delinquency rates reflect that this variable captures \textit{any} delinquent tradeline, not just severely delinquent debt. The 90-day threshold would yield much lower rates (typically 5--10\% at the individual level).} These variables capture different aspects of household financial health and access to credit markets.

\subsection{Social Capital Data}

We use the Social Capital II dataset from \citet{chetty2022social1,chetty2022social2}, which constructs social capital measures from de-identified Facebook friendship data covering 21 billion friendships. The key variable is ``economic connectedness'' (EC), a normalized index measuring the extent of cross-class friendships among low-SES individuals relative to what would be expected given exposure. Values above 1 indicate more cross-class friendships than the national average, while values below 1 indicate fewer. This measure is strongly predictive of upward economic mobility.

For 3,018 counties, we observe EC ranging from 0.29 to 1.36 with a mean of 0.81 (SD = 0.18). We also use related measures including ``friending bias'' (the tendency to befriend individuals of one's own SES after controlling for exposure opportunities) and ``clustering'' (the density of friend networks). The underlying Facebook friendship data was collected in 2018--2019 and published in 2022 \citep{chetty2022social1}, predating the 2020 election outcomes we analyze.

Additionally, we utilize the national 100$\times$100 friendship matrix, which reports the probability that an individual at SES percentile $p$ forms a friendship with someone at percentile $p'$, for all combinations of $p$ and $p'$. This matrix vividly illustrates class-based homophily in American social networks.

\subsection{Voting Data}

We compile county-level presidential election returns for 2016, 2020, and 2024 from the MIT Election Data and Science Lab and related sources. For each county-election, we observe vote counts and calculate Republican and Democratic vote shares. Our primary outcome is the Republican vote share in 2020 (mean = 65\%, SD = 15\%). We also construct the change in Republican vote share from 2016 to 2024, which captures political shifts over the Trump era. The mean county shifted 3.3 percentage points toward Republicans over this period.

\subsection{County Covariates}

From Opportunity Insights and the Opportunity Atlas, we obtain county-level covariates including median household income, college attainment share, employment rates, poverty rates, Gini coefficients, racial composition (share white, share Black), single-parent household rates, and population density. These variables are measured primarily in 2010 to predate our outcome measures.

\subsection{Sample Construction}

We merge all datasets by county FIPS code. Our baseline sample includes 2,992 counties with complete data on credit scores, economic connectedness, and 2020 voting outcomes. Sample sizes vary slightly across specifications due to missing values in particular covariates (e.g., friending bias, clustering), as noted in each table. Table~\ref{tab:summary} presents summary statistics for key variables, with N reflecting variable-specific availability before the final merge.

\begin{table}[H]
\centering
\caption{Summary Statistics}
\begin{threeparttable}
\small
\begin{tabular}{l S[table-format=3.2] S[table-format=2.2] S[table-format=3.2] S[table-format=3.2] S[table-format=5.0]}
\toprule
Variable & {Mean} & {SD} & {Min} & {Max} & {N} \\
\midrule
\multicolumn{6}{l}{\textit{Panel A: Credit Access}} \\
Credit Score & 677 & 29 & 593 & 754 & 3089 \\
Student Loan Balance (\$) & 10012 & 2511 & 2706 & 25119 & 3089 \\
Mortgage Balance (\$) & 70191 & 22729 & 14193 & 184429 & 3089 \\
Delinquency Rate & 0.45 & 0.12 & 0.14 & 0.82 & 3089 \\
\midrule
\multicolumn{6}{l}{\textit{Panel B: Social Capital}} \\
Economic Connectedness & 0.81 & 0.18 & 0.29 & 1.36 & 3018 \\
Friending Bias & 0.06 & 0.05 & -0.11 & 0.23 & 3012 \\
Clustering & 0.12 & 0.02 & 0.07 & 0.19 & 3018 \\
\midrule
\multicolumn{6}{l}{\textit{Panel C: Political}} \\
GOP Vote Share 2020 & 0.65 & 0.15 & 0.04 & 0.96 & 3061 \\
GOP Change 2016--2024 & 0.03 & 0.04 & -0.12 & 0.19 & 3052 \\
\midrule
\multicolumn{6}{l}{\textit{Panel D: Covariates}} \\
Median HH Income (2010) & 61786 & 16036 & 26952 & 157804 & 3087 \\
College Share (2010) & 0.19 & 0.09 & 0.04 & 0.70 & 3087 \\
Share White (2010) & 0.83 & 0.17 & 0.10 & 0.99 & 3087 \\
Population Density & 253 & 1785 & 0.04 & {69,468} & 3087 \\
\bottomrule
\end{tabular}
\begin{tablenotes}[flushleft]
\small
\item Notes: Unit of observation is the county. N reflects variable-specific availability before merging; the final analysis sample with complete data on all key variables includes 2,992 counties. Credit and social capital data from Opportunity Insights. Voting data from MIT Election Lab. Panel D covariates from Opportunity Atlas.
\end{tablenotes}
\end{threeparttable}
\label{tab:summary}
\end{table}

\section{Part I: The Geography of Credit and Social Capital}

\subsection{The Credit Score Map of America}

Figure~\ref{fig:credit_map} presents the geographic distribution of average credit scores across U.S. counties. A clear spatial pattern emerges: credit scores are highest (shown in yellow) in the Upper Midwest, Great Plains, and Pacific Northwest, and lowest (shown in dark purple) across the Deep South, Appalachia, and the Mississippi Delta. The mean credit score ranges from below 600 in some Delta counties to above 750 in parts of Minnesota and North Dakota.

\begin{figure}[H]
\centering
\includegraphics[width=\textwidth]{figures/fig1_credit_score_map.png}
\caption{The Credit Score Map of America}
\label{fig:credit_map}
\floatfoot{\textit{Notes}: County-level average credit scores, 2020. Source: Opportunity Insights Credit Access Data. Scores winsorized at 600--750 for visualization.}
\end{figure}

This geographic pattern correlates strongly with historical and contemporary measures of economic opportunity. The high-credit-score regions of the Upper Midwest are the same areas identified by \citet{chetty2014land} as having high rates of intergenerational mobility. The low-credit-score regions of the South overlap substantially with areas of persistent poverty and limited economic mobility.

\subsection{The Social Capital Map of America}

Figure~\ref{fig:ec_map} shows the geographic distribution of economic connectedness---the extent to which low-SES individuals form friendships with high-SES individuals. The geographic pattern bears striking resemblance to the credit score map. Economic connectedness is highest in the same Upper Midwest and Plains regions where credit scores are highest, and lowest in the South and Appalachia.

\begin{figure}[H]
\centering
\includegraphics[width=\textwidth]{figures/fig2_economic_connectedness_map.png}
\caption{The Social Capital Map of America}
\label{fig:ec_map}
\floatfoot{\textit{Notes}: Economic connectedness is a normalized index of cross-class friendships among low-SES individuals (values $>1$ indicate above-national-average cross-class connectivity). Source: Chetty et al. (2022), Social Capital II.}
\end{figure}

The correlation between credit scores and economic connectedness across counties is remarkable: $r = 0.82$. This extraordinarily high correlation---among the highest we have seen across county-level socioeconomic measures---suggests that credit access and bridging social capital are deeply intertwined aspects of local opportunity structures. Places that provide access to financial markets are also places where social networks span class boundaries.

\subsection{America's Credit Divide: Delinquency}

Figure~\ref{fig:delinq_map} maps the delinquency rate---the share of individuals with any delinquent debt on their credit record. This measure captures credit distress and the inability to keep up with debt payments. The pattern is essentially the inverse of the credit score map: delinquency rates are highest in the South, Appalachia, and minority-majority counties, and lowest in the Upper Midwest. The correlation between credit scores and delinquency is $r = -0.98$, indicating these measures capture the same underlying construct of credit health.

\begin{figure}[H]
\centering
\includegraphics[width=\textwidth]{figures/fig3_delinquency_map.png}
\caption{America's Credit Divide: Debt Delinquency}
\label{fig:delinq_map}
\floatfoot{\textit{Notes}: Share of individuals with any delinquent debt on credit record, 2020. Source: Opportunity Insights Credit Access Data.}
\end{figure}

\subsection{The Friendship Matrix: Who Befriends Whom?}

Figure~\ref{fig:friendship_matrix} visualizes the national 100$\times$100 friendship matrix, showing the probability of friendship between individuals at each pair of SES percentiles. The diagonal concentration of probability mass reveals strong homophily: Americans are most likely to befriend others at similar socioeconomic levels. The probability of friendship declines monotonically as the SES gap increases.

\begin{figure}[H]
\centering
\includegraphics[width=0.85\textwidth]{figures/fig4_friendship_matrix.png}
\caption{The Friendship Matrix: Who Befriends Whom in America?}
\label{fig:friendship_matrix}
\floatfoot{\textit{Notes}: Each cell shows the probability that an individual at the row SES percentile forms a friendship with someone at the column percentile. Based on 21 billion Facebook friendships. Source: Chetty et al. (2022), Social Capital II.}
\end{figure}

The matrix is not symmetric around the diagonal: high-SES individuals (top rows) show a stronger diagonal concentration than low-SES individuals. This pattern is consistent with ``opportunity hoarding'' by elites, who may have more control over their social environments and thus more ability to limit cross-class interaction.

\subsection{Credit and Social Capital: A Tight Relationship}

Figure~\ref{fig:credit_ec} presents a scatter plot of economic connectedness against credit scores across counties, with point size proportional to population. The strong positive relationship ($r = 0.82$) is visually apparent: counties with higher credit scores have higher economic connectedness, with relatively little scatter around the regression line.

\begin{figure}[H]
\centering
\includegraphics[width=0.9\textwidth]{figures/fig5_credit_social_capital.png}
\caption{Credit Access and Social Capital Move Together}
\label{fig:credit_ec}
\floatfoot{\textit{Notes}: Each point is a county, sized by population. Line shows OLS fit. Source: Opportunity Insights Credit Access (2020) and Social Capital II (2022).}
\end{figure}

This tight relationship raises important questions about mechanisms. Does credit access enable social mobility that leads to cross-class friendships? Do bridging social networks provide information and connections that improve credit outcomes? Or do both credit access and social capital reflect deeper structural features of local economies and institutions? Our correlational design cannot distinguish among these possibilities, but the strength of the relationship suggests they are worth investigating causally.

\section{Part II: Credit, Social Capital, and Political Polarization}

\subsection{America's Political Geography}

Figure~\ref{fig:gop_map} maps Republican vote share in the 2020 presidential election. The familiar red--blue map shows Republican dominance across the interior and South, with Democratic strength concentrated in urban counties, the coasts, and majority-minority areas. Notably, 82\% of U.S. counties gave a majority to Trump in 2020, though these counties represent a smaller share of the total population.

\begin{figure}[H]
\centering
\includegraphics[width=\textwidth]{figures/fig6_gop_2020_map.png}
\caption{America's Political Geography: 2020 Presidential Election}
\label{fig:gop_map}
\floatfoot{\textit{Notes}: County-level Republican vote share, 2020. Source: MIT Election Data and Science Lab.}
\end{figure}

Figure~\ref{fig:gop_shift} shows the change in Republican vote share from 2016 to 2024---the political shift over the Trump era. Red counties became more Republican; blue counties shifted toward Democrats. The map reveals that Republican gains were concentrated in Hispanic-majority counties along the Texas border and in Florida, while Democratic gains occurred in suburban counties and parts of the Sun Belt.

\begin{figure}[H]
\centering
\includegraphics[width=\textwidth]{figures/fig7_gop_shift_16_24_map.png}
\caption{America's Political Shift: 2016 to 2024}
\label{fig:gop_shift}
\floatfoot{\textit{Notes}: Change in Republican vote share from 2016 to 2024. Positive values (red) indicate shift toward Republicans. Source: MIT Election Data and Science Lab.}
\end{figure}

\subsection{Credit and Political Alignment}

What is the relationship between credit access and political behavior? Figure~\ref{fig:credit_politics} plots GOP vote share against credit scores. The raw correlation is essentially zero ($r = 0.04$)---counties with high and low credit scores appear similarly likely to vote Republican.

\begin{figure}[H]
\centering
\includegraphics[width=0.85\textwidth]{figures/fig8_credit_politics.png}
\caption{Credit Access and Political Alignment}
\label{fig:credit_politics}
\floatfoot{\textit{Notes}: Each small point is a county. Large points show binned means. Line shows OLS fit. Source: Opportunity Insights (2020), MIT Election Lab.}
\end{figure}

But this zero correlation masks important heterogeneity. Our regression results progressively adding controls reveal important patterns. In the bivariate specification (column 1), credit scores are essentially uncorrelated with Republican voting. Adding median household income (column 2) reveals a positive partial correlation: conditional on income, higher credit scores are associated with \textit{more} Republican voting (3.4 pp per SD). Adding education (column 3) strengthens this positive association (4.8 pp per SD). But adding demographic controls (column 4) reverses the sign: conditional on income, education, and demographics, higher credit scores are associated with \textit{less} Republican voting ($-2.4$ pp per SD). This reversal reflects the complex relationships among race, education, credit access, and political preferences.

In the full specification with demographics and density (column 6), one standard deviation higher credit scores are associated with 5.5 percentage points lower Republican vote share, holding other factors constant. This suggests that credit access captures something distinct from income or education that is associated with more moderate or Democratic political preferences.

% Table 3: Credit and Politics

\begin{table}[htbp]
   \caption{\label{tab:credit_politics} Credit Scores and Republican Vote Share, 2020}
   \bigskip
   \centering
   \begin{tabular}{lcccccc}
      \toprule
       & \multicolumn{6}{c}{GOP Vote Share 2020}\\
                       & (1)            & (2)             & (3)            & (4)             & (5)             & (6)\\  
      \midrule 
      Constant         & 0.6487$^{***}$ & 0.9054$^{***}$  & 0.8713$^{***}$ & 0.4069$^{***}$  & 0.3728$^{***}$  & 0.2715$^{***}$\\   
                       & (0.0029)       & (0.0123)        & (0.0101)       & (0.0225)        & (0.0211)        & (0.0500)\\   
      Credit Score (Z) & 0.0024         & 0.0339$^{***}$  & 0.0480$^{***}$ & -0.0237$^{***}$ & -0.0431$^{***}$ & -0.0546$^{***}$\\   
                       & (0.0029)       & (0.0031)        & (0.0026)       & (0.0028)        & (0.0028)        & (0.0032)\\   
      hhinc\_1000      &                & -0.0041$^{***}$ & 0.0006$^{***}$ & 0.0007$^{***}$  & 0.0017$^{***}$  & 0.0009$^{***}$\\   
                       &                & (0.0002)        & (0.0002)       & (0.0002)        & (0.0002)        & (0.0002)\\   
      College Share    &                &                 & -1.362$^{***}$ & -1.013$^{***}$  & -0.8054$^{***}$ & -0.8640$^{***}$\\   
                       &                &                 & (0.0360)       & (0.0306)        & (0.0302)        & (0.0353)\\   
      Share White      &                &                 &                & 0.4872$^{***}$  & 0.5336$^{***}$  & 0.4833$^{***}$\\   
                       &                &                 &                & (0.0213)        & (0.0200)        & (0.0211)\\   
      Share Black      &                &                 &                & -0.1400$^{***}$ & -0.0695$^{***}$ & -0.1144$^{***}$\\   
                       &                &                 &                & (0.0238)        & (0.0225)        & (0.0227)\\   
      log\_popdensity  &                &                 &                &                 & -0.0296$^{***}$ & -0.0293$^{***}$\\   
                       &                &                 &                &                 & (0.0014)        & (0.0014)\\   
      Poverty Share    &                &                 &                &                 &                 & -0.4314$^{***}$\\   
                       &                &                 &                &                 &                 & (0.0612)\\   
      Gini Coefficient &                &                 &                &                 &                 & 0.4226$^{***}$\\   
                       &                &                 &                &                 &                 & (0.0678)\\   
      Employment Rate  &                &                 &                &                 &                 & 0.1382$^{***}$\\   
                       &                &                 &                &                 &                 & (0.0314)\\   
       \\
      R$^2$            & 0.00023        & 0.13325         & 0.41449        & 0.61271         & 0.66220         & 0.67468\\  
      Observations     & 2,992          & 2,991           & 2,991          & 2,991           & 2,991           & 2,991\\  
      \bottomrule
   \end{tabular}
\end{table}




\subsection{Economic Connectedness and Political Alignment}

Figure~\ref{fig:ec_politics} plots GOP vote share against economic connectedness. Again, the raw correlation is weak ($r = 0.06$).

\begin{figure}[H]
\centering
\includegraphics[width=0.85\textwidth]{figures/fig9_ec_politics.png}
\caption{Social Capital and Political Alignment}
\label{fig:ec_politics}
\floatfoot{\textit{Notes}: Economic Connectedness is a normalized index (values $>1$ = above-average cross-class connectivity). Large points show binned means. Source: Chetty et al. (2022), MIT Election Lab.}
\end{figure}

Regression analysis reveals similar patterns. In the bivariate specification, economic connectedness is weakly positively correlated with Republican voting. But with full controls (column 6), one SD higher economic connectedness is associated with 1.8 pp lower Republican vote share. The effect is smaller than for credit scores, and less robust across specifications.

% Table 4: Economic Connectedness and Politics

\begin{table}[htbp]
   \caption{\label{tab:ec_politics} Economic Connectedness and Republican Vote Share, 2020}
   \bigskip
   \centering
   \begin{tabular}{lcccccc}
      \toprule
       & \multicolumn{6}{c}{GOP Vote Share 2020}\\
                                 & (1)            & (2)             & (3)            & (4)             & (5)             & (6)\\  
      \midrule 
      Constant                   & 0.6488$^{***}$ & 0.9564$^{***}$  & 0.9246$^{***}$ & 0.4937$^{***}$  & 0.4128$^{***}$  & 0.3160$^{***}$\\   
                                 & (0.0029)       & (0.0126)        & (0.0102)       & (0.0253)        & (0.0246)        & (0.0531)\\   
      Economic Connectedness (Z) & 0.0089$^{***}$ & 0.0523$^{***}$  & 0.0657$^{***}$ & 0.0032          & -0.0185$^{***}$ & -0.0182$^{***}$\\   
                                 & (0.0029)       & (0.0032)        & (0.0026)       & (0.0029)        & (0.0031)        & (0.0032)\\   
      hhinc\_1000                &                & -0.0050$^{***}$ & -0.0002        & 0.0003$^{*}$    & 0.0015$^{***}$  & 0.0014$^{***}$\\   
                                 &                & (0.0002)        & (0.0002)       & (0.0002)        & (0.0002)        & (0.0003)\\   
      College Share              &                &                 & -1.378$^{***}$ & -1.098$^{***}$  & -0.9057$^{***}$ & -0.9943$^{***}$\\   
                                 &                &                 & (0.0343)       & (0.0308)        & (0.0315)        & (0.0369)\\   
      Share White                &                &                 &                & 0.4245$^{***}$  & 0.5035$^{***}$  & 0.4789$^{***}$\\   
                                 &                &                 &                & (0.0230)        & (0.0224)        & (0.0235)\\   
      Share Black                &                &                 &                & -0.0965$^{***}$ & -0.0020         & -0.0247\\   
                                 &                &                 &                & (0.0236)        & (0.0231)        & (0.0233)\\   
      log\_popdensity            &                &                 &                &                 & -0.0259$^{***}$ & -0.0265$^{***}$\\   
                                 &                &                 &                &                 & (0.0015)        & (0.0015)\\   
      Poverty Share              &                &                 &                &                 &                 & -0.3044$^{***}$\\   
                                 &                &                 &                &                 &                 & (0.0634)\\   
      Gini Coefficient           &                &                 &                &                 &                 & 0.4808$^{***}$\\   
                                 &                &                 &                &                 &                 & (0.0707)\\   
      Employment Rate            &                &                 &                &                 &                 & -0.0257\\   
                                 &                &                 &                &                 &                 & (0.0311)\\   
       \\
      R$^2$                      & 0.00318        & 0.17373         & 0.46307        & 0.60359         & 0.63928         & 0.64606\\  
      Observations               & 2,992          & 2,991           & 2,991          & 2,991           & 2,991           & 2,991\\  
      \bottomrule
   \end{tabular}
\end{table}




\subsection{Combined Models and Horse Race}

Table~\ref{tab:combined} presents models including both credit scores and economic connectedness. Given their high correlation ($r = 0.82$), multicollinearity is a concern. In column 3, we include both measures with basic controls: the credit score coefficient is $-4.7$ pp per SD, while the EC coefficient is $+0.9$ pp per SD. When we add delinquency rates (column 4), the credit score effect attenuates while the delinquency effect is positive and significant: counties with higher delinquency rates vote \textit{more} Republican, conditional on other factors.

The full specification (column 6) includes credit scores, economic connectedness, delinquency rates, friending bias, clustering, student loan balances, and the full set of controls. The R-squared reaches 0.71, indicating these variables explain substantial variation in county-level voting. The key finding is that friending bias---the tendency to befriend one's own SES group---is strongly \textit{negatively} associated with Republican voting. The coefficient of $-0.59$ on the raw friending bias variable implies that a one standard deviation increase (0.05 units; see Table~\ref{tab:summary}) is associated with approximately 3 percentage points lower Republican vote share ($-0.59 \times 0.05 \approx -0.03$). This means that counties with \textit{more} within-class friendship (higher friending bias) vote \textit{less} Republican, conditional on other factors. This is consistent with the interpretation that areas with stronger within-group social ties may have different political orientations than areas with more cross-class mixing.

% Table 5: Combined Models

\begin{table}[htbp]
   \caption{\label{tab:combined} Credit, Social Capital, and Republican Vote Share: Combined Models}
   \bigskip
   \centering
   \begin{tabular}{lcccccc}
      \toprule
       & \multicolumn{6}{c}{pct\_gop\_2020}\\
                           & (1)             & (2)             & (3)             & (4)             & (5)             & (6)\\  
      \midrule 
      Constant             & 0.3728$^{***}$  & 0.4128$^{***}$  & 0.3994$^{***}$  & 0.4259$^{***}$  & 0.4584$^{***}$  & 0.3697$^{***}$\\   
                           & (0.0211)        & (0.0246)        & (0.0238)        & (0.0242)        & (0.0236)        & (0.0492)\\   
      credit\_score\_z     & -0.0431$^{***}$ &                 & -0.0473$^{***}$ & 0.0095          & 0.0323$^{***}$  & 0.0166\\   
                           & (0.0028)        &                 & (0.0033)        & (0.0107)        & (0.0105)        & (0.0103)\\   
      hhinc\_1000          & 0.0017$^{***}$  & 0.0015$^{***}$  & 0.0016$^{***}$  & 0.0014$^{***}$  & 0.0017$^{***}$  & 0.0008$^{***}$\\   
                           & (0.0002)        & (0.0002)        & (0.0002)        & (0.0002)        & (0.0002)        & (0.0002)\\   
      frac\_coll\_2010     & -0.8054$^{***}$ & -0.9057$^{***}$ & -0.8221$^{***}$ & -0.8151$^{***}$ & -0.7678$^{***}$ & -0.7759$^{***}$\\   
                           & (0.0302)        & (0.0315)        & (0.0310)        & (0.0309)        & (0.0303)        & (0.0373)\\   
      share\_white\_2010   & 0.5336$^{***}$  & 0.5035$^{***}$  & 0.5132$^{***}$  & 0.4993$^{***}$  & 0.4601$^{***}$  & 0.4871$^{***}$\\   
                           & (0.0200)        & (0.0224)        & (0.0217)        & (0.0217)        & (0.0214)        & (0.0222)\\   
      share\_black\_2010   & -0.0695$^{***}$ & -0.0020         & -0.0813$^{***}$ & -0.0884$^{***}$ & -0.1216$^{***}$ & 0.0546$^{**}$\\   
                           & (0.0225)        & (0.0231)        & (0.0230)        & (0.0230)        & (0.0225)        & (0.0276)\\   
      log\_popdensity      & -0.0296$^{***}$ & -0.0259$^{***}$ & -0.0286$^{***}$ & -0.0298$^{***}$ & -0.0262$^{***}$ & -0.0299$^{***}$\\   
                           & (0.0014)        & (0.0015)        & (0.0015)        & (0.0015)        & (0.0015)        & (0.0015)\\   
      ec\_z                &                 & -0.0185$^{***}$ & 0.0085$^{**}$   & 0.0076$^{**}$   & -0.0179$^{***}$ & -0.0220$^{***}$\\   
                           &                 & (0.0031)        & (0.0035)        & (0.0035)        & (0.0040)        & (0.0038)\\   
      delinquency\_z       &                 &                 &                 & 0.0555$^{***}$  & 0.0657$^{***}$  & 0.0500$^{***}$\\   
                           &                 &                 &                 & (0.0099)        & (0.0097)        & (0.0094)\\   
      friending\_bias      &                 &                 &                 &                 & -0.5626$^{***}$ & -0.5931$^{***}$\\   
                           &                 &                 &                 &                 & (0.0432)        & (0.0420)\\   
      clustering           &                 &                 &                 &                 &                 & -1.020$^{***}$\\   
                           &                 &                 &                 &                 &                 & (0.1188)\\   
      student\_loan\_z     &                 &                 &                 &                 &                 & -0.0242$^{***}$\\   
                           &                 &                 &                 &                 &                 & (0.0024)\\   
      poor\_share\_2010    &                 &                 &                 &                 &                 & -0.2504$^{***}$\\   
                           &                 &                 &                 &                 &                 & (0.0590)\\   
      gini\_2010           &                 &                 &                 &                 &                 & 0.4084$^{***}$\\   
                           &                 &                 &                 &                 &                 & (0.0653)\\   
      emp\_2010            &                 &                 &                 &                 &                 & 0.1568$^{***}$\\   
                           &                 &                 &                 &                 &                 & (0.0298)\\   
       \\
      R$^2$                & 0.66220         & 0.63928         & 0.66286         & 0.66635         & 0.68409         & 0.71277\\  
      Observations         & 2,991           & 2,991           & 2,991           & 2,991           & 2,985           & 2,985\\  
      \bottomrule
   \end{tabular}
\end{table}




\subsection{Polarization: What Predicts Political Shift?}

Table~\ref{tab:polarization} examines what predicts the change in Republican vote share from 2016 to 2024. The bivariate correlations are modest: higher credit scores predict less Republican shift ($-0.6$ pp per SD), and higher economic connectedness predicts less Republican shift ($-0.7$ pp per SD). These patterns suggest that credit-constrained, socially isolated counties shifted \textit{toward} Republicans during the Trump era.

With full controls (column 5), the pattern changes: credit scores and delinquency rates have strong effects, while economic connectedness has a small positive effect. Counties with higher delinquency rates experienced significantly larger shifts toward Republicans ($+3.9$ pp per SD), while counties with higher college attainment experienced shifts \textit{away} from Republicans (approximately $-1.8$ pp for each 10-percentage-point increase in college share).

% Table 6: Polarization

\begin{table}[htbp]
   \caption{\label{tab:polarization} Credit, Social Capital, and Political Shift 2016--2024}
   \bigskip
   \centering
   \begin{tabular}{lcccccc}
      \toprule
       & \multicolumn{6}{c}{GOP Shift 2016-24}\\
                                 & (1)             & (2)             & (3)             & (4)                     & (5)                     & (6)\\  
      \midrule 
      Constant                   & 0.0330$^{***}$  & 0.0330$^{***}$  & 0.0330$^{***}$  & 0.1580$^{***}$          & 0.0510$^{***}$          & 0.2205$^{***}$\\   
                                 & (0.0006)        & (0.0006)        & (0.0006)        & (0.0083)                & (0.0146)                & (0.0077)\\   
      Credit Score (Z)           & -0.0057$^{***}$ &                 & 0.0007          & 0.0011                  & 0.0450$^{***}$          & 0.0340$^{***}$\\   
                                 & (0.0007)        &                 & (0.0011)        & (0.0011)                & (0.0036)                & (0.0033)\\   
      Economic Connectedness (Z) &                 & -0.0072$^{***}$ & -0.0078$^{***}$ & 0.0020$^{*}$            & 0.0025$^{**}$           & 0.0023$^{**}$\\   
                                 &                 & (0.0006)        & (0.0011)        & (0.0012)                & (0.0012)                & (0.0011)\\   
      hhinc\_1000                &                 &                 &                 & -0.0003$^{***}$         & 0.0003$^{***}$          & -0.0002$^{***}$\\   
                                 &                 &                 &                 & ($6.09\times 10^{-5}$)  & ($8.34\times 10^{-5}$)  & ($5.53\times 10^{-5}$)\\    
      College Share              &                 &                 &                 & -0.1344$^{***}$         & -0.1801$^{***}$         & -0.2325$^{***}$\\   
                                 &                 &                 &                 & (0.0109)                & (0.0122)                & (0.0103)\\   
      Share White                &                 &                 &                 & -0.0807$^{***}$         & -0.0573$^{***}$         & -0.0113\\   
                                 &                 &                 &                 & (0.0076)                & (0.0077)                & (0.0073)\\   
      Share Black                &                 &                 &                 & -0.0742$^{***}$         & -0.0587$^{***}$         & -0.0840$^{***}$\\   
                                 &                 &                 &                 & (0.0081)                & (0.0079)                & (0.0071)\\   
      log\_popdensity            &                 &                 &                 & -0.0010$^{**}$          & -0.0022$^{***}$         & -0.0056$^{***}$\\   
                                 &                 &                 &                 & (0.0005)                & (0.0005)                & (0.0005)\\   
      Delinquency Rate (Z)       &                 &                 &                 &                         & 0.0385$^{***}$          & 0.0390$^{***}$\\   
                                 &                 &                 &                 &                         & (0.0033)                & (0.0031)\\   
      Poverty Share              &                 &                 &                 &                         & 0.2471$^{***}$          &   \\   
                                 &                 &                 &                 &                         & (0.0198)                &   \\   
      Gini Coefficient           &                 &                 &                 &                         & 0.0456$^{**}$           &   \\   
                                 &                 &                 &                 &                         & (0.0231)                &   \\   
      pct\_gop\_2016             &                 &                 &                 &                         &                         & -0.1443$^{***}$\\   
                                 &                 &                 &                 &                         &                         & (0.0053)\\   
       \\
      R$^2$                      & 0.02529         & 0.04059         & 0.04073         & 0.19182                 & 0.26497                 & 0.37390\\  
      Observations               & 2,983           & 2,983           & 2,983           & 2,983                   & 2,983                   & 2,983\\  
      \bottomrule
   \end{tabular}
\end{table}




\subsection{A Typology of American Counties}

Figure~\ref{fig:typology} presents a county typology based on whether each county is above or below the median on credit scores, economic connectedness, and Republican vote share. This yields eight types:

\begin{itemize}
\item \textbf{Affluent Blue}: High credit, high EC, Democratic (e.g., coastal metros)
\item \textbf{Prosperous Red}: High credit, high EC, Republican (e.g., wealthy exurbs)
\item \textbf{Urban Isolated}: High credit, low EC, Democratic (e.g., some urban cores)
\item \textbf{Suburban Red}: High credit, low EC, Republican (e.g., outer suburbs)
\item \textbf{Connected Blue}: Low credit, high EC, Democratic (e.g., diverse urban)
\item \textbf{Rural Connected}: Low credit, high EC, Republican (e.g., Upper Midwest rural)
\item \textbf{Struggling Blue}: Low credit, low EC, Democratic (e.g., minority-majority South)
\item \textbf{Struggling Red}: Low credit, low EC, Republican (e.g., Appalachia)
\end{itemize}

\begin{figure}[H]
\centering
\includegraphics[width=\textwidth]{figures/fig10_county_typology.png}
\caption{America's Divided Counties: A Typology}
\label{fig:typology}
\floatfoot{\textit{Notes}: Classification based on whether counties are above/below median on credit scores, economic connectedness, and Republican vote share 2020. Source: Opportunity Insights, Chetty et al. (2022), MIT Election Lab.}
\end{figure}

The geographic patterns are striking. Affluent Blue counties cluster on the coasts and in metropolitan cores. Prosperous Red counties ring major metros and dominate the Mountain West. Struggling Red counties blanket Appalachia, the Ozarks, and much of the rural South. Rural Connected counties---an interesting type with low credit but high social capital, voting Republican---are concentrated in the Upper Midwest, Great Plains, and parts of the Pacific Northwest.

The existence of the Rural Connected type is notable: it shows that high economic connectedness alone does not predict Democratic voting. These counties have strong bridging social capital but lack access to credit markets, and they voted strongly Republican. This suggests that credit access and social capital may operate through different channels in shaping political preferences.

\section{Discussion}

Our findings reveal a tight interrelationship between credit access, social capital, and political geography in America. The extraordinarily high correlation between credit scores and economic connectedness ($r = 0.82$) suggests that these are not independent dimensions of local opportunity---places with good credit markets also tend to have bridging social networks. Whether this reflects common causes (e.g., local institutions, history of development) or causal relationships between credit and social capital remains an important question for future research.

The relationship between these economic-social factors and political behavior is more complex. Raw correlations are weak, but controlling for socioeconomic fundamentals reveals that higher credit access is associated with less Republican voting. This pattern could reflect several mechanisms: individuals with better credit access may feel more economically secure and less drawn to populist appeals; they may have different information environments or social influences; or credit access may proxy for unmeasured aspects of local economic dynamism. Our data cannot distinguish among these possibilities.

The finding that friending bias---the tendency toward within-class friendship---is strongly associated with Republican voting is consistent with theories linking social insularity to partisan extremism. When social networks are class-segregated, individuals may have less exposure to cross-cutting perspectives, potentially contributing to more polarized political attitudes. However, we emphasize that this is a correlation, and reverse causality (political sorting leading to homophilous networks) is equally plausible.

Our findings also speak to the broader literature on place-based economic distress and political outcomes. The trade shock literature, exemplified by \citet{rodriguez2018revenge}, documents how economic decline in certain regions has contributed to political disaffection and support for populist movements. Our evidence on delinquency and GOP shifts is consistent with this narrative: counties experiencing greater financial distress---as measured by higher delinquency rates---shifted more strongly toward Republicans between 2016 and 2024. This suggests that credit market outcomes may serve as a useful proxy for the broader economic stress that shapes political behavior.

The strong positive correlation between credit scores and economic connectedness ($r = 0.82$) is perhaps our most striking finding. This suggests that places with good financial health are also places with strong bridging social capital. While we cannot establish causality, this correlation is consistent with multiple mechanisms: financially secure individuals may have more time and resources to form diverse social networks; cross-class networks may provide information and connections that improve financial outcomes; or both may be driven by common underlying factors like educational institutions, local labor market conditions, or historical patterns of investment and development.

\subsection{Methodological Considerations}

Several methodological issues deserve discussion. First, our standard errors do not account for spatial autocorrelation in county-level outcomes. Counties are not independent observations---neighboring counties share economic conditions, social networks, and political cultures. Future work should employ spatial econometric methods such as Conley standard errors or state-level clustering to provide more conservative inference. While such corrections would likely widen our confidence intervals, the strong associations we document would likely remain statistically meaningful.

Second, the high correlation between credit scores and delinquency rates ($r \approx -0.98$) raises concerns about multicollinearity in models including both variables. These measures capture related but distinct aspects of credit market health: credit scores reflect creditworthiness and access to favorable terms, while delinquency captures financial distress and inability to meet obligations. Future work could use factor analysis or principal components to create composite credit health indices that avoid these collinearity concerns.

Third, our analysis focuses on levels and changes in Republican vote share, but alternative political outcomes could yield different patterns. Turnout, third-party voting, primary polarization, and down-ballot races may show different relationships with credit and social capital. Understanding these alternative margins is an important direction for future research.

\subsection{Limitations}

Several additional limitations deserve emphasis. First, this is a correlational study. We do not identify causal effects, and the relationships we document could reflect reverse causality or omitted variables. Second, our county-level analysis is subject to ecological fallacy: individual-level relationships may differ from the county-level patterns we observe. Third, our credit and social capital measures are from specific years (2018-2020) and may not capture longer-term patterns or recent changes. Fourth, we focus on presidential voting, which may not capture local political engagement or policy preferences. Fifth, our measures are aggregated to the county level, which obscures important within-county heterogeneity, particularly in large metropolitan counties that contain both affluent and distressed neighborhoods.

\section{Conclusion}

This paper documents the geographic clustering of credit access, cross-class social networks, and political polarization across U.S. counties. We find that credit scores and economic connectedness are extraordinarily correlated ($r = 0.82$), suggesting that financial and social opportunity structures are deeply intertwined. After controlling for socioeconomic fundamentals, higher credit access is associated with less Republican voting, while counties with higher delinquency and lower college attainment experienced the largest shifts toward Republicans from 2016--2024.

Our county typology reveals a complex mosaic of American places: affluent coastal metros with high credit and high social capital voting Democratic; struggling interior counties with low credit and low social capital voting Republican; and intriguing ``Rural Connected'' counties with strong social networks but poor credit access voting Republican. These patterns suggest that understanding America's political divide requires examining both economic opportunity structures and social network configurations.

The policy implications of our findings are speculative but potentially significant. If credit access and social capital do causally influence political attitudes, then policies aimed at improving financial inclusion or fostering cross-class social connections could have political consequences beyond their direct economic effects. Community development financial institutions (CDFIs), credit counseling programs, and place-based policies targeting economically distressed areas could potentially affect political dynamics in addition to their primary economic goals. However, we emphasize that these policy implications remain speculative in the absence of causal identification.

Future research should investigate the causal mechanisms linking credit, social capital, and political behavior. Does improving credit access reduce support for populist politics? Do bridging social networks moderate political attitudes? How do these relationships differ across demographic groups and community types? Answering these questions will require research designs that can isolate causal effects, such as policy shocks affecting credit access or quasi-experimental variation in social network formation. Promising identification strategies might include bank branch closures or openings, state-level consumer credit regulations, Community Reinvestment Act enforcement changes, or natural experiments affecting social network formation such as school desegregation or public housing policies.

Our descriptive evidence suggests these are fruitful directions for understanding America's divided geography. The tight co-movement of credit access, social connectivity, and political polarization across space suggests that these are not independent phenomena but rather interconnected dimensions of local economic and social conditions. Understanding these connections is essential for both academic analysis of political economy and for policy efforts to address regional inequality and political polarization in the United States.

\section*{Acknowledgements}

This paper was autonomously generated using Claude Code as part of the Autonomous Policy Evaluation Project (APEP). The analysis uses publicly available data from Opportunity Insights and the MIT Election Data and Science Lab. We thank these organizations for making their data accessible for research.

\noindent\textbf{Project Repository:} \url{https://github.com/SocialCatalystLab/auto-policy-evals}

\noindent\textbf{Replication Materials:} Available at the project repository.

\label{apep_main_text_end}
\newpage

\begin{thebibliography}{99}

\bibitem[Chetty et al.(2014)]{chetty2014land}
Chetty, R., Hendren, N., Kline, P., \& Saez, E. (2014). Where is the land of opportunity? The geography of intergenerational mobility in the United States. \textit{Quarterly Journal of Economics}, 129(4), 1553--1623.

\bibitem[Chetty et al.(2018)]{chetty2018opportunity}
Chetty, R., Friedman, J. N., Hendren, N., Jones, M. R., \& Porter, S. R. (2018). The Opportunity Atlas: Mapping the childhood roots of social mobility. \textit{NBER Working Paper No. 25147}.

\bibitem[Chetty et al.(2022a)]{chetty2022social1}
Chetty, R., Jackson, M. O., Kuchler, T., Stroebel, J., et al. (2022). Social capital I: Measurement and associations with economic mobility. \textit{Nature}, 608, 108--121.

\bibitem[Chetty et al.(2022b)]{chetty2022social2}
Chetty, R., Jackson, M. O., Kuchler, T., Stroebel, J., et al. (2022). Social capital II: Determinants of economic connectedness. \textit{Nature}, 608, 122--134.

\bibitem[Chetty et al.(2023)]{chetty2023diversifying}
Chetty, R., Deming, D. J., \& Friedman, J. N. (2023). Diversifying society's leaders? The determinants and causal effects of admission to highly selective private colleges. \textit{NBER Working Paper No. 31492}.

\bibitem[Cramer(2016)]{cramer2016politics}
Cramer, K. J. (2016). \textit{The Politics of Resentment: Rural Consciousness in Wisconsin and the Rise of Scott Walker}. University of Chicago Press.

\bibitem[Guillen et al.(2024)]{guillen2024social}
Guillen, J., Pratt, K., \& Schwartz, S. (2024). Social capital and political polarization in the United States. \textit{The Annals of the American Academy of Political and Social Science}.

\bibitem[Putnam(2000)]{putnam2000bowling}
Putnam, R. D. (2000). \textit{Bowling Alone: The Collapse and Revival of American Community}. Simon \& Schuster.

\bibitem[Rodr{\'i}guez-Pose(2018)]{rodriguez2018revenge}
Rodr{\'i}guez-Pose, A. (2018). The revenge of the places that don't matter (and what to do about it). \textit{Cambridge Journal of Regions, Economy and Society}, 11(1), 189--209.

\end{thebibliography}

\newpage
\appendix

\section{Data Appendix}

\subsection{Data Sources and Access}

\textbf{Credit Access Data:} Downloaded from Opportunity Insights (\url{https://opportunityinsights.org/data/}) in January 2026. Files include county-level averages for credit scores, student loans, mortgages, auto loans, credit cards, and delinquency rates, all measured in 2020.

\textbf{Social Capital Data:} Downloaded from Humanitarian Data Exchange (HDX) hosting of the Chetty et al. (2022) Social Capital II replication data. County-level measures include economic connectedness, friending bias, and clustering, derived from de-identified Facebook friendship data.

\textbf{Voting Data:} Compiled from MIT Election Data and Science Lab county returns and supplementary sources for 2024 data. Files provide vote counts by party for 2016, 2020, and 2024 presidential elections.

\textbf{County Covariates:} From Opportunity Atlas and Changing Opportunity study. Variables include decennial census measures (income, education, demographics) and constructed covariates.

\subsection{Variable Construction}

\textbf{Credit Score (Z):} Average credit score standardized to mean 0, standard deviation 1.

\textbf{Economic Connectedness (Z):} EC measure standardized to mean 0, standard deviation 1.

\textbf{GOP Vote Share:} Republican votes divided by total votes.

\textbf{GOP Change 2016--2024:} GOP vote share in 2024 minus GOP vote share in 2016.

\textbf{County Type:} Eight-way classification based on above/below median on credit score, economic connectedness, and GOP vote share 2020.

\section{Additional Figures}

\begin{figure}[H]
\centering
\includegraphics[width=0.9\textwidth]{figures/figA1_student_loan_map.pdf}
\caption{Student Loan Geography}
\label{fig:student_loan_app}
\end{figure}

\begin{figure}[H]
\centering
\includegraphics[width=0.9\textwidth]{figures/figA2_mortgage_map.pdf}
\caption{Mortgage Balance Geography}
\label{fig:mortgage_app}
\end{figure}

\begin{figure}[H]
\centering
\includegraphics[width=0.9\textwidth]{figures/figA3_credit_card_map.pdf}
\caption{Credit Card Balance Geography}
\label{fig:credit_card_app}
\end{figure}

\begin{figure}[H]
\centering
\includegraphics[width=0.9\textwidth]{figures/figA4_auto_loan_map.pdf}
\caption{Auto Loan Balance Geography}
\label{fig:auto_loan_app}
\end{figure}

\begin{figure}[H]
\centering
\includegraphics[width=0.9\textwidth]{figures/figA5_friending_bias_map.pdf}
\caption{Friending Bias Geography}
\label{fig:friending_bias_app}
\end{figure}

\begin{figure}[H]
\centering
\includegraphics[width=\textwidth]{figures/figA6_urban_rural.pdf}
\caption{Urban-Rural Divide: Key Variables by Population Density Quartile}
\label{fig:urban_rural_app}
\end{figure}

\begin{figure}[H]
\centering
\includegraphics[width=\textwidth]{figures/figA7_regional.pdf}
\caption{Regional Variation in Credit, Social Capital, and Politics}
\label{fig:regional_app}
\end{figure}

\begin{figure}[H]
\centering
\includegraphics[width=0.85\textwidth]{figures/figA8_correlation_heatmap.pdf}
\caption{Correlation Matrix: Credit, Social Capital, and Political Variables}
\label{fig:corr_heatmap_app}
\floatfoot{\textit{Notes}: Correlations computed on N=2,977 counties with complete data on all 12 variables. Correlations differ slightly from main-text figures (e.g., Figure 8), which use pairwise-complete samples.}
\end{figure}

\begin{figure}[H]
\centering
\includegraphics[width=0.85\textwidth]{figures/figA9_shift_decomposition.pdf}
\caption{Political Shifts: 2016--2020 vs 2020--2024}
\label{fig:shift_decomp_app}
\end{figure}

\begin{figure}[H]
\centering
\includegraphics[width=0.85\textwidth]{figures/figA10_coefficient_stability.pdf}
\caption{Coefficient Stability: Adding Controls}
\label{fig:coef_stability_app}
\end{figure}

\end{document}
