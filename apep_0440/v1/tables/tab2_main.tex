\begin{table}[htbp]
\centering
\caption{RDD Estimates at Social Insurance Eligibility Thresholds}
\label{tab:main}
\begin{tabular}{lcc}
\toprule
 & Age 62 (Social Security) & Age 65 (Medicare) \\
\midrule
\textit{Part-Time} & 0.44 & 3.12*** \\
 & (0.66) & (0.38) \\
 & [-0.86, 1.74] & [2.37, 3.87] \\
[0.3em]
\textit{Overqualified} & 0.25*** & 0.10 \\
 & (0.07) & (0.09) \\
 & [0.12, 0.38] & [-0.08, 0.28] \\
[0.3em]
\textit{Earnings Mismatch} & -0.25 & 0.48 \\
 & (0.23) & (0.45) \\
 & [-0.71, 0.21] & [-0.40, 1.36] \\
[0.3em]
\textit{Involuntary PT} & 0.44 & 1.82*** \\
 & (0.26) & (0.29) \\
 & [-0.06, 0.95] & [1.25, 2.38] \\
[0.3em]
\textit{Composite Index} & 0.47 & 3.01*** \\
 & (0.45) & (0.41) \\
 & [-0.40, 1.34] & [2.21, 3.81] \\
\midrule
Bandwidth & 5.0 & 5.0 \\
Observations & 556,121 & 420,705 \\
Kernel & Uniform & Uniform \\
\bottomrule
\multicolumn{3}{p{0.85\textwidth}}{\footnotesize \textit{Notes:} 
Each cell reports the RDD estimate (in percentage points) from a local linear regression using the 
\texttt{fixest} package with a bandwidth of 5 age-years. 
Standard errors (in parentheses) clustered at the age level (11 clusters per threshold). 
95\% confidence intervals in brackets. 
* $p < 0.10$, ** $p < 0.05$, *** $p < 0.01$.} \\
\end{tabular}
\end{table}
