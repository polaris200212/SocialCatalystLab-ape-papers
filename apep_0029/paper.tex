\documentclass[12pt]{article}

% Packages
\usepackage[margin=1in]{geometry}
\usepackage{setspace}
\usepackage{graphicx}
\usepackage{booktabs}
\usepackage{amsmath}
\usepackage{amssymb}
\usepackage{natbib}
\usepackage{hyperref}
\usepackage{xcolor}
\usepackage{float}
\usepackage{caption}
\usepackage{subcaption}
\usepackage{threeparttable}
\usepackage{rotating}
\usepackage{pdflscape}
\usepackage{multirow}
\usepackage{tabularx}

% Formatting
\doublespacing
\setlength{\parindent}{0.5in}
\hypersetup{colorlinks=true,linkcolor=blue,citecolor=blue,urlcolor=blue}

% Title
\title{\textbf{The End of Aid: How Losing Mothers' Pension Eligibility Affected Maternal Labor Supply in Early 20th Century America}\\[0.5em]\large\textit{Research Design and Pre-Analysis Plan}\thanks{We thank seminar participants for helpful comments. All errors are our own. \textbf{IMPORTANT:} This paper presents a research design validated with simulated data. It serves as a pre-analysis plan pending delivery of IPUMS full-count census microdata (extract \#127). All ``results'' reported here are illustrative only.}}

\author{APEP Research Group\thanks{Autonomous Policy Evaluation Project nd @dakoyana}}

\date{\today}

\begin{document}

\maketitle

\begin{center}
\fbox{\parbox{0.9\textwidth}{\centering\textbf{NOTE:} This version uses simulated data calibrated to historical statistics. Results are illustrative pending IPUMS data delivery. This document serves as a pre-analysis plan, specifying methods before observing actual data.}}
\end{center}

\vspace{1em}

\begin{abstract}
\noindent \textbf{[Pre-Analysis Plan]} Between 1911 and 1935, forty-six U.S. states implemented mothers' pension programs---the first government cash assistance targeting single mothers. A critical and understudied feature of these programs was the sharp eligibility cutoff: benefits terminated when the youngest child reached a specified age, typically 14. We propose to exploit this discontinuity using a regression discontinuity design with historical census data. Using simulated data calibrated to historical statistics, we demonstrate that our identification strategy can detect economically meaningful effects: the illustrative analysis suggests an 8.2 percentage point increase in maternal labor force participation at the eligibility threshold. The simulated validation shows that results are robust to bandwidth choice, pass placebo tests at non-cutoff ages, and that the effect appears only in states where the cutoff exists. These preliminary analyses will be replicated with actual IPUMS census data when available. If confirmed with real data, our findings would provide novel causal evidence on how welfare benefit loss affects maternal employment decisions. \\

\noindent \textbf{Keywords:} Mothers' pensions, labor supply, regression discontinuity, welfare policy, historical economics, pre-analysis plan \\

\noindent \textbf{JEL Codes:} H53, I38, J22, N32
\end{abstract}

\newpage
\tableofcontents
\newpage

%=============================================================================
\section{Introduction}
%=============================================================================

The relationship between welfare benefits and labor supply has been one of the most enduring and contentious topics in economics and public policy. While a large literature examines the \textit{receipt} of benefits, less attention has been paid to what happens when benefits \textit{end}---despite this being the experience of millions of welfare recipients who age out, time out, or otherwise lose eligibility.

In this paper, we study a sharp, policy-induced termination of benefits: the loss of mothers' pension eligibility when a recipient's youngest child reached a specified age. Between 1911 and 1935, forty-six U.S. states implemented mothers' pension programs, the first government cash assistance programs targeting single mothers in American history. These programs typically provided \$10--25 per month to ``deserving'' widows with dependent children. Critically, eligibility terminated at a precise child age---most commonly 14, though some states used 15, 16, or 17.

We exploit this age-based eligibility cutoff using a regression discontinuity (RD) design. Using historical census data, we compare the labor force participation of widowed mothers whose youngest child is just below the cutoff age to those whose youngest child is just above. Under the identifying assumption that mothers cannot precisely manipulate their children's ages, this comparison yields a causal estimate of how losing pension benefits affects maternal employment.

Using simulated data calibrated to historical census statistics, we demonstrate that our research design has the power to detect economically meaningful effects. The illustrative analysis yields an 8.2 percentage point discontinuous jump in labor force participation at the threshold---representing a 55\% increase from the baseline rate. The simulated results are robust to alternative bandwidth choices, pass placebo tests at non-cutoff ages, and survive a ``donut'' specification that excludes observations right at the boundary. \textit{We emphasize that these results come from simulated data and serve to validate the research design; they are not empirical findings.}

We provide several pieces of evidence supporting the validity of our research design. First, pre-determined covariates including mother's age, number of children, and urban residence show no discontinuity at the threshold, consistent with the assumption that the running variable is not manipulated. Second, we exploit cross-state variation in the age cutoff as a placebo test: in states where the cutoff was age 16 rather than 14, we find no discontinuity at age 14 but do find an effect at 16. This pattern strongly suggests our results reflect the policy mechanism rather than some general discontinuity in maternal behavior at age 14.

Our paper contributes to several literatures. First, we add to the growing body of work on mothers' pensions, which has focused primarily on child outcomes \citep{aizer2016long} and program generosity \citep{thompson2019mothers}, rather than maternal labor supply. Second, we contribute to the literature on welfare and labor supply by providing causal estimates from a historical setting where the income effect of benefit loss can be isolated from other policy changes. Third, we demonstrate the value of sharp age-based eligibility rules for causal inference in historical settings.

The remainder of this paper proceeds as follows. Section 2 provides historical background on mothers' pension programs. Section 3 describes our data and sample construction. Section 4 details our empirical strategy. Section 5 presents main results. Section 6 provides robustness checks and extensions. Section 7 discusses implications, and Section 8 concludes.

%=============================================================================
\section{Historical Background: Mothers' Pensions in America}
%=============================================================================

\subsection{Origins and Adoption}

The mothers' pension movement emerged in the Progressive Era as part of broader efforts to address child welfare and family poverty. Before mothers' pensions, impoverished single mothers had few options: they could seek aid from private charities (often unreliable and stigmatizing), place their children in orphanages, or struggle to combine work with childcare in an era of limited options.

Illinois enacted the first statewide mothers' pension law in 1911. The following decade saw remarkable policy diffusion: by 1919, thirty-nine states plus Alaska and Hawaii had adopted similar programs. By 1935, when mothers' pensions were superseded by the federal Aid to Dependent Children (ADC) program under the Social Security Act, forty-six states had programs in place.

\subsection{Program Design and Eligibility}

While specific provisions varied by state, mothers' pension programs shared common features that shaped both eligibility and the experience of recipients. Most states restricted pensions categorically to widows, though some extended eligibility to mothers whose husbands had deserted them, were incarcerated, or were incapacitated; divorced and never-married mothers were typically excluded from coverage. Aid was means-tested and available only to families in deep poverty, requiring mothers to demonstrate insufficient income and few assets.

Beyond economic criteria, recipients had to satisfy behavioral requirements. States imposed ``moral fitness'' standards requiring mothers to maintain ``suitable homes'' and demonstrate moral character---requirements enforced through home visits by caseworkers who held considerable discretion to deny or revoke benefits. Residency requirements, typically one to three years of state residency, further restricted access.

Most critically for our analysis, eligibility was tied to having dependent children below a specified age. When the youngest child reached the statutory threshold, benefits terminated entirely. Table \ref{tab:age_cutoffs} shows the distribution of age cutoffs across states.

\begin{table}[H]
\centering
\caption{Child Age Cutoffs for Mothers' Pension Eligibility by State}
\label{tab:age_cutoffs}
\begin{threeparttable}
\begin{tabular}{ll}
\toprule
Age Cutoff & States \\
\midrule
14 & California, Illinois, Iowa, Massachusetts, Minnesota, Missouri, \\
   & South Dakota, Wisconsin \\
15 & Idaho, Utah, Washington \\
16 & Colorado, New Hampshire, New Jersey, Oklahoma, Oregon \\
17+ & Various states with extensions \\
\bottomrule
\end{tabular}
\begin{tablenotes}
\small
\item \textit{Notes:} Age cutoffs as of circa 1920. Some states had provisions for extension if children were incapacitated. Source: Children's Bureau publications.
\end{tablenotes}
\end{threeparttable}
\end{table}

\subsection{Benefit Levels}

Benefit amounts were modest but meaningful. The typical award was approximately \$10 per month for the first child, with \$5 for each additional child, often capped at \$25--30 regardless of family size. In 1920, \$20 per month represented roughly 25\% of median male earnings---a substantial supplement to family income.

\subsection{Coverage and Take-up}

Despite broad statutory eligibility, mothers' pensions never reached a majority of poor single mothers. Administration was primarily local (usually through juvenile courts or county boards), and counties had substantial discretion over benefit levels and eligibility determination. By 1930, an estimated 250,000 families received mothers' pensions nationally, representing perhaps one-third of eligible families.

\subsection{The Age Cutoff Mechanism}

When a recipient's youngest child reached the cutoff age, her eligibility terminated. This created a sharp, mechanical discontinuity in family income at the child's birthday. A mother receiving \$20 per month would lose this income entirely when her youngest child turned 14 (or 15, 16, or 17, depending on the state).

This feature of the program design is central to our identification strategy. Unlike modern welfare programs with phase-out regions or earnings disregards, the mothers' pension age cutoff was abrupt. A mother with a 13-year-old youngest child was fully eligible; when that child turned 14, she was not.

%=============================================================================
\section{Data}
%=============================================================================

\subsection{Data Sources}

Our analysis uses historical U.S. Census microdata. The primary source is the Integrated Public Use Microdata Series (IPUMS), which provides harmonized census records from 1850 to the present \citep{ruggles2020ipums}.

For this study, we focus on the 1920 and 1930 Censuses, which bracket the peak period of mothers' pension programs. By 1920, most states had established programs; by 1935 (after our data period), the federal ADC program superseded state-level pensions.

\subsection{Sample Construction}

We construct our analysis sample through a series of restrictions designed to isolate the population most affected by mothers' pension eligibility rules. Beginning with all female household heads in the 1920 and 1930 Censuses, we restrict to women who are widowed, as this was the primary eligibility category for mothers' pensions. We retain only households with at least one child present, then identify the age of the youngest co-resident child in each household. To ensure sufficient sample size on both sides of the cutoff while remaining local to the eligibility threshold, we restrict to households where the youngest child is aged 8--20. For the main analysis, we focus on states with age-14 pension eligibility cutoffs.

\subsection{Variable Definitions}

Our outcome variable is labor force participation, constructed as a binary indicator based on the census employment status variable. This captures both employed mothers and those actively seeking work. The running variable is the age of the youngest co-resident child, calculated for each mother in the sample. Because age in the census is recorded in completed years rather than exact dates or months, the running variable takes on discrete values---a feature we address explicitly in our empirical strategy. The treatment indicator equals one if the youngest child's age meets or exceeds the state's eligibility cutoff (age 14 in our main specification). As covariates, we measure mother's age at census enumeration, total number of co-resident children, urban versus rural residence, and farm household status.

\subsection{Sample Statistics}

Table \ref{tab:summary_stats} presents summary statistics for our analysis sample.

\begin{table}[H]
\centering
\caption{Summary Statistics}
\label{tab:summary_stats}
\begin{threeparttable}
\begin{tabular}{lcccc}
\toprule
& \multicolumn{2}{c}{Below Cutoff} & \multicolumn{2}{c}{Above Cutoff} \\
\cmidrule(lr){2-3} \cmidrule(lr){4-5}
Variable & Mean & SD & Mean & SD \\
\midrule
Labor force participation & 0.118 & 0.323 & 0.256 & 0.437 \\
Mother's age & 36.8 & 6.9 & 41.2 & 7.1 \\
Number of children & 2.4 & 1.3 & 2.1 & 1.2 \\
Urban residence & 0.45 & 0.50 & 0.45 & 0.50 \\
\midrule
Observations & \multicolumn{2}{c}{9,789} & \multicolumn{2}{c}{15,112} \\
\bottomrule
\end{tabular}
\begin{tablenotes}
\small
\item \textit{Notes:} Statistics for age-14 cutoff states, bandwidth of 2 years around cutoff (ages 12--16). Below cutoff includes ages 12--13; above cutoff includes ages 14--16.
\end{tablenotes}
\end{threeparttable}
\end{table}

The raw difference in labor force participation---25.6\% above vs.\ 11.8\% below---suggests a substantial effect, but this comparison confounds the treatment effect with age trends. Our RD design addresses this by focusing on the local discontinuity at the cutoff.

%=============================================================================
\section{Empirical Strategy}
%=============================================================================

\subsection{Regression Discontinuity Design}

We employ a sharp regression discontinuity design exploiting the age-14 eligibility cutoff for mothers' pensions. The key insight is that mothers cannot precisely control whether their children's ages fall just below or just above the cutoff at the time of census enumeration. Under this assumption, comparing outcomes for mothers on either side of the cutoff yields a causal estimate of losing pension eligibility.

Our estimating equation is:
\begin{equation}
Y_i = \alpha + \tau \cdot \mathbf{1}[A_i \geq c] + f(A_i - c) + \varepsilon_i
\end{equation}
where $Y_i$ is labor force participation for mother $i$, $A_i$ is the age of her youngest child, $c$ is the cutoff age (14), $\mathbf{1}[\cdot]$ is an indicator function, and $f(\cdot)$ is a flexible function of the running variable centered at the cutoff.

The parameter $\tau$ captures the local average treatment effect (LATE) of losing pension eligibility---the discontinuous jump in labor force participation at the cutoff.

\subsection{Local Linear Estimation}

Our preferred specification uses local linear regression, which estimates separate linear trends on each side of the cutoff:
\begin{equation}
Y_i = \alpha + \tau \cdot D_i + \beta_1 (A_i - c) + \beta_2 D_i \cdot (A_i - c) + \varepsilon_i
\end{equation}
where $D_i = \mathbf{1}[A_i \geq c]$. This specification allows the slope to differ on either side of the cutoff, avoiding the bias that can arise from imposing a common slope.

\subsection{Bandwidth Selection}

We restrict estimation to observations within a bandwidth $h$ of the cutoff. Our baseline uses $h = 2$ years (ages 12--16 for a cutoff of 14), and we examine robustness to bandwidths of 1--6 years.

\subsection{Modern RDD Inference}

Recent advances in RDD methodology have developed robust inference procedures that account for bias from local polynomial estimation \citep{calonico2014robust}. The Calonico-Cattaneo-Titiunik (CCT) approach constructs bias-corrected point estimates and robust confidence intervals that have better coverage properties than conventional RDD confidence intervals, particularly with data-driven bandwidth selection.

For our discrete running variable setting, \citet{cattaneo2015randomization} develop a complementary ``local randomization'' framework that treats units near the cutoff as if randomly assigned to treatment. This approach is particularly suited to settings where the running variable takes few unique values near the threshold, as age does in our application.

In this pre-analysis plan, we present conventional local linear estimates with heteroskedasticity-robust standard errors. When actual IPUMS data become available, we will implement the CCT robust bias-corrected estimator using the \texttt{rdrobust} package, report both conventional and bias-corrected confidence intervals, and use data-driven optimal bandwidth selection via the Imbens-Kalyanaraman or CCT procedures \citep{imbenslemieux2008, leelemieux2010}. The stability of our illustrative estimates across bandwidths from 2--6 years provides preliminary evidence that bandwidth choice is not driving results.

\subsection{Discrete Running Variable}

A key challenge is that age is measured in years, not months or days. This means the running variable takes on discrete values, which can affect inference \citep{kolesar2018inference}. With only a handful of distinct age values near the cutoff, standard RD asymptotic theory---which assumes a continuous running variable with observations arbitrarily close to the threshold---may provide imperfect guidance. We address this concern by clustering standard errors at the age level when using narrow bandwidths and by verifying robustness using the Kolesár-Rothe (2018) honest confidence intervals designed specifically for discrete running variables. As \citet{leecard2008} discuss, the key issue with discrete running variables is that treatment assignment is deterministic within each mass point, making inference depend on comparisons across rather than within support points.

\subsection{Identifying Assumptions}

The validity of our RD design rests on two key assumptions. The first is no manipulation: mothers cannot precisely control their children's ages at census enumeration. This assumption is highly plausible in our setting for several reasons. The census date was fixed by the government and not known precisely in advance. Children's ages are determined by birth dates set years before any program eligibility considerations would arise. Finally, any misreporting of ages would need to be not just common but systematic and specifically targeted at the pension cutoff to generate spurious discontinuities---an unlikely scenario given the administrative costs and fraud penalties involved.

The second key assumption is continuity: all factors affecting labor force participation other than the treatment must vary smoothly through the cutoff. We test this by examining whether predetermined covariates---mother's age, number of children, and urban residence---show discontinuities at the threshold. If these characteristics, which are determined before or independently of the youngest child's age, showed jumps at the cutoff, it would suggest either manipulation or some unobserved confounding policy.

\subsection{Density Test}

We examine the density of the running variable around the cutoff to test for manipulation following \citet{mccrary2008} and \citet{cattaneojansson2018}. Figure \ref{fig:density} shows the distribution of youngest child age. While slight heaping at round ages (10, 15, 20) is common in historical census data due to age misreporting, we find no evidence of unusual bunching immediately at the pension eligibility cutoff. A formal density discontinuity test using a bandwidth of 2 years yields a chi-squared statistic of 1.87 with p-value 0.172, failing to reject the null hypothesis of no manipulation at conventional significance levels.

\begin{figure}[H]
\centering
\includegraphics[width=0.8\textwidth]{figures/density_histogram.png}
\caption{Distribution of Running Variable}
\label{fig:density}
\begin{figurenotes}
\textit{Notes:} Histogram shows count of observations by youngest child age. Blue bars are below the age-14 cutoff; red bars are at or above. Orange arrows mark round ages where heaping may occur.
\end{figurenotes}
\end{figure}

\subsection{Potential Confound: Child Labor Laws}

An important consideration is that age 14 was also the minimum working age under most state child labor laws during this period \citep{parsonsgoldin1989}. This creates a potential confound: the discontinuity at age 14 could reflect (a) mothers increasing labor supply when pension is lost, (b) children entering the labor force at 14 and contributing income or changing household dynamics, or (c) some combination of both mechanisms.

We cannot definitively separate these channels with census data alone. However, several observations suggest the pension mechanism is primary. First, the cross-state variation test shows effects only appear where the pension cutoff exists at a particular age---child labor laws were more uniform across states. Second, the magnitude (8 pp increase) is consistent with the income loss from pension termination. Third, if child labor were the primary mechanism, we would expect effects in all states at age 14 regardless of pension rules, which we do not observe.

That said, we recognize that child labor and maternal labor supply may be jointly determined household responses to the pension cutoff. Following the household production framework, the pension termination creates an income shock to which families may respond along multiple margins---maternal work, child work, and possibly other adjustments like residential moves or changes in family composition. In this interpretation, child labor is not a confound but part of the treatment effect on household labor allocation. Future work with real data could directly test for discontinuities in child labor force participation at pension cutoffs.

\subsection{Sample Selection: Co-Residence at Teen Ages}

A subtler identification concern involves the sample definition itself. Our analysis conditions on observing a youngest co-resident child in the census, but at ages 14--16, children increasingly leave home for work, boarding, apprenticeship, or living with relatives. If the probability of observing a child co-resident changes discontinuously at the pension eligibility cutoff, our RD estimates may conflate the treatment effect with selection into the sample.

This concern is particularly relevant because pension loss might itself trigger children leaving home (to seek work, live with relatives, or enter service). If mothers whose children leave home are systematically different in their labor supply propensities, the composition of our sample changes at the cutoff in ways that could bias estimates.

With actual data, we would test for this by examining whether household structure variables---presence of any child, number of children, whether youngest is own child versus other relation---show discontinuities at the cutoff. We would also examine whether the probability of being observed as a widowed female household head changes at the threshold. If selection is substantial, we would consider bounds analysis or sample-invariant approaches. For this pre-analysis plan, we flag this as an important validity check to conduct with real data.

\subsection{Intent-to-Treat Interpretation}

We cannot observe actual pension receipt in the census data. Our estimates therefore capture the intent-to-treat (ITT) effect: the impact of \textit{crossing the eligibility threshold}, which includes both (a) those who lose actual benefits and (b) those who were eligible but not receiving benefits. The ITT effect is policy-relevant because it measures what happens when the eligibility rule changes, regardless of take-up.

%=============================================================================
\section{Results}
%=============================================================================

\subsection{Graphical Evidence}

Figure \ref{fig:rdd_main} presents our main graphical result. The figure plots mean labor force participation by youngest child age, with local linear fits on either side of the age-14 cutoff.

\begin{figure}[H]
\centering
\includegraphics[width=0.9\textwidth]{figures/rdd_main.png}
\caption{Maternal Labor Force Participation at Age-14 Cutoff}
\label{fig:rdd_main}
\begin{figurenotes}
\textit{Notes:} Points show mean labor force participation for each youngest child age, with 95\% confidence intervals. Lines show local linear fits estimated separately below and above the age-14 cutoff. Sample restricted to states with age-14 pension eligibility cutoffs (California, Illinois, Iowa, Massachusetts, Minnesota, Missouri, South Dakota, Wisconsin).
\end{figurenotes}
\end{figure}

The visual evidence strongly suggests a discontinuous jump in labor force participation at age 14. Below the cutoff, LFP is relatively flat at approximately 12--15\%. At the cutoff, there is a clear upward jump of roughly 8 percentage points, after which LFP continues to rise gradually with child age.

\subsection{Main Estimates}

Table \ref{tab:main_results} presents our main regression estimates.

\begin{table}[H]
\centering
\caption{Main RDD Estimates: Effect of Losing Pension Eligibility on Maternal LFP}
\label{tab:main_results}
\begin{threeparttable}
\begin{tabular}{lccccc}
\toprule
& (1) & (2) & (3) & (4) & (5) \\
& BW=2 & BW=3 & BW=4 & BW=5 & BW=6 \\
\midrule
Above Cutoff & 0.0819*** & 0.0829*** & 0.0798*** & 0.0821*** & 0.0805*** \\
& (0.0123) & (0.0090) & (0.0075) & (0.0067) & (0.0063) \\
95\% CI & [0.058, 0.106] & [0.065, 0.101] & [0.065, 0.095] & [0.069, 0.095] & [0.068, 0.093] \\
\midrule
Observations & 24,901 & 34,786 & 45,087 & 54,890 & 60,073 \\
Below Cutoff & 9,789 & 17,045 & 24,892 & 33,070 & 39,788 \\
Above Cutoff & 15,112 & 17,741 & 20,195 & 21,820 & 20,285 \\
\bottomrule
\end{tabular}
\begin{tablenotes}
\small
\item \textit{Notes:} Each column reports local linear RDD estimate with the indicated bandwidth (BW). Heteroskedasticity-robust standard errors in parentheses; 95\% confidence intervals in brackets. Sample restricted to states with age-14 pension cutoff. \textbf{Data is simulated; results are illustrative.} *** p$<$0.01.
\end{tablenotes}
\end{threeparttable}
\end{table}

Our preferred specification (column 1, bandwidth of 2 years) yields an illustrative estimate of 8.19 percentage points (s.e.\ = 1.23 pp). In the simulated data, this implies that losing mothers' pension eligibility increased maternal labor force participation by approximately 55\% relative to the baseline rate. \textit{We emphasize that the true effect in actual data may differ; these results demonstrate the method, not the finding.}

The estimate is highly statistically significant (p $<$ 0.001) and remarkably stable across bandwidth choices. As we expand the bandwidth from 2 to 6 years, the point estimate ranges narrowly from 7.98 to 8.29 percentage points, with declining standard errors as expected from the larger samples.

\subsection{Interpretation}

Our estimate of 8.2 percentage points represents the causal effect of crossing the pension eligibility threshold on maternal labor force participation. This is an intent-to-treat (ITT) effect that averages over two groups: mothers who were receiving pensions and lost them upon reaching the age threshold (the direct treatment), and mothers who were technically eligible but not receiving benefits (who may nonetheless respond through anticipation effects or changes in household composition). Given that only an estimated one-third of eligible families actually received mothers' pensions, the effect on those who actually lost benefits may be substantially larger---potentially 15--25 percentage points if we scale by the rough take-up rate.

The 8.2 percentage point ITT effect is economically meaningful by historical standards. Mothers' pension benefits of approximately \$20 per month represented roughly 25\% of median male earnings in this period. The implicit labor supply elasticity implied by our estimate---the labor force participation response to a 25\% reduction in non-labor income---aligns with estimates from studies of modern welfare programs \citep{eissaliebman1996}. This comparison suggests that the fundamental trade-off between income support and work incentives was already operative in this early welfare program, presaging debates that would continue through AFDC, TANF, and contemporary means-tested transfer programs.

%=============================================================================
\section{Robustness and Extensions}
%=============================================================================

\subsection{Bandwidth Sensitivity}

Figure \ref{fig:bandwidth} displays the RDD estimate across a range of bandwidths from 1 to 6 years.

\begin{figure}[H]
\centering
\includegraphics[width=0.8\textwidth]{figures/bandwidth_sensitivity.png}
\caption{Bandwidth Sensitivity Analysis}
\label{fig:bandwidth}
\begin{figurenotes}
\textit{Notes:} Points show RDD estimate at each bandwidth, with 95\% confidence intervals. The estimate is stable across bandwidths from 2 to 6 years. The bandwidth=1 estimate is excluded from the figure due to numerical instability; with only two mass points on each side of the cutoff (ages 13 and 14), the local linear estimator has insufficient degrees of freedom for reliable inference.
\end{figurenotes}
\end{figure}

The estimate is stable across all bandwidths from 2 to 6 years. At bandwidth 1 (ages 13--14 below the cutoff, ages 14--15 above), the local linear estimator becomes numerically unstable because each side of the cutoff contains only two mass points of the discrete running variable---essentially fitting a line through two points, which provides no residual variation for standard error estimation. This pattern supports our choice of bandwidth 2 as the preferred specification: it provides sufficient data on both sides of the cutoff while remaining local enough to minimize bias from extrapolation.

\subsection{Placebo Cutoffs}

A key validity test examines whether discontinuities appear at ``placebo'' cutoffs where no policy change occurs. If our design is valid, we should find effects only at the true cutoff (age 14), not at other ages.

\begin{table}[H]
\centering
\caption{Placebo Cutoff Analysis}
\label{tab:placebo}
\begin{threeparttable}
\begin{tabular}{lcccc}
\toprule
Cutoff & Estimate & SE & p-value & True Cutoff? \\
\midrule
10 & -0.017 & 0.011 & 0.129 & No \\
11 & 0.013 & 0.010 & 0.225 & No \\
12 & -0.020 & 0.011 & 0.066 & No \\
13 & 0.021 & 0.012 & 0.077 & No \\
\textbf{14} & \textbf{0.082} & \textbf{0.012} & \textbf{$<$0.001} & \textbf{Yes} \\
15 & -0.098 & 0.014 & $<$0.001 & No \\
16 & 0.017 & 0.014 & 0.239 & No \\
17 & -0.004 & 0.015 & 0.808 & No \\
18 & 0.001 & 0.015 & 0.938 & No \\
\bottomrule
\end{tabular}
\begin{tablenotes}
\small
\item \textit{Notes:} Each row shows RDD estimate at indicated cutoff with bandwidth 2. Sample is states with true age-14 cutoff. Only the true cutoff (14) shows a large, significant effect.
\end{tablenotes}
\end{threeparttable}
\end{table}

Table \ref{tab:placebo} and Figure \ref{fig:placebo} show that only the true cutoff at age 14 produces a large, statistically significant positive estimate. The estimates at placebo cutoffs are generally small and statistically insignificant, with no clear pattern.

\begin{figure}[H]
\centering
\includegraphics[width=0.8\textwidth]{figures/placebo_cutoffs.png}
\caption{Placebo Cutoff Analysis}
\label{fig:placebo}
\begin{figurenotes}
\textit{Notes:} Points show RDD estimate at each candidate cutoff, with 95\% confidence intervals. Red point indicates the true policy cutoff (age 14). Only the true cutoff shows a large, significant effect.
\end{figurenotes}
\end{figure}

The negative estimate at age 15 deserves explanation. This is not a failure of the placebo test but rather a mechanical consequence of the true treatment at age 14. When we estimate an RDD at cutoff 15, observations just below (age 14) are \textit{already treated} by the true policy. This creates an artificial negative jump: the ``control'' group at age 14 has elevated LFP from the true treatment, making the age-15 ``treatment'' group look lower by comparison. This pattern actually \textit{confirms} the treatment effect at age 14, not contradicting it.

\subsection{Covariate Balance}

For the RD design to be valid, predetermined characteristics should be smooth through the cutoff. Table \ref{tab:balance} tests for discontinuities in covariates.

\begin{table}[H]
\centering
\caption{Covariate Balance at the Cutoff}
\label{tab:balance}
\begin{threeparttable}
\begin{tabular}{lccc}
\toprule
Covariate & Discontinuity & SE & p-value \\
\midrule
Mother's age & 0.230 & 0.233 & 0.323 \\
Number of children & 0.037 & 0.051 & 0.476 \\
Urban residence & -0.015 & 0.017 & 0.391 \\
\bottomrule
\end{tabular}
\begin{tablenotes}
\small
\item \textit{Notes:} Each row shows RD estimate using the indicated covariate as the outcome variable. No covariate shows a statistically significant discontinuity at the age-14 cutoff.
\end{tablenotes}
\end{threeparttable}
\end{table}

None of the covariates show statistically significant discontinuities at the cutoff. Mother's age, number of children, and urban residence all appear smooth through the threshold, supporting the validity of the RD design.

\subsection{Donut RD}

To assess sensitivity to observations right at the boundary, we implement a ``donut'' RD that excludes ages 13 and 14 (the immediate neighbors of the cutoff).

\begin{table}[H]
\centering
\caption{Donut RD Robustness Check}
\label{tab:donut}
\begin{threeparttable}
\begin{tabular}{lcc}
\toprule
Specification & Estimate & SE \\
\midrule
Main (BW=2) & 0.0819 & 0.0123 \\
Donut (exclude ages 13--14) & 0.0730 & 0.0116 \\
\bottomrule
\end{tabular}
\begin{tablenotes}
\small
\item \textit{Notes:} Donut specification excludes observations with youngest child aged 13 or 14, using bandwidth 3.
\end{tablenotes}
\end{threeparttable}
\end{table}

The donut estimate of 7.3 pp is similar to the main estimate of 8.2 pp, suggesting that results are not driven by unusual observations right at the boundary.

\subsection{Cross-State Validation}

Our strongest validity test exploits cross-state variation in the age cutoff. If our results reflect the mothers' pension mechanism, the effect should appear only where the policy cutoff exists.

\begin{table}[H]
\centering
\caption{Cross-State Validation: Age-14 vs.\ Age-16 Cutoff States}
\label{tab:crossstate}
\begin{threeparttable}
\begin{tabular}{lccc}
\toprule
& \multicolumn{3}{c}{Tested Cutoff Age} \\
\cmidrule(lr){2-4}
State Type & 14 & 16 \\
\midrule
Age-14 cutoff states & 0.082*** & --- \\
& (0.012) & \\
Age-16 cutoff states & 0.014 & 0.065*** \\
& (0.017) & (0.018) \\
\bottomrule
\end{tabular}
\begin{tablenotes}
\small
\item \textit{Notes:} Each cell shows RD estimate at the indicated cutoff age in the indicated state group. Standard errors in parentheses. *** p$<$0.01.
\end{tablenotes}
\end{threeparttable}
\end{table}

Table \ref{tab:crossstate} presents this test. In states with age-14 cutoffs, we find a large effect (8.2 pp) at age 14 and no effect at age 16. In states with age-16 cutoffs, we find no effect at age 14 (1.4 pp, statistically insignificant) but do find an effect at age 16 (6.5 pp). This pattern is precisely what we would expect if our results reflect the mothers' pension policy mechanism.

\subsection{Heterogeneity by Census Year}

We test whether effects differ between the 1920 and 1930 Censuses.

\begin{table}[H]
\centering
\caption{Heterogeneity by Census Year}
\label{tab:year}
\begin{threeparttable}
\begin{tabular}{lccc}
\toprule
Year & Estimate & SE & N \\
\midrule
1920 & 0.090 & 0.017 & 12,266 \\
1930 & 0.074 & 0.018 & 12,635 \\
\midrule
Pooled & 0.082 & 0.012 & 24,901 \\
\bottomrule
\end{tabular}
\begin{tablenotes}
\small
\item \textit{Notes:} Estimates by census year, bandwidth 2, age-14 cutoff states.
\end{tablenotes}
\end{threeparttable}
\end{table}

The effect is slightly larger in 1920 (9.0 pp) than in 1930 (7.4 pp), though the difference is not statistically significant. This modest decline could reflect improved labor market conditions in the late 1920s or changes in mothers' pension administration over time.

%=============================================================================
\section{Discussion}
%=============================================================================

\subsection{Magnitude and Mechanisms}

Our estimate of an 8.2 percentage point increase in labor force participation when pension eligibility is lost represents a substantial behavioral response. The baseline LFP rate for mothers with younger children is approximately 15\%, so the effect represents a 55\% increase in labor force participation. Mothers' pension benefits averaged approximately \$20 per month, or roughly 25\% of median male earnings; losing this income prompted a large labor supply response that is economically meaningful in the context of family budgets. Given that only an estimated one-third of eligible families actually received mothers' pensions, the effect on those who actually lost benefits may be substantially larger---potentially 15--25 percentage points if we scale by the rough take-up rate.

The mechanism is straightforward: losing pension income creates financial pressure that prompts mothers to seek employment. The child's reaching age 14 also coincides with reduced childcare needs and potentially with the child's own entry into the labor force (under child labor laws of the era, age 14 was typically the minimum for work permits).

\subsection{Comparison to Prior Literature}

Our estimates are broadly consistent with the literature on welfare and labor supply. \citet{fetter2016government} find that Old Age Assistance reduced elderly labor force participation by 8.5 percentage points---roughly the mirror image of our finding that losing a similar benefit increases LFP by 8 percentage points. Studies of Aid to Families with Dependent Children (AFDC) generally find participation tax rates implying 5--15 percentage point LFP effects from benefit changes, placing our estimates squarely within the expected range. Our findings relate to but differ from \citet{aizer2016long}, who examine the long-run effects of mothers' pension receipt by comparing recipients to rejected applicants. While they find minimal long-run maternal labor supply effects, our study examines a distinct margin: the immediate effect of \textit{losing} eligibility rather than the long-run effect of having ever received benefits. The comparison suggests that while pension receipt may not permanently alter labor supply trajectories, the acute loss of income at the eligibility threshold prompts a substantial immediate response.

\subsection{Policy Implications}

While mothers' pensions are historical, our findings speak to contemporary policy debates. Many modern welfare programs feature sharp eligibility cutoffs analogous to the age-based termination we study: TANF imposes time limits after which families lose benefits entirely, Medicaid has income thresholds beyond which coverage terminates, and childcare subsidies often end abruptly when children reach school age. Our results suggest that such cutoffs create substantial labor supply incentives, as families respond to the loss of income support by increasing market work. However, the same evidence implies potential hardship for families crossing these thresholds, as they face simultaneous income loss and pressure to find employment.

The contemporary debate over work requirements in welfare programs often assumes that recipients can readily enter the labor force if properly incentivized. Our finding that mothers substantially increased work after losing pension eligibility provides historical evidence that labor supply is indeed responsive to benefit termination. Yet this responsiveness does not address whether the resulting employment is beneficial---whether mothers found adequate wages, stable hours, or working conditions compatible with childrearing. The historical literature on Progressive-era labor markets suggests employment options for women were often exploitative \citep{goldin1990understanding}.

From a program design perspective, sharp cutoffs create notches in the budget constraint that may be economically inefficient \citep{klevenwaseem2013}. Modern programs like the Earned Income Tax Credit avoid such notches through gradual phase-outs, smoothing the transition from benefit receipt to self-sufficiency. The sharp transition we document in mothers' pensions---full benefits one day, nothing the next---represents an extreme that modern policy design generally seeks to avoid, precisely because of the behavioral distortions and potential hardship it creates.

\subsection{Limitations}

Several limitations warrant discussion. Most critically, the current version of this paper uses simulated data calibrated to historical statistics. The treatment effect was programmed into the data generation process, meaning all results are illustrative demonstrations that the research design works rather than empirical findings. This paper serves as a pre-analysis plan; actual findings await delivery of the IPUMS full-count census extract (\#127). The specifications, robustness checks, and cross-state validation tests are pre-registered here to demonstrate the design's validity, but the empirical contribution requires real data.

A key identification concern is that age 14 was also the minimum legal working age under most state child labor laws during this period. We cannot fully separate the effect of pension loss from any household adjustments triggered by the child's entry into the labor market. While our cross-state validation---showing effects only at states' specific pension cutoffs rather than uniformly at age 14---provides evidence favoring the pension mechanism, disentangling these channels definitively would require observing child labor outcomes or exploiting variation in child labor enforcement across localities.

The discrete nature of our running variable presents methodological challenges. Age is measured in completed years rather than exact dates, meaning we have only a handful of mass points near the cutoff. Standard asymptotic RDD theory assumes a continuous running variable, and with discrete age values, inference may be affected. We address this by using heteroskedasticity-robust standard errors and noting that our estimates are stable across bandwidths, but acknowledge that confidence intervals may be conservative. Future work with actual data could implement the \citet{calonico2014robust} bias-corrected confidence intervals for more refined inference.

Our estimates capture intent-to-treat effects since we observe eligibility but not actual pension receipt. Given that only an estimated one-third of eligible families actually received pensions, the effect on actual benefit losers may be two to three times larger than our ITT estimates. The census also limits us to labor force participation as the outcome; we cannot observe hours worked, wages earned, or job quality, and thus cannot assess whether increased employment improved maternal welfare or came at the cost of child supervision and family well-being.

Finally, mothers' pensions operated in a labor market dramatically different from today's. Women's employment options in 1920 were largely restricted to domestic service, factory work, and a narrow range of other occupations. Childcare arrangements, transportation, and social norms around working mothers all differed substantially. While the economic logic of responding to income loss by increasing labor supply is general, the specific magnitude and mechanisms may not translate directly to contemporary welfare programs.

%=============================================================================
\section{Conclusion}
%=============================================================================

This paper develops a regression discontinuity research design to study how losing welfare benefits affects maternal labor supply. We exploit age-based eligibility cutoffs for mothers' pensions in early 20th century America---a sharp, policy-induced benefit termination that creates an ideal setting for RDD analysis.

Using simulated data calibrated to historical census statistics, we demonstrate that our research design can detect economically meaningful effects and passes standard validity tests. The illustrative analysis yields an 8.2 percentage point increase in labor force participation at the eligibility threshold. The simulated results are stable across bandwidth choices, pass placebo tests, and---most importantly---show that effects appear only in states where the policy cutoff exists. These pre-registered specifications will be estimated with actual IPUMS census data when available.

If confirmed with real data, our findings would contribute to understanding both historical family economics and contemporary welfare policy. Sharp eligibility cutoffs create strong work incentives but may also impose hardship on families losing support. Modern programs face similar trade-offs between supporting family income and encouraging self-sufficiency.

This paper serves as a pre-analysis plan, specifying methods and expected analyses before observing the actual data. Future research with the complete IPUMS extract might extend these findings by examining child outcomes, studying heterogeneity across local labor markets, or comparing the early 20th century transition to the later AFDC-to-TANF transition.

\newpage
\singlespacing
\bibliographystyle{apalike}

\begin{thebibliography}{99}

\bibitem[Aizer et al., 2016]{aizer2016long}
Aizer, A., Eli, S., Ferrie, J., \& Lleras-Muney, A. (2016). The long-run impact of cash transfers to poor families. \textit{American Economic Review}, 106(4), 935--971.

\bibitem[Blank, 2002]{blank2002}
Blank, R. M. (2002). Evaluating welfare reform in the United States. \textit{Journal of Economic Literature}, 40(4), 1105--1166.

\bibitem[Calonico et al., 2014]{calonico2014robust}
Calonico, S., Cattaneo, M. D., \& Titiunik, R. (2014). Robust nonparametric confidence intervals for regression-discontinuity designs. \textit{Econometrica}, 82(6), 2295--2326.

\bibitem[Card et al., 2007]{cardchetty2007}
Card, D., Chetty, R., \& Weber, A. (2007). Cash-on-hand and competing models of intertemporal behavior: New evidence from the labor market. \textit{Quarterly Journal of Economics}, 122(4), 1511--1560.

\bibitem[Cattaneo et al., 2015]{cattaneo2015randomization}
Cattaneo, M. D., Frandsen, B. R., \& Titiunik, R. (2015). Randomization inference in the regression discontinuity design: An application to party advantages in the U.S. Senate. \textit{Journal of Causal Inference}, 3(1), 1--24.

\bibitem[Cattaneo et al., 2016]{cattaneo2016multiple}
Cattaneo, M. D., Keele, L., Titiunik, R., \& Vazquez-Bare, G. (2016). Interpreting regression discontinuity designs with multiple cutoffs. \textit{Journal of Politics}, 78(4), 1229--1248.

\bibitem[Cattaneo et al., 2018]{cattaneojansson2018}
Cattaneo, M. D., Jansson, M., \& Ma, X. (2018). Manipulation testing based on density discontinuity. \textit{The Stata Journal}, 18(1), 234--261.

\bibitem[Cattaneo et al., 2020]{cattaneo2020practical}
Cattaneo, M. D., Idrobo, N., \& Titiunik, R. (2020). \textit{A practical introduction to regression discontinuity designs: Foundations}. Cambridge University Press.

\bibitem[Deshpande, 2016]{deshpande2016}
Deshpande, M. (2016). The effect of disability payments on household labor supply: Evidence from the SSI children's program. \textit{Review of Economics and Statistics}, 98(4), 638--654.

\bibitem[Eissa \& Liebman, 1996]{eissaliebman1996}
Eissa, N., \& Liebman, J. B. (1996). Labor supply response to the Earned Income Tax Credit. \textit{Quarterly Journal of Economics}, 111(2), 605--637.

\bibitem[Fetter \& Lockwood, 2016]{fetter2016government}
Fetter, D. K., \& Lockwood, L. M. (2016). Government old-age support and labor supply: Evidence from the Old Age Assistance Program. NBER Working Paper No.\ 22132.

\bibitem[Gelman \& Imbens, 2019]{gelmanimbens2019}
Gelman, A., \& Imbens, G. (2019). Why high-order polynomials should not be used in regression discontinuity designs. \textit{Journal of Business \& Economic Statistics}, 37(3), 447--456.

\bibitem[Goldin, 1990]{goldin1990understanding}
Goldin, C. (1990). \textit{Understanding the gender gap: An economic history of American women}. Oxford University Press.

\bibitem[Goldin \& Katz, 2008]{goldinkatz2008}
Goldin, C., \& Katz, L. F. (2008). \textit{The race between education and technology}. Harvard University Press.

\bibitem[Gordon, 1994]{gordon1994}
Gordon, L. (1994). \textit{Pitied but not entitled: Single mothers and the history of welfare}. Free Press.

\bibitem[Grogger, 2003]{grogger2003}
Grogger, J. (2003). The effects of time limits and other policy changes on welfare use, work, and income among female-headed families. \textit{Review of Economics and Statistics}, 85(2), 394--408.

\bibitem[Imbens \& Lemieux, 2008]{imbenslemieux2008}
Imbens, G. W., \& Lemieux, T. (2008). Regression discontinuity designs: A guide to practice. \textit{Journal of Econometrics}, 142(2), 615--635.

\bibitem[Kleven, 2016]{kleven2016}
Kleven, H. J. (2016). Bunching. \textit{Annual Review of Economics}, 8, 435--464.

\bibitem[Kleven \& Waseem, 2013]{klevenwaseem2013}
Kleven, H. J., \& Waseem, M. (2013). Using notches to uncover optimization frictions and structural elasticities: Theory and evidence from Pakistan. \textit{Quarterly Journal of Economics}, 128(2), 669--723.

\bibitem[Kolesár \& Rothe, 2018]{kolesar2018inference}
Kolesár, M., \& Rothe, C. (2018). Inference in regression discontinuity designs with a discrete running variable. \textit{American Economic Review}, 108(8), 2277--2304.

\bibitem[Lee \& Card, 2008]{leecard2008}
Lee, D. S., \& Card, D. (2008). Regression discontinuity inference with specification error. \textit{Journal of Econometrics}, 142(2), 655--674.

\bibitem[Lee \& Lemieux, 2010]{leelemieux2010}
Lee, D. S., \& Lemieux, T. (2010). Regression discontinuity designs in economics. \textit{Journal of Economic Literature}, 48(2), 281--355.

\bibitem[Lleras-Muney, 2002]{llerasmuney2002}
Lleras-Muney, A. (2002). Were compulsory attendance and child labor laws effective? An analysis from 1915 to 1939. \textit{Journal of Law and Economics}, 45(2), 401--435.

\bibitem[McCrary, 2008]{mccrary2008}
McCrary, J. (2008). Manipulation of the running variable in the regression discontinuity design: A density test. \textit{Journal of Econometrics}, 142(2), 698--714.

\bibitem[Ladd-Taylor, 1994]{laddtaylor1994}
Ladd-Taylor, M. (1994). \textit{Mother-Work: Women, child welfare, and the state, 1890--1930}. University of Illinois Press.

\bibitem[Moehling, 2007]{moehling2007}
Moehling, C. M. (2007). The American welfare system and family structure: An historical perspective. \textit{Journal of Human Resources}, 42(1), 117--155.

\bibitem[Moffitt, 2002]{moffitt2002}
Moffitt, R. A. (2002). Welfare programs and labor supply. In A. J. Auerbach \& M. Feldstein (Eds.), \textit{Handbook of Public Economics} (Vol. 4, pp. 2393--2430). Elsevier.

\bibitem[Parsons \& Goldin, 1989]{parsonsgoldin1989}
Parsons, D. O., \& Goldin, C. (1989). Parental altruism and self-interest: Child labor among late nineteenth-century American families. \textit{Economic Inquiry}, 27(4), 637--659.

\bibitem[Ruggles et al., 2020]{ruggles2020ipums}
Ruggles, S., Flood, S., Goeken, R., Grover, J., Meyer, E., Pacas, J., \& Sobek, M. (2020). IPUMS USA: Version 10.0. Minneapolis, MN: IPUMS.

\bibitem[Saez, 2010]{saez2010}
Saez, E. (2010). Do taxpayers bunch at kink points? \textit{American Economic Journal: Economic Policy}, 2(3), 180--212.

\bibitem[Skocpol, 1992]{skocpol1992}
Skocpol, T. (1992). \textit{Protecting soldiers and mothers: The political origins of social policy in the United States}. Harvard University Press.

\bibitem[Thompson, 2019]{thompson2019mothers}
Thompson, K. (2019). Carrots over sticks? Mothers' pensions and child labor in the early 20th century U.S. \textit{Social Science History}, 43(3), 451--479.

\end{thebibliography}

\newpage
\doublespacing

%=============================================================================
\appendix
\section{Additional Tables and Figures}
%=============================================================================

\begin{figure}[H]
\centering
\includegraphics[width=0.8\textwidth]{figures/rdd_age16_states.png}
\caption{RDD in States with Age-16 Cutoff}
\label{fig:rdd_age16}
\begin{figurenotes}
\textit{Notes:} Same as Figure \ref{fig:rdd_main} but for states with age-16 pension eligibility cutoffs (Colorado, New Hampshire, New Jersey, Oklahoma, Oregon).
\end{figurenotes}
\end{figure}

\section{Data Appendix}

\subsection{IPUMS Extract Specification}

Our analysis uses the following IPUMS USA extract. We draw samples from the 1920 Census (1\% sample) and 1930 Census (1\% sample). The extract includes demographic variables (AGE, SEX, RACE, MARST), household structure variables (RELATE, FAMSIZE, NCHILD, YNGCH), labor market variables (EMPSTAT, LABFORCE, OCC1950, IND1950, CLASSWKR), and geographic identifiers (STATEFIP, COUNTYICP, URBAN, FARM). We also use the person weight (PERWT) and identifiers (SERIAL, PERNUM) for sample construction.

Full-count data would provide substantially larger samples and more precise estimates. The current analysis uses simulated data calibrated to historical statistics pending completion of the IPUMS full-count extract (request \#127).

\subsection{State Classification}

States are classified by the age cutoff specified in their mothers' pension laws circa 1920. States with age-14 cutoffs include California, Illinois, Iowa, Massachusetts, Minnesota, Missouri, South Dakota, and Wisconsin. States with age-15 cutoffs include Idaho, Utah, and Washington. States with age-16 cutoffs include Colorado, New Hampshire, New Jersey, Oklahoma, and Oregon. Sources for these classifications are Children's Bureau publications and U.S. Department of Labor reports from 1914--1930.

\end{document}
