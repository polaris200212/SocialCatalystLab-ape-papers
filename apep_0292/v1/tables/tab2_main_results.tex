\begin{table}[htbp]
\centering
\caption{Main RDD Results: Effect of LL84 Disclosure on Property Values}
\label{tab:main_results}
\small
\begin{tabular}{lcccc}
\hline\hline
 & (1) & (2) & (3) & (4) \\
 & Log Assessed & Log Assessed & Log Value & Log Assessed \\
 & Value & Value & per Sq Ft & Value \\
\hline
LL84 Disclosure & -0.0398 & -0.0282 & -0.0395 & 0.0052 \\
 & (0.0593) & (0.0683) & (0.0593) & (0.0211) \\
 & [-0.156, 0.076] & [-0.164, 0.108] & [-0.156, 0.076] & [-0.036, 0.047] \\
\hline
Polynomial & Linear & Quadratic & Linear & Linear (param.) \\
Kernel & Triangular & Triangular & Triangular & --- \\
Bandwidth & MSE-Optimal & MSE-Optimal & MSE-Optimal & Full Sample \\
Eff. $N$ & 3,740 & 5,596 & 3,740 & 18,627 \\
$p$-value & 0.591 & 0.825 & 0.591 & 0.805 \\
\hline\hline
\end{tabular}
\begin{minipage}{0.95\textwidth}
\vspace{0.3em}
\footnotesize \textit{Notes:} Columns (1)--(3) report local polynomial RDD estimates using \texttt{rdrobust} (Cattaneo, Idrobo, and Titiunik 2020). Standard errors in parentheses are bias-corrected robust; 95\% confidence intervals in brackets. Column (4) reports a parametric OLS estimate with borough fixed effects. The outcome in columns (1), (2), and (4) is log assessed total value; column (3) uses log assessed value per square foot. The running variable is building gross floor area with a cutoff at 25,000 sq ft. MSE-optimal bandwidths are selected independently for each specification using the procedure in Cattaneo and Vazquez-Bare (2020). The quadratic specification (column 2) selects a wider MSE-optimal bandwidth than the linear specification (column 1) because higher-order polynomials have lower bias but higher variance, so the MSE-optimal bandwidth increases to compensate---this is standard \texttt{rdrobust} behavior. ``Eff.\ $N$'' reports the number of observations within each specification's selected bandwidth.
\end{minipage}
\end{table}

