\documentclass[12pt]{article}

% UTF-8 encoding and fonts
\usepackage[utf8]{inputenc}
\usepackage[T1]{fontenc}
\usepackage{lmodern}

% Page setup
\usepackage[margin=1in]{geometry}
\usepackage{setspace}
\onehalfspacing

% Math and symbols
\usepackage{amsmath,amssymb}

% Graphics
\usepackage{graphicx}
\usepackage{float}

% Tables
\usepackage{booktabs}
\usepackage{array}
\usepackage{multirow}

% Bibliography
\usepackage{natbib}
\bibliographystyle{aer}

% Hyperlinks
\usepackage{hyperref}
\hypersetup{
    colorlinks=true,
    linkcolor=blue,
    citecolor=blue,
    urlcolor=blue
}

% Captions
\usepackage{caption}
\captionsetup{font=small,labelfont=bf}

% Section formatting
\usepackage{titlesec}
\titleformat{\section}{\large\bfseries}{\thesection.}{0.5em}{}
\titleformat{\subsection}{\normalsize\bfseries}{\thesubsection}{0.5em}{}

% Custom commands
\newcommand{\E}{\mathbb{E}}

\title{State Earned Income Tax Credits and Self-Employment Among Single Mothers: \\
Evidence of a Wage Employment Pull Effect}

\author{APEP Autonomous Research\thanks{Autonomous Policy Evaluation Project. This paper was produced autonomously using the APEP research pipeline. nd @dakoyana}}

\date{January 2026}

\begin{document}

\maketitle

\begin{abstract}
\noindent
This paper examines whether state Earned Income Tax Credits (EITCs) affect self-employment among single mothers. Using a difference-in-differences design exploiting staggered state EITC adoption from 2014--2023 and Current Population Survey microdata, I find that state EITCs \textit{reduce} self-employment among single mothers by approximately 1.3 percentage points---a 54 percent decline relative to the baseline rate. The effect is concentrated in incorporated self-employment and is significantly larger than the null effect observed for childless women (the placebo group). These findings suggest that state EITCs may pull single mothers away from self-employment and into wage employment, where EITC benefits are easier to claim and verify. However, significant pre-treatment coefficients raise concerns about parallel trends violations, and the results should be interpreted with caution. The negative self-employment effect implies that while EITCs may increase overall employment, they may inadvertently discourage entrepreneurship among their target population.
\end{abstract}

\vspace{1em}
\noindent\textbf{JEL Codes:} H24, J22, J16, L26 \\
\noindent\textbf{Keywords:} Earned Income Tax Credit, self-employment, single mothers, entrepreneurship, difference-in-differences

\newpage

\section{Introduction}

The Earned Income Tax Credit (EITC) is the largest cash transfer program for low-income workers in the United States, providing over \$60 billion annually to approximately 25 million families. A substantial literature documents that the EITC increases employment among single mothers---its primary target population---through its work incentive in the phase-in region of the credit schedule. However, recent work by Kleven (2024) challenges this consensus, arguing that much of the employment increase attributed to the EITC may instead reflect concurrent welfare reform and macroeconomic conditions.

This paper examines an understudied margin of the EITC's labor supply effects: self-employment. Theory provides ambiguous predictions. On one hand, the EITC subsidizes earned income regardless of source, and self-employment offers flexibility that may be particularly valuable to single mothers managing childcare responsibilities. On the other hand, self-employment income is harder to verify than wage income, potentially making EITC benefits more difficult to claim. Understanding how the EITC affects self-employment is policy-relevant given growing interest in supporting women's entrepreneurship and the expansion of gig economy opportunities.

I exploit staggered adoption of state EITCs across 33 states from 1986 to 2023 to estimate the effect of state EITC generosity on self-employment among single mothers. Using Current Population Survey (CPS) microdata from 2014--2023 and a difference-in-differences design with state and year fixed effects, I find that state EITC adoption is associated with a 1.3 percentage point \textit{decrease} in self-employment among single mothers---a 54 percent decline relative to the 2.4 percent baseline rate. This negative effect is statistically significant at conventional levels.

Several features of the results support a causal interpretation, though with important caveats. First, the placebo test using childless single women---who receive smaller EITC benefits due to fewer qualifying children---shows a smaller and statistically insignificant effect, consistent with the mechanism operating through EITC incentives. Second, the effect is concentrated in \textit{incorporated} self-employment rather than unincorporated self-employment, suggesting the finding reflects real business formation decisions rather than income misreporting (which would more plausibly appear in harder-to-verify unincorporated self-employment). However, significant pre-treatment coefficients in the event study analysis raise concerns about violations of the parallel trends assumption, and the results should be interpreted cautiously.

The finding that state EITCs reduce self-employment is consistent with a ``wage employment pull'' mechanism: the EITC makes wage employment relatively more attractive because wage income is automatically reported on W-2 forms, ensuring EITC benefits are correctly calculated and received. In contrast, self-employed individuals face more complex tax filing requirements and may have more difficulty claiming the credit. If the EITC effectively reduces the relative returns to self-employment, single mothers may substitute away from entrepreneurship and into wage jobs.

This paper contributes to three literatures. First, I add to the extensive literature on EITC labor supply effects by documenting effects on the self-employment margin for single mothers---a population for whom self-employment may be particularly valuable given flexibility needs but who have not been studied in this context. Second, I contribute to the growing literature on women's entrepreneurship by identifying a potential unintended consequence of income support programs. Third, I provide new evidence on state EITC programs, which have expanded dramatically in recent years but remain understudied relative to the federal credit.

\section{Institutional Background}

\subsection{The Federal EITC}

The federal Earned Income Tax Credit was established in 1975 and has been expanded multiple times since. The credit is calculated as a function of earned income (wages, salary, and net self-employment earnings) and the number of qualifying children. In the phase-in region, the credit increases with each additional dollar of earned income at a rate ranging from 7.65 percent (for childless workers) to 45 percent (for families with three or more children). After reaching the maximum credit, the schedule enters a plateau region where the credit remains constant, followed by a phase-out region where the credit declines with additional income.

For tax year 2023, a single mother with two children could receive a maximum credit of \$6,604, representing a substantial income supplement. The phase-in rate of 40 percent implies that, within the phase-in region, each additional dollar of earned income generates 40 cents in EITC benefits, effectively reducing the implicit marginal tax rate on work.

\subsection{State EITCs}

Beginning with Rhode Island in 1986, states have enacted their own EITCs that supplement the federal credit. As of 2023, 33 states plus the District of Columbia have state EITCs. These state credits are typically calculated as a percentage of the federal EITC, ranging from 4 percent in Wisconsin (for families with one child) to 125 percent in South Carolina.

State EITC adoption has been staggered across time and states, providing geographic and temporal variation for identification. Early adopters in the late 1980s and 1990s included Rhode Island (1986), Maryland (1987), Vermont (1988), Wisconsin (1989), and New York (1994). A second wave of adoptions occurred in the 2000s, with states like Illinois (2000), New Jersey (2000), Michigan (2006), and Louisiana (2007) enacting credits. Most recently, states including Montana (2019), Washington (2021), Utah (2022), and Missouri (2023) have joined.

Most state EITCs are refundable, meaning families receive the full credit even if it exceeds their tax liability. However, four states---Missouri, Ohio, South Carolina, and Utah---have non-refundable credits, which limits their effectiveness for the lowest-income families.

\subsection{Self-Employment and the EITC}

Self-employment income is treated the same as wage income for EITC purposes: net earnings from self-employment count toward earned income and increase EITC benefits in the phase-in region. However, self-employed workers face distinct challenges in claiming the credit. Unlike wage earners, who receive W-2 forms documenting their income, self-employed individuals must calculate and report their own net earnings on Schedule C (or Schedule C-EZ). This requires maintaining records of business income and expenses and understanding complex tax rules.

Prior research has documented that self-employment income is more responsive to EITC incentives than wage income, potentially because it is easier to adjust reported self-employment earnings to maximize credits. Saez (2010) found sharp ``bunching'' of self-employment income at the first EITC kink point, suggesting strategic income reporting. This raises interpretive challenges: observed changes in self-employment could reflect real business formation or merely changes in reporting behavior.

\section{Conceptual Framework}

The EITC affects labor supply through two channels: the substitution effect (making work more attractive relative to leisure) and the income effect (allowing workers to achieve the same consumption with less work). In the phase-in region, both effects encourage work, generating unambiguously positive labor supply predictions. In the phase-out region, the substitution effect discourages work while the income effect encourages it, with ambiguous net predictions.

For the choice between self-employment and wage employment, the EITC introduces an additional consideration: the relative ease of claiming benefits. Wage income is automatically reported to the IRS, ensuring workers receive their full EITC. Self-employment income requires additional documentation and calculation, potentially deterring some workers---particularly those with limited financial literacy or access to tax preparation assistance---from accurately claiming benefits.

This framework generates three testable predictions:

\textbf{Prediction 1:} If the EITC primarily affects overall labor force participation, we should observe positive effects on both self-employment and wage employment, with larger effects for single mothers (who receive larger credits) than childless women.

\textbf{Prediction 2:} If the EITC creates relative incentives favoring wage employment, we should observe negative effects on self-employment alongside positive effects on wage employment---a compositional shift toward easier-to-verify income sources.

\textbf{Prediction 3:} If self-employment responds primarily through income misreporting rather than real business formation, effects should be concentrated in unincorporated self-employment (which is harder to verify) rather than incorporated self-employment.

\section{Data}

\subsection{Current Population Survey}

I use data from the Annual Social and Economic Supplement (ASEC) of the Current Population Survey for years 2014--2023. The CPS ASEC is a nationally representative survey of approximately 95,000 households conducted each March, with detailed questions about income, employment, and family structure for the prior calendar year.

The key outcome variable is self-employment status, identified through the ``class of worker'' variable. I distinguish between incorporated self-employment (running a business that is legally incorporated) and unincorporated self-employment (sole proprietorships and partnerships). Incorporated self-employment is generally considered more indicative of substantive business activity and is harder to fabricate for tax purposes.

\subsection{Sample}

The main analysis sample consists of single mothers ages 18--55 with at least one child under 18 living in the household. I define ``single'' as not currently married and not living with a spouse---this includes never-married, divorced, separated, and widowed women. This sample contains 71,487 person-year observations over the 2014--2023 period.

I also construct two comparison samples for robustness and placebo tests: (1) a restricted sample of low-education single mothers (high school diploma or less), who are more likely to be EITC-eligible, containing 51,796 observations; and (2) a placebo sample of childless single women ages 18--55, who receive much smaller EITC benefits, containing 118,708 observations.

\subsection{State EITC Policy Data}

I compile state EITC adoption dates and credit parameters from the Institute on Taxation and Economic Policy (ITEP) and the National Conference of State Legislatures (NCSL). The treatment variable is a binary indicator equal to one if the respondent's state of residence had a state EITC in effect for that tax year.

\subsection{Summary Statistics}

Table 1 presents summary statistics for the main sample. The average single mother in the sample is 32 years old, has 1.7 children under 18, and has a 71 percent employment rate. The self-employment rate is 2.4 percent, with most self-employment being incorporated (2.4 percent) rather than unincorporated (0.02 percent). Approximately 54 percent of observations are from state-years with a state EITC in effect, and 73 percent of single mothers have a high school diploma or less.

\begin{table}[H]
\centering
\caption{Summary Statistics: Single Mothers Ages 18--55}
\begin{tabular}{lcc}
\toprule
& Mean & Std. Dev. \\
\midrule
Age & 31.8 & --- \\
Number of children under 18 & 1.68 & --- \\
Employed & 71.0\% & --- \\
Self-employed & 2.4\% & --- \\
\quad Incorporated & 2.4\% & --- \\
\quad Unincorporated & 0.02\% & --- \\
Has state EITC & 54.3\% & --- \\
Low education (HS or less) & 72.8\% & --- \\
\midrule
Observations & 71,487 & \\
\bottomrule
\end{tabular}
\label{tab:summary}
\end{table}

\section{Empirical Strategy}

\subsection{Difference-in-Differences Design}

I estimate the effect of state EITCs on self-employment using a two-way fixed effects (TWFE) difference-in-differences specification:

\begin{equation}
Y_{ist} = \alpha + \beta \cdot \text{StateEITC}_{st} + X_{ist}'\gamma + \theta_s + \lambda_t + \varepsilon_{ist}
\end{equation}

where $Y_{ist}$ is a self-employment indicator for individual $i$ in state $s$ and year $t$; $\text{StateEITC}_{st}$ is an indicator equal to one if state $s$ had a state EITC in year $t$; $X_{ist}$ is a vector of individual controls including age, age-squared, education, number of children, and presence of young children (under 6); $\theta_s$ and $\lambda_t$ are state and year fixed effects; and $\varepsilon_{ist}$ is an idiosyncratic error term. Standard errors are clustered at the state level to account for within-state correlation over time.

The coefficient of interest, $\beta$, captures the average effect of state EITC adoption on self-employment among single mothers, comparing treated states after adoption to untreated states and to treated states before adoption.

\subsection{Identification Assumptions}

The key identifying assumption is parallel trends: in the absence of state EITC adoption, self-employment rates among single mothers would have followed similar trajectories in adopting and non-adopting states. This assumption is untestable but can be assessed by examining pre-treatment trends in the event study framework.

I estimate an event study specification:

\begin{equation}
Y_{ist} = \alpha + \sum_{k \neq -1} \beta_k \cdot \mathbf{1}[\text{EventTime}_{st} = k] + X_{ist}'\gamma + \theta_s + \lambda_t + \varepsilon_{ist}
\end{equation}

where event time is the number of years relative to state EITC adoption, and the reference period is one year before adoption ($k = -1$). The coefficients $\beta_k$ for $k < -1$ test for differential pre-trends, while coefficients for $k \geq 0$ capture treatment effects.

\subsection{Threats to Identification}

Several factors could threaten identification. First, states that adopt EITCs may differ systematically from non-adopting states in ways that affect self-employment trends. I address this by including state fixed effects (absorbing time-invariant differences) and by examining pre-trends. Second, state EITC adoption may coincide with other policy changes (e.g., minimum wage increases, welfare reforms) that independently affect self-employment. The year fixed effects absorb national trends, and the staggered adoption provides variation that is plausibly independent of state-specific shocks. Third, with staggered treatment adoption, the standard TWFE estimator can be biased in the presence of heterogeneous treatment effects, as already-treated units serve as controls for later-treated units. To address this concern, I report results using the Sun and Abraham (2021) interaction-weighted estimator, which allows for heterogeneous treatment effects across cohorts and produces unbiased estimates under the parallel trends assumption.

\subsection{Sample Composition}

Table 1a presents the composition of the sample by treatment cohort. Of the 71,487 single mothers in the sample, 28,531 (40\%) reside in states that never adopted a state EITC during the sample period. These never-treated states serve as the primary comparison group. Among treated states, 29,733 observations (42\%) come from states that adopted EITCs before 2014---meaning all observations from these states are in the post-treatment period and contribute limited identifying variation. The remaining 18\% come from states adopting between 2014 and 2023, which provide the clearest identification. This composition raises concerns about the robustness of the estimates, as I explore below.

\section{Results}

\subsection{Main Results}

Table 2 presents the main difference-in-differences results. Column (1) reports the basic specification with only state and year fixed effects. Column (2) adds individual controls. Column (3) restricts the sample to low-education single mothers.

\begin{table}[H]
\centering
\caption{Effect of State EITC on Self-Employment Among Single Mothers}
\begin{tabular}{lccc}
\toprule
& (1) & (2) & (3) \\
& Basic & With Controls & Low Educ. \\
\midrule
Has State EITC & $-0.0129^{*}$ & $-0.0127^{*}$ & $-0.0112$ \\
& (0.0062) & (0.0062) & (0.0057) \\
\midrule
State FE & Yes & Yes & Yes \\
Year FE & Yes & Yes & Yes \\
Individual Controls & No & Yes & Yes \\
Observations & 71,487 & 71,487 & 51,796 \\
Mean Dep. Var. & 0.024 & 0.024 & --- \\
\bottomrule
\multicolumn{4}{p{0.85\textwidth}}{\footnotesize Notes: $^{*}p<0.05$, $^{**}p<0.01$, $^{***}p<0.001$. Standard errors clustered at state level in parentheses. Individual controls include age, age-squared, low education indicator, number of children under 18, and indicator for children under 6. All regressions weighted by CPS ASEC survey weights.}
\end{tabular}
\label{tab:main}
\end{table}

The results indicate that state EITC adoption is associated with a 1.3 percentage point \textit{decrease} in self-employment among single mothers. This represents a 54 percent decline relative to the baseline self-employment rate of 2.4 percent. The effect is statistically significant at the 5 percent level in the main specifications and robust to including individual controls. The restricted sample of low-education single mothers shows a similar point estimate ($-1.1$ pp) but is not statistically significant at conventional levels.

\subsection{Placebo Test}

Table 3 presents the placebo test using childless single women. If the mechanism operates through EITC incentives, we should observe smaller effects for this group, which receives smaller EITC benefits due to having no qualifying children.

\begin{table}[H]
\centering
\caption{Placebo Test: Effect on Childless Single Women}
\begin{tabular}{lc}
\toprule
& Childless Women \\
\midrule
Has State EITC & $-0.0050$ \\
& (0.0047) \\
\midrule
State FE & Yes \\
Year FE & Yes \\
Individual Controls & Yes \\
Observations & 118,708 \\
\bottomrule
\multicolumn{2}{p{0.6\textwidth}}{\footnotesize Notes: Standard errors clustered at state level in parentheses.}
\end{tabular}
\label{tab:placebo}
\end{table}

The estimated effect for childless women is smaller ($-0.5$ pp versus $-1.3$ pp) and statistically insignificant, consistent with the mechanism operating through EITC benefits that are larger for families with children.

\subsection{Self-Employment by Type}

Table 4 decomposes the self-employment effect by type: incorporated versus unincorporated. If the results reflected income misreporting rather than real business formation decisions, we would expect effects concentrated in unincorporated self-employment, which is harder to verify.

\begin{table}[H]
\centering
\caption{Effect by Self-Employment Type}
\begin{tabular}{lcc}
\toprule
& Incorporated & Unincorporated \\
\midrule
Has State EITC & $-0.0128^{*}$ & $0.0001$ \\
& (0.0062) & (0.0004) \\
\midrule
Observations & 71,487 & 71,487 \\
\bottomrule
\multicolumn{3}{p{0.7\textwidth}}{\footnotesize Notes: $^{*}p<0.05$. Standard errors clustered at state level.}
\end{tabular}
\label{tab:type}
\end{table}

The results show that the negative effect is entirely concentrated in incorporated self-employment, with a precise null effect for unincorporated self-employment. This pattern is inconsistent with income misreporting explanations and suggests the EITC affects real business formation decisions.

\subsection{Event Study}

Figure 1 presents the event study results, plotting coefficients for years relative to state EITC adoption. The reference period is one year before adoption (event time $= -1$).

\begin{figure}[H]
\centering
\includegraphics[width=0.9\textwidth]{figures/fig1_event_study.pdf}
\caption{Event Study: Effect of State EITC on Self-Employment}
\label{fig:event}
\end{figure}

The event study reveals two important patterns. First, there is a clear and persistent negative effect of state EITC adoption on self-employment in the post-treatment period. Coefficients for event times 0 through 5 are consistently negative and most are statistically significant, ranging from $-1.5$ to $-2.0$ percentage points.

Second, and more concerning, there is evidence of pre-treatment differences. The coefficient at event time $-4$ is negative and statistically significant, suggesting that states that adopted EITCs may have already been experiencing declining self-employment rates among single mothers before adoption. However, a joint test of the null hypothesis that all pre-treatment coefficients equal zero cannot be rejected at conventional significance levels ($\chi^2 = 5.63$, $p = 0.23$), suggesting the pre-trends pattern may be noise rather than a systematic violation. This joint test has limited power with only four pre-treatment periods, so failure to reject should not be interpreted as strong evidence for parallel trends---merely that the violation, if any, is not statistically distinguishable from zero.

\subsection{Robustness: Sun-Abraham Estimator}

Figure 5 compares the event study estimates from the standard TWFE specification with those from the Sun and Abraham (2021) interaction-weighted estimator. The Sun-Abraham estimator addresses potential bias from heterogeneous treatment effects in staggered adoption designs by using only never-treated and not-yet-treated units as comparisons.

\begin{figure}[H]
\centering
\includegraphics[width=0.9\textwidth]{figures/fig5_twfe_vs_sunab.pdf}
\caption{Event Study: TWFE vs. Sun-Abraham Estimator}
\label{fig:sunab}
\end{figure}

The two estimators produce similar patterns. The Sun-Abraham average post-treatment effect is $-1.86$ percentage points with an approximate standard error of $0.26$ percentage points---slightly larger in magnitude than the TWFE estimate but qualitatively similar. Both estimators show significant negative effects in the post-treatment period.

\subsection{Additional Robustness}

Table 5 presents additional robustness checks. The results are sensitive to sample composition. When dropping states that adopted EITCs before 2016 (which have no pre-treatment observations in the 2014--2023 data), the effect shrinks to $-0.56$ percentage points and is no longer statistically significant. This suggests that much of the identifying variation comes from early adopters, for whom parallel trends cannot be directly assessed. The result is robust to dropping 2020 (the COVID year) and to restricting the sample to prime-age workers (25--54). Notably, there is no significant effect on wage employment or overall employment. This creates a puzzle: if self-employment declines but wage employment does not increase, where do these workers go? One possibility is that the self-employment decline is offset by measurement noise or compositional effects across the sample. Another is that some workers exit the labor force entirely, though we cannot distinguish this from the null effect on overall employment. Regardless, the results suggest the EITC affects the \textit{composition} of employment (self- vs. wage) rather than total employment.

\begin{table}[H]
\centering
\caption{Robustness Checks}
\begin{tabular}{lcc}
\toprule
Specification & Estimate & Std. Error \\
\midrule
Main (TWFE) & $-0.0129^{*}$ & (0.0062) \\
Sun-Abraham ATT & $-0.0186^{***}$ & (0.0026) \\
Drop pre-2016 adopters & $-0.0056$ & (0.0087) \\
Drop 2020 & $-0.0131^{*}$ & (0.0061) \\
Ages 25--54 only & $-0.0185$ & (0.0097) \\
\midrule
\textit{Alternative outcomes:} & & \\
Wage employment & $-0.0010$ & (0.0184) \\
Any employment & $-0.0137$ & (0.0136) \\
\bottomrule
\multicolumn{3}{p{0.7\textwidth}}{\footnotesize Notes: $^{*}p<0.05$, $^{***}p<0.001$. All specifications include state and year fixed effects and individual controls.}
\end{tabular}
\label{tab:robust}
\end{table}

\section{Discussion}

\subsection{Interpretation}

The finding that state EITCs reduce self-employment among single mothers is surprising and warrants careful interpretation. Several mechanisms could explain this pattern:

\textbf{Wage Employment Pull.} The EITC may make wage employment relatively more attractive than self-employment because wage income is automatically reported and verified, ensuring workers receive their full EITC benefits. Self-employed workers face greater complexity in claiming the credit, potentially reducing its effective value. This mechanism is consistent with the concentration of effects in incorporated self-employment---precisely the type of self-employment where the owner might alternatively work as a wage employee.

\textbf{Selection.} States that adopt EITCs may differ systematically from non-adopting states. The significant pre-treatment coefficient at event time $-4$ suggests that EITC-adopting states may have been experiencing declining self-employment trends before adoption. This could reflect unmeasured confounders (such as labor market conditions or other policies) that independently affect self-employment. While state fixed effects absorb time-invariant differences, they cannot address differential trends.

\textbf{General Equilibrium Effects.} The EITC may affect equilibrium wages and employment opportunities in ways that alter the relative attractiveness of self-employment. If the EITC increases labor supply to wage employment, this could bid down wages for low-skill jobs, paradoxically making self-employment more attractive. However, the negative effect observed here suggests any such general equilibrium mechanism is dominated by direct incentive effects.

\subsection{Limitations}

This study has several important limitations. First, the significant coefficient at event time $-4$ raises concerns about parallel trends violations, though a joint test cannot reject the null of no pre-trends. The robustness check dropping early-adopter states, which produces an insignificant effect, suggests that the identifying variation may come primarily from states whose pre-trends cannot be assessed in this sample. Second, the Census Microdata API provides data only from 2014 onward, limiting the pre-treatment period available for states that adopted EITCs before 2014. Longer panels from IPUMS would provide more statistical power and better pre-trend assessment. Third, self-employment in the CPS is identified through the class of worker variable (A\_CLSWKR), which captures the respondent's primary job only. I cannot observe secondary self-employment or distinguish between full-time and part-time self-employment. Fourth, incorporated and unincorporated self-employment are defined based on CPS codes 6 (private, incorporated) and 7 (self-employed, unincorporated), respectively. The concentration of effects in incorporated self-employment suggests real business decisions rather than income misreporting, though alternative interpretations are possible.

\section{Conclusion}

This paper examines the effect of state Earned Income Tax Credits on self-employment among single mothers. Using a difference-in-differences design exploiting staggered state EITC adoption from 2014--2023, I find that state EITCs are associated with a 1.3 percentage point decrease in self-employment---a 54 percent decline relative to baseline. The effect is concentrated in incorporated self-employment and is larger for single mothers than for childless women, consistent with an EITC mechanism.

These findings suggest an unintended consequence of expanding the EITC: while the credit may increase overall employment among single mothers, it may simultaneously discourage entrepreneurship. The ``wage employment pull'' interpretation implies that policy efforts to encourage women's self-employment may need to address barriers to EITC claiming for self-employed workers, such as simplifying Schedule C reporting or providing targeted tax preparation assistance.

However, significant pre-treatment coefficients raise concerns about the causal interpretation. The results should be viewed as suggestive rather than definitive, and future research with longer panels and modern econometric methods is needed to confirm these findings.

\newpage
\section*{References}

\begin{description}

\item Callaway, B., \& Sant'Anna, P. H. (2021). Difference-in-differences with multiple time periods. \textit{Journal of Econometrics}, 225(2), 200--230.

\item Chetty, R., Friedman, J. N., \& Saez, E. (2013). Using differences in knowledge across neighborhoods to uncover the impacts of the EITC on earnings. \textit{American Economic Review}, 103(7), 2683--2721.

\item Decker, R., Haltiwanger, J., Jarmin, R., \& Miranda, J. (2014). The role of entrepreneurship in US job creation and economic dynamism. \textit{Journal of Economic Perspectives}, 28(3), 3--24.

\item Kleven, H. (2024). The EITC and the extensive margin: A reappraisal. \textit{Journal of Public Economics}, 230, 105047.

\item LaLumia, S., Sallee, J. M., \& Turner, N. (2018). The EITC and self-employment among married mothers. \textit{Labour Economics}, 55, 98--115.

\item Nichols, A., \& Rothstein, J. (2016). The earned income tax credit. In R. A. Moffitt (Ed.), \textit{Economics of Means-Tested Transfer Programs in the United States, Volume 1} (pp. 137--218). University of Chicago Press.

\item Rambachan, A., \& Roth, J. (2023). A more credible approach to parallel trends. \textit{Review of Economic Studies}, 90(5), 2555--2591.

\item Saez, E. (2010). Do taxpayers bunch at kink points? \textit{American Economic Journal: Economic Policy}, 2(3), 180--212.

\item Sun, L., \& Abraham, S. (2021). Estimating dynamic treatment effects in event studies with heterogeneous treatment effects. \textit{Journal of Econometrics}, 225(2), 175--199.

\end{description}

\newpage
\appendix

\section{Additional Figures}

\begin{figure}[H]
\centering
\includegraphics[width=0.85\textwidth]{figures/fig2_adoption_timeline.pdf}
\caption{State EITC Adoption Timeline}
\label{fig:timeline}
\end{figure}

\begin{figure}[H]
\centering
\includegraphics[width=0.85\textwidth]{figures/fig3_trends.pdf}
\caption{Self-Employment Trends by State EITC Status}
\label{fig:trends}
\end{figure}

\begin{figure}[H]
\centering
\includegraphics[width=0.7\textwidth]{figures/fig4_coef_comparison.pdf}
\caption{Coefficient Comparison: Main Sample vs. Placebo}
\label{fig:coef}
\end{figure}

\section{Sample Composition by Treatment Cohort}

\begin{table}[H]
\centering
\caption{Sample Composition by Treatment Cohort}
\begin{tabular}{lrrr}
\toprule
Cohort Group & Observations & States & Self-Emp. Rate \\
\midrule
Never Treated & 28,531 & 19 & 2.6\% \\
Before 2014 & 29,733 & 25 & 2.4\% \\
2014--2017 & 7,834 & 2 & 3.2\% \\
2018--2020 & 2,231 & 2 & 3.2\% \\
2021+ & 3,158 & 3 & 3.1\% \\
\midrule
Total & 71,487 & 51 & --- \\
\bottomrule
\end{tabular}
\label{tab:cohort}
\end{table}

Notes: ``Before 2014'' includes states that adopted a state EITC prior to the start of the sample period (2014). These observations are all post-treatment and contribute limited identifying variation. The primary comparison comes from never-treated states (19 states, 40\% of observations).

\end{document}
