\begin{table}[H]
\centering
\caption{Heterogeneity in Treatment Effects by Census Region}
\label{tab:heterogeneity}
\begin{threeparttable}
\begin{tabular}{lccccc}
\toprule
Region & ATT & SE & 95\% CI & Ever-Treated$^a$ & Never-Treated \\
\midrule
\textbf{Overall} & -0.540 & (0.446) & [-1.414, 0.334] & 31 (16) & 20 \\
\addlinespace
Midwest & -0.644 & (0.977) & [-2.559, 1.271] & 7 & 5 \\
South & -0.472** & (0.228) & [-0.920, -0.025] & 7 & 10 \\
West & 0.655 & (0.811) & [-0.935, 2.245] & 10 & 3 \\
\bottomrule
\end{tabular}
\begin{tablenotes}[flushleft]
\small
\item \textit{Notes:} Each row reports the overall ATT from a separate CS-DiD estimation on the regional subsample.
Overall: $N = 357$ (51 states $\times$ 7 years); Midwest: $N = 84$ (12 states); South: $N = 119$ (17 states); West: $N = 91$ (13 states). Northeast omitted due to insufficient never-treated states.
$^a$Ever-treated states in region (includes always-treated states that do not contribute to CS-DiD ATT). In parentheses for overall row: 16 contributing treated states (cohorts 2016--2021).
Standard errors clustered at the state level.
* $p<0.10$, ** $p<0.05$, *** $p<0.01$.
\end{tablenotes}
\end{threeparttable}
\end{table}
