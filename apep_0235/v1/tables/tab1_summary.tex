\begin{table}[H]
\centering
\caption{Summary Statistics}
\label{tab:summary}
\begin{threeparttable}
\begin{tabular}{l S[table-format=6.1] S[table-format=5.1] S[table-format=6.1] S[table-format=6.1] S[table-format=4.0]}
\toprule
Variable & {Mean} & {Std. Dev.} & {Min} & {Max} & {N} \\
\midrule
\multicolumn{6}{l}{\textit{Panel A: Monetary Policy Shock}} \\
MP Shock (JK) & -0.006 & 0.049 & -0.391 & 0.133 & 397 \\
\midrule
\multicolumn{6}{l}{\textit{Panel B: Employment (thousands)}} \\
Total Nonfarm & 133180.0 & 12281.0 & 108253.0 & 157032.0 & 397 \\
Manufacturing & 14285.9 & 2175.8 & 11382.0 & 17637.0 & 397 \\
Construction & 6445.2 & 952.6 & 4570.0 & 8123.0 & 397 \\
Mining/Logging & 681.0 & 86.9 & 539.0 & 904.0 & 397 \\
Retail Trade & 14822.6 & 818.6 & 12809.5 & 15872.6 & 397 \\
Financial Activities & 7911.8 & 684.4 & 6520.0 & 9193.0 & 397 \\
Prof./Business Svcs. & 17193.7 & 3241.1 & 10697.0 & 22850.0 & 397 \\
Education/Health Svcs. & 18513.3 & 4113.1 & 11349.0 & 26033.0 & 397 \\
Leisure/Hospitality & 12996.6 & 2138.2 & 8719.0 & 16889.0 & 397 \\
\midrule
\multicolumn{6}{l}{\textit{Panel C: Macro Controls}} \\
Federal Funds Rate & 2.6 & 2.2 & 0.1 & 6.9 & 397 \\
Unemployment Rate & 5.8 & 1.8 & 3.4 & 14.8 & 397 \\
CPI Inflation (\%) & 2.6 & 1.6 & -2.0 & 9.0 & 397 \\
IP Growth (\%) & 1.6 & 4.3 & -17.3 & 16.6 & 397 \\
\bottomrule
\end{tabular}
\begin{tablenotes}[flushleft]
\small
\item Notes: Monthly data, 1991:01--2024:01. Employment from BLS Current Employment Statistics (CES), in thousands. MP Shock is the Jarocinski-Karadi (2020) monetary policy surprise, aggregated to monthly frequency. CPI Inflation is 12-month percentage change in CPI-U. IP Growth is 12-month percentage change in industrial production index.
\end{tablenotes}
\end{threeparttable}
\end{table}