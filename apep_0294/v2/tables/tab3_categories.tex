\begin{table}[H]
\centering
\caption{Medicaid Spending by Service Category}
\label{tab:categories}
\begin{threeparttable}
\begin{tabular}{lrrrr}
\toprule
Category & Spending (\$B) & \% Total & Claims (B) & Codes \\
\midrule
CPT Professional Services & 367.6 & 33.6 & 10.3 & 5,731 \\
HCBS/State (T-codes) & 321.1 & 29.4 & 2.4 & 154 \\
Behavioral Health (H-codes) & 148.6 & 13.6 & 1.3 & 145 \\
Temporary/State (S-codes) & 95.7 & 8.8 & 1.2 & 299 \\
Other & 70.4 & 6.4 & 1.8 & 1,775 \\
Ambulance/DME (A-codes) & 40.3 & 3.7 & 0.9 & 584 \\
CMS Procedures (G-codes) & 27.7 & 2.5 & 0.5 & 832 \\
Drugs (J-codes) & 10.8 & 1.0 & 0.4 & 567 \\
DME (E-codes) & 9.5 & 0.9 & 0.1 & 410 \\
Orthotics/Prosthetics (L-codes) & 1.9 & 0.2 & 0.0 & 384 \\
\midrule
Total & 1093.6 & 100.0 & 18.8 & 10,881 \\
\bottomrule
\end{tabular}
\begin{tablenotes}[flushleft]
\small
\item \textit{Notes:} Categories defined by HCPCS code prefix. T-codes = HCBS/state-specific services; H-codes = behavioral health; S-codes = temporary/state codes; numeric = CPT professional services. Most T, H, and S codes have no Medicare equivalent.
\end{tablenotes}
\end{threeparttable}
\end{table}
