\begin{table}[htbp]
\centering
\caption{MDE Dilution Mapping: Population-Level vs.\ Treated-Group Detectable Effects}
\label{tab:dilution}
\begin{tabular}{ccccc}
\hline\hline
Treated Share & $s$ (\%) & Pop-Level MDE & Treated-Group MDE & MDE (\% Baseline) \\
\hline
0.03 & 3\% & 5.50 & 183.18 & 765.2\% \\
0.05 & 5\% & 5.50 & 109.91 & 459.1\% \\
0.10 & 10\% & 5.50 & 54.95 & 229.6\% \\
0.15 & 15\% & 5.50 & 36.64 & 153.0\% \\
0.20 & 20\% & 5.50 & 27.48 & 114.8\% \\
\hline\hline
\end{tabular}
\begin{tablenotes}
\small
\item \textit{Notes:} Dilution algebra: $\text{ATT}_{\text{pop}} = s \times \Delta_T$, where $s$ is the share of mortality attributable to the treated subpopulation (privately insured diabetics using insulin), and $\Delta_T$ is the true effect on that group.
\item Population-level MDE is computed at 80\% power, 5\% significance using TWFE SE.
\item Mean baseline diabetes mortality rate: 23.94 per 100,000.
\item For realistic treated shares ($s = 3$--$5\%%$), the MDE on the treated group exceeds 100\% of baseline mortality, meaning the design cannot detect plausible effects without subpopulation-specific data.
\end{tablenotes}
\end{table}
