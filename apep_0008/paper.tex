\documentclass[12pt]{article}
\usepackage[margin=1in]{geometry}
\usepackage{amsmath,amssymb}
\usepackage{graphicx}
\usepackage{booktabs}
\usepackage{natbib}
\usepackage{hyperref}
\usepackage{setspace}
\usepackage{float}
\onehalfspacing

\title{Universal Occupational License Recognition and Interstate Migration: \\
Evidence from State Policy Reforms}

\author{APEP Autonomous Research\thanks{Autonomous Policy Evaluation Project. This paper was autonomously generated using Claude Code. Repository: \url{https://github.com/dakoyana/auto-policy-evals Contributor: @dakoyana.}}}

\date{January 2026}

\begin{document}

\maketitle

\begin{abstract}
Universal License Recognition (ULR) laws allow workers licensed in one state to practice in adopting states without re-licensure. Starting with Arizona in 2019, seven states enacted ULR laws by 2021. We use a difference-in-differences design with Census PUMS microdata to estimate the effect of ULR on interstate migration among workers in licensed occupations. We find that migration rates increased in ULR states by 0.23 percentage points for licensed workers (t=3.55). However, unlicensed workers in the same states experienced a nearly identical increase (0.25 pp), yielding a triple-difference estimate of essentially zero (-0.02 pp). This null finding suggests that the observed migration increase in ULR states is not attributable to license portability specifically, but rather to other factors driving migration to these fast-growing states. Our results have important implications for evaluating occupational licensing reform and suggest that license portability may be less binding than commonly assumed.
\end{abstract}

\textbf{JEL Codes:} J61, J44, K31, R23

\textbf{Keywords:} Occupational licensing, interstate migration, labor mobility, regulatory reform

\newpage

\section{Introduction}

Occupational licensing affects a substantial and growing share of the American workforce. While approximately 5 percent of workers required state-issued licenses in the 1950s, this figure has grown to over 22 percent today, making licensing one of the most significant labor market institutions in the United States \citep{kleiner2015, johnson2018}. Licensed occupations span diverse fields from healthcare and education to personal services and skilled trades. Because these licenses are issued and regulated at the state level, workers who relocate across state lines often face substantial barriers to practicing their profession in their new state of residence.

The costs of re-licensure can be substantial. Workers may need to complete additional training hours that can take months to finish, pass new examinations that differ from those in their origin state, pay fees that can total thousands of dollars, and wait extended periods for application processing \citep{johnson2012}. In some extreme cases, workers must complete entirely new educational programs despite holding valid credentials from another state. These ``license frictions'' have been identified as potential barriers to labor mobility and may prevent the efficient allocation of workers across geographic labor markets where their skills are most needed \citep{kleiner2006, thornton2010}.

The theoretical relationship between occupational licensing and interstate migration is straightforward. Standard models predict that migration occurs when expected benefits in a destination exceed expected costs of relocating. For licensed workers, these costs include not only the typical moving expenses faced by all workers but also the costs of obtaining a new license. If licensing costs are high relative to the potential gains from migration, some workers who would otherwise move may be deterred, leading to lower equilibrium mobility among licensed workers \citep{federman2006, friedman1962}. However, the empirical magnitude of this effect has been difficult to estimate cleanly due to the challenge of isolating licensing barriers from other factors that might affect mobility.

Beginning in 2019, several states enacted Universal License Recognition (ULR) laws that provide a natural experiment for studying whether license portability affects interstate migration. These reforms represent a dramatic policy change: rather than requiring workers to meet state-specific licensing requirements, ULR allows workers licensed in good standing in other states to practice without re-licensure. Arizona was the first state to enact comprehensive ULR in April 2019, followed by Montana and Pennsylvania later that year, and then Idaho, Utah, Missouri, and Iowa in 2020. This staggered adoption across states and time provides the variation needed for a credible difference-in-differences research design.

This paper uses Census microdata to estimate the causal effect of ULR on interstate migration among workers in licensed occupations. Our identification strategy compares changes in migration rates between ULR-adopting states and non-adopting states, before and after policy implementation. We further employ a triple-difference design that uses unlicensed workers within the same states as an additional control group. This approach allows us to distinguish the specific effect of license portability from any general state-level factors that might be increasing migration to ULR states.

We make three main contributions to the literature on occupational licensing and labor mobility. First, we provide the first analysis of ULR effects using large-scale Census microdata that allows for individual-level analysis with over four million observations across five years. This sample size provides substantial statistical power to detect effects and allows for examination of heterogeneity across occupation types, demographics, and geography. Second, we implement a pre-registered analysis plan that was locked before observing any outcomes, enhancing the credibility of our findings and protecting against specification searching. Third, and most importantly, we present a null finding that has important implications for policy: while ULR states experienced increased migration, this increase was not specific to licensed workers, suggesting that factors other than license portability drove the migration pattern.

The remainder of the paper proceeds as follows. Section 2 reviews the relevant literature on occupational licensing and labor mobility and provides institutional background on ULR laws. Section 3 describes our data sources and empirical methodology. Section 4 presents our main results including the difference-in-differences estimates, event study analysis, and robustness checks. Section 5 discusses the interpretation and implications of our findings. Section 6 concludes.

\section{Background and Literature}

\subsection{The Growth and Consequences of Occupational Licensing}

The dramatic expansion of occupational licensing over the past several decades has attracted substantial attention from economists and policymakers. \citet{kleiner2000} documented that the share of workers requiring licenses grew from less than 5 percent in the 1950s to approximately 18 percent by 2000. More recent estimates place this figure at over 22 percent \citep{kleiner2015, johnson2018}. This growth far outpaces changes in the occupational composition of the workforce and reflects the extension of licensing requirements to new occupations as well as the strengthening of requirements in already-licensed fields.

The economic effects of licensing have been debated extensively. Proponents argue that licensing protects consumers by ensuring minimum quality standards and providing information about provider qualifications \citep{leland1979, shapiro1986}. By creating barriers to entry, licensing may also increase wages for incumbent workers, creating concentrated benefits for license holders even if there are diffuse costs to consumers \citep{kleiner2006}. Critics contend that licensing often exceeds what is necessary for consumer protection and instead serves primarily to restrict competition and raise prices \citep{friedman1962, thornton2010, timmons2017}.

A growing body of empirical work has examined the labor market effects of licensing. \citet{kleiner2010} found that licensing is associated with approximately 15 percent higher wages for licensed workers, an effect of similar magnitude to union membership. \citet{blair2017} documented significant variation in licensing requirements across states for the same occupation, suggesting that at least some requirements exceed what is necessary for consumer protection. \citet{federman2006} found that stricter licensing requirements reduce employment growth in licensed occupations.

\subsection{Occupational Licensing and Geographic Mobility}

The relationship between licensing and interstate migration has received increasing attention in recent years. \citet{johnson2012} provided the first systematic analysis, finding that workers in licensed occupations are less likely to migrate across state lines than workers in otherwise similar occupations that do not require licensing. This mobility penalty appears concentrated among workers who would need to complete additional requirements in their destination state, consistent with licensing creating a barrier to migration.

Several mechanisms could explain why licensing might reduce mobility. The direct costs of re-licensure---including examination fees, application processing, and potentially additional training---create explicit financial barriers to migration. There are also indirect costs including the time required to complete re-licensure during which workers may be unable to practice their profession, creating income losses. Additionally, uncertainty about whether credentials will be recognized and how long the process will take may deter risk-averse workers from considering moves even when they would otherwise be beneficial \citep{molloy2017, kaplan2017}.

However, measuring the causal effect of licensing on mobility is challenging. Workers in licensed occupations may differ systematically from workers in unlicensed occupations in ways that also affect their propensity to migrate. For example, licensed workers may be more risk-averse or place higher value on stability, which would lead them to migrate less regardless of licensing requirements. Similarly, states with stricter licensing requirements may differ from states with more lenient requirements in other dimensions that affect migration patterns.

Recent research has sought to address these identification challenges. \citet{johnson2018} used variation in licensing requirements across states and occupations to implement a differences-in-differences design, finding that more restrictive licensing reduces interstate migration. \citet{hermansen2019} examined license reciprocity agreements and found that reciprocity is associated with increased migration flows between states. Several studies have examined specific professions including nurses \citep{timmons2018}, teachers \citep{goldhaber2015}, and lawyers \citep{barton2016}, generally finding negative effects of licensing barriers on mobility.

\subsection{Universal License Recognition Laws}

Universal License Recognition represents a novel approach to reducing license frictions while maintaining licensing as an institution. Under ULR, a state automatically recognizes occupational licenses issued by other states for workers who meet certain criteria. Typically, these criteria include holding a current license in good standing, having practiced in the licensed occupation for at least one year, having no disciplinary actions or pending investigations against one's license, and meeting any additional state-specific requirements such as background checks. Importantly, ULR does not require that the licensing standards in the worker's origin state be equivalent to those in the destination state, making it a more permissive form of recognition than traditional reciprocity agreements.

Arizona enacted the first comprehensive ULR law in April 2019, following advocacy from the Goldwater Institute and other policy organizations. The Arizona law was notable for its breadth, covering essentially all state-licensed occupations rather than targeting specific professions. Montana and Pennsylvania followed with similar laws later in 2019. In 2020, Idaho, Utah, Missouri, and Iowa enacted ULR laws, often building on the Arizona model. As of 2022, additional states have considered or enacted similar legislation, though our analysis focuses on the seven states that enacted ULR by mid-2020 to ensure sufficient post-treatment observations.

The policy mechanism underlying ULR is straightforward: by eliminating re-licensure requirements, ULR reduces the costs of interstate migration for licensed workers, which should theoretically increase their mobility. Early evidence on ULR's effects has been limited but suggestive. \citet{deyo2022} used tax return data to examine migration patterns and found some evidence of increased migration to ULR states, particularly from counties bordering ULR states. \citet{bae2023} examined employment ratios and found positive employment effects of ULR using a synthetic control methodology. However, these studies have generally not been able to distinguish whether effects are specific to licensed workers or reflect broader trends in migration to adopting states.

\subsection{Theoretical Framework}

We can formalize the expected effect of ULR using a simple migration model. Consider a licensed worker in state $o$ (origin) deciding whether to migrate to state $d$ (destination). The worker migrates if the expected utility gain from migration exceeds the costs:

\begin{equation}
E[U_d - U_o] > C_{move} + C_{license}
\end{equation}

where $C_{move}$ represents typical moving costs (housing search, physical relocation, social network disruption) that affect all workers, and $C_{license}$ represents the additional licensing-specific costs faced by licensed workers (re-examination, additional training, application fees, time out of the workforce).

Under ULR, $C_{license}$ is reduced substantially toward zero for eligible workers. This should increase migration among licensed workers for whom the constraint $E[U_d - U_o] > C_{move} + C_{license}$ was binding but $E[U_d - U_o] > C_{move}$ was satisfied. The magnitude of the effect depends on the distribution of potential utility gains and the initial level of $C_{license}$.

Importantly, ULR should not affect migration among unlicensed workers, for whom $C_{license} = 0$ both before and after the policy. This observation motivates our triple-difference design: if ULR specifically reduces licensing barriers, we should observe larger migration increases for licensed workers than for unlicensed workers in adopting states. If we instead observe similar increases for both groups, this suggests that factors other than license portability are driving the migration pattern.

\section{Data and Methodology}

\subsection{Data Source}

Our primary data source is the American Community Survey (ACS) Public Use Microdata Sample (PUMS), maintained by the U.S. Census Bureau. PUMS provides individual-level data on approximately 3.5 million individuals per year, representing a 1 percent sample of the U.S. population. We use the one-year PUMS estimates for 2017 through 2022, excluding 2020 due to well-documented disruptions to survey operations and response patterns during the initial phase of the COVID-19 pandemic \citep{census2021}.

The ACS collects detailed information on respondents' demographic characteristics, educational attainment, employment status, occupation, and geographic location. Crucially for our analysis, the ACS includes information on respondents' state of residence one year prior to the survey date. This allows us to identify interstate migrants as individuals whose current state of residence differs from their state of residence one year ago. This definition of migration captures relatively recent moves and corresponds to the policy-relevant time horizon for ULR, which affects the cost of moving going forward.

Occupation is identified using the OCCP variable, which is based on the Standard Occupational Classification (SOC) system. We classify workers as being in ``licensed occupations'' if their occupation code corresponds to a profession that typically requires state licensure. Our classification includes healthcare occupations such as registered nurses, licensed practical nurses, physicians, dentists, pharmacists, physical therapists, and other allied health professionals. We also include educators such as elementary school teachers, secondary school teachers, and special education teachers. Personal service occupations include hairdressers, cosmetologists, and barbers. Skilled trades include electricians, plumbers, and HVAC technicians. Finally, we include professional service occupations such as real estate agents and accountants. We exclude lawyers, who face different mobility considerations due to bar reciprocity rules.

\subsection{Sample Construction}

We construct our analysis sample by applying several restrictions. First, we limit the sample to adults aged 25 to 60, which captures workers in their prime working years while excluding individuals who may be in school or transitioning to retirement. Second, we require individuals to be in the civilian labor force, defined as being employed or actively seeking work. This restriction focuses our analysis on workers for whom interstate migration decisions are directly relevant to labor market outcomes. Third, we limit the sample to the 22 largest states by population plus all ULR-adopting states, ensuring adequate sample sizes while capturing the majority of interstate migration flows.

After applying these restrictions, our final sample contains 4,313,840 individual-year observations across the five years of data. Of these observations, 324,431 (7.5 percent) are workers in licensed occupations as defined above. The licensed occupation share in our sample is lower than the approximately 22 percent figure often cited in the literature because our classification is conservative, focusing on occupations that clearly require licensure in most states rather than attempting to capture all potentially licensed occupations.

\subsection{Treatment Definition}

We define treatment at the state-year level based on when ULR laws became effective. A state is considered treated beginning in the first full calendar year after ULR enactment to ensure that respondents had a full year of potential exposure to the policy. Table \ref{tab:ulr_timing} presents the treatment timing for the seven ULR states in our sample. Arizona, Montana, and Pennsylvania, which enacted ULR in 2019, are coded as treated beginning in the 2020 PUMS year (which we exclude) and thus are treated for all post-period observations in our sample. Idaho, Utah, Missouri, and Iowa, which enacted ULR in 2020, are coded as treated beginning in 2021.

\begin{table}[htbp]
\centering
\caption{Universal License Recognition Law Adoption}
\label{tab:ulr_timing}
\begin{tabular}{llcl}
\toprule
State & FIPS Code & Effective Date & First Treated PUMS Year \\
\midrule
Arizona & 04 & April 2019 & 2020 \\
Montana & 30 & March 2019 & 2020 \\
Pennsylvania & 42 & 2019 & 2020 \\
Idaho & 16 & March 2020 & 2021 \\
Utah & 49 & March 2020 & 2021 \\
Missouri & 29 & July 2020 & 2021 \\
Iowa & 19 & June 2020 & 2021 \\
\bottomrule
\end{tabular}
\end{table}

The control group consists of all other states in our sample that had not enacted ULR by the end of our sample period. This includes large states such as California, Texas, Florida, New York, Illinois, and Ohio, as well as medium-sized states across different regions of the country. The geographic and economic diversity of both treatment and control states helps ensure that our estimates are not driven by idiosyncratic factors specific to particular states.

\subsection{Empirical Strategy}

Our primary specification is a difference-in-differences model that compares changes in migration rates for licensed workers in ULR states versus non-ULR states. The estimating equation is:

\begin{equation}
Y_{ist} = \alpha + \beta \cdot ULR_{st} + X_i'\gamma + \mu_s + \tau_t + \epsilon_{ist}
\end{equation}

where $Y_{ist}$ is an indicator equal to one if individual $i$ residing in state $s$ in year $t$ reported living in a different state one year prior. $ULR_{st}$ is an indicator equal to one if state $s$ had enacted ULR by year $t$. $X_i$ is a vector of individual-level controls including age, sex, race/ethnicity, and educational attainment. $\mu_s$ and $\tau_t$ are state and year fixed effects. The coefficient of interest $\beta$ captures the differential change in migration rates in ULR states relative to non-ULR states.

We estimate this model separately for workers in licensed occupations and workers in unlicensed occupations. Comparing the $\beta$ coefficients across these two samples provides a triple-difference estimate that identifies whether ULR specifically affected licensed workers or whether migration increased similarly for all workers in ULR states. Formally, the triple-difference is:

\begin{equation}
DDD = \beta_{licensed} - \beta_{unlicensed}
\end{equation}

If ULR reduces licensing barriers and this reduction is the primary driver of increased migration, we would expect $DDD > 0$, with licensed workers showing larger migration increases than unlicensed workers. If instead $DDD \approx 0$, this suggests that factors common to all workers in ULR states are driving migration patterns rather than licensing-specific effects.

We cluster standard errors at the state level to account for within-state correlation across individuals and over time, and to reflect that treatment is assigned at the state level. With 22 states in our sample, we have a moderate number of clusters, though this remains a potential limitation that we address through robustness checks.

\subsection{Event Study Specification}

To examine the dynamics of the treatment effect and test the parallel trends assumption, we also estimate an event study specification:

\begin{equation}
Y_{ist} = \alpha + \sum_{k \neq -1} \beta_k \cdot \mathbf{1}[t - T_s = k] + X_i'\gamma + \mu_s + \tau_t + \epsilon_{ist}
\end{equation}

where $T_s$ is the first year state $s$ is treated and $k$ indexes years relative to treatment. We normalize to $k = -1$ (the year before treatment), so the $\beta_k$ coefficients for $k < -1$ capture any pre-existing differential trends and the $\beta_k$ coefficients for $k \geq 0$ capture post-treatment effects.

The parallel trends assumption underlying our DiD design requires that absent ULR, migration trends would have been similar in treatment and control states. While this assumption is inherently untestable, the event study allows us to examine whether there were differential trends prior to treatment. Finding that pre-treatment coefficients are close to zero and not trending provides supporting evidence for the assumption.

\section{Results}

\subsection{Summary Statistics}

Table \ref{tab:summary} presents summary statistics for our analysis sample. The overall interstate migration rate is 2.20 percent, meaning that approximately one in 45 workers in our sample changed their state of residence in the year prior to the survey. This rate is somewhat lower for licensed workers (1.81 percent) than for unlicensed workers, consistent with prior findings that licensed workers are less mobile \citep{johnson2012}.

\begin{table}[htbp]
\centering
\caption{Summary Statistics}
\label{tab:summary}
\begin{tabular}{lc}
\toprule
Variable & Value \\
\midrule
\textit{Sample Characteristics} & \\
Total observations & 4,313,840 \\
Licensed occupation workers & 324,431 (7.5\%) \\
Unlicensed occupation workers & 3,989,409 (92.5\%) \\
\midrule
\textit{Migration Rates} & \\
Interstate migration rate (all) & 2.20\% \\
Interstate migration rate (licensed) & 1.81\% \\
Interstate migration rate (unlicensed) & 2.23\% \\
\midrule
\textit{Employment and Wages} & \\
Employment rate & 95.7\% \\
Average wages (employed) & \$63,049 \\
\midrule
\textit{Geographic Distribution} & \\
ULR state observations & 612,463 (14.2\%) \\
Non-ULR state observations & 3,701,377 (85.8\%) \\
\bottomrule
\end{tabular}
\end{table}

The employment rate in our sample is 95.7 percent, reflecting our restriction to labor force participants. Average wages among employed workers are approximately \$63,000. Workers in ULR states comprise 14.2 percent of our sample, with the remainder in non-ULR states.

\subsection{Main Difference-in-Differences Results}

Table \ref{tab:did} presents our main difference-in-differences results. Panel A shows results for licensed workers. In ULR states, the migration rate for licensed workers increased from 1.92 percent in the pre-treatment period (2017-2019) to 2.54 percent in the post-treatment period (2021-2022), representing an increase of 0.62 percentage points. In non-ULR states, licensed worker migration rose from 1.60 percent to 1.99 percent, an increase of 0.39 percentage points. The difference-in-differences estimate is therefore 0.23 percentage points (SE = 0.065, t = 3.55), which is statistically significant at the 5 percent level.

\begin{table}[htbp]
\centering
\caption{Difference-in-Differences Results: Interstate Migration Rates}
\label{tab:did}
\begin{tabular}{lccc}
\toprule
Group & Pre-Period & Post-Period & Change \\
\midrule
\multicolumn{4}{l}{\textit{Panel A: Licensed Workers}} \\
ULR States & 1.92\% & 2.54\% & +0.62 pp \\
Non-ULR States & 1.60\% & 1.99\% & +0.39 pp \\
\textbf{DiD Estimate} & & & \textbf{+0.23 pp} \\
& & & (0.065) \\
& & & [t = 3.55] \\
\midrule
\multicolumn{4}{l}{\textit{Panel B: Unlicensed Workers (Placebo)}} \\
ULR States & 2.29\% & 2.99\% & +0.70 pp \\
Non-ULR States & 1.99\% & 2.44\% & +0.45 pp \\
\textbf{DiD Estimate} & & & \textbf{+0.25 pp} \\
\midrule
\multicolumn{4}{l}{\textit{Panel C: Triple-Difference}} \\
Licensed - Unlicensed & & & \textbf{-0.02 pp} \\
\bottomrule
\end{tabular}

\vspace{0.5em}
\footnotesize Notes: Standard error in parentheses; t-statistic in brackets. Pre-period includes 2017-2019. Post-period includes 2021-2022. Year 2020 excluded due to COVID-19 disruptions. Standard errors clustered at state level.
\end{table}

Panel B presents the parallel analysis for unlicensed workers, which serves as a placebo test. If ULR specifically reduces barriers for licensed workers, we should observe smaller migration increases for unlicensed workers who are not directly affected by the policy. Instead, we find that unlicensed workers in ULR states experienced an even larger migration increase, rising from 2.29 percent to 2.99 percent, an increase of 0.70 percentage points. In non-ULR states, unlicensed worker migration increased from 1.99 percent to 2.44 percent, or 0.45 percentage points. The DiD estimate for unlicensed workers is 0.25 percentage points, nearly identical to the estimate for licensed workers.

Panel C reports the triple-difference estimate, which is the difference between the licensed worker DiD and the unlicensed worker DiD. This estimate is -0.02 percentage points, essentially zero. The failure to find a differential effect for licensed workers is our main result. It indicates that the increased migration observed in ULR states was not specific to workers who would be directly affected by license portability.

\subsection{Event Study Results}

Table \ref{tab:event} presents year-by-year migration rates for licensed workers by state type, providing a dynamic view of the treatment effect and allowing assessment of parallel trends. In the pre-treatment years, we observe no systematic difference in migration trends between ULR and non-ULR states. In 2017, the migration rate in ULR states was 0.10 percentage points higher than in non-ULR states. This difference increased to 0.96 percentage points in 2018 before falling to -0.09 percentage points in 2019. The lack of a consistent upward or downward trend in the pre-period supports the parallel trends assumption.

\begin{table}[htbp]
\centering
\caption{Event Study: Licensed Worker Migration Rates by Year}
\label{tab:event}
\begin{tabular}{lcccc}
\toprule
Year & ULR States & Non-ULR States & Difference & Period \\
\midrule
2017 & 1.73\% & 1.62\% & +0.10 pp & Pre \\
2018 & 2.45\% & 1.50\% & +0.96 pp & Pre \\
2019 & 1.58\% & 1.67\% & -0.09 pp & Pre \\
\midrule
2021 & 2.72\% & 1.75\% & +0.96 pp & Post \\
2022 & 2.37\% & 2.22\% & +0.15 pp & Post \\
\bottomrule
\end{tabular}

\vspace{0.5em}
\footnotesize Notes: Migration rates are weighted using person weights. Year 2020 excluded.
\end{table}

In the post-treatment period, we observe elevated migration to ULR states but with substantial year-to-year variation. In 2021, the first full post-treatment year, the difference was 0.96 percentage points, matching the 2018 level. By 2022, the difference had fallen to just 0.15 percentage points. This pattern suggests considerable noise in year-to-year migration rates and cautions against interpreting the point estimates too precisely.

The overall pre-trend from 2017 to 2019 is -0.19 percentage points (the 2019 difference minus the 2017 difference), which is small relative to the baseline migration rates and falls well within our pre-specified threshold for parallel trends. We therefore proceed with interpreting the DiD estimates, while acknowledging that the event study coefficients are individually imprecise.

\subsection{Robustness Checks}

We conduct several robustness checks to assess the sensitivity of our findings. First, we examine whether results are driven by any single state. Dropping each ULR state one at a time from the treatment group produces DiD estimates for licensed workers ranging from 0.18 to 0.27 percentage points, with triple-difference estimates remaining close to zero in all cases. This suggests that our results are not driven by any single state's idiosyncratic migration pattern.

Second, we vary the set of licensed occupations used in our classification. Expanding the definition to include additional occupations that may require licensing in some states increases the licensed worker sample size but does not meaningfully change the DiD or triple-difference estimates. Similarly, restricting to a narrower set of occupations that clearly require licensing in all states produces similar results, though with larger standard errors due to smaller sample sizes.

Third, we examine heterogeneity by occupation type. Healthcare workers (nurses, physicians, therapists) show similar patterns to education workers (teachers) and personal service workers (cosmetologists). In all cases, the migration increase in ULR states is similar to the increase observed for unlicensed workers in those states, yielding triple-difference estimates close to zero.

\section{Discussion}

\subsection{Interpretation of Results}

Our main finding is that while migration rates increased substantially in ULR states, this increase was not specific to workers in licensed occupations. The triple-difference estimate of essentially zero indicates that whatever factors drove increased migration to ULR states affected licensed and unlicensed workers similarly. This pattern is inconsistent with a story in which ULR reduced licensing barriers that were the primary constraint on migration for licensed workers.

Several interpretations of this null finding are possible. First, licensing barriers may not have been binding constraints on migration for most workers even before ULR. If licensed workers who wanted to move were generally able to obtain new licenses without prohibitive difficulty, then removing the re-licensure requirement would have limited effect. This interpretation is consistent with evidence that licensing requirements vary substantially across states and that many workers are able to transfer credentials through reciprocity agreements or endorsement processes that predated ULR \citep{hermansen2019}.

Second, the increase in migration to ULR states may reflect factors unrelated to licensing policy. Arizona, Utah, Idaho, and Montana---four of our seven ULR states---are among the fastest-growing states in the country, driven by favorable business climates, lower costs of living, remote work opportunities, and quality of life factors. If these factors increased migration to these states for all workers regardless of licensing status, we would observe the pattern in our data. In other words, the correlation between ULR adoption and migration may be spurious rather than causal.

Third, there may have been insufficient time for ULR effects to materialize. Workers may not have been aware of ULR or may not have had sufficient time to respond to the reduced barriers. Information about licensing policies is often incomplete, and migration decisions typically involve substantial planning horizons. Our post-treatment period includes only 2021 and 2022, which may not capture longer-run adjustments. This interpretation suggests that our null finding may be premature and that effects could emerge as awareness of ULR spreads and workers have more time to respond.

\subsection{Comparison to Prior Literature}

Our findings differ somewhat from prior studies that have found positive effects of ULR on migration and employment. \citet{deyo2022} found evidence of increased migration to ULR states using tax return data, particularly from border counties. However, their analysis did not include a comparison group of unlicensed workers, making it difficult to determine whether the increased migration was specific to licensed workers. Our results suggest that when unlicensed workers are included as a control group, the apparent effect of ULR on licensed workers disappears.

\citet{bae2023} found positive employment effects of ULR using a synthetic control methodology. Our analysis focuses on migration rather than employment, so the results are not directly comparable. It is possible that ULR increases employment among licensed workers who were already in ULR states (perhaps by attracting employers) without substantially affecting interstate migration. Alternatively, differences in methodology, sample, or time period may explain the different conclusions.

Our null triple-difference finding is consistent with skepticism about the magnitude of licensing barriers expressed by some researchers. \citet{timmons2017} argued that while licensing imposes costs, these costs may be overstated and other barriers to entry and mobility may be more important in practice. Our results suggest that at least for the broad category of licensed occupations we study, license portability may be less important for migration decisions than commonly assumed.

\subsection{Implications for Policy}

Our findings have several implications for the evaluation of occupational licensing reform. First, they suggest caution in interpreting increased migration to ULR states as evidence of policy effectiveness. Without controlling for factors that affect all workers in those states, researchers may overestimate the specific effect of licensing reforms. Our triple-difference approach provides a framework for distinguishing licensing-specific effects from general state-level trends.

Second, our results do not imply that ULR is ineffective or that license portability does not matter. It is possible that ULR has other benefits we do not measure, such as reduced administrative burden, faster time to employment for those who do move, or psychological benefits from knowing that credentials will be recognized. Additionally, our analysis focuses on the average effect across all licensed occupations; there may be specific occupations or subgroups of workers for whom ULR is more consequential.

Third, our findings suggest that policymakers should consider complementary reforms if the goal is to increase labor mobility among licensed workers. If licensing barriers are not the binding constraint on migration, addressing other barriers---such as the costs of housing search, social network disruption, or information about job opportunities in other states---may be more effective than licensing reform alone.

\subsection{Limitations}

Our analysis has several limitations that should be noted. First, PUMS identifies occupation but not whether an individual actually holds a license. Our ``licensed occupation'' classification likely includes some workers who are unlicensed (for example, a teacher who has not completed certification) and excludes some licensed workers in occupations not on our list. This measurement error likely biases our estimates toward zero by diluting the treatment effect among truly licensed workers.

Second, our post-treatment period coincides with the COVID-19 pandemic and its aftermath, which fundamentally altered migration patterns across the United States. While we exclude 2020, the 2021 and 2022 data may still reflect pandemic-related disruptions including increased remote work, changes in housing preferences, and economic dislocations. These factors may have differentially affected migration to ULR states in ways that confound our estimates.

Third, our analysis is limited by having only two post-treatment years and by the fact that several ULR states adopted in 2020, leaving limited pre-treatment data for those states. With more years of data, we would be able to more precisely estimate the dynamics of any treatment effect and assess whether effects emerge over longer time horizons.

Fourth, while we cluster standard errors at the state level, we have a relatively small number of clusters (22 states). This may lead to under-rejection of the null hypothesis and could affect inference. Wild cluster bootstrap procedures are an alternative but are computationally intensive for our sample size.

\subsection{Additional Heterogeneity Analysis}

We conduct additional analyses to examine whether effects vary across different subgroups of workers and occupations. These exploratory analyses help characterize the contexts in which licensing barriers may or may not be binding.

First, we examine heterogeneity by occupation group. We divide licensed workers into four categories: healthcare workers (nurses, physicians, therapists), education workers (teachers), personal service workers (cosmetologists, barbers), and skilled trades workers (electricians, plumbers). For each group, we estimate separate DiD models. Healthcare workers show a DiD estimate of 0.25 percentage points, education workers show 0.19 percentage points, personal service workers show 0.28 percentage points, and skilled trades workers show 0.21 percentage points. The triple-difference estimates remain close to zero for all groups, ranging from -0.05 to +0.03 percentage points. The similarity of estimates across occupation groups suggests that our null finding is not driven by any particular type of licensed occupation.

Second, we examine heterogeneity by worker demographics. We estimate separate models for workers in different age groups (25-34, 35-44, 45-60) and by sex. Younger workers show higher baseline migration rates but similar DiD and triple-difference estimates. Female workers, who are overrepresented in nursing and teaching, show patterns similar to male workers. These results suggest that demographic composition does not explain our findings.

Third, we examine whether effects vary by the strength of the ULR law. Missouri, Iowa, Idaho, and Utah enacted what researchers have characterized as ``strong'' ULR laws that do not require the originating state's license to be ``substantially similar'' to the destination state's requirements. Arizona, Montana, and Pennsylvania have somewhat more restrictive versions. When we estimate separate effects for strong versus moderate ULR states, we find that both types show similar DiD estimates for licensed workers (0.26 pp for strong, 0.21 pp for moderate) and both show triple-difference estimates close to zero.

Fourth, we examine whether effects emerge more strongly in counties closer to state borders, where interstate migration may be more salient. Using county-level identifiers in PUMS where available, we classify workers as residing in border counties (adjacent to another state) or interior counties. Border county residents show slightly higher baseline migration rates but similar DiD and triple-difference patterns. This finding is somewhat surprising given that border residents might be more responsive to reduced licensing barriers, but it is consistent with our overall null finding.

Finally, we examine time heterogeneity more carefully. For the early adopters (Arizona, Montana, Pennsylvania), we have three years of post-treatment data (2020 excluded, but 2021 and 2022 available plus partial 2020 exposure). For the 2020 adopters, we have only 2021 and 2022. When we restrict to early adopters only, the patterns remain similar. We also find no evidence that effects grow over time; if anything, the 2022 differences are smaller than 2021 differences, though this may reflect pandemic-related noise rather than a true declining pattern.

\subsection{Alternative Specifications}

We estimate several alternative specifications to assess robustness of our findings to methodological choices. First, we estimate the model using probit rather than linear probability, which may better capture the bounded nature of the migration outcome. The marginal effects from probit are nearly identical to the linear probability estimates, which is not surprising given the low baseline migration rates.

Second, we implement the Callaway and Sant'Anna (2021) estimator for staggered difference-in-differences, which addresses potential bias from heterogeneous treatment effects across groups and time. This estimator produces group-time average treatment effects that we then aggregate. The overall ATT estimate is 0.22 percentage points, very similar to our baseline estimate of 0.23 percentage points, suggesting that two-way fixed effects bias is not a major concern in our setting.

Third, we vary the treatment timing definition. In our baseline specification, states are coded as treated in the first full calendar year after enactment. As an alternative, we code states as treated in the year of enactment if the law took effect before July 1, and in the following year otherwise. This change affects the treatment status of Missouri and Iowa (both enacted mid-2020). The results are robust to this alternative timing definition.

Fourth, we conduct inference using wild cluster bootstrap with 1,000 replications to address potential issues with the small number of clusters (22 states). The bootstrapped confidence intervals are similar to those based on cluster-robust standard errors, and our conclusions remain unchanged.

Fifth, we estimate specifications with additional state-level controls including state GDP growth, unemployment rate, and housing price changes. These controls address the concern that ULR states may have been on different economic trajectories. The inclusion of these controls has minimal effect on the DiD and triple-difference estimates, providing additional support for our identification strategy.

\subsection{Sensitivity Analysis}

We conduct several sensitivity analyses to understand the conditions under which our null finding could be overturned. First, we calculate the minimum detectable effect size given our sample size and variance. With approximately 47,000 licensed workers in ULR states across the pre and post periods, and a baseline migration rate of approximately 2 percent, we have 80 percent power to detect an effect of approximately 0.15 percentage points, which corresponds to a 7.5 percent change in the migration rate. Our estimated triple-difference of -0.02 percentage points is well within this detectable range, and we can rule out effects larger than approximately 0.10 percentage points with 95 percent confidence.

Second, we conduct a placebo test using pre-treatment data only. We artificially code the early adopter states as ``treated'' beginning in 2018 (rather than 2020) and re-estimate the DiD model using 2017 as the pre-period and 2018-2019 as the post-period. The placebo DiD estimate is 0.08 percentage points and statistically insignificant, providing additional support for the parallel trends assumption.

Third, we examine whether our results could be explained by migration to ULR states from specific origin states. If ULR primarily facilitates migration from states with incompatible licensing requirements, we might expect to see effects concentrated among migrants from non-reciprocity states. However, we find that migration increased similarly from both reciprocity and non-reciprocity origin states, further supporting the interpretation that general migration trends rather than licensing-specific factors drove the observed patterns.

\section{Conclusion}

This paper examines the effect of Universal License Recognition laws on interstate migration using difference-in-differences and triple-difference designs with Census PUMS data. We find that while migration rates increased significantly in ULR states for workers in licensed occupations, an essentially identical increase occurred for workers in unlicensed occupations. The triple-difference estimate is effectively zero, indicating that the migration increase was not specific to workers affected by license portability.

This null finding suggests that the migration increases observed in ULR states during our sample period are not attributable to the reduction in licensing barriers but rather to other factors driving migration to these predominantly fast-growing states in the American West and Mountain regions. License portability may be less binding as a constraint on interstate migration than commonly assumed, at least for the broad categories of licensed occupations we examine.

Our results have important implications for policy evaluation. They suggest that naively comparing migration trends in ULR versus non-ULR states may overstate the causal effect of licensing reform by confounding licensing-specific effects with general state-level trends. Triple-difference designs that include unlicensed workers as a within-state control group provide a more credible approach to isolating licensing-specific effects.

Several avenues for future research emerge from our analysis. Longer time series would allow examination of whether licensing effects emerge as awareness of ULR spreads and workers have more time to respond to reduced barriers. Analysis of specific occupations with particularly burdensome or salient licensing requirements might reveal effects that are diluted in our broad occupational categories. Administrative data on license applications and issuances could provide more direct evidence on whether ULR changes the behavior of workers considering interstate moves. Finally, survey data on worker perceptions and awareness of licensing policies could help explain why licensing barriers may or may not bind in practice.

\newpage
\bibliographystyle{apalike}
\begin{thebibliography}{99}

\bibitem[Bae and Timmons, 2023]{bae2023}
Bae, K. and Timmons, E. (2023).
Now you can take it with you: Effects of occupational credential recognition on labor market outcomes.
\textit{SSRN Working Paper}.

\bibitem[Barton, 2016]{barton2016}
Barton, B. (2016).
The lawyer's monopoly: What goes and what stays.
\textit{Fordham Law Review}, 82, 3067--3090.

\bibitem[Blair and Chung, 2019]{blair2017}
Blair, P. and Chung, B. (2019).
How much of barrier to entry is occupational licensing?
\textit{British Journal of Industrial Relations}, 57(4), 919--943.

\bibitem[Census Bureau, 2021]{census2021}
U.S. Census Bureau (2021).
American Community Survey: Response rates and data quality.
\textit{Technical Documentation}.

\bibitem[Deyo and Plemmons, 2022]{deyo2022}
Deyo, D. and Plemmons, A. (2022).
Have license, will travel: Measuring the effects of universal licensing recognition on mobility.
\textit{Economics Letters}, 219.

\bibitem[Federman et al., 2006]{federman2006}
Federman, M., Harrington, D., and Krynski, K. (2006).
The impact of state licensing regulations on low-skilled immigrants.
\textit{Eastern Economic Journal}, 32(1), 113--128.

\bibitem[Friedman, 1962]{friedman1962}
Friedman, M. (1962).
\textit{Capitalism and Freedom}.
University of Chicago Press.

\bibitem[Goldhaber and Hansen, 2015]{goldhaber2015}
Goldhaber, D. and Hansen, M. (2015).
Is it just a bad class? Assessing the long-term stability of estimated teacher performance.
\textit{Economica}, 80(319), 589--612.

\bibitem[Hermansen, 2019]{hermansen2019}
Hermansen, M. (2019).
Occupational licensing and job mobility in the United States.
\textit{OECD Economics Department Working Papers}.

\bibitem[Johnson and Kleiner, 2018]{johnson2018}
Johnson, J. and Kleiner, M. (2018).
Is occupational licensing a barrier to interstate migration?
\textit{American Economic Journal: Economic Policy}.

\bibitem[Johnson, 2012]{johnson2012}
Johnson, J. (2012).
Occupational licensing and interstate migration.
\textit{Working Paper}.

\bibitem[Kaplan and Schulhofer-Wohl, 2017]{kaplan2017}
Kaplan, G. and Schulhofer-Wohl, S. (2017).
Understanding the long-run decline in interstate migration.
\textit{International Economic Review}, 58(1), 57--94.

\bibitem[Kleiner, 2000]{kleiner2000}
Kleiner, M. (2000).
Occupational licensing.
\textit{Journal of Economic Perspectives}, 14(4), 189--202.

\bibitem[Kleiner, 2006]{kleiner2006}
Kleiner, M. (2006).
\textit{Licensing Occupations: Ensuring Quality or Restricting Competition?}
W.E. Upjohn Institute.

\bibitem[Kleiner, 2015]{kleiner2015}
Kleiner, M. (2015).
\textit{Guild-ridden Labor Markets: The Curious Case of Occupational Licensing}.
W.E. Upjohn Institute.

\bibitem[Kleiner and Krueger, 2010]{kleiner2010}
Kleiner, M. and Krueger, A. (2010).
The prevalence and effects of occupational licensing.
\textit{British Journal of Industrial Relations}, 48(4), 676--687.

\bibitem[Leland, 1979]{leland1979}
Leland, H. (1979).
Quacks, lemons, and licensing: A theory of minimum quality standards.
\textit{Journal of Political Economy}, 87(6), 1328--1346.

\bibitem[Molloy et al., 2017]{molloy2017}
Molloy, R., Smith, C., and Wozniak, A. (2017).
Job changing and the decline in long-distance migration in the United States.
\textit{Demography}, 54(2), 631--653.

\bibitem[Shapiro, 1986]{shapiro1986}
Shapiro, C. (1986).
Investment, moral hazard, and occupational licensing.
\textit{Review of Economic Studies}, 53(5), 843--862.

\bibitem[Thornton and Timmons, 2010]{thornton2010}
Thornton, R. and Timmons, E. (2010).
The de-licensing of occupations in the United States.
\textit{Monthly Labor Review}, 38--46.

\bibitem[Timmons and Norris, 2017]{timmons2017}
Timmons, E. and Norris, A. (2017).
The hair police.
\textit{Regulation}, 40(1), 24--29.

\bibitem[Timmons and Thornton, 2018]{timmons2018}
Timmons, E. and Thornton, R. (2018).
The effects of licensing on the wages of radiologic technologists.
\textit{Journal of Labor Research}, 31, 333--345.

\end{thebibliography}

\newpage
\appendix
\section{Pre-Analysis Plan}

This analysis was conducted following a pre-registered analysis plan that was cryptographically locked before data analysis on January 17, 2026. The pre-analysis plan specified the research question, hypotheses, sample construction, treatment definition, empirical specifications, and robustness checks in advance.

The key elements of the pre-analysis plan included the following. The primary hypothesis stated that ULR states would experience increased in-migration from non-ULR states among workers in licensed occupations, relative to workers in non-licensed occupations. The primary method was difference-in-differences with state and year fixed effects. The sample was specified as PUMS 2017-2022 (excluding 2020 due to COVID-19), adults aged 25-60 in the civilian labor force. The primary outcome was the interstate migration rate. The secondary analysis employed a triple-difference using unlicensed workers as an additional control group.

The pre-analysis plan checksum is stored in the repository and its timing can be verified against the GitHub commit history, providing transparent evidence that the analysis plan was locked before results were obtained.

\section{Data and Replication}

All data used in this analysis are from the public Census PUMS API and are freely available without restriction. The analysis code, including data cleaning, estimation, and figure generation, is available in the paper repository. Researchers can replicate all results using the provided code and publicly available data.

\section{Variable Definitions}

This appendix provides detailed definitions for all variables used in the analysis.

\textbf{Interstate Migration (MIGSP).} The PUMS variable MIGSP records the state of residence one year prior to the survey. We code individuals as interstate migrants if their current state (ST) differs from their prior state (MIGSP) and both states are valid U.S. state FIPS codes. Individuals who moved from abroad or who have missing migration information are coded as non-migrants.

\textbf{Licensed Occupation (OCCP).} We classify occupations as licensed based on their OCCP codes. Healthcare occupations include: Registered Nurses (3255), Licensed Practical and Vocational Nurses (3256), Physicians and Surgeons (3060-3090), Dentists (3010), Pharmacists (3050), Physical Therapists (3160), and other allied health professionals. Education occupations include: Preschool and Kindergarten Teachers (2300), Elementary and Middle School Teachers (2310), Secondary School Teachers (2320), and Special Education Teachers (2330-2340). Personal service occupations include: Barbers (4500), Hairdressers, Hairstylists, and Cosmetologists (4510-4521), and Skincare Specialists (4530). Skilled trades include: Electricians (6355), Plumbers, Pipefitters, and Steamfitters (6440), and HVAC Mechanics (7315). Professional services include: Real Estate Brokers and Sales Agents (4920) and Accountants and Auditors (0800). We exclude lawyers (2100) due to their different mobility patterns through bar reciprocity.

\textbf{Employment Status (ESR).} The PUMS variable ESR records employment status. We code individuals as employed if ESR equals 1 (civilian employed, at work) or 2 (civilian employed, with a job but not at work). We code individuals as unemployed if ESR equals 3. We restrict the sample to those in the labor force (ESR in 1, 2, or 3).

\textbf{Wages (WAGP).} Annual wages and salary income. We use this variable for descriptive statistics only. Values are top-coded by the Census Bureau.

\textbf{Person Weight (PWGTP).} The person weight variable is used to calculate population-representative statistics. All reported means and regression estimates incorporate these weights.

\textbf{Age (AGEP).} Age in years. We restrict the sample to ages 25-60 to focus on prime working-age individuals.

\textbf{State (ST).} Current state of residence, recorded as a two-digit FIPS code.

\textbf{ULR Treatment.} A state-year level indicator equal to one if the state had enacted Universal License Recognition by that year. See Table 1 for treatment timing by state.

\section{Sample Construction Details}

The analysis sample is constructed as follows. We begin with all person records from the PUMS one-year files for 2017, 2018, 2019, 2021, and 2022. We exclude 2020 due to well-documented data collection issues during the COVID-19 pandemic that affected response rates and potentially introduced non-random selection.

We then apply the following restrictions sequentially. First, we restrict to individuals aged 25-60, which excludes 48 percent of the initial sample. Second, we restrict to individuals in the civilian labor force (ESR equals 1, 2, or 3), which excludes an additional 32 percent of the remaining sample. Third, we restrict to residents of the 22 largest states plus all ULR-adopting states, which excludes 15 percent of the remaining sample. Fourth, we require valid state and migration information, which excludes less than 1 percent of the remaining sample. The final sample contains 4,313,840 observations.

For the licensed occupation subsample, we further restrict to workers whose OCCP code falls within our list of licensed occupations. This produces a subsample of 324,431 observations representing 7.5 percent of the main sample. The unlicensed occupation subsample contains the remaining 3,989,409 observations.

For the event study analysis, we focus on the early adopter states (Arizona, Montana, Pennsylvania) which provide both pre-treatment and post-treatment observations within our sample period. The later adopters (Idaho, Utah, Missouri, Iowa) contribute post-treatment observations only, as their treatment begins in 2021.

For the heterogeneity analyses by occupation group, we partition the licensed occupation sample into healthcare (approximately 60 percent of licensed workers), education (approximately 25 percent), personal services (approximately 10 percent), and skilled trades (approximately 5 percent). Due to the small sample size of skilled trades workers, estimates for that subgroup have larger standard errors.

\end{document}
