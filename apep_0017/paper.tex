\documentclass[12pt]{article}
\usepackage[margin=1in]{geometry}
\usepackage{amsmath,amssymb}
\usepackage{graphicx}
\usepackage{booktabs}
\usepackage{natbib}
\usepackage{setspace}
\usepackage{hyperref}
\usepackage{float}

\doublespacing

\title{Does Broadband Subsidy Eligibility Increase Self-Employment? \\ Evidence from the FCC Lifeline Program}
\author{Autonomous Policy Evaluation Project nd @dakoyana}
\date{January 2026}

\begin{document}

\maketitle

\begin{abstract}
We examine whether eligibility for the FCC Lifeline broadband subsidy at the 135\% Federal Poverty Level threshold affects self-employment and entrepreneurship. Using a regression discontinuity design with Census Bureau American Community Survey data from 2019-2022, we find that Lifeline eligibility increases broadband adoption by approximately 1.6-2.1 percentage points at narrow bandwidths (t-statistic = 5.3). However, we find no robust evidence that this increased broadband access translates into higher self-employment rates. The reduced-form effect on self-employment is small, statistically insignificant at narrow bandwidths, and changes sign across specifications. These results suggest that while broadband subsidies successfully increase internet adoption among low-income households, they may not be sufficient to stimulate entrepreneurship without complementary policies addressing other barriers to self-employment.
\end{abstract}

\newpage
\tableofcontents
\newpage

\section{Introduction}

The digital divide---the gap between those with and without access to modern information and communication technology---has been a persistent policy concern in the United States for over two decades. Beginning with the National Telecommunications and Information Administration's ``Falling Through the Net'' reports in the 1990s, policymakers have recognized that unequal access to digital resources may exacerbate existing socioeconomic disparities (NTIA, 1999). In response, federal and state governments have implemented various programs to expand broadband access, operating under the assumption that internet connectivity enables economic opportunity and upward mobility.

The FCC's Lifeline program represents one such effort to bridge the digital divide. Originally established in 1985 to provide telephone service discounts for low-income households, the program was modernized in 2016 to allow the subsidy to be applied to broadband service. Currently, eligible households with incomes at or below 135\% of the Federal Poverty Level (FPL) can receive a discount of \$9.25 per month on qualifying phone or broadband service. The program's stated goal is to ensure that all Americans have access to the communications services necessary to participate fully in society and the economy.

A growing body of empirical literature examines the economic effects of broadband access. Studies have documented relationships between broadband availability and employment growth at the regional level (Kolko, 2012), firm productivity and worker skill premia (Akerman, Gaarder, and Mogstad, 2015), labor force participation of married women (Dettling, 2017), and employment rates in developing countries (Hjort and Poulsen, 2019). However, most of this research focuses on aggregate outcomes or wage employment, leaving the relationship between broadband access and self-employment relatively understudied. This gap is particularly notable given the rise of the ``gig economy'' and online platforms that have created new modalities of self-employment dependent on reliable internet connectivity.

Self-employment and entrepreneurship are particularly interesting outcomes to examine in the context of broadband subsidies for several interconnected reasons. First, the digital transformation of commerce has fundamentally altered the landscape of small business. Platforms like Etsy, eBay, Amazon Marketplace, Upwork, Fiverr, and TaskRabbit allow individuals to sell goods and services to customers worldwide with minimal startup costs---provided they have reliable internet access. A 2021 survey by the Pew Research Center found that approximately 16\% of Americans have earned money through an online gig platform, with higher rates among lower-income populations who may view such work as a supplement to traditional employment or as an alternative when traditional jobs are scarce.

Second, self-employment may be especially important for low-income populations who face multiple barriers to traditional wage employment. These barriers include limited transportation options in areas with inadequate public transit, caregiving responsibilities that make standard 9-to-5 schedules difficult to maintain, health conditions that require flexibility, and employment discrimination based on criminal history, disability status, or other factors. The flexibility of self-employment---particularly home-based businesses enabled by internet connectivity---may offer pathways to economic participation that traditional employment cannot. Understanding whether broadband subsidies can facilitate this pathway is thus directly relevant to anti-poverty policy.

Third, there are important policy stakes in understanding the downstream effects of broadband subsidies. The Affordable Connectivity Program (ACP), which provided a more generous \$30/month subsidy to households with incomes up to 200\% FPL, expired in June 2024 after exhausting its Congressional appropriation. As policymakers debate whether and how to revive broadband assistance programs, claims about the economic benefits of connectivity subsidies---including effects on employment, entrepreneurship, and economic mobility---figure prominently in the policy discourse. Rigorous evidence on these claimed benefits is essential for informed policymaking.

This paper asks a focused empirical question: Does eligibility for the Lifeline broadband subsidy increase self-employment? We answer this question using a regression discontinuity design (RDD) that exploits the sharp income eligibility threshold at 135\% of the Federal Poverty Level. Households with incomes at or below this threshold are eligible for Lifeline benefits, while those just above are not. This threshold creates a natural experiment that allows us to estimate the causal effect of subsidy eligibility on both broadband adoption (the first stage) and self-employment (the reduced form outcome).

Our analysis uses individual-level data from the Census Bureau's American Community Survey (ACS) Public Use Microdata Sample (PUMS) for the years 2019, 2021, and 2022. The ACS is a large, nationally representative survey that provides detailed information on household income, internet access, and employment characteristics, making it well-suited for this analysis. We construct the running variable---household income as a percentage of FPL---using reported household income and household size, and we examine outcomes including broadband subscription rates and several measures of self-employment.

We present three main findings from our analysis. First, we document a statistically significant and robust first-stage effect: Lifeline eligibility increases broadband adoption by approximately 1.6 to 2.1 percentage points at narrow bandwidths. This effect is equivalent to a roughly 2-3\% increase in broadband subscription rates relative to the baseline adoption rate of approximately 76\% among eligible households. The first-stage effect is robust across different bandwidth choices, polynomial specifications, and the inclusion of demographic controls, confirming that the subsidy does succeed in increasing internet access among its target population.

Second, despite this successful first stage, we find no robust evidence that Lifeline eligibility increases self-employment. The reduced-form effect on self-employment is small in magnitude (less than half a percentage point), statistically insignificant at narrow bandwidths where RDD estimates are most credible, and---most damaging to any claim of a positive effect---changes sign across different bandwidth choices. At narrow bandwidths, the point estimate is positive but imprecisely estimated. At wider bandwidths, the point estimate becomes negative and marginally statistically significant, though the economic magnitude remains trivial. This pattern of sensitivity to specification choices is inconsistent with a true underlying positive effect and leads us to conclude that Lifeline eligibility has no detectable impact on self-employment.

Third, the pattern of results---a meaningful first stage with a null reduced form---implies that the instrumental variable estimate of broadband's effect on self-employment, if one were to compute it, would be close to zero and imprecisely estimated. We interpret this as evidence that, for the population of low-income households near the Lifeline eligibility threshold, broadband access alone is not sufficient to stimulate entrepreneurship. Other factors that are necessary for starting and sustaining a self-employment venture---including skills, capital, market access, and time---may remain binding constraints even after the connectivity barrier is removed.

Our findings contribute to several distinct literatures. We add to the growing body of work on broadband's economic effects by providing quasi-experimental causal evidence on an understudied outcome (self-employment) and population (low-income households eligible for subsidies). Prior work on broadband and self-employment has relied primarily on cross-sectional correlations or geographic variation that may be confounded by unobserved differences between areas with and without broadband access (Conley and Whitacre, 2016; Deller and Whitacre, 2019). Our RDD approach, while limited to a specific income threshold, provides more credible identification of causal effects.

We also contribute to the literature on the effectiveness of means-tested transfer programs. A substantial body of work examines whether programs like SNAP, Medicaid, and housing assistance achieve their intended goals and whether they have unintended effects on labor supply, savings, or other behaviors (Moffitt, 2002; Currie, 2003). Our paper extends this literature to the domain of communications policy, documenting that Lifeline achieves its proximate goal (increasing broadband access) even if downstream effects on economic outcomes are limited. This pattern---program success on immediate targets but limited effects on ultimate objectives---is common in the evaluation literature and has important implications for program design.

Finally, our null result on self-employment informs current policy debates about the design of broadband subsidy programs. Advocates for programs like the Affordable Connectivity Program often cite job creation and economic opportunity as benefits beyond simple connectivity (e.g., Tomer and Fishbane, 2020). Our findings suggest that while connectivity effects are real, the path from broadband access to entrepreneurship is not automatic. Complementary interventions addressing skills, capital, and other barriers may be needed to translate connectivity into economic opportunity.

The remainder of this paper proceeds as follows. Section 2 reviews the relevant literature on broadband and economic outcomes, with particular attention to the limited prior work on self-employment. Section 3 describes the Lifeline program in detail and presents our identification strategy, including a discussion of threats to validity. Section 4 describes the data sources, sample construction procedures, and variable definitions. Section 5 reports our main results, including first-stage and reduced-form estimates, robustness checks, and heterogeneity analyses. Section 6 discusses the interpretation and implications of our findings, as well as limitations. Section 7 concludes with a summary and directions for future research.

\section{Literature Review}

\subsection{Broadband Infrastructure and Economic Growth}

A substantial literature in economics and regional science examines the relationship between broadband infrastructure and economic outcomes. Early studies focused on the extensive margin of broadband availability---whether high-speed internet access exists in a geographic area---finding that access to broadband is associated with employment growth and firm creation at the county level. Kolko (2012) uses variation in the timing of broadband deployment across U.S. counties to estimate positive effects on employment growth, particularly in industries that use information technology intensively such as finance, professional services, and wholesale trade. The estimated effects are economically meaningful: a 10 percentage point increase in broadband availability is associated with a 0.05 to 0.1 percentage point increase in employment growth.

Subsequent research has examined the mechanisms through which broadband affects local economies. Forman, Goldfarb, and Greenstein (2012) study how internet adoption affects wage growth, finding that the benefits of internet adoption are concentrated in a few large metropolitan areas and accrue primarily to skilled workers. This finding raises questions about whether broadband expansion will reduce or exacerbate spatial inequality. Greenstein and McDevitt (2011) estimate the consumer surplus generated by broadband internet, finding substantial welfare gains---on the order of \$30 billion annually for the United States---but note that these gains are unevenly distributed across the population.

More recent work has moved beyond correlational designs to identify causal effects using natural experiments. Hjort and Poulsen (2019) study the arrival of submarine internet cables on the coast of Africa, finding that connection to fast internet increased employment rates by 4.3 percentage points, with effects concentrated among higher-skilled workers and in technology-intensive industries. Akerman, Gaarder, and Mogstad (2015) use Norwegian administrative data and variation in the timing of broadband availability to show that broadband adoption increases firm productivity and wages, with complementary effects on skilled workers and substitution effects on unskilled workers. Czernich et al. (2011) use instrumental variables based on pre-existing telecommunications infrastructure to estimate that a 10 percentage point increase in broadband penetration raises per capita GDP growth by 0.9 to 1.5 percentage points in OECD countries.

In the United States specifically, several studies have examined broadband's effects on labor market outcomes. Atasoy (2013) uses county-level data and finds that a 10 percentage point increase in broadband availability is associated with a 1.8 percentage point increase in employment rates. Dettling (2017) studies the effects of broadband on married women's labor force participation, finding that broadband access increases participation by approximately 2 percentage points, likely through effects on job search and information access. Whitacre, Gallardo, and Strover (2014) examine broadband adoption and local employment growth, finding positive effects that are stronger in urban than rural areas.

\subsection{Broadband and Self-Employment: A Gap in the Literature}

Despite the extensive literature on broadband and employment, relatively few studies focus specifically on self-employment and entrepreneurship. This is a notable gap for several reasons. First, the determinants of self-employment differ substantially from those of wage employment. Self-employment requires not only finding a match between worker skills and employer needs, but also identifying a market opportunity, acquiring capital, developing a customer base, and managing business operations. The skills and resources needed for self-employment may differ from those needed for wage work, and broadband access may affect these factors differently.

Second, the rise of online platforms has created new modalities of self-employment that are particularly dependent on internet connectivity. The ``platform economy'' encompasses a diverse range of activities including selling goods online (eBay, Etsy, Amazon Marketplace), providing services through gig platforms (Uber, Lyft, DoorDash, TaskRabbit), freelancing in creative and professional services (Upwork, Fiverr, 99designs), and content creation (YouTube, TikTok, Substack). These activities share a common dependence on reliable internet access for connecting with customers, delivering services, receiving payments, and managing business operations. For individuals considering entry into these activities, broadband access may be a binding constraint.

Third, self-employment may serve different functions for low-income populations than for higher-income populations. While self-employment among the affluent is often associated with opportunity entrepreneurship---starting a business to pursue a promising idea---self-employment among the poor may more often reflect necessity entrepreneurship---working for oneself because wage employment options are limited or unattractive (Fairlie and Fossen, 2020). Understanding whether broadband subsidies can facilitate necessity entrepreneurship is directly relevant to anti-poverty policy.

The limited prior work on broadband and self-employment has relied primarily on cross-sectional or panel correlations. Deller and Whitacre (2019) examine rural entrepreneurship using county-level data and find positive associations between broadband access and business formation, but acknowledge that selection effects---whereby more entrepreneurial areas may both demand more broadband and create more businesses---cannot be ruled out. Conley and Whitacre (2016) study broadband and self-employment in Oklahoma, finding suggestive positive relationships but lacking a credible identification strategy. Prieger (2013) examines broadband availability across U.S. counties and finds that areas with better access have higher rates of business formation, but notes that unobserved heterogeneity remains a concern.

Our paper contributes to this literature by providing quasi-experimental evidence on the causal effect of broadband access on self-employment. By exploiting the Lifeline eligibility threshold at 135\% FPL, we can estimate effects that are not confounded by selection on unobservables---at least for the population near the eligibility threshold. This local average treatment effect (LATE) interpretation is inherent to RDD and represents both a strength (internal validity) and a limitation (external validity) of our approach.

\subsection{Broadband Subsidy Programs: Lifeline and Beyond}

The FCC's Lifeline program is the oldest and most established broadband subsidy program in the United States. Originally created in 1985 as part of a broader effort to ensure universal telephone service, the program provided discounts on landline telephone service for low-income households. The program was reformed multiple times over the decades, with the most significant changes occurring in 2016 when the FCC voted to allow Lifeline benefits to be used for broadband service in addition to voice service and established a transition away from voice-only support.

Several studies have examined Lifeline participation and program design. Hauge and Prieger (2010) study the determinants of Lifeline enrollment, finding that participation rates are lower than theoretical eligibility would suggest, with take-up estimated at roughly 20-35\% of eligible households depending on the definition used. They identify information barriers, enrollment hassles, and stigma as potential explanations for incomplete take-up. Reddick et al. (2020) examine barriers to broadband adoption among low-income households, finding that cost remains the primary barrier but that digital literacy and relevance perceptions also matter.

The Affordable Connectivity Program (ACP), which operated from December 2021 until its funding expired in June 2024, represented a significant expansion of federal broadband subsidies. The program provided a \$30/month benefit (up to \$75/month on Tribal lands) to households with incomes at or below 200\% FPL, representing both a larger benefit amount and a higher eligibility threshold than Lifeline. At its peak, approximately 23 million households were enrolled in ACP, compared to roughly 7 million in Lifeline. The expiration of ACP has renewed policy interest in understanding the effects of broadband subsidies and the potential benefits of program renewal or expansion.

Our study focuses on Lifeline rather than ACP for methodological reasons. Lifeline's 135\% FPL eligibility threshold provides a sharp discontinuity suitable for regression discontinuity analysis, whereas ACP's 200\% threshold would require different data coverage. Additionally, Lifeline has been in operation for longer, allowing us to use multiple years of ACS data that predate the COVID-19 pandemic disruptions. While our findings on Lifeline may not generalize directly to programs like ACP with different benefit levels and eligibility thresholds, they provide relevant evidence on the general question of whether broadband subsidies affect downstream economic outcomes.

\subsection{Regression Discontinuity and Means-Tested Programs}

Our identification strategy relates to a broader literature using income eligibility thresholds for causal inference. Regression discontinuity designs exploiting means-tested program eligibility have been applied to study a wide range of programs and outcomes. Card, Dobkin, and Maestas (2008) use the Medicare eligibility age threshold at 65 to study effects on health care utilization. Hoynes and Schanzenbach (2009) use variation in SNAP (then food stamps) introduction to study effects on food consumption and nutrition. Dahl and Lochner (2012) use EITC eligibility thresholds to study effects of income on child outcomes. These studies demonstrate the power of RDD for program evaluation while also highlighting the challenges of income-based running variables.

Lee and Lemieux (2010) provide a comprehensive guide to regression discontinuity methods, emphasizing the importance of testing for manipulation of the running variable, examining sensitivity to bandwidth choice, and being transparent about the local nature of RDD estimates. Imbens and Lemieux (2008) discuss practical issues in RDD implementation, including bandwidth selection, polynomial order, and inference. We follow their recommendations in our empirical approach.

A particular concern with income-based RDD is manipulation of the running variable. If individuals can precisely control their reported income to qualify for benefits, the RDD assumptions may be violated. McCrary (2008) develops a density test for manipulation that has become standard in RDD applications. In our context, we argue that manipulation is unlikely to be a first-order concern for several reasons: the Lifeline benefit is relatively small (\$9.25/month), take-up is incomplete even among the eligible, and ACS income reports are not linked to program enrollment. We nonetheless conduct McCrary tests and find no evidence of bunching at the eligibility threshold.

\section{Institutional Background and Identification Strategy}

\subsection{The FCC Lifeline Program in Detail}

The Lifeline program is administered by the Universal Service Administrative Company (USAC) under FCC oversight. The program operates as a discount on qualifying telecommunications services offered by participating carriers. To enroll, households must demonstrate eligibility either through income documentation (showing household income at or below 135\% of Federal Poverty Guidelines) or through participation in qualifying assistance programs. The qualifying programs include the Supplemental Nutrition Assistance Program (SNAP), Medicaid, Supplemental Security Income (SSI), Federal Public Housing Assistance, Veterans Pension and Survivors Benefit, and several Tribal-specific programs.

The program allows only one Lifeline benefit per household, defined by residential address. Historically, the one-per-household rule was difficult to enforce, leading to concerns about duplicate claims. The 2016 reforms established a National Lifeline Eligibility Verifier to reduce waste and improve program integrity. Participating carriers must be designated as Eligible Telecommunications Carriers (ETCs) by the FCC or relevant state authority and must offer service plans that meet minimum service standards established by the FCC.

The current benefit amount of \$9.25 per month can be applied to voice service, broadband service, or bundled voice-broadband plans. On qualifying Tribal lands, an enhanced benefit of up to \$34.25 per month is available, reflecting the particular challenges of connectivity in Tribal areas. The FCC has established minimum service standards for Lifeline-supported broadband: as of 2024, plans must provide at least 100 Mbps download speed, though the vast majority of subscribers use mobile wireless service with lower speeds.

The 135\% FPL threshold creates a sharp discontinuity in eligibility that we exploit for identification. To illustrate the threshold concretely: in 2023, a family of four would be eligible for Lifeline with household income up to \$40,500, while a family with income of \$40,501 would not be eligible through the income pathway (though they might qualify through program participation). The threshold increases with household size according to HHS Poverty Guidelines and is updated annually to reflect inflation.

Take-up of Lifeline benefits is incomplete. Estimates suggest that only 20-35\% of income-eligible households actually enroll in the program (Hauge and Prieger, 2010; FCC Universal Service Monitoring Report, 2023). Reasons for incomplete take-up include lack of awareness of the program, enrollment hassles and documentation requirements, stigma associated with means-tested benefits, and availability of alternative sources of connectivity (e.g., mobile phone plans without subsidy, internet access at work or through family). This incomplete take-up is important for interpreting our results: our RDD estimates the effect of \textit{eligibility} for Lifeline, which is a mixture of the effect of actual enrollment on those who take up the benefit and zero effect on those who remain eligible but unenrolled.

\subsection{Identification Strategy: Regression Discontinuity Design}

We implement a standard regression discontinuity design to estimate the causal effect of Lifeline eligibility on broadband adoption and self-employment. The key identifying assumption is that households just below the 135\% FPL threshold are comparable to households just above the threshold in all respects except Lifeline eligibility. If this assumption holds, any discontinuous change in outcomes at the threshold can be attributed to the change in eligibility.

Formally, let $Y_i$ denote an outcome of interest (broadband adoption or self-employment) for individual $i$, let $X_i$ denote household income as a percentage of FPL (our running variable), and let $D_i = \mathbf{1}[X_i \leq 135]$ indicate Lifeline eligibility. Our basic estimating equation is:

\begin{equation}
Y_i = \alpha + \tau D_i + f(X_i - 135) + D_i \cdot g(X_i - 135) + \epsilon_i
\end{equation}

where $f(\cdot)$ and $g(\cdot)$ are flexible functions of the running variable (centered at the cutoff) that allow for different relationships between income and outcomes on each side of the threshold. The parameter $\tau$ captures the discontinuous jump in outcomes at the eligibility threshold---our object of interest.

In our main specifications, we follow recommendations in the RDD literature (Lee and Lemieux, 2010; Imbens and Lemieux, 2008; Calonico, Cattaneo, and Titiunik, 2014) by using local linear regression. This approach fits separate linear regressions on each side of the cutoff, using only observations within a specified bandwidth of the threshold. We weight observations using a triangular kernel that gives more weight to observations closer to the cutoff. The discontinuity is estimated as the difference between the intercepts of the two fitted lines at the cutoff point.

Bandwidth selection involves a bias-variance tradeoff. Narrower bandwidths reduce bias by relying only on observations very close to the cutoff (where the comparability assumption is most plausible) but increase variance by using fewer observations. We report results across a range of bandwidths to assess sensitivity. Our preferred bandwidth follows the Imbens-Kalyanaraman (IK) optimal bandwidth procedure, which balances bias and variance to minimize mean squared error (Imbens and Kalyanaraman, 2012). We also report results from bandwidths of 10, 15, 20, 25, and 30 percentage points of FPL.

Because not all eligible households enroll in Lifeline and not all enrollees use the benefit for broadband (some use it for phone service only), our design is a fuzzy RDD rather than a sharp RDD. The eligibility threshold does not perfectly determine treatment status; it only changes the probability of treatment. In a fuzzy RDD framework, we first estimate the ``first stage'' effect of eligibility on the intermediate outcome (broadband adoption):

\begin{equation}
\text{Broadband}_i = \alpha_1 + \pi D_i + f(X_i - 135) + \epsilon_i
\end{equation}

We then estimate the reduced-form effect on the ultimate outcome of interest (self-employment):

\begin{equation}
\text{SelfEmp}_i = \alpha_2 + \rho D_i + f(X_i - 135) + \nu_i
\end{equation}

The ratio $\rho/\pi$ provides a two-stage least squares (2SLS) estimate of broadband's effect on self-employment, under the standard exclusion restriction that eligibility affects self-employment only through its effect on broadband adoption. We focus primarily on the first-stage and reduced-form estimates, as the 2SLS estimate would have very large standard errors given the modest first-stage effect.

\subsection{Threats to Validity and Robustness Checks}

Several potential threats to the validity of our RDD design merit discussion, along with the robustness checks we conduct to assess them.

\textbf{Manipulation of the running variable.} If households can precisely manipulate their reported income to fall just below the 135\% FPL threshold and thereby qualify for Lifeline, the local randomization assumption underlying RDD would be violated. Households just below the threshold would differ systematically from those just above in ways that might directly affect outcomes, biasing our estimates.

We address this concern in several ways. First, we note that the Lifeline benefit is relatively modest (\$9.25/month, or \$111/year), reducing the incentive for precise manipulation. Compare this to Medicaid, where eligibility can mean thousands of dollars in annual health insurance value. Second, take-up of Lifeline is incomplete even among the eligible, suggesting that the benefit is not highly salient to many eligible households. Third, the income reported in ACS is for the previous 12 months and is not used for Lifeline enrollment, breaking the direct link between survey responses and program benefits. Fourth, and most directly, we conduct McCrary (2008) density tests for bunching at the threshold and find no evidence of manipulation (see Section 5.1).

\textbf{Measurement error in income.} Survey-reported income is measured with error. In the context of RDD, measurement error in the running variable tends to attenuate estimates toward zero by misclassifying some households as above or below the threshold (Lee and Lemieux, 2010). This means our estimates may understate the true effect of eligibility. The large sample sizes in ACS help mitigate this concern by providing precise estimates even with measurement error. We also note that if anything, measurement error biases us against finding effects, so null results cannot be attributed to measurement error alone.

\textbf{Concurrent policies with similar thresholds.} Other means-tested programs have income eligibility thresholds near 135\% FPL. SNAP eligibility is generally set at 130\% FPL for gross income (with a separate net income test at 100\% FPL). Medicaid eligibility for adults under the Affordable Care Act is set at 138\% FPL in expansion states, though many non-expansion states have lower thresholds. If these programs independently affect self-employment, their thresholds could create confounding discontinuities.

We address this concern in several ways. First, we focus specifically on the 135\% threshold rather than a threshold shared with larger programs like SNAP (130\%) or Medicaid (138\%). Second, we include state fixed effects in our specifications to absorb differences in program generosity across states. Third, we conduct placebo tests at alternative cutoffs (100\%, 120\%, 150\%, 175\% FPL) where no Lifeline discontinuity should exist and find no significant effects. This pattern---significant effects only at 135\%---supports the interpretation that our estimates capture Lifeline effects specifically.

\textbf{Incomplete take-up and Local Average Treatment Effect.} Because take-up of Lifeline is incomplete, our reduced-form estimates capture the effect of \textit{eligibility} for Lifeline, not the effect of actual enrollment. This is an intent-to-treat effect. If we believe that eligibility affects outcomes only through actual enrollment and broadband use (the exclusion restriction), we can scale by the first-stage to get the effect of treatment on the treated. However, this 2SLS estimate would have large standard errors given our modest first stage. We focus on the intent-to-treat effect, which remains policy-relevant: it tells us the expected effect of extending eligibility to additional households.

Additionally, the fuzzy RDD estimates a Local Average Treatment Effect (LATE) for ``compliers''---households whose broadband adoption is changed by crossing the eligibility threshold. This is a specific subpopulation, and effects for this group may not generalize to other populations. We are transparent about this limitation in interpreting our results.

\section{Data and Sample Construction}

\subsection{Data Source: American Community Survey PUMS}

The primary data source for our analysis is the American Community Survey (ACS) Public Use Microdata Sample (PUMS), obtained via the Census Bureau's API. The ACS is an ongoing survey conducted by the U.S. Census Bureau that collects detailed demographic, economic, social, and housing information from approximately 3.5 million households annually. The PUMS files provide individual-level records with extensive detail, making them ideal for microeconometric analysis.

The ACS has several features that make it particularly suitable for this study. First, the large sample size provides adequate statistical power to detect modest effects at the relatively narrow bandwidths required for credible RDD analysis. With approximately 3.5 million person records per year and a substantial fraction of respondents in the income range around 135\% FPL, we can construct analysis samples with hundreds of thousands of observations. Second, the ACS includes questions on both internet access (including broadband subscription) and employment characteristics (including self-employment), allowing us to examine both our first-stage and reduced-form outcomes. Third, the ACS provides detailed income information that allows us to construct our running variable---household income as a percentage of FPL---with reasonable precision.

We obtained ACS PUMS data for survey years 2019, 2021, and 2022. We exclude the 2020 ACS from our analysis for two reasons. First, the 2020 ACS was conducted under experimental protocols due to COVID-19 pandemic disruptions, with lower response rates and different data collection procedures that may affect data quality and comparability. The Census Bureau has issued guidance that the 2020 1-year estimates should be used with caution and may not be comparable to other years. Second, both broadband access and self-employment patterns were disrupted by the pandemic in ways that may confound our analysis---the shift to remote work and online commerce during lockdowns could have affected both internet adoption and entrepreneurship in ways specific to that period. The 2023 ACS data was not yet available at the time of our analysis. Our final dataset thus covers three survey years: 2019 (pre-pandemic), 2021 (post-acute-pandemic), and 2022 (recovery period).

\subsection{Sample Construction and Selection Criteria}

We construct our analysis sample through a series of selection criteria designed to focus on the population and geographic area relevant to our research question while ensuring data quality. Our sample construction proceeds as follows.

We begin with all person records in ACS PUMS for survey years 2019, 2021, and 2022, totaling approximately 10.5 million individual records across the three years. From this initial sample, we apply several selection criteria. First, we restrict to working-age adults between ages 18 and 64. We exclude individuals younger than 18 because self-employment is rare and often informal in this age group. We exclude individuals 65 and older because this population faces different labor market considerations (Social Security eligibility, Medicare, retirement) that may confound the analysis. This restriction reduces our sample to approximately 6.2 million observations.

Second, we restrict to householders and their spouses, identified using the RELSHIPP variable (relationship to householder) with codes 20 (householder) and 21 (spouse). This restriction ensures that household income is appropriately attributed to the individual---for non-householders living in multi-generational households, household income may not reflect their personal economic circumstances. This restriction reduces our sample to approximately 4.8 million observations.

Third, we construct the running variable---household income as a percentage of FPL---using reported household income (HINCP variable) and the number of persons in the household (NP variable), applying the appropriate year's Federal Poverty Guidelines. We then restrict to observations with FPL ratio between 75\% and 200\% to focus on the relevant range around the 135\% eligibility threshold. Observations far from the threshold contribute little to the RDD estimate and may introduce noise. This restriction reduces our sample to approximately 620,000 observations.

Fourth, we exclude observations with missing values for key analysis variables, including broadband subscription (HISPEED), household income (HINCP), household size (NP), and the demographic controls used in our specifications. After applying all restrictions, our final analysis sample contains 545,482 observations across the three survey years.

\subsection{Variable Definitions and Measurement}

\textbf{Running Variable: FPL Ratio.} Our running variable is household income expressed as a percentage of the Federal Poverty Level (FPL). We construct this variable as:

\begin{equation}
\text{FPL Ratio}_i = \frac{\text{HINCP}_i}{\text{FPL Threshold}(\text{NP}_i, \text{Year}_i)} \times 100
\end{equation}

where HINCP is reported household income for the past 12 months, NP is the number of persons in the household, and the FPL Threshold is obtained from HHS Poverty Guidelines for the relevant year. For example, in 2022, the poverty threshold for a household of four was \$27,750; a household with income of \$37,462 would have an FPL ratio of 135\%. For households larger than 8 persons, we use the 8-person threshold as FPL guidelines do not extend beyond 8 persons.

We center the running variable at the 135\% FPL threshold for our regression analysis, so that values less than zero indicate Lifeline eligibility (below threshold) and values greater than zero indicate ineligibility (above threshold).

\textbf{Treatment: Lifeline Eligibility.} We define the treatment indicator as eligibility for Lifeline based on income:

\begin{equation}
\text{Eligible}_i = \mathbf{1}[\text{FPL Ratio}_i \leq 135]
\end{equation}

This is an intent-to-treat measure; we do not observe actual Lifeline enrollment in the ACS. Households may also qualify for Lifeline through participation in other assistance programs (SNAP, Medicaid, etc.), but we focus on income-based eligibility as this provides the sharp threshold needed for RDD.

\textbf{First Stage Outcome: Broadband Adoption.} Our primary measure of broadband access is the HISPEED variable in ACS, which indicates whether the household has a broadband internet subscription. The variable is coded as 1 = Yes (household subscribes to high-speed internet service) and 2 = No. We recode this as a binary indicator equal to 1 if HISPEED = 1 and 0 if HISPEED = 2. Observations with missing or other values for HISPEED are excluded from the analysis.

It is worth noting what this variable captures and what it does not. HISPEED measures subscription to broadband service at the household level, as reported by survey respondents. It does not capture the quality or speed of the connection, the intensity of internet use, or access to internet through other means (e.g., at work, library, or through mobile data plans). If Lifeline subsidies induce subscriptions without meaningful increases in internet use, our first-stage estimate would overstate the effect on actual connectivity.

\textbf{Primary Outcomes: Self-Employment Measures.} We construct two primary measures of self-employment from ACS variables:

The first measure uses the class of worker variable (COW), which categorizes employed individuals by their relationship to their employer. We define a binary self-employment indicator as:

\begin{equation}
\text{SelfEmp}_i = \mathbf{1}[\text{COW}_i \in \{6, 7\}]
\end{equation}

where COW = 6 indicates ``self-employed in own not incorporated business, professional practice, or farm'' and COW = 7 indicates ``self-employed in own incorporated business, professional practice, or farm.'' This measure captures self-employment status at the time of the survey for currently employed individuals.

The second measure uses the self-employment income variable (SEMP), which reports self-employment income (or loss) over the past 12 months from the individual's own business, professional practice, or farm. We define a binary indicator for positive self-employment income:

\begin{equation}
\text{AnySEMP}_i = \mathbf{1}[\text{SEMP}_i > 0]
\end{equation}

This measure captures receipt of any self-employment income during the year, which may include individuals who are primarily wage workers but have some self-employment income on the side, as well as individuals who transition into or out of self-employment during the year.

\textbf{Control Variables.} In specifications that include controls, we use the following demographic and geographic covariates: age (AGEP) and age squared to capture nonlinear life-cycle effects; sex (SEX, coded as female indicator); race (RAC1P, with categories for White, Black, Asian, and Other); educational attainment (SCHL, categorized as less than high school, high school graduate, some college, and bachelor's degree or higher); marital status (MAR); and state and year fixed effects. All regression analyses use ACS person weights (PWGTP) to produce nationally representative estimates.

\subsection{Weighting and Inference}

All estimates in our analysis use ACS person weights (PWGTP) to account for the complex survey design and produce nationally representative estimates for the population of working-age householders and spouses. The person weights adjust for differential sampling probabilities across geographic areas and demographic groups, as well as for survey nonresponse and coverage of the ACS sampling frame.

For standard error estimation, we use robust (heteroskedasticity-consistent) standard errors clustered at the state level. Clustering at the state level accounts for potential within-state correlation in outcomes due to state-level policies, economic conditions, or other factors. As a robustness check, we also report results using standard errors clustered at the PUMA (Public Use Microdata Area) level, which provides finer geographic clustering.

The ACS provides a set of 80 replicate weights (PWGTP1-PWGTP80) that can be used for variance estimation using successive difference replication. In sensitivity analyses, we use these replicate weights to compute standard errors and find similar results to our main clustering approach.

\subsection{Summary Statistics}

Table 1 presents descriptive statistics for our analysis sample, separately for observations below and above the 135\% FPL threshold. The statistics are computed using survey weights to be representative of the target population.

The sample includes 91,957 observations below the 135\% threshold (eligible for Lifeline) and 102,139 observations above the threshold (ineligible), reflecting the rightward skew of the income distribution. The mean FPL ratio is 117.2\% below the threshold and 152.8\% above, by construction.

Broadband adoption rates are high in both groups but slightly lower among those below the threshold: 76.4\% versus 77.8\%. This 1.4 percentage point gap reflects the well-documented income gradient in internet adoption---higher-income households are more likely to have broadband. Our RDD approach will control for this gradient to isolate the effect of the eligibility discontinuity itself.

Self-employment rates are nearly identical on both sides of the threshold, at approximately 11.2-11.3\% by the COW-based measure. The rate of positive self-employment income is slightly lower, at about 8.2-8.4\%, reflecting individuals who are classified as self-employed but report zero or negative self-employment income. These raw means suggest no discontinuity in self-employment at the threshold, which our formal RDD analysis will confirm.

Demographic characteristics are similar on both sides of the threshold, though not identical. The mean age is slightly higher above the threshold (43.1 years vs. 42.3 years). We control for these demographic differences in our regression specifications to improve precision and to verify that they do not confound our estimates.

\section{Results}

\subsection{McCrary Density Test for Manipulation}

Before presenting our main RDD estimates, we assess the validity of the design by testing for manipulation of the running variable. If households strategically adjust their reported income to fall just below the 135\% FPL threshold and qualify for Lifeline, we would expect to see excess density of observations just below the cutoff relative to just above---a bunching pattern that would invalidate our RDD approach.

Figure 1 presents the distribution of our running variable (FPL ratio) around the 135\% threshold. Visual inspection shows no obvious bunching at the cutoff. The histogram displays a smooth distribution through the threshold, with slightly higher density above than below reflecting the natural right skew of the income distribution. The bars on both sides of the threshold follow a continuous pattern without an evident spike just below the cutoff.

To formalize this visual assessment, we compute a simple density discontinuity test following the approach of McCrary (2008). We compare the weighted count of observations in bins just below versus just above the threshold. The estimated log density jump is 0.04 with a standard error of 0.03---close to zero and not statistically significant. This confirms the visual impression that there is no manipulation at the threshold.

The absence of manipulation is consistent with our institutional expectations. The Lifeline benefit is relatively modest (\$9.25/month), take-up is incomplete even among the eligible, and ACS income reports are not linked to program enrollment. Additionally, precise income manipulation would require households to accurately estimate the FPL threshold for their household size and adjust their income accordingly---a level of sophistication that seems unlikely for a modest monthly benefit.

\subsection{First Stage Results: Broadband Adoption}

Figure 2 presents the RD plot for our first-stage outcome, broadband adoption. Each point represents the weighted mean broadband adoption rate within a narrow bin of the running variable. The vertical dashed line marks the 135\% FPL threshold. Separate linear regression lines are fitted to observations below (blue) and above (red) the cutoff.

The figure reveals several notable patterns. First, there is a clear positive relationship between income and broadband adoption on both sides of the threshold. This gradient reflects the well-documented fact that higher-income households are more likely to subscribe to broadband. Second, and more importantly for our purposes, there is a visible positive jump at the cutoff---the fitted line from the left (eligible side) intersects the threshold at a higher value than the fitted line from the right (ineligible side). This jump represents the first-stage effect of Lifeline eligibility on broadband adoption.

Table 2 presents formal estimates of the first-stage effect across different bandwidths. At the narrowest bandwidth (10\% FPL, corresponding to roughly \$3,000-4,000 in annual income depending on household size), we estimate that Lifeline eligibility increases broadband adoption by 2.11 percentage points with a standard error of 0.40 (t-statistic = 5.29, p < 0.001). At a bandwidth of 15\% FPL, the estimate is 1.76 percentage points (SE = 0.33, t = 5.33). At wider bandwidths of 20-30\% FPL, the estimate ranges from 0.76 to 1.63 percentage points, remaining statistically significant throughout.

The pattern of estimates across bandwidths is worth noting. The point estimate decreases somewhat as bandwidth increases, which is typical in RDD applications where wider bandwidths introduce more observations that are farther from the threshold and may be affected by specification of the control function. However, the estimate remains positive and statistically significant across all bandwidths, demonstrating robustness.

We also verify the first-stage using specifications with and without demographic controls. Including controls for age, sex, race, education, marital status, and state-year fixed effects has minimal effect on the point estimate, suggesting that these covariates are balanced at the threshold (as they should be in a valid RDD). The controlled estimates are similar in magnitude and precision to the uncontrolled estimates.

These first-stage results confirm that Lifeline eligibility has a meaningful causal effect on broadband adoption. The magnitude---roughly 1.5-2 percentage points---may seem modest in absolute terms, but it represents a 2-3\% increase relative to the baseline adoption rate of approximately 76\% among eligible households. Given the low take-up rate of Lifeline (estimated at 20-35\% of eligible households), the effect among those who actually enroll and use the benefit for broadband could be substantially larger.

\subsection{Reduced Form Results: Self-Employment}

Figure 3 presents the RD plot for our primary outcome, self-employment (defined as COW = 6 or 7). In stark contrast to the broadband plot, there is no visually apparent discontinuity at the 135\% FPL threshold. The binned means fluctuate on both sides of the cutoff without a clear pattern, and the fitted lines from the left and right appear to nearly intersect at the threshold.

Table 3 presents formal reduced-form estimates across bandwidths. At the narrowest bandwidth (10\% FPL), the estimated effect of eligibility on self-employment is 0.36 percentage points with a standard error of 0.30 (t = 1.21, p = 0.23)---positive but not statistically significant. At a bandwidth of 15\%, the estimate switches sign to -0.17 percentage points (t = -0.69). At wider bandwidths of 20-30\%, the estimates are negative and marginally statistically significant (around -0.46 to -0.50 percentage points, with t-statistics around -2.3 to -2.5).

This pattern is deeply inconsistent with a true positive effect of Lifeline eligibility on self-employment. A well-identified positive effect should be present (and ideally largest) at narrow bandwidths where the RDD assumptions are most credible, with the point estimate remaining relatively stable as bandwidth increases. Instead, we observe: (1) a positive but insignificant estimate at the narrowest bandwidth; (2) a sign change as bandwidth increases; (3) marginally significant negative estimates at wider bandwidths. This sensitivity to bandwidth choice is a hallmark of a null result in RDD---the appearance of effects at some bandwidths likely reflects specification choices rather than a true underlying effect.

We conduct parallel analyses for our alternative self-employment measure---any positive self-employment income (SEMP > 0)---and find substantively identical results. The point estimates are slightly smaller in magnitude (self-employment income is less common than self-employment status), but the pattern of a null result at narrow bandwidths and sensitivity to specification at wider bandwidths is the same.

\subsection{Bandwidth Robustness Analysis}

Figure 4 presents the bandwidth robustness analysis graphically, showing point estimates and 95\% confidence intervals for both the first stage (Panel A) and reduced form (Panel B) across bandwidths ranging from 5\% to 35\% FPL.

Panel A confirms the robustness of the first-stage effect on broadband. The point estimate is consistently positive across all bandwidths, ranging from approximately 0.8 to 2.5 percentage points. The 95\% confidence intervals exclude zero at all bandwidths except the very narrowest (5\% FPL), where the small sample size leads to imprecision. This pattern supports the conclusion that Lifeline eligibility causally increases broadband adoption.

Panel B demonstrates the lack of robustness in the reduced-form effect on self-employment. The point estimates fluctuate around zero, with confidence intervals that consistently include zero except at the widest bandwidths where negative estimates become marginally significant. The sign of the point estimate changes from positive to negative as bandwidth increases. This instability across specification choices is inconsistent with a true underlying effect and supports the conclusion that Lifeline eligibility does not detectably affect self-employment.

\subsection{Placebo Tests at Alternative Cutoffs}

To further assess whether our significant first-stage effect at 135\% FPL reflects a true policy discontinuity rather than a statistical artifact, we conduct placebo tests at alternative cutoffs where no Lifeline discontinuity should exist. Specifically, we re-estimate the first-stage model at placebo cutoffs of 100\%, 120\%, 150\%, and 175\% FPL, using the same bandwidth (25\% FPL) and specification as our main analysis.

Table 4 presents the placebo test results. At the 100\% FPL placebo cutoff, the estimated discontinuity is 0.23 percentage points (t = 0.74). At 120\% FPL, it is -0.15 percentage points (t = -0.54). At 150\% FPL (the first placebo above our true cutoff), it is 0.08 percentage points (t = 0.28). At 175\% FPL, it is -0.21 percentage points (t = -0.66). None of these placebo estimates are statistically significant, and all are substantially smaller in magnitude than our true estimate at 135\% FPL (1.76 percentage points, t = 5.33).

This pattern strongly supports the validity of our identification strategy. Significant effects appear only at the 135\% FPL threshold where Lifeline eligibility changes, not at other arbitrary points in the income distribution. If our first-stage estimate reflected general misspecification or data artifacts, we would expect to find similarly sized effects at placebo cutoffs. The absence of such effects at placebo cutoffs increases our confidence that the effect at 135\% FPL represents a genuine policy discontinuity.

\subsection{Heterogeneity Analysis}

We explore whether the null average effect on self-employment masks heterogeneous effects across subgroups. If Lifeline eligibility increases self-employment for some populations but decreases it for others, the average effect could be close to zero even with meaningful effects in subgroups.

Table 5 presents reduced-form estimates separately by education level, age group, and state type. By education, we find no significant effects in any group: the point estimate is 0.21 percentage points for less than high school (t = 0.44), 0.15 for high school graduates (t = 0.47), -0.28 for some college (t = -0.80), and -0.42 for bachelor's degree or higher (t = -1.02). If anything, there is a pattern of more positive (but still insignificant) effects for lower-educated groups, which could reflect the fact that highly educated individuals are less constrained by broadband access. However, none of these differences are statistically significant.

By age group, we find no significant effects for any cohort: 18-34 year-olds (0.18 pp, t = 0.47), 35-49 year-olds (-0.25 pp, t = -0.74), and 50-64 year-olds (-0.32 pp, t = -1.00). By state type, we find a suggestive but insignificant positive effect in rural states including South Dakota, Montana, and Wyoming (0.85 pp, t = 1.04) and no effect in urban states including Massachusetts, New York, and California (-0.18 pp, t = -0.75). The rural result is interesting given policy arguments that broadband subsidies may be particularly valuable in rural areas with fewer connectivity alternatives, but the small sample size and lack of statistical significance prevent strong conclusions.

Overall, the heterogeneity analysis does not reveal subgroups for which Lifeline eligibility has a statistically significant effect on self-employment. While we cannot rule out heterogeneous effects in populations we do not examine or effects that are too small to detect with our sample size, the consistent null results across subgroups strengthen our conclusion that Lifeline eligibility does not detectably affect self-employment for this population.

\section{Discussion}

\subsection{Interpretation of the Null Finding}

Our analysis presents a puzzle that warrants careful interpretation: Lifeline eligibility has a clear, robust, and statistically significant positive effect on broadband adoption, yet this increased broadband access does not translate into detectably higher self-employment rates. How should we understand this finding?

The simplest interpretation is that broadband access is necessary but not sufficient for self-employment. Starting and sustaining a business requires more than internet connectivity---it also requires business skills, financial capital, access to customers, products or services to sell, and time to devote to the enterprise. Low-income households eligible for Lifeline may face binding constraints on many of these other inputs. A household that gains broadband access through Lifeline but lacks the savings to purchase inventory, the skills to market effectively online, or the time beyond wage work to develop a side business will not show up as newly self-employed in our data.

This interpretation is consistent with research on the barriers to entrepreneurship among low-income populations. Studies have documented that aspiring entrepreneurs from disadvantaged backgrounds face greater difficulty accessing capital, have less valuable business networks, and may have skills gaps in areas like financial literacy and marketing (Fairlie and Robb, 2008). Broadband access addresses one barrier (connectivity) but leaves these other barriers in place.

A second interpretation focuses on the margin of behavior induced by the Lifeline subsidy. The households who adopt broadband because of the \$9.25/month discount may not be the households for whom broadband access would unlock entrepreneurship. Those with genuine entrepreneurial intentions or opportunities may have already found ways to access the internet---through work, public libraries, family members, or by prioritizing internet costs in their budgets. The households induced into broadband adoption by Lifeline may be those with lower underlying propensity for self-employment, making the Local Average Treatment Effect (LATE) for compliers close to zero even if broadband would have positive effects for other populations.

A third interpretation emphasizes statistical power. Self-employment is a relatively rare outcome (about 11\% in our sample), and the first-stage effect on broadband is modest (about 2 percentage points). Even if broadband access meaningfully increased the probability of self-employment---say, by 20\%---the reduced-form effect would be only $0.11 \times 0.20 \times 0.02 = 0.00044$, or 0.04 percentage points. This effect would be far too small to detect with our sample and methods. We cannot rule out that small positive effects exist but are below our detection threshold.

Fourth, it is possible that broadband affects the type or quality of self-employment rather than the probability of self-employment. Someone who already operates a small business might use newly acquired broadband to expand online, reach new customers, or improve efficiency without changing their employment classification. A freelancer might use broadband to find additional clients or move to higher-value work. These intensive margin effects would not show up in our binary self-employment measures. Unfortunately, the ACS does not provide data on business revenues, customer reach, or other intensive margin outcomes that would allow us to test this hypothesis.

Finally, our analysis covers a specific population (low-income households near 135\% FPL) and a specific time period (2019-2022). The effects of broadband on self-employment may differ for other populations, for more generous subsidies (like the now-expired ACP), or in other time periods. The rise of online gig work and e-commerce suggests that broadband may be increasingly important for self-employment over time, and future studies may find effects that we do not detect.

\subsection{Policy Implications}

Our findings have several implications for broadband policy.

First and most directly, the strong first-stage effect validates the core theory of change for broadband subsidies: making internet access more affordable does increase adoption. Households just below the Lifeline eligibility threshold are 1.5-2 percentage points more likely to have broadband than similar households just above the threshold. This is a meaningful effect on the program's proximate objective. Whatever downstream effects broadband may or may not have, the subsidy succeeds in getting more households connected.

Second, the null effect on self-employment suggests caution about claims that broadband subsidies will directly stimulate entrepreneurship among low-income populations. Policy advocates sometimes argue that connectivity is a pathway to economic opportunity, with self-employment and small business creation as examples of the opportunities enabled by the internet. Our findings suggest that this pathway is not automatic. Connectivity may be necessary for certain types of entrepreneurship, but it is not sufficient---other barriers remain binding. Policymakers should not expect broadband subsidies alone to produce significant increases in self-employment.

Third, our results suggest potential value in complementary interventions that address the non-connectivity barriers to entrepreneurship. If broadband access is necessary but not sufficient, then policy packages that combine connectivity subsidies with training programs, microfinance, business development services, or other supports might be more effective than connectivity alone. Evidence from developing countries suggests that bundled interventions addressing multiple constraints can be more effective than single-factor interventions (Banerjee et al., 2015). Similar approaches might be effective in the U.S. context.

Fourth, the findings inform ongoing debates about renewing or replacing the Affordable Connectivity Program. Advocates for ACP renewal often cite economic benefits including employment and entrepreneurship as reasons for continued funding. Our analysis suggests that while connectivity benefits are real, effects on economic outcomes like self-employment may be more limited than advocates claim. This does not mean broadband subsidies are unwarranted---connectivity has many benefits beyond labor market outcomes---but it does suggest that employment and entrepreneurship arguments should be advanced cautiously.

\subsection{Limitations and Future Directions}

Several limitations of our analysis warrant acknowledgment.

First, as with any RDD, our estimates are local to the cutoff. We identify effects for households near 135\% FPL whose broadband adoption is affected by Lifeline eligibility. Effects for households in deep poverty, for those well above the eligibility threshold, or for households whose adoption does not respond to the subsidy may differ. Extrapolating our null result to other populations or policy designs should be done cautiously.

Second, we observe broadband subscription rather than actual internet use or quality. The HISPEED variable indicates whether a household subscribes to broadband, not how they use it or whether the connection is adequate for running an online business. If Lifeline induces subscriptions without meaningful increases in entrepreneurially-relevant internet use, our first stage would overstate the effect on true connectivity.

Third, the ACS measures self-employment at a point in time and may miss transitions that occur between surveys or self-employment spells that are too brief to be captured. If broadband access increases the duration of self-employment spells or affects transition rates without changing the point-in-time prevalence, we would not detect these effects.

Fourth, our analysis period (2019-2022) includes the COVID-19 pandemic. While we exclude the experimental 2020 ACS, the 2021 and 2022 surveys may reflect pandemic-related disruptions to both internet access and self-employment. The pandemic accelerated digital adoption and remote work in ways that may not persist, and it disrupted small businesses in ways that could affect our estimates. Future research with longer post-pandemic data could assess whether effects differ in more normal economic conditions.

Fifth, we focus on Lifeline's \$9.25/month subsidy and 135\% FPL eligibility threshold. Effects may differ for programs with larger benefits (like ACP's \$30/month) or higher eligibility thresholds. Larger subsidies might induce broadband adoption among households with different underlying propensities for entrepreneurship, potentially producing different effects.

Future research could extend our analysis in several directions. Linking survey data to administrative records on business formation would enable more precise measurement of entrepreneurship. Studying the ACP program during its operation, or any future successor programs, would provide evidence on whether more generous subsidies produce different results. Qualitative research on how low-income entrepreneurs use the internet and what barriers they face could inform the design of more effective policy bundles. And studying longer-term outcomes---whether households that gain broadband eventually transition into self-employment after acquiring relevant skills---would address the possibility that effects emerge over time rather than immediately.

\section{Conclusion}

This paper examines whether eligibility for the FCC Lifeline broadband subsidy affects self-employment, using a regression discontinuity design at the 135\% Federal Poverty Level threshold. Our analysis yields two main findings.

First, we document a robust and statistically significant first-stage effect: households just below the Lifeline eligibility threshold are 1.5-2.1 percentage points more likely to have broadband internet than comparable households just above the threshold. This effect represents a 2-3\% increase relative to baseline adoption rates and confirms that the Lifeline subsidy achieves its proximate goal of increasing internet access among low-income households. The effect is robust across bandwidth choices, polynomial specifications, and the inclusion of demographic controls, and we find no evidence of manipulation at the eligibility threshold that would invalidate our RDD approach.

Second, we find no robust evidence that this increased broadband access translates into higher self-employment rates. The reduced-form effect of Lifeline eligibility on self-employment is small in absolute magnitude (less than half a percentage point), statistically insignificant at the narrow bandwidths where RDD estimates are most credible, and---critically---changes sign across different specifications. This pattern of instability is inconsistent with a true positive effect and leads us to conclude that Lifeline eligibility does not detectably increase self-employment for this population. The null finding persists across alternative self-employment measures, demographic subgroups, and geographic areas.

Our results carry implications for policy and research. For policy, we find that while broadband subsidies do work to increase connectivity, their effects on downstream economic outcomes like self-employment may be more limited than advocates sometimes claim. Broadband access appears to be necessary but not sufficient for entrepreneurship---other barriers related to skills, capital, and opportunity may remain binding even after connectivity is addressed. This suggests potential value in policy bundles that combine connectivity subsidies with complementary interventions addressing other constraints.

For research, our findings underscore the importance of distinguishing between effects on proximate targets (broadband adoption) and effects on ultimate objectives (economic outcomes). Programs can succeed on their immediate measures while having limited effects on the outcomes that motivate the program's existence. Our analysis also demonstrates both the power and the limitations of regression discontinuity designs for program evaluation: we can credibly identify local average treatment effects at the eligibility threshold, but these effects may not generalize to other populations or policy designs.

The digital divide remains a significant policy challenge, and broadband subsidies are an important tool for addressing it. Our findings confirm that these subsidies work for their intended purpose of increasing internet access. But their effects on economic mobility and entrepreneurship may be more modest than hoped, suggesting the need for realistic expectations and comprehensive approaches that address the multiple barriers facing low-income households seeking economic advancement.

\newpage
\section*{References}

\noindent Akerman, A., Gaarder, I., \& Mogstad, M. (2015). The Skill Complementarity of Broadband Internet. \textit{Quarterly Journal of Economics}, 130(4), 1781-1824.

\noindent Atasoy, H. (2013). The Effects of Broadband Internet Expansion on Labor Market Outcomes. \textit{ILR Review}, 66(2), 315-345.

\noindent Banerjee, A., Duflo, E., Goldberg, N., Karlan, D., Osei, R., Pariente, W., Shapiro, J., Thuysbaert, B., \& Udry, C. (2015). A Multifaceted Program Causes Lasting Progress for the Very Poor: Evidence from Six Countries. \textit{Science}, 348(6236), 1260799.

\noindent Calonico, S., Cattaneo, M. D., \& Titiunik, R. (2014). Robust Nonparametric Confidence Intervals for Regression-Discontinuity Designs. \textit{Econometrica}, 82(6), 2295-2326.

\noindent Card, D., Dobkin, C., \& Maestas, N. (2008). The Impact of Nearly Universal Insurance Coverage on Health Care Utilization: Evidence from Medicare. \textit{American Economic Review}, 98(5), 2242-2258.

\noindent Conley, K. L., \& Whitacre, B. E. (2016). Does Broadband Matter for Rural Entrepreneurs and Creative Class Employees? \textit{Review of Regional Studies}, 46(2), 171-190.

\noindent Currie, J. (2003). U.S. Food and Nutrition Programs. In R. A. Moffitt (Ed.), \textit{Means-Tested Transfer Programs in the United States} (pp. 199-290). University of Chicago Press.

\noindent Czernich, N., Falck, O., Kretschmer, T., \& Woessmann, L. (2011). Broadband Infrastructure and Economic Growth. \textit{Economic Journal}, 121(552), 505-532.

\noindent Dahl, G. B., \& Lochner, L. (2012). The Impact of Family Income on Child Achievement: Evidence from the Earned Income Tax Credit. \textit{American Economic Review}, 102(5), 1927-1956.

\noindent Deller, S. C., \& Whitacre, B. E. (2019). Broadband's Relationship to Rural Housing Values. \textit{Papers in Regional Science}, 98(5), 2125-2143.

\noindent Dettling, L. J. (2017). Broadband in the Labor Market: The Impact of Residential High-Speed Internet on Married Women's Labor Force Participation. \textit{ILR Review}, 70(2), 451-482.

\noindent Fairlie, R. W., \& Fossen, F. M. (2020). Defining Opportunity Versus Necessity Entrepreneurship: Two Components of Business Creation. In \textit{Research in Labor Economics} (Vol. 48, pp. 253-289). Emerald Publishing.

\noindent Fairlie, R. W., \& Robb, A. M. (2008). \textit{Race and Entrepreneurial Success: Black-, Asian-, and White-Owned Businesses in the United States}. MIT Press.

\noindent Forman, C., Goldfarb, A., \& Greenstein, S. (2012). The Internet and Local Wages: A Puzzle. \textit{American Economic Review}, 102(1), 556-575.

\noindent Greenstein, S., \& McDevitt, R. C. (2011). The Broadband Bonus: Estimating Broadband Internet's Economic Value. \textit{Telecommunications Policy}, 35(7), 617-632.

\noindent Hauge, J. A., \& Prieger, J. E. (2010). Demand-Side Programs to Stimulate Adoption of Broadband: What Works? \textit{Review of Network Economics}, 9(3), Article 4.

\noindent Hjort, J., \& Poulsen, J. (2019). The Arrival of Fast Internet and Employment in Africa. \textit{American Economic Review}, 109(3), 1032-1079.

\noindent Hoynes, H. W., \& Schanzenbach, D. W. (2009). Consumption Responses to In-Kind Transfers: Evidence from the Introduction of the Food Stamp Program. \textit{American Economic Journal: Applied Economics}, 1(4), 109-139.

\noindent Imbens, G. W., \& Kalyanaraman, K. (2012). Optimal Bandwidth Choice for the Regression Discontinuity Estimator. \textit{Review of Economic Studies}, 79(3), 933-959.

\noindent Imbens, G. W., \& Lemieux, T. (2008). Regression Discontinuity Designs: A Guide to Practice. \textit{Journal of Econometrics}, 142(2), 615-635.

\noindent Kolko, J. (2012). Broadband and Local Growth. \textit{Journal of Urban Economics}, 71(1), 100-113.

\noindent Lee, D. S., \& Lemieux, T. (2010). Regression Discontinuity Designs in Economics. \textit{Journal of Economic Literature}, 48(2), 281-355.

\noindent McCrary, J. (2008). Manipulation of the Running Variable in the Regression Discontinuity Design: A Density Test. \textit{Journal of Econometrics}, 142(2), 698-714.

\noindent Moffitt, R. A. (2002). Welfare Programs and Labor Supply. In A. J. Auerbach \& M. Feldstein (Eds.), \textit{Handbook of Public Economics} (Vol. 4, pp. 2393-2430). Elsevier.

\noindent NTIA. (1999). \textit{Falling Through the Net: Defining the Digital Divide}. U.S. Department of Commerce.

\noindent Prieger, J. E. (2013). The Broadband Digital Divide and the Economic Benefits of Mobile Broadband for Rural Areas. \textit{Telecommunications Policy}, 37(6-7), 483-502.

\noindent Reddick, C. G., Enriquez, R., Harris, R. J., \& Sharma, B. (2020). Determinants of Broadband Access and Affordability: An Analysis of a Community Survey on the Digital Divide. \textit{Cities}, 106, 102904.

\noindent Tomer, A., \& Fishbane, L. (2020). \textit{Bridging the Digital Divide Through Workforce Development}. Brookings Institution.

\noindent Whitacre, B., Gallardo, R., \& Strover, S. (2014). Does Rural Broadband Impact Jobs and Income? Evidence from Spatial and First-Differenced Regressions. \textit{Annals of Regional Science}, 53(3), 649-670.

\newpage
\appendix
\section{Data Appendix}

\subsection{Variable Definitions from ACS PUMS}

This appendix provides detailed definitions of the key variables used in our analysis, drawn from the American Community Survey Public Use Microdata Sample technical documentation.

\textbf{HISPEED (Broadband Internet Subscription).} This variable indicates whether the household has a subscription to high-speed internet service. The variable is coded as: 1 = Yes, household subscribes to high-speed internet service; 2 = No, household does not subscribe; blank = GQ/vacant. High-speed internet is defined as any type of internet service with faster download speeds than dial-up, including DSL, cable, fiber-optic, satellite, or fixed wireless. Mobile data plans (cellular) are not included in this definition.

\textbf{COW (Class of Worker).} This variable categorizes workers by their relationship to their employer. The variable is coded as: 1 = Employee of a private for-profit company or business, or of an individual, for wages, salary, or commissions; 2 = Employee of a private not-for-profit, tax-exempt, or charitable organization; 3 = Local government employee; 4 = State government employee; 5 = Federal government employee; 6 = Self-employed in own not incorporated business, professional practice, or farm; 7 = Self-employed in own incorporated business, professional practice, or farm; 8 = Working without pay in family business or farm. We define self-employment as COW = 6 or 7.

\textbf{SEMP (Self-Employment Income).} This variable reports self-employment income (or loss) received during the past 12 months from the person's own nonfarm business (including proprietorships and partnerships). The values range from -\$10,000 to \$640,000+ (with top-coding at the upper limit). Losses are reported as negative values. We define ``any self-employment income'' as SEMP > 0.

\textbf{HINCP (Household Income).} This variable reports total household income for the past 12 months, summing income of all household members from all sources. The values range from approximately -\$59,999 to \$9,999,999 (with top-coding). We use this variable in combination with NP to construct our FPL ratio running variable.

\textbf{NP (Number of Persons in Household).} This variable reports the number of persons living in the housing unit. Values range from 1 to 20. We use this variable to select the appropriate FPL threshold for the household.

\textbf{PWGTP (Person Weight).} This variable provides the person-level survey weight for producing nationally representative estimates. All analyses in this paper use PWGTP as the analysis weight.

\subsection{Federal Poverty Level Guidelines}

Table A1 presents the Federal Poverty Level guidelines by household size and year for the years covered by our analysis. The Lifeline eligibility threshold is 135\% of these values.

\begin{table}[H]
\centering
\caption{Federal Poverty Guidelines by Year and Household Size}
\begin{tabular}{lrrrrr}
\toprule
HH Size & 2019 & 2021 & 2022 & 135\% (2022) \\
\midrule
1 & \$12,490 & \$12,880 & \$13,590 & \$18,347 \\
2 & \$16,910 & \$17,420 & \$18,310 & \$24,719 \\
3 & \$21,330 & \$21,960 & \$23,030 & \$31,091 \\
4 & \$25,750 & \$26,500 & \$27,750 & \$37,463 \\
5 & \$30,170 & \$31,040 & \$32,470 & \$43,835 \\
6 & \$34,590 & \$35,580 & \$37,190 & \$50,207 \\
7 & \$39,010 & \$40,120 & \$41,910 & \$56,579 \\
8 & \$43,430 & \$44,660 & \$46,630 & \$62,951 \\
\bottomrule
\end{tabular}
\end{table}

For households with more than 8 persons, we add \$4,720 (in 2022 dollars) for each additional person and apply the 135\% multiplier.

\end{document}
