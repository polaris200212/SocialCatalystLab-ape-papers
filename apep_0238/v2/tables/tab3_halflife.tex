\begin{table}[H]
\centering
\caption{Employment Persistence: Half-Lives and Recovery Measures}
\begin{threeparttable}
\begin{tabular}{lcc}
\toprule
& Great Recession & COVID Recession \\
\midrule
Peak response ($\hat{\beta}_{peak}$) & $-0.0828$ & $0.5586$ \\
Peak horizon (months) & 51 & 3 \\
Half-life (months) & 45 & 9 \\
$\hat{\beta}_{48}$ & $-0.0732$ & $-0.2308$ \\
Persistence ratio ($\hat{\beta}_{48} / \hat{\beta}_{peak}$) & $0.883$ & $-0.413$ \\
\midrule
Instrument & Housing price boom & Bartik (industry shares) \\
States & 46 & 48 \\
\bottomrule
\end{tabular}
\begin{tablenotes}[flushleft]
\small
\item \textit{Notes:} For the Great Recession, peak response is the most negative $\hat{\beta}_h$ across all horizons; for COVID, it is the most positive (capturing the initial strong relationship between exposure and employment loss that fades as states recover). Half-life is the number of months after peak until $|\hat{\beta}_h|$ decays to half its peak value. Persistence ratio measures how much of the peak effect remains at $h=48$ months.
\end{tablenotes}
\end{threeparttable}
\label{tab:halflife}
\end{table}
