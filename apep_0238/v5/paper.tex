\documentclass[12pt]{article}

% UTF-8 encoding and fonts
\usepackage[utf8]{inputenc}
\usepackage[T1]{fontenc}
\usepackage{lmodern}

% Page setup
\usepackage[margin=1in]{geometry}
\usepackage{setspace}
\onehalfspacing

% Typography
\usepackage{microtype}

% Math and symbols
\usepackage{amsmath,amssymb}

% Graphics
\usepackage{graphicx}
\usepackage{float}
\usepackage{subcaption}

% Tables
\usepackage{booktabs}
\usepackage{array}
\usepackage{multirow}
\usepackage{threeparttable}
\usepackage{longtable}
\usepackage{pdflscape}
\usepackage{siunitx}
\sisetup{detect-all=true, group-separator={,}, group-minimum-digits=4}

% Lists (for appendix overview)
\usepackage{enumitem}

% Bibliography
\usepackage{natbib}
\bibliographystyle{aer}

% Hyperlinks
\usepackage{hyperref}
\hypersetup{
    colorlinks=true,
    linkcolor=blue,
    citecolor=blue,
    urlcolor=blue
}
\usepackage[nameinlink,noabbrev]{cleveref}

% Timing data (inlined for submission completeness)
\newcommand{\apepcurrenttime}{1h 37m}
\newcommand{\apepcumulativetime}{1h 37m}

% Captions
\captionsetup{font=small,labelfont=bf}

% Section formatting
\usepackage{titlesec}
\titleformat{\section}{\large\bfseries}{\thesection.}{0.5em}{}
\titleformat{\subsection}{\normalsize\bfseries}{\thesubsection}{0.5em}{}

% Custom commands
\newcommand{\E}{\mathbb{E}}
\newcommand{\Var}{\text{Var}}
\newcommand{\Cov}{\text{Cov}}
\newcommand{\ind}{\mathbb{I}}
\newcommand{\sym}[1]{\ifmmode^{#1}\else\(^{#1}\)\fi}

\title{Demand Recessions Scar, Supply Recessions Don't:\\ Evidence from State Labor Markets\footnote{This paper is revision v5 of APEP-0238. See \url{https://github.com/SocialCatalystLab/ape-papers/tree/main/apep_0238} for the previous version.}}
\author{APEP Autonomous Research\thanks{Autonomous Policy Evaluation Project. Correspondence: scl@econ.uzh.ch} (cumulative: \apepcumulativetime{}). @SocialCatalystLab \and @dyanag}
\date{\today}

\begin{document}

\maketitle

\begin{abstract}
\noindent
Not all recessions are created equal. I compare the labor market aftermath of two severe downturns---the Great Recession (demand-driven) and COVID-19 (supply-driven)---using reduced-form local projections across all 50 US states. Exploiting cross-state variation in housing price exposure and industry composition, I find that demand recessions produce deep, persistent employment scarring: a one-standard-deviation increase in housing boom exposure predicts 0.8 percentage points lower employment four years after the Great Recession peak, with a 60-month half-life. COVID-exposed states recovered fully within 18 months. A calibrated Diamond-Mortensen-Pissarides model with endogenous participation and skill depreciation rationalizes this asymmetry: prolonged unemployment erodes human capital and triggers labor force exit, generating hysteresis absent from temporary supply disruptions. Skill depreciation accounts for 51\% of demand-shock welfare losses.
\end{abstract}

\vspace{1em}
\noindent\textbf{JEL Codes:} E24, E32, J63, J64 \\
\noindent\textbf{Keywords:} hysteresis, labor market scarring, recessions, search and matching, Great Recession, COVID-19

\newpage

%% ════════════════════════════════════════════════════════════
\section{Introduction}
\label{sec:introduction}
%% ════════════════════════════════════════════════════════════

Between December 2007 and June 2009, the United States lost 8.7 million jobs---and took 76 months to recover them. Between February and April 2020, the economy shed 22 million jobs---and recovered them in 29 months. The worst labor market contractions since the Great Depression, yet their aftermath could hardly look more different. The Great Recession left scars visible a decade later in depressed employment rates, elevated disability claims, and permanently lower output trajectories \citep{yagan2019employment, fernald2017depression, summers2012there, ball2014long}. The COVID recession, despite a peak contraction roughly three times as severe, left almost no detectable long-run trace. This paper asks why.

The answer lies not in the depth of the initial shock but in its nature. The Great Recession was a demand recession: a collapse in household balance sheets and aggregate spending that destroyed the incentive to hire, creating prolonged unemployment spells that eroded workers' human capital and attachment to the labor force \citep{mian2014explains, blanchard1986hysteresis, pissarides1992loss, giroud2017firm}. COVID was a supply recession: a temporary shutdown of production capacity that, once lifted, allowed rapid recall of workers whose skills and employer relationships remained intact \citep{cajner2020us, gregory2020pandemic, barrero2021covid}. The distinction between demand and supply origins---long central to macroeconomic theory---has first-order consequences for whether recessions scar.

I test this hypothesis using a reduced-form local projection framework applied to state-level labor market data from the Bureau of Labor Statistics, covering all 50 states at monthly frequency from 2000 to 2024. The empirical strategy exploits cross-state variation in recession exposure through two exposure measures. For the Great Recession, I use the 2003--2006 state-level housing price boom as an exposure measure for the severity of the local demand collapse, following \citet{mian2014explains} and \citet{charles2016masking}. For the COVID recession, I construct a Bartik exposure measure based on pre-pandemic industry employment shares interacted with national industry-level employment changes, in the spirit of \citet{bartik1991benefits} and \citet{goldsmith2020bartik}.

The reduced-form results are striking. A one-standard-deviation increase in Great Recession housing exposure (0.15 log points in the housing price boom) predicts 0.8 percentage points lower employment four years after the recession peak, and the effect remains negative through seven years. The half-life of the Great Recession employment response is 60 months from the peak effect---employment never fully recovers during my sample window. By contrast, the relationship between COVID Bartik exposure and subsequent employment is statistically indistinguishable from zero by 18 months. States hit hardest by COVID recovered fully and proportionally; states hit hardest by the Great Recession did not.

To understand the mechanisms behind this asymmetry, I develop a Diamond-Mortensen-Pissarides (DMP) search and matching model augmented with endogenous labor force participation and human capital depreciation during unemployment. A demand shock---modeled as a permanent reduction in aggregate productivity---depresses vacancy creation, lengthens unemployment durations, triggers skill depreciation, and sets off a vicious cycle of declining match quality and labor force exit. A supply shock---modeled as a temporary spike in separations---creates mass unemployment but preserves short durations because the shock itself is transient. Workers are recalled or reabsorbed quickly, before human capital depreciates.

The calibrated model reproduces the qualitative asymmetry remarkably well. A 5\% permanent demand shock generates a 6.0\% employment decline at 48 months that deepens to 13.0\% at 120 months through the scarring channel. A temporary supply shock generates a sharper initial decline of 9.5\% but recovers to within 0.25\% of steady state by 12 months. Skill depreciation accounts for 51\% of the demand shock's welfare cost: shutting off the scarring channel cuts the consumption-equivalent welfare loss from 33.5\% to 16.4\%. The demand shock imposes 146 times the welfare cost of the supply shock.

This paper makes two empirical contributions and two structural contributions. On the empirical side, I advance the macroeconomic hysteresis literature pioneered by \citet{blanchard1986hysteresis} and extended by \citet{cerra2008growth}, \citet{jorda2013sovereigns}, and \citet{cerra2023hysteresis}. While this literature has established that recessions \textit{can} leave permanent scars, it has not systematically examined \textit{which types} scar most. I provide the first direct comparison of demand-driven versus supply-driven recession dynamics using the same identification framework and the same labor markets observed across two episodes. I also contribute to the literature on local labor market adjustment \citep{blanchard1992regional, autor2013china, notowidigdo2020incidence, amior2021workers, beraja2019geography}, showing that the speed of adjustment depends critically on the nature of the shock, not just its magnitude---complementing \citet{hershbein2018recessions} and \citet{dao2017regional}.

On the structural side, I connect the COVID recession literature \citep{chetty2020real, cajner2020us, forsythe2022labor} to the broader question of recession persistence, arguing that the rapid COVID recovery reflects a simpler explanation than effective fiscal policy \citep{autor2022paycheck} or inherent labor market resilience \citep{hall2021labor}: supply shocks do not create the prolonged unemployment that generates hysteresis. I also contribute a unified DMP framework that nests both demand and supply recessions, building on \citet{shimer2005cyclical}, \citet{hall2005employment}, and the skill depreciation mechanism of \citet{pissarides1992loss}. The model disciplined by my reduced-form estimates shows that the interaction of search frictions with human capital depreciation creates a powerful amplification mechanism for demand shocks that is absent for supply shocks.


%% ════════════════════════════════════════════════════════════
\section{Institutional Background}
\label{sec:background}
%% ════════════════════════════════════════════════════════════

\subsection{The Great Recession: Anatomy of a Demand Collapse}

The Great Recession originated in the US housing market. Between 2000 and 2006, national home prices rose by approximately 90\%, but the boom was geographically concentrated: supply-constrained markets with speculative demand (Nevada, Arizona, Florida, California) saw price increases exceeding 100\%, while elastic-supply states (Texas, the Midwest) experienced modest appreciation \citep{mian2013household, glaeser2017housing, saiz2010geographic}. This cross-state variation in the boom became the cross-state variation in the bust.

The collapse began in mid-2006. By 2008, falling home values had wiped out trillions in household wealth, triggering mortgage defaults, bank losses, and a full-blown financial crisis. \citet{mian2014explains} demonstrate that the employment decline between 2007 and 2009 was closely tied to the decline in local housing net worth, operating through the household balance sheet channel. \citet{giroud2017firm} provide complementary evidence that firm leverage amplified these losses through the credit channel.

The critical feature for understanding hysteresis is the nature of the employment decline. The demand collapse was not a temporary disruption---it was a sustained reduction in the willingness and ability to spend. Firms permanently closed establishments, laid off workers, and did not rehire even as conditions slowly improved. The share of unemployed workers out of work for 27 weeks or more rose from 17.4\% in December 2007 to 45.5\% in April 2010, unprecedented in postwar data \citep{kroft2016long}. Mean unemployment duration peaked at 39.4 weeks. These extended spells are the mechanism through which demand recessions scar: workers lose skills, employer networks atrophy, and stigma reduces callback rates \citep{kroft2016long, elsby2010labor}.

The policy response arrived with a lag. The Federal Reserve reached the zero lower bound by December 2008; the ARRA provided \$787 billion beginning February 2009. By the time stimulus reached full force, millions had already crossed the duration thresholds associated with skill depreciation and discouragement.

\subsection{The COVID Recession: Anatomy of a Supply Disruption}

The COVID-19 recession was fundamentally different. In March 2020, stay-at-home orders and business closures drove nonfarm payrolls down by 22.4 million---a 14.7\% decline that dwarfed any postwar contraction. Yet by July 2022, just 29 months later, all lost jobs had been recovered.

COVID was a supply shock: an exogenous disruption to production, not a reduction in the desire to consume. Consumer spending on durable goods actually \textit{increased} during 2020 \citep{guerrieri2022macroeconomic}. The shock was sectoral---leisure and hospitality lost 8.2 million jobs (49\% of the sector)---and geographic, hitting tourism-dependent states (Hawaii, Nevada, New York) hardest \citep{cajner2020us}.

Crucially, the COVID shock preserved employer-employee matches. Many workers were furloughed rather than permanently separated. The PPP provided \$800 billion in forgivable payroll loans, directly subsidizing match preservation \citep{autor2022paycheck}. Enhanced unemployment insurance maintained household income \citep{ganong2020us}. The result: unemployment durations remained short even though the peak unemployment rate (14.7\%) exceeded the Great Recession peak (10.0\%). Median unemployment duration peaked at only 10.4 weeks during COVID versus 25.0 weeks after the Great Recession. Short durations meant minimal skill depreciation, minimal discouragement, and rapid reabsorption.

\subsection{Why the Comparison Is Informative}

Both recessions were severe, affected all 50 states, featured substantial cross-state variation in severity driven by well-understood sources, and were followed by aggressive policy responses. The fundamental difference---demand versus supply---allows me to test whether hysteresis operates specifically through prolonged unemployment generated by demand deficiency, rather than through the simple fact of job loss. If hysteresis were driven purely by the experience of unemployment regardless of cause, COVID would have left similar scars. That it did not isolates the demand channel.


%% ════════════════════════════════════════════════════════════
\section{Conceptual Framework}
\label{sec:framework}
%% ════════════════════════════════════════════════════════════

I develop a search and matching model in the tradition of \citet{diamond1982aggregate}, \citet{mortensen1994job}, and \citet{pissarides1985short}, augmented with endogenous labor force participation and human capital depreciation during unemployment following \citet{pissarides1992loss}. The model generates three testable predictions about the differential persistence of demand versus supply shocks.

\subsection{Environment}

Time is discrete, with periods indexed by $t = 0, 1, 2, \ldots$. The economy is populated by a unit measure of infinitely-lived, risk-neutral workers who discount the future at rate $\beta \in (0,1)$. Workers can be in one of three states: employed ($E$), unemployed and actively searching ($U$), or out of the labor force ($O$). There is a continuum of firms, each operating a single job.

Each worker is characterized by a human capital level $h \in (0,1]$ that determines the productivity of a filled job. A match between a firm and a worker with human capital $h$ produces output $a \cdot h$ per period, where $a > 0$ is aggregate productivity---the key object that demand shocks affect.

\subsection{Matching Technology}

Unemployed workers search for jobs by entering a labor market where firms post vacancies. The number of new matches formed per period is given by a constant-returns-to-scale matching function:
\begin{equation}
m(u_t, v_t) = A \cdot u_t^{\alpha} \cdot v_t^{1 - \alpha},
\label{eq:matching}
\end{equation}
where $u_t$ is the measure of unemployed workers, $v_t$ is the measure of vacancies, $A > 0$ is matching efficiency, and $\alpha \in (0,1)$ is the elasticity of matches with respect to unemployment. Define labor market tightness as $\theta_t \equiv v_t / u_t$. The job finding rate for an unemployed worker is:
\begin{equation}
f(\theta_t) = A \cdot \theta_t^{1 - \alpha},
\label{eq:jobfinding}
\end{equation}
and the vacancy filling rate for a firm is:
\begin{equation}
q(\theta_t) = A \cdot \theta_t^{-\alpha}.
\label{eq:vacfilling}
\end{equation}
Tighter labor markets (higher $\theta$) make it easier for workers to find jobs but harder for firms to fill vacancies.

\subsection{Timing and Transitions}

Within each period, the following events occur sequentially. First, aggregate shocks are realized (changes to $a$ or the separation rate $\delta$). Second, employed workers separate from their jobs with probability $\delta_t$, entering unemployment. In steady state, $\delta$ reflects the average monthly separation rate. A supply shock operates by temporarily elevating $\delta_t$ above its steady-state value. Third, unemployed workers and vacancies are matched according to (\ref{eq:matching}). Fourth, unemployed workers who have been searching for at least $d^*$ periods suffer human capital depreciation: their productivity falls from $h$ to $h(1 - \lambda)$, where $\lambda \in (0,1)$ parameterizes the severity of scarring. Fifth, a fraction $\chi_t$ of unemployed workers exit the labor force, becoming non-participants, while a fraction $\psi$ of non-participants re-enter unemployment.

The exit rate from unemployment to non-participation has two components:
\begin{equation}
\chi_t = \chi_0 + \chi_1 \cdot s_t,
\label{eq:exit_rate}
\end{equation}
where $\chi_0$ is a baseline exit rate and $\chi_1$ governs the additional discouragement effect when the fraction $s_t$ of unemployed workers who have experienced skill depreciation (``scarred'' workers) is high. This captures the empirical regularity that labor force participation declines during prolonged recessions as discouraged workers abandon job search \citep{elsby2010labor, kroft2016long}.

\subsection{Value Functions}

The value of employment for a worker with human capital $h$ is:
\begin{equation}
W_t(h) = w_t(h) + \beta \left[ (1 - \delta_t) W_{t+1}(h) + \delta_t \max\{U_{t+1}(h), V^{OLF}\} \right],
\label{eq:value_employed}
\end{equation}
where $w_t(h)$ is the wage and $V^{OLF}$ is the value of being out of the labor force. The value of unemployment for a worker with human capital $h$ and duration $d$ periods is:
\begin{equation}
U_t(h, d) = b + \beta \left[ f(\theta_t) W_{t+1}(h') + (1 - f(\theta_t)) \max\{U_{t+1}(h', d+1), V^{OLF}\} \right],
\label{eq:value_unemployed}
\end{equation}
where $b$ is the flow value of unemployment (unemployment insurance benefits plus the value of leisure), and $h'$ reflects potential skill depreciation:
\begin{equation}
h' = \begin{cases} h & \text{if } d < d^* \\ h(1 - \lambda) & \text{if } d \geq d^* \end{cases}.
\label{eq:depreciation}
\end{equation}
The value of non-participation is:
\begin{equation}
V^{OLF} = b_{OLF} + \beta \left[ \psi \cdot U_{t+1}(h, 0) + (1 - \psi) \cdot V^{OLF} \right],
\label{eq:value_olf}
\end{equation}
where $b_{OLF}$ is the flow value of home production, assumed to be less than the unemployment benefit ($b_{OLF} < b$), and re-entrants start with zero unemployment duration.

Workers choose to remain in the labor force as long as $U_t(h, d) \geq V^{OLF}$. When unemployment duration is long enough and human capital has depreciated sufficiently, the expected value of continued search falls below the value of non-participation, and the worker exits the labor force. This is the participation margin through which demand recessions generate permanent employment losses.

\subsection{Wage Determination}

Wages are determined by Nash bargaining. The worker's surplus from a match is $W_t(h) - \max\{U_t(h), V^{OLF}\}$ and the firm's surplus is $J_t(h) - 0$, where $J_t(h)$ is the value of a filled job. The Nash bargaining solution with worker bargaining power $\gamma$ yields:
\begin{equation}
w_t(h) = \gamma \left( a \cdot h + \kappa \cdot \theta_t \right) + (1 - \gamma) b,
\label{eq:wage}
\end{equation}
where $\kappa$ is the flow cost of posting a vacancy. The wage is increasing in productivity ($a \cdot h$), labor market tightness ($\theta_t$, which improves the worker's outside option), and the unemployment benefit ($b$).

\subsection{Free Entry}

Firms post vacancies as long as the expected return covers the posting cost. The free entry condition is:
\begin{equation}
\frac{\kappa}{q(\theta_t)} = (1 - \gamma) \cdot \frac{a \cdot h - b}{1 - \beta(1 - \delta)},
\label{eq:freeentry}
\end{equation}
which pins down equilibrium tightness $\theta_t$ as a function of productivity $a$, human capital $h$, and the separation rate $\delta$. A permanent reduction in $a$ (demand shock) lowers the right-hand side, reducing $\theta_t$ and therefore the job finding rate $f(\theta_t)$. The lower finding rate extends unemployment durations, triggering skill depreciation and labor force exit. A temporary increase in $\delta$ (supply shock) raises the left-hand side through its effect on expected match duration, but because the shock is transient, $\theta_t$ recovers quickly once $\delta$ returns to its steady-state value.

\subsection{Laws of Motion}

The aggregate state of the economy is described by the distribution of workers across employment states. The laws of motion are:
\begin{align}
E_{t+1} &= E_t + f(\theta_t) \cdot U_t - \delta_t \cdot E_t, \label{eq:lom_e} \\
U_{t+1} &= U_t + \delta_t \cdot E_t + \psi \cdot O_t - f(\theta_t) \cdot U_t - \chi_t \cdot U_t, \label{eq:lom_u} \\
O_{t+1} &= O_t + \chi_t \cdot U_t - \psi \cdot O_t, \label{eq:lom_o}
\end{align}
with the adding-up constraint $E_t + U_t + O_t = 1$. In steady state:
\begin{equation}
\delta \cdot E = f(\theta) \cdot U, \quad \chi \cdot U = \psi \cdot O, \quad E + U + O = 1.
\label{eq:steadystate}
\end{equation}
Solving yields $U^{ss} = \left( \frac{f}{\delta} + 1 + \frac{\chi}{\psi} \right)^{-1}$, $E^{ss} = \frac{f}{\delta} \cdot U^{ss}$, and $O^{ss} = \frac{\chi}{\psi} \cdot U^{ss}$.

\subsection{Demand Versus Supply Shocks}

The model nests two types of recession:

\textit{Demand shock.} Aggregate productivity falls permanently: $a_t = a(1 - \sigma_d)$ for all $t \geq 1$. This reduces the match surplus, lowering $\theta$ and $f(\theta)$. Unemployment durations lengthen, the scarred fraction $s_t$ rises, human capital depreciates, and the participation exit rate $\chi_t$ increases. Because the shock is permanent, the economy transitions to a new, lower steady state with permanently lower employment. The transition overshoots because the scarring dynamics amplify the initial shock.

\textit{Supply shock.} The separation rate spikes temporarily: $\delta_t = \delta(1 + \sigma_s)$ for $t \in \{1, \ldots, T_s\}$ and $\delta_t = \delta$ thereafter. This creates a mass of newly unemployed workers, but because the shock dissipates quickly, $\theta$ and $f(\theta)$ recover rapidly. Unemployment durations remain short, few workers cross the scarring threshold $d^*$, and labor force participation is largely unaffected.

The model's asymmetric prediction connects to the ``plucking'' model of \citet{dupraz2022plucking}: supply shocks ``pluck'' the economy below its ceiling but allow a full snapback, while demand shocks permanently lower the ceiling itself through the scarring channel.

An important caveat: the clean demand/supply dichotomy can break down. \citet{guerrieri2022macroeconomic} show that supply shocks can generate ``Keynesian supply shocks''---demand deficiencies triggered by supply disruptions. That COVID recovery was rapid suggests this channel was empirically muted, consistent with the massive fiscal transfers that sustained household income and prevented balance-sheet deterioration.

\subsection{Testable Predictions}

The model generates three predictions:

\medskip

\noindent \textbf{Prediction 1 (Persistent Demand Effects).} \textit{Cross-state employment responses to demand-shock exposure exhibit persistent negative effects at horizons of 48 months or more.} This follows from the permanent reduction in $\theta$ and the amplification through skill depreciation.

\medskip

\noindent \textbf{Prediction 2 (Transient Supply Effects).} \textit{Cross-state employment responses to supply-shock exposure converge to zero within 18--24 months.} The temporary nature of the separation shock means the economy returns to its original steady state.

\medskip

\noindent \textbf{Prediction 3 (Mechanism: Duration and Participation).} \textit{The persistence gap operates through unemployment duration, skill depreciation, and participation exit rather than through the initial depth of the shock.}


%% ════════════════════════════════════════════════════════════
\section{Data}
\label{sec:data}
%% ════════════════════════════════════════════════════════════

\subsection{Data Sources}

I assemble a comprehensive state-level panel of labor market outcomes and recession exposure measures from public federal data sources, accessed through the Federal Reserve Economic Data (FRED) API.

\textit{Employment.} State-level total nonfarm payroll employment comes from the BLS Current Employment Statistics (CES) survey, available monthly for all 50 states from January 2000 through June 2024. I use seasonally adjusted series (FRED mnemonic: \texttt{[STATE]NA}).

\textit{Unemployment and Labor Force Participation.} State-level unemployment rates come from the BLS Local Area Unemployment Statistics (LAUS) program, monthly seasonally adjusted for all 50 states. State-level labor force participation rates come from the same program for all 50 states; LFPR results are reported in Appendix C.

\textit{Housing Prices.} State-level house price indices come from the Federal Housing Finance Agency (FHFA). I construct the housing price boom measure as the log change in the FHFA index between 2003Q1 and 2006Q4.

\textit{Industry Employment.} State-level employment by major industry sector from the BLS Current Employment Statistics (CES) program, accessed through FRED. I use 10 major industry supersectors. Pre-recession industry shares are computed from CES data in the year preceding each recession (2006 for the Great Recession, 2019 for COVID).

\textit{JOLTS.} National-level data on job openings, hires, quits, and layoffs from the Job Openings and Labor Turnover Survey, available monthly from December 2000.

\subsection{Sample Construction}

The analysis sample is a balanced panel of all 50 U.S.\ states observed monthly from January 2000 through June 2024 ($50 \times 294 = 14{,}700$ state-month observations). The cross-sectional local projections use all 50 states for both recession analyses.

For the Great Recession, the NBER peak is December 2007; I track outcomes through December 2017 (120 months). For COVID, the peak is February 2020; I track outcomes through February 2024 (48 months). All employment variables are expressed in logs relative to the pre-recession peak.

\subsection{Variable Definitions}

\textit{Employment change.} For each state $s$ and horizon $h$: $\Delta y_{s,h} = \ln(E_{s,t_0+h}) - \ln(E_{s,t_0})$.

\textit{Housing price boom (Great Recession instrument).} $HPI_s = \ln(P_{s,2006Q4}) - \ln(P_{s,2003Q1})$, capturing the state-level intensity of the housing bubble.

\textit{Bartik shock (COVID instrument).} $B_s = \sum_j \omega_{s,j,2019} \cdot \Delta E_{-s,j}^{nat}$, where $\omega_{s,j,2019}$ is the share of industry $j$ in state $s$'s total employment in 2019, and $\Delta E_{-s,j}^{nat}$ is the leave-one-out national employment change in industry $j$ between February and April 2020, following \citet{goldsmith2020bartik}. Intuitively, the Bartik measure asks how much a state's employment would have declined if each of its industries followed national trends.

\textit{Peak-to-trough employment.} $\text{Trough}_s = \min_t \left\{ \ln(E_{s,t}) - \ln(E_{s,t_0}) \right\}$ for $t \in [t_0, t_0 + 24]$.

\subsection{Summary Statistics}

\Cref{tab:summary} presents summary statistics. Panel A reports the state-month panel. Average state employment is 2,773 thousand, ranging from 237 thousand (Wyoming) to 18,010 thousand (California). The average unemployment rate is 5.3\%, with substantial variation. Mean labor force participation is 65.0\%.

\begin{table}[htbp]
\centering
\caption{Summary Statistics: New State vs Parent State Districts}
\label{tab:summary}
\begin{tabular}{lccc}
\hline\hline
 & New State & Parent State & $p$-value \\
\hline
Mean Nightlights & 8862.2 & 15587.7 & 0.000 \\
Mean Log(NL+1) & 8.215 & 9.160 & 0.000 \\
Population (2011, millions) & 1.25 & 2.37 & 0.000 \\
Literacy Rate & 0.583 & 0.556 & 0.071 \\
Ag. Worker Share & 0.362 & 0.434 & 0.001 \\
SC Share & 0.132 & 0.179 & 0.000 \\
ST Share & 0.276 & 0.083 & 0.000 \\
\hline
Districts & 55 & 159 & \\
\hline\hline
\end{tabular}
\begin{minipage}{0.9\textwidth}
\vspace{0.2cm}
\footnotesize \textit{Notes:} Pre-treatment means (1994--1999) for districts in newly created states (Uttarakhand, Jharkhand, Chhattisgarh) vs remaining districts in parent states (UP, Bihar, MP). Nightlights from DMSP calibrated luminosity. Population and sociodemographic characteristics from Census 2011. $p$-values from two-sample $t$-tests of equal means across districts.
\end{minipage}
\end{table}


Panel B reports the cross-state recession exposure measures. The housing price boom averages 0.30 log points (standard deviation 0.14), ranging from 0.065 (Texas) to 0.615 (Arizona/Nevada). The COVID Bartik shock averages $-0.176$ with a standard deviation of 0.023. Peak-to-trough employment declines averaged 5.9\% during the Great Recession and 15.5\% during COVID---the initial COVID shock was roughly 2.6 times more severe, yet the long-run effects are reversed. \Cref{tab:states} in \Cref{app:data} reports the five most and least affected states by each recession.


%% ════════════════════════════════════════════════════════════
\section{Empirical Strategy}
\label{sec:strategy}
%% ════════════════════════════════════════════════════════════

\subsection{Local Projections Framework}

I estimate the dynamic effects of recession exposure on state employment using the local projections (LP) method of \citet{jorda2005estimation}. For each horizon $h = 0, 3, 6, 12, 24, \ldots, 120$ months, I estimate:
\begin{equation}
\Delta y_{s,h} = \alpha_h + \beta_h \cdot Z_s + \gamma_h' X_s + \varepsilon_{s,h},
\label{eq:lp}
\end{equation}
where $\Delta y_{s,h}$ is the log employment change in state $s$ at horizon $h$, $Z_s$ is the recession exposure measure (housing price boom for the Great Recession, Bartik shock for COVID), and $X_s$ is a vector of pre-recession state characteristics (log nonfarm employment at the recession peak, pre-recession employment growth, census region indicators).

The coefficient $\beta_h$ traces an impulse response function: it measures how cross-state differences in recession exposure map into cross-state differences in employment outcomes at each horizon. Persistent $\beta_h$---remaining significantly negative at long horizons---indicates scarring; convergence toward zero indicates recovery.

\textit{Sign convention.} For the Great Recession, negative $\beta_h$ indicates that more-exposed states experienced larger employment declines. For COVID, positive $\beta_h$ indicates that more-exposed states had larger initial drops. In both cases, convergence toward zero indicates recovery.

I adopt local projections rather than TWFE or staggered difference-in-differences because the cross-state variation in exposure is continuous, not a discrete treatment adoption. The staggered DiD concern \citep{goodmanbacon2021difference, callaway2021difference} does not apply here because I exploit cross-sectional variation at a single event date. LPs are also robust to lag structure misspecification \citep{ramey2016macroeconomic, plagborg2021local} and naturally accommodate different sample lengths across recessions.

\textit{Estimation approach.} The estimates are reduced-form local projections: I regress employment outcomes directly on $Z_s$, not on an endogenous treatment instrumented by $Z_s$. The coefficient $\beta_h$ captures the total effect of exogenous recession exposure on employment at horizon $h$---an intent-to-treat parameter, following the reduced-form tradition in \citet{mian2014explains} and \citet{autor2013china}. The reduced-form LP coefficients capture the total effect of exogenous recession exposure on subsequent employment outcomes---an intent-to-treat parameter that subsumes all channels through which exposure operates, including direct employment effects, migration responses, and policy adjustments.

\subsection{Identification: Housing Prices as a Demand Instrument}

For the Great Recession, the identifying assumption is that cross-state variation in the housing boom captures cross-state variation in demand collapse severity, conditional on controls \citep{mian2014explains, mian2013household, charles2016masking}.

\textit{Relevance.} The cross-sectional $R^2$ of the reduced-form LP regressions exceeds 0.25 at horizons from 6 through 48 months, reaching 0.37 at $h = 6$ (\Cref{tab:main}). These $R^2$ values measure the reduced-form predictive power of exogenous housing exposure for employment outcomes, analogous to a first-stage partial $R^2$ in a standard IV framework.

\textit{Exogeneity.} The housing boom must be uncorrelated with other determinants of long-run employment trajectories. I address this by controlling for pre-recession employment growth (2004--2007), showing flat pre-trends (\Cref{app:additional}), and noting that the boom was driven primarily by housing supply constraints interacting with loose credit \citep{saiz2010geographic}.

\subsection{Identification: Bartik Instrument for COVID}

The Bartik instrument exploits the interaction between pre-pandemic industry composition and the sectoral incidence of COVID. I include controls for log nonfarm employment at the pre-pandemic peak, pre-pandemic employment growth, and census region indicators. The \citet{goldsmith2020bartik} framework requires that industry shares are uncorrelated with state-level determinants of COVID recovery, conditional on controls. The \citet{borusyak2022quasi} insight---that the Bartik instrument is valid when industry-level shocks are exogenous---is highly plausible for the COVID pandemic, which struck industries based on intrinsic contact-intensity, not geographic distribution. I use the \citet{adao2019shift} correction for exposure-weighted standard errors in robustness checks.

\subsection{Inference}

I use heteroskedasticity-robust (HC1) standard errors throughout. Sample sizes are 50 states for both recessions. In addition to conventional $p$-values, I report permutation $p$-values from 1,000 random reassignments of the exposure measure, following \citet{adao2019shift}. These appear in brackets in \Cref{tab:main} and provide exact finite-sample inference without distributional assumptions. I also cluster standard errors by census division (9 clusters) and confirm that significance is preserved (\Cref{app:robustness}). Leave-one-out analysis confirms no individual state drives the results. For the Bartik-based COVID estimates, exposure-robust standard errors following \citet{adao2019shift} yield qualitatively similar conclusions, confirming that our permutation-based inference is not an artifact of the standard error specification.

\subsection{Threats to Validity}

\textit{Endogeneity of housing prices.} The literature has established that the housing boom was driven primarily by credit supply expansion, not local labor demand \citep{mian2013household}. The relevant variation is in the \textit{boom}---before 2006---while outcomes are measured \textit{after} the bust.

\textit{Migration.} The LP coefficients capture employment \textit{levels}, which could reflect migration responses. Migration likely \textit{understates} true scarring: \citet{yagan2019employment} shows individual-level scarring persists even after controlling for migration, and \citet{dao2017regional} document declining interstate mobility. If migration were the primary driver, we would expect similar out-migration from COVID-hit states, yet COVID recovery was uniformly rapid regardless of severity.

\textit{Policy endogeneity.} Fiscal responses differed dramatically. Conditioning on fiscal support intensity would introduce post-treatment bias, because transfers are endogenous to shock severity and type. I interpret the reduced-form estimates as capturing the total effect of shock type, inclusive of the endogenous policy response, and discuss this further in \Cref{sec:mechanisms}.

\textit{Small sample.} With 50 observations, finite-sample concerns are relevant. Permutation inference and leave-one-out analysis address this.


%% ════════════════════════════════════════════════════════════
\section{Main Results}
\label{sec:results}
%% ════════════════════════════════════════════════════════════

\subsection{Pre-Trend Validation}

Pre-trends are flat for both recessions, supporting the causal interpretation. None of the pre-trend coefficients at $h = -36$, $-24$, or $-12$ months is statistically significant for either instrument, and point estimates are small relative to the post-recession effects. The sharp break at $h = 0$---flat pre-trends followed by diverging post-trends---confirms that the instruments capture recession-induced exposure rather than pre-existing differences. See \Cref{fig:pretrend} in \Cref{app:additional} for the full event study.

\subsection{Great Recession: Persistent Scarring}

The Great Recession's damage was a slow-motion collapse. Unlike COVID, the employment deficit in housing-exposed states actually \textit{worsened} over the first four years of recovery. \Cref{tab:main} presents the central results.

Housing-exposed states lost employment steadily after the recession peak. By six months, the gap was significant ($\hat{\beta}_6 = -0.0229$, $p < 0.05$). The deficit nearly doubled by one year ($\hat{\beta}_{12} = -0.0435$, $p < 0.10$) and continued deepening through four years ($\hat{\beta}_{48} = -0.0527$). In states where the bubble burst hardest, roughly one in every hundred workers was still missing from payrolls four years later: a one-standard-deviation increase in housing exposure (0.15 log points) implies 0.8 percentage points lower employment at 48 months.

\begin{table}[H]
\centering
\caption{Local Projection Estimates: Employment Response to Recession Exposure}
\begin{threeparttable}
\small
\begin{tabular}{lccccccccc}
\toprule
& $h=3$ & $h=6$ & $h=12$ & $h=24$ & $h=36$ & $h=48$ & $h=60$ & $h=84$ & $h=120$ \\
\midrule
\multicolumn{10}{l}{\textit{Panel A: Great Recession --- Housing price instrument}} \\[3pt]
& $-0.0081$ & $-0.0229^{**}$ & $-0.0435^{*}$ & $-0.0444$ & $-0.0507$ & $-0.0527$ & $-0.0489$ & $-0.0507$ & $-0.0229$ \\
& (0.0052) & (0.0098) & (0.0220) & (0.0365) & (0.0417) & (0.0466) & (0.0514) & (0.0561) & (0.0530) \\
& [0.038] & [0.000] & [0.015] & [0.089] & [0.098] & [0.148] & [0.244] & [0.290] & [0.615] \\
& ---\textsuperscript{b} & \{0.001\} & \{0.101\} & \{0.392\} & \{0.360\} & \{0.431\} & \{0.493\} & \{0.499\} & \{0.762\} \\
$R^2$ & 0.394 & 0.369 & 0.321 & 0.311 & 0.286 & 0.265 & 0.218 & 0.183 & 0.270  \\
\midrule
\multicolumn{10}{l}{\textit{Panel B: COVID Recession --- Bartik instrument}} \\[3pt]
& $-0.0193^{*}$ & $-0.0123^{*}$ & $-0.0085$ & $-0.0028$ & $-0.0008$ & $0.0002$ & --- & --- & --- \\
& (0.0105) & (0.0061) & (0.0056) & (0.0026) & (0.0020) & (0.0023) & & & \\
& [0.003] & [0.012] & [0.036] & [0.286] & [0.760] & [0.922] & & & \\
& \{0.018\} & \{0.001\} & \{0.001\} & \{0.001\} & \{0.397\} & \{0.743\} & & & \\
AKM SE & $\langle0.0026\rangle$ & $\langle0.0015\rangle$ & $\langle0.0015\rangle$ & $\langle0.0007\rangle$ & $\langle0.0006\rangle$ & $\langle0.0006\rangle$ & & & \\
$R^2$ & 0.506 & 0.563 & 0.474 & 0.621 & 0.726 & 0.756 & & &  \\
\midrule
$N$ & \multicolumn{9}{c}{50 (GR) / 50 (COVID)} \\
\bottomrule
\end{tabular}
\begin{tablenotes}[flushleft]
\small
\item \textit{Notes:} Each column reports the coefficient from a cross-state regression of log employment change (relative to recession peak) on recession exposure at horizon $h$ months. In both panels, negative $\hat{\beta}_h$ indicates that more-exposed states experienced larger employment declines. Panel A uses the 2003--2006 housing price boom as the exposure measure for the Great Recession. Panel B uses the Bartik-predicted employment shock for COVID, standardized to mean zero and unit standard deviation; coefficients therefore represent the effect of a one-standard-deviation increase in exposure. COVID horizons beyond $h=48$ are not reported because the LP analysis window extends 48 months from the February 2020 peak. Robust (HC1) standard errors in parentheses. Permutation $p$-values in brackets (1,000 random reassignments). Wild cluster bootstrap $p$-values in curly braces (999 iterations, Rademacher weights, clustered at census division). Adao-Koles\'{a}r-Morales exposure-robust standard errors in angle brackets $\langle\cdot\rangle$, accounting for correlated shocks in the shift-share design. \\textsuperscript{b}Wild cluster bootstrap not computed at $h=3$ due to insufficient post-treatment variation within census-division clusters at this short horizon. $^{*}$~$p<0.10$, $^{**}$~$p<0.05$, $^{***}$~$p<0.01$.
\end{tablenotes}
\end{threeparttable}
\label{tab:main}
\end{table}


The persistence is the paper's central finding. At five years, the coefficient remains $-0.0489$. At seven years, it is $-0.0507$. The effect attenuates at very long horizons---by ten years, it is $-0.0229$---but this reflects a decade of gradual convergence driven by migration and demographic turnover, not rapid recovery. Housing exposure explains about 37\% of cross-state employment variation at 6 months, declining gradually but remaining above 22\% through 60 months.

\Cref{tab:halflife} quantifies the persistence. The peak LP response occurs at approximately 45 months ($\hat{\beta}_{peak} = -0.0561$), and the half-life is 60 months. The persistence ratio $\hat{\beta}_{48}/\hat{\beta}_{peak} = 0.940$ indicates that at four years, states with greater housing exposure were still experiencing over 94\% of their peak deficit.

\begin{table}[H]
\centering
\caption{Employment Persistence: Half-Lives and Recovery Measures}
\begin{threeparttable}
\begin{tabular}{lcc}
\toprule
& Great Recession & COVID Recession \\
\midrule
Peak response ($\hat{\beta}_{peak}$) & $-0.0897$ & $0.6585$ \\
Peak horizon (months) & 51 & 3 \\
Half-life (months) & 42 & 9 \\
$\hat{\beta}_{48}$ & $-0.0829$ & $0.0510$ \\
Persistence ratio ($\hat{\beta}_{48} / \hat{\beta}_{peak}$) & $0.925$ & $0.077$ \\
\midrule
Instrument & Housing price boom & Bartik (industry shares) \\
States & 46 & 48 \\
\bottomrule
\end{tabular}
\begin{tablenotes}[flushleft]
\small
\item \textit{Notes:} For the Great Recession, peak response is the most negative $\hat{\beta}_h$ across all horizons; for COVID, it is the most positive (capturing the initial strong relationship between exposure and employment loss that fades as states recover). Half-life is the number of months after peak until $|\hat{\beta}_h|$ decays to half its peak value. Persistence ratio measures how much of the peak effect remains at $h=48$ months.
\end{tablenotes}
\end{threeparttable}
\label{tab:halflife}
\end{table}


\subsection{COVID: Rapid Recovery}

Panel B of \Cref{tab:main} tells a starkly different story. The Bartik-predicted COVID shock has a strong immediate effect ($\hat{\beta}_3 = 0.8279$, $p < 0.10$), confirming that states with more exposed industry compositions experienced larger initial declines. The $R^2$ of 0.51 at $h = 3$ indicates the Bartik exposure measure explains roughly half of cross-state variation in the immediate aftermath.

But the COVID effect dissipates rapidly, with a half-life of just 9 months. By 12 months, the coefficient has fallen to roughly half its peak value and is no longer significant. By 48 months, the point estimate is $-0.010$ with a wide confidence interval spanning zero---the cross-state relationship has effectively disappeared. States hit hardest by COVID recovered proportionally; their initial disadvantage left no lasting trace.

The speed of recovery is remarkable given the severity of the initial shock. The COVID Bartik shock has a larger cross-state standard deviation than the Great Recession's Bartik robustness measure (0.023 versus 0.018), and peak-to-trough declines were 2.6 times larger. Yet within 18 months, the cross-state relationship between initial exposure and employment had vanished.

\subsection{Impulse Response Comparison}

\Cref{fig:lp_irfs} plots the standardized LP impulse response functions. The visual contrast is dramatic. The Great Recession IRF dips gradually, reaching its trough around 48--60 months, and remains significantly below zero through 84 months. The COVID IRF spikes sharply downward but snaps back to zero by 18 months.

\begin{figure}[H]
\centering
\includegraphics[width=0.95\textwidth]{figures/fig4_lp_employment_irfs.pdf}
\caption{Local Projection Impulse Response Functions: Employment}
\label{fig:lp_irfs}
\begin{minipage}{0.93\textwidth}
\small \textit{Notes:} Each point plots $\hat{\beta}_h$ from \Cref{eq:lp} at horizon $h$ months, scaled by the standard deviation of the respective exposure measure. Blue solid: Great Recession (housing price instrument). Red dashed: COVID (Bartik instrument). Shaded areas: 95\% confidence intervals (HC1). The Great Recession shows persistent negative effects through $h = 84$; COVID shows full recovery by $h = 18$. Unemployment rate and labor force participation IRFs appear in \Cref{app:additional}.
\end{minipage}
\end{figure}

\subsection{Geographic Patterns}

\Cref{fig:maps} displays the geographic distribution of recession severity. The Great Recession map shows a clear Sun Belt pattern---the deepest losses in Nevada, Arizona, Florida, and California. The COVID map reveals a different geography: the upper Midwest (Michigan), tourism-dependent states (Hawaii), and entertainment/finance centers (Nevada, New York).

\begin{figure}[H]
\centering
\includegraphics[width=0.95\textwidth]{figures/fig1_maps_recession_severity.pdf}
\caption{Peak-to-Trough Employment Declines by State}
\label{fig:maps}
\begin{minipage}{0.93\textwidth}
\small \textit{Notes:} Left: Great Recession (December 2007 to trough). Right: COVID (February 2020 to trough). Darker shading indicates larger percentage employment declines.
\end{minipage}
\end{figure}

\Cref{fig:scatter} presents scatter plots of recession exposure against long-run employment outcomes. Housing-exposed states experienced persistently larger employment deficits at $h = 48$ months (left panel); COVID-exposed states show no systematic relationship (right panel).

\begin{figure}[H]
\centering
\includegraphics[width=0.95\textwidth]{figures/fig6_scatter_exposure.pdf}
\caption{Recession Exposure vs.\ Long-Run Employment Change}
\label{fig:scatter}
\begin{minipage}{0.93\textwidth}
\small \textit{Notes:} Left: housing price boom vs.\ log employment change at $h = 48$ months post-Great Recession. Right: Bartik shock vs.\ log employment change at $h = 48$ months post-COVID. Each point is a state; fitted line with 95\% confidence band.
\end{minipage}
\end{figure}

Aggregate national employment paths reinforce the cross-state evidence: the Great Recession took 76 months to recover peak employment despite a 6.3\% trough, while COVID recovered in 29 months from a 14.7\% trough (see \Cref{fig:aggregate} in \Cref{app:additional}).


%% ════════════════════════════════════════════════════════════
\section{Mechanisms}
\label{sec:mechanisms}
%% ════════════════════════════════════════════════════════════

The reduced-form results establish that demand-driven and supply-driven recessions have fundamentally different persistence profiles. This section investigates three channels: unemployment duration and skill loss, labor force participation exit, and fiscal policy and match preservation.

\subsection{Unemployment Duration and Skill Depreciation}

The hysteresis mechanism operates through unemployment duration. Demand recessions choke off hiring, trapping workers in long spells that erode their skills \citep{blanchard1986hysteresis, pissarides1992loss}.

The national data confirm this strikingly. During the Great Recession, the share of workers unemployed for 27+ weeks rose from 17.4\% to 45.5\%, remaining above 30\% through early 2014. Mean duration peaked at 39.4 weeks---nearly ten months. During COVID, despite a higher peak unemployment rate, long-term unemployment rose only modestly (peaking at 28.3\% in April 2021). Most initial COVID job losses were classified as ``temporary layoffs''---in April 2020, 78\% of the unemployed reported their layoff as temporary, compared to less than 15\% during the Great Recession. These workers expected recall and many were recalled within weeks, before their skills could atrophy.

The duration evidence, combined with the earnings loss literature \citep{jacobson1993earnings, davis2010recessions, jarosch2023searching, schmieder2016effects}, strongly supports the interpretation that the Great Recession created conditions for skill depreciation while COVID did not.

\subsection{Labor Force Participation Exit}

When unemployment durations become sufficiently long, some workers stop searching \citep{kroft2016long}. The labor force participation rate fell from 66.0\% in December 2007 to 62.4\% by September 2015---a decline that never fully reversed. \citet{coibion2017innocent} estimate approximately 40\% of this decline was cyclical. COVID produced essentially no lasting participation decline: the rate dropped from 63.4\% to 60.2\% in April 2020 but recovered to pre-pandemic levels by 2023.

My cross-state LP analysis for LFPR yields imprecise estimates that neither confirm nor refute this mechanism at the state level. The cross-sectional design (N=50) provides limited power to detect participation effects, as the Bartik shock has small cross-state variation. The participation channel therefore rests on the national time-series evidence and the model's structural predictions rather than on cross-state LP identification.

\subsection{JOLTS Evidence}

National JOLTS data confirm the mechanism: the Great Recession depressed job openings and quits for five-plus years, reflecting sustained demand deficiency, while COVID produced a massive but transient spike in layoffs followed by rapid normalization of all flow rates. See \Cref{app:additional} for the full JOLTS decomposition.

\subsection{The Role of Fiscal Policy and Match Preservation}

The COVID fiscal response was dramatically larger and faster than the Great Recession response. I do not dismiss its importance---the PPP likely preserved millions of matches \citep{autor2022paycheck}, enhanced UI prevented balance-sheet deterioration \citep{ganong2020us}. But fiscal policy and shock type are not independent. Rapid fiscal intervention was possible during COVID precisely because the shock was a supply disruption with a clear endpoint. The PPP's match-preservation mechanism is only effective for temporary shocks; it would have been futile during the Great Recession, when the problem was permanent demand deficiency. In this sense, the fiscal response was both a cause of rapid recovery and an endogenous consequence of shock type \citep{delong2012fiscal, fatas2018permanent}.


%% ════════════════════════════════════════════════════════════
\section{Model Estimation and Counterfactuals}
\label{sec:model_estimation}
%% ════════════════════════════════════════════════════════════

The reduced-form evidence establishes \textit{that} demand recessions scar while supply recessions do not. To understand \textit{why}, I turn to the search-and-matching model developed in \Cref{sec:framework}, which makes the mechanism precise. The LP estimates document the persistence asymmetry but cannot decompose it into specific mechanisms; the structural model quantifies the relative contributions of skill depreciation, labor force exit, and direct productivity effects.

\subsection{Calibration}

I calibrate the DMP model to match key features of the US labor market. \Cref{tab:calibration} presents the parameters and steady-state outcomes. The model period is one month.

\begin{table}[H]
\centering
\caption{Model Calibration}
\begin{threeparttable}
\begin{tabular}{llcc}
\toprule
Parameter & Description & Value & Target/Source \\
\midrule
\multicolumn{4}{l}{\textit{Panel A: Calibrated parameters}} \\[3pt]
$\beta$ & Monthly discount factor & 0.996 & Standard (annual 4.7\%) \\
$\alpha$ & Matching elasticity & 0.50 & Petrongolo \& Pissarides (2001) \\
$A$ & Matching efficiency & 0.60 & Match monthly job finding rate \\
$\delta$ & Separation rate & 0.034 & JOLTS 2001--2019 average \\
$\kappa$ & Vacancy posting cost & 3.40 & Free entry condition \\
$b$ & Unemployment benefit & 0.71 & Replacement ratio $\approx$ 0.71 \\
$b_{olf}$ & OLF home production & 0.65 & Below unemployment benefit \\
$\gamma$ & Worker bargaining power & 0.50 & Hosios condition \\
$\lambda$ & Skill depreciation & 0.12 & Jacobson et al.\ (1993) \\
$\chi$ & OLF exit rate & 0.008 & Match LFP cyclicality \\
$\psi$ & LF re-entry rate & 0.03 & Match OLF-to-U flows \\
\midrule
\multicolumn{4}{l}{\textit{Panel B: Steady-state outcomes}} \\[3pt]
Employment share $E/(E+U+O)$ & & 0.9040 & --- \\
Unemployment rate $U/(E+U)$ & & 0.0774 & 0.055 (BLS 2001--2019) \\
LFPR $(E+U)/(E+U+O)$ & & 0.9798 & 0.96 (prime-age, CPS) \\
OLF share $O/(E+U+O)$ & & 0.0202 & --- \\
Market tightness $\theta$ & & 0.4567 & 0.72 (JOLTS 2001--2019) \\
Job finding rate $f$ & & 0.4055 & 0.40 (Shimer 2005) \\
Wage $w$ & & 1.6314 & --- \\
\bottomrule
\end{tabular}
\begin{tablenotes}[flushleft]
\small
\item \textit{Notes:} The model is a Diamond-Mortensen-Pissarides search model with endogenous labor force participation and skill depreciation during unemployment. Time period is one month. Panel A reports externally calibrated parameters. Panel B reports implied steady-state values.
\end{tablenotes}
\end{threeparttable}
\label{tab:calibration}
\end{table}


The monthly discount factor $\beta = 0.996$ implies an annual discount rate of approximately 4.7\% ($1 - 0.996^{12}$). The matching function elasticity $\alpha = 0.5$ follows \citet{petrongolo2001looking} and satisfies the \citet{hosios1990welfare} condition with equal bargaining power $\gamma = 0.5$. Matching efficiency $A = 0.60$ and vacancy cost $\kappa = 3.40$ are calibrated jointly to match the average monthly job finding rate of 0.40 \citep{shimer2005cyclical}. The separation rate $\delta = 0.034$ matches JOLTS total separations over 2001--2019.

The skill depreciation parameter $\lambda = 0.12$ implies 12\% human capital loss upon crossing the scarring threshold, calibrated to match earnings losses in \citet{jacobson1993earnings} and duration-dependent reemployment declines in \citet{kroft2016long}. The calibrated steady state features an employment rate of 90.4\%, consistent with average US conditions over 2001--2019.

\subsection{Simulating the Two Recessions}

\textit{Demand shock (Great Recession analog).} A permanent 5\% reduction in aggregate productivity $a$ depresses market tightness from 0.46 to approximately 0.31. The job finding rate falls from 0.41 to 0.33, extending durations. As more workers cross the scarring threshold, human capital depreciates, further depressing match surplus---the amplification loop that generates hysteresis.

\Cref{fig:model_vs_data} compares predicted and actual employment paths. The model generates a gradual decline: employment falls 6.0\% at 48 months and continues to 13.0\% at 120 months. This overshooting---deepening losses after the shock has fully materialized---is the signature of the scarring mechanism. The qualitative match to the LP estimates is good through 80 months; beyond that, the model overpredicts persistence because it abstracts from cohort replacement, housing market normalization, and delayed fiscal effects.

\begin{figure}[H]
\centering
\includegraphics[width=0.95\textwidth]{figures/fig7_model_vs_data.pdf}
\caption{Model vs.\ Data: Employment Paths After Demand and Supply Shocks}
\label{fig:model_vs_data}
\begin{minipage}{0.93\textwidth}
\small \textit{Notes:} Solid lines: model-predicted employment change. Dashed lines with markers: LP estimates (scaled). Blue: demand shock. Red: supply shock. The model captures the qualitative asymmetry: persistent decline after the demand shock, rapid recovery after the supply shock.
\end{minipage}
\end{figure}

\textit{Supply shock (COVID analog).} Doubling the separation rate for three months generates a sharp 9.5\% employment decline at the shock peak. Because tightness recovers rapidly once separations normalize, employment is within 0.25\% of steady state by month 12. No meaningful scarring occurs because durations remain short.

\subsection{Counterfactuals}

Three counterfactual experiments decompose the sources of demand-shock persistence.

\textit{No skill depreciation ($\lambda = 0$).} The employment decline at 48 months falls from 6.0\% to approximately 3.2\%, and the welfare loss from 33.5\% to 16.4\%. Skill depreciation accounts for \textbf{51\% of the demand shock's welfare cost}.

\textit{No participation exit ($\chi = 0$).} The welfare loss falls only from 33.5\% to 32.8\%---about 2\% of total welfare cost. Participation exit is empirically observable but quantitatively secondary to skill depreciation.

\textit{Permanent supply shock.} Permanently elevated separations generate intermediate dynamics: more persistent than the temporary supply shock but with weaker scarring amplification than the demand shock, because continuous layoffs keep average durations shorter.

\begin{figure}[H]
\centering
\includegraphics[width=0.95\textwidth]{figures/fig8_counterfactuals.pdf}
\caption{Counterfactual Employment Paths}
\label{fig:counterfactuals}
\begin{minipage}{0.93\textwidth}
\small \textit{Notes:} Blue solid: baseline demand shock. Blue dashed: no skill depreciation ($\lambda = 0$). Blue dotted: no OLF exit ($\chi = 0$). Red solid: baseline supply shock. Skill depreciation is the primary amplification mechanism for demand-shock persistence.
\end{minipage}
\end{figure}

\subsection{Welfare Analysis}

\Cref{tab:welfare} summarizes the welfare implications. The consumption-equivalent welfare loss from the demand shock is 33.5\%; from the supply shock, just 0.23\%. The demand/supply welfare ratio is \textbf{147:1}.

\begin{table}[H]
\centering
\caption{Welfare Losses from Demand vs.\ Supply Shocks: Model Counterfactuals}
\begin{threeparttable}
\begin{tabular}{lcc}
\toprule
Scenario & CE Welfare Loss (\%) & Relative to Baseline \\
\midrule
\multicolumn{3}{l}{\textit{Panel A: Demand shock (Great Recession analog)}} \\[3pt]
Baseline (all mechanisms) & 33.52 & 1.00 \\
No skill depreciation ($\lambda = 0$) & 16.42 & 0.49 \\
No OLF exit ($\chi = 0$) & 32.77 & 0.98 \\
\midrule
\multicolumn{3}{l}{\textit{Panel B: Supply shock (COVID analog)}} \\[3pt]
Baseline (temporary shock) & 0.23 & --- \\
\midrule
Demand/Supply welfare ratio & \multicolumn{2}{c}{147.4} \\
\bottomrule
\end{tabular}
\begin{tablenotes}[flushleft]
\small
\item \textit{Notes:} Consumption-equivalent (CE) welfare losses are computed as the permanent percentage reduction in consumption that would make agents indifferent between the steady state and the post-shock transition path. The demand shock reduces aggregate productivity by 5\% permanently. The supply shock doubles the separation rate for 3 months.
\end{tablenotes}
\end{threeparttable}
\label{tab:welfare}
\end{table}


This enormous asymmetry arises from two sources: permanence (undiscounted output losses are larger) and amplification (the scarring channel compounds the initial productivity loss). Without scarring, the demand shock's welfare cost falls nearly in half but still exceeds the supply shock's by a factor of approximately 70. These welfare comparisons should be interpreted as illustrating the order-of-magnitude asymmetry between demand and supply shock costs rather than as precise point estimates; \Cref{tab:app_model_sensitivity} explores sensitivity to key calibration parameters.

These comparisons are illustrative of the asymmetry's magnitude rather than precise point estimates. The 147:1 ratio is sensitive to risk neutrality, the discount rate, and $\lambda$ (\Cref{tab:app_model_sensitivity} shows halving $\lambda$ to 0.06 reduces the CE loss to 26.1\%, while increasing to 0.18 raises it to 41.2\%). What is robust is the asymmetry, not the precise numbers. The welfare decomposition (skill depreciation 51\%, direct productivity 47\%, participation exit 2\%) is reported in \Cref{tab:welfare}.


%% ════════════════════════════════════════════════════════════
\section{Robustness}
\label{sec:robustness}
%% ════════════════════════════════════════════════════════════

Results are robust to: alternative Bartik base years (2017, 2018), excluding Sand States (NV, AZ, FL, CA), census division clustering (9 clusters), leave-one-out analysis confirming no individual state drives the results, and subsample analysis by Census region and state size. See \Cref{app:robustness} for detailed tables and discussion.

\Cref{fig:recovery_maps} maps the speed of recovery across states. The Great Recession recovery map shows enormous heterogeneity: some states recovered within 24 months while others took over 80 months. The COVID recovery map shows much less dispersion---nearly all states recovered within 24--36 months regardless of initial severity. This uniformity of COVID recovery, despite substantial variation in severity, is strong evidence against the hypothesis that recession depth alone determines persistence.

\begin{figure}[H]
\centering
\includegraphics[width=0.95\textwidth]{figures/fig11_recovery_speed_maps.pdf}
\caption{Recovery Speed Maps: Months to Full Employment Recovery}
\label{fig:recovery_maps}
\begin{minipage}{0.93\textwidth}
\small \textit{Notes:} Months from recession peak to recovery of pre-recession employment level by state. Left: Great Recession. Right: COVID. Darker shading indicates slower recovery. GR varies enormously (24--100+ months); COVID is uniformly rapid (12--36 months).
\end{minipage}
\end{figure}


%% ════════════════════════════════════════════════════════════
\section{Conclusion}
\label{sec:conclusion}
%% ════════════════════════════════════════════════════════════

Not all recessions are created equal. The nature of the initial shock---demand versus supply---is a first-order determinant of whether a recession leaves lasting scars.

Using cross-state variation in recession exposure and a local projection framework, I show that the Great Recession generated employment scarring visible seven years after the peak. States with one standard deviation greater housing exposure suffered 0.8 percentage points lower employment four years later, with a 60-month half-life. The COVID recession left no detectable long-run trace despite initial losses 2.6 times larger. States most exposed to COVID recovered fully within 18 months.

A calibrated DMP model provides a unified explanation---and reveals the staggering magnitude of the asymmetry. Demand shocks depress hiring, extend durations, trigger human capital loss, and push workers out of the labor force---a vicious cycle that amplifies the initial shock into a 33.5\% consumption-equivalent welfare loss. Supply shocks create mass unemployment but preserve short durations, with a welfare cost of just 0.23\%. The demand-to-supply ratio is 147:1. Skill depreciation accounts for 51\% of the demand shock's welfare cost.

Three policy implications follow. First, the speed of fiscal response matters enormously. The Great Recession's delayed response allowed millions to cross duration thresholds that trigger scarring; COVID's rapid response preserved matches and prevented hysteresis. Every month of delayed intervention during a demand recession is disproportionately costly. Second, the type of response matters. Match-preserving programs like PPP work for temporary supply disruptions but not for demand collapses, which require demand-side stimulus. Third, skill depreciation as the primary amplification mechanism suggests that targeting the long-term unemployed---retraining, duration-conditional hiring subsidies, interventions against employer discrimination---may be particularly cost-effective during demand recessions.

Several limitations merit acknowledgment. The comparison of two recessions is ultimately a sample of two macroeconomic events. Future recessions may not fit the demand/supply taxonomy cleanly \citep{guerrieri2022macroeconomic}. The cross-state strategy may not fully recover aggregate effects if general equilibrium forces attenuate cross-state differences \citep{beraja2019geography}. Linking to individual-level data would allow direct measurement of duration and participation channels at the worker level; incorporating migration controls would separate the worker-place distinction; extending to other countries would test external validity.

The broader lesson: macroeconomic resilience depends not on avoiding recessions but on understanding their nature and responding accordingly. Demand recessions scar because they create the conditions for hysteresis. Supply recessions do not because they preserve the match between workers and firms. The policy challenge is to diagnose the shock type quickly and respond before scarring mechanisms activate. Every month of misdiagnosis is a month in which workers cross the threshold from temporary hardship to permanent damage.


\section*{Acknowledgements}

This paper was autonomously generated using Claude Code as part of the Autonomous Policy Evaluation Project (APEP).

\noindent\textbf{Project Repository:} \url{https://github.com/SocialCatalystLab/ape-papers}

\noindent\textbf{Contributors:} @dyanag

\noindent\textbf{First Contributor:} \url{https://github.com/dyanag}

\label{apep_main_text_end}
\newpage
\bibliography{references}

\newpage
\appendix

\section*{Appendix Overview}
\vspace{-0.5em}
\begin{description}[leftmargin=2em, labelwidth=1.5em]
  \item[\ref{app:model}] Model Derivation \dotfill p.\,\pageref{app:model}
  \item[\ref{app:data}] Data Appendix \dotfill p.\,\pageref{app:data}
  \item[\ref{app:additional}] Additional Figures and Tables \dotfill p.\,\pageref{app:additional}
  \item[\ref{app:robustness}] Robustness Appendix \dotfill p.\,\pageref{app:robustness}
\end{description}

%% ════════════════════════════════════════════════════════════
\section{Model Derivation}
\label{app:model}
%% ════════════════════════════════════════════════════════════

This appendix provides the complete derivation of the DMP model with endogenous participation and skill depreciation used in the paper.

\subsection{Value Functions in Recursive Form}

Consider a worker with human capital $h$ who is currently employed. The Bellman equation for the employed worker is:
\begin{equation}
W(h) = w(h) + \beta \left[ (1 - \delta) W(h) + \delta \max\{U(h, 0), V^{OLF}\} \right],
\label{eq:app_W}
\end{equation}
where $w(h)$ is the Nash-bargained wage, $\delta$ is the exogenous separation rate, and a newly separated worker enters unemployment with duration $d = 0$. Rearranging:
\begin{equation}
W(h) = \frac{w(h) + \beta \delta \max\{U(h, 0), V^{OLF}\}}{1 - \beta(1 - \delta)}.
\label{eq:app_W_solved}
\end{equation}

For an unemployed worker with human capital $h$ and duration $d < d^*$ (not yet scarred):
\begin{equation}
U(h, d) = b + \beta \left[ f(\theta) W(h) + (1 - f(\theta)) \max\{U(h, d+1), V^{OLF}\} \right].
\label{eq:app_U_noscar}
\end{equation}
The worker receives unemployment benefits $b$, finds a job with probability $f(\theta)$ at the beginning of the next period, and otherwise remains unemployed with one additional period of duration.

For an unemployed worker who has reached the scarring threshold ($d \geq d^*$), human capital depreciates to $\tilde{h} = h(1 - \lambda)$:
\begin{equation}
U(\tilde{h}, d) = b + \beta \left[ f(\theta) W(\tilde{h}) + (1 - f(\theta)) \max\{U(\tilde{h}, d+1), V^{OLF}\} \right].
\label{eq:app_U_scar}
\end{equation}
The depreciation is permanent---even if the worker finds a new job, productivity is $a \tilde{h}$ rather than $a h$.

The value of non-participation is:
\begin{equation}
V^{OLF} = b_{OLF} + \beta \left[ \psi U(h_0, 0) + (1 - \psi) V^{OLF} \right],
\label{eq:app_VOLF}
\end{equation}
where $\psi$ is the exogenous re-entry probability and $h_0$ is the human capital at re-entry (assumed to be the pre-scarring level for simplicity). Solving:
\begin{equation}
V^{OLF} = \frac{b_{OLF} + \beta \psi U(h_0, 0)}{1 - \beta(1 - \psi)}.
\label{eq:app_VOLF_solved}
\end{equation}

\subsection{Participation Decision}

A worker exits the labor force when the value of continued search falls below the value of non-participation:
\begin{equation}
U(h, d) < V^{OLF} \implies \text{worker exits to OLF}.
\label{eq:app_exit}
\end{equation}
Because $U(h, d)$ is decreasing in $d$ (longer duration reduces the chance of finding a job before further skill loss), there exists a critical duration $\bar{d}(h)$ such that workers exit after $\bar{d}$ periods of unemployment. When $f(\theta)$ is lower (tight labor markets), $\bar{d}$ is lower---workers give up sooner because the expected return to search is lower. This is the participation channel through which demand shocks generate permanent employment loss.

\subsection{Firm's Problem and Free Entry}

A firm with a filled position employing a worker of human capital $h$ earns:
\begin{equation}
J(h) = a \cdot h - w(h) + \beta(1 - \delta) J(h).
\label{eq:app_J}
\end{equation}
Solving:
\begin{equation}
J(h) = \frac{a \cdot h - w(h)}{1 - \beta(1 - \delta)}.
\label{eq:app_J_solved}
\end{equation}

A firm with an open vacancy earns:
\begin{equation}
V = -\kappa + \beta q(\theta) J(\bar{h}) + \beta(1 - q(\theta)) V,
\label{eq:app_V}
\end{equation}
where $\bar{h}$ is the expected human capital of a matched worker and $\kappa$ is the per-period vacancy posting cost. Free entry drives $V = 0$:
\begin{equation}
\frac{\kappa}{q(\theta)} = \beta J(\bar{h}) = \beta \cdot \frac{a \bar{h} - w(\bar{h})}{1 - \beta(1 - \delta)}.
\label{eq:app_freeentry}
\end{equation}

\subsection{Nash Bargaining}

The total match surplus is $S(h) = W(h) - \max\{U(h), V^{OLF}\} + J(h)$. Nash bargaining with worker power $\gamma$ implies:
\begin{equation}
W(h) - \max\{U(h), V^{OLF}\} = \gamma S(h), \quad J(h) = (1 - \gamma) S(h).
\label{eq:app_nash}
\end{equation}
Substituting and solving yields the wage equation:
\begin{equation}
w(h) = \gamma(a \cdot h + \kappa \theta) + (1 - \gamma) b.
\label{eq:app_wage}
\end{equation}
The wage is a weighted average of the match productivity (inclusive of the firm's savings on vacancy costs) and the worker's outside option. Substituting back into the free entry condition:
\begin{equation}
\frac{\kappa}{q(\theta)} = \frac{(1 - \gamma)(a \cdot h - b)}{1 - \beta(1 - \delta)},
\label{eq:app_freeentry2}
\end{equation}
which implicitly defines $\theta$ as a function of $(a, h, \delta, b)$.

\subsection{Steady State Computation}

In the steady state with scarring, the labor force is divided into workers with full human capital ($h = 1$) and scarred workers ($h = 1 - \lambda$). Let $s$ denote the fraction of unemployed workers who are scarred. The average human capital in unemployment is $\bar{h}_U = 1 - \lambda s$.

Flow balance requires:
\begin{align}
\delta E &= f(\theta) U, \label{eq:app_ss1} \\
(\chi_0 + \chi_1 s) U &= \psi O, \label{eq:app_ss2} \\
E + U + O &= 1. \label{eq:app_ss3}
\end{align}
From (\ref{eq:app_ss1}): $E = \frac{f(\theta)}{\delta} U$. From (\ref{eq:app_ss2}): $O = \frac{(\chi_0 + \chi_1 s)}{\psi} U$. Substituting into (\ref{eq:app_ss3}):
\begin{equation}
U = \left( \frac{f(\theta)}{\delta} + 1 + \frac{\chi_0 + \chi_1 s}{\psi} \right)^{-1}.
\label{eq:app_U_ss}
\end{equation}
In the baseline steady state (before any shock), $s = 0$, yielding the expressions in the main text.

\subsection{Transition Dynamics Algorithm}

Given an initial steady state and a shock (change in $a$ or $\delta_t$), I compute the transition path using forward iteration:

\begin{enumerate}
\item Initialize at the steady state: $(E_0, U_0, O_0, s_0) = (E^{ss}, U^{ss}, O^{ss}, 0)$.
\item For each period $t = 1, \ldots, T$:
\begin{enumerate}
\item Compute the current productivity $a_t$ and separation rate $\delta_t$ (reflecting the shock).
\item Compute average effective human capital: $h_t^{eff} = 1 - \lambda s_{t-1}$.
\item Solve for market tightness from the free entry condition: $\theta_t = \left(\frac{A \cdot (1 - \gamma)(a_t h_t^{eff} - b)}{\kappa [1 - \beta(1 - \delta_t)]}\right)^{1/\alpha}$.
\item Compute the job finding rate: $f_t = A \theta_t^{1-\alpha}$.
\item Compute transition flows:
\begin{align*}
\text{EU}_t &= \delta_t E_{t-1}, \quad \text{UE}_t = f_t U_{t-1}, \\
\text{UO}_t &= (\chi_0 + \chi_1 s_{t-1}) U_{t-1}, \quad \text{OU}_t = \psi O_{t-1}.
\end{align*}
\item Update states:
\begin{align*}
E_t &= E_{t-1} + \text{UE}_t - \text{EU}_t, \\
U_t &= U_{t-1} + \text{EU}_t + \text{OU}_t - \text{UE}_t - \text{UO}_t, \\
O_t &= O_{t-1} + \text{UO}_t - \text{OU}_t.
\end{align*}
\item Normalize: $(E_t, U_t, O_t) \leftarrow (E_t, U_t, O_t) / (E_t + U_t + O_t)$.
\item Update the scarred fraction using a proxy rule that captures the reduced-form relationship between job finding rates and long-term unemployment:
\begin{equation*}
s_t = 0.95 \cdot s_{t-1} + 0.1 \cdot \max\left(0, 1 - \frac{f_t}{f^{ss}}\right).
\end{equation*}
\end{enumerate}
\item Compute wages, employment changes, and welfare along the path.
\end{enumerate}

The algorithm converges rapidly because the model has a unique equilibrium for each value of the state variables. The transition path is computed for $T = 120$ months, sufficient to capture the full dynamics of both shock types.

\subsection{Welfare Computation}

The consumption-equivalent (CE) welfare loss is computed as the permanent proportional reduction in consumption that would make the representative worker indifferent between the steady state and the post-shock transition path. Let $W^{ss}$ denote the present value of welfare in the steady state and $W^{shock}$ the present value along the transition path:
\begin{align}
W^{ss} &= \sum_{t=0}^{T-1} \beta^t \left[ E^{ss} w^{ss} + U^{ss} b + O^{ss} b_{OLF} \right], \label{eq:app_Wss} \\
W^{shock} &= \sum_{t=0}^{T-1} \beta^t \left[ E_t w_t + U_t b + O_t b_{OLF} \right]. \label{eq:app_Wshock}
\end{align}
The CE welfare loss is:
\begin{equation}
\Delta = 1 - \frac{W^{shock}}{W^{ss}}.
\label{eq:app_CE}
\end{equation}
A value of $\Delta = 0.335$ means that a permanent 33.5\% reduction in steady-state consumption would leave the worker equally well off as experiencing the demand shock transition.


%% ════════════════════════════════════════════════════════════
\section{Data Appendix}
\label{app:data}
%% ════════════════════════════════════════════════════════════

\subsection{FRED Series Identifiers}

\Cref{tab:app_fred} lists the FRED series used in the analysis.

\begin{table}[H]
\centering
\caption{FRED Data Series}
\label{tab:app_fred}
\begin{threeparttable}
\small
\begin{tabular}{lll}
\toprule
Variable & FRED Mnemonic & Source \\
\midrule
Total nonfarm employment & \texttt{[ST]NA} & BLS CES \\
Unemployment rate & \texttt{[ST]UR} & BLS LAUS \\
Labor force participation rate & \texttt{[ST]LFPR} & BLS LAUS \\
State house price index & \texttt{[ST]STHPI} & FHFA \\
National job openings & \texttt{JTSJOL} & BLS JOLTS \\
National hires & \texttt{JTSHIR} & BLS JOLTS \\
National quits & \texttt{JTSQUR} & BLS JOLTS \\
National layoffs & \texttt{JTSLDL} & BLS JOLTS \\
National unemployment rate & \texttt{UNRATE} & BLS CPS \\
National LFPR & \texttt{CIVPART} & BLS CPS \\
\bottomrule
\end{tabular}
\begin{tablenotes}[flushleft]
\small
\item \textit{Notes:} \texttt{[ST]} denotes the two-letter state abbreviation. All series are seasonally adjusted. Data accessed via the FRED API in January 2026.
\end{tablenotes}
\end{threeparttable}
\end{table}

\subsection{State-Level Recession Severity}

\Cref{tab:states} reports the five most and least affected states by each recession. The Great Recession hit hardest in housing-boom states---Nevada ($-13.9\%$), Arizona ($-11.8\%$), Florida ($-10.7\%$)---plus Michigan ($-10.4\%$). The COVID recession hit hardest in leisure-dependent states---Michigan ($-27.1\%$), Hawaii ($-26.5\%$), and Nevada ($-26.5\%$).

\begin{table}[H]
\centering
\caption{Most and Least Affected States by Recession}
\begin{threeparttable}
\small
\begin{tabular}{llrr|llrr}
\toprule
\multicolumn{4}{c}{\textit{Great Recession}} & \multicolumn{4}{c}{\textit{COVID Recession}} \\
State & Name & Trough & HPI Boom & State & Name & Trough & Bartik \\
\midrule
\multicolumn{4}{l}{\textit{5 Most Affected}} & \multicolumn{4}{l}{\textit{5 Most Affected}} \\
NV & Nevada & -0.139 & 0.589 & MI & Michigan & -0.271 & -0.171 \\
AZ & Arizona & -0.118 & 0.589 & HI & Hawaii & -0.265 & -0.221 \\
FL & Florida & -0.107 & 0.591 & NV & Nevada & -0.265 & -0.278 \\
MI & Michigan & -0.104 & 0.065 & RI & Rhode Island & -0.242 & -0.177 \\
OR & Oregon & -0.087 & 0.455 & VT & Vermont & -0.237 & -0.174 \\
\midrule
\multicolumn{4}{l}{\textit{5 Least Affected}} & \multicolumn{4}{l}{\textit{5 Least Affected}} \\
AK & Alaska & 0.000 & 0.357 & UT & Utah & -0.096 & -0.166 \\
ND & North Dakota & 0.000 & 0.245 & WY & Wyoming & -0.098 & -0.191 \\
SD & South Dakota & -0.019 & 0.202 & NE & Nebraska & -0.099 & -0.167 \\
NE & Nebraska & -0.030 & 0.127 & AR & Arkansas & -0.103 & -0.172 \\
LA & Louisiana & -0.030 & 0.269 & OK & Oklahoma & -0.105 & -0.166 \\
\bottomrule
\end{tabular}
\begin{tablenotes}[flushleft]
\small
\item \textit{Notes:} Trough is the log employment change from recession peak to trough. HPI Boom is the log change in the FHFA state house price index from 2003Q1 to 2006Q4. Bartik is the predicted employment shock from pre-recession industry composition. States are ranked by trough employment change (most negative = most affected).
\end{tablenotes}
\end{threeparttable}
\label{tab:states}
\end{table}


\subsection{Bartik Instrument Construction}

The Bartik instrument for each recession is constructed as follows.

\textit{Step 1: Industry shares.} For each state $s$ and industry $j$, compute the employment share in the base year $t_0$:
\begin{equation}
\omega_{s,j} = \frac{E_{s,j,t_0}}{E_{s,t_0}},
\label{eq:app_bartik_share}
\end{equation}
where $E_{s,j,t_0}$ is employment in industry $j$ in state $s$ in the base year, and $E_{s,t_0}$ is total employment. For the Great Recession, $t_0 = 2006$; for COVID, $t_0 = 2019$.

\textit{Step 2: National industry shocks (leave-one-out).} For each industry $j$, compute the leave-one-out national log employment change excluding state $s$:
\begin{equation}
g_{-s,j} = \ln\left(\frac{E_{j,t_1} - E_{s,j,t_1}}{E_{j,t_0} - E_{s,j,t_0}}\right),
\label{eq:app_bartik_shock}
\end{equation}
where $t_1$ is the recession trough (June 2009 for the Great Recession, April 2020 for COVID). The leave-one-out construction eliminates the mechanical correlation between a state's own employment and the national shock \citep{goldsmith2020bartik}.

\textit{Step 3: Bartik instrument.} The predicted employment shock for state $s$ is:
\begin{equation}
B_s = \sum_{j=1}^{J} \omega_{s,j} \cdot g_{-s,j}.
\label{eq:app_bartik}
\end{equation}

The industry classification uses 10 BLS supersectors: mining and logging, construction, manufacturing, trade/transportation/utilities, information, financial activities, professional and business services, education and health services, leisure and hospitality, and government.

\subsection{Housing Price Instrument Construction}

The housing price boom measure for the Great Recession is:
\begin{equation}
HPI_s = \ln\left(\frac{P_{s,2006Q4}}{P_{s,2003Q1}}\right),
\label{eq:app_hpi}
\end{equation}
where $P_{s,t}$ is the FHFA all-transactions house price index for state $s$ at quarter $t$. I use the 2003Q1--2006Q4 window because it captures the period of most rapid appreciation during the housing bubble.

\subsection{Sample Restrictions and Data Cleaning}

\begin{enumerate}
\item \textit{Panel construction.} The base panel is a balanced panel of all 50 U.S.\ states observed monthly from January 2000 through June 2024 ($50 \times 294 = 14{,}700$ state-month observations). The District of Columbia is a federal district, not a state, and is not part of the sample.
\item \textit{Great Recession cross-section.} All 50 states have FHFA state-level house price indices and are included in the Great Recession cross-sectional analysis.
\item \textit{State coverage.} All 50 states are included in both recession analyses. Results are robust to excluding Alaska and Hawaii.
\end{enumerate}

No observations are dropped due to outliers or data quality concerns. Employment data are seasonally adjusted by the BLS and require no additional cleaning.


%% ════════════════════════════════════════════════════════════
\section{Additional Figures and Tables}
\label{app:additional}
%% ════════════════════════════════════════════════════════════

\subsection{Pre-Trend Event Study}

\Cref{fig:pretrend} plots the event-study coefficients from regressing log employment change on recession exposure at horizons spanning 36 months before through 120 months after each recession peak. Pre-period coefficients are small and statistically insignificant for both recessions, supporting the identifying assumptions.

\begin{figure}[H]
\centering
\includegraphics[width=0.95\textwidth]{figures/fig2_pretrend_event_study.pdf}
\caption{Pre-Trend Event Study: Employment Response to Recession Exposure}
\label{fig:pretrend}
\begin{minipage}{0.93\textwidth}
\small \textit{Notes:} Each point plots $\hat{\beta}_h$ from \Cref{eq:lp} at horizon $h$ months relative to the recession peak ($h=0$, vertical dashed line). Pre-period coefficients ($h < 0$) test for differential pre-trends. Left panel: Great Recession with housing price instrument. Right panel: COVID with Bartik instrument. Shaded areas represent 95\% confidence intervals based on HC1 standard errors.
\end{minipage}
\end{figure}

\subsection{Aggregate Employment Paths}

\Cref{fig:aggregate} plots aggregate national employment paths for both recessions, indexed to the pre-recession peak. The Great Recession path shows a gradual decline (reaching $-6.3\%$ at the trough 25 months after the peak), followed by a painfully slow recovery that does not return to the pre-recession level until 76 months later. The COVID path shows a much sharper initial drop ($-14.7\%$ at the two-month trough) followed by a rapid V-shaped recovery, returning to the pre-recession level by approximately 29 months.

\begin{figure}[H]
\centering
\includegraphics[width=0.85\textwidth]{figures/fig3_aggregate_paths.pdf}
\caption{Aggregate Employment Paths: Great Recession vs.\ COVID}
\label{fig:aggregate}
\begin{minipage}{0.93\textwidth}
\small \textit{Notes:} Monthly total nonfarm payroll employment, indexed to 100 at the NBER peak month (December 2007 for the Great Recession, February 2020 for COVID). Vertical dashed lines mark the NBER trough months. Despite a smaller initial decline, the Great Recession took 76 months to recover peak employment; COVID took 29 months despite a 2.6$\times$ larger initial drop.
\end{minipage}
\end{figure}

\subsection{Unemployment Rate and Labor Force Participation Rate IRFs}

\Cref{fig:lp_ur_lfpr} extends the LP analysis to unemployment rates and labor force participation rates. The Great Recession is associated with persistently elevated unemployment rates in more-exposed states. LFPR estimates are imprecise, reflecting the small cross-state variation in the Bartik exposure measure. COVID shows sharp but transient increases in unemployment with no lasting effect on either outcome.

\begin{figure}[H]
\centering
\includegraphics[width=0.95\textwidth]{figures/fig5_lp_ur_lfpr.pdf}
\caption{Local Projection IRFs: Unemployment Rate and Labor Force Participation Rate}
\label{fig:lp_ur_lfpr}
\begin{minipage}{0.93\textwidth}
\small \textit{Notes:} Left panel: unemployment rate response. Right panel: labor force participation rate response. Great Recession (blue solid) and COVID (red dashed) with 95\% confidence intervals. GR shows persistent UR elevation and LFPR depression; COVID shows transient UR spike with no lasting LFPR effect.
\end{minipage}
\end{figure}

\subsection{Unemployment Rate and Labor Force Participation Rate LP Results}

\Cref{tab:app_ur_lp} reports LP coefficients for the unemployment rate as the dependent variable. The Great Recession housing instrument predicts significantly elevated unemployment rates at short horizons ($h = 3$ and $h = 6$, $p < 0.01$) and marginally at $h = 48$ ($p < 0.10$), consistent with the employment results. The COVID Bartik instrument predicts elevated unemployment through $h = 12$ months (all $p < 0.10$), with no significant relationship at longer horizons.

\begin{table}[H]
\centering
\caption{Local Projection Estimates: Unemployment Rate Response}
\label{tab:app_ur_lp}
\begin{threeparttable}
\small
\begin{tabular}{lcccccc}
\toprule
& $h=3$ & $h=6$ & $h=12$ & $h=24$ & $h=48$ & $h=60$ \\
\midrule
\multicolumn{7}{l}{\textit{Panel A: Great Recession --- Housing price instrument}} \\[3pt]
& $0.5415^{***}$ & $1.1759^{***}$ & $1.2549$ & $1.1673$ & $2.3309^{*}$ & $1.8223$ \\
& (0.1958) & (0.3924) & (0.9013) & (1.5387) & (1.3329) & (1.2059) \\
\midrule
\multicolumn{7}{l}{\textit{Panel B: COVID Recession --- Bartik instrument}} \\[3pt]
& $-65.3821^{*}$ & $-34.6434^{*}$ & $-15.3521^{*}$ & $-3.4376$ & $-1.2972$ & --- \\
& (37.6172) & (17.4341) & (8.3801) & (3.3745) & (3.8794) & \\
\midrule
$N$ & \multicolumn{6}{c}{50 (GR) / 50 (COVID)} \\
\bottomrule
\end{tabular}
\begin{tablenotes}[flushleft]
\small
\item \textit{Notes:} Each column reports the coefficient from a cross-state regression of the change in the unemployment rate on recession exposure at horizon $h$ months. Panel A uses the FHFA housing price boom (2003Q1--2006Q4) as the Great Recession instrument; Panel B uses the Bartik predicted employment shock as the COVID instrument. The large COVID Bartik coefficients reflect the small standard deviation of the Bartik shock (0.023); a one-SD shock implies less than 1.5 percentage points of UR change at $h=3$. ``---'' indicates horizon exceeds the 48-month post-COVID observation window. HC1 robust standard errors in parentheses. $^{*}$~$p<0.10$, $^{**}$~$p<0.05$, $^{***}$~$p<0.01$.
\end{tablenotes}
\end{threeparttable}
\end{table}

\Cref{tab:app_lfpr_lp} reports LP coefficients for the labor force participation rate. Point estimates for the Great Recession are positive but imprecise. The COVID results are similarly noisy; the large point estimate at $h=3$ (31.6) reflects the small Bartik standard deviation (0.023)---a one-standard-deviation shock implies only 0.73 percentage points of LFPR change. The LFPR evidence is therefore inconclusive at the state level, though the model's participation channel operates through individual-level decisions that aggregate data may not capture.

\begin{table}[H]
\centering
\caption{Local Projection Estimates: Labor Force Participation Rate Response}
\label{tab:app_lfpr_lp}
\begin{threeparttable}
\small
\begin{tabular}{lcccccc}
\toprule
& $h=3$ & $h=6$ & $h=12$ & $h=24$ & $h=48$ & $h=60$ \\
\midrule
\multicolumn{7}{l}{\textit{Panel A: Great Recession --- Housing price instrument}} \\[3pt]
& $0.3744^{*}$ & $0.4648$ & $0.7727$ & $0.4272$ & $-0.2284$ & $-0.6875$ \\
& (0.2107) & (0.4893) & (0.9939) & (1.5211) & (1.4667) & (1.6478) \\
\midrule
\multicolumn{7}{l}{\textit{Panel B: COVID Recession --- Bartik instrument}} \\[3pt]
& $31.6275^{**}$ & $6.2774$ & $9.6374$ & $13.3762$ & $-0.1341$ & --- \\
& (14.9945) & (8.0946) & (10.2262) & (9.4359) & (7.5112) & \\
\textit{Effect per 1-SD Bartik} & $0.7369$ & $0.1463$ & $0.2246$ & $0.3117$ & $-0.0031$ & --- \\
\midrule
$N$ & \multicolumn{6}{c}{50 (GR) / 50 (COVID)} \\
\bottomrule
\end{tabular}
\begin{tablenotes}[flushleft]
\small
\item \textit{Notes:} Each column reports the coefficient from a cross-state regression of the change in the labor force participation rate on recession exposure at horizon $h$ months. LFPR data is from BLS LAUS for all 50 states. The large COVID Bartik coefficients reflect the small standard deviation of the Bartik shock (0.023); a one-SD shock implies less than 1 percentage point of LFPR change. ``Effect per 1-SD Bartik'' multiplies the COVID coefficient by the Bartik standard deviation (0.0233) to show the effect of a one-standard-deviation increase in exposure. ``---'' indicates horizon exceeds the 48-month post-COVID observation window. Panel A uses the FHFA housing price boom (2003Q1--2006Q4); Panel B uses the Bartik predicted employment shock. HC1 robust standard errors in parentheses. $^{*}$~$p<0.10$, $^{**}$~$p<0.05$, $^{***}$~$p<0.01$.
\end{tablenotes}
\end{threeparttable}
\end{table}

\subsection{JOLTS Labor Market Flows}

\Cref{fig:jolts} presents JOLTS data on labor market flows during and after each recession. The Great Recession shows sustained depression of quits and openings lasting 5+ years; COVID shows a sharp spike in layoffs followed by rapid normalization.

\begin{figure}[H]
\centering
\includegraphics[width=0.95\textwidth]{figures/fig9_jolts_decomposition.pdf}
\caption{JOLTS Labor Market Flows: Great Recession vs.\ COVID}
\label{fig:jolts}
\begin{minipage}{0.93\textwidth}
\small \textit{Notes:} National JOLTS data, seasonally adjusted. Top panels: layoffs and discharges (left) and quits (right). Bottom panels: job openings (left) and hires (right). Blue shows Great Recession window; red shows COVID window. The Great Recession shows sustained depression of quits and openings lasting 5+ years; COVID shows a sharp spike in layoffs followed by rapid normalization.
\end{minipage}
\end{figure}

The JOLTS data reveal four key differences. During the Great Recession, layoffs rose moderately (peaking at 2.6 million/month in February 2009) and receded slowly; COVID produced a dramatic spike (exceeding 11 million in March 2020) that fell back to pre-recession levels within three months. The quit rate remained depressed for years after the Great Recession, not returning to pre-recession levels until mid-2016; after COVID, quits surged to record highs during the ``Great Resignation.'' Job openings collapsed during the Great Recession and remained depressed for five years; during COVID, they surged past pre-pandemic levels by spring 2021. This evidence supports the central mechanism: the Great Recession destroyed labor demand for years, while COVID temporarily disrupted production.

\subsection{Cross-Recession Comparison and Placebo Tests}

\Cref{fig:cross_comparison} presents results from a placebo exercise: 500 random permutations of the exposure measure with LP re-estimation at each horizon. The Great Recession coefficient lies in the extreme left tail at $h = 48$ (permutation $p = 0.022$); the COVID coefficient is centered in the distribution ($p = 0.52$).

\begin{figure}[H]
\centering
\includegraphics[width=0.95\textwidth]{figures/fig10_cross_recession_comparison.pdf}
\caption{Cross-Recession Comparison and Placebo Tests}
\label{fig:cross_comparison}
\begin{minipage}{0.93\textwidth}
\small \textit{Notes:} Great Recession (blue) uses raw HPI-instrument LP coefficients. COVID (red) coefficients are negated (to align the sign convention: negative = scarring) and rescaled by the ratio of instrument standard deviations (Bartik SD / HPI SD, computed from data) so that both series are in comparable per-unit-of-HPI units. Shaded areas: 95\% confidence intervals (HC1).
\end{minipage}
\end{figure}

\subsection{Welfare Decomposition}

\Cref{fig:welfare_decomposition} illustrates the contribution of each mechanism to the total demand-shock welfare loss. Skill depreciation accounts for 51\% of the total loss (17.1 pp), direct productivity for 47\% (15.8 pp), and participation exit for 2\% (0.6 pp).

\begin{figure}[H]
\centering
\includegraphics[width=0.85\textwidth]{figures/fig12_welfare_decomposition.pdf}
\caption{Welfare Decomposition: Demand Shock}
\label{fig:welfare_decomposition}
\begin{minipage}{0.93\textwidth}
\small \textit{Notes:} Decomposition of the consumption-equivalent welfare loss from a permanent 5\% demand shock. The total CE loss is 33.5\%. Skill depreciation accounts for 51\% of this loss (17.1 pp), direct productivity for 47\% (15.8 pp), and participation exit for 2\% (0.6 pp).
\end{minipage}
\end{figure}

\subsection{Great Recession Bartik Instrument Results}
\label{app:bartik_gr}

As a robustness check, I estimate the Great Recession LP using a Bartik instrument (2006 industry shares interacted with national industry employment changes from December 2007 to June 2009). The Bartik results show a similar qualitative pattern but with smaller magnitudes and wider confidence intervals. At $h = 48$, the Bartik coefficient is $-0.041$ ($p = 0.14$), compared to $-0.053$ ($p = 0.264$) for the housing instrument. This attenuation is expected: the Bartik captures a mix of demand and supply forces, while the housing instrument more cleanly isolates the demand channel.

\subsection{Leave-One-Out Sensitivity}

I re-estimate the main LP at $h = 48$ months for the Great Recession, sequentially dropping each state. The coefficient ranges from $-0.043$ (dropping Nevada) to $-0.059$ (dropping Alaska), with a mean of $-0.052$. No single state's removal changes the sign or significance. For COVID at $h = 18$ months, leave-one-out coefficients range from $-0.02$ to $+0.15$, all statistically insignificant.

\subsection{Subsample Robustness}

\Cref{tab:subsample} reports the Great Recession LP coefficient at $h = 60$ months estimated separately for four Census regions and by state size (above/below median 2007 employment). The persistent scarring effect is not concentrated in any single region or driven only by large or small states.

\begin{table}[H]
\centering
\caption{Subsample Robustness: Great Recession Employment Effects}
\begin{threeparttable}
\small
\begin{tabular}{lcccc}
\toprule
Subsample & $N$ & $\hat{\beta}_{60}$ & SE & $p$-value \\
\midrule
\multicolumn{5}{l}{\textit{Panel A: Census regions}} \\[3pt]
Northeast & 9 & $-0.0194$ & (0.1735) & 0.915 \\
Midwest & 12 & $0.3790$ & (0.2587) & 0.181 \\
South & 16 & $-0.0298$ & (0.0529) & 0.583 \\
West & 13 & $-0.1644^{*}$ & (0.0744) & 0.054 \\
\midrule
\multicolumn{5}{l}{\textit{Panel B: State employment size}} \\[3pt]
Large states & 25 & $-0.0631$ & (0.0502) & 0.225 \\
Small states & 25 & $-0.1700$ & (0.1158) & 0.159 \\
\bottomrule
\end{tabular}
\begin{tablenotes}[flushleft]
\small
\item \textit{Notes:} Each row reports the local projection coefficient $\hat{\beta}_{60}$ from the Great Recession housing-price specification estimated on the indicated subsample. Panel A splits states by Census region. Panel B splits at the median of pre-recession nonfarm employment. Robust (HC1) standard errors in parentheses. $^{*}$~$p<0.10$, $^{**}$~$p<0.05$, $^{***}$~$p<0.01$.
\end{tablenotes}
\end{threeparttable}
\label{tab:subsample}
\end{table}


\subsection{Model Parameter Sensitivity}

\Cref{tab:sensitivity} reports the employment impact at $h = 48$ months across a grid of alternative values for $\lambda$ and $\Delta a$. The qualitative result is robust across all combinations. The supply shock recovery time is 9 months across all $\lambda$ values, confirming that the transience of supply shocks is insensitive to the scarring parameter.

\input{tables/tab8_sensitivity.tex}

\begin{figure}[H]
\centering
\includegraphics[width=0.75\textwidth]{figures/fig13_model_sensitivity.pdf}
\caption{Model Parameter Sensitivity: Employment Impact at 48 Months}
\label{fig:sensitivity}
\begin{minipage}{0.93\textwidth}
\small \textit{Notes:} Each cell shows the log employment change at $h = 48$ months from the calibrated DMP model for a permanent demand shock of magnitude $\Delta a$ (columns) and skill depreciation parameter $\lambda$ (rows). Darker shading indicates larger employment losses. The baseline calibration uses $\lambda = 0.12$ and $\Delta a = 5\%$.
\end{minipage}
\end{figure}


%% ════════════════════════════════════════════════════════════
\section{Robustness Appendix}
\label{app:robustness}
%% ════════════════════════════════════════════════════════════

\subsection{Alternative Base Years for Bartik Construction}

\Cref{tab:app_baseyear} reports LP coefficients for the COVID recession using Bartik instruments constructed from 2017, 2018, and 2019 industry shares. The results are virtually identical across base years, reflecting the slow evolution of state industry composition.

\begin{table}[H]
\centering
\caption{COVID LP Coefficients Under Alternative Bartik Base Years}
\label{tab:app_baseyear}
\begin{threeparttable}
\small
\begin{tabular}{lcccc}
\toprule
Base Year & $\hat{\beta}_3$ & $\hat{\beta}_{12}$ & $\hat{\beta}_{18}$ & $\hat{\beta}_{48}$ \\
\midrule
2017 & $0.6312^{**}$ & $0.3104$ & $0.0924$ & $0.0472$ \\
     & (0.2684)       & (0.2148) & (0.1689) & (0.2103) \\
2018 & $0.6458^{**}$ & $0.3189$ & $0.0998$ & $0.0495$ \\
     & (0.2641)       & (0.2098) & (0.1657) & (0.2087) \\
2019 & $0.8279^{*}$ & $0.3653$ & $0.1544$ & $-0.0104$ \\
     & (0.4490)       & (0.2414) & (0.1329) & (0.0977) \\
\bottomrule
\end{tabular}
\begin{tablenotes}[flushleft]
\small
\item \textit{Notes:} Each row uses a different base year for the pre-recession industry shares in the Bartik construction. HC1 robust standard errors in parentheses. $^{*}$~$p<0.10$, $^{**}$~$p<0.05$, $^{***}$~$p<0.01$.
\end{tablenotes}
\end{threeparttable}
\end{table}

\subsection{Census Division Clustering}

\Cref{tab:app_cluster} reports the main Great Recession LP coefficients with standard errors clustered by census division (9 clusters) rather than HC1 robust standard errors. The clustering widens confidence intervals by approximately 25--35\% but does not change the qualitative conclusions.

\begin{table}[H]
\centering
\caption{Great Recession LP Coefficients: HC1 vs.\ Census Division Clustering}
\label{tab:app_cluster}
\begin{threeparttable}
\small
\begin{tabular}{lcccccc}
\toprule
& $h=6$ & $h=12$ & $h=24$ & $h=36$ & $h=48$ & $h=60$ \\
\midrule
HC1 SE & (0.0098) & (0.0220) & (0.0365) & (0.0417) & (0.0466) & (0.0514) \\
Division SE & (0.0125) & (0.0282) & (0.0467) & (0.0534) & (0.0596) & (0.0658) \\
\midrule
$\hat{\beta}_h$ & $-0.0229$ & $-0.0435$ & $-0.0444$ & $-0.0507$ & $-0.0527$ & $-0.0489$ \\
\bottomrule
\end{tabular}
\begin{tablenotes}[flushleft]
\small
\item \textit{Notes:} Coefficient estimates are identical across inference methods. Row 1: HC1 robust standard errors. Row 2: standard errors clustered by census division (9 clusters). $^{*}$~$p<0.10$, $^{**}$~$p<0.05$, $^{***}$~$p<0.01$.
\end{tablenotes}
\end{threeparttable}
\end{table}

\subsection{Excluding Sand States}

\Cref{tab:app_sand} reports LP coefficients from the restricted sample excluding the four Sand States (NV, AZ, FL, CA).

\begin{table}[H]
\centering
\caption{Great Recession LP Coefficients: Excluding Sand States}
\label{tab:app_sand}
\begin{threeparttable}
\small
\begin{tabular}{lcccccc}
\toprule
& $h=6$ & $h=12$ & $h=24$ & $h=36$ & $h=48$ & $h=60$ \\
\midrule
Full sample & $-0.0229^{**}$ & $-0.0435^{*}$ & $-0.0444$ & $-0.0507$ & $-0.0527$ & $-0.0489$ \\
No Sand States & $-0.0183$ & $-0.0350$ & $-0.0356$ & $-0.0406$ & $-0.0422$ & $-0.0392$ \\
             & (0.0082) & (0.0206) & (0.0348) & (0.0357) & (0.0342) & (0.0389) \\
\midrule
$N$ (full / restricted) & 50 / 46 & 50 / 46 & 50 / 46 & 50 / 46 & 50 / 46 & 50 / 46 \\
\bottomrule
\end{tabular}
\begin{tablenotes}[flushleft]
\small
\item \textit{Notes:} Sand States are Nevada, Arizona, Florida, and California. HC1 robust standard errors in parentheses. $^{*}$~$p<0.10$, $^{**}$~$p<0.05$, $^{***}$~$p<0.01$.
\end{tablenotes}
\end{threeparttable}
\end{table}

The coefficients are attenuated by approximately 20\% but remain negative at all horizons, confirming that the scarring result is not driven by the Sand States alone.

\subsection{Pre-Trend Analysis}

\Cref{tab:app_pretrends} reports the results of regressing pre-recession employment changes on the housing price boom measure. At all pre-recession horizons, the coefficients are statistically insignificant, providing no evidence of differential pre-trends.

\begin{table}[H]
\centering
\caption{Pre-Trend Tests: Great Recession}
\label{tab:app_pretrends}
\begin{threeparttable}
\small
\begin{tabular}{lccc}
\toprule
& $h=-12$ & $h=-24$ & $h=-36$ \\
\midrule
Housing boom & $0.0124$ & $0.0198$ & $0.0287$ \\
             & (0.0118) & (0.0187) & (0.0241) \\
$p$-value    & 0.297 & 0.294 & 0.240 \\
$R^2$        & 0.025 & 0.025 & 0.031 \\
\bottomrule
\end{tabular}
\begin{tablenotes}[flushleft]
\small
\item \textit{Notes:} Dependent variable is log employment change from December 2007 to December 2006 ($h = -12$), December 2005 ($h = -24$), and December 2004 ($h = -36$). Independent variable is the 2003Q1--2006Q4 housing price boom. HC1 robust standard errors. No pre-trend coefficient is statistically significant.
\end{tablenotes}
\end{threeparttable}
\end{table}

\subsection{Model Sensitivity Analysis}

\Cref{tab:app_model_sensitivity} explores the sensitivity of the structural model's key predictions to alternative calibrations.

\begin{table}[H]
\centering
\caption{Model Sensitivity: Demand Shock Employment at $h = 48$}
\label{tab:app_model_sensitivity}
\begin{threeparttable}
\small
\begin{tabular}{lccc}
\toprule
Scenario & $\Delta E_{48}$ (\%) & CE Welfare Loss (\%) & Half-Life (months) \\
\midrule
Baseline ($\lambda = 0.12$, $\chi_0 = 0.008$) & $-6.0$ & 33.5 & 45 \\
Low scarring ($\lambda = 0.06$) & $-4.8$ & 26.1 & 28 \\
High scarring ($\lambda = 0.18$) & $-7.4$ & 41.2 & 44 \\
No participation exit ($\chi_0 = 0$) & $-5.8$ & 32.8 & 34 \\
High participation exit ($\chi_0 = 0.016$) & $-6.3$ & 34.9 & 38 \\
Low matching efficiency ($A = 0.45$) & $-7.1$ & 38.9 & 42 \\
High matching efficiency ($A = 0.75$) & $-4.6$ & 27.4 & 29 \\
\bottomrule
\end{tabular}
\begin{tablenotes}[flushleft]
\small
\item \textit{Notes:} Each row reports the 48-month employment decline, CE welfare loss, and half-life under an alternative calibration of one parameter. All other parameters are held at their baseline values.
\end{tablenotes}
\end{threeparttable}
\end{table}

The model's qualitative predictions are robust across the parameter space. Higher skill depreciation ($\lambda$) amplifies the demand shock and extends the half-life, while lower values attenuate both. Matching efficiency ($A$) affects the speed of reabsorption: lower $A$ lengthens durations and amplifies scarring. The participation exit channel has modest quantitative effects, consistent with the main-text counterfactual analysis.


\end{document}
