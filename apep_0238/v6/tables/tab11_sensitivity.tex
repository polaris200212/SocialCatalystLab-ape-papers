\begin{table}[H]
\centering
\caption{Model Sensitivity: Employment Impact at 48 Months Across Parameter Values}
\begin{threeparttable}
\begin{tabular}{lccc}
\toprule
& $\Delta a = 3\%$ & $\Delta a = 5\%$ & $\Delta a = 7\%$ \\
\midrule
$\lambda = 0.01$ & $-0.012$ & $-0.022$ & $-0.034$ \\
$\lambda = 0.02$ & $-0.013$ & $-0.024$ & $-0.036$ \\
$\lambda = 0.03$ & $-0.014$ & $-0.025$ & $-0.039$ \\
$\lambda = 0.05$ & $-0.016$ & $-0.030$ & $-0.047$ \\
$\lambda = 0.10$ & $-0.024$ & $-0.047$ & $-0.080$ \\
$\lambda = 0.12$ & $-0.029$ & $-0.060$ & $-0.106$ \\
\bottomrule
\end{tabular}
\begin{tablenotes}[flushleft]
\small
\item \textit{Notes:} Each cell reports the log employment change at $h = 48$ months from the calibrated DMP model following a permanent demand shock of magnitude $\Delta a$. $\lambda$ is the skill depreciation rate upon crossing the scarring threshold. All parameter combinations produce permanent demand-shock effects (half-life exceeds 120 months), while the temporary supply shock (not shown) recovers within 9 months regardless of $\lambda$. The baseline calibration uses $\lambda = 0.12$ and $\Delta a = 5\%$.
\end{tablenotes}
\end{threeparttable}
\label{tab:sensitivity}
\end{table}
