\begin{table}[H]
\centering
\caption{Employment Persistence: Half-Lives and Recovery Measures}
\begin{threeparttable}
\begin{tabular}{lcc}
\toprule
& Great Recession & COVID Recession \\
\midrule
Peak response ($\hat{\beta}_{peak}$) & $-0.0527$ & $-0.8279$ \\
Peak horizon (months) & 48 & 3 \\
Half-life (months) & 60 & 9 \\
$\hat{\beta}_{48}$ & $-0.0527$ & $0.0104$ \\
Persistence ratio ($\hat{\beta}_{48} / \hat{\beta}_{peak}$) & $1.000$ & $-0.013$ \\
\midrule
Instrument & Housing price boom & Bartik (industry shares) \\
States & 50 & 50 \\
\bottomrule
\end{tabular}
\begin{tablenotes}[flushleft]
\small
\item \textit{Notes:} Peak response is the most negative $\hat{\beta}_h$ among the discrete horizons reported in Table~\\ref{tab:main}. COVID coefficients are sign-reversed (multiplied by $-1$) to match the Table~\\ref{tab:main} convention where negative values indicate employment loss. Half-life is the number of months after peak until $|\\hat{\\beta}_h|$ decays to half its peak value, computed from the full LP horizon grid at 3-month intervals (not only the sparse horizons in Table~\\ref{tab:main}). For the Great Recession, the effect remains near its peak through $h = 84$ and crosses the half-peak threshold between $h = 108$ and $h = 111$. Persistence ratio measures how much of the peak effect remains at $h=48$ months.
\end{tablenotes}
\end{threeparttable}
\label{tab:halflife}
\end{table}
