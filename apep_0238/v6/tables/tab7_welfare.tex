\begin{table}[H]
\centering
\caption{Welfare Losses from Demand vs.\ Supply Shocks: Model Counterfactuals}
\begin{threeparttable}
\begin{tabular}{lcc}
\toprule
Scenario & CE Welfare Loss (\%) & Relative to Baseline \\
\midrule
\multicolumn{3}{l}{\textit{Panel A: Demand shock (Great Recession analog)}} \\[3pt]
Baseline (all mechanisms) & 39.61 & 1.00 \\
No skill depreciation ($\lambda = 0$) & 16.60 & 0.42 \\
No OLF exit ($\chi = 0$) & 38.82 & 0.98 \\
\midrule
\multicolumn{3}{l}{\textit{Panel B: Supply shock (COVID analog)}} \\[3pt]
Baseline (temporary shock) & 0.09 & --- \\
\midrule
Demand/Supply welfare ratio & \multicolumn{2}{c}{442.0} \\
\bottomrule
\end{tabular}
\begin{tablenotes}[flushleft]
\small
\item \textit{Notes:} Consumption-equivalent (CE) welfare losses are computed as the permanent percentage reduction in consumption that would make agents indifferent between the steady state and the post-shock transition path. The demand shock reduces aggregate productivity by 5\% permanently. The supply shock doubles the separation rate for 3 months.
\end{tablenotes}
\end{threeparttable}
\label{tab:welfare}
\end{table}
