\begin{table}[H]
\centering
\caption{Summary Statistics}
\begin{threeparttable}
\begin{tabular}{lrrrrr}
\toprule
Variable & Mean & Std. Dev. & Min & Max & N \\
\midrule
\multicolumn{6}{l}{\textit{Panel A: State-level outcomes (monthly, 2000--2024)}} \\[3pt]
Nonfarm payrolls (thousands) & 2,773.1 & 2,963.4 & 237.3 & 18,010.0 & 14,700 \\
Unemployment rate (\%) & 5.3 & 2.2 & 1.7 & 30.5 & 14,700 \\
Labor force participation rate (\%) & 64.6 & 4.0 & 53.8 & 73.6 & 5,880 \\
\midrule
\multicolumn{6}{l}{\textit{Panel B: Recession exposure measures}} \\[3pt]
Bartik shock: Great Recession & -0.0558 & 0.0166 & -0.1196 & -0.0283 & 50 \\
Bartik shock: COVID & -0.1770 & 0.0293 & -0.3164 & -0.1018 & 50 \\
Housing price boom (log, 2003--2006) & 0.3033 & 0.1466 & 0.0653 & 0.6146 & 50 \\
Peak-to-trough employment (GR) & -0.0593 & 0.0272 & -0.1393 & 0.0000 & 50 \\
Peak-to-trough employment (COVID) & -0.1547 & 0.0447 & -0.2708 & -0.0963 & 50 \\
\bottomrule
\end{tabular}
\begin{tablenotes}[flushleft]
\small
\item \textit{Notes:} Panel A reports summary statistics for the balanced state-month panel. Nonfarm payrolls are from BLS Current Employment Statistics via FRED. Labor force participation rate is available from BLS LAUS for 20 states over the full sample period, yielding 5,880 state-month observations; LFPR results (reported in Appendix C) use this subsample. Panel B reports cross-state recession exposure measures. Bartik shocks are constructed using pre-recession industry employment shares weighted by national industry employment changes. Housing price boom is the log change in the FHFA state-level house price index from 2003Q1 to 2006Q4.
\end{tablenotes}
\end{threeparttable}
\label{tab:summary}
\end{table}
