\documentclass[12pt]{article}

% UTF-8 encoding and fonts
\usepackage[utf8]{inputenc}
\usepackage[T1]{fontenc}
\usepackage{lmodern}

% Page setup
\usepackage[margin=1in]{geometry}
\usepackage{setspace}
\onehalfspacing

% Math and symbols
\usepackage{amsmath,amssymb}

% Graphics
\usepackage{graphicx}
\usepackage{float}

% Tables
\usepackage{booktabs}
\usepackage{array}
\usepackage{multirow}
\usepackage{tabularx}

% Hyperlinks
\usepackage{hyperref}
\hypersetup{
    colorlinks=true,
    linkcolor=blue,
    citecolor=blue,
    urlcolor=blue
}

% Captions
\usepackage{caption}
\captionsetup{font=small,labelfont=bf}

% Section formatting
\usepackage{titlesec}
\titleformat{\section}{\large\bfseries}{\thesection.}{0.5em}{}
\titleformat{\subsection}{\normalsize\bfseries}{\thesubsection}{0.5em}{}

% Custom commands
\newcommand{\E}{\mathbb{E}}

\title{Betting on Jobs? The Employment Effects of \\
Legal Sports Betting in the United States}
\author{APEP Autonomous Research\thanks{Prepared for AEJ: Economic Policy. Replication materials available upon request. We thank seminar participants for helpful comments. All errors are our own.} nd @dakoyana}}
\date{January 2026}

\begin{document}

\maketitle

\begin{abstract}
\noindent
The legalization of sports betting following the Supreme Court's 2018 \emph{Murphy v. NCAA} decision created a natural experiment affecting 38 states by 2024. This paper provides the first rigorous causal estimates of the employment effects of sports betting legalization using a difference-in-differences design that exploits staggered state adoption. Employing the Callaway-Sant'Anna estimator to address heterogeneous treatment effects across adoption cohorts, we find that legalization increased gambling industry employment by approximately 1,122 jobs per state (95\% CI: [787, 1,457]), with effects concentrated in states that permitted mobile betting. The estimates are robust to alternative estimators including Sun-Abraham and two-way fixed effects specifications. Pre-treatment event study coefficients show no evidence of differential pre-trends, and placebo tests on unrelated industries confirm the specificity of our findings. However, these gains appear modest relative to industry projections of 200,000+ jobs nationally, suggesting that much of the predicted job creation reflected either formalization of existing informal betting or displacement from other gambling sectors. Our findings inform ongoing policy debates as remaining states consider legalization, while highlighting important caveats regarding measurement and confounding from the COVID-19 pandemic.
\end{abstract}

\vspace{1em}
\noindent\textbf{JEL Codes:} J21, L83, H71, K23 \\
\noindent\textbf{Keywords:} sports betting, gambling, employment, difference-in-differences, state policy

\newpage

\section{Introduction}

On May 14, 2018, the Supreme Court fundamentally reshaped the American gambling landscape. In \emph{Murphy v. National Collegiate Athletic Association}, the Court struck down the Professional and Amateur Sports Protection Act (PASPA) of 1992, which had effectively prohibited sports betting outside Nevada for over two decades. Within weeks, states began legalizing sports wagering, creating a staggered natural experiment across the country. By the end of 2024, 38 states plus the District of Columbia had legalized some form of sports betting, generating over \$10 billion in annual handle and transforming a once-underground industry into a mainstream entertainment sector.

This paper asks a simple but policy-relevant question: did sports betting legalization create jobs? The gambling industry and its advocates have promoted sports betting as an engine of economic development, projecting that nationwide legalization could support over 200,000 jobs and generate \$8 billion in tax revenue (American Gaming Association, 2018). State legislators considering legalization have cited job creation as a primary justification, and labor unions have negotiated agreements with sportsbook operators for worker protections. Yet despite the magnitude of these claims, no study has rigorously estimated the causal employment effects of sports betting legalization.

We address this gap using a difference-in-differences (DiD) research design that exploits the staggered adoption of sports betting across states following the \emph{Murphy} decision. Our identification strategy compares employment outcomes in states that legalized sports betting to outcomes in states that did not (or had not yet) legalized, before and after legalization. The key identifying assumption is that, absent legalization, employment in treated states would have evolved similarly to employment in control states---the parallel trends assumption. We provide visual and statistical evidence supporting this assumption.

A methodological challenge in our setting is that treatment effects may vary across adoption cohorts and over time since treatment. Recent econometric research has shown that standard two-way fixed effects (TWFE) estimators can be severely biased under such heterogeneity, potentially producing estimates with the wrong sign (de Chaisemartin and D'Haultfoeuille, 2020; Goodman-Bacon, 2021). We address this concern by implementing the Callaway and Sant'Anna (2021) estimator, which constructs group-time average treatment effects that are robust to treatment effect heterogeneity, and aggregates these into interpretable summary measures.

Our main finding is that sports betting legalization increased gambling industry employment by approximately 1,122 jobs per state, with a standard error of 171. This estimate is statistically significant at conventional levels (t-statistic = 6.56). Translating to the national level, our estimates imply that legalization created roughly 38,000--45,000 jobs across all adopting states---a substantial figure, but notably below the 200,000+ jobs projected by industry advocates.

We document important heterogeneity in treatment effects. States that permitted both retail and mobile sports betting experienced employment gains roughly double those of states that legalized only retail betting. This pattern is consistent with mobile betting's larger market penetration and the workforce required to operate digital platforms, including customer service, compliance, and technology personnel. We also find that employment effects grow over time following legalization, consistent with gradual market development.

Our estimates survive a battery of robustness checks. We show that pre-treatment event study coefficients are statistically indistinguishable from zero, supporting the parallel trends assumption. Results are qualitatively similar using the Sun-Abraham (2021) interaction-weighted estimator and traditional TWFE (though the latter is somewhat attenuated, consistent with known biases). Placebo tests on industries unlikely to be affected by sports betting---such as manufacturing and agriculture---show no effects. We also conduct sensitivity analyses to address potential confounding from the COVID-19 pandemic, which overlapped substantially with the sports betting legalization wave.

Several important caveats apply to our analysis. First, we exclude Nevada from our main sample because it permitted legal sports betting throughout our study period, making it unsuitable as either a treated or control unit in our DiD framework. Second, our outcome measure---NAICS 7132 (Gambling Industries)---captures employment at gambling establishments but may not fully capture jobs at technology companies providing backend services to sportsbooks, nor remote workers who may be employed by out-of-state firms. Third, while we address COVID confounding through sensitivity analysis, we cannot fully rule out pandemic-related bias given the substantial overlap between COVID-19 and the sports betting expansion period.

This paper contributes to several literatures. First, we contribute to the economics of gambling by providing credible causal estimates of employment effects, complementing existing work on problem gambling (Grinols, 2004) and household financial outcomes (Baker et al., 2024). Second, we contribute methodologically by applying recent advances in staggered DiD to a policy context where heterogeneous treatment effects are likely, demonstrating the practical importance of using appropriate estimators. Third, we contribute to policy debates by providing empirical grounding for job creation claims that have influenced legislative deliberations across the country.

The paper proceeds as follows. Section 2 reviews related literature. Section 3 provides institutional background on sports betting legalization. Section 4 describes our data sources and variable construction. Section 5 presents our empirical strategy. Section 6 reports results. Section 7 discusses robustness. Section 8 concludes.

\section{Related Literature}

Our paper relates to three main strands of literature: the economics of gambling, the labor market effects of regulatory policy, and methodological advances in difference-in-differences estimation.

\subsection{Economics of Gambling}

A substantial literature examines the economic effects of casino gambling, which preceded and in many ways shaped the sports betting industry. Grinols and Mustard (2006) study the effect of casinos on county-level crime rates, finding increases in property and violent crime. Evans and Topoleski (2002) examine the effects of Indian casino openings on local employment and poverty, finding significant positive employment effects concentrated in nearby counties. Fink and Rork (2003) study the fiscal effects of state lotteries, while Garrett and Sobel (2004) examine lottery earmarking and fungibility.

More directly relevant to our setting, Baker et al. (2024) study the household-level effects of sports betting legalization, finding that access to legal sports betting increases gambling expenditure and financial distress among vulnerable populations. Their work complements ours by examining the demand-side household impacts, while we focus on supply-side employment effects.

The literature on gambling and employment has produced mixed results. Cotti (2008) finds that casino gambling reduces mortality in counties with casinos, potentially through income effects. Humphreys and Marchand (2013) study the relationship between casinos and local employment in Canada, finding modest positive effects. Nichols and Tosun (2017) examine the regional effects of casino gambling more broadly. Our paper contributes to this literature by providing the first causal estimates for the specific case of sports betting, which differs from casino gambling in important ways---particularly its mobile-first business model and integration with professional sports.

\subsection{Labor Market Effects of State Regulatory Policy}

Our paper also relates to a broader literature studying how state-level regulatory changes affect labor markets. This literature has examined policies ranging from minimum wage increases (Dube, Lester, and Reich, 2010) to occupational licensing (Kleiner and Krueger, 2013) to marijuana legalization (Nicholas and Maclean, 2019). A common challenge in this literature is credibly identifying causal effects when states self-select into policy adoption based on unobserved factors correlated with outcomes.

The staggered adoption of sports betting following an exogenous Supreme Court decision provides a relatively clean setting for causal identification. Unlike policy adoptions driven by state-level political economy, the timing of state-level sports betting legalization post-\emph{Murphy} was largely determined by pre-existing legislative capacity, relationships with gaming interests, and idiosyncratic political factors rather than anticipated employment trends.

\subsection{Difference-in-Differences Methodology}

Our methodological approach draws on recent advances in difference-in-differences estimation with staggered treatment timing. Building on the foundational semiparametric DiD framework of Abadie (2005), the literature has documented important pitfalls in traditional TWFE estimators when treatment effects are heterogeneous across groups or over time (Borusyak, Jaravel, and Spiess, 2024; de Chaisemartin and D'Haultfoeuille, 2020; Goodman-Bacon, 2021; Sun and Abraham, 2021).

Goodman-Bacon (2021) provides the foundational decomposition showing that TWFE estimates are weighted averages of all possible two-group/two-period DiD estimates, with some weights potentially negative. This means that if treatment effects vary across cohorts, TWFE can produce estimates with the wrong sign. Callaway and Sant'Anna (2021) propose an estimator that avoids these problems by using only ``clean'' comparisons---either never-treated or not-yet-treated units as controls. Sun and Abraham (2021) propose an alternative interaction-weighted estimator with similar robustness properties. Cengiz et al. (2019) demonstrate the practical value of ``stacked'' event study designs for policy evaluation.

We follow best practices from this methodological literature. We implement the Callaway-Sant'Anna estimator as our main specification, use Sun-Abraham as a robustness check, and report TWFE for comparison while acknowledging its limitations. We also examine pre-treatment event study coefficients to assess the parallel trends assumption, following Roth (2022) who cautions about the low power of such tests and recommends complementary sensitivity analysis.

Bertrand, Duflo, and Mullainathan (2004) highlight the problem of serially correlated errors in DiD designs and recommend clustering standard errors at the level of policy variation. Cameron and Miller (2015) provide guidance on cluster-robust inference with few clusters. Donald and Lang (2007) and Conley and Taber (2011) discuss inference challenges specific to DiD designs with a small number of policy changes, which is relevant to our setting with approximately 50 state clusters. We cluster at the state level, our unit of treatment assignment, and report wild cluster bootstrap p-values as a robustness check given our modest number of clusters.

\section{Institutional Background}

\subsection{The Professional and Amateur Sports Protection Act}

From 1992 to 2018, the Professional and Amateur Sports Protection Act (PASPA) effectively banned sports betting in all but four states that had existing legal frameworks: Nevada, Delaware, Montana, and Oregon. Of these, only Nevada permitted comprehensive single-game sports wagering; the others were limited to parlay betting or sports lotteries with restrictions that limited market activity. For example, Delaware's sports lottery was limited to parlay bets on NFL games, while Montana and Oregon operated modest sports pools. As a practical matter, Nevada was the only state with a meaningful legal sports betting market during the PASPA era.

PASPA did not make sports betting a federal crime but rather prohibited states from ``authorizing'' or ``licensing'' sports gambling. This anti-commandeering approach---directing states what they could not do rather than criminalizing conduct directly---would prove to be the statute's constitutional vulnerability.

The law was championed by Senator Bill Bradley of New Jersey, a former professional basketball player concerned about the integrity of sporting events. Professional sports leagues strongly supported the legislation, viewing legal sports betting as a threat to the legitimacy of competition. For over two decades, the law remained largely unchallenged.

\subsection{New Jersey's Challenge and Murphy v. NCAA}

New Jersey's path to legal sports betting began in 2011, when voters approved a constitutional amendment permitting sports wagering at Atlantic City casinos and horse racing tracks. The state's motivation was economic: Atlantic City's casino industry faced increasing competition from neighboring states and was in severe decline. State legislators hoped that sports betting could revitalize the struggling gambling sector.

New Jersey's initial attempt to implement sports betting was blocked by the professional sports leagues in federal court. The Third Circuit ruled that New Jersey's legalization violated PASPA. In response, New Jersey pursued a creative legal strategy: rather than affirmatively ``authorizing'' sports betting, the state would simply \emph{repeal} its prohibitions, allowing sports betting to occur without state sanction. This too was challenged and blocked.

The Supreme Court granted certiorari in the New Jersey case, styled \emph{Murphy v. National Collegiate Athletic Association} after New Jersey's governor. On May 14, 2018, the Court ruled 7-2 that PASPA violated the Tenth Amendment's anti-commandeering principle. Justice Alito, writing for the majority, held that Congress cannot ``issue direct orders to state legislatures'' requiring them to maintain laws prohibiting sports betting. The entire statute was invalidated as inseverable.

\subsection{Post-Murphy State Adoption}

The \emph{Murphy} decision immediately opened the door for state-by-state legalization. Unlike federal legalization of an activity, the Court's ruling simply removed a federal barrier, leaving each state to determine its own approach. This created the staggered adoption pattern we exploit for identification.

\textbf{First movers (2018):} Delaware moved fastest, launching legal sports betting on June 5, 2018, less than a month after the \emph{Murphy} decision. Delaware had existing gaming infrastructure and enabling legislation that required minimal modification. New Jersey followed on June 14, launching what would become the nation's largest market outside Nevada. Mississippi began accepting sports bets in August, followed by West Virginia, Pennsylvania, and Rhode Island later in 2018. These six states form our first treatment cohort.

\textbf{Second wave (2019):} Eight additional states legalized in 2019, including major markets like Indiana, Iowa, New Hampshire, and Tennessee. This wave included the first states to launch with mobile-only betting (Tennessee) and states with diverse regulatory models.

\textbf{2020 and COVID-19:} The 2020 cohort included Colorado, Illinois, Michigan, Virginia, and Washington DC. However, market launches were complicated by the COVID-19 pandemic. Illinois temporarily closed its in-person registration requirement; Virginia launched entirely during the pandemic. The sports betting industry itself faced uncertainty as major sports leagues suspended play from March through July 2020.

\textbf{Continued expansion (2021-2024):} The industry continued expanding despite pandemic challenges. Arizona launched in September 2021 timed to the NFL season. New York's mobile launch in January 2022 created the nation's largest market almost overnight. By late 2024, 38 states plus DC had legalized sports betting.

\textbf{Non-adopting states:} Twelve states have not legalized as of 2024. These include populous states like California (which rejected a ballot initiative in November 2022), Texas, and Florida (where legalization faces tribal gaming complications), as well as states with constitutional prohibitions (Utah, Hawaii) or limited political will for gambling expansion.

\subsection{Treatment Definition and Nevada}

A key issue for our research design is how to handle the four states---Nevada, Delaware, Montana, and Oregon---that had pre-existing sports betting authorization before \emph{Murphy}. We treat these states differently based on the nature of their pre-2018 markets:

\textbf{Nevada:} We exclude Nevada from our main sample entirely. Nevada had comprehensive legal sports betting throughout our study period (and for decades prior), making it unsuitable as either a treated or control unit in our DiD framework. Including Nevada as a control would violate the clean comparison assumption; including it as ``always treated'' would require strong assumptions about treatment effect homogeneity over very long time horizons.

\textbf{Delaware, Montana, Oregon:} These states had limited sports betting operations under PASPA exemptions but expanded significantly post-\emph{Murphy}. Delaware transitioned from parlay-only betting to full single-game wagering; Montana and Oregon similarly expanded their offerings. We code these states as treated in 2018 when they expanded, effectively treating the pre-\emph{Murphy} limited operations as equivalent to no treatment. This is consistent with our focus on the employment effects of the post-\emph{Murphy} sports betting expansion rather than the effects of limited pre-existing sports pools.

In robustness checks, we explore alternative treatments of these states, including excluding them entirely. Results are qualitatively similar.

\subsection{Implementation Heterogeneity}

States have implemented sports betting in diverse ways, creating important sources of heterogeneity:

\textbf{Retail vs. Mobile:} Some states initially permitted only retail (in-person) betting at casinos, racetracks, or licensed facilities. Others launched immediately with mobile betting via smartphone apps, or added mobile within months of retail launch. The distinction matters enormously for market size: in mature markets, mobile betting accounts for 80-90\% of handle. The workforce implications also differ---mobile operations require customer service centers, compliance teams, and technology staff that may be located remotely.

\textbf{Licensing and Market Structure:} States vary in the number of licensed operators (from single-operator monopolies in Oregon and New Hampshire to competitive markets with 10+ licensees in states like New Jersey and Illinois), tax rates (from 6.75\% in Nevada to 51\% in New York), and tie-ins to existing gaming interests (casinos, horse racing, tribal gaming).

\textbf{Online-Only Registration:} During the COVID-19 pandemic, several states waived in-person registration requirements, allowing bettors to sign up entirely online. This accelerated market development but also means that pandemic-era adoption differs from earlier periods.

For our analysis, we focus on the date of the first legal sports bet in each state, which captures when market activity---and associated employment---began. We also code implementation type (retail, mobile, or both) to examine heterogeneous effects.

\section{Data}

\subsection{Data Sources}

Our primary employment data source is the Quarterly Census of Employment and Wages (QCEW), a comprehensive database of employment and wages drawn from state unemployment insurance (UI) records. The QCEW covers approximately 97\% of U.S. wage and salary civilian employment and provides industry-level employment counts at the state-by-quarter level.

We focus on NAICS industry code 7132 (Gambling Industries), which includes ``Establishments primarily engaged in operating gambling facilities, such as casinos, bingo halls, and video gaming terminals, or in the provision of gambling services, such as lotteries and off-track betting'' (BLS, 2024). This category captures both traditional casino employment and newer sports betting operations. We aggregate quarterly data to annual to match our policy timing variable and reduce noise from seasonal fluctuations.

Policy timing data comes from Legal Sports Report, an industry publication that tracks state-by-state legalization dates, and is verified against state gaming commission announcements and news reports. Our timing variable captures the calendar year of the first legal sports bet in each state.

\subsection{Measurement Considerations}

Several measurement issues merit discussion:

\textbf{NAICS 7132 scope:} NAICS 7132 captures employment at gambling establishments but has limitations for measuring sports betting employment specifically. First, sports betting is often integrated into existing casino operations, with workers performing multiple functions across gaming types. Second, the major mobile sportsbook operators (DraftKings, FanDuel, BetMGM, Caesars) employ substantial workforces in technology, customer service, and corporate functions that may be coded to other industries (e.g., data processing, corporate headquarters) or located in states other than where bets are placed. Third, the category includes traditional casino and lottery employment, creating a noisy measure of sports-betting-specific employment.

\textbf{Suppression and imputation:} QCEW data are suppressed when publication could reveal information about individual establishments. This is common in states with few gambling establishments. We use linear interpolation for occasional missing values and exclude state-years with systematic suppression.

\textbf{Geographic attribution:} Workers in the mobile sports betting industry may be located in states different from where customers place bets. Customer service centers, for example, may be concentrated in states with favorable labor markets. Our state-level employment measure attributes jobs to the state where the worker is located (based on UI records), not where revenue is generated.

\subsection{Sample Construction}

Our analysis sample spans 2014-2024, providing four years of pre-treatment data (before the May 2018 \emph{Murphy} decision) and six years of post-treatment data. We begin in 2014 to have sufficient pre-treatment observations for event study analysis while avoiding excessive extrapolation.

The sample includes 50 states plus the District of Columbia, minus Nevada which we exclude as always-treated. Our final estimation sample contains 50 units $\times$ 11 years = 550 state-year observations, though some specifications using never-treated controls have fewer effective observations.

\subsection{Variable Definitions}

Our primary outcome variable is state-level gambling industry employment (number of employees in NAICS 7132, annual average). Our treatment variable is an indicator for whether a state had legalized sports betting by a given year. The treatment turns on in the calendar year of first legal wagering (e.g., a state with first bet in November 2019 is coded as treated starting in 2019).

We code treatment cohorts ($G_s$) as the year of first legal wagering. States that never legalized through 2024 are coded as $G = \infty$ (never-treated) and serve as our primary control group in the Callaway-Sant'Anna estimator. In robustness checks, we also use not-yet-treated states as controls.

We construct implementation type indicators based on whether states permitted retail betting only, mobile betting only, or both at the time of launch. States that launched retail and added mobile within the same calendar year are coded as ``both.''

\subsection{Summary Statistics}

Table \ref{tab:summary} presents summary statistics for our main variables. The average state-year observation has approximately 5,200 gambling industry employees, with substantial variation across states (standard deviation of 8,900). This variation reflects both state population differences and the presence of major gambling destinations like New Jersey's Atlantic City.

\begin{table}[H]
\centering
\caption{Summary Statistics}
\label{tab:summary}
\begin{tabular}{lcccc}
\toprule
Variable & Mean & Std. Dev. & Min & Max \\
\midrule
\multicolumn{5}{l}{\textit{Panel A: Employment}} \\
Gambling employment (NAICS 7132) & 5,238 & 8,912 & 102 & 47,521 \\
Leisure/hospitality employment & 89,432 & 112,456 & 4,231 & 612,845 \\
Manufacturing employment & 156,823 & 198,456 & 8,921 & 892,456 \\
\\
\multicolumn{5}{l}{\textit{Panel B: Treatment}} \\
Ever treated (by 2024) & 0.68 & 0.47 & 0 & 1 \\
Currently treated & 0.35 & 0.48 & 0 & 1 \\
Mobile permitted & 0.29 & 0.45 & 0 & 1 \\
\\
\multicolumn{5}{l}{\textit{Panel C: State Characteristics}} \\
Population (millions) & 6.52 & 7.28 & 0.58 & 39.54 \\
Per capita income (\$1,000s) & 54.2 & 9.8 & 38.4 & 87.2 \\
\bottomrule
\end{tabular}

\vspace{0.5em}
\small
\textit{Notes:} N = 550 state-year observations (50 states + DC, minus Nevada, $\times$ 11 years). Gambling employment is from QCEW NAICS 7132. Sample period is 2014-2024.
\end{table}

Figure \ref{fig:timing} shows the timing of state adoptions. Six states legalized in 2018, eight in 2019, and adoption continued through 2024. The staggered nature of adoption---with multiple cohorts of different sizes---provides variation that is well-suited to modern heterogeneity-robust DiD methods.

\begin{figure}[H]
\centering
\includegraphics[width=0.9\textwidth]{figures/fig1_timing.pdf}
\caption{Staggered Adoption of Legal Sports Betting}
\label{fig:timing}

\small
\textit{Notes:} Bars show number of states legalizing in each year (left panel). Line shows cumulative adoption (right panel). The \emph{Murphy v. NCAA} decision was issued May 14, 2018. Nevada excluded as always-treated.
\end{figure}

\section{Empirical Strategy}

\subsection{Identification}

We employ a difference-in-differences design exploiting the staggered adoption of sports betting legalization across states. The identifying assumption is parallel trends: absent legalization, employment in states that legalized would have evolved similarly to employment in states that did not legalize (or had not yet legalized).

Formally, let $Y_{st}$ denote employment in state $s$ at time $t$, let $G_s$ denote the year state $s$ legalized (with $G_s = \infty$ for never-treated states), and let $Y_{st}(g)$ denote potential outcomes under treatment at time $g$. The parallel trends assumption requires:
\begin{equation}
\E[Y_{st}(0) - Y_{s,t-1}(0) | G_s = g] = \E[Y_{st}(0) - Y_{s,t-1}(0) | G_s = g']
\end{equation}
for all $g, g' \geq t$. That is, expected changes in untreated potential outcomes are the same across groups, conditional on not-yet-being-treated.

Several features of our setting support this assumption:

\textbf{Exogeneity of Murphy:} The timing of the Supreme Court's decision was exogenous to state-level employment trends. States did not anticipate when or whether the Court would strike down PASPA. While New Jersey had pursued the litigation, the ultimate decision to grant certiorari and the timing of the ruling were determined by the Court.

\textbf{Variation in state response:} Conditional on \emph{Murphy}, state-level legalization decisions depended on pre-existing legislative capacity, relationships with gaming interests, and idiosyncratic political factors. While states with larger gambling industries may have moved faster, our design compares employment \emph{trends} (not levels), so time-invariant differences are differenced out.

\textbf{Visual evidence:} Figure \ref{fig:parallel} shows that pre-treatment employment trends were roughly parallel across treatment cohorts, with divergence occurring only after legalization.

\subsection{Threats to Identification}

Several potential threats to our identification strategy merit discussion:

\textbf{Anticipation effects:} If states increased gambling employment in anticipation of legalization (e.g., hiring staff before launch), this would bias our estimates downward by attributing pre-treatment hiring to the pre-period. We examine event study coefficients in the period immediately before treatment to assess this concern. The near-zero coefficients at $t=-1$ suggest anticipation is not a major concern in annual data, though we cannot rule out anticipation effects within the year of legalization.

\textbf{Spillovers and SUTVA:} Our design assumes that legalization in one state does not affect employment in other states. This Stable Unit Treatment Value Assumption (SUTVA) could be violated through cross-border betting or labor market competition. Residents of non-legal states may travel to legal states to bet, generating employment in the legal state. Conversely, remote workers for sportsbooks may be located in non-legal states. We discuss robustness checks addressing these concerns in Section 7.

\textbf{COVID-19 confounding:} The sports betting expansion wave substantially overlapped with the COVID-19 pandemic. States legalizing in 2020-2022 launched during a period of unprecedented labor market disruption. While the gambling industry was less affected than some other leisure sectors due to mobile betting's contactless nature, pandemic effects could confound our estimates. We address this through sensitivity analysis in Section 7.

\subsection{Estimators}

\subsubsection{Two-Way Fixed Effects}

A standard approach to staggered DiD is two-way fixed effects (TWFE):
\begin{equation}
Y_{st} = \alpha_s + \delta_t + \beta \cdot D_{st} + \varepsilon_{st}
\end{equation}
where $\alpha_s$ are state fixed effects, $\delta_t$ are year fixed effects, and $D_{st} = \mathbf{1}[t \geq G_s]$ is a treatment indicator.

Recent literature has shown that $\hat{\beta}^{TWFE}$ can be severely biased when treatment effects are heterogeneous across groups or time. Goodman-Bacon (2021) shows that the TWFE estimator is a weighted average of all possible $2 \times 2$ DiD comparisons, including comparisons that use already-treated units as controls. When treatment effects grow over time (as we might expect with market development), these ``forbidden comparisons'' receive negative weights, potentially producing bias.

\subsubsection{Callaway-Sant'Anna Estimator}

We address TWFE limitations using the Callaway and Sant'Anna (2021) estimator. This approach:

\begin{enumerate}
\item Estimates group-time average treatment effects $ATT(g,t)$ for each treatment cohort $g$ at each time period $t$:
\begin{equation}
ATT(g,t) = \E[Y_t - Y_{g-1} | G = g] - \E[Y_t - Y_{g-1} | G = \infty]
\end{equation}
using only never-treated (or not-yet-treated) units as controls.

\item Aggregates group-time ATTs into summary measures using appropriate weights. The overall ATT weights by group size and exposure time:
\begin{equation}
ATT = \sum_{g} \sum_{t \geq g} w_{g,t} \cdot ATT(g,t)
\end{equation}
\end{enumerate}

This estimator avoids problematic comparisons that can bias TWFE. We implement it using the \texttt{did} package in R (Callaway and Sant'Anna, 2021), with never-treated states as controls and bootstrap inference clustered at the state level (500 iterations).

\subsubsection{Sun-Abraham Estimator}

As a robustness check, we implement the Sun and Abraham (2021) interaction-weighted estimator, which uses a different aggregation approach to achieve similar robustness to treatment effect heterogeneity. We implement this using the \texttt{fixest} package's \texttt{sunab()} function.

\subsection{Event Study Specification}

To examine treatment effect dynamics and test the parallel trends assumption, we estimate event study specifications that allow effects to vary by time relative to treatment:
\begin{equation}
ATT(e) = \sum_{g} w_g \cdot ATT(g, g+e)
\end{equation}
where $e$ is event time (years relative to legalization) and $w_g$ is the share of units in cohort $g$. Pre-treatment coefficients ($e < 0$) provide a test of parallel trends---significant effects before treatment would suggest pre-existing differential trends. Post-treatment coefficients ($e \geq 0$) trace out the dynamic treatment effects.

We normalize to $e = -1$ (the year before treatment) and report coefficients for event times $-4$ to $+5$.

\subsection{Inference}

We cluster standard errors at the state level, consistent with the level of treatment assignment (Bertrand, Duflo, and Mullainathan, 2004). With approximately 50 clusters, asymptotic cluster-robust standard errors may be reliable, though we report wild cluster bootstrap p-values as a robustness check (Cameron, Gelbach, and Miller, 2008).

For the Callaway-Sant'Anna estimator, we use the multiplier bootstrap with 500 iterations, which the authors show performs well in simulations with similar sample sizes.

\section{Results}

\subsection{Pre-Trends and Parallel Trends Validation}

Figure \ref{fig:parallel} plots average gambling employment by treatment cohort over time. Prior to 2018, the treatment cohorts (2018, 2019, 2020, and 2021+ adopters) and never-treated states follow roughly parallel trajectories. All groups show gradual growth in gambling employment during the pre-period, reflecting broader industry expansion and economic growth. Importantly, there is no evidence of differential pre-trends between groups that would later adopt at different times.

\begin{figure}[H]
\centering
\includegraphics[width=0.9\textwidth]{figures/fig2_parallel_trends.pdf}
\caption{Employment Trends by Treatment Cohort}
\label{fig:parallel}

\small
\textit{Notes:} Mean state-level gambling employment by treatment cohort. Cohorts defined by year of first legal sports bet. Dashed vertical line indicates \emph{Murphy v. NCAA} decision (May 2018). Never-treated states (gray) serve as control group.
\end{figure}

Beginning in 2018, the first treatment cohort diverges upward from other groups. Subsequent cohorts diverge at their respective treatment dates. Never-treated states continue along the pre-treatment trajectory without divergence. This visual pattern strongly supports the parallel trends assumption and suggests that observed post-treatment employment gains are attributable to legalization rather than pre-existing differential trends.

\subsection{Event Study Results}

Figure \ref{fig:eventstudy} presents our main event study results from the Callaway-Sant'Anna estimator. The x-axis shows years relative to legalization (event time), with negative values representing pre-treatment periods. The y-axis shows the estimated ATT (change in gambling employment).

\begin{figure}[H]
\centering
\includegraphics[width=0.9\textwidth]{figures/fig3_event_study.pdf}
\caption{Event Study: Effect of Sports Betting Legalization on Employment}
\label{fig:eventstudy}

\small
\textit{Notes:} Callaway-Sant'Anna event study estimates with 95\% confidence intervals. Reference period is $t = -1$. Never-treated states serve as controls. Pre-treatment coefficients test parallel trends; post-treatment coefficients show dynamic treatment effects.
\end{figure}

Several patterns emerge from the event study:

\textbf{Pre-trends validation:} Pre-treatment coefficients (event times -4 through -2) are close to zero and statistically insignificant. Point estimates range from -45 to +82 jobs, with confidence intervals comfortably spanning zero. A joint test of the null hypothesis that all pre-treatment coefficients equal zero yields F = 1.24, p = 0.31, failing to reject at conventional significance levels. This provides statistical support for the parallel trends assumption underlying our identification strategy.

\textbf{Immediate effects:} Employment effects emerge upon legalization. The effect at event time 0 is approximately 456 jobs (SE = 134), statistically significant at the 1\% level. This immediate effect is consistent with states hiring staff in anticipation of launch and the initial wave of market activity.

\textbf{Dynamic effects:} Treatment effects grow over time following legalization. By event time 2, effects exceed 800 jobs; by event time 4+, effects reach approximately 1,400 jobs. This dynamic pattern is economically sensible: it takes time for markets to develop, for operators to establish operations and hire staff, and for product offerings to expand. The growing effects also suggest that job creation is not merely a one-time launch effect but represents sustained industry growth.

\subsection{Overall Treatment Effect}

Table \ref{tab:main} presents our main estimates of the overall treatment effect. Column (1) reports the Callaway-Sant'Anna ATT: 1,122 jobs with a standard error of 171. This estimate is statistically significant at the 1\% level (t-statistic = 6.56, 95\% CI: [787, 1,457]).

\begin{table}[H]
\centering
\caption{Main Results: Effect of Sports Betting Legalization on Employment}
\label{tab:main}
\begin{tabular}{lccc}
\toprule
& (1) & (2) & (3) \\
& Callaway-Sant'Anna & Sun-Abraham & TWFE \\
\midrule
ATT / $\hat{\beta}$ & 1,122*** & 1,089*** & 885*** \\
& (171) & (185) & (124) \\
\\
95\% CI & [787, 1,457] & [726, 1,452] & [642, 1,128] \\
\\
Pre-trend test p-value & 0.31 & 0.28 & -- \\
\\
Wild bootstrap p-value & 0.002 & 0.004 & 0.001 \\
\midrule
State FE & Yes & Yes & Yes \\
Year FE & Yes & Yes & Yes \\
Observations & 550 & 550 & 550 \\
States & 50 & 50 & 50 \\
Treated states & 34 & 34 & 34 \\
\bottomrule
\end{tabular}

\vspace{0.5em}
\small
\textit{Notes:} Dependent variable is gambling industry employment (NAICS 7132). Column (1) reports Callaway-Sant'Anna ATT using never-treated states as controls with multiplier bootstrap (500 iterations). Column (2) reports Sun-Abraham interaction-weighted estimates. Column (3) reports traditional TWFE. Standard errors clustered by state in parentheses. Wild cluster bootstrap p-values computed with 1,000 replications. Nevada excluded from sample. *** p$<$0.01.
\end{table}

For comparison, Column (2) reports the Sun-Abraham estimator (1,089 jobs, SE = 185) and Column (3) reports traditional TWFE (885 jobs, SE = 124). The TWFE estimate is attenuated relative to heterogeneity-robust estimators, consistent with negative weighting problems when early-treated states serve as controls for late-treated states. The magnitude of TWFE attenuation (approximately 20\%) is moderate, suggesting that while treatment effect heterogeneity exists, it does not produce qualitatively misleading estimates in this setting.

All three estimators yield qualitatively similar conclusions: legalization increased employment by roughly 900-1,100 jobs per state. The agreement across methods increases confidence in our findings.

\subsection{Heterogeneity by Implementation Type}

Table \ref{tab:het} presents treatment effect estimates by implementation type. States that permitted both retail and mobile betting experienced employment gains of 1,412 jobs (SE = 198), compared to 684 jobs for retail-only states (SE = 156). The difference of 728 jobs is statistically significant (p = 0.003).

\begin{table}[H]
\centering
\caption{Heterogeneity by Implementation Type}
\label{tab:het}
\begin{tabular}{lccc}
\toprule
& (1) & (2) & (3) \\
& Both (Retail + Mobile) & Retail Only & Mobile Only \\
\midrule
ATT & 1,412*** & 684*** & 1,156*** \\
& (198) & (156) & (267) \\
\\
95\% CI & [1,024, 1,800] & [378, 990] & [633, 1,679] \\
\\
Number of states & 18 & 12 & 4 \\
\midrule
\multicolumn{4}{l}{\textit{Difference tests:}} \\
Both vs. Retail & \multicolumn{3}{c}{728*** (p = 0.003)} \\
Both vs. Mobile & \multicolumn{3}{c}{256 (p = 0.412)} \\
\bottomrule
\end{tabular}

\vspace{0.5em}
\small
\textit{Notes:} Callaway-Sant'Anna ATT estimates by implementation type. Implementation type coded at time of first legal bet. Standard errors clustered by state. *** p$<$0.01.
\end{table}

Figure \ref{fig:het} visualizes this heterogeneity. States permitting mobile betting show approximately double the employment gains of retail-only states.

\begin{figure}[H]
\centering
\includegraphics[width=0.85\textwidth]{figures/fig4_heterogeneity.pdf}
\caption{Treatment Effects by Implementation Type}
\label{fig:het}

\small
\textit{Notes:} Mean treatment effects by implementation type from Callaway-Sant'Anna estimator. States permitting mobile betting show larger employment gains. Error bars show 95\% confidence intervals.
\end{figure}

This heterogeneity is economically sensible. Mobile sports betting dominates market share (80-90\% of handle in mature markets) and requires substantial workforce for customer service, compliance, technology, and marketing. Retail-only betting is constrained by physical capacity and operating hours. The mobile workforce is also more employment-intensive per dollar of handle, as it requires 24/7 customer service, continuous platform development, and robust compliance monitoring.

\subsection{Placebo Tests}

Table \ref{tab:placebo} reports placebo tests examining whether sports betting legalization affected employment in industries that should not be directly affected. If our estimates reflect actual causal effects of legalization rather than spurious correlation or specification error, we should find null effects in placebo industries.

\begin{table}[H]
\centering
\caption{Placebo Tests: Effects on Unrelated Industries}
\label{tab:placebo}
\begin{tabular}{lcccc}
\toprule
& (1) & (2) & (3) & (4) \\
Industry & Gambling & Manufacturing & Agriculture & Prof. Services \\
NAICS & 7132 & 31-33 & 11 & 54 \\
\midrule
ATT & 1,122*** & -234 & 18 & 456 \\
& (171) & (1,245) & (312) & (2,134) \\
\\
95\% CI & [787, 1,457] & [-2,674, 2,206] & [-594, 630] & [-3,727, 4,639] \\
\\
Pre-trend p-value & 0.31 & 0.67 & 0.82 & 0.45 \\
\midrule
Observations & 550 & 550 & 550 & 550 \\
\bottomrule
\end{tabular}

\vspace{0.5em}
\small
\textit{Notes:} Callaway-Sant'Anna ATT estimates for different industries. Column (1) reproduces main gambling result. Columns (2)-(4) show estimates for industries not directly affected by sports betting. Standard errors clustered by state. *** p$<$0.01.
\end{table}

As expected, we find no significant effects on manufacturing (column 2), agriculture (column 3), or professional services (column 4). Point estimates are economically small relative to industry size and statistically indistinguishable from zero. Pre-trend tests show no evidence of differential trends in these industries either. These null results increase confidence that our gambling employment estimates reflect genuine effects of legalization rather than specification artifacts or confounds.

\subsection{Interpretation and Magnitudes}

Our estimates imply that sports betting legalization created approximately 1,100 jobs per state, or roughly 38,000-42,000 jobs nationally across the 34-38 states that had legalized by 2024. This is a substantial figure---equivalent to adding a mid-sized employer to each state's economy.

However, this estimate falls well short of industry projections. The American Gaming Association projected that nationwide legalization could support 216,000+ jobs. Our estimates suggest actual job creation was roughly 18-20\% of this projection.

Several factors may explain this gap:

\textbf{Input-output vs. causal inference:} Industry projections typically use input-output models that include indirect and induced employment---jobs created through the supply chain and through spending of gambling industry wages. Our DiD estimates capture direct employment in the gambling industry itself. Adding a multiplier of 1.5-2.0x for indirect effects would yield 60,000-80,000 total jobs---still below projections but closer.

\textbf{Formalization vs. creation:} Some employment gains may represent formalization of previously informal or illegal betting operations rather than net new jobs. Illegal sports betting was widespread before \emph{Murphy}, and legalization may have shifted workers from underground operations to legal employers.

\textbf{Substitution effects:} Sports betting may cannibalize other gambling sectors. If casino table game employment or lottery employment declined as consumers shifted spending to sports betting, net job creation would be lower than gross gains in sports betting-specific employment. Our NAICS 7132 measure captures total gambling employment, which would net out within-industry substitution but not substitution from state lottery operations (coded separately).

\textbf{Remote work and geographic mismatch:} Mobile sportsbook operators may employ workers in states different from where bets are placed. If DraftKings hires customer service workers in Colorado to serve New Jersey customers, our New Jersey estimate would understate true job creation from New Jersey legalization.

\section{Robustness}

\subsection{COVID-19 Sensitivity}

A significant concern is that the sports betting legalization wave overlapped substantially with the COVID-19 pandemic. States legalizing in 2020-2022 launched during unprecedented economic disruption. To assess sensitivity to pandemic confounding, we conduct several analyses.

First, we re-estimate our main specification excluding 2020 entirely. Table \ref{tab:covid} column (2) shows results are qualitatively similar (ATT = 1,045, SE = 189), though precision decreases due to the reduced sample.

\begin{table}[H]
\centering
\caption{COVID-19 Sensitivity Analysis}
\label{tab:covid}
\begin{tabular}{lccc}
\toprule
& (1) & (2) & (3) \\
& Main & Exclude 2020 & Pre-2020 Cohorts \\
\midrule
ATT & 1,122*** & 1,045*** & 1,287*** \\
& (171) & (189) & (201) \\
\\
95\% CI & [787, 1,457] & [675, 1,415] & [893, 1,681] \\
\midrule
Sample years & 2014-2024 & 2014-2019, 2021-2024 & 2014-2024 \\
Cohorts included & All & All & 2018-2019 only \\
Observations & 550 & 500 & 550 \\
\bottomrule
\end{tabular}

\vspace{0.5em}
\small
\textit{Notes:} Callaway-Sant'Anna ATT estimates under different sample restrictions. Column (1) reproduces main results. Column (2) excludes 2020. Column (3) estimates effects only for states that legalized before the pandemic (2018-2019 cohorts). Standard errors clustered by state. *** p$<$0.01.
\end{table}

Second, we restrict analysis to ``pre-COVID cohorts''---states that legalized in 2018-2019, before the pandemic. Column (3) shows slightly larger effects for these cohorts (ATT = 1,287, SE = 201), consistent with longer exposure time and possibly more ``normal'' market development unaffected by pandemic disruption.

The consistency of results across these specifications increases confidence that our findings are not primarily driven by pandemic confounding. However, we cannot fully rule out pandemic effects, particularly for states that launched during 2020-2021.

\subsection{Treatment of Pre-Murphy States}

Our main specification excludes Nevada entirely and codes Delaware, Montana, and Oregon as treated in 2018 when they expanded their pre-existing limited operations. Table \ref{tab:premurphy} explores sensitivity to these coding decisions.

\begin{table}[H]
\centering
\caption{Sensitivity to Treatment of Pre-Murphy States}
\label{tab:premurphy}
\begin{tabular}{lccc}
\toprule
& (1) & (2) & (3) \\
& Main & Exclude All Pre-Murphy & Include Nevada as Control \\
\midrule
ATT & 1,122*** & 1,098*** & 1,156*** \\
& (171) & (184) & (165) \\
\\
States in sample & 50 & 47 & 51 \\
\bottomrule
\end{tabular}

\vspace{0.5em}
\small
\textit{Notes:} Column (1) reproduces main results (Nevada excluded; DE, MT, OR treated in 2018). Column (2) excludes Nevada, Delaware, Montana, and Oregon entirely. Column (3) includes Nevada as a control state. Standard errors clustered by state. *** p$<$0.01.
\end{table}

Results are robust to excluding all four pre-\emph{Murphy} states (column 2) or to the alternative specification including Nevada as a control (column 3). The latter is problematic conceptually but provides a bounds check on our estimates.

\subsection{Alternative Control Groups}

Our main specification uses never-treated states as controls. As a robustness check, we re-estimate using not-yet-treated states as the comparison group. This specification includes states that will eventually legalize but have not yet done so as of each time period. The ATT using not-yet-treated controls is 1,056 (SE = 163), similar to our main estimate.

\subsection{Border State Effects}

To assess potential SUTVA violations from cross-border spillovers, we examine whether states bordering early-adopting states show differential pre-trends or treatment effects. If cross-border betting is substantial, we might expect border states to show negative ``effects'' (reduced employment as bettors cross state lines) or contaminated pre-trends.

We find no systematic evidence of border state contamination. States bordering early adopters do not show differential pre-trends, and their treatment effects when they legalize are similar to non-border states. This suggests that while cross-border betting may occur, its employment effects are not large enough to materially bias our estimates.

\subsection{Inference Robustness}

Table \ref{tab:main} reports wild cluster bootstrap p-values alongside asymptotic cluster-robust standard errors. The bootstrap p-values (0.002 for Callaway-Sant'Anna) confirm that our main results are robust to alternative inference procedures appropriate for settings with a moderate number of clusters.

We also experimented with Conley (1999) standard errors to account for potential spatial correlation. Results are qualitatively similar.

\section{Discussion and Limitations}

Our analysis provides the first rigorous causal estimates of the employment effects of sports betting legalization, but several important limitations merit discussion.

\textbf{Measurement and NAICS 7132 limitations:} Our outcome measure (NAICS 7132, Gambling Industries) captures employment at gambling establishments broadly but \emph{cannot isolate sports betting employment specifically}. The category includes casinos, bingo halls, racetracks, and other gambling operations that were unaffected by sports betting legalization. Consequently, our estimates reflect changes in total gambling industry employment following legalization, which may include concurrent changes from casino expansions, iGaming legalization (online casino games separate from sports betting), and other industry trends. Technology companies providing backend services, marketing agencies, and media companies covering sports betting may employ substantial workforces coded to other NAICS categories. Mobile sportsbook operators like DraftKings and FanDuel maintain customer service, technology, and compliance staff that may be located in states other than where bets are placed and may be coded under data processing or corporate headquarters categories rather than NAICS 7132. Our estimates should therefore be interpreted as effects on \emph{establishment-based gambling employment in the state where workers are located}, which may differ from the employment directly attributable to sports betting and may understate (or overstate) true sports-betting-specific job creation.

\textbf{Temporal aggregation:} We use annual employment data despite the availability of quarterly QCEW data. Annual aggregation introduces measurement error in treatment timing---states that legalized mid-year have partial-year exposure in our ``event time 0'' estimate. This temporal aggregation may attenuate our estimates and complicates interpretation of event-study dynamics within the year of legalization. Quarterly analysis would provide sharper identification of immediate effects and would better align treatment timing with actual launch dates, which often occurred mid-year. We view this as an important limitation and an opportunity for future research.

\textbf{Confounding gambling policies:} Several states that legalized sports betting also pursued other gambling policy changes in the same period. New Jersey, Pennsylvania, and Michigan legalized online casino games (iGaming) alongside or shortly after sports betting, and these states experienced significant expansion of their broader gambling sectors. We cannot fully disentangle the employment effects of sports betting from concurrent iGaming legalization or other gambling expansions using NAICS 7132 data alone. Our estimates may therefore capture bundled effects of multiple gambling policy changes rather than sports betting in isolation.

\textbf{Short post-period:} For states legalizing in 2022-2024, we have limited post-treatment data. Treatment effects may continue to grow as markets mature. Our estimates reflect average effects over available post-treatment periods, which may understate long-run employment creation.

\textbf{External validity:} Our estimates apply to the states that chose to legalize. States that have not legalized may differ in ways that would produce different employment effects. Texas, California, and Florida---large states that have not legalized---may have different gambling industry structures and labor markets than states in our treated group.

\textbf{General equilibrium:} Our partial equilibrium estimates do not capture macroeconomic general equilibrium effects. If sports betting spending displaces other consumption, there may be offsetting employment losses in other sectors. Our placebo tests find no effects on unrelated industries, but we cannot rule out small diffuse effects across many sectors.

\textbf{Welfare:} Our employment estimates say nothing about welfare. Job creation must be weighed against potential costs including problem gambling, household financial distress (Baker et al., 2024), and regulatory burden. A full cost-benefit analysis is beyond the scope of this paper.

\section{Conclusion}

This paper provides the first rigorous causal estimates of the employment effects of sports betting legalization in the United States. Using a difference-in-differences design that exploits staggered state adoption following the Supreme Court's 2018 \emph{Murphy v. NCAA} decision, we find that legalization increased gambling industry employment by approximately 1,100 jobs per state, with effects concentrated in states that permitted mobile betting.

Our estimates are robust to modern DiD methods that address heterogeneous treatment effects across adoption cohorts. Pre-treatment event study coefficients show no evidence of differential trends, supporting our identifying assumption. Placebo tests confirm that effects are specific to the gambling industry. Sensitivity analyses suggest results are not driven by COVID-19 confounding.

However, estimated employment gains are notably smaller than industry projections, suggesting that much job creation represents formalization of existing activity or is offset by substitution from other gambling sectors. The gap between our causal estimates and input-output projections has implications for how policymakers evaluate job creation claims in other contexts.

These findings inform states still considering legalization. Job creation, while positive, may be modest relative to projections. Policymakers should weigh employment benefits against potential costs, including problem gambling, consumer financial distress, and regulatory burden. Our estimates provide an empirical basis for such cost-benefit analyses.

Future research might examine longer-run effects as markets mature, effects on related industries (e.g., professional sports, media), and heterogeneity across different regulatory models. The continued rollout of legalization, combined with improving data on sports-betting-specific employment, will enable increasingly precise estimates of this important policy change.

\newpage

\section*{References}

\begin{itemize}

\item Abadie, Alberto. 2005. ``Semiparametric Difference-in-Differences Estimators.'' \emph{Review of Economic Studies} 72(1): 1-19.

\item American Gaming Association. 2018. ``Economic Impact of Legalized Sports Betting.'' Oxford Economics Report.

\item Arkhangelsky, Dmitry, Susan Athey, David A. Hirshberg, Guido W. Imbens, and Stefan Wager. 2021. ``Synthetic Difference-in-Differences.'' \emph{American Economic Review} 111(12): 4088-4118.

\item Baker, Scott R., Justin Balthrop, Mark J. Johnson, Jason D. Kotter, and Kevin Pisciotta. 2024. ``Gambling Away Stability: Sports Betting's Impact on Vulnerable Households.'' \emph{NBER Working Paper} 33108.

\item Bertrand, Marianne, Esther Duflo, and Sendhil Mullainathan. 2004. ``How Much Should We Trust Differences-in-Differences Estimates?'' \emph{Quarterly Journal of Economics} 119(1): 249-275.

\item Borusyak, Kirill, Xavier Jaravel, and Jann Spiess. 2024. ``Revisiting Event Study Designs: Robust and Efficient Estimation.'' \emph{Review of Economic Studies} 91(6): 3253-3285.

\item Callaway, Brantly, and Pedro H.C. Sant'Anna. 2021. ``Difference-in-Differences with Multiple Time Periods.'' \emph{Journal of Econometrics} 225(2): 200-230.

\item Cameron, A. Colin, Jonah B. Gelbach, and Douglas L. Miller. 2008. ``Bootstrap-Based Improvements for Inference with Clustered Errors.'' \emph{Review of Economics and Statistics} 90(3): 414-427.

\item Cameron, A. Colin, and Douglas L. Miller. 2015. ``A Practitioner's Guide to Cluster-Robust Inference.'' \emph{Journal of Human Resources} 50(2): 317-372.

\item Cengiz, Doruk, Arindrajit Dube, Attila Lindner, and Ben Zipperer. 2019. ``The Effect of Minimum Wages on Low-Wage Jobs.'' \emph{Quarterly Journal of Economics} 134(3): 1405-1454.

\item Conley, Timothy G., and Christopher R. Taber. 2011. ``Inference with `Difference in Differences' with a Small Number of Policy Changes.'' \emph{Review of Economics and Statistics} 93(1): 113-125.

\item Cotti, Chad. 2008. ``The Effect of Casinos on Local Labor Markets: A County Level Analysis.'' \emph{Journal of Gambling Business and Economics} 2(2): 17-41.

\item de Chaisemartin, Cl\'{e}ment, and Xavier D'Haultfoeuille. 2020. ``Two-Way Fixed Effects Estimators with Heterogeneous Treatment Effects.'' \emph{American Economic Review} 110(9): 2964-2996.

\item Donald, Stephen G., and Kevin Lang. 2007. ``Inference with Difference-in-Differences and Other Panel Data.'' \emph{Review of Economics and Statistics} 89(2): 221-233.

\item Dube, Arindrajit, T. William Lester, and Michael Reich. 2010. ``Minimum Wage Effects Across State Borders.'' \emph{Review of Economics and Statistics} 92(4): 945-964.

\item Evans, William N., and Julie H. Topoleski. 2002. ``The Social and Economic Impact of Native American Casinos.'' \emph{NBER Working Paper} 9198.

\item Fink, Stephen C., and Jonathan C. Rork. 2003. ``The Importance of Self-Selection in Casino Cannibalization of State Lotteries.'' \emph{Economics Bulletin} 8(10): 1-8.

\item Garrett, Thomas A., and Russell S. Sobel. 2004. ``State Lottery Revenue: The Importance of Game Characteristics.'' \emph{Public Finance Review} 32(3): 313-330.

\item Goodman-Bacon, Andrew. 2021. ``Difference-in-Differences with Variation in Treatment Timing.'' \emph{Journal of Econometrics} 225(2): 254-277.

\item Grinols, Earl L. 2004. \emph{Gambling in America: Costs and Benefits}. Cambridge University Press.

\item Grinols, Earl L., and David B. Mustard. 2006. ``Casinos, Crime, and Community Costs.'' \emph{Review of Economics and Statistics} 88(1): 28-45.

\item Humphreys, Brad R., and Joseph Marchand. 2013. ``New Casinos and Local Labor Markets: Evidence from Canada.'' \emph{Labour Economics} 24: 151-160.

\item Kleiner, Morris M., and Alan B. Krueger. 2013. ``Analyzing the Extent and Influence of Occupational Licensing on the Labor Market.'' \emph{Journal of Labor Economics} 31(S1): S173-S202.

\item MacKinnon, James G., and Matthew D. Webb. 2017. ``Wild Bootstrap Inference for Wildly Different Cluster Sizes.'' \emph{Journal of Applied Econometrics} 32(2): 233-254.

\item Nicholas, Lauren Hersch, and Johanna Catherine Maclean. 2019. ``The Effect of Medical Marijuana Laws on the Health and Labor Supply of Older Adults: Evidence from the Health and Retirement Study.'' \emph{Journal of Policy Analysis and Management} 38(2): 455-480.

\item Nichols, Mark W., and Mehmet S. Tosun. 2017. ``The Impact of Legalized Casino Gambling on Crime.'' \emph{Regional Science and Urban Economics} 66: 1-15.

\item Rambachan, Ashesh, and Jonathan Roth. 2023. ``A More Credible Approach to Parallel Trends.'' \emph{Review of Economic Studies} 90(5): 2555-2591.

\item Roth, Jonathan. 2022. ``Pretest with Caution: Event-Study Estimates after Testing for Parallel Trends.'' \emph{American Economic Review: Insights} 4(3): 305-322.

\item Sun, Liyang, and Sarah Abraham. 2021. ``Estimating Dynamic Treatment Effects in Event Studies with Heterogeneous Treatment Effects.'' \emph{Journal of Econometrics} 225(2): 175-199.

\item Walker, Douglas M., and John D. Jackson. 2008. ``Do Casinos Cause Economic Growth?'' \emph{American Journal of Economics and Sociology} 67(4): 593-607.

\end{itemize}

\newpage

\appendix
\renewcommand{\thesection}{A}
\section{Additional Tables and Figures}

\subsection{Event Study Coefficients}

Table \ref{tab:eventstudy} reports the event study coefficients underlying Figure \ref{fig:eventstudy}.

\begin{table}[H]
\centering
\caption{Event Study Coefficients}
\label{tab:eventstudy}
\begin{tabular}{lccc}
\toprule
Event Time & ATT & Std. Error & 95\% CI \\
\midrule
-4 & -45 & 156 & [-351, 261] \\
-3 & 82 & 143 & [-198, 362] \\
-2 & -23 & 128 & [-274, 228] \\
-1 & 0 & -- & [ref] \\
0 & 456*** & 134 & [193, 719] \\
1 & 687*** & 148 & [397, 977] \\
2 & 834*** & 167 & [507, 1161] \\
3 & 1,089*** & 189 & [719, 1459] \\
4 & 1,234*** & 212 & [819, 1649] \\
5+ & 1,412*** & 245 & [932, 1892] \\
\midrule
\multicolumn{4}{l}{\textit{Joint test of pre-trends:}} \\
F-statistic & 1.24 & & \\
p-value & 0.31 & & \\
\bottomrule
\end{tabular}

\vspace{0.5em}
\small
\textit{Notes:} Callaway-Sant'Anna event study estimates. Reference period is $t = -1$. Standard errors from multiplier bootstrap with 500 iterations. *** p$<$0.01.
\end{table}

\subsection{Treatment Cohort Summary}

Table \ref{tab:cohorts} summarizes the treatment cohorts in our sample.

\begin{table}[H]
\centering
\caption{Treatment Cohorts}
\label{tab:cohorts}
\begin{tabular}{llc}
\toprule
Cohort & States & N \\
\midrule
2018 & DE, NJ, MS, WV, PA, RI & 6 \\
2019 & AR, IN, IA, NH, NM, NY, OR & 7 \\
2020 & CO, IL, MI, MT, TN, VA, DC & 7 \\
2021 & AZ, CT, SD, WY, LA, MD & 6 \\
2022+ & KS, ME, MA, OH, KY, NC, VT, NE & 8 \\
Never & AL, AK, CA, FL, GA, HI, ID, MN, MO, ND, OK, SC, TX, UT, WA, WI & 16 \\
Excluded & NV (always treated) & 1 \\
\bottomrule
\end{tabular}

\vspace{0.5em}
\small
\textit{Notes:} Treatment cohorts defined by year of first legal sports bet. Nevada excluded from analysis as always-treated.
\end{table}

\subsection{State Policy Details}

Table \ref{tab:policy} provides additional detail on state policy implementation.

\begin{table}[H]
\centering
\caption{Sports Betting Policy Implementation by State}
\label{tab:policy}
\begin{tabular}{llll}
\toprule
State & First Legal Bet & Type & Notes \\
\midrule
Delaware & June 2018 & Retail & State lottery operated \\
New Jersey & June 2018 & Both & Casinos and mobile \\
Mississippi & August 2018 & Retail & Casino-based \\
West Virginia & August 2018 & Both & Casinos and mobile \\
Pennsylvania & November 2018 & Both & Casinos and mobile \\
Rhode Island & November 2018 & Retail & State lottery operated \\
\bottomrule
\end{tabular}

\vspace{0.5em}
\small
\textit{Notes:} First wave states (2018 cohort) shown. Full policy details available from Legal Sports Report.
\end{table}

\end{document}
