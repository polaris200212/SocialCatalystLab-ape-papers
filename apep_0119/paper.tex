\documentclass[12pt]{article}

% UTF-8 encoding and fonts
\usepackage[utf8]{inputenc}
\usepackage[T1]{fontenc}
\usepackage{lmodern}

% Page setup
\usepackage[margin=1in]{geometry}
\usepackage{setspace}
\onehalfspacing

% Typography
\usepackage{microtype}

% Math and symbols
\usepackage{amsmath,amssymb}

% Graphics
\usepackage{graphicx}
\usepackage{float}
\usepackage{subcaption}

% Tables
\usepackage{booktabs}
\usepackage{array}
\usepackage{multirow}
\usepackage{threeparttable}
\usepackage{longtable}
\usepackage{pdflscape}
\usepackage{siunitx}
\sisetup{detect-all=true, group-separator={,}, group-minimum-digits=4}
\usepackage{tabularx}

% Bibliography
\usepackage{natbib}
\bibliographystyle{aer}

% Hyperlinks
\usepackage{hyperref}
\hypersetup{
    colorlinks=true,
    linkcolor=blue,
    citecolor=blue,
    urlcolor=blue
}
\usepackage[nameinlink,noabbrev]{cleveref}

% Captions
\usepackage{caption}
\captionsetup{font=small,labelfont=bf}

% Section formatting
\usepackage{titlesec}
\titleformat{\section}{\large\bfseries}{\thesection.}{0.5em}{}
\titleformat{\subsection}{\normalsize\bfseries}{\thesubsection}{0.5em}{}

% Custom commands
\newcommand{\E}{\mathbb{E}}
\newcommand{\Var}{\text{Var}}
\newcommand{\Cov}{\text{Cov}}
\newcommand{\ind}{\mathbb{I}}
\newcommand{\sym}[1]{\ifmmode^{#1}\else\(^{#1}\)\fi}
\newcommand{\attgt}{\text{ATT}(g,t)}

\title{Do Energy Efficiency Resource Standards Reduce Electricity Consumption?\\Evidence from Staggered State Adoption}
\author{APEP Autonomous Research\thanks{Autonomous Policy Evaluation Project. Correspondence: scl@econ.uzh.ch} \\ @ai1scl}
\date{\today}

\begin{document}

\maketitle

\begin{abstract}
\noindent
Do state-level mandates requiring utilities to achieve energy savings targets actually reduce electricity consumption? Energy Efficiency Resource Standards (EERS) are among the most widespread energy policies in the United States, with 28 jurisdictions (27 states plus DC) adopting mandatory standards between 1998 and 2020. Despite their prevalence, no rigorous causal evaluation using modern econometric methods exists. I exploit the staggered adoption of EERS across states using the Callaway and Sant'Anna (2021) heterogeneity-robust difference-in-differences estimator with never-treated states as controls. Using state-level data from the Energy Information Administration covering 1990--2023, I estimate that EERS adoption is associated with a 3.9 percent reduction in per-capita residential electricity consumption (SE = 0.025), though the 95\% confidence interval includes zero $[-0.087, 0.009]$. The event-study analysis reveals suggestive dynamics: pre-treatment coefficients are centered on zero, while post-treatment coefficients become progressively more negative, reaching 5--8 percent after 15 years. The point estimate is larger among early adopters ($-4.0\%$, $p < 0.05$) than late adopters ($-3.0\%$, not significant), consistent with cumulative program maturation. I find no statistically significant effect on residential electricity prices (4.5\%, SE = 0.065). These results provide suggestive but imprecisely estimated evidence that EERS mandates reduce residential electricity consumption, with the imprecision likely reflecting substantial heterogeneity across states and adoption cohorts.
\end{abstract}

\vspace{1em}
\noindent\textbf{JEL Codes:} Q48, Q41, H76, L94 \\
\noindent\textbf{Keywords:} energy efficiency, utility regulation, electricity consumption, difference-in-differences, staggered adoption

\newpage

%==============================================================================
\section{Introduction}
%==============================================================================

Energy efficiency is widely regarded as the ``first fuel'' of a sustainable energy transition---the cheapest and most accessible resource for reducing greenhouse gas emissions and lowering consumer costs. In the United States, state governments have taken the lead in mandating energy savings through Energy Efficiency Resource Standards (EERS), which require electric and gas utilities to achieve specified annual reductions in customer energy consumption. By 2020, 28 jurisdictions had adopted mandatory EERS, making it one of the most widespread state-level energy policies alongside Renewable Portfolio Standards (RPS).

Despite this prevalence, a fundamental question remains unresolved: do these mandates actually reduce electricity consumption? The theoretical prediction is ambiguous. On one hand, EERS programs fund a range of demand-side interventions---appliance rebates, weatherization subsidies, building energy audits, and industrial process improvements---that should directly reduce energy use. On the other hand, several mechanisms could attenuate or even reverse these savings. Free-ridership occurs when programs subsidize efficiency investments that consumers would have made voluntarily, inflating reported savings without generating additional conservation. The rebound effect predicts that efficiency gains lower the effective price of energy services, inducing additional consumption that partially offsets engineering savings. And states that adopt EERS may simultaneously pursue complementary policies (building codes, appliance standards) that confound attribution of savings to the EERS itself.

This paper provides the first rigorous causal estimate of EERS effectiveness using modern econometric methods designed for staggered policy adoption. I exploit the fact that 28 jurisdictions (27 states plus the District of Columbia) adopted mandatory EERS at different times between 1998 and 2020, while 23 states never adopted such mandates. Using a balanced panel of 51 jurisdictions (50 states plus the District of Columbia) observed annually from 1990 to 2023, I apply the Callaway and Sant'Anna (2021) heterogeneity-robust difference-in-differences estimator to estimate the average treatment effect on the treated (ATT). This approach avoids the well-documented biases of conventional two-way fixed effects (TWFE) estimators in staggered adoption settings \citep{goodmanbacon2021, roth2023}.

The main result is a point estimate suggesting that EERS adoption reduces per-capita residential electricity consumption by approximately 3.9 percent, though this estimate is not statistically significant at conventional levels (SE = 0.025, 95\% CI: $[-0.087, 0.009]$). The event-study analysis, however, reveals a suggestive dynamic pattern: pre-treatment coefficients are centered on zero from 10 years before adoption, while post-treatment coefficients become progressively more negative, reaching 5--8 percent reductions after 10--15 years. This dynamic pattern aligns with the institutional reality that EERS programs require time to ramp up---utilities must design programs, recruit participants, and build contractor networks before achieving mandated savings levels.

The direction of the effect is consistent across specifications, though precision varies. Using not-yet-treated states as an alternative comparison group yields a point estimate of $-3.1$ percent (SE = 0.017, $p < 0.10$), the most precisely estimated specification. The Sun and Abraham (2021) interaction-weighted estimator produces qualitatively similar event-study dynamics, with post-treatment coefficients of $-0.01$ to $-0.08$ log points. Conventional TWFE estimation yields a coefficient of $-0.024$ (SE = 0.018), also not statistically significant, though this estimate is potentially attenuated by ``bad comparisons'' between late-adopting and earlier-treated states \citep{goodmanbacon2021}.

I investigate two channels through which EERS could affect welfare beyond consumption reductions. First, I examine the price channel. If EERS programs reduce sales volumes without reducing utilities' fixed costs, regulators may allow rate increases to maintain utility revenue adequacy---the so-called ``utility death spiral'' concern. I find no statistically significant effect on residential electricity prices (point estimate: $+4.5\%$, SE = 0.065). Second, I examine heterogeneity by adoption timing. Early adopters (pre-2008) show a statistically significant effect of $-4.0\%$ (SE = 0.019, $p < 0.05$) while late adopters show a similar-magnitude but imprecisely estimated $-3.0\%$ (SE = 0.031), consistent with either program maturation over time or positive selection of more committed states into early adoption.

This paper contributes to several literatures. First, it provides the first application of modern heterogeneity-robust DiD methods to EERS policy evaluation. Previous studies have used simple panel regressions, descriptive comparisons, or engineering estimates that do not credibly address selection into treatment \citep{barbose2013, gillingham2018}. By applying the Callaway-Sant'Anna estimator with never-treated controls, I address the key identification threats in this setting: differential pre-treatment trends, treatment effect heterogeneity across cohorts, and contamination of TWFE estimates by ``bad comparisons'' of later-treated to earlier-treated states \citep{goodmanbacon2021, sun2021, dechaisemartin2020}.

Second, the paper contributes to the broader debate on the effectiveness of ``command and control'' energy regulation versus market-based approaches. The literature on Renewable Portfolio Standards---a closely related mandate requiring utilities to procure renewable generation---has found mixed effects on electricity prices and generation mix \citep{deschenes2023, greenstone2024}. EERS addresses the demand side rather than the supply side, targeting the efficiency of energy use rather than its source. The suggestive evidence that EERS mandates are associated with consumption reductions---particularly among early adopters, where the effect is statistically significant---implies that utility-administered efficiency programs, despite concerns about free-ridership and administrative costs, may be a useful component of energy and climate policy, though more precise estimation is needed.

Third, this paper speaks to the growing literature on the environmental effectiveness of state-level climate and energy policies \citep{auffhammer2014, borenstein2016}. In the absence of comprehensive federal carbon pricing, U.S. climate policy has relied on a patchwork of state-level regulations including EERS, RPS, cap-and-trade programs, and building codes. Understanding which policies actually reduce energy consumption is essential for designing effective climate policy portfolios. My finding of a 4 percent reduction in residential electricity consumption suggests that EERS is one of the more impactful state-level interventions, comparable in magnitude to estimates of RPS effects on renewable generation \citep{deschenes2023}.

The remainder of the paper proceeds as follows. Section 2 provides institutional background on EERS design and implementation. Section 3 describes the conceptual framework and expected mechanisms. Section 4 presents the data sources and sample construction. Section 5 details the empirical strategy. Section 6 presents the main results. Section 7 provides robustness checks and sensitivity analyses. Section 8 examines heterogeneity. Section 9 discusses implications and limitations. Section 10 concludes.

%==============================================================================
\section{Institutional Background}
%==============================================================================

\subsection{What Are Energy Efficiency Resource Standards?}

An Energy Efficiency Resource Standard (EERS) is a state-level regulatory mandate requiring electric utilities, gas utilities, or both to achieve specified reductions in customer energy consumption through demand-side management (DSM) programs. Unlike Renewable Portfolio Standards, which mandate minimum shares of renewable generation on the supply side, EERS operate on the demand side by requiring utilities to help customers use less energy.

EERS mandates typically specify annual savings targets as a percentage of retail electricity sales (e.g., 1.5\% per year) or as absolute energy savings in megawatt-hours or therms. Utilities comply by designing and administering portfolios of customer-facing programs, which may include residential appliance rebate programs, commercial building retrofits, industrial process optimization, low-income weatherization assistance, and behavioral programs such as home energy reports. States vary considerably in the stringency of their targets, with annual electricity savings requirements ranging from 0.4\% (Texas) to over 2.0\% (Massachusetts, Illinois).

EERS programs are funded through ratepayer surcharges---small per-kWh charges added to electricity bills that finance the utility's efficiency program portfolio. These surcharges typically range from 1--3 cents per kWh, adding \$5--15 per month to a typical residential bill. In exchange, participating customers receive subsidized efficiency upgrades that are expected to reduce their energy consumption by more than the surcharge cost, generating net savings. However, the surcharge is paid by all ratepayers regardless of participation, creating a cross-subsidy from non-participants to participants.

\subsection{Staggered Adoption Across States}

The first EERS was adopted by Connecticut in 1998, followed by Texas in 1999 and Vermont in 2000. Adoption accelerated in the mid-2000s, with a particularly large cohort of eight jurisdictions (DC, Maryland, Massachusetts, Michigan, New Mexico, New York, North Carolina, Pennsylvania) adopting EERS mandates in 2008---a year in which energy policy was prominent on state legislative agendas due to rising gasoline prices and growing climate concern. Additional waves of adoption occurred in 2010 (Arizona, Arkansas), 2016 (Oregon), 2018 (New Hampshire, New Jersey), and 2019--2020 (Iowa, Maine, Virginia).

By 2020, 28 jurisdictions---27 states plus the District of Columbia---had adopted mandatory EERS mandates. The 23 states that never adopted EERS---predominantly located in the Southeast and Mountain West---form the ``never-treated'' comparison group in my analysis. These states tend to have lower electricity prices, more fossil-fuel-dependent economies, and more conservative political environments, characteristics I account for through state fixed effects in the identification strategy.

The staggered nature of EERS adoption across states and over time provides the identifying variation for the difference-in-differences analysis. Critically, while states that adopt EERS may differ systematically from non-adopters in levels, the DiD approach requires only that they would have followed parallel trends in the absence of treatment. I examine this assumption extensively in the empirical strategy and robustness sections.

\subsection{Why States Adopt EERS}

Understanding the determinants of EERS adoption is important for assessing the plausibility of the parallel trends assumption. States adopt EERS for several reasons. First, environmental advocacy groups and public utility commissions in states with progressive regulatory traditions have pushed for efficiency mandates as a cost-effective alternative to new power plant construction. Second, states with high electricity prices face stronger political pressure to reduce consumer costs, and EERS programs are marketed as reducing long-run energy bills. Third, the American Recovery and Reinvestment Act of 2009 provided federal funding for state energy efficiency programs, lowering the cost of program administration and encouraging adoption.

Importantly, states do not appear to adopt EERS in direct response to trends in electricity consumption---the key threat to identification. Rather, adoption reflects political and institutional factors (governor's party, utility commission structure, environmental group activity) that are largely time-invariant or slowly-evolving, and thus absorbed by state fixed effects. I provide additional evidence on this point through pre-treatment trend tests in Section 7.

\subsection{Mechanisms of Effect}

EERS mandates can reduce electricity consumption through several channels. The \textit{direct program channel} operates through utility-administered efficiency programs that subsidize specific energy-saving investments. These programs include appliance rebates (e.g., \$100 off an ENERGY STAR refrigerator), weatherization services (insulation, air sealing, window upgrades), commercial building retrofits, and industrial process improvements. Engineering estimates suggest these programs achieve savings of 2--5\% per participating customer, though actual savings may be lower due to free-ridership and rebound effects.

The \textit{information channel} operates through mandatory energy audits, home energy reports, and energy benchmarking requirements that accompany many EERS programs. By providing consumers with information about their energy use relative to neighbors or efficiency potential, these programs can induce behavioral changes even without direct subsidies \citep{allcott2011}.

The \textit{market transformation channel} operates through the cumulative effect of efficiency programs on local contractor markets, appliance availability, and building practices. As utilities fund efficiency programs year after year, the local market for energy-efficient products and services expands, reducing costs and increasing adoption even beyond directly subsidized installations.

Countervailing forces include the \textit{rebound effect}, whereby efficiency improvements lower the effective price of energy services and induce additional consumption, and \textit{free-ridership}, whereby programs subsidize actions that would have occurred without the program, inflating reported savings without generating additional conservation.

%==============================================================================
\section{Conceptual Framework}
%==============================================================================

Consider a state $s$ that adopts an EERS mandate in year $g$, requiring utilities to achieve annual electricity savings of $\theta_s$ percent of retail sales through customer efficiency programs. The expected effect on state-level per-capita residential electricity consumption can be decomposed as:

\begin{equation}
\Delta \ln(E_{st}) = \underbrace{-\theta_s \cdot (1 - \phi_s)}_{\text{Net program savings}} + \underbrace{\eta_s \cdot \theta_s \cdot (1-\phi_s)}_{\text{Rebound effect}} + \underbrace{\gamma_s}_{\text{Market transformation}} + \underbrace{\epsilon_{st}}_{\text{Other factors}}
\label{eq:decomposition}
\end{equation}

\noindent where $\phi_s \in [0,1]$ is the free-ridership rate (fraction of program savings that would have occurred without the program), $\eta_s \in [0,1]$ is the rebound elasticity, and $\gamma_s$ captures net spillover effects---including market transformation (which reduces consumption, $\gamma < 0$) and behavioral responses such as the ``licensing effect'' or increased amenity consumption (which may increase it, $\gamma > 0$). The sign and magnitude of $\gamma$ is an empirical question.

Simplifying, the net effect is:
\begin{equation}
\Delta \ln(E_{st}) = -\theta_s(1-\phi_s)(1-\eta_s) + \gamma_s + \epsilon_{st}
\end{equation}

The overall treatment effect is negative (consumption falls) when direct net program savings exceed any positive spillovers:
$\theta_s(1-\phi_s)(1-\eta_s) > \gamma_s$. This condition may fail if free-ridership is near complete ($\phi \to 1$), the rebound effect is very large ($\eta \to 1$), or positive spillovers ($\gamma > 0$) dominate. With typical parameter values from the engineering literature ($\theta \approx 1.5\%$, $\phi \approx 0.2$, $\eta \approx 0.1$, $\gamma \approx 0$), the predicted annual net savings are approximately 1.1\%, which would cumulate to 5--10\% over 5--10 years of program operation. This provides a quantitative benchmark for interpreting my empirical estimates.

The EERS mandate also affects electricity prices. Utility revenue requirements include the costs of efficiency program administration, which are recovered through ratepayer surcharges. At the same time, reduced electricity sales reduce the variable costs of electricity generation. The net price effect depends on the relative magnitudes of these forces and on the state's utility regulatory framework (cost-of-service vs.\ performance-based regulation, decoupling provisions).

I test three predictions derived from this framework:
\begin{enumerate}
\item \textbf{Prediction 1 (Consumption).} EERS adoption reduces per-capita residential electricity consumption, with effects growing over time as programs mature.
\item \textbf{Prediction 2 (Prices).} EERS adoption may increase per-unit electricity prices due to program cost recovery, but the magnitude depends on the regulatory framework.
\item \textbf{Prediction 3 (Heterogeneity).} Effects are larger in states with more stringent targets and longer post-adoption periods, consistent with cumulative program savings.
\end{enumerate}

%==============================================================================
\section{Data}
%==============================================================================

\subsection{Electricity Consumption and Prices}

The primary outcome variable is state-level per-capita residential electricity consumption. I construct this from two sources from the U.S. Energy Information Administration (EIA).

\textit{State Energy Data System (SEDS).} SEDS provides annual estimates of total energy consumption by state, sector (residential, commercial, industrial, transportation), and fuel type, from 1960 to 2023. I use the ``Electricity consumed by the residential sector'' series (ESRCB), measured in billion Btu. SEDS data are derived from utility reports and are considered the most comprehensive source of state-level energy consumption data.

\textit{EIA Retail Sales Data.} The retail sales dataset provides annual electricity sales (MWh), revenue (thousand dollars), and average retail price (cents per kWh) by state and sector from 1990 to 2023. I use residential sales and prices as outcome and explanatory variables.

I access both datasets via the EIA's open API (v2), which provides machine-readable JSON data for all states and years. The API is freely accessible without authentication using the demonstration API key.

\subsection{Population Data}

I obtain annual state population estimates from the U.S. Census Bureau. For 2000--2023, I use intercensal and annual estimates from the Population Estimates Program (PEP), accessed via the Census API. For 1990--1999, I linearly interpolate between the 1990 Decennial Census count and the April 1, 2000 Census base, following standard practice in the state-level panel data literature. This yields a complete population series for all 51 jurisdictions across the full 1990--2023 study period.

\subsection{Treatment Coding}

I code each state's EERS adoption year based on the ACEEE State Energy Efficiency Resource Standards database, cross-referenced with the Database of State Incentives for Renewables \& Efficiency (DSIRE) and the National Conference of State Legislatures (NCSL) energy policy database. I classify a state as ``treated'' in the year it first adopted a \textit{binding mandatory} EERS with quantitative energy savings targets. States with voluntary goals, non-binding targets, or RPS provisions that include optional efficiency compliance pathways are classified as never-treated to maintain a sharp treatment definition.

This coding yields 28 treated jurisdictions (27 states plus DC) with adoption years ranging from 1998 (Connecticut) to 2020 (Maine, Virginia), and 23 never-treated states. Table~\ref{tab:cohorts} lists the adoption cohorts and constituent states.

\subsection{Sample Construction}

The analysis sample is a balanced panel of 51 jurisdictions observed annually from 1990 to 2023 (34 years), yielding 1,734 state-year observations. Population data are available for all jurisdiction-years: 2000--2023 from the Census Population Estimates Program and 1990--1999 from linear interpolation between the 1990 and 2000 Decennial Census counts.

\subsection{Summary Statistics}

Table~\ref{tab:summary_stats} presents summary statistics separately for EERS and non-EERS states, both in the full sample and restricted to pre-treatment years. Several patterns are notable. First, EERS states tend to have lower per-capita residential electricity consumption than non-EERS states, reflecting the concentration of non-adopters in hot-climate Southeastern states with high cooling demand. This level difference is absorbed by state fixed effects. Second, EERS states have higher average electricity prices, consistent with their location in more expensive electricity markets (Northeast, Pacific). Third, pre-treatment balance is similar to full-sample balance, suggesting that treatment adoption did not dramatically change group composition.

\begin{table}[htbp]
\centering
\caption{Summary Statistics by Treatment Status}
\label{tab:summary_stats}
\begin{tabular}{lccc}
\hline\hline
Variable & Not Afraid & Afraid & Overall \\
\hline
$N$ & 28,704 & 18,384 & 47,088 \\
\hline
Age & 46.5 & 47.4 & 46.9 \\
Female (%) & 43.6 & 74.1 & 55.5 \\
Black (%) & 12.0 & 18.2 & 14.4 \\
Education (years) & 13.3 & 12.8 & 13.1 \\
College+ (%) & 26.7 & 22.2 & 24.9 \\
Married (%) & 54.9 & 45.8 & 51.4 \\
Parent Educ (years) & 11.4 & 11.0 & 11.3 \\
Real Income ($) & \$35,466 & \$28,156 & \$32,647 \\
Conservative (%) & 34.3 & 31.9 & 33.4 \\
Urban (%) & 27.7 & 42.1 & 33.3 \\
\hline
Death Pen. Support (%) & 70.4 & 67.2 & 69.1 \\
Courts Lenient (%) & 74.4 & 80.9 & 77.0 \\
More Crime Spend (%) & 65.3 & 72.2 & 68.1 \\
\hline\hline
\end{tabular}
\begin{tablenotes}
\small
\item \textit{Notes:} Data from the General Social Survey, 1973--2024.
Treatment is defined as reporting fear of walking alone at night near home.
Real income in 2024 dollars.
\end{tablenotes}
\end{table}


\begin{table}[htbp]
\centering
\caption{EERS Adoption Cohorts}
\label{tab:cohorts}
\begin{tabular}{ccp{10cm}}
\toprule
Year & States & State Abbreviations \\
\midrule
1998 & 1 & CT \\
1999 & 1 & TX \\
2000 & 1 & VT \\
2004 & 1 & CA \\
2005 & 2 & NV, WI \\
2006 & 2 & RI, WA \\
2007 & 3 & CO, IL, MN \\
2008 & 8 & DC, MA, MD, MI, NC, NM, NY, PA \\
2009 & 1 & HI \\
2010 & 2 & AR, AZ \\
2016 & 1 & OR \\
2018 & 2 & NH, NJ \\
2019 & 1 & IA \\
2020 & 2 & ME, VA \\
\midrule
Total & 28 & \\
\bottomrule
\end{tabular}
\begin{tablenotes}
\small
\item \textit{Notes:} Year indicates the first year with a binding mandatory EERS. States with voluntary goals only are classified as never-treated.
\end{tablenotes}
\end{table}



%==============================================================================
\section{Empirical Strategy}
%==============================================================================

\subsection{Identification}

I estimate the causal effect of EERS adoption on electricity consumption using a difference-in-differences design that exploits the staggered timing of adoption across states. The identifying assumption is that, in the absence of EERS adoption, treated and never-treated states would have followed parallel trends in (log) per-capita residential electricity consumption.

Formally, let $Y_{st}(0)$ denote the potential outcome for state $s$ in year $t$ without EERS, and $Y_{st}(1)$ the potential outcome with EERS. The average treatment effect on the treated for group $g$ (states adopting in year $g$) at time $t$ is:
\begin{equation}
\text{ATT}(g,t) = \E[Y_{st}(1) - Y_{st}(0) \mid G_s = g]
\end{equation}

The parallel trends assumption states:
\begin{equation}
\E[Y_{st}(0) - Y_{s,t-1}(0) \mid G_s = g] = \E[Y_{st}(0) - Y_{s,t-1}(0) \mid G_s = \infty]
\end{equation}

\noindent for all $t \geq g$, where $G_s = \infty$ denotes never-treated states. That is, absent treatment, states that adopted EERS in year $g$ would have experienced the same changes in electricity consumption as states that never adopted EERS.

This assumption is most plausible when treatment adoption is driven by political and institutional factors (governor's party, utility commission structure, environmental group activity) rather than by differential trends in electricity consumption. If states adopted EERS specifically because their electricity consumption was rising faster than other states, the parallel trends assumption would be violated, and estimated treatment effects would be biased toward finding consumption reductions.

I provide several pieces of evidence supporting the parallel trends assumption. First, the event-study plot (Figure~\ref{fig:event_study}) shows that pre-treatment coefficients are centered on zero from 10 years before adoption, with no visible pre-trend. Second, I examine robustness to alternative comparison groups. Third, I conduct a placebo test using industrial electricity consumption, which should not be directly affected by EERS programs that primarily target residential customers.

\subsection{Estimation}

I use the Callaway and Sant'Anna (2021) estimator, which provides heterogeneity-robust estimates of the ATT in staggered adoption settings. The estimator proceeds in two steps. First, it estimates group-time average treatment effects $\widehat{\text{ATT}}(g,t)$ for each adoption cohort $g$ and time period $t$ using a doubly-robust approach that combines outcome regression with inverse probability weighting. Second, these group-time effects are aggregated into summary measures using appropriate weighted averages.

The key advantage of this estimator over conventional TWFE is that it avoids ``forbidden comparisons'' that use already-treated units as controls for later-treated units. As \citet{goodmanbacon2021} demonstrated, such comparisons can produce biased and even sign-reversed estimates when treatment effects vary across cohorts or over time---a concern that is particularly relevant for EERS, where programs take years to reach full effectiveness and states differ in target stringency.

I estimate the following specifications:

\begin{enumerate}
\item \textbf{Main specification.} CS-DiD with never-treated states as the comparison group, doubly-robust estimation, and universal base period.

\item \textbf{Alternative control.} CS-DiD with not-yet-treated states as an additional comparison group, which includes states that adopt EERS after the focal period.

\item \textbf{TWFE comparison.} Standard two-way fixed effects as a benchmark:
\begin{equation}
\ln E_{st}^{\text{pc}} = \alpha_s + \lambda_t + \beta \cdot \text{EERS}_{st} + \varepsilon_{st}
\label{eq:twfe}
\end{equation}
where $\alpha_s$ and $\lambda_t$ are state and year fixed effects, $\text{EERS}_{st}$ is an indicator equal to one after state $s$ adopts EERS, and $\varepsilon_{st}$ is an idiosyncratic error. Standard errors are clustered at the state level.

\item \textbf{Sun-Abraham.} The interaction-weighted estimator of \citet{sun2021}, implemented via \texttt{sunab()} in the \texttt{fixest} R package, which provides a cohort-specific event study.
\end{enumerate}

I aggregate group-time effects into four summary measures: (a) an overall ATT averaging across all cohorts and post-treatment periods; (b) group-level ATTs showing the average effect for each adoption cohort; (c) dynamic ATTs showing the average effect at each event time (years since adoption), which produce the event-study plot; and (d) calendar-time ATTs showing the average effect in each calendar year.

\subsection{Threats to Validity}

Several threats to the identifying assumption merit discussion.

\textit{Selection into treatment.} States that adopt EERS are not randomly selected; they tend to be wealthier, more urban, and more politically progressive. However, DiD identification requires only parallel trends, not random assignment. State fixed effects absorb all time-invariant differences between treated and control states, including climate, political culture, economic structure, and baseline consumption levels. The key question is whether time-varying confounders differentially affect treated and control states.

\textit{Concurrent policies.} EERS states may simultaneously adopt other energy or environmental policies (RPS, building codes, appliance standards) that also affect electricity consumption. If these policies are correlated with EERS adoption, my estimates capture the combined effect of the EERS and its policy complement rather than the isolated effect of EERS alone. I interpret my estimates as the ``EERS package'' effect, noting that this is the policy-relevant parameter for states considering EERS adoption.

\textit{Anticipation.} If utilities or consumers adjust behavior in anticipation of EERS adoption (e.g., utilities begin offering efficiency programs before the mandate takes effect), treatment effects may appear before the coded adoption year, violating the no-anticipation assumption. I examine this possibility through the event-study analysis, looking for pre-treatment effects in the years immediately before adoption.

\textit{Composition effects.} If EERS adoption changes the composition of economic activity in a state (e.g., driving energy-intensive industry to non-EERS states), per-capita consumption could fall through compositional shifts rather than actual efficiency improvements. I address this by examining industrial electricity consumption as a placebo outcome: if the residential effect is driven by targeted efficiency programs rather than compositional shifts, we should not observe a significant effect on industrial consumption.

%==============================================================================
\section{Results}
%==============================================================================

\subsection{Treatment Rollout}

Figure~\ref{fig:rollout} displays the staggered adoption of EERS across states. The earliest adopters (Connecticut, 1998; Texas, 1999; Vermont, 2000) are followed by a cluster of adoptions in 2005--2008 (11 states) and a later wave in 2016--2020 (6 states). The largest single adoption cohort is 2008, with eight states adopting EERS mandates simultaneously. This variation in timing is the key source of identification.

\begin{figure}[H]
\centering
\includegraphics[width=0.85\textwidth]{figures/fig1_treatment_rollout.pdf}
\caption{Staggered Adoption of Energy Efficiency Resource Standards}
\label{fig:rollout}
\end{figure}

\subsection{Raw Trends}

Figure~\ref{fig:trends} shows mean per-capita residential electricity consumption for EERS and non-EERS states over the sample period. Both groups show similar trajectories through the early 2000s, with consumption rising through approximately 2005 and then declining. The divergence between groups appears to begin around 2005--2008, coinciding with the major wave of EERS adoptions. However, raw trends do not control for pre-existing level differences or other time-varying factors, motivating the formal DiD analysis.

\begin{figure}[H]
\centering
\includegraphics[width=0.85\textwidth]{figures/fig2_raw_trends.pdf}
\caption{Mean Per-Capita Residential Electricity Consumption by EERS Status}
\label{fig:trends}
\end{figure}

\subsection{Main Results: Callaway-Sant'Anna Estimation}

Table~\ref{tab:main_results} presents the main results. Column (1) reports the preferred specification: the Callaway-Sant'Anna doubly-robust estimator with never-treated states as the comparison group. The overall ATT is $-0.0386$ (SE = $0.0245$), corresponding to a point estimate of approximately 3.9 percent lower per-capita residential electricity consumption in EERS states relative to never-treated states. However, the 95\% confidence interval $[-0.087, 0.009]$ includes zero, so this estimate is not statistically significant at conventional levels ($t = -1.58$, $p = 0.12$).

To put this magnitude in context, the average annual EERS savings target across states is approximately 1.0--1.5\% of retail sales. If the point estimate is taken at face value, a 3.9\% reduction in per-capita consumption after an average of 8 years of treatment would imply average annual realized savings of approximately 0.5\%, suggesting that about one-third to one-half of mandated savings translate into measurable population-level consumption reductions. The remainder would reflect free-ridership, rebound effects, or measurement differences between engineering estimates and econometric estimates.

Column (2) reports the conventional TWFE estimate of $-0.024$ (SE = $0.018$), also not statistically significant. The similarity in magnitude to the CS estimate suggests that in this setting, TWFE contamination from ``bad comparisons'' \citep{goodmanbacon2021} does not dramatically alter the point estimate, though the CS estimator remains preferred for valid inference under treatment effect heterogeneity.

Column (3) uses not-yet-treated states as an alternative comparison group, yielding an ATT of $-0.031$ (SE = $0.017$, $p < 0.10$). This is the most precisely estimated specification and is marginally significant at the 10\% level. The similar magnitude across comparison groups suggests that the direction of the effect is not an artifact of the choice of control group.

Columns (4) and (5) examine alternative outcome variables. The effect on total per-capita electricity consumption is $-0.013$ (SE = $0.010$), smaller than the residential-only effect and not statistically significant. The effect on residential electricity prices is $+0.045$ (SE = $0.065$), positive but far from statistically significant, providing no evidence that EERS programs increase per-unit electricity costs through program cost recovery.

\begin{table}[htbp]
\centering
\caption{Main Results: Effect of Energy Community Designation on Clean Energy Investment}
\label{tab:main_results}
\small
\begin{tabular}{lcccc}
\toprule
 & (1) & (2) & (3) & (4) \\
 & Sharp RDD & + Covariates & Quadratic & OLS (BW) \\
\midrule
Energy Community & -5.279 & -8.144 & -6.46 & -4.06 \\
 & (4.098) & (3.333) & (5.235) & (2.344) \\
 & [0.198] & [0.015] & [0.217] & \\
95\% CI & [-13.31, 2.75] & [-14.68, -1.61] & [-16.72, 3.8] & [-8.65, 0.53] \\
\midrule
Polynomial & Linear & Linear & Quadratic & Linear \\
Covariates & No & Yes & No & Yes \\
Bandwidth & 0.069 & 0.071 & 0.09 & 0.069 \\
N (left) & 27 & 28 & 35 & 27 \\
N (right) & 13 & 14 & 16 & 13 \\
\bottomrule
\end{tabular}
\begin{minipage}{0.95\textwidth}
\vspace{0.3em}
\footnotesize
\textit{Notes:} Dependent variable is post-IRA (2023+) clean energy generating capacity in megawatts per 1,000 employees. Columns (1)--(3) report robust bias-corrected estimates from \texttt{rdrobust} with Calonico-Cattaneo-Titiunik optimal bandwidth selection. Column (4) reports OLS within the optimal bandwidth. Standard errors in parentheses; $p$-values in brackets. Covariates include log population, median household income, percent with bachelor's degree, and percent white. Running variable: fossil fuel employment as percent of total employment (2021 CBP). Threshold: 0.17\% (IRA statutory cutoff). Sample: MSAs/non-MSAs with unemployment $\geq$ national average.
\end{minipage}
\end{table}


\subsection{Event Study: Dynamic Treatment Effects}

Figure~\ref{fig:event_study} presents the event-study analysis, plotting the estimated ATT at each event time (years relative to EERS adoption). The figure provides two critical pieces of evidence.

First, the pre-treatment coefficients (event times $-10$ to $-1$) are centered on zero and show no systematic pre-trend. This is the strongest available evidence for the parallel trends assumption: in the decade before EERS adoption, treated states were on the same consumption trajectory as never-treated states. The absence of pre-trends makes it unlikely that differential trends---rather than the EERS mandate itself---explain the post-treatment divergence.

Second, the post-treatment coefficients show a gradual, monotonic decline consistent with cumulative program effects. In the adoption year itself (event time 0), the point estimate is approximately $-0.01$ log points. By event time 5, the effect has grown to approximately $-0.025$ log points. By event time 10--15, effects reach $-0.05$ to $-0.08$ log points (5--8\% consumption reductions), though individual event-time estimates have wide confidence intervals. This dynamic pattern is consistent with the institutional reality that EERS programs require several years to reach full scale: utilities must design programs, hire contractors, recruit participants, and iteratively improve program delivery before achieving mandated savings levels.

\begin{figure}[H]
\centering
\includegraphics[width=0.9\textwidth]{figures/fig3_event_study_main.pdf}
\caption{Dynamic Treatment Effects of EERS on Residential Electricity Consumption}
\label{fig:event_study}
\end{figure}

The Sun-Abraham estimator produces qualitatively similar dynamics. Post-treatment coefficients range from $-0.011$ at event time 0 to $-0.079$ at event time 16, with the magnitude of the point estimates growing steadily over time. Pre-treatment coefficients at far-distant event times (beyond $-20$) show some noise, which is expected given that these are identified from a small number of early-adopting states with long pre-treatment histories.

\subsection{Group-Level Effects}

Figure~\ref{fig:group_att} presents the group-level ATT by adoption cohort (omitting the 2009 cohort, a single-state cohort where standard errors cannot be reliably estimated, and the 2020 cohort, for which the CS estimator did not return group-level estimates). The pattern reveals meaningful heterogeneity across cohorts. Early adopters (1998--2006 cohorts) show larger average treatment effects than later adopters (2008--2019 cohorts), which is expected given that early adopters have longer post-treatment periods over which savings accumulate. The 2008 cohort---the largest, with 8 jurisdictions---shows a moderate negative effect. Some later cohorts (2018--2019) have imprecise estimates due to short post-treatment periods.

\begin{figure}[H]
\centering
\includegraphics[width=0.85\textwidth]{figures/fig5_group_att.pdf}
\caption{Group-Level Average Treatment Effects by Adoption Cohort. Two of the 14 cohorts are omitted: the 2009 cohort (HI) is excluded because the CS estimator's clustered bootstrap did not converge on a standard error for this group, and the 2020 cohort (ME, VA) is excluded because the estimator did not return group-level estimates given the limited post-treatment variation (4 years).}
\label{fig:group_att}
\end{figure}

%==============================================================================
\section{Robustness}
%==============================================================================

\subsection{Alternative Control Groups}

Figure~\ref{fig:control_compare} overlays the event-study estimates using never-treated and not-yet-treated comparison groups. Both specifications yield similar pre-treatment patterns (flat, centered on zero) and post-treatment dynamics (gradually declining). The not-yet-treated specification produces somewhat smaller post-treatment estimates, which may reflect the mechanical reduction in comparison group size as more states enter treatment over time. The concordance of both specifications supports the robustness of the main finding.

\begin{figure}[H]
\centering
\includegraphics[width=0.9\textwidth]{figures/fig4_control_group_comparison.pdf}
\caption{Robustness: Alternative Control Groups}
\label{fig:control_compare}
\end{figure}

\subsection{Placebo Outcomes}

As a partial placebo test, I examine the effect of EERS on industrial electricity consumption. While EERS mandates in some states cover industrial customers, the direct effect should be concentrated in the residential sector, which is the primary target of most efficiency programs (appliance rebates, weatherization, home energy audits). I find a small positive and statistically insignificant effect on industrial consumption ($+0.045$, SE = $0.031$), consistent with EERS not meaningfully affecting industrial usage. This supports the interpretation that the residential consumption reduction reflects targeted demand-side programs rather than a spurious general trend.

\subsection{Alternative Outcome: Electricity Prices}

The effect of EERS on residential electricity prices provides insight into the welfare implications of the mandate. I estimate a positive but statistically insignificant coefficient of $+0.045$ (SE = $0.065$), corresponding to an approximate 4.5\% price increase that is statistically indistinguishable from zero. While the sign is consistent with utilities recovering efficiency program costs through rate increases, the imprecision prevents strong conclusions about the magnitude of price pass-through.

\subsection{Summary of Robustness}

Figure~\ref{fig:forest} presents a forest plot summarizing the ATT estimates across all specifications. All residential and total electricity specifications yield negative point estimates, indicating a consistent direction of effect across estimators (CS-DiD, TWFE), comparison groups (never-treated, not-yet-treated), and outcome measures (residential, total), though most individual estimates are not statistically significant at the 5\% level.

\begin{figure}[H]
\centering
\includegraphics[width=0.85\textwidth]{figures/fig7_robustness_forest.pdf}
\caption{Summary of ATT Estimates Across Specifications}
\label{fig:forest}
\end{figure}

%==============================================================================
\section{Heterogeneity}
%==============================================================================

\subsection{Early vs.\ Late Adopters}

I split the treated sample into early adopters (jurisdictions adopting EERS before 2008, $N=11$ states) and late adopters (2008 or later, $N=17$ jurisdictions including DC). Early adopters show a larger average treatment effect of $-4.0\%$ (SE = $1.9\%$) compared to late adopters at $-3.0\%$ (SE = $3.1\%$). Two interpretations are consistent with this pattern. First, early adopters have longer post-treatment periods, allowing cumulative savings to accumulate. Given the dynamic pattern in the event study---where effects grow over time---this mechanical explanation accounts for much of the difference. Second, early adopters may be positively selected on commitment to energy efficiency, implementing more stringent targets and better-funded programs.

Distinguishing between these explanations is important for policy. If the difference is primarily mechanical (more time = more savings), then late adopters will eventually reach similar cumulative reductions. If it reflects selection on commitment, then the marginal state considering EERS adoption may achieve smaller effects than the average treated state.

\subsection{Implications for EERS Design}

The heterogeneity results have implications for EERS program design. First, the growing dynamic effects suggest that program duration matters: states should not expect immediate large-scale savings, but rather a gradual ramp-up as utility programs mature and contractor markets develop. This argues for multi-year program commitments rather than annual targets that may lead to short-term program cycling.

Second, the difference between early and late adopters suggests that first-mover states may capture larger benefits, perhaps through earlier establishment of program infrastructure and supply chains. Late-adopting states may benefit from learning spillovers but face different market conditions.

%==============================================================================
\section{Discussion}
%==============================================================================

\subsection{Interpretation of Results}

The point estimate from the preferred specification---that EERS adoption is associated with approximately 3.9\% lower per-capita residential electricity consumption---is not statistically significant at conventional levels (95\% CI: $[-0.087, 0.009]$). However, several features of the results support the interpretation that EERS mandates likely reduce electricity consumption. First, the direction of the effect is consistently negative across all specifications, estimators, and comparison groups. Second, the not-yet-treated comparison group specification yields a marginally significant estimate of $-3.1\%$ ($p < 0.10$). Third, the event-study reveals a plausible dynamic pattern with flat pre-trends and gradually growing post-treatment effects. Fourth, the early-adopter subsample---states with the longest exposure to EERS---shows a statistically significant reduction of 4.0\% ($p < 0.05$).

The imprecision of the main estimate reflects the challenge of detecting moderate-sized effects in a state-level panel with 51 units and substantial cross-state heterogeneity in electricity consumption patterns, climate, economic structure, and program design. The minimum detectable effect at 80\% power in this design is approximately 5--6\%, which exceeds the point estimate.

If the point estimate is taken at face value, a 3.9\% reduction in residential electricity consumption across the 28 EERS jurisdictions would correspond to approximately 49 billion kWh per year in avoided electricity generation, equivalent to the annual output of approximately 10 large coal-fired power plants. At current average electricity prices, this would represent roughly \$5 billion in annual consumer savings (before accounting for program costs).

The positive but imprecisely estimated effect on electricity prices ($+4.5\%$, SE = $6.5\%$) prevents drawing welfare conclusions. If the point estimate were taken at face value, a household consuming 3.9\% less electricity at 4.5\% higher prices would experience a net change in its bill of $0.961 \times 1.045 - 1 \approx +0.4\%$---roughly offsetting consumption savings with price increases. However, both the consumption and price estimates have wide confidence intervals, making such calculations illustrative rather than definitive.

\subsection{Comparison to Prior Literature}

My point estimates are broadly consistent with prior engineering and econometric estimates, though less precise. The American Council for an Energy-Efficient Economy (ACEEE) reports that state-level efficiency programs achieved verified savings of 0.7--1.5\% of retail sales annually in leading states, which would cumulate to 5--10\% over a decade of sustained effort. My point estimate of 3.9\% after approximately 8 years of average treatment is within this range, though the confidence interval is wide enough to accommodate both negligible and substantial effects.

The comparison to RPS is instructive. \citet{deschenes2023} estimate that RPS increased wind generation capacity by 44\% in adopting states. While EERS and RPS operate on different margins (demand vs. supply), both appear to move outcomes in the intended direction when evaluated using modern causal methods, though the precision of estimates varies across settings.

\subsection{Limitations}

Several limitations merit acknowledgment. First, the state-year panel provides limited degrees of freedom, and my estimates are identified from variation across 28 treated and 23 never-treated jurisdictions over 34 years. While the CS estimator is designed for this setting, precision is limited for subgroup analyses and for isolating mechanisms.

Second, I cannot observe individual household behavior or program participation, so I cannot decompose the aggregate effect into contributions from specific program types (rebates, weatherization, behavioral programs). Individual-level data from utility administrative records would enable such decomposition but are not publicly available.

Third, the ``EERS package'' interpretation means that my estimates capture the combined effect of the EERS mandate and any correlated policies adopted simultaneously. States that adopt EERS may also strengthen building codes, appliance standards, or RPS requirements. Disentangling these policy interactions would require additional variation.

Fourth, the concentration of never-treated states in the Southeast and Mountain West raises questions about the external validity of the counterfactual. If these states have fundamentally different consumption dynamics (e.g., due to climate, housing stock, or economic structure), the estimated treatment effects may not generalize to the average treated state.

%==============================================================================
\section{Conclusion}
%==============================================================================

This paper provides the first causal evaluation of Energy Efficiency Resource Standards using modern heterogeneity-robust difference-in-differences methods. Exploiting the staggered adoption of mandatory EERS across 28 U.S. jurisdictions between 1998 and 2020, I estimate a point reduction in per-capita residential electricity consumption of approximately 3.9 percent, though the 95\% confidence interval includes zero. The direction of the effect is consistent across estimators, comparison groups, and outcome measures, and the event-study analysis reveals flat pre-trends and a plausible dynamic pattern of growing post-treatment effects. The early-adopter subsample yields a statistically significant estimate of $-4.0\%$ ($p < 0.05$), suggesting that the imprecision of the full-sample estimate reflects heterogeneity and limited statistical power rather than a true null effect.

These findings have implications for energy and climate policy. First, the consistent direction of point estimates across specifications suggests that state-level energy efficiency mandates are likely associated with real-world consumption reductions, though more precise estimation---perhaps using utility-level data or individual program records---is needed to establish this definitively. Second, the gradual ramp-up of effects underscores the importance of sustained, multi-year program commitments rather than short-term mandates. Third, the imprecisely estimated effect on electricity prices prevents conclusions about whether efficiency program costs are passed through to ratepayers, leaving open distributional concerns about non-participating households.

The growing interest in energy efficiency as a climate policy tool makes rigorous evaluation essential. As more states consider adopting or strengthening EERS mandates, and as the federal government explores national efficiency standards, evidence on the actual effectiveness of existing mandates is critical for informed policy design. This paper provides suggestive evidence that EERS mandates reduce consumption, but the imprecision of the estimates highlights the need for richer data sources and continued program evaluation.

\section*{Acknowledgements}

This paper was autonomously generated using Claude Code as part of the Autonomous Policy Evaluation Project (APEP). All electricity consumption and price data are from the U.S. Energy Information Administration. Population data are from the U.S. Census Bureau. EERS treatment coding is based on the ACEEE State Energy Efficiency Resource Standards database, the Database of State Incentives for Renewables \& Efficiency (DSIRE), and the National Conference of State Legislatures (NCSL).

\noindent\textbf{Project Repository:} \url{https://github.com/SocialCatalystLab/auto-policy-evals}

\noindent\textbf{Contributors:} APEP Autonomous Research

\label{apep_main_text_end}
\newpage

\begin{thebibliography}{99}

\bibitem[Allcott(2011)]{allcott2011}
Allcott, H. (2011). Social norms and energy conservation. \textit{Journal of Public Economics}, 95(9--10), 1082--1095.

\bibitem[Auffhammer and Mansur(2014)]{auffhammer2014}
Auffhammer, M., and Mansur, E.T. (2014). Measuring climatic impacts on energy consumption: A review of the empirical literature. \textit{Energy Economics}, 46, 522--530.

\bibitem[Baker et al.(2025)]{baker2025}
Baker, A.C., Callaway, B., Cunningham, S., Goodman-Bacon, A., and Sant'Anna, P.H.C. (2025). Difference-in-Differences Designs: A Practitioner's Guide. arXiv:2503.13323.

\bibitem[Barbose et al.(2013)]{barbose2013}
Barbose, G.L., Goldman, C.A., Hoffman, I.M., and Billingsley, M. (2013). The future of utility customer-funded energy efficiency programs in the United States: Projected spending and savings to 2025. \textit{Energy Efficiency}, 6, 475--493.

\bibitem[Borenstein and Bushnell(2016)]{borenstein2016}
Borenstein, S., and Bushnell, J. (2016). The U.S. electricity industry after 20 years of restructuring. \textit{Annual Review of Economics}, 7, 437--463.

\bibitem[Callaway and Sant'Anna(2021)]{callaway2021}
Callaway, B., and Sant'Anna, P.H.C. (2021). Difference-in-Differences with multiple time periods. \textit{Journal of Econometrics}, 225(2), 200--230.

\bibitem[de Chaisemartin and D'Haultfoeuille(2020)]{dechaisemartin2020}
de Chaisemartin, C., and D'Haultfoeuille, X. (2020). Two-way fixed effects estimators with heterogeneous treatment effects. \textit{American Economic Review}, 110(9), 2964--2996.

\bibitem[Deschenes, Malloy, and McDonald(2023)]{deschenes2023}
Deschenes, O., Malloy, C., and McDonald, G. (2023). Causal Effects of Renewable Portfolio Standards on Renewable Investments and Generation: The Role of Heterogeneity and Dynamics. NBER Working Paper 31568.

\bibitem[Gillingham et al.(2018)]{gillingham2018}
Gillingham, K., Rapson, D., and Wagner, G. (2018). The rebound effect and energy efficiency policy. \textit{Review of Environmental Economics and Policy}, 10(1), 68--88.

\bibitem[Goodman-Bacon(2021)]{goodmanbacon2021}
Goodman-Bacon, A. (2021). Difference-in-differences with variation in treatment timing. \textit{Journal of Econometrics}, 225(2), 254--277.

\bibitem[Greenstone and Nath(2024)]{greenstone2024}
Greenstone, M., and Nath, I. (2024). Do Renewable Portfolio Standards deliver cost-effective carbon abatement? University of Chicago Energy Policy Institute Working Paper.

\bibitem[Ito(2014)]{ito2014}
Ito, K. (2014). Do consumers respond to marginal or average price? Evidence from nonlinear electricity pricing. \textit{American Economic Review}, 104(2), 537--563.

\bibitem[Jessoe and Rapson(2014)]{jessoe2014}
Jessoe, K., and Rapson, D. (2014). Knowledge is (less) power: Experimental evidence from residential energy use. \textit{American Economic Review}, 104(4), 1417--1438.

\bibitem[Joskow(2014)]{joskow2014}
Joskow, P. (2014). Incentive regulation in theory and practice: Electricity distribution and transmission networks. In \textit{Economic Regulation and Its Reform}, pp. 291--344. University of Chicago Press.

\bibitem[Levinson(2016)]{levinson2016}
Levinson, A. (2016). How much energy do building energy codes save? Evidence from California houses. \textit{American Economic Review}, 106(10), 2867--2894.

\bibitem[Myers(2019)]{myers2019}
Myers, E. (2019). Are home buyers inattentive? Evidence from capitalization of energy costs. \textit{American Economic Journal: Economic Policy}, 11(2), 165--188.

\bibitem[Rambachan and Roth(2023)]{rambachan2023}
Rambachan, A., and Roth, J. (2023). A more credible approach to parallel trends. \textit{Review of Economic Studies}, 90(5), 2555--2591.

\bibitem[Roth(2022)]{roth2022}
Roth, J. (2022). Pretest with Caution: Event-Study Estimates after Testing for Parallel Trends. \textit{American Economic Review: Insights}, 4(3), 305--322.

\bibitem[Roth et al.(2023)]{roth2023}
Roth, J., Sant'Anna, P.H.C., Bilinski, A., and Poe, J. (2023). What's Trending in Difference-in-Differences? A Synthesis of the Recent Econometrics Literature. \textit{Journal of Econometrics}, 235(2), 2218--2244.

\bibitem[Sant'Anna and Zhao(2020)]{santanna2020}
Sant'Anna, P.H.C., and Zhao, J. (2020). Doubly Robust Difference-in-Differences Estimators. \textit{Journal of Econometrics}, 219(1), 101--122.

\bibitem[Sun and Abraham(2021)]{sun2021}
Sun, L., and Abraham, S. (2021). Estimating Dynamic Treatment Effects in Event Studies with Heterogeneous Treatment Effects. \textit{Journal of Econometrics}, 225(2), 175--199.

\bibitem[Fowlie, Greenstone, and Wolfram(2018)]{fowlie2018}
Fowlie, M., Greenstone, M., and Wolfram, C. (2018). Do energy efficiency investments deliver? Evidence from the Weatherization Assistance Program. \textit{Quarterly Journal of Economics}, 133(3), 1597--1644.

\bibitem[Allcott and Greenstone(2012)]{allcott2012}
Allcott, H., and Greenstone, M. (2012). Is there an energy efficiency gap? \textit{Journal of Economic Perspectives}, 26(1), 3--28.

\bibitem[Davis, Fuchs, and Gertler(2014)]{davis2014}
Davis, L.W., Fuchs, A., and Gertler, P. (2014). Cash for coolers: Evaluating a large-scale appliance replacement program in Mexico. \textit{American Economic Journal: Economic Policy}, 6(4), 207--238.

\bibitem[Jacobsen and Kotchen(2013)]{jacobsen2013}
Jacobsen, G.D., and Kotchen, M.J. (2013). Are building codes effective at saving energy? Evidence from residential billing data in Florida. \textit{Review of Economics and Statistics}, 95(1), 34--49.

\bibitem[Arimura et al.(2012)]{arimura2012}
Arimura, T.H., Li, S., Newell, R.G., and Palmer, K. (2012). Cost-effectiveness of electricity energy efficiency programs. \textit{Energy Journal}, 33(2), 63--99.

\end{thebibliography}

\newpage
\appendix

%==============================================================================
\section{Data Appendix}
%==============================================================================

\subsection{Data Sources and Access}

All data used in this paper are publicly accessible through government APIs and databases.

\textit{EIA State Energy Data System (SEDS).} Accessed via \texttt{api.eia.gov/v2/seds/data/} with the DEMO\_KEY. Series ESRCB (residential electricity consumption in Billion Btu), ESTCB (total), ESCCB (commercial), and ESICB (industrial) were downloaded for all states for the period 1990--2023. Data were last accessed on January 27, 2026.

\textit{EIA Retail Sales.} Accessed via \texttt{api.eia.gov/v2/electricity/retail-sales/data/}. Annual residential and commercial sector data including price (cents/kWh), revenue (thousand \$), and sales (MWh) were downloaded for 1990--2023.

\textit{Census Population Estimates.} Accessed via \texttt{api.census.gov}. Intercensal estimates for 2000--2009 (PEP/int\_population endpoint), annual estimates for 2010--2019 (PEP/population), and vintage 2023 estimates for 2020--2023 were combined. For 1990--1999, state populations were linearly interpolated between the 1990 Decennial Census count and the April 1, 2000 Census base from the intercensal estimates, yielding a complete state-year population panel for 1990--2023.

\textit{EERS Treatment Coding.} Compiled from the ACEEE State Energy Efficiency Resource Standards database (\url{database.aceee.org}), cross-referenced with DSIRE (\url{dsireusa.org}) and NCSL (\url{ncsl.org/energy}). Treatment is defined as the first year of a binding mandatory EERS with quantitative savings targets.

\subsection{Variable Definitions}

\begin{itemize}
\item \textbf{Per-capita residential electricity consumption:} SEDS series ESRCB (Billion Btu) divided by state population. Measured in Billion Btu per person.
\item \textbf{Log per-capita residential electricity:} Natural logarithm of per-capita residential electricity consumption. This is the primary dependent variable.
\item \textbf{Residential electricity price:} Average retail price of electricity to residential customers, in cents per kilowatt-hour, from EIA retail sales data.
\item \textbf{EERS indicator:} Binary variable equal to 1 in all years $\geq$ the state's EERS adoption year, and 0 otherwise. Set to 0 for all years in never-treated states.
\item \textbf{First treatment year:} The year the state first adopted a binding mandatory EERS. Set to 0 for never-treated states (as required by the \texttt{did} R package).
\end{itemize}

\subsection{Sample Restrictions}

The panel consists of 51 jurisdictions (50 states + DC) $\times$ 34 years (1990--2023) = 1,734 state-year observations. Population data for 2000--2023 come from the Census Bureau's Population Estimates Program (intercensal estimates for 2000--2009, annual estimates for 2010--2019, and vintage 2023 estimates for 2020--2023). For 1990--1999, I linearly interpolate between the 1990 Decennial Census count and the April 1, 2000 Census base, following standard practice in the state-level panel data literature. Energy consumption data from EIA SEDS are available for all 51 jurisdictions across the full 1990--2023 period, yielding a complete balanced panel of 1,734 state-year observations with no missing values.

%==============================================================================
\section{Identification Appendix}
%==============================================================================

\subsection{Adoption Cohort Details}

Table~\ref{tab:cohorts} in the main text lists all 14 adoption cohorts and their constituent states. The largest cohort is 2008 (8 states), followed by 2007 (3 states) and several years with 2 states each. Five cohorts consist of a single state. This distribution provides reasonable variation in treatment timing, though the concentration of adoptions in 2007--2008 means that a substantial fraction of the treatment effect estimate is identified from this period.

\subsection{Pre-Treatment Covariate Balance}

The summary statistics in Table~\ref{tab:summary_stats} show that EERS and non-EERS states differ in levels of consumption and prices. EERS states tend to have lower per-capita consumption (reflecting concentration in the Northeast and Pacific regions with moderate climates and older housing stock) and higher electricity prices (reflecting higher-cost electricity markets). These level differences are absorbed by state fixed effects and do not threaten identification, which relies on parallel trends rather than level equivalence.

\subsection{Goodman-Bacon Decomposition}

I decompose the TWFE estimate using the \citet{goodmanbacon2021} method to illustrate the sources of identification. The decomposition reveals that 74.3\% of the TWFE weight comes from ``clean'' treated-vs-untreated comparisons (average estimate: $-0.029$), 15.9\% from earlier-vs-later-treated comparisons ($-0.020$), and 9.8\% from later-vs-earlier-treated comparisons ($+0.008$). The positive coefficient on the later-vs-earlier component reflects the ``forbidden comparisons'' that contaminate TWFE in staggered settings: when later-treated states are compared to already-treated states whose outcomes have already declined, the estimand is biased toward zero or positive values. The overall TWFE estimate of $-0.024$ is attenuated relative to the CS estimate of $-0.039$ partly because of this contamination, though the dominant clean-comparison component ($-0.029$) is close to the CS point estimate. This decomposition confirms that the CS estimator is preferred, though TWFE contamination is modest in this application.

%==============================================================================
\section{Robustness Appendix}
%==============================================================================

\subsection{Alternative Outcomes}

The effect on total per-capita electricity consumption ($-0.013$, SE = $0.010$) is smaller than the residential-only effect ($-0.039$). This is consistent with the residential sector being the primary target of most EERS programs (through appliance rebates, weatherization, and home energy audits), while commercial and industrial consumption may be less affected by programs designed primarily for households. Additionally, the total electricity measure aggregates across sectors with different consumption dynamics, potentially diluting the residential-specific signal.

\subsection{Industrial Electricity as Partial Placebo}

The effect on industrial electricity consumption ($+0.045$, SE = $0.031$) is small, positive, and statistically insignificant. This is reassuring as a partial placebo test: EERS mandates primarily target residential and commercial customers, and the absence of a significant industrial effect supports the interpretation that the residential consumption reduction reflects targeted demand-side programs rather than spurious general trends. A cleaner placebo would require an outcome that is mechanically unrelated to EERS---such as transportation energy use or heating fuel consumption---which I leave for future work.

\subsection{Price Effects}

The positive but imprecise effect on residential electricity prices ($+0.045$, SE = $0.065$) is consistent in sign with the theoretical prediction that utilities recover efficiency program costs through ratepayer surcharges. However, the large standard error makes the estimate statistically indistinguishable from zero, preventing strong conclusions about the magnitude or even the existence of price pass-through. Better identification of price effects may require finer geographic or temporal variation than the state-year panel provides.

%==============================================================================
\section{Heterogeneity Appendix}
%==============================================================================

\subsection{Early vs.\ Late Adopters: Detailed Results}

The early-adopter subsample (11 states adopting before 2008: CT, TX, VT, CA, NV, WI, RI, WA, CO, IL, MN) shows an ATT of $-0.040$ (SE = $0.019$, $p < 0.05$). These states have an average of 17 years of post-treatment data, allowing cumulative savings to accumulate substantially.

The late-adopter subsample (17 jurisdictions adopting 2008 or later: DC, MD, MA, MI, NM, NY, NC, PA, HI, AZ, AR, OR, NH, NJ, IA, ME, VA) shows an ATT of $-0.030$ (SE = $0.031$). These jurisdictions have an average of 10 years of post-treatment data. The smaller and less precisely estimated effect is consistent with the event-study finding that treatment effects grow over time.

The difference between early and late adopter effects ($-0.040$ vs. $-0.030$) could also reflect positive selection: states that adopted EERS earliest may have been more committed to energy efficiency, implemented more stringent targets, and invested more heavily in program infrastructure. Disentangling timing from selection requires additional variation (e.g., instruments for adoption timing) that is beyond the scope of this paper.

%==============================================================================
\section{Additional Figures and Tables}
%==============================================================================

\begin{figure}[H]
\centering
\includegraphics[width=0.85\textwidth]{figures/fig6_alternative_outcomes.pdf}
\caption{EERS Effects Across Outcome Variables: Residential Electricity, Total Electricity, and Prices}
\label{fig:alt_outcomes}
\end{figure}

\end{document}
