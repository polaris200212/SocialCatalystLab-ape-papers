\documentclass[12pt]{article}

% UTF-8 encoding and fonts
\usepackage[utf8]{inputenc}
\usepackage[T1]{fontenc}
\usepackage{lmodern}

% Page setup
\usepackage[margin=1in]{geometry}
\usepackage{setspace}
\onehalfspacing

% Typography
\usepackage{microtype}

% Math and symbols
\usepackage{amsmath,amssymb}

% Graphics
\usepackage{graphicx}
\usepackage{float}
\usepackage{subcaption}

% Tables
\usepackage{booktabs}
\usepackage{array}
\usepackage{multirow}
\usepackage{threeparttable}
\usepackage{longtable}
\usepackage{pdflscape}
\usepackage{siunitx}
\sisetup{detect-all=true, group-separator={,}, group-minimum-digits=4}

% Bibliography
\usepackage{natbib}
\bibliographystyle{aer}

% Hyperlinks
\usepackage{hyperref}
\hypersetup{
    colorlinks=true,
    linkcolor=blue,
    citecolor=blue,
    urlcolor=blue
}
\usepackage[nameinlink,noabbrev]{cleveref}

% Captions
\usepackage{caption}
\captionsetup{font=small,labelfont=bf}

% Section formatting
\usepackage{titlesec}
\titleformat{\section}{\large\bfseries}{\thesection.}{0.5em}{}
\titleformat{\subsection}{\normalsize\bfseries}{\thesubsection}{0.5em}{}

% Custom commands
\newcommand{\E}{\mathbb{E}}
\newcommand{\Var}{\text{Var}}
\newcommand{\Cov}{\text{Cov}}
\newcommand{\ind}{\mathbb{I}}
\newcommand{\sym}[1]{\ifmmode^{#1}\else\(^{#1}\)\fi}

\title{State Minimum Wage Increases and Business Establishments:\\ Evidence from Staggered Policy Adoption}
\author{Autonomous Policy Evaluation Project (APEP) \\ @“ai1scl”}
\date{\today}

\begin{document}

\maketitle

\begin{abstract}
\noindent
Does raising the minimum wage discourage business formation? Standard economic theory predicts that higher labor costs reduce firm entry, yet empirical evidence on this extensive margin remains surprisingly sparse. This paper exploits staggered adoption of above-federal minimum wages across U.S. states between 2012 and 2021 to estimate causal effects on business establishment counts using a modern difference-in-differences framework. Drawing on Census Bureau County Business Patterns data covering 51 jurisdictions over 10 years (510 state-year observations), I find precisely estimated null effects: the elasticity of establishment counts with respect to the real minimum wage is $-0.018$ (SE = 0.036), implying that a 10 percent minimum wage increase is associated with a 0.18 percent decrease in establishments---an economically small effect that is statistically indistinguishable from zero. Callaway-Sant'Anna heterogeneity-robust estimators confirm the null result (ATT = $-0.013$, SE = 0.012), and event study estimates show no differential pre-trends among states with within-sample policy adoption. Results are robust to excluding large states, adding state-specific trends, and alternative treatment definitions. The findings suggest that minimum wage increases in the range observed during this period do not detectably affect aggregate business formation, contributing to the broader debate about whether minimum wage policy poses meaningful barriers to entrepreneurship.
\end{abstract}

\vspace{1em}
\noindent\textbf{JEL Codes:} J38, L26, J23, J31, M13 \\
\noindent\textbf{Keywords:} minimum wage, business formation, entrepreneurship, difference-in-differences, state policy, labor costs

\newpage

\section{Introduction}

The minimum wage remains one of the most contested labor market policies in the United States. Since the seminal work of \citet{card1994minimum}, researchers have intensively studied the employment effects of minimum wage increases, with a vast literature debating whether mandated wage floors reduce jobs for low-skill workers \citep{neumark2008minimum, dube2019minimum, cengiz2019effect}. Far less attention, however, has been devoted to the extensive margin of firm creation---whether minimum wage increases deter entrepreneurs from starting new businesses in the first place.

This question matters for several reasons. If minimum wage increases discourage business formation, the long-run consequences for job creation could extend well beyond contemporaneous employment effects. New businesses are disproportionately responsible for job creation \citep{decker2014role, haltiwanger2013creates}, and policies that reduce firm entry could have persistent effects on economic dynamism. Moreover, as policymakers increasingly consider substantial minimum wage increases---with several states and cities moving toward \$15 per hour or higher---understanding the full range of potential responses becomes essential for informed policy design.

Standard economic theory offers competing predictions about how minimum wages should affect business formation. On one hand, higher labor costs raise the hurdle for profitable operation, potentially deterring marginal entrants who cannot profitably pay the mandated wage. This is the traditional ``job killer'' narrative: entrepreneurs considering whether to open a restaurant, retail store, or other labor-intensive business may decide the numbers no longer work when facing higher wage floors. On the other hand, efficiency wage considerations suggest that higher mandated wages might improve worker quality, reduce turnover, and enhance productivity \citep{autor2016contribution}, potentially making some business models more attractive. The net effect is therefore an empirical question---one that has received surprisingly little direct study.

This paper estimates the causal effect of state minimum wage increases on business establishment counts using variation from the staggered adoption of above-federal minimum wages across U.S. states. The sample includes 50 states plus the District of Columbia, providing a natural laboratory for policy evaluation. Between 2012 and 2021, 29 of these 51 jurisdictions raised their minimum wage above the federal floor of \$7.25 per hour, while the remaining 22 jurisdictions maintained wages at exactly the federal minimum throughout this period. This staggered adoption generates identifying variation that I exploit using both traditional two-way fixed effects (TWFE) estimation and the heterogeneity-robust estimator of \citet{callaway2021difference}.

The main finding is a precisely estimated null effect. Using Census Bureau County Business Patterns (CBP) data on annual establishment counts, I estimate an elasticity of establishment counts with respect to the real minimum wage of $-0.018$ (SE = 0.036). This implies that a 10 percent increase in the real minimum wage is associated with a mere 0.18 percent decrease in establishments---an effect that is both economically trivial and statistically insignificant. The 95 percent confidence interval for this estimate implies that a 10 percent minimum wage increase changes establishments by between roughly $-0.9$ percent and $+0.5$ percent, allowing me to rule out economically meaningful effects in either direction.

These results are confirmed using the Callaway-Sant'Anna estimator, which addresses concerns about heterogeneous treatment effects in difference-in-differences designs with staggered adoption \citep{goodman2021difference, sun2021estimating, roth2023s}. The aggregate treatment effect on the treated (ATT) is $-0.013$ (SE = 0.012), with a 95 percent confidence interval of ($-0.036$, 0.010). Event study estimates support the identifying assumption of parallel trends: pre-treatment coefficients for years $t = -2$ through $t = -5$ are small and statistically insignificant, and post-treatment coefficients remain flat and close to zero throughout the sample period.

The null results are robust across an extensive battery of sensitivity checks. Excluding the three largest states by population (California, New York, and Texas) yields nearly identical estimates ($-0.016$, SE = 0.037). Adding state-specific linear time trends attenuates the point estimate toward zero (0.004, SE = 0.014). Alternative treatment specifications using binary indicators (above federal vs.\ at federal) also yield insignificant effects. The Goodman-Bacon decomposition confirms that the TWFE estimates are dominated by clean comparisons between treated and never-treated states, with 82 percent of the weight coming from these comparisons.

This paper contributes to several literatures. First, it extends minimum wage research beyond employment effects to examine the extensive margin of business creation. While the employment literature is vast, direct evidence on business formation is sparse, with most related work focusing on firm-level outcomes like employment per establishment rather than establishment counts themselves \citep{harasztosi2019labor}. Second, the paper contributes to research on the determinants of entrepreneurship \citep{hurst2004entrepreneurship, fairlie2018entrepreneurship}. Finding that minimum wage policy---at observed levels---does not appear to be a meaningful barrier adds to our understanding of what does (and does not) drive business formation decisions. Third, methodologically, the paper demonstrates the application of modern difference-in-differences techniques \citep{callaway2021difference, goodman2021difference} to a policy setting with staggered adoption, explicitly addressing concerns about heterogeneous treatment effects that have recently received considerable attention.

The remainder of the paper proceeds as follows. Section 2 reviews the relevant literature on minimum wages and business formation. Section 3 provides institutional background on minimum wage policy in the United States. Section 4 describes the data sources and sample construction. Section 5 presents the empirical strategy, including both TWFE and Callaway-Sant'Anna estimation. Section 6 reports the main results, robustness checks, and event study evidence. Section 7 discusses mechanisms, limitations, and policy implications, and Section 8 concludes.


\section{Related Literature}

This paper contributes to three strands of the economics literature: the vast empirical literature on minimum wage effects, the growing literature on business dynamics and entrepreneurship, and the methodological literature on difference-in-differences estimation with staggered policy adoption.

\subsection{Minimum Wage Effects on Employment}

The modern empirical literature on minimum wage employment effects traces to \citet{card1994minimum}, whose study of New Jersey's 1992 minimum wage increase found no evidence of employment losses in fast food restaurants. This finding challenged the standard competitive labor market prediction and sparked decades of subsequent research.

The literature has produced divergent findings depending on methodology and data. \citet{neumark2008minimum} provide a comprehensive review arguing that minimum wages reduce employment for low-skilled workers, with elasticities typically in the range of $-0.1$ to $-0.3$. In contrast, \citet{dube2019minimum} reviews international evidence and concludes that employment effects are generally small, with most estimates clustered around zero.

More recent work has emphasized the importance of research design. \citet{cengiz2019effect} use bunching estimators to examine employment distributions around the minimum wage, finding minimal job losses even for substantial increases. This approach sidesteps concerns about comparing dissimilar geographic units that plague earlier state-panel studies.

While this employment literature is extensive, it focuses almost exclusively on incumbent workers and establishments. The question of whether minimum wages affect business formation---the extensive margin of firm entry---has received far less attention, despite its potential importance for long-run job creation.

\subsection{Business Dynamics and Entrepreneurship}

A parallel literature has documented the importance of new businesses for economic dynamism. \citet{decker2014role} show that young firms are disproportionately responsible for job creation in the United States, with high-growth startups playing an outsized role. \citet{haltiwanger2013creates} demonstrate that firm age, rather than size, is the key driver of job creation---small firms create jobs primarily because they are young, not because they are small.

This literature has also documented a secular decline in business dynamism in the United States. Entry rates have fallen, job reallocation has declined, and the share of employment at young firms has shrunk \citep{decker2014role}. Understanding the determinants of business formation---including the role of labor market policies---is therefore increasingly important for understanding aggregate economic performance.

The entrepreneurship literature has examined various determinants of entry, including access to capital \citep{hurst2004entrepreneurship}, regulatory burdens, and local economic conditions. \citet{fairlie2018entrepreneurship} distinguish between ``opportunity'' and ``necessity'' entrepreneurship, noting that the motivations and outcomes of business formation vary substantially. Minimum wage policy could potentially affect both types: higher mandated wages could deter opportunity entrepreneurs for whom labor-intensive business models become unprofitable, while simultaneously pushing necessity entrepreneurs toward self-employment as wage work becomes scarce.

Surprisingly, direct evidence on the minimum wage--business formation nexus is sparse. A few studies examine related outcomes---such as firm profits \citep{harasztosi2019labor} or employment growth \citep{autor2016contribution}---but establishment counts and entry rates have received little direct attention.

\subsection{Difference-in-Differences with Staggered Adoption}

The econometric literature on difference-in-differences estimation has undergone substantial development in recent years, with particular attention to designs involving staggered policy adoption. Traditional two-way fixed effects (TWFE) estimators, long the workhorse of policy evaluation, have been shown to produce potentially biased estimates when treatment effects vary across units or over time \citep{goodman2021difference}.

The core issue is that TWFE implicitly uses already-treated units as controls for later-treated units, creating ``forbidden comparisons'' that can contaminate estimates. \citet{goodman2021difference} provides a decomposition showing how TWFE aggregates many underlying 2$\times$2 comparisons, some of which can receive negative weights under heterogeneous effects.

Several solutions have been proposed. \citet{callaway2021difference} develop a group-time average treatment effect framework that avoids using already-treated units as controls. \citet{sun2021estimating} propose an interaction-weighted estimator for event studies. \citet{roth2023s} synthesize these developments and provide practical guidance for applied researchers.

This paper applies these modern methods to the minimum wage--business formation question, using both traditional TWFE and the Callaway-Sant'Anna estimator. The staggered adoption of above-federal minimum wages across states provides a natural setting for these techniques, while the availability of never-treated states enables clean identification using the recommended control group.


\section{Institutional Background}

\subsection{History of the Federal Minimum Wage}

The federal minimum wage was established by the Fair Labor Standards Act (FLSA) of 1938, initially set at \$0.25 per hour. The original legislation covered approximately 20 percent of the workforce, primarily in manufacturing and interstate commerce. Over the following decades, Congress periodically raised the minimum wage and expanded coverage to include additional sectors, including retail, service, and agricultural workers.

The trajectory of the federal minimum wage has been uneven. In real terms, the federal minimum reached its peak purchasing power in 1968 at approximately \$12 in 2020 dollars. Since then, the real value has declined substantially due to inflation outpacing nominal increases. Congress has raised the federal minimum wage 22 times since 1938, but the intervals between increases have grown longer, and the real value has trended downward.

The current federal minimum wage of \$7.25 per hour took effect in July 2009, concluding a three-step increase from \$5.15 that was enacted in 2007. This level has remained unchanged for over 15 years---the longest period without an increase in the history of the federal minimum wage. In real terms, the \$7.25 federal minimum has lost approximately 25 percent of its purchasing power since 2009, leaving workers at the federal floor with substantially reduced real earnings.

\subsection{State Authority and Preemption}

The FLSA establishes a federal floor but explicitly permits states to set higher minimum wages. Under the doctrine of federal preemption, the higher of the state or federal rate applies in any jurisdiction. This dual system creates substantial policy variation, as states may choose to supplement the federal floor or simply default to it.

States' authority to set minimum wages derives from their general police powers under the Tenth Amendment. Unlike some areas of labor law where federal preemption limits state action, minimum wage policy permits concurrent state regulation. This has important implications for policy evaluation: researchers can exploit variation in state policies to identify causal effects while controlling for federal-level changes that affect all states simultaneously.

As of the study period (2012--2021), states exhibited three broad patterns. First, 22 jurisdictions maintained minimum wages at exactly the federal floor of \$7.25, effectively ceding minimum wage policy to the federal government. Second, 13 jurisdictions had established above-federal minimum wages prior to 2012 and maintained them throughout the study period. Third, 16 jurisdictions crossed above the federal threshold at some point during 2012--2021, providing within-sample variation for identification.

\subsection{State Minimum Wage Variation}

As of December 2021, 29 of the 51 U.S. jurisdictions (50 states plus the District of Columbia) maintained minimum wages above the federal floor. State policies exhibit substantial variation in both levels and timing. At one extreme, California and the District of Columbia led the way with minimums exceeding \$14 per hour by 2021, while Washington State maintained an above-federal minimum since the early 2000s due to its inflation-indexed system. At the other extreme, states like Mississippi, Alabama, and Louisiana have never enacted their own minimum wage laws and remain at exactly \$7.25.

The timing of adoption also varies considerably. ``Early movers'' like Washington, Oregon, and California have maintained above-federal minimums since well before the study period. Other states crossed the federal threshold more recently: Florida exceeded \$7.25 in 2019 and subsequently approved a ballot initiative in 2020 for a phased increase to \$15. This staggered adoption pattern---with different states raising their minimums at different times---provides the identifying variation for this study.

\subsection{The Political Economy of State Minimum Wages}

Understanding the political economy of minimum wage adoption helps contextualize the identification strategy. States that raise their minimum wages tend to differ systematically from states that do not. Higher-minimum-wage states are typically more urban, have higher costs of living, stronger labor movements, and more liberal electorates. California, New York, and Massachusetts exemplify this pattern.

Importantly for identification, however, the timing of increases within states is largely driven by political factors---election cycles, gubernatorial priorities, and legislative coalitions---rather than by trends in business formation. A new governor from the opposing party, a shift in legislative control, or a ballot initiative can accelerate or delay minimum wage changes in ways that are plausibly orthogonal to underlying business cycle conditions.

\subsection{Minimum Wage Increases During the Study Period}

Between 2012 and 2021, minimum wage policy became increasingly dynamic. The ``Fight for \$15'' movement, which began in 2012 with fast-food worker strikes, galvanized political momentum for substantial increases. By 2016, California and New York had enacted legislation phasing in \$15 minimums, and several cities followed with their own local increases.

Most state increases during this period were gradual and predictable. Typical legislated increases were \$0.50 to \$1.00 per year, often announced years in advance. California's path to \$15, for example, was legislated in 2016 with implementation phased through 2022 for large employers. This predictability is important because it means entrepreneurs contemplating entry had substantial lead time to incorporate expected wage changes into their business plans---potentially dampening any contemporaneous response to actual implementation.


\section{Data}

\subsection{Business Establishment Counts}

The primary outcome variable comes from the Census Bureau's County Business Patterns (CBP) program. CBP provides annual counts of business establishments by geographic area, including state-level totals. An establishment is defined as a single physical location where business is conducted; multi-location firms contribute multiple establishments to the count.

CBP covers establishments with paid employees subject to the Federal Insurance Contributions Act (FICA) taxes, which captures the vast majority of formal business activity. The data exclude self-employed individuals without employees, agricultural production, most government establishments, and private households. For the purposes of this study, CBP provides a comprehensive measure of formal business presence at the state level.

The sample covers all 51 jurisdictions (50 states plus DC) from 2012 to 2021, yielding a balanced panel of 510 state-year observations. One limitation of establishment counts is that they measure the stock rather than flow of businesses. The analysis therefore captures net effects on formal business activity---the combined result of entries and exits---rather than gross entry alone. This is appropriate for assessing overall policy effects but may mask offsetting dynamics.

\subsection{Minimum Wage Data}

State minimum wage data are compiled from Department of Labor records and state labor department publications. For each state-year, I compute the annual average effective minimum wage, defined as the higher of the state and federal rate. This continuous measure captures both whether a state is above the federal floor and by how much.

Real minimum wages are deflated to 2020 dollars using the Consumer Price Index for All Urban Consumers (CPI-U), following standard practice in the minimum wage literature. The resulting variable ranges from \$7.10 to \$13.70 in real 2020 dollars across the sample.

For binary treatment specifications, I construct an indicator for whether the state's effective minimum wage exceeds the federal floor of \$7.25. This classification identifies states with any binding state-level minimum wage policy.

\subsection{Treatment Timing}

Identifying treatment timing is crucial for the event study and Callaway-Sant'Anna analyses. I define a state as ``treated'' in the first year its effective minimum wage exceeds the federal floor. Of the 29 treated jurisdictions, 13 were already above the federal minimum at the sample start in 2012 (``early adopters''), while 16 crossed the federal threshold at some point during the sample period (``within-sample adopters'').

For the Callaway-Sant'Anna estimator and event study, I exclude the 13 early adopters because they have no pre-treatment observations in the sample. This restriction leaves 38 jurisdictions for these analyses: 22 never-treated and 16 with within-sample adoption. The exclusion is necessary for credible identification---without pre-treatment data, one cannot test for parallel trends or properly implement the cohort-based estimator.

\subsection{Sample Construction}

The final analysis sample is constructed by merging the CBP establishment data with the minimum wage data at the state-year level. Several sample restrictions are applied to ensure data quality and consistency across the panel.

First, I restrict the sample to the 50 states plus the District of Columbia, excluding U.S. territories (Puerto Rico, Guam, Virgin Islands, American Samoa, Northern Mariana Islands). These territories have distinct labor market institutions and minimum wage policies that make them unsuitable for comparison with the mainland states.

Second, I restrict the time period to 2012--2021, a decade-long span that provides sufficient pre-period for most within-sample adopters while avoiding the COVID-19 pandemic recovery period when business dynamics were highly abnormal. The sample begins in 2012 to ensure coverage of the ``Fight for \$15'' movement's emergence and the subsequent wave of state-level minimum wage increases.

Third, I exclude any state-year observations with missing data on either establishment counts or minimum wage levels. In practice, both data sources provide complete coverage for all 51 jurisdictions across all 10 years, yielding a balanced panel of 510 observations.

\subsection{Variation in Treatment}

The staggered adoption of above-federal minimum wages generates substantial variation for identification. Figure \ref{fig:variation} (available in the online appendix) illustrates the geographic and temporal distribution of treatment. States are classified into three groups based on their treatment status throughout the sample period.

The 13 early-adopter states---California, Colorado, Connecticut, Illinois, Massachusetts, Maine, Michigan, New Mexico, Ohio, Oregon, Rhode Island, Vermont, and Washington---had minimum wages above the federal floor before the sample period began and maintained them throughout. These states provide no within-sample variation in treatment status and are excluded from the event study and Callaway-Sant'Anna analyses, though they contribute to the TWFE estimates.

The 16 within-sample adopters crossed above the federal threshold at various points during 2012--2021. Adoption years range from 2014 (Delaware, Minnesota, New Jersey, New York) to 2020 (Nevada). This staggered timing provides the identifying variation for the event study and heterogeneity-robust estimators.

The 22 never-treated states remained at exactly the federal minimum of \$7.25 throughout the sample period. These states serve as the comparison group for the Callaway-Sant'Anna estimator and provide a stable control for the event study analysis.

\subsection{Summary Statistics}

Table \ref{tab:summary} presents summary statistics for the analysis sample. Mean annual establishments per state are approximately 152,500, with substantial variation (SD = 170,800) reflecting the wide range in state sizes from Wyoming (approximately 20,000 establishments) to California (approximately 1 million). The annual average effective minimum wage is \$8.17 nominally, or \$8.69 in 2020 dollars, with a range from the federal floor of \$7.25 to \$14.00 in high-wage states.

Approximately 45 percent of state-year observations have effective minimum wages above the federal floor. This relatively balanced split between treated and untreated observations provides good statistical power for detecting effects. The balance is somewhat skewed toward treated observations in later years, as more states crossed the federal threshold over time, but the comparison group of never-treated states provides a stable baseline throughout.

The distribution of minimum wages conditional on being above federal shows substantial variation. Among above-federal observations, the mean minimum wage is \$9.42 nominally, with a range from just above \$7.25 (states that barely exceeded the federal floor) to over \$14.00 (high-cost-of-living states like California and Washington). This variation in treatment intensity provides additional identifying information beyond the simple binary above/below classification.

\begin{table}[H]
\centering
\caption{Summary Statistics}
\begin{threeparttable}
\begin{tabular}{lcccc}
\toprule
Variable & Mean & Std. Dev. & Min & Max \\
\midrule
Establishments (annual) & 152,521 & 170,802 & 20,426 & 998,578 \\
Effective MW (Nominal \$) & 8.17 & 1.42 & 7.25 & 14.00 \\
Real MW (2020 \$) & 8.69 & 1.32 & 7.10 & 13.70 \\
Above Federal (indicator) & 0.451 & 0.498 & 0 & 1 \\
\bottomrule
\end{tabular}
\begin{tablenotes}[flushleft]
\small
\item Notes: N = 510 state-year observations (51 jurisdictions $\times$ 10 years). Establishment counts from Census Bureau County Business Patterns. Real minimum wage deflated using CPI-U (2020 = 100). ``Above Federal'' equals one if the state's effective minimum wage exceeds \$7.25.
\end{tablenotes}
\end{threeparttable}
\label{tab:summary}
\end{table}


\section{Empirical Strategy}

\subsection{Two-Way Fixed Effects Estimation}

The baseline empirical specification employs a standard two-way fixed effects (TWFE) regression:
\begin{equation}
\log(Y_{st}) = \alpha + \beta \log(MW_{st}) + \gamma_s + \delta_t + \varepsilon_{st}
\end{equation}
where $Y_{st}$ is the establishment count in state $s$ and year $t$, $MW_{st}$ is the annual average real minimum wage, $\gamma_s$ are state fixed effects, and $\delta_t$ are year fixed effects. The log-log specification means that the coefficient $\beta$ is interpretable as an elasticity: the percent change in establishments associated with a 1 percent increase in the minimum wage.

State fixed effects absorb all time-invariant differences across states---their industry composition, regulatory environment, political culture, and other factors that might correlate with both minimum wage policy and business formation. Year fixed effects absorb national trends affecting all states simultaneously, including business cycle effects and changes in federal policy.

Standard errors are clustered at the state level throughout to account for serial correlation within states and the state-level nature of the policy treatment. With 51 clusters, inference should be reliable, though I also report results excluding large states that might drive the findings.

\subsection{Callaway-Sant'Anna Estimator}

Recent econometric research has highlighted potential biases in TWFE estimation under staggered adoption with heterogeneous treatment effects \citep{goodman2021difference, sun2021estimating, roth2023s}. The concern is that TWFE estimates can be contaminated by ``forbidden comparisons'' using already-treated units as controls, potentially yielding misleading estimates even when parallel trends hold.

To address this concern, I implement the group-time average treatment effect estimator of \citet{callaway2021difference}. This approach defines treatment effect parameters for each cohort (states first treated in year $g$) at each time period $t$, computing:
\begin{equation}
ATT(g,t) = \E[Y_t - Y_{g-1} | G = g] - \E[Y_t - Y_{g-1} | C = 1]
\end{equation}
where $G = g$ indicates cohort $g$ and $C = 1$ indicates the never-treated comparison group. The estimator avoids using already-treated units as controls, ensuring clean identification.

I use the never-treated states as the comparison group, which is appropriate when there are sufficient never-treated units (22 in my sample). The group-time ATTs are then aggregated to an overall ATT using appropriate weights. I also report dynamic treatment effects by time relative to treatment, providing event-study-style evidence on pre-trends and dynamic responses.

For the Callaway-Sant'Anna analysis, I necessarily exclude the 13 states already treated at sample start, as these lack any pre-treatment observations. This leaves 38 states: 22 never-treated and 16 with within-sample adoption.

\subsection{Event Study Specification}

The event study approach estimates treatment effects for each period relative to adoption:
\begin{equation}
\log(Y_{st}) = \sum_{k \neq -1} \beta_k \cdot \ind[\text{RelYear}_{st} = k] + \gamma_s + \delta_t + \varepsilon_{st}
\end{equation}
where $\text{RelYear}_{st}$ is the number of years relative to first treatment for state $s$. The coefficients $\beta_k$ trace out the dynamic treatment effect profile. The period immediately before treatment ($k = -1$) is the omitted reference category.

Pre-treatment coefficients ($\beta_{-2}, \beta_{-3}$, etc.) serve as a test of the parallel trends assumption. Under parallel trends, these should be zero: treated states should not be systematically different from control states before treatment occurs. Post-treatment coefficients ($\beta_0, \beta_1$, etc.) capture the dynamic treatment effect.

Event time is binned at $\pm 5$ years to maintain sample size in each bin. The event study sample excludes already-treated states, consistent with the Callaway-Sant'Anna analysis.

\subsection{Identification Assumptions and Threats}

The key identifying assumption is parallel trends: in the absence of treatment, treated and control states would have followed similar establishment trajectories. Several considerations support this assumption in the present context.

First, the event study provides direct evidence. Finding null pre-treatment coefficients suggests that treated states were not on different trajectories before their minimum wage increases.

Second, the timing of minimum wage increases appears largely determined by political factors---election outcomes, legislative priorities, and ballot initiatives---rather than by trends in business formation. A state does not typically raise its minimum wage because it expects business formation to increase; rather, it does so due to political pressures and ideological commitments.

Third, the robustness of results to including state-specific linear trends suggests that differential trends are not driving the findings. If treated states were on systematically different trajectories, controlling for trends would substantially change the estimates.

Several potential threats deserve mention. Anticipation effects are one concern: since minimum wage increases are typically announced well in advance, entrepreneurs might adjust before formal implementation. This would attenuate measured effects around the treatment date but should not bias the overall finding of null effects. Concurrent policy changes are another concern: states that raise minimum wages might simultaneously enact other business-relevant policies. I cannot fully rule this out but note that the variety of states in the treatment group---from liberal California to moderate Florida---makes systematic confounding less likely.


\section{Results}

\subsection{Main TWFE Results}

Table \ref{tab:main} reports the main TWFE estimates. Column (1) presents the baseline specification with state and year fixed effects. The elasticity of establishment counts with respect to the real minimum wage is $-0.018$ (SE = 0.036). This small negative coefficient implies that a 10 percent increase in the real minimum wage is associated with a $10 \times 0.018 = 0.18$ percent decrease in establishment counts---an economically negligible effect.

The estimate is also statistically insignificant, with a t-statistic of $-0.5$ and a p-value well above conventional thresholds. The 95 percent confidence interval, approximately ($-0.089$, $0.053$), allows us to rule out elasticities larger than about 0.09 in absolute value. For a 10 percent minimum wage increase, this rules out effects on establishments larger than about 0.9 percent in magnitude.

Column (2) adds state-specific linear time trends to the baseline specification. This controls for any differential trends in business formation across states that might confound the estimate. The point estimate attenuates to 0.004 (SE = 0.014)---effectively zero. The tighter standard error reflects the additional structure imposed by the trends, but the substantive conclusion is unchanged: no detectable effect.

Columns (3) and (4) use a binary treatment indicator (Above Federal) rather than the continuous log minimum wage. Results are similarly null: the coefficient in Column (3) is 0.0003 (SE = 0.011), and with state trends in Column (4) it is $-0.0004$ (SE = 0.005). These binary specifications confirm that simply being above the federal floor, regardless of by how much, has no detectable effect on establishment counts.

\begin{table}[H]
\centering
\caption{Effect of Minimum Wage on Business Establishments}
\begin{threeparttable}
\begin{tabular}{lcccc}
\toprule
& (1) & (2) & (3) & (4) \\
& Log(Est) & Log(Est) & Log(Est) & Log(Est) \\
\midrule
Log(Real MW) & $-0.018$ & 0.004 & & \\
             & (0.036) & (0.014) & & \\
Above Federal & & & 0.0003 & $-0.0004$ \\
              & & & (0.011) & (0.005) \\
\midrule
State FE & Yes & Yes & Yes & Yes \\
Year FE & Yes & Yes & Yes & Yes \\
State Trends & No & Yes & No & Yes \\
N & 510 & 510 & 510 & 510 \\
\bottomrule
\end{tabular}
\begin{tablenotes}[flushleft]
\small
\item Notes: * p$<$0.10, ** p$<$0.05, *** p$<$0.01. Standard errors clustered at state level in parentheses. Dependent variable is log of annual establishment count from CBP. Coefficient in Columns (1)--(2) is an elasticity: a 10\% MW increase implies $\approx$0.18\% change in establishments.
\end{tablenotes}
\end{threeparttable}
\label{tab:main}
\end{table}

\subsection{Callaway-Sant'Anna Estimates}

The heterogeneity-robust Callaway-Sant'Anna estimator yields results consistent with the TWFE findings. The aggregate ATT is $-0.013$ (SE = 0.012), indicating that states that raised their minimum wages above the federal floor experienced, on average, 1.3 percent lower log establishment counts---again statistically insignificant and economically small.

The 95 percent confidence interval for the ATT is ($-0.036$, 0.010). This rules out large negative effects (greater than 3.6 percent) at conventional significance levels. The ATT is identified using clean comparisons between newly-treated cohorts and never-treated states, avoiding the contamination concerns that arise in standard TWFE with staggered adoption.

\subsection{Event Study Evidence}

Figure \ref{fig:event} displays the event study estimates from the Callaway-Sant'Anna framework. The figure plots treatment effects by years relative to adoption, with the year immediately before treatment ($t = -1$) as the reference period.

The pre-treatment coefficients provide a key test of parallel trends. For $t = -2$ through $t = -5$, all coefficients are small (less than 0.02 in absolute value) and statistically insignificant. The 95 percent confidence intervals comfortably include zero. This pattern supports the parallel trends assumption: treated states were not on systematically different trajectories before their minimum wage increases.

Post-treatment coefficients ($t = 0$ through $t = 5$) remain flat and centered near zero. There is no evidence of immediate effects at adoption, delayed effects emerging over time, or cumulating effects as the treatment duration lengthens. The null effect appears to be stable across the post-treatment horizon.

\begin{figure}[H]
\centering
\includegraphics[width=0.85\textwidth]{figures/fig3_event_study.pdf}
\caption{Event Study: Effect of Above-Federal Minimum Wage on Establishments}
\label{fig:event}
\begin{figurenotes}
Notes: Callaway-Sant'Anna estimates. Sample excludes 13 states already above federal in 2012. Reference period is $t = -1$ (year before treatment). Shaded region shows 95\% confidence intervals. Coefficients represent effect on log establishment counts relative to never-treated states.
\end{figurenotes}
\end{figure}

\subsection{Robustness Checks}

Table \ref{tab:robust} presents a series of robustness checks. The main estimate is stable across specifications.

\textit{Excluding large states.} California, New York, and Texas together account for a substantial share of U.S. establishments and minimum wage variation. Excluding these three states yields an estimate of $-0.016$ (SE = 0.037), virtually unchanged from the baseline.

\textit{State-specific trends.} Adding state-specific linear trends yields a point estimate of 0.004 (SE = 0.014). The attenuation toward zero and tighter standard error suggest that any residual concern about differential trends does not drive the null finding.

\textit{Binary treatment.} Using an indicator for above-federal minimum wage yields a coefficient of 0.0003 (SE = 0.011), confirming null effects under an alternative treatment definition.

\begin{table}[H]
\centering
\caption{Robustness Checks}
\begin{threeparttable}
\begin{tabular}{lccc}
\toprule
Specification & Coefficient & SE & N \\
\midrule
Main (TWFE) & $-0.018$ & (0.036) & 510 \\
Exclude CA, NY, TX & $-0.016$ & (0.037) & 480 \\
State-Specific Trends & 0.004 & (0.014) & 510 \\
Binary Treatment & 0.0003 & (0.011) & 510 \\
\bottomrule
\end{tabular}
\begin{tablenotes}[flushleft]
\small
\item Notes: * p$<$0.10, ** p$<$0.05, *** p$<$0.01. Standard errors clustered at state level. All specifications include state and year fixed effects.
\end{tablenotes}
\end{threeparttable}
\label{tab:robust}
\end{table}

\subsection{Goodman-Bacon Decomposition}

To understand the sources of variation driving the TWFE estimate, I implement the Goodman-Bacon decomposition \citep{goodman2021difference}. This decomposition reveals how much weight different 2x2 difference-in-differences comparisons receive in the overall TWFE estimate.

The results show that 82 percent of the weight comes from treated-versus-never-treated comparisons---the ``clean'' comparisons that use never-treated states as controls. Early-versus-late treated comparisons receive 8 percent of the weight, and late-versus-early comparisons receive 10 percent. The dominance of clean comparisons explains why the TWFE and Callaway-Sant'Anna estimates are so similar: the TWFE is not substantially contaminated by problematic comparisons using already-treated units as controls.


\section{Discussion}

\subsection{Interpreting the Null Finding}

The consistent null findings across specifications and estimators suggest that minimum wage increases---at the levels observed during 2012--2021---do not detectably affect aggregate business formation. This conclusion deserves careful interpretation along several dimensions.

First, a null effect on establishment counts does not imply no effects on any margin. Entrepreneurs might respond to minimum wage increases along other margins---reducing planned employment, raising prices, choosing different locations, or accepting lower profits---while still opening businesses. The analysis captures only the extensive margin of whether a business exists, not the intensive margin of how it operates. A complete assessment of minimum wage effects on entrepreneurship would require examining employment per establishment, revenue, profitability, and survival rates in addition to entry.

Second, the policy variation may not be large enough to detect effects. The minimum wage increases during this period, while politically contentious, were relatively modest in economic terms. A typical increase of \$1--\$2 above the federal floor represents a 15--30 percent increase in mandated wages, but wages are only one component of business costs. If labor costs account for 20--30 percent of total costs for a typical new business, even a 30 percent wage increase translates to only a 6--9 percent increase in total costs---potentially too small to deter many marginal entrants.

Third, the null finding applies to aggregate establishment counts but may mask heterogeneity across industries or establishment types. Minimum-wage-intensive sectors like food service and retail might show effects that are diluted when combined with sectors like finance and professional services where minimum wages rarely bind. Future research should examine industry-specific effects using detailed NAICS codes to test whether the aggregate null masks meaningful sectoral heterogeneity.

Fourth, the outcome variable measures the stock of establishments rather than gross entry or exit flows. A null effect on the stock could result from offsetting effects on entry and exit---for example, if minimum wages deter some entries while simultaneously reducing exits by increasing worker retention and reducing turnover costs. Disentangling these flows would require establishment-level panel data with birth and death indicators, such as the Business Dynamics Statistics.

\subsection{Theoretical Framework for the Null}

Standard economic theory offers several mechanisms through which minimum wage increases might affect business formation. On the negative side, higher mandated wages increase labor costs, potentially pushing some business models below the profitability threshold. This is particularly relevant for labor-intensive, low-margin industries where wages constitute a large share of operating costs. Entrepreneurs evaluating whether to enter such markets must consider whether they can profitably operate while paying the mandated wage.

However, several countervailing forces might offset these negative effects, helping to explain the null finding. First, minimum wage increases may improve worker productivity through efficiency wage mechanisms. Higher wages attract better workers, reduce turnover, and increase effort---all of which can partially offset the direct cost increase. If productivity gains are substantial, the net effect on profitability (and hence entry incentives) may be small.

Second, minimum wage increases may be largely passed through to consumers. In markets with inelastic demand, firms can raise prices to offset higher labor costs without losing substantial sales volume. Empirical evidence suggests substantial price pass-through in sectors like restaurants, where estimated pass-through rates often exceed 100 percent of the labor cost increase. If pass-through is complete, minimum wage increases need not affect profitability or entry incentives.

Third, minimum wage increases may serve as a quality signal, attracting entrepreneurs who intend to compete on service quality rather than low prices. In this view, minimum wage policy acts as a barrier to entry for low-road business models while leaving high-road entrants unaffected. The net effect on establishment counts depends on the distribution of potential entrants across business model types.

Fourth, entrepreneurs may have limited information about minimum wage levels and may not incorporate them into entry decisions. Survey evidence suggests that many small business owners have imprecise knowledge of applicable wage floors, particularly when state and local rates differ from the federal minimum. If entrepreneurs do not accurately forecast labor costs, minimum wage changes may have smaller effects on entry than rational models predict.

\subsection{Mechanisms}

Several mechanisms could explain why minimum wage increases have null effects on establishment counts, even if standard theory predicts negative effects. I discuss each in turn.

\textit{Minimum wages as a small share of startup costs.} For many businesses, labor costs during the startup phase are modest relative to capital requirements, real estate, inventory, and other fixed costs. A new restaurant may face \$100,000 or more in build-out costs before opening, making the \$1--\$2 per hour minimum wage difference a second-order consideration. Entrepreneurs may be more sensitive to interest rates, commercial rents, and regulatory compliance costs than to the minimum wage. This is particularly true for capital-intensive businesses or those in high-rent locations.

\textit{Adjustment along other margins.} Businesses facing higher wage floors may adjust employment levels, hours, automation investments, or output prices rather than deciding not to enter. These adjustments allow entry to remain viable even when one input cost rises. A prospective restaurant owner might plan to hire fewer workers, use more self-service technology, or adjust the menu and pricing---all without abandoning the entry decision entirely. The flexibility of business models in response to input costs may preserve entry incentives even when individual cost components increase.

\textit{Worker quality improvements.} Higher wages may attract better workers, reducing training costs and turnover. If efficiency wages operate, the higher labor cost may be partially offset by productivity gains, making entry calculations less sensitive to the wage floor than naive models suggest. This mechanism is particularly relevant in sectors with high turnover, where the costs of recruiting, training, and losing workers can be substantial. Entrepreneurs anticipating these gains may view higher minimum wages as less costly than the statutory increase implies.

\textit{Pass-through to consumers.} In markets with inelastic demand or limited competition, businesses may pass higher labor costs through to prices with little effect on profitability. Fast food, for example, has shown substantial price pass-through in response to minimum wage increases \citep{harasztosi2019labor}, preserving margins for incumbents and entrants alike. If entrepreneurs anticipate that they can raise prices without losing customers, the deterrent effect of higher wages on entry is correspondingly reduced.

\textit{Selection effects.} Minimum wage increases might deter some potential entrants while encouraging others. If higher wages signal a favorable business environment---characterized by high consumer demand, quality labor supply, and economic vitality---some entrepreneurs might be drawn to high-minimum-wage states even as others are deterred. States that raise their minimum wages may be precisely those with strong economies and growing consumer bases, making them attractive entry locations despite higher labor costs. The net effect on entry could be close to zero if these offsetting selection forces roughly balance.

\textit{Geographic substitution.} Entrepreneurs deterred by higher minimum wages in one state might enter in a neighboring lower-wage state instead, rather than abandoning entry entirely. This spatial substitution would reduce measured effects within treated states while potentially increasing entry in control states. To the extent that border effects or cross-state competition matter, state-level estimates may understate the effect on local entry while accurately capturing the aggregate (null) effect.

\subsection{Policy Implications}

The null findings have several implications for the policy debate over minimum wage increases. Most directly, they suggest that concerns about minimum wages deterring business formation find limited empirical support---at least for the policy levels observed during 2012--2021 and for aggregate establishment counts. This is one margin of potential adjustment that does not appear to respond to the minimum wage policies implemented during this period.

However, several caveats limit the policy conclusions that can be drawn. First, the study period preceded the full implementation of \$15 minimum wage policies in California, New York, and other jurisdictions. Larger minimum wage increases might eventually affect business formation even if the moderate increases studied here do not. The relationship between minimum wage levels and business formation need not be linear, and threshold effects might emerge at higher wage levels.

Second, the null finding on aggregate establishments does not preclude effects on business composition. Minimum wage increases might shift entry toward capital-intensive or high-productivity business models while deterring labor-intensive, low-margin operations. Such compositional shifts would not appear as changes in total establishment counts but could have meaningful implications for employment, wages, and economic welfare.

Third, local minimum wage policies---increasingly common in major cities---may have different effects than state-level policies. Local minimum wages create sharper geographic discontinuities and may generate more pronounced entry effects at city borders. The state-level analysis here cannot speak to these local policy effects.

For policymakers, the findings suggest that minimum wage policy can be evaluated primarily on its labor market effects---wages, employment, and worker welfare---without substantial concern that moderate increases will dramatically reduce business formation. The extensive margin of firm entry appears to be relatively insensitive to minimum wage policy within the observed range, leaving the intensive margins (employment per establishment, hours, wages) as the primary channels of policy impact.

\subsection{Limitations}

Several limitations of this analysis deserve acknowledgment. These limitations should inform both the interpretation of the findings and the design of future research on this question.

\textit{Aggregate outcomes.} Establishment counts at the state-year level represent a highly aggregated outcome. Effects in specific industries, especially minimum-wage-intensive sectors like food service and retail, might be masked when combined with industries where minimum wages rarely bind, such as finance, technology, and professional services. Ideally, one would analyze industry-specific establishment counts using detailed NAICS codes, though data limitations and sample size concerns constrain this approach at the state level.

\textit{Net versus gross formation.} CBP establishment counts measure the stock of businesses, reflecting both entry and exit. If minimum wages simultaneously deter some entries and cause some exits, the effects would cumulate in the same direction. However, if they deter entries while reducing exits (through, say, reduced competition or improved worker retention), effects might offset. Disentangling entry from exit requires different data sources, such as the Census Bureau's Business Dynamics Statistics, which provides establishment birth and death counts by state, industry, and firm age.

\textit{Informal businesses excluded.} CBP covers only establishments with paid employees subject to FICA taxes. Informal businesses, self-employed individuals without employees, and ``gig'' workers are excluded. If minimum wage increases push economic activity from the formal to informal sector---for example, by encouraging self-employment or gig work as alternatives to low-wage employment---this shift would not appear as reduced establishment counts. To the contrary, it might appear as a reduction in formal establishments even if overall entrepreneurial activity is unchanged.

\textit{Geographic aggregation.} State-level analysis ignores substantial within-state variation. Effects might differ across urban and rural areas, border regions, or local labor markets. Metropolitan areas with local minimum wage ordinances (common in California, Washington, and elsewhere) introduce additional complexity not captured by state-level measures. A more refined analysis would use county-level data with geographic fixed effects for contiguous cross-border county pairs, following the influential approach of \citet{dube2019minimum}. Such a design would better control for unobserved regional factors while exploiting sharper policy discontinuities.

\textit{Anticipation and dynamics.} Since minimum wage increases are typically announced years in advance, entrepreneurs may adjust before formal implementation. California's path to \$15, for example, was legislated in 2016 for phase-in through 2022, giving potential entrants six years to incorporate the planned increases into their decisions. The event study finds no pre-treatment effects within the five-year window examined, but anticipatory responses occurring earlier---or coinciding with legislative announcements rather than implementation dates---would not be captured by the implementation-based event timing used here.

\textit{External validity.} The findings apply to the specific policy variation observed during 2012--2021, when minimum wage increases were typically modest (\$1--\$2 above the federal floor). The results may not generalize to larger increases, such as the \$15 minimum wage policies being phased in during the 2020s. The relationship between minimum wages and business formation could be nonlinear, with threshold effects emerging only at higher wage levels. The findings also pertain specifically to states that chose to raise their minimum wages during this period---a selected sample that may differ systematically from states that did not.

\textit{Measurement of treatment intensity.} The analysis uses a binary indicator (above federal vs. at federal) for the Callaway-Sant'Anna estimator and log real minimum wage for the TWFE specification. Neither approach fully captures the intensity of treatment, which depends not just on the minimum wage level but also on the share of workers affected, the distribution of wages in the state, and the cost of living. More sophisticated measures of ``bindingness''---such as the Kaitz index (minimum wage relative to median wage) or the share of workers earning at or below the minimum---might reveal effects not apparent with cruder treatment measures.


\section{Conclusion}

This paper exploits staggered minimum wage increases across U.S. states to estimate effects on business establishment counts. Using Census Bureau County Business Patterns data and modern difference-in-differences methods that address heterogeneous treatment effects, I find precisely estimated null effects: a 10 percent minimum wage increase is associated with a 0.18 percent decrease in establishments, an effect that is economically trivial and statistically insignificant.

The main empirical finding---an elasticity of $-0.018$ with standard error 0.036---implies that minimum wage policy has no detectable effect on the extensive margin of business creation at the state-year level. The 95 percent confidence interval allows us to rule out elasticities larger than about 0.09 in absolute value, corresponding to effects on establishments of less than 1 percent for a 10 percent minimum wage increase.

The null finding is robust across specifications. TWFE and Callaway-Sant'Anna estimators yield similar results, reflecting the dominance of clean treated-versus-never-treated comparisons in the TWFE decomposition. Estimates are stable when excluding large states (California, New York, Texas), adding state-specific linear trends, or using alternative treatment definitions (binary above-federal indicator versus continuous log real minimum wage). Event study evidence supports the parallel trends assumption, with pre-treatment coefficients indistinguishable from zero for years $t = -2$ through $t = -5$ relative to treatment.

\subsection{Contributions}

This paper makes several contributions to the literature. First, it provides direct evidence on the minimum wage--business formation nexus, an understudied margin in the otherwise extensive minimum wage literature. While employment effects have received enormous attention, the extensive margin of firm entry has been largely ignored despite its potential importance for long-run job creation and economic dynamism.

Second, the paper demonstrates the application of modern difference-in-differences techniques to minimum wage policy evaluation. By implementing both traditional TWFE and the heterogeneity-robust Callaway-Sant'Anna estimator, the analysis addresses concerns about bias from staggered adoption that have received considerable recent attention. The consistency of results across estimators provides reassurance that the null finding is not an artifact of method choice.

Third, the paper documents the extent of minimum wage policy variation across U.S. states during the 2012--2021 period and provides a framework for studying this variation. The classification of states into early adopters, within-sample adopters, and never-treated provides a template for future research on minimum wage policy effects using staggered adoption designs.

\subsection{Implications for Policy}

The findings have several implications for the policy debate. Most directly, they suggest that concerns about minimum wages deterring entrepreneurship and business formation are not supported by evidence from the moderate policy changes observed during 2012--2021. Policymakers considering minimum wage increases need not anticipate substantial reductions in business entry as a first-order consequence, at least for increases in the range examined.

However, the findings should be interpreted cautiously. The null applies to aggregate establishment counts at the state level, not to specific industries or localities. It applies to the stock of establishments, not to gross entry flows. And it applies to the moderate minimum wage increases that prevailed during the study period, not necessarily to the more dramatic \$15 policies now being implemented.

The most defensible policy conclusion is narrow: within this design and data, minimum wage increases of the magnitude observed during 2012--2021 do not detectably reduce state-level establishment counts. This leaves open the possibility of effects on other margins, in other settings, or at other policy levels.

\subsection{Directions for Future Research}

Several extensions would strengthen the evidence base on minimum wage effects on business formation. First, future work should examine entry and exit flows separately using Business Dynamics Statistics data, rather than relying solely on establishment stocks. This would allow researchers to determine whether the null effect on stocks reflects genuinely no effect or offsetting effects on entry and exit.

Second, industry-specific analyses would help determine whether aggregate nulls mask meaningful sectoral heterogeneity. Food service, accommodation, and retail---where minimum wages are most likely to bind---warrant particular attention. If effects are concentrated in these sectors but diluted by aggregation, industry-level analysis would reveal patterns that state-level totals conceal.

Third, local minimum wage policies merit separate study. The proliferation of city-level minimum wage ordinances creates sharp geographic discontinuities that may be more suitable for identification than state-level variation. Border-pair designs comparing establishments on opposite sides of city boundaries could provide compelling evidence on local entry effects.

Fourth, the phase-in of \$15 minimum wage policies during the 2020s will provide an opportunity to study larger minimum wage changes than those observed in the present sample. If effects emerge only at higher wage levels, the null findings here would not preclude meaningful effects of more substantial policy changes.

Finally, complementary evidence from surveys of entrepreneurs and small business owners could illuminate the mechanisms underlying the null finding. Understanding why entrepreneurs appear unresponsive to minimum wage policy---whether due to inattention, offsetting adjustments, or other factors---would inform both academic understanding and policy design.

In sum, this paper finds that minimum wage increases during 2012--2021 had no detectable effect on state-level business establishment counts. While important caveats apply, the finding suggests that the extensive margin of firm entry is relatively insensitive to moderate minimum wage changes. Future research should examine gross entry flows, industry heterogeneity, and the effects of larger minimum wage increases to provide a more complete picture of how minimum wage policy affects business formation.


\section*{Acknowledgements}

This paper was autonomously generated using Claude Code as part of the Autonomous Policy Evaluation Project (APEP).

\noindent\textbf{Repository:} \url{https://github.com/SocialCatalystLab/auto-policy-evals}

\label{apep_main_text_end}
\newpage

\begin{thebibliography}{99}

\bibitem[Autor, Manning, and Smith(2016)]{autor2016contribution}
Autor, D.H., Manning, A. and Smith, C.L., 2016. The contribution of the minimum wage to US wage inequality over three decades: A reassessment. \textit{American Economic Journal: Applied Economics}, 8(1), pp.58--99.

\bibitem[Callaway and Sant'Anna(2021)]{callaway2021difference}
Callaway, B. and Sant'Anna, P.H., 2021. Difference-in-differences with multiple time periods. \textit{Journal of Econometrics}, 225(2), pp.200--230.

\bibitem[Card and Krueger(1994)]{card1994minimum}
Card, D. and Krueger, A.B., 1994. Minimum wages and employment: A case study of the fast-food industry in New Jersey and Pennsylvania. \textit{American Economic Review}, 84(4), pp.772--793.

\bibitem[Cengiz et al.(2019)]{cengiz2019effect}
Cengiz, D., Dube, A., Lindner, A. and Zipperer, B., 2019. The effect of minimum wages on low-wage jobs. \textit{Quarterly Journal of Economics}, 134(3), pp.1405--1454.

\bibitem[Decker et al.(2014)]{decker2014role}
Decker, R., Haltiwanger, J., Jarmin, R. and Miranda, J., 2014. The role of entrepreneurship in US job creation and economic dynamism. \textit{Journal of Economic Perspectives}, 28(3), pp.3--24.

\bibitem[Dube(2019)]{dube2019minimum}
Dube, A., 2019. Impacts of minimum wages: Review of the international evidence. HM Treasury, UK Government.

\bibitem[Fairlie and Fossen(2018)]{fairlie2018entrepreneurship}
Fairlie, R.W. and Fossen, F.M., 2018. Opportunity versus necessity entrepreneurship: Two components of business creation. \textit{IZA Discussion Papers}, No. 11258.

\bibitem[Goodman-Bacon(2021)]{goodman2021difference}
Goodman-Bacon, A., 2021. Difference-in-differences with variation in treatment timing. \textit{Journal of Econometrics}, 225(2), pp.254--277.

\bibitem[Haltiwanger, Jarmin, and Miranda(2013)]{haltiwanger2013creates}
Haltiwanger, J., Jarmin, R.S. and Miranda, J., 2013. Who creates jobs? Small versus large versus young. \textit{Review of Economics and Statistics}, 95(2), pp.347--361.

\bibitem[Harasztosi and Lindner(2019)]{harasztosi2019labor}
Harasztosi, P. and Lindner, A., 2019. Who pays for the minimum wage? \textit{American Economic Review}, 109(8), pp.2693--2727.

\bibitem[Hurst and Lusardi(2004)]{hurst2004entrepreneurship}
Hurst, E. and Lusardi, A., 2004. Liquidity constraints, household wealth, and entrepreneurship. \textit{Journal of Political Economy}, 112(2), pp.319--347.

\bibitem[Neumark and Wascher(2008)]{neumark2008minimum}
Neumark, D. and Wascher, W.L., 2008. \textit{Minimum Wages}. MIT Press.

\bibitem[Roth et al.(2023)]{roth2023s}
Roth, J., Sant'Anna, P.H., Bilinski, A. and Poe, J., 2023. What's trending in difference-in-differences? A synthesis of the recent econometrics literature. \textit{Journal of Econometrics}, 235(2), pp.2218--2244.

\bibitem[Sun and Abraham(2021)]{sun2021estimating}
Sun, L. and Abraham, S., 2021. Estimating dynamic treatment effects in event studies with heterogeneous treatment effects. \textit{Journal of Econometrics}, 225(2), pp.175--199.

\end{thebibliography}

\newpage
\appendix

\section{Data Appendix}

\subsection{County Business Patterns}

Business establishment counts come from the Census Bureau's County Business Patterns (CBP) program via the Census API. CBP provides annual counts of establishments with paid employees, classified by geography and industry. For this study, I use state-level total establishment counts.

The data cover 2012--2021, providing 10 years of observations. All 50 states plus the District of Columbia are included, yielding 510 state-year observations. Establishment counts range from approximately 20,000 (Wyoming) to nearly 1 million (California).

\subsection{Minimum Wage Data}

State minimum wage data are compiled from the U.S. Department of Labor Wage and Hour Division and state labor department records. For each state-year, I record the state's minimum wage rate as of January 1, taking the annual average when rates changed mid-year.

The effective minimum wage is the higher of the state rate and federal rate (\$7.25). Real wages are computed by deflating nominal wages using the CPI-U with 2020 as the base year (2020 = 100).

\subsection{Treatment Timing}

Table \ref{tab:treatment} reports treatment timing for states with minimum wages above the federal floor at any point during the sample.

Of 51 jurisdictions:
\begin{itemize}
\item 29 ever above federal
\item 13 already above federal in 2012 (excluded from event study)
\item 16 with within-sample adoption (used in event study)
\item 22 never-treated (at federal floor throughout)
\end{itemize}

States already above federal in 2012 (13 states): CA, CO, CT, IL, MA, ME, MI, NM, OH, OR, RI, VT, WA. States with within-sample adoption (16 states): AK (2015), AR (2015), AZ (2017), DE (2014), FL (2019), HI (2015), MD (2015), MN (2014), MO (2019), MT (2016), NE (2015), NJ (2014), NV (2020), NY (2014), SD (2015), WV (2015). States at federal floor throughout (22 states): AL, DC, GA, ID, IN, IA, KS, KY, LA, MS, NC, ND, NH, OK, PA, SC, TN, TX, UT, VA, WI, WY.


\section{Robustness Appendix}

\subsection{Goodman-Bacon Decomposition}

The Goodman-Bacon decomposition partitions the TWFE estimate into weighted averages of 2$\times$2 DiD comparisons:

\begin{itemize}
\item Treated vs. never-treated: 82\% weight
\item Earlier vs. later treated: 8\% weight
\item Later vs. earlier treated: 10\% weight
\end{itemize}

The dominance of clean treated-vs-untreated comparisons explains the consistency between TWFE and Callaway-Sant'Anna estimates.

\subsection{Sensitivity to Large State Exclusion}

Excluding California, New York, and Texas removes approximately 35\% of national establishments but only 6\% of state-year observations. The estimate remains stable at $-0.016$ (SE = 0.037), indicating that results are not driven by these large states.

\end{document}
