\documentclass[12pt]{article}

% Packages
\usepackage[margin=1in]{geometry}
\usepackage{amsmath,amssymb,amsthm}
\usepackage{graphicx}
\usepackage{booktabs}
\usepackage{natbib}
\usepackage{setspace}
\usepackage{hyperref}
\usepackage{caption}
\usepackage{subcaption}
\usepackage{float}
\usepackage{rotating}
\usepackage{pdflscape}
\usepackage{longtable}
\usepackage{array}
\usepackage{tabularx}
\usepackage{threeparttable}
\usepackage{adjustbox}
\usepackage{xcolor}
\usepackage{soul}
\usepackage{multirow}
\usepackage{enumitem}

% Formatting
\doublespacing
\setlength{\parindent}{0.5in}
\hypersetup{colorlinks=true, linkcolor=blue, citecolor=blue, urlcolor=blue}

% Title
\title{\textbf{Salary Transparency Laws and Wage Outcomes: \\
Evidence from Staggered State Adoption}\thanks{We thank the CPS for providing public access to microdata. All errors are our own.}}

\author{
  APEP Working Paper 0066\\[0.5em]
  \textit{Autonomous Policy Evaluation Project and @olafdrw}
}

\date{January 2026}

\begin{document}

\maketitle

\begin{abstract}
\noindent Between 2021 and 2025, thirteen U.S. states adopted laws requiring employers to disclose salary ranges in job postings. Using a difference-in-differences design with staggered adoption, we estimate the causal effect of these salary transparency laws on wage outcomes. Drawing on CPS Merged Outgoing Rotation Group data covering over 1.1 million workers from 2016--2024, we find that transparency laws are associated with a statistically significant 4.2 log point \textit{decrease} in weekly earnings (SE = 0.007, $p < 0.001$). We find suggestive evidence that the gender wage gap narrows in treated states, with a 1.2 percentage point reduction, though this effect is not statistically significant. The negative wage effects are larger for male workers ($-4.3\%$) than female workers ($-3.1\%$), consistent with wage compression reducing previously higher male wages. Heterogeneity analysis reveals that effects are concentrated among middle-aged workers (ages 35-49: $-5.6\%$) and college-educated workers ($-4.7\%$), while effects at the 10th and 90th percentiles ($-5.2\%$ each) are larger than at the median ($-3.0\%$), suggesting compression from both tails. These findings contribute to a growing literature on pay transparency, highlighting the potential unintended consequences of well-intentioned labor market policies. However, significant pre-trend violations in the event study raise concerns about the parallel trends assumption, and we urge caution in interpreting these estimates as causal.

\vspace{1em}
\noindent \textbf{JEL Codes:} J31, J38, J71, K31

\vspace{0.5em}
\noindent \textbf{Keywords:} Salary transparency, pay equity, wage determination, gender wage gap, difference-in-differences
\end{abstract}

\newpage
\tableofcontents
\newpage

%============================================================================
\section{Introduction}
%============================================================================

The question of how information shapes labor market outcomes is central to understanding wage determination. Traditional economic models often assume that both employers and workers possess complete information about market wages, yet in practice, significant information asymmetries exist. Workers frequently lack knowledge of prevailing wages for their position, limiting their ability to negotiate effectively and potentially contributing to wage inequality, including the persistent gender wage gap \citep{card2012workplace,cullen2023equilibrium}.

In response to concerns about pay inequity, a wave of state-level salary transparency legislation has swept across the United States. Beginning with Colorado in 2021, states have adopted laws requiring employers to disclose salary ranges in job postings. By the end of 2025, thirteen states covering approximately 40\% of the U.S. workforce will have implemented such requirements. These laws represent a significant intervention in labor market information flows, with potentially far-reaching consequences for wage setting, bargaining, and worker welfare.

The theoretical predictions for salary transparency laws are ambiguous. On one hand, transparency may benefit workers by reducing information asymmetries, enabling more effective wage negotiation, and exposing discriminatory pay practices \citep{bennedsen2022firms}. On the other hand, transparency may reduce worker bargaining power by eliminating the possibility of obtaining wages above the posted range, compress wage distributions, and lead firms to post wider or more conservative ranges \citep{cullen2023equilibrium}. Understanding which effects dominate is an empirical question with important policy implications.

This paper provides causal evidence on the effects of state salary transparency laws using a difference-in-differences research design. We exploit the staggered adoption of transparency requirements across states between 2021 and 2025, comparing wage outcomes in treated states to those in not-yet-treated and never-treated states. Our analysis draws on the Current Population Survey Merged Outgoing Rotation Groups (CPS MORG), which provides high-quality earnings data for a representative sample of U.S. workers.

\subsection{Main Findings}

Our main findings are threefold. First, we find that salary transparency laws are associated with a statistically significant 4.2 log point decrease in average weekly earnings. This negative effect emerges immediately upon law implementation and persists through subsequent years. Second, we find suggestive evidence that transparency laws narrow the gender wage gap, with a 1.2 percentage point reduction in the male-female earnings differential, though this effect is not statistically significant at conventional levels. Third, we document heterogeneous effects by gender: male earnings decline by 4.3\%, while female earnings decline by only 3.1\%, consistent with wage compression reducing previously higher wages.

Beyond these core results, we provide extensive heterogeneity analysis. Effects are larger for middle-aged workers (ages 35-49: $-5.6\%$, SE = 0.011) than for younger workers (ages 18-34: $-2.1\%$, SE = 0.008), potentially reflecting the greater wage dispersion among more experienced workers that transparency compresses. College-educated workers experience larger effects ($-4.7\%$, SE = 0.019) than those without college education ($-1.8\%$, SE = 0.012, not significant), consistent with compression of skilled wage premiums.

We also examine effects across the wage distribution. Interestingly, effects at the 10th percentile ($-5.2\%$, SE = 0.012) are as large as effects at the 90th percentile ($-5.2\%$, SE = 0.011), with smaller effects at the median ($-3.0\%$, SE = 0.015). This suggests that transparency compresses the wage distribution from both tails rather than simply pulling down high wages. The mechanism for lower-tail compression may involve anchoring effects from posted ranges or reduced wage negotiation at all levels.

\subsection{Contribution to the Literature}

These results contribute to a growing literature on pay transparency policies. \citet{baker2019pay} study a Danish law requiring firms to report gender-disaggregated wages and find a 7\% reduction in the gender pay gap, driven primarily by wage compression among male workers. \citet{bennedsen2022firms} examine mandatory pay gap disclosure in the UK and find that firms respond by reducing bonuses for men rather than increasing pay for women. \citet{cullen2023equilibrium} develop a model showing that transparency can shift bargaining power from workers to firms when wage distributions are compressed. Our findings align with this theoretical prediction and extend the literature by examining laws that require disclosure in job postings, a more salient form of transparency than internal reporting requirements.

However, we urge caution in interpreting our results. Event study analysis reveals significant pre-treatment coefficients, suggesting potential violations of the parallel trends assumption. States adopting transparency laws may have been on different wage trajectories prior to adoption, and the observed post-treatment effects may reflect continuation of these trends rather than causal policy impacts. We conduct extensive robustness checks, including alternative control groups and specification tests, but cannot fully rule out selection bias.

\subsection{Policy Implications}

Our findings have direct policy relevance as more states and the federal government consider transparency legislation. While the stated goals of transparency laws---reducing wage inequality and closing gender gaps---are laudable, our results suggest the mechanisms through which these goals are achieved may differ from policymaker expectations. Rather than raising wages for underpaid workers, transparency appears to lower wages across the board, achieving equity through compression rather than uplift. Policymakers should weigh these potential consequences against the benefits of improved information flow.

The remainder of this paper proceeds as follows. Section~\ref{sec:background} provides institutional background on salary transparency laws. Section~\ref{sec:data} describes our data sources and sample construction. Section~\ref{sec:method} outlines our empirical strategy. Section~\ref{sec:results} presents main results, and Section~\ref{sec:robust} discusses robustness checks. Section~\ref{sec:conclude} concludes.

%============================================================================
\section{Institutional Background}\label{sec:background}
%============================================================================

\subsection{The Rise of Salary Transparency Laws}

Salary transparency laws requiring disclosure of pay ranges in job postings represent a relatively new policy intervention in U.S. labor markets. While equal pay laws have existed for decades, requiring employers to pay men and women equally for equal work, these newer transparency requirements take a different approach: mandating that employers reveal wage information \textit{before} hiring decisions are made.

The policy rationale for salary transparency laws rests on several foundations. First, proponents argue that information asymmetries disadvantage workers in wage negotiations. Employers typically possess superior information about market wages and internal pay structures, while workers must negotiate with incomplete information \citep{card2012workplace}. By requiring salary range disclosure, transparency laws aim to level the informational playing field.

Second, transparency laws are viewed as a tool for addressing pay discrimination, particularly the persistent gender wage gap. When salaries are secret, discriminatory pay practices can persist undetected. Transparency requirements force employers to justify their compensation practices and may expose inconsistencies that cannot be explained by legitimate factors \citep{blau2017gender}.

Third, transparency may improve labor market efficiency by reducing search frictions. When job seekers can observe wage ranges, they can more efficiently match to positions that meet their reservation wage, reducing time spent on applications for unsuitable positions \citep{marinescu2020wages}.

Fourth, and perhaps most importantly from a political economy perspective, transparency laws enjoy broad public support. Surveys consistently show that workers prefer jobs with transparent pay practices, and advocacy groups have successfully framed transparency as a fairness issue. This political momentum has driven rapid adoption across progressive-leaning states.

\subsection{State-Level Adoption}

Colorado became the first state to require salary range disclosure in job postings when its Equal Pay for Equal Work Act took effect on January 1, 2021. The Colorado law requires all employers, regardless of size, to include the compensation range and a general description of benefits in job postings. Notably, the law also prohibits employers from retaliating against employees who discuss wages and requires employers to make reasonable efforts to notify employees of advancement opportunities.

The Colorado law faced initial resistance from some employers. Reports emerged of companies explicitly excluding Colorado applicants from remote positions to avoid compliance \citep{arnold2023impact}. This response highlighted both the binding nature of the requirement and the potential for unintended consequences. Major employers including Johnson \& Johnson, Nike, and others reportedly posted remote job listings with disclaimers that Colorado applicants would not be considered.

Following Colorado's lead, a wave of states adopted similar legislation. California and Washington implemented transparency requirements effective January 1, 2023, each applying to employers with 15 or more employees. New York State's law took effect on September 17, 2023, with a lower threshold of 4 or more employees. Hawaii, Maryland, and the District of Columbia followed in 2024, and Illinois, Minnesota, Vermont, and Massachusetts enacted laws effective in 2025.

Table~\ref{tab:treatment} summarizes the adoption timeline and key features of state salary transparency laws. Several patterns emerge from the adoption sequence. First, early adopters tend to be politically liberal states with strong labor movements. Second, the employer size thresholds vary substantially, from all employers (Colorado, D.C., Maryland) to 50 or more employees (Hawaii). Third, the scope of required disclosure has expanded over time, with later laws often requiring disclosure of benefits information in addition to salary ranges.

\begin{table}[htbp]
\centering
\caption{State Salary Transparency Law Adoption}
\label{tab:treatment}
\begin{threeparttable}
\begin{tabular}{llcl}
\toprule
State & Effective Date & Employer Threshold & Key Features \\
\midrule
Colorado & January 1, 2021 & All employers & Salary range + benefits \\
California & January 1, 2023 & 15+ employees & Salary range required \\
Washington & January 1, 2023 & 15+ employees & Salary + benefits + other comp \\
New York & September 17, 2023 & 4+ employees & Salary range required \\
Hawaii & January 1, 2024 & 50+ employees & Salary range required \\
D.C. & June 30, 2024 & All employers & Salary + healthcare benefits \\
Maryland & October 1, 2024 & All employers & Salary range + benefits \\
Illinois & January 1, 2025 & 15+ employees & Salary range + benefits \\
Minnesota & January 1, 2025 & 30+ employees & Starting salary + benefits \\
Vermont & July 1, 2025 & 5+ employees & Compensation range \\
Massachusetts & October 29, 2025 & 25+ employees & Pay range required \\
Maine & January 1, 2026 & 10+ employees & Salary range required \\
\bottomrule
\end{tabular}
\begin{tablenotes}
\small
\item \textit{Notes:} Table shows states adopting salary transparency laws requiring disclosure in job postings by end of 2025. Employer threshold indicates minimum firm size for law applicability. Features vary by state but typically require disclosure of minimum and maximum salary for the position.
\end{tablenotes}
\end{threeparttable}
\end{table}

Figure~\ref{fig:map} displays the geographic distribution of states adopting transparency laws, with earlier adopters concentrated in the West Coast and Northeast corridor.

\subsection{Theoretical Predictions}

Economic theory provides ambiguous predictions for the effects of salary transparency on wage levels and distributions. We briefly summarize the key theoretical channels.

\textbf{Information Channel.} Standard search models predict that reducing information frictions should improve match quality and worker welfare \citep{stigler1962information}. If workers lack information about prevailing wages, they may accept offers below their market value or spend excessive time searching. Transparency should enable workers to more accurately assess their outside options, potentially increasing wages by improving bargaining positions.

\textbf{Bargaining Channel.} However, transparency may also affect the bargaining game itself. When salary ranges are public, workers can no longer credibly claim to have higher outside offers, reducing their leverage in negotiations. \citet{cullen2023equilibrium} formalize this intuition, showing that when all wages become observable, firms can more easily coordinate on wage offers, effectively shifting surplus from workers to employers.

\textbf{Compression Channel.} Transparency may also compress wage distributions within firms. When employees can observe each other's pay, firms face pressure to maintain internal equity. \citet{card2012workplace} document that workers reduce effort when they learn they are paid below peers, suggesting that transparency creates incentives for firms to narrow pay differentials. This compression could manifest as reductions in wages at the top of the distribution.

\textbf{Discrimination Channel.} For the gender wage gap specifically, transparency may expose discriminatory practices and create accountability. \citet{bennedsen2022firms} find that mandatory disclosure of gender pay gaps leads firms to narrow gaps, primarily by reducing male wages. If gender-based pay differences are partly attributable to information asymmetries that disadvantage women in negotiation \citep{bowles2007social}, transparency could disproportionately benefit women.

\textbf{Selection Channel.} Finally, transparency may affect employer location and hiring decisions. If compliance is costly, some employers may relocate jobs to non-transparency states or reduce hiring in treated states. The early evidence from Colorado of employers excluding state residents from remote positions suggests this channel may be operative.

\textbf{Anchoring Channel.} A less-discussed possibility is that posted salary ranges serve as anchors in wage negotiations. If employers post ranges with lower midpoints than they would otherwise offer, workers may anchor on these ranges and negotiate less aggressively. This could lead to wage reductions even when workers have full information about posted ranges.

The net effect of these channels is theoretically ambiguous and must be determined empirically.

\subsection{Prior Literature}

The empirical literature on pay transparency has grown rapidly in recent years, though most studies focus on contexts outside the United States or on different forms of transparency.

\citet{baker2019pay} study a 2006 Danish law requiring firms to report gender-disaggregated wages. Using matched employer-employee data, they find that the law reduced the gender pay gap by 7\%, with the effect driven primarily by slower wage growth for men. Importantly, they find no evidence of negative effects on firm profitability or employment.

\citet{bennedsen2022firms} examine the UK's 2017 requirement that firms report gender pay gaps. They find that firms responded by reducing the gap, but primarily through reductions in bonuses for men rather than increases in base pay for women. The study highlights how transparency can lead to wage compression rather than wage increases for disadvantaged groups.

\citet{perez2022gender} study Austria's 2011 introduction of mandatory gender pay reporting for firms above certain size thresholds. They find modest reductions in the gender pay gap, concentrated in firms just above the reporting threshold, suggesting that compliance costs limit the effectiveness of such policies.

In the U.S. context, \citet{arnold2023impact} provides early evidence on Colorado's salary transparency law using job posting data. He finds that the law increased the share of postings with salary information by 30 percentage points and increased average posted wages by 3.6\%. However, this analysis focuses on posted wages rather than realized wages and cannot assess effects on the gender wage gap.

\citet{cullen2023equilibrium} develop a model of salary transparency with heterogeneous productivity workers. They show that when firms must post wages, they optimally narrow wage ranges, reducing high wages while potentially increasing low wages. The model generates a prediction of negative effects on average wages combined with reduced wage dispersion.

\citet{obloj2022pay} examine within-firm wage transparency using a natural experiment in a large U.S. firm. They find that when employees gained access to co-workers' salaries, turnover increased among relatively underpaid workers, while firm profits declined due to reduced worker effort among those learning they were paid below peers.

Our contribution to this literature is to estimate the causal effect of U.S. salary transparency laws on realized wages using a difference-in-differences design with staggered state adoption. We are among the first to examine effects on the gender wage gap and wage distribution in the U.S. context, and we provide the most comprehensive heterogeneity analysis to date.

%============================================================================
\section{Data}\label{sec:data}
%============================================================================

\subsection{CPS Merged Outgoing Rotation Groups}

Our primary data source is the Current Population Survey Merged Outgoing Rotation Groups (CPS MORG) from 2016 to 2024. The CPS is a monthly household survey conducted by the Census Bureau, providing nationally representative data on employment, earnings, and demographics. The MORG extracts include the earnings supplement administered to outgoing rotation groups (rotation groups 4 and 8), which contains detailed wage and salary information.

We use the NBER MORG extracts, which harmonize variables across years and provide analysis weights appropriate for earnings studies. The MORG sample includes approximately 280,000 individuals per year, with about half reporting earnings information. The CPS has several advantages for our analysis: it covers all 50 states plus D.C., it provides consistent earnings measures over time, and it includes detailed demographic information enabling heterogeneity analysis.

\subsection{Sample Construction}

We impose the following sample restrictions:

\begin{enumerate}[label=(\roman*)]
\item \textbf{Age:} We restrict to workers aged 18--64, the prime working age population.
\item \textbf{Employment Status:} We include wage and salary workers, excluding the self-employed (who are not subject to transparency requirements).
\item \textbf{Valid Earnings:} We require positive weekly earnings below the CPS top code (\$2,885 per week).
\item \textbf{Complete Demographics:} We require non-missing values for key demographic variables (age, sex, education).
\end{enumerate}

After applying these restrictions, our analysis sample contains 1,180,455 individual-year observations spanning 51 states (including D.C.) and 9 years (2016--2024). Table~\ref{tab:datastructure} shows the sample size by year and treatment status.

\begin{table}[htbp]
\centering
\caption{Sample Size by Year and Treatment Status}
\label{tab:datastructure}
\begin{threeparttable}
\begin{tabular}{lcccc}
\toprule
Year & States & Treated States & Control States & Total Observations \\
\midrule
2016 & 51 & 0 & 51 & 154,169 \\
2017 & 51 & 0 & 51 & 151,820 \\
2018 & 51 & 0 & 51 & 147,985 \\
2019 & 51 & 0 & 51 & 142,731 \\
2020 & 51 & 0 & 51 & 122,901 \\
2021 & 51 & 1 & 50 & 123,288 \\
2022 & 51 & 1 & 50 & 118,770 \\
2023 & 51 & 4 & 47 & 111,864 \\
2024 & 51 & 7 & 44 & 106,927 \\
\midrule
Total & --- & --- & --- & 1,180,455 \\
\bottomrule
\end{tabular}
\begin{tablenotes}
\small
\item \textit{Notes:} Sample includes wage/salary workers ages 18--64 from CPS MORG 2016--2024. Treated states count includes states with transparency laws effective during that year. The decline in sample size over time reflects CPS sampling changes and response rate trends.
\end{tablenotes}
\end{threeparttable}
\end{table}

\subsection{Variable Construction}

Our primary outcome variable is \textbf{log weekly earnings}, constructed from the CPS variable \texttt{earnwke} (weekly earnings not including overtime, tips, or commissions). We take the natural log to facilitate interpretation of coefficients as percentage changes.

For the \textbf{gender wage gap} analysis, we compute the male-female log earnings differential within each state-year cell. Specifically, we estimate the regression:
\begin{equation}
\ln W_{ist} = \alpha_{st} + \beta_{st} \text{Female}_i + X_i'\gamma_{st} + \varepsilon_{ist}
\end{equation}
for each state $s$ and year $t$, where $X_i$ includes controls for age, age squared, education, and race/ethnicity. The coefficient $\beta_{st}$ captures the conditional gender wage gap in state $s$ and year $t$.

We also construct \textbf{distributional outcomes} including the 10th, 25th, 50th, 75th, and 90th percentiles of log weekly earnings within each state-year cell. These allow us to examine how transparency affects the wage distribution beyond mean effects.

We construct \textbf{treatment indicators} based on the effective dates of state transparency laws. For states that adopted laws by our sample end date (2024), we code the treatment year as the first full calendar year the law was in effect. For the Sun-Abraham event study, we also construct event-time indicators relative to each state's treatment year.

\subsection{Summary Statistics}

Table~\ref{tab:summ} presents summary statistics for our analysis sample, separately for treated and control states.

\begin{table}[htbp]
\centering
\caption{Summary Statistics}
\label{tab:summ}
\begin{threeparttable}
\begin{tabular}{lccc}
\toprule
& Treated States & Control States & Full Sample \\
\midrule
Weekly Earnings (\$) & 1,284 & 1,142 & 1,175 \\
Log Weekly Earnings & 7.02 & 6.91 & 6.94 \\
Female (\%) & 49.2 & 48.7 & 48.8 \\
Age (years) & 40.8 & 41.2 & 41.1 \\
College+ (\%) & 38.4 & 31.2 & 32.8 \\
Full-time (\%) & 82.1 & 80.5 & 80.9 \\
\midrule
Observations & 289,299 & 891,156 & 1,180,455 \\
States & 9 & 42 & 51 \\
\bottomrule
\end{tabular}
\begin{tablenotes}
\small
\item \textit{Notes:} Sample includes wage/salary workers ages 18--64 from CPS MORG 2016--2024. Treated states are those adopting salary transparency laws by 2024 (CO, CA, WA, NY, HI, DC, MD). Weekly earnings top-coded at \$2,885. College+ indicates bachelor's degree or higher. Full-time indicates 35+ usual hours per week.
\end{tablenotes}
\end{threeparttable}
\end{table}

Several patterns emerge from the summary statistics. Workers in treated states have higher average earnings (\$1,284 vs. \$1,142 weekly), higher education levels (38.4\% vs. 31.2\% college+), and similar demographic compositions. These differences highlight the potential for selection bias if states with high wages are more likely to adopt transparency laws, a concern we address in our empirical strategy.

Table~\ref{tab:balance} presents a more formal balance assessment using only pre-treatment data (2016--2020).

\begin{table}[htbp]
\centering
\caption{Pre-Treatment Balance}
\label{tab:balance}
\begin{threeparttable}
\begin{tabular}{lccc}
\toprule
& Treated States & Control States & Difference \\
\midrule
Log Weekly Earnings & 6.74 & 6.61 & 0.13 \\
Age (years) & 40.6 & 41.1 & $-0.5$ \\
Female (\%) & 48.8 & 49.2 & $-0.4$ \\
\midrule
Observations & 178,062 & 541,544 & --- \\
\bottomrule
\end{tabular}
\begin{tablenotes}
\small
\item \textit{Notes:} Balance assessment using pre-treatment data only (2016--2020). Treated states are those that would later adopt transparency laws.
\end{tablenotes}
\end{threeparttable}
\end{table}

The pre-treatment differences are non-trivial: workers in (future) treated states earned approximately 13 log points more than those in (future) control states before any laws took effect. This gap underscores the importance of the difference-in-differences design, which identifies effects from within-state changes over time rather than cross-state level comparisons.

%============================================================================
\section{Empirical Strategy}\label{sec:method}
%============================================================================

\subsection{Difference-in-Differences Design}

We estimate the effect of salary transparency laws using a difference-in-differences (DiD) research design. The core identifying assumption is that, absent treatment, wages in treated states would have evolved in parallel to wages in control states.

\subsubsection{Simple DiD Specification}

Our baseline specification estimates:
\begin{equation}
Y_{st} = \alpha_s + \lambda_t + \beta \cdot \text{Treated}_{st} + \varepsilon_{st}
\end{equation}
where $Y_{st}$ is the outcome (log weekly earnings or gender wage gap) in state $s$ and year $t$, $\alpha_s$ are state fixed effects, $\lambda_t$ are year fixed effects, and $\text{Treated}_{st}$ is an indicator for whether state $s$ has an active transparency law in year $t$.

The coefficient $\beta$ captures the average treatment effect on the treated (ATT), interpreted as the percentage change in earnings attributable to transparency laws. We weight observations by the number of individuals in each state-year cell and cluster standard errors at the state level to account for serial correlation within states \citep{bertrand2004much}.

\subsubsection{Staggered Adoption and Sun-Abraham Estimator}

A key feature of our setting is staggered treatment adoption: states implement transparency laws in different years (2021, 2023, 2024, 2025). This creates two related challenges for traditional two-way fixed effects (TWFE) estimation. First, with heterogeneous treatment effects across cohorts or over time, TWFE can produce biased estimates by using already-treated units as controls \citep{goodman2021difference}. Second, differential treatment timing can generate negative weights on some group-time ATTs, potentially yielding estimates with the wrong sign \citep{dechaisemartin2020two}.

We address these concerns using the \citet{sun2021estimating} interaction-weighted estimator, implemented via the \texttt{sunab()} function in the \texttt{fixest} R package. The Sun-Abraham estimator constructs event-time dummies relative to each cohort's treatment date and interacts them with cohort indicators. The overall ATT is recovered as a weighted average of cohort-specific effects, with weights that ensure never-treated or not-yet-treated units serve as controls.

Our event study specification estimates:
\begin{equation}
Y_{st} = \alpha_s + \lambda_t + \sum_{k \neq -1} \beta_k \cdot \mathbf{1}[K_{st} = k] + \varepsilon_{st}
\end{equation}
where $K_{st}$ denotes event time (years since treatment) for state $s$ at time $t$, and the reference period is $k = -1$ (the year before treatment). Pre-treatment coefficients ($\beta_k$ for $k < 0$) provide a test of the parallel trends assumption: if these coefficients are statistically indistinguishable from zero, we have greater confidence that control states provide a valid counterfactual.

\subsubsection{Treatment Definition}

We define treatment based on the effective date of state transparency laws. A state is coded as treated starting in the first full calendar year after the law's effective date. For example, Colorado's law took effect January 1, 2021, so Colorado is coded as treated from 2021 onward. For states with mid-year effective dates (e.g., New York on September 17, 2023), we code treatment as beginning the following year to ensure a clean pre/post comparison.

This coding may introduce some measurement error, as laws may affect behavior before their formal effective date (anticipation) or have delayed impacts as compliance spreads. We address these concerns through event study analysis and robustness checks.

\subsection{Threats to Identification}

Several threats to identification warrant discussion.

\textbf{Selection into Treatment.} States adopting transparency laws may differ systematically from non-adopting states. Treated states tend to have higher wages, more educated workforces, and more progressive political environments. If these factors are correlated with wage trends, simple comparisons may be biased. Our fixed effects control for time-invariant state characteristics, and the event study tests for differential pre-trends.

\textbf{Concurrent Policies.} Transparency laws are often enacted alongside other labor market reforms (e.g., minimum wage increases, equal pay laws). If these policies affect wages, our estimates may capture a package of reforms rather than transparency specifically. We address this by examining robustness to inclusion of state-level controls.

\textbf{Spillovers and Anticipation.} Transparency laws may affect wages before their official effective dates (anticipation) or influence wages in neighboring states (spillovers). Anticipation would cause us to underestimate effects by including anticipation effects in the control period. Spatial spillovers could bias estimates in either direction.

\textbf{Remote Work.} Many transparency laws apply to remote positions if the employee works in (or reports to a supervisor in) a covered state. This complicates the treatment definition, as workers in non-treated states may be affected. Our analysis uses state of residence rather than state of employment, potentially introducing measurement error.

\textbf{Composition Changes.} If transparency laws affect the composition of employed workers (e.g., by discouraging high-wage workers from seeking jobs in treated states), observed wage changes may reflect selection rather than effects on wages for a fixed set of workers. The CPS provides repeated cross-sections rather than panel data, limiting our ability to distinguish composition effects from within-worker wage changes.

%============================================================================
\section{Results}\label{sec:results}
%============================================================================

\subsection{Main Effects on Wages}

Table~\ref{tab:main} presents our main results on the effect of salary transparency laws on log weekly earnings.

\begin{table}[htbp]
\centering
\caption{Effect of Salary Transparency Laws on Log Weekly Earnings}
\label{tab:main}
\begin{threeparttable}
\begin{tabular}{lcc}
\toprule
& (1) & (2) \\
& Simple DiD & Sun-Abraham ATT \\
\midrule
Treated & $-0.039^{***}$ & $-0.042^{***}$ \\
& (0.010) & (0.007) \\
\midrule
State FE & Yes & Yes \\
Year FE & Yes & Yes \\
Observations & 459 & 459 \\
States & 51 & 51 \\
$R^2$ (within) & 0.070 & 0.165 \\
\bottomrule
\end{tabular}
\begin{tablenotes}
\small
\item \textit{Notes:} Dependent variable is mean log weekly earnings at the state-year level. Observations weighted by state-year sample size. Column (1) reports simple DiD with state and year fixed effects. Column (2) reports the Sun-Abraham interaction-weighted ATT. Standard errors clustered at state level in parentheses. $^{***}$ $p<0.01$, $^{**}$ $p<0.05$, $^{*}$ $p<0.10$.
\end{tablenotes}
\end{threeparttable}
\end{table}

Column (1) reports the simple DiD estimate: transparency laws are associated with a 3.9 log point decrease in weekly earnings, statistically significant at the 1\% level. Column (2) reports the Sun-Abraham ATT, which accounts for heterogeneous treatment effects with staggered adoption. The magnitude is slightly larger at 4.2 log points.

These results suggest that salary transparency laws \textit{reduce} average wages, contrary to the hypothesis that transparency benefits workers. The magnitude is economically significant: a 4.2\% reduction in wages represents approximately \$50 per week for the average worker, or roughly \$2,600 annually.

\subsection{Event Study}

Figure~\ref{fig:eventstudy} presents the event study, plotting estimated coefficients for each year relative to treatment.

\begin{figure}[htbp]
\centering
\includegraphics[width=0.9\textwidth]{figures/fig2_event_study.png}
\caption{Event Study: Effect of Salary Transparency Laws on Log Weekly Earnings}
\label{fig:eventstudy}
\begin{flushleft}
\small \textit{Notes:} Figure plots Sun-Abraham event study coefficients with 95\% confidence intervals. The omitted reference period is $t = -1$ (year before treatment). Vertical line indicates treatment timing. Sample includes 459 state-year observations from 2016--2024.
\end{flushleft}
\end{figure}

The event study reveals several important patterns. First, the negative treatment effect emerges immediately upon implementation: the $t = 0$ coefficient is $-0.035$ (SE = 0.008), and the $t = 1$ coefficient is $-0.057$ (SE = 0.007). Second, the effect appears to partially dissipate by $t = 2$, with a coefficient near zero, before becoming negative again at $t = 3$.

However, the event study also reveals a concerning pattern: the $t = -3$ coefficient is \textit{positive} (0.027, SE = 0.007) and statistically significant. This suggests that treated states experienced wage \textit{increases} relative to control states three years before treatment---the opposite of what a clean parallel trends story would predict. This pre-trend raises questions about whether the parallel trends assumption holds and whether our estimates capture causal effects or pre-existing differences.

Table~\ref{tab:eventstudy} provides the full set of event study coefficients.

\begin{table}[htbp]
\centering
\caption{Event Study Coefficients}
\label{tab:eventstudy}
\begin{threeparttable}
\begin{tabular}{lcccc}
\toprule
Event Time & Estimate & SE & 95\% CI & Stars \\
\midrule
$-8$ & 0.000 & 0.027 & [$-0.052$, 0.052] & \\
$-7$ & $-0.015$ & 0.012 & [$-0.039$, 0.009] & \\
$-6$ & $-0.001$ & 0.007 & [$-0.015$, 0.013] & \\
$-5$ & $-0.011$ & 0.007 & [$-0.024$, 0.003] & \\
$-4$ & 0.005 & 0.007 & [$-0.009$, 0.019] & \\
$-3$ & 0.027 & 0.007 & [0.012, 0.041] & *** \\
$-2$ & 0.005 & 0.004 & [$-0.003$, 0.013] & \\
0 & $-0.035$ & 0.008 & [$-0.052$, $-0.019$] & *** \\
1 & $-0.057$ & 0.007 & [$-0.072$, $-0.043$] & *** \\
2 & 0.001 & 0.007 & [$-0.013$, 0.015] & \\
3 & $-0.017$ & 0.008 & [$-0.033$, $-0.001$] & ** \\
\bottomrule
\end{tabular}
\begin{tablenotes}
\small
\item \textit{Notes:} Sun-Abraham event study coefficients from specification with state and year fixed effects. Reference period is $t = -1$. Standard errors clustered at state level. $^{***}$ $p<0.01$, $^{**}$ $p<0.05$, $^{*}$ $p<0.10$.
\end{tablenotes}
\end{threeparttable}
\end{table}

\subsection{Gender Wage Gap}

Table~\ref{tab:gender} presents results for the effect of transparency laws on the gender wage gap. The gender gap is defined as the difference between male and female mean log earnings within each state-year cell; a negative coefficient indicates a reduction in the gap.

\begin{table}[htbp]
\centering
\caption{Effect of Salary Transparency Laws on Gender Wage Gap}
\label{tab:gender}
\begin{threeparttable}
\begin{tabular}{lcc}
\toprule
& (1) & (2) \\
& DiD & Event Study \\
\midrule
Treated & $-0.012$ & --- \\
& (0.011) & \\
Event time = 0 & --- & $-0.005$ \\
& & (0.009) \\
Event time = 1 & --- & $-0.023^{***}$ \\
& & (0.008) \\
Event time = 2 & --- & $-0.077^{***}$ \\
& & (0.007) \\
Event time = 3 & --- & $-0.073^{***}$ \\
& & (0.008) \\
\midrule
State FE & Yes & Yes \\
Year FE & Yes & Yes \\
Observations & 459 & 459 \\
\bottomrule
\end{tabular}
\begin{tablenotes}
\small
\item \textit{Notes:} Dependent variable is the male-female log earnings differential (gender wage gap) at the state-year level. Negative coefficients indicate reduction in the gender wage gap. Standard errors clustered at state level in parentheses. $^{***}$ $p<0.01$, $^{**}$ $p<0.05$, $^{*}$ $p<0.10$.
\end{tablenotes}
\end{threeparttable}
\end{table}

The simple DiD estimate suggests a 1.2 percentage point reduction in the gender wage gap, but this effect is not statistically significant (SE = 0.011). However, the event study reveals a more nuanced pattern: while there is no immediate effect at $t = 0$, the gender gap reduction grows over time, reaching 7.7 percentage points by $t = 2$. This delayed effect is consistent with transparency requiring time to affect bargaining dynamics and employer practices.

Figure~\ref{fig:gendergapes} shows the event study for the gender gap outcome.

\begin{figure}[htbp]
\centering
\includegraphics[width=0.9\textwidth]{figures/fig9_gender_gap_es.png}
\caption{Event Study: Effect on Gender Wage Gap}
\label{fig:gendergapes}
\begin{flushleft}
\small \textit{Notes:} Sun-Abraham event study coefficients for gender wage gap (male minus female log earnings). Negative values indicate gap reduction. Shaded area shows 95\% confidence interval.
\end{flushleft}
\end{figure}

Importantly, the pre-treatment coefficients for the gender gap are generally small and insignificant, suggesting that the parallel trends assumption may be more credible for this outcome than for the overall wage effect.

\subsection{Heterogeneity by Gender}

Table~\ref{tab:het} examines heterogeneous effects by gender.

\begin{table}[htbp]
\centering
\caption{Heterogeneous Effects by Gender}
\label{tab:het}
\begin{threeparttable}
\begin{tabular}{lccc}
\toprule
& All Workers & Male & Female \\
\midrule
ATT & $-0.039$ & $-0.043$ & $-0.031$ \\
SE & (0.010) & (0.010) & (0.012) \\
95\% CI & $[-0.059, -0.020]$ & $[-0.064, -0.022]$ & $[-0.055, -0.007]$ \\
\bottomrule
\end{tabular}
\begin{tablenotes}
\small
\item \textit{Notes:} Table reports DiD estimates of the effect of salary transparency laws on log weekly earnings by gender. Standard errors clustered at state level.
\end{tablenotes}
\end{threeparttable}
\end{table}

Both male and female workers experience wage declines following transparency laws, but the effect is larger for men ($-4.3\%$) than for women ($-3.1\%$). This 1.2 percentage point differential is consistent with transparency compressing the wage distribution by reducing high wages (disproportionately earned by men) more than low wages. The pattern aligns with evidence from Denmark and the UK, where transparency narrowed gender gaps primarily through reductions in male wages \citep{baker2019pay,bennedsen2022firms}.

Figure~\ref{fig:het} displays these heterogeneous effects graphically.

\subsection{Heterogeneity by Demographics}

Table~\ref{tab:demohet} presents heterogeneous effects by age group and education level.

\begin{table}[htbp]
\centering
\caption{Heterogeneous Effects by Demographics}
\label{tab:demohet}
\begin{threeparttable}
\begin{tabular}{lcccc}
\toprule
Group & ATT & SE & 95\% CI & Stars \\
\midrule
\multicolumn{5}{l}{\textit{By Age Group}} \\
Young (18--34) & $-0.021$ & 0.008 & $[-0.037, -0.006]$ & *** \\
Middle (35--49) & $-0.056$ & 0.011 & $[-0.077, -0.036]$ & *** \\
Older (50--64) & $-0.050$ & 0.021 & $[-0.090, -0.010]$ & ** \\
\midrule
\multicolumn{5}{l}{\textit{By Education}} \\
No college & $-0.018$ & 0.012 & $[-0.041, 0.006]$ & \\
Some college & $-0.026$ & 0.008 & $[-0.042, -0.011]$ & *** \\
College+ & $-0.047$ & 0.019 & $[-0.084, -0.010]$ & ** \\
\bottomrule
\end{tabular}
\begin{tablenotes}
\small
\item \textit{Notes:} DiD estimates by demographic subgroup. Standard errors clustered at state level. $^{***}$ $p<0.01$, $^{**}$ $p<0.05$, $^{*}$ $p<0.10$.
\end{tablenotes}
\end{threeparttable}
\end{table}

Effects are largest for middle-aged workers (35--49: $-5.6\%$) and smallest for younger workers (18--34: $-2.1\%$). This pattern may reflect the greater wage dispersion among more experienced workers, which provides more scope for compression. Alternatively, younger workers may already face more standardized wages that leave less room for reduction.

By education, effects are largest for college-educated workers ($-4.7\%$) and smallest and insignificant for those without college education ($-1.8\%$). This suggests that transparency compresses wage premiums for skilled workers, potentially by revealing that some employers pay above-market rates for college graduates.

\subsection{Effects Across the Wage Distribution}

Table~\ref{tab:dist} examines effects at different percentiles of the wage distribution.

\begin{table}[htbp]
\centering
\caption{Effects on Wage Distribution}
\label{tab:dist}
\begin{threeparttable}
\begin{tabular}{lcccc}
\toprule
Percentile & ATT & SE & 95\% CI & Stars \\
\midrule
10th & $-0.052$ & 0.012 & $[-0.075, -0.028]$ & *** \\
25th & $-0.015$ & 0.009 & $[-0.033, 0.004]$ & \\
50th (Median) & $-0.030$ & 0.015 & $[-0.059, 0.000]$ & ** \\
75th & $-0.048$ & 0.019 & $[-0.086, -0.010]$ & ** \\
90th & $-0.052$ & 0.011 & $[-0.074, -0.029]$ & *** \\
\bottomrule
\end{tabular}
\begin{tablenotes}
\small
\item \textit{Notes:} DiD estimates for percentiles of log weekly earnings distribution. Standard errors clustered at state level. $^{***}$ $p<0.01$, $^{**}$ $p<0.05$, $^{*}$ $p<0.10$.
\end{tablenotes}
\end{threeparttable}
\end{table}

The distributional effects reveal a surprising U-shaped pattern. Effects are largest at the tails of the distribution (10th percentile: $-5.2\%$; 90th percentile: $-5.2\%$) and smaller at the median ($-3.0\%$). The effect at the 25th percentile is smaller and not statistically significant ($-1.5\%$).

This pattern differs from the simple compression story, which would predict larger effects at the top of the distribution than at the bottom. The large negative effect at the 10th percentile suggests that transparency may reduce wages for low-wage workers as well, perhaps through anchoring effects when posted ranges are lower than what employers would otherwise pay.

%============================================================================
\section{Robustness}\label{sec:robust}
%============================================================================

\subsection{Alternative Control Groups and Specifications}

Table~\ref{tab:robust} examines robustness to alternative specifications.

\begin{table}[htbp]
\centering
\caption{Robustness to Alternative Specifications}
\label{tab:robust}
\begin{threeparttable}
\begin{tabular}{lcccc}
\toprule
Specification & ATT & SE & N States & Stars \\
\midrule
Main (weighted) & $-0.039$ & 0.010 & 51 & *** \\
Never-treated control & $-0.039$ & 0.010 & 51 & *** \\
Excluding Colorado & $-0.048$ & 0.006 & 50 & *** \\
Excluding California & $-0.033$ & 0.015 & 50 & ** \\
Unweighted & $-0.034$ & 0.015 & 51 & ** \\
\bottomrule
\end{tabular}
\begin{tablenotes}
\small
\item \textit{Notes:} All specifications include state and year fixed effects. Standard errors clustered at state level. $^{***}$ $p<0.01$, $^{**}$ $p<0.05$, $^{*}$ $p<0.10$.
\end{tablenotes}
\end{threeparttable}
\end{table}

Results are robust to using only never-treated states as controls (ATT = $-0.039$), excluding Colorado, the first mover (ATT = $-0.048$), and using unweighted regressions (ATT = $-0.034$). Interestingly, excluding California---the largest treated state---yields a smaller but still significant estimate (ATT = $-0.033$), while excluding Colorado yields a larger estimate, suggesting Colorado may have experienced smaller-than-average effects.

\subsection{Placebo Test}

Figure~\ref{fig:placebo} presents a placebo test using only pre-treatment data, with treatment artificially assigned in 2019.

\begin{figure}[htbp]
\centering
\includegraphics[width=0.9\textwidth]{figures/fig11_placebo.png}
\caption{Placebo Test: Fake Treatment in 2019}
\label{fig:placebo}
\begin{flushleft}
\small \textit{Notes:} Event study using only pre-treatment data (2016--2020), with placebo treatment assigned in 2019 to states that would later adopt transparency laws. Non-zero effects would suggest pre-existing differential trends.
\end{flushleft}
\end{figure}

The placebo test reveals some concerning patterns. The post-placebo coefficients show variation, though confidence intervals are wider than in the main analysis due to the shorter time series. The presence of any systematic pattern in the placebo test reinforces concerns about pre-treatment trends.

\subsection{Pre-Trend Concerns}

The significant positive coefficient at $t = -3$ in our event study raises concerns about violations of parallel trends. Several interpretations are possible:

\begin{enumerate}[label=(\roman*)]
\item \textbf{Policy Anticipation:} States may have experienced wage growth in anticipation of transparency laws, perhaps due to employer pre-compliance or correlated policy changes. However, anticipation effects typically appear closer to treatment (e.g., $t = -1$).

\item \textbf{Selection:} States with positive wage shocks may have been more likely to adopt transparency laws. This would bias our estimates if the positive shocks persist into the post-treatment period.

\item \textbf{Timing Mismeasurement:} If laws were announced well before their effective dates, employers may have begun responding earlier than our treatment coding captures.

\item \textbf{Confounding Policies:} The $t = -3$ coefficient coincides with 2018 for the 2021 Colorado cohort. If other policies affecting wages were adopted in 2018, this could generate the observed pattern.
\end{enumerate}

We cannot definitively distinguish among these interpretations with our research design. Future work with longer pre-periods and more detailed policy timing data may help resolve this question.

\subsection{State-by-State Heterogeneity}

Figure~\ref{fig:stateeffects} shows simple before-after comparisons for each treated state.

\begin{figure}[htbp]
\centering
\includegraphics[width=0.9\textwidth]{figures/fig10_state_effects.png}
\caption{State-Specific Wage Changes}
\label{fig:stateeffects}
\begin{flushleft}
\small \textit{Notes:} Simple pre/post wage comparisons by state. These are descriptive comparisons that do not control for trends in control states. Colors indicate treatment year.
\end{flushleft}
\end{figure}

There is considerable heterogeneity across states. Some states show large negative changes (e.g., Colorado), while others show smaller or even positive changes. This heterogeneity suggests that either the true effects vary across states or that state-specific factors confound the simple comparisons.

\subsection{Effects on Wage Inequality}

Figure~\ref{fig:inequality} presents the event study for the 90/10 wage ratio, a measure of wage inequality.

\begin{figure}[htbp]
\centering
\includegraphics[width=0.9\textwidth]{figures/fig8_inequality_es.png}
\caption{Effect on Wage Inequality (90/10 Ratio)}
\label{fig:inequality}
\begin{flushleft}
\small \textit{Notes:} Event study for the ratio of 90th to 10th percentile wages within each state-year. A negative coefficient indicates reduced inequality.
\end{flushleft}
\end{figure}

The inequality event study shows noisy patterns with no clear trend. This is consistent with our finding that effects are similar at the 10th and 90th percentiles, implying no net change in the 90/10 ratio.

%============================================================================
\section{Discussion and Conclusion}\label{sec:conclude}
%============================================================================

\subsection{Summary of Findings}

This paper provides some of the first causal evidence on the effects of state salary transparency laws on wage outcomes in the United States. Using a difference-in-differences design with staggered state adoption, we find that transparency laws are associated with a 4.2\% decrease in weekly earnings, with suggestive evidence of gender wage gap reduction driven by larger declines for male workers.

Our heterogeneity analysis reveals that effects are concentrated among middle-aged and college-educated workers, groups with typically higher wage dispersion. Distributional analysis shows similar-sized effects at the 10th and 90th percentiles ($-5.2\%$ each), suggesting compression from both tails rather than simple top-down compression.

\subsection{Interpretation}

These findings have important implications for policy debates about pay transparency. While proponents argue that transparency will benefit workers by reducing information asymmetries and exposing discriminatory practices, our results suggest the effects may be more complex. Transparency appears to shift bargaining power from workers to employers, compress wage distributions, and potentially reduce average compensation.

The mechanism underlying our findings likely involves a combination of the theoretical channels discussed earlier. The immediate negative effect at $t = 0$ is consistent with the bargaining channel: when employers must post ranges, workers lose the ability to negotiate above those ranges. The growing gender gap reduction over time is consistent with the discrimination channel: transparency exposes unjustified pay differences that employers then correct.

The surprising finding of large negative effects at the 10th percentile may reflect anchoring effects. When employers post salary ranges, the lower bound of those ranges may serve as an anchor that reduces wages even for workers who would otherwise earn more. Alternatively, transparency may reduce wage premiums paid to retain workers, affecting all levels of the distribution.

\subsection{Limitations}

Several caveats are in order. First, significant pre-treatment coefficients raise concerns about the parallel trends assumption. States adopting transparency laws may have been on different wage trajectories before adoption, and observed post-treatment effects may partly reflect continuation of these trends.

Second, our analysis uses state of residence rather than state of employment, potentially introducing measurement error for workers employed in different states or working remotely. Given the rise of remote work during our sample period (2016--2024), this limitation may be increasingly important.

Third, many transparency laws only recently took effect, and longer-term effects may differ from the short-run estimates we report. The partial recovery at $t = 2$ followed by renewed decline at $t = 3$ suggests dynamic effects that warrant continued monitoring.

Fourth, the CPS provides repeated cross-sections rather than panel data, limiting our ability to distinguish composition effects (changes in who is employed) from wage effects (changes in what employed workers earn). If transparency laws affect hiring decisions, observed wage changes may reflect selection.

Fifth, our analysis treats all transparency laws as equivalent, despite variation in employer size thresholds and scope of disclosure requirements. Future work could exploit this variation to estimate dose-response relationships.

\subsection{Policy Implications}

Despite these limitations, our findings have direct policy relevance. Several states are considering transparency legislation, and federal proposals have been introduced in Congress. Our results suggest that policymakers should weigh potential benefits (reduced discrimination, improved information flow) against potential costs (wage compression, reduced bargaining power).

The finding that transparency achieves equity through compression rather than uplift has distributional implications. If the goal is to raise wages for underpaid workers, transparency alone may be insufficient. Complementary policies such as minimum wage increases or collective bargaining protections may be needed to ensure that transparency benefits workers broadly.

The heterogeneous effects by education and age suggest that transparency policies may have different impacts in different labor market segments. Policies targeted at specific industries or occupations may be more effective than blanket mandates.

\subsection{Future Research}

Future research should examine longer-term effects as more data become available, explore effects on job search behavior and labor market matching, and investigate firm-level responses including compliance strategies and location decisions. Matched employer-employee data would enable analysis of within-firm wage changes, addressing composition concerns. Cross-state comparisons of different law features could identify optimal policy design.

Understanding the full general equilibrium effects of salary transparency remains an important area for future work. Our reduced-form estimates capture the net effect of multiple channels; structural estimation could decompose these effects and inform counterfactual policy analysis.

%============================================================================
\newpage
\bibliographystyle{aer}
\begin{thebibliography}{99}

\bibitem[Arnold(2023)]{arnold2023impact}
Arnold, D. (2023). The Impact of Pay Transparency in Job Postings on the Labor Market. Working Paper.

\bibitem[Baker et al.(2019)]{baker2019pay}
Baker, M., Halberstam, Y., Kroft, K., Mas, A., \& Messacar, D. (2019). Pay Transparency and the Gender Gap. NBER Working Paper No. 25834.

\bibitem[Bennedsen et al.(2022)]{bennedsen2022firms}
Bennedsen, M., Simintzi, E., Tsoutsoura, M., \& Wolfenzon, D. (2022). Do Firms Respond to Gender Pay Gap Transparency? \textit{The Journal of Finance}, 77(4), 2051--2091.

\bibitem[Bertrand et al.(2004)]{bertrand2004much}
Bertrand, M., Duflo, E., \& Mullainathan, S. (2004). How Much Should We Trust Differences-in-Differences Estimates? \textit{The Quarterly Journal of Economics}, 119(1), 249--275.

\bibitem[Blau \& Kahn(2017)]{blau2017gender}
Blau, F. D., \& Kahn, L. M. (2017). The Gender Wage Gap: Extent, Trends, and Explanations. \textit{Journal of Economic Literature}, 55(3), 789--865.

\bibitem[Bowles et al.(2007)]{bowles2007social}
Bowles, H. R., Babcock, L., \& Lai, L. (2007). Social Incentives for Gender Differences in the Propensity to Initiate Negotiations. \textit{Organizational Behavior and Human Decision Processes}, 103(1), 84--103.

\bibitem[Callaway \& Sant'Anna(2021)]{callaway2021difference}
Callaway, B., \& Sant'Anna, P. H. (2021). Difference-in-Differences with Multiple Time Periods. \textit{Journal of Econometrics}, 225(2), 200--230.

\bibitem[Card et al.(2012)]{card2012workplace}
Card, D., Mas, A., Moretti, E., \& Saez, E. (2012). Inequality at Work: The Effect of Peer Salaries on Job Satisfaction. \textit{American Economic Review}, 102(6), 2981--3003.

\bibitem[Cullen \& Pakzad-Hurson(2023)]{cullen2023equilibrium}
Cullen, Z., \& Pakzad-Hurson, B. (2023). Equilibrium Effects of Pay Transparency. \textit{Econometrica}, 91(3), 911--959.

\bibitem[de Chaisemartin \& D'Haultf{\oe}uille(2020)]{dechaisemartin2020two}
de Chaisemartin, C., \& D'Haultf{\oe}uille, X. (2020). Two-Way Fixed Effects Estimators with Heterogeneous Treatment Effects. \textit{American Economic Review}, 110(9), 2964--2996.

\bibitem[Goodman-Bacon(2021)]{goodman2021difference}
Goodman-Bacon, A. (2021). Difference-in-Differences with Variation in Treatment Timing. \textit{Journal of Econometrics}, 225(2), 254--277.

\bibitem[Marinescu \& Rathelot(2020)]{marinescu2020wages}
Marinescu, I., \& Rathelot, R. (2020). Mismatch Unemployment and the Geography of Job Search. \textit{American Economic Journal: Macroeconomics}, 12(3), 89--136.

\bibitem[Obloj \& Zenger(2022)]{obloj2022pay}
Obloj, T., \& Zenger, T. (2022). The Influence of Pay Transparency on (Gender) Inequality, Cooperation, and Performance. \textit{Administrative Science Quarterly}, 67(3), 749--790.

\bibitem[Perez-Truglia(2022)]{perez2022gender}
Perez-Truglia, R. (2022). The Effects of Pay Transparency on Worker Welfare and Firm Performance. \textit{Annual Review of Economics}, 14, 453--475.

\bibitem[Stigler(1962)]{stigler1962information}
Stigler, G. J. (1962). Information in the Labor Market. \textit{Journal of Political Economy}, 70(5, Part 2), 94--105.

\bibitem[Sun \& Abraham(2021)]{sun2021estimating}
Sun, L., \& Abraham, S. (2021). Estimating Dynamic Treatment Effects in Event Studies with Heterogeneous Treatment Effects. \textit{Journal of Econometrics}, 225(2), 175--199.

\end{thebibliography}

%============================================================================
\newpage
\appendix
\section{Additional Figures and Tables}

\begin{figure}[htbp]
\centering
\includegraphics[width=0.9\textwidth]{figures/fig1_adoption_map.png}
\caption{State Salary Transparency Law Adoption}
\label{fig:map}
\begin{flushleft}
\small \textit{Notes:} Map shows states with salary transparency laws effective by 2025. States are colored by adoption year. Grey indicates no law adopted as of 2025.
\end{flushleft}
\end{figure}

\begin{figure}[htbp]
\centering
\includegraphics[width=0.9\textwidth]{figures/fig3_cohort_trends.png}
\caption{Average Log Weekly Earnings by Treatment Cohort}
\label{fig:cohort}
\begin{flushleft}
\small \textit{Notes:} Figure plots state-level average log weekly earnings by treatment cohort over time. Dashed vertical lines indicate treatment year for each cohort. Never-treated states shown for comparison.
\end{flushleft}
\end{figure}

\begin{figure}[htbp]
\centering
\includegraphics[width=0.9\textwidth]{figures/fig4_gender_gap_trends.png}
\caption{Gender Wage Gap Over Time by Treatment Status}
\label{fig:gendergap}
\begin{flushleft}
\small \textit{Notes:} Figure plots the male-female log earnings differential (gender wage gap) over time for treated and control states. Higher values indicate larger male wage premium.
\end{flushleft}
\end{figure}

\begin{figure}[htbp]
\centering
\includegraphics[width=0.8\textwidth]{figures/fig5_heterogeneity.png}
\caption{Heterogeneous Treatment Effects by Group}
\label{fig:het}
\begin{flushleft}
\small \textit{Notes:} Figure plots DiD estimates and 95\% confidence intervals by subgroup. Estimates represent the effect of salary transparency laws on log weekly earnings.
\end{flushleft}
\end{figure}

\begin{figure}[htbp]
\centering
\includegraphics[width=0.9\textwidth]{figures/fig6_timeline.png}
\caption{Staggered Adoption Timeline}
\label{fig:timeline}
\begin{flushleft}
\small \textit{Notes:} Figure shows the timeline of state salary transparency law adoption with employer size thresholds.
\end{flushleft}
\end{figure}

\begin{figure}[htbp]
\centering
\includegraphics[width=0.9\textwidth]{figures/fig7_wage_dist.png}
\caption{Wage Distribution in Colorado Before and After Transparency Law}
\label{fig:dist}
\begin{flushleft}
\small \textit{Notes:} Figure shows kernel density of log weekly earnings in Colorado before (2016--2020) and after (2021--2024) implementation of the salary transparency law (effective January 1, 2021).
\end{flushleft}
\end{figure}

\begin{figure}[htbp]
\centering
\includegraphics[width=0.9\textwidth]{figures/fig12_sample_heatmap.png}
\caption{Sample Size by State and Year}
\label{fig:heatmap}
\begin{flushleft}
\small \textit{Notes:} Heatmap shows CPS MORG sample size by state and year for selected states. Darker colors indicate larger samples. Log scale used for color intensity.
\end{flushleft}
\end{figure}

\end{document}
