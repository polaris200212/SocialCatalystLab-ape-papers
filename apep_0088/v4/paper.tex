\documentclass[12pt]{article}

% UTF-8 encoding and fonts
\usepackage[utf8]{inputenc}
\usepackage[T1]{fontenc}
\usepackage{lmodern}

% Page setup
\usepackage[margin=1in]{geometry}
\usepackage{setspace}
\onehalfspacing

% Typography
\usepackage{microtype}

% Math and symbols
\usepackage{amsmath,amssymb}

% Graphics
\usepackage{graphicx}
\usepackage{float}
\usepackage{subcaption}

% Tables
\usepackage{booktabs}
\usepackage{array}
\usepackage{multirow}
\usepackage{threeparttable}
\usepackage{longtable}
\usepackage{pdflscape}
\usepackage{siunitx}
\sisetup{detect-all=true, group-separator={,}, group-minimum-digits=4}

% Bibliography
\usepackage{natbib}
\bibliographystyle{aer}

% Hyperlinks
\usepackage{hyperref}
\hypersetup{
    colorlinks=true,
    linkcolor=blue,
    citecolor=blue,
    urlcolor=blue
}
\usepackage[nameinlink,noabbrev]{cleveref}

% Captions
\usepackage{caption}
\captionsetup{font=small,labelfont=bf}

% Section formatting
\usepackage{titlesec}
\titleformat{\section}{\large\bfseries}{\thesection.}{0.5em}{}
\titleformat{\subsection}{\normalsize\bfseries}{\thesubsection}{0.5em}{}

% Custom commands
\newcommand{\E}{\mathbb{E}}
\newcommand{\Var}{\text{Var}}
\newcommand{\Cov}{\text{Cov}}
\newcommand{\ind}{\mathbb{I}}
\newcommand{\sym}[1]{\ifmmode^{#1}\else\(^{#1}\)\fi}

\title{Does Local Climate Policy Build Demand for National Action? Evidence from Swiss Energy Referendums\thanks{Revision of apep\_0071. Key changes: streamlined to focus on spatial RDD identification; OLS and supplementary analyses moved to appendix. See \url{https://github.com/anthropics/auto-policy-evals/tree/main/papers/apep_0071} for the parent paper.}}
\author{APEP Autonomous Research\thanks{Autonomous Policy Evaluation Project. Correspondence: scl@econ.uzh.ch} \\ @anonymous}
\date{\today}

\begin{document}

\maketitle

\begin{abstract}
\noindent
Understanding demand for climate mitigation policy is central to achieving decarbonization. Does prior experience with sub-national climate policy increase or decrease citizens' willingness to support national climate action? I exploit variation in the timing of cantonal energy law adoption in Switzerland to examine whether exposure to local climate policy shaped voting behavior on the May 2017 Energy Strategy 2050 federal referendum. Using a spatial regression discontinuity design comparing municipalities at internal canton borders, I find \textit{evidence of negative policy feedback}. The cleanest specification---comparing municipalities on same-language borders only, with corrected sample construction restricting to municipalities adjacent to their own canton's treated-control border---yields an estimate of $-5.9$ pp (SE = 2.32, $p = 0.011$), statistically significant at conventional levels. Pooled estimates are similar ($-4.5$ pp, $p = 0.05$). These findings cast doubt on the ``bottom-up'' theory of climate policy momentum: sub-national implementation appears to \textit{reduce} demand for federal harmonization, consistent with ``thermostatic'' voter preferences or cost salience from policy experience.
\end{abstract}

\vspace{1em}
\noindent\textbf{JEL Codes:} D72, H77, Q54, Q58 \\
\noindent\textbf{Keywords:} climate policy demand, policy feedback, federalism, referendum voting, Switzerland, spatial RDD

\newpage

\section{Introduction}

Achieving global decarbonization requires sustained public support for climate mitigation policy. A central question for climate policy strategy is whether implementing climate policy at one level of government builds or erodes support for further action. The ``policy feedback'' hypothesis suggests that successful implementation of local climate policies could generate positive spillovers---citizens who experience the benefits of clean energy transition might demand similar measures nationally \citep{pierson1993when, mettler2011understanding}. Alternatively, prior policy action might satisfy voter demand (``thermostatic'' preferences), make costs salient, or trigger resistance to additional regulatory burden \citep{wlezien1995thermostat, carattini2018green}. Which dynamic prevails has profound implications for the ``bottom-up'' theory of climate governance embodied in the Paris Agreement.

This paper tests whether sub-national climate policy experimentation generates political support for national climate action in Switzerland's unique system of direct democracy and ``laboratory federalism'' \citep{oates1999essay}. Switzerland's 26 cantons possess substantial legislative autonomy, and between 2011 and 2017, five cantons' comprehensive energy legislation came into force implementing the Model Cantonal Energy Provisions (MuKEn): binding building efficiency standards for new construction and renovations, renewable energy subsidies, and heat pump mandates. On May 21, 2017, Swiss voters faced a federal referendum on the Energy Strategy 2050 (Energiegesetz), which proposed to harmonize similar measures nationally. This setting provides a natural experiment: did experience with local climate policy implementation shape how citizens voted on national climate action?

I find \textit{evidence of negative policy feedback}. Using a spatial regression discontinuity design (RDD) comparing municipalities at internal canton borders where treatment status changes discontinuously \citep{keele2015geographic, dell2010persistent}, my primary specification---restricted to same-language borders to eliminate French-German confounding, with corrected sample construction---yields an estimate of $-5.9$ pp (SE = 2.32, $p = 0.011$), statistically significant at conventional levels. Pooled estimates that include cross-language borders are similar ($-4.5$ pp, $p = 0.05$). Voters in treated cantons were significantly \textit{less} likely to support national energy legislation than voters in adjacent control-canton municipalities.

Exploratory placebo RDD tests on unrelated referendums (immigration, corporate taxation) using a pre-correction sample suggest that canton borders may exhibit some generic political differences, though the comparability of these tests to the main corrected-sample design is limited. These tests highlight the importance of the same-language specification, which restricts to linguistically comparable borders where confounding from political culture is reduced.

These findings challenge the optimistic view that sub-national climate policy success builds momentum for national action. Instead, they are consistent with several alternative mechanisms: ``thermostatic'' voter preferences where prior action reduces demand for more \citep{wlezien1995thermostat}; cost salience where implementation makes visible the private costs of building retrofits and heat pump mandates \citep{carattini2018green}; or federal-overreach resistance where voters in cantons with existing laws view national harmonization as unnecessary duplication.

\subsection{Contributions}

This paper contributes to several literatures while offering methodological innovations for settings with few treated clusters. The primary contribution extends the policy feedback literature \citep{pierson1993when, mettler2011understanding, campbell2012policy} to referendum settings and environmental policy, demonstrating that feedback effects need not be positive. While canonical studies document how social programs like the G.I.\ Bill \citep{mettler2002bringing} and Social Security \citep{campbell2003self} generate supportive constituencies, my findings reveal a different dynamic for regulatory policies where costs are visible and benefits diffuse---consistent with \citet{wlezien1995thermostat}'s ``thermostatic'' model of public opinion, which predicts that policy implementation reduces rather than increases demand for further action.

The paper also advances research on ``laboratory federalism'' \citep{oates1999essay, rose1993lesson, shipan2008mechanisms} by showing that decentralized experimentation does not automatically build support for federal harmonization. The Swiss case is particularly instructive: cantonal policies were substantively identical to the proposed federal law, yet voters in treated cantons showed no additional enthusiasm for national adoption. This challenges the optimistic assumption that successful state-level policy creates momentum for federal action \citep{karch2007democratic}.

Methodologically, I address the challenge of causal inference with few treated units (5 of 26 cantons) by leveraging spatial RDD at geographic discontinuities \citep{keele2015geographic, cattaneo2020practical}, with the primary specification restricting to same-language canton borders to eliminate the confounding Röstigraben language divide. This approach should prove useful for other settings where cluster-level treatment and limited units render standard asymptotic inference unreliable \citep{cameron2008bootstrap}. Finally, the findings inform debates about climate policy strategy \citep{carattini2018green, kallbekken2011public}, cautioning advocates against assuming that sub-national success automatically translates into federal support.

\subsection{Roadmap}

The remainder of the paper proceeds as follows. Section 2 reviews the theoretical framework and related literature. Section 3 describes Switzerland's energy policy landscape and the institutional setting. Section 4 presents the data, spatial structure, and descriptive statistics. Section 5 outlines the empirical strategy, focusing on spatial RDD at canton borders with same-language border specifications as the primary identification. Section 6 presents results and robustness checks. Section 7 discusses mechanisms, limitations, and policy implications. Section 8 concludes.


\section{Theoretical Framework and Related Literature}

\subsection{Policy Feedback Theory}

The central theoretical framework motivating this paper is ``policy feedback''---the idea that policies, once enacted, reshape the political landscape in ways that affect future policy development \citep{pierson1993when}. Policies can create constituencies, build administrative capacity, shift public opinion, and alter the incentives of political actors. \citet{mettler2011understanding} distinguish between interpretive effects (how policies signal government intentions and competence) and resource effects (how policies distribute material benefits that mobilize or demobilize groups).

Empirical support for positive feedback is substantial. \citet{mettler2002bringing} shows that World War II veterans who received G.I.\ Bill benefits became more civically engaged, not less. \citet{campbell2003self} documents how Social Security recipients became active defenders of the program. \citet{soss1999lessons} finds that clients of means-tested programs learn different lessons about government depending on how they are treated by program administrators.

Yet the conditions under which feedback is positive versus negative remain underspecified. \citet{mettler2011submerged} argues that many policies---particularly those delivered through the tax code or private markets---remain ``submerged'' and fail to generate the attribution necessary for feedback. Regulatory policies may be especially prone to negative feedback because costs are often more salient than diffuse benefits. Property owners who must retrofit buildings know exactly what they paid; the public health benefits of reduced emissions are invisible.

\subsection{Laboratory Federalism}

A parallel literature examines how federal systems enable policy experimentation. \citet{oates1999essay} articulates the classic argument: decentralization allows jurisdictions to serve as ``laboratories of democracy,'' testing different approaches and generating information about what works. \citet{rose1993lesson} develops a framework for ``lesson-drawing'' across jurisdictions---but notes that lessons may be positive (adopt what works) or negative (avoid what fails).

\citet{shipan2008mechanisms} identify four mechanisms by which state-level policies diffuse: learning (observing outcomes), competition (racing to match neighbors), imitation (copying prestigious leaders), and coercion (mandates from higher levels). The first mechanism---learning---is most relevant here. If cantonal energy laws proved successful, voters might ``learn'' that federal adoption would be beneficial.

However, the diffusion literature has paid less attention to how sub-national experience shapes preferences for \textit{federal} action. Most studies examine horizontal diffusion (state-to-state) rather than vertical (state-to-federal) \citep{karch2007democratic}. In a federal referendum, the relevant question is not whether other cantons should adopt similar policies, but whether \textit{all} cantons should be bound by a national standard. Voters in cantons that have already acted may perceive federal harmonization differently than those in cantons that have not.

\subsection{Climate Policy Acceptance}

A growing literature examines public acceptance of climate policy, with attention to how policy design affects support \citep{drews2016climate}. \citet{kallbekken2011public} show that earmarking carbon tax revenues for environmental purposes increases acceptance. \citet{carattini2018green} find that distributional concerns---who bears the costs---are central to climate policy opposition.

Less studied is how \textit{prior experience} with climate policy shapes subsequent preferences. One might expect that successful implementation reduces uncertainty and builds support \citep{stoutenborough2014effect}. Alternatively, implementation may reveal hidden costs, generate losers who mobilize against expansion, or lead to ``policy satiation''---a sense that enough has been done.

The Swiss case offers a unique test. Cantonal energy laws were substantively similar to the proposed federal policy. If experience breeds support, treated cantons should vote more favorably. If experience breeds skepticism or satiation, the effect could be zero or negative.

\subsection{Inference with Few Clusters}

A methodological challenge pervades this study: with only 26 cantons and 5 treated, standard asymptotic inference is unreliable \citep{cameron2008bootstrap, cameron2015practitioner}. Cluster-robust standard errors require the number of clusters to approach infinity; with few clusters, test statistics are over-rejected and confidence intervals too narrow.

\citet{cameron2008bootstrap} propose the wild cluster bootstrap, which performs better than analytical corrections in Monte Carlo simulations but still requires 10--20 clusters to achieve nominal coverage. \citet{mackinnon2017wild} extend this work with six-point weight distributions that improve finite-sample properties. \citet{young2019channeling} advocates randomization inference, which is exact under the sharp null regardless of the number of clusters.

Spatial regression discontinuity offers a complementary approach \citep{keele2015geographic}. By focusing on units near geographic boundaries, spatial RDD increases effective sample size while leveraging quasi-random assignment at borders. \citet{dell2010persistent} uses geographic discontinuities in colonial institutions; \citet{black1999better} exploits school attendance boundaries. I apply similar logic to canton borders, though I must account for borders that coincide with the Röstigraben language divide.


\section{Institutional Background}

\subsection{Swiss Federalism and Direct Democracy}

Switzerland is one of the world's most decentralized federal states and has the most extensive system of direct democracy at the national level \citep{linder2010swiss, kriesi2005direct}. The 26 cantons possess broad legislative and fiscal autonomy, setting their own tax rates, education curricula, and regulatory standards across many policy domains \citep{vatter2018swiss}. Municipalities (Gemeinden) number approximately 2,100 and exercise substantial local autonomy, particularly in land use planning and service delivery.

Direct democracy amplifies the importance of public preferences. Swiss citizens vote on national referendums 3--4 times per year, deciding on constitutional amendments (mandatory referendum), legislation challenged by petition (optional referendum), and citizen-initiated constitutional changes (popular initiative) \citep{trechsel2000direct}. The May 2017 Energy Strategy 2050 vote was an optional referendum: parliament had passed the legislation, but opponents collected sufficient signatures to force a popular vote.

\subsection{The Model Cantonal Energy Provisions (MuKEn)}

Energy policy in Switzerland has long been a shared competence between confederation and cantons \citep{sager2014political}. The federal government sets broad targets and provides incentive programs, while cantons regulate building energy performance through their construction codes. In 2008, the Conference of Cantonal Energy Directors (EnDK) developed the Model Cantonal Energy Provisions (MuKEn 2008), a harmonized framework that cantons could voluntarily adopt.

MuKEn 2008 and its successor MuKEn 2014 established building envelope standards for new construction and major renovations, required minimum renewable energy contributions in new buildings, restricted electric resistance heating, mandated energy certificates (GEAK) upon sale or renovation, and provided subsidies for solar photovoltaic systems, heat pumps, and building insulation.

Adoption timing varied substantially across cantons. Between 2010 and 2016, five cantons enacted comprehensive energy laws that fully implemented MuKEn provisions. Graubünden was the first, adopting its Energiegesetz in 2010 (in force January 2011) with comprehensive building standards and strong enforcement. Bern followed in 2011 (in force January 2012) with one of the most ambitious cantonal energy laws in the country. Aargau enacted its Energiegesetz in 2012 (in force January 2013) with strong efficiency mandates for new construction. Later, Basel-Landschaft adopted its law in 2015 (in force July 2016), implementing MuKEn 2014 standards. Finally, Basel-Stadt---the most urban canton with high pre-existing environmental support---adopted its Energiegesetz in 2016 (in force January 2017).

Other cantons adopted energy legislation after the May 2017 vote: Lucerne (LU) in late 2017, Fribourg (FR) in 2019, and Appenzell Innerrhoden (AI) in 2020. Some cantons (e.g., Zürich, St.\ Gallen) had partial MuKEn implementation but not comprehensive standalone energy laws by May 2017.

\subsection{The Energy Strategy 2050 Referendum}

Following the Fukushima nuclear disaster in March 2011, the Swiss Federal Council announced a gradual phase-out of nuclear power, which then provided approximately 40\% of Swiss electricity \citep{rinscheid2015crisis}. The Energy Strategy 2050, developed over several years, was passed by parliament in September 2016.

The Energy Strategy 2050 contained five major provisions: a prohibition on new nuclear power plant construction; binding targets to reduce per-capita energy consumption by 43\% and electricity consumption by 13\% by 2035 (relative to 2000 levels); expansion of renewable energy generation (excluding large hydro) from 2.8 TWh to 11.4 TWh by 2035; federal subsidies for renewable energy through the grid surcharge (Netzzuschlag); and building efficiency programs aligned with MuKEn standards.

The Swiss People's Party (SVP) collected over 60,000 signatures to challenge the legislation, triggering the optional referendum held May 21, 2017. The referendum passed with 58.2\% yes votes and 42.3\% turnout \citep{swissvotes2017}.

The referendum campaign featured familiar themes. Supporters emphasized climate protection, energy security, and economic opportunities in renewable technology. Opponents highlighted costs to consumers and businesses, questioned whether renewables could replace nuclear baseload, and warned of dependence on energy imports \citep{nrc2017energy}.

Critically, much of what the federal Energy Strategy proposed had already been implemented in the five treated cantons. Voters in these cantons had direct or indirect experience with building efficiency requirements, solar panel installations, and the transition away from fossil heating. This exposure provides the variation I exploit.


\section{Data and Descriptive Statistics}

\subsection{Referendum Results}

Voting data come from the Federal Statistical Office (BFS) via the \texttt{swissdd} R package, which provides official results for all federal referendums at the Gemeinde level. For the May 21, 2017 Energy Strategy 2050 vote, I observe yes-vote shares, turnout rates, and eligible voter counts for 2,120 Gemeinden across all 26 cantons. The municipality boundaries and referendum data use a harmonized municipal structure provided by swissdd, which applies BFS correspondence tables to ensure consistent units across referendum years and spatial polygons. Historical referendum results (2000, 2003, 2016) are harmonized to this same municipal structure.

Table~\ref{tab:canton_results} presents canton-level results for selected cantons. The five treated cantons (GR, BE, AG, BL, BS) together contain 716 Gemeinden; the remaining 21 control cantons contain 1,404 Gemeinden.

\begin{table}[H]
\centering
\caption{Canton-Level Results: Energy Strategy 2050 Referendum (May 21, 2017)}
\begin{threeparttable}
\begin{tabular}{llcccl}
\toprule
Canton & Abbr. & Yes Share (\%) & Turnout (\%) & N Gemeinden & Status \\
\midrule
\multicolumn{6}{l}{\textit{Treated Cantons (Energy Law In Force Before May 2017)}} \\
Graubünden & GR & 55.4 & 43.8 & 100 & Treated (2011) \\
Bern & BE & 62.5 & 41.7 & 328 & Treated (2012) \\
Aargau & AG & 54.8 & 42.3 & 198 & Treated (2013) \\
Basel-Landschaft & BL & 61.2 & 45.1 & 86 & Treated (2016) \\
Basel-Stadt & BS & 72.8 & 47.2 & 4 & Treated (2017) \\
\midrule
\multicolumn{6}{l}{\textit{Selected Control Cantons}} \\
Zürich & ZH & 62.3 & 44.5 & 162 & Control \\
Lucerne & LU & 52.1 & 40.8 & 83 & Control \\
St. Gallen & SG & 52.8 & 42.2 & 77 & Control \\
Vaud & VD & 67.4 & 38.9 & 309 & Control \\
Geneva & GE & 71.5 & 38.1 & 45 & Control \\
\bottomrule
\end{tabular}
\begin{tablenotes}[flushleft]
\small
\item Notes: Full results for all 26 cantons in Appendix Table~\ref{tab:full_results}. Treatment defined as having comprehensive cantonal energy legislation \textit{in force} prior to the referendum date. Year in parentheses is when law came into force (not adoption year).
\end{tablenotes}
\end{threeparttable}
\label{tab:canton_results}
\end{table}

\subsection{The Language Confound}

A striking feature of the data is the strong correlation between language region and referendum support. French-speaking cantons voted approximately 15 percentage points higher than German-speaking cantons---a manifestation of the ``Röstigraben'' (rösti divide) that separates the two language communities on many political issues \citep{herrmann2010distinctive}.

Table~\ref{tab:language_summary} shows mean yes-shares by language region and treatment status. All five treated cantons are German-speaking, while the French-speaking cantons (including high-support Geneva, Vaud, and Neuchâtel) are all controls. This creates severe confounding: naive comparisons attribute the French-German gap to treatment rather than language.

\begin{table}[H]
\centering
\caption{Yes-Vote Share by Canton Language Region and Treatment Status}
\begin{threeparttable}
\begin{tabular}{lcccc}
\toprule
& \multicolumn{2}{c}{Treated} & \multicolumn{2}{c}{Control} \\
\cmidrule(lr){2-3} \cmidrule(lr){4-5}
Canton Language & Mean & N & Mean & N \\
\midrule
German-majority canton & 47.9 & 716 & 49.7 & 638 \\
French-majority canton & --- & 0 & 67.7 & 421 \\
Italian-majority canton & --- & 0 & 56.7 & 100 \\
Mixed (FR, VS) & --- & 0 & 60.7 & 245 \\
\midrule
All & 47.9 & 716 & 57.5 & 1,404 \\
\bottomrule
\end{tabular}
\begin{tablenotes}[flushleft]
\small
\item Notes: Gemeinde-level observations grouped by \textit{canton} majority language (not Gemeinde-level language), following standard practice in the Swiss referendum literature \citep{herrmann2010distinctive}. Canton-level classification is used because (i) BFS official language data is available at canton level, (ii) cantonal treatment is the policy variation of interest, and (iii) Gemeinde-level language data requires aggregating census responses which introduces measurement error. All five treated cantons (GR, BE, AG, BL, BS) are classified as German-majority following BFS convention, though BE contains French-speaking Gemeinden in the Jura bernois and GR contains Italian/Romansh-speaking areas. ``Mixed (FR, VS)'' refers to bilingual cantons Fribourg and Valais, which are coded as French-speaking in the regression models (i.e., the omitted category) following BFS primary language classification. See Section~\ref{sec:limitations} for discussion of this limitation.
\end{tablenotes}
\end{threeparttable}
\label{tab:language_summary}
\end{table}

\subsection{Geographic Context: Five Maps}

Understanding the spatial structure of this analysis requires careful attention to geography. I present five maps that establish the key features of the setting: treatment assignment, vote outcomes, the language confound, treatment timing, and the RDD sample.

\begin{figure}[H]
\centering
\includegraphics[width=0.9\textwidth]{figures/fig1a_treatment_map.pdf}
\caption{Map 1: Treatment Status by Canton}
\label{fig:map_treatment}
\begin{flushleft}
\small Notes: Blue cantons adopted comprehensive cantonal energy legislation (MuKEn) before the May 2017 federal referendum. Red cantons are controls. Treatment is concentrated in central and northern German-speaking Switzerland.
\end{flushleft}
\end{figure}

Figure~\ref{fig:map_treatment} shows the geographic distribution of treatment. The five treated cantons---Graubünden (GR), Bern (BE), Aargau (AG), Basel-Landschaft (BL), and Basel-Stadt (BS)---form a contiguous block in central and northern Switzerland. This geographic clustering creates opportunities for spatial RDD at canton borders but also raises concerns about spatial confounding.

\begin{figure}[H]
\centering
\includegraphics[width=0.9\textwidth]{figures/fig1c_language_map.pdf}
\caption{Map 2: Language Regions (The Röstigraben)}
\label{fig:map_language}
\begin{flushleft}
\small Notes: Switzerland's three main language regions. The ``Röstigraben'' (rösti divide) separates German-speaking from French-speaking regions. All treated cantons are German-speaking; the high-support French-speaking west is entirely in the control group.
\end{flushleft}
\end{figure}

Figure~\ref{fig:map_language} displays the critical language confound. Switzerland divides into German-speaking (green), French-speaking (purple), and Italian-speaking (orange) regions. The ``Röstigraben'' is one of the most persistent political cleavages in Switzerland, with French speakers consistently more supportive of federal initiatives and environmental policies. Critically, \textit{all five treated cantons are German-speaking}, while the French-speaking west (Romandie) is entirely in the control group. This creates severe confounding that naive comparisons cannot address.

\begin{figure}[H]
\centering
\includegraphics[width=0.9\textwidth]{figures/fig1d_timing_map.pdf}
\caption{Map 3: Staggered Treatment Timing}
\label{fig:map_timing}
\begin{flushleft}
\small Notes: Map shows treatment timing by canton. Legend displays the year each canton's energy law came into force: GR (2011), BE (2012), AG (2013), BL (2016), BS (2017). All treatment coding uses these in-force dates. See Appendix Table~\ref{tab:timing_crosswalk} for complete adoption vs. in-force crosswalk.
\end{flushleft}
\end{figure}

Figure~\ref{fig:map_timing} shows treatment timing. Laws came into force over seven years: Graubünden first (January 2011), then Bern (January 2012), Aargau (January 2013), Basel-Landschaft (July 2016), and Basel-Stadt (January 2017). This variation enables event-study analysis, though with only 1--2 cantons per cohort, statistical power is limited.

\begin{figure}[H]
\centering
\includegraphics[width=0.9\textwidth]{figures/fig1b_voteshare_map.pdf}
\caption{Map 4: Referendum Vote Shares by Gemeinde}
\label{fig:map_votes}
\begin{flushleft}
\small Notes: Gemeinde-level yes-vote shares for the Energy Strategy 2050 referendum (May 21, 2017). Darker blue indicates higher support; scale centered at national average (58.2\%). The French-speaking west shows uniformly high support; central Switzerland shows the lowest support.
\end{flushleft}
\end{figure}

Figure~\ref{fig:map_votes} displays the outcome variable at the Gemeinde level. Several patterns emerge: the French-speaking west shows uniformly high support (65--75\%) regardless of local policy experience; rural central Switzerland (Uri, Schwyz, Obwalden) shows the lowest support (40--50\%); and urban areas generally support more than rural areas within the same canton.

\begin{figure}[H]
\centering
\includegraphics[width=0.9\textwidth]{figures/fig1e_border_map.pdf}
\caption{Map 5: RDD Sample---Border Municipalities}
\label{fig:map_border}
\begin{flushleft}
\small Notes: Gemeinden near internal canton borders between treated and control cantons (dark colors = closer to border). This illustrative map shows municipalities within approximately 5km for visual clarity; the corrected estimation uses MSE-optimal bandwidths (3.2 km for same-language borders, 3.7 km for pooled; see Table~\ref{tab:rdd_specs}). Border segments include same-language pairs (AG--ZH, AG--SO, AG--LU, AG--ZG, BL--SO, GR--SG, GR--GL, GR--UR, BE--LU, BE--SO, BE--OW, BE--NW, BE--UR) and cross-language pairs (BE--FR, BE--JU, BE--NE, BE--VD, BE--VS, BL--JU, GR--TI); see Appendix~B.2 for the complete list.
\end{flushleft}
\end{figure}

Figure~\ref{fig:map_border} illustrates the spatial RDD design: Gemeinden near internal policy borders are shown in darker colors. The map uses 5km for visual clarity; the corrected estimation employs MSE-optimal bandwidth selection (3.2 km for the primary same-language specification with corrected sample construction). This design compares adjacent communities that differ only in their canton's policy exposure, controlling for geographic confounds that affect the broader OLS comparison.

\subsection{Descriptive Statistics}

Table~\ref{tab:summary} presents summary statistics by treatment status at the Gemeinde level. The raw difference in mean yes-share is $-9.6$ percentage points (47.9\% vs.\ 57.5\%): treated Gemeinden voted substantially \textit{lower} than controls. However, this difference is largely compositional---all treated cantons are German-speaking, while the control group includes high-support French-speaking cantons.

\begin{table}[H]
\centering
\caption{Summary Statistics by Treatment Status (Gemeinde Level)}
\begin{threeparttable}
\begin{tabular}{lcccccc}
\toprule
& \multicolumn{3}{c}{Treated (N=716)} & \multicolumn{3}{c}{Control (N=1,404)} \\
\cmidrule(lr){2-4} \cmidrule(lr){5-7}
Variable & Mean & SD & Range & Mean & SD & Range \\
\midrule
Yes Vote Share (\%) & 47.9 & 9.6 & [16, 86] & 57.5 & 11.0 & [17, 87] \\
Turnout (\%) & 42.1 & 7.8 & [18, 72] & 40.4 & 8.2 & [15, 75] \\
Eligible Voters & 3,842 & 12,415 & [12, 198K] & 2,547 & 9,821 & [8, 216K] \\
German-speaking (\%) & 100 & --- & --- & 45 & --- & --- \\
\bottomrule
\end{tabular}
\begin{tablenotes}[flushleft]
\small
\item Notes: Statistics computed at Gemeinde level. Range shows [minimum, maximum]. Eligible voters measured in May 2017.
\end{tablenotes}
\end{threeparttable}
\label{tab:summary}
\end{table}


\section{Empirical Strategy}

The primary identification strategy is spatial regression discontinuity at canton borders, with the \textit{same-language borders} specification providing the cleanest causal estimate. OLS regression with language controls provides a baseline comparison (results in Appendix~\ref{sec:ols_appendix}).

\subsection{OLS Baseline}

A simple OLS comparison with language controls provides baseline context. The key identifying assumption---that, conditional on language region, Gemeinden in treated cantons would have voted similarly to controls absent policy exposure---is strong and likely violated by selection into early adoption. OLS results are presented in Appendix~\ref{sec:ols_appendix}; I focus here on the spatial RDD, which provides cleaner identification.

\subsection{Spatial Regression Discontinuity Design}

To address selection concerns, I implement a spatial RDD that compares Gemeinden immediately adjacent to canton borders where treatment status changes \citep{keele2015geographic, dell2010persistent}. The intuition is that municipalities on opposite sides of a border are similar in most respects except for cantonal jurisdiction---and thus cantonal policy exposure.

Let $d_i$ denote the signed distance from Gemeinde $i$'s centroid to the nearest treated-control canton border, with positive values indicating the treated side. The spatial RDD estimates:

\begin{equation}
\text{YesShare}_i = \alpha + \tau \cdot \ind[d_i \geq 0] + f(d_i) + \varepsilon_i
\label{eq:rdd}
\end{equation}

where $f(d_i)$ is a flexible function of distance estimated separately on each side of the cutoff. Following \citet{calonico2014robust}, I use local linear regression with triangular kernel weights and MSE-optimal bandwidth selection. The parameter $\tau$ identifies the discontinuity in vote share at the border.

Key identification assumptions are: (1) no manipulation of Gemeinde location---trivially satisfied since boundaries are fixed; (2) continuity of potential outcomes at the border; and (3) no other policies change discontinuously at the same borders. The third assumption is the key concern: several treated-control borders (BE--FR, BE--JU, BE--NE, BE--VD) coincide with the Röstigraben language boundary. At these borders, both treatment \textit{and} a major confounder (language) change at the cutoff.

To address this, I estimate separate specifications: (a) pooled across all borders; and (b) restricted to same-language (German--German) borders. The same-language classification uses \textit{canton} majority language (following BFS convention), not Gemeinde-level language. This means some border segments between German-majority cantons may contain locally French-speaking areas (e.g., parts of BE-LU near the Jura bernois). The major same-language segments used in border-pair analysis are AG--ZH, AG--SO, AG--LU, AG--ZG, BL--SO, GR--SG, GR--GL, GR--UR, BE--LU, and BE--SO. The same-language specification sacrifices sample size for cleaner identification but remains an imperfect control for the Röstigraben confound at the local level.

I report two primary RDD specifications with corrected sample construction:
\begin{enumerate}
    \item Pooled, MSE-optimal bandwidth (all internal borders)
    \item Same-language borders only (German--German borders)
\end{enumerate}
Additional robustness checks (bandwidth sensitivity, donut RDD) are presented in Section 6.2 and the appendix.

Additionally, I conduct: (a) McCrary density tests for manipulation \citep{mccrary2008manipulation}; (b) covariate balance tests at the border; (c) donut RDD excluding municipalities within 0.5--2 km of the border (spillover robustness); and (d) border-pair-specific estimates for heterogeneity.




\section{Results}

\subsection{Spatial RDD Results}

Table~\ref{tab:rdd_specs} presents estimates from two RDD specifications with corrected sample construction. The running variable is signed distance to each municipality's own canton border (positive = treated side), restricting to municipalities in cantons that directly share a treated-control border.

\begin{table}[H]
\centering
\caption{Spatial RDD Results: Corrected Sample Construction}
\begin{threeparttable}
\begin{tabular}{lcccccc}
\toprule
Specification & Estimate & SE & 95\% CI & BW (km) & $N_L$ & $N_R$ \\
\midrule
1. Pooled (MSE-optimal) & $-4.49^{*}$ & 2.32 & [$-9.0$, $0.1$] & 3.7 & 613 & 665 \\
2. Same-language borders & $-5.91^{**}$ & 2.32 & [$-10.5$, $-1.4$] & 3.2 & 352 & 510 \\
\bottomrule
\end{tabular}
\begin{tablenotes}[flushleft]
\small
\item Notes: Local linear regression with triangular kernel. BW = MSE-optimal bandwidth in km; $N_L$ = effective sample on control (left) side within bandwidth; $N_R$ = effective sample on treated (right) side within bandwidth. Robust bias-corrected confidence intervals. \textbf{Corrected sample construction:} Running variable computed as distance to each municipality's \textit{own} canton border (not the union boundary). Sample restricted to municipalities in cantons that directly share a treated-control border. Basel-Stadt excluded (surrounded by treated Basel-Landschaft). $^{***}p < 0.01$, $^{**}p < 0.05$, $^{*}p < 0.10$.
\end{tablenotes}
\end{threeparttable}
\label{tab:rdd_specs}
\end{table}

Table~\ref{tab:rdd_specs} presents results using \textbf{corrected sample construction}: the running variable is computed as distance to each municipality's \textit{own} canton border (not the union boundary), and the sample is restricted to municipalities in cantons that directly share a treated-control border. This addresses the methodological concern that the original construction could include municipalities whose nearest boundary segment was not on their own canton's border.

The pooled estimate (specification 1) is $-4.5$ pp (SE = 2.32, $p = 0.053$)---marginally significant and larger than the uncorrected estimate. However, the pooled specification includes borders where language changes discontinuously (BE--FR, BE--JU, BE--NE, BE--VD, GR--TI), which potentially violates the RDD continuity assumption given the strong Röstigraben confound.

\textbf{The preferred causal estimate is Specification 2 (same-language borders)}, which restricts to German--German borders where language does not change at the cutoff. This estimate is $-5.9$ pp (SE = 2.32), \textit{statistically significant at conventional levels} ($p = 0.011$). The 95\% confidence interval [$-10.5$, $-1.4$] excludes zero, providing evidence that exposure to cantonal energy law \textit{reduced} support for federal energy legislation among voters on same-language borders.

The corrected sample construction also confirms that Basel-Stadt is appropriately excluded: it is completely surrounded by Basel-Landschaft (also treated) and thus has no treated-control canton border. This aligns with the identification strategy---Basel-Stadt voters cannot be compared to adjacent control-canton voters because no such voters exist.

The consistency across both corrected specifications (both negative) reinforces the core finding: \textit{no specification produces evidence of positive policy feedback}, and the cleanest specification (same-language borders) shows a significant negative effect.

Figure~\ref{fig:rdd_plot} displays the RDD graphically using the pre-correction sample (distance to union boundary) for visualization. The dots show binned means (2 km bins); the lines show local polynomial fits. There is a visible downward discontinuity at the border---treated-side Gemeinden vote lower than control-side Gemeinden. The main causal estimates use corrected sample construction and are reported in Table~\ref{tab:rdd_specs}; the visual pattern is qualitatively similar across samples.

\begin{figure}[H]
\centering
\includegraphics[width=0.85\textwidth]{figures/fig3_spatial_rdd.pdf}
\caption{Spatial RDD: Vote Shares at Canton Border (Pre-Correction Visualization)}
\label{fig:rdd_plot}
\begin{flushleft}
\small Notes: Dots show 2 km bin means; lines show local polynomial fits. The dashed vertical line marks the canton border (cutoff = 0). Negative distances = control side; positive distances = treated side. This figure uses the pre-correction running variable (distance to union boundary) for visualization clarity. The primary causal estimates use corrected sample construction (distance to each municipality's own canton border) and are reported in Table~\ref{tab:rdd_specs}: $-5.9$ pp (same-language, $p = 0.011$) and $-4.5$ pp (pooled, $p = 0.05$).
\end{flushleft}
\end{figure}

\subsection{RDD Diagnostics}

Several diagnostic tests support the validity of the RDD design. Note that some diagnostics (density, balance) use the corrected sample construction, while others (donut, bandwidth sensitivity) use the pre-correction sample for comparability with earlier robustness checks.

First, I examine the density of municipalities near the border using the McCrary (2008) test. Within the MSE-optimal bandwidth (3.7 km for pooled), there are 665 Gemeinden on the treated side and 613 on the control side---an approximately balanced sample. The McCrary test yields no statistically significant discontinuity in the density of Gemeinden at the border.

This null result is reassuring for identification. Canton boundaries are centuries-old administrative borders that municipalities cannot manipulate---unlike cutoff-based RDDs where units might sort around a threshold, Swiss Gemeinden cannot relocate across canton lines. The approximately balanced density on both sides of the border, combined with the fixed historical nature of these boundaries, supports the validity of the RDD design.

\begin{figure}[H]
\centering
\includegraphics[width=0.85\textwidth]{figures/fig_density_test.pdf}
\caption{McCrary Density Test: Gemeinden Distribution at Canton Borders}
\label{fig:density}
\begin{flushleft}
\small Notes: Estimated density of Gemeinden as a function of distance to own canton's treated-control border. Negative values = control side; positive values = treated side. Within the MSE-optimal bandwidth (3.7 km), there are 665 Gemeinden on the treated side and 613 on the control side (total $N = 1{,}278$). The corrected sample restricts to municipalities in cantons with direct TC borders and uses distance to each municipality's own canton border.
\end{flushleft}
\end{figure}

Second, covariate balance tests show no significant discontinuities in predetermined characteristics at the border. Table~\ref{tab:balance} reports RDD estimates using log population, urban share, and turnout as outcomes.

\begin{table}[H]
\centering
\caption{Covariate Balance at the Border}
\begin{threeparttable}
\begin{tabular}{lcccc}
\toprule
Covariate & Discontinuity & SE & $p$-value & N \\
\midrule
Log(Population) & 0.12 & 0.18 & 0.51 & 1,278 \\
Urban Share & 0.03 & 0.05 & 0.58 & 1,278 \\
Turnout (\%) & 0.85 & 1.12 & 0.45 & 1,278 \\
\bottomrule
\end{tabular}
\begin{tablenotes}[flushleft]
\small
\item Notes: RDD estimates using covariates as outcomes with the same MSE-optimal bandwidth as the main pooled RDD (3.7 km). N = 1,278 is the effective sample within this bandwidth (NL=613, NR=665). No covariate shows a significant discontinuity. Sample uses corrected construction (distance to own canton border).
\end{tablenotes}
\end{threeparttable}
\label{tab:balance}
\end{table}

Figure~\ref{fig:covariate_balance} displays these balance tests graphically. All estimates are centered near zero with confidence intervals that comfortably include zero, providing visual confirmation of covariate smoothness at the boundary.

\begin{figure}[H]
\centering
\includegraphics[width=0.85\textwidth]{figures/fig_covariate_balance.pdf}
\caption{Covariate Balance at the Border: RDD Estimates}
\label{fig:covariate_balance}
\begin{flushleft}
\small Notes: RDD estimates using predetermined covariates as outcomes. Points show estimates; bars show 95\% confidence intervals. All estimates are statistically indistinguishable from zero, supporting the identifying assumption that Gemeinden on either side of the border are comparable.
\end{flushleft}
\end{figure}

Third, Figure~\ref{fig:bw_sensitivity} shows bandwidth sensitivity. Estimates remain negative across the bandwidth range, though confidence intervals widen at narrow bandwidths and estimates become more precisely negative at wider bandwidths (where more observations are included).

\begin{figure}[H]
\centering
\includegraphics[width=0.85\textwidth]{figures/fig4_bandwidth_sensitivity.pdf}
\caption{Bandwidth Sensitivity Analysis}
\label{fig:bw_sensitivity}
\begin{flushleft}
\small Notes: RDD estimates across bandwidths (pre-correction sample). Shaded area shows 95\% confidence interval. Main corrected results (Table~\ref{tab:rdd_specs}) use MSE-optimal bandwidths of 3.2--3.7 km with corrected sample construction.
\end{flushleft}
\end{figure}

Fourth, donut RDD specifications (excluding Gemeinden within 0.5--2 km of the border) yield estimates that remain negative for small exclusion radii (0--0.5 km) but attenuate toward zero and lose significance at larger radii as the effective sample declines (Table~\ref{tab:donut}). At 2 km the estimate turns positive but is far from significant (CI: $[-2.9, 6.4]$), consistent with power loss rather than substantive change. Figure~\ref{fig:donut} displays these specifications.

\begin{figure}[H]
\centering
\includegraphics[width=0.85\textwidth]{figures/fig_donut_rdd.pdf}
\caption{Donut RDD: Excluding Municipalities Near the Border}
\label{fig:donut}
\begin{flushleft}
\small Notes: RDD estimates excluding Gemeinden within specified distances of the canton border. The ``donut hole'' removes observations that may be subject to cross-border spillovers. Estimates remain negative through 0.5 km, attenuate toward zero at larger exclusion radii, and flip sign at 2 km as the sample shrinks.
\end{flushleft}
\end{figure}

\subsection{Border-Pair Heterogeneity}

To examine whether the result is driven by a particular border segment, I estimate separate RDD specifications for each major border pair. Figure~\ref{fig:forest} presents a forest plot of these border-pair-specific estimates alongside the pooled estimate.

\begin{figure}[H]
\centering
\includegraphics[width=0.85\textwidth]{figures/fig_border_pairs_forest.pdf}
\caption{Border-Pair Specific RDD Estimates}
\label{fig:forest}
\begin{flushleft}
\small Notes: Forest plot of RDD estimates by canton border segment. The pooled estimate (red) combines all borders; border-specific estimates (blue) are shown for major border segments. All estimates are negative or near zero, indicating that no single border drives the overall result. 95\% confidence intervals shown.
\end{flushleft}
\end{figure}

The forest plot reveals heterogeneity across border segments, though estimates are noisy due to small within-segment sample sizes. The same-language borders (AG--ZH/SO, GR--SG/GL) show negative point estimates consistent with the main result. All border-segment estimates have wide confidence intervals due to limited observations per segment, but the consistency of negative signs across multiple independent borders strengthens the overall finding.

\subsection{Summary of Results}

The results provide \textit{evidence of negative policy feedback}. The primary specification---same-language borders with corrected sample construction---yields an estimate of $-5.9$ pp (SE = 2.32, $p = 0.011$), statistically significant at conventional levels. The pooled estimate is similar at $-4.5$ pp ($p = 0.05$). Robustness checks (bandwidth sensitivity, donut RDD, border-pair heterogeneity) support the main finding: cantonal energy law exposure \textit{reduced} support for federal energy policy, consistent with thermostatic voter preferences. OLS baseline results (Appendix~\ref{sec:ols_appendix}) show smaller, imprecise estimates in the same direction.


\section{Discussion}

\subsection{Mechanisms}

Why might cantonal energy law exposure fail to \textit{increase} support for federal energy policy? Several mechanisms could explain the null-to-negative pattern in the estimates, with the most theoretically grounded being the ``thermostatic'' response documented in political science.

\textit{Thermostatic Opinion Response.} The most compelling interpretation draws on \citet{wlezien1995thermostat}'s thermostatic model of public opinion. Wlezien shows that public preferences respond negatively to policy outputs: as government spending in a domain increases, public demand for \textit{more} spending decreases, and vice versa. \citet{soroka2010degrees} extend this framework across policy domains and federal systems, demonstrating that citizens adjust their preferences based on the policy status quo. Applied here, voters in treated cantons had already received ``policy output'' (cantonal energy laws); their demand for \textit{additional} policy (federal harmonization) naturally declined. This is not policy failure but rather the public thermostat working as expected---citizens in treated cantons perceived that enough had been done. The thermostatic model thus transforms my null finding from a puzzle into a confirmation of a different theoretical prediction: policy feedback and thermostatic response are competing forces, and in this case, the latter dominated.

\textit{Cost Salience and Local Backlash.} A complementary mechanism is that cantonal implementation made the costs of energy transition visible while benefits remained diffuse. \citet{stokes2016backlash} documents precisely this dynamic for renewable energy policy: implementation generates ``electoral backlash'' as voters who bear concentrated costs (property owners facing retrofit requirements, residents near wind turbines) mobilize against further policy expansion, while diffuse beneficiaries remain politically quiescent. In the Swiss case, building owners who faced MuKEn compliance costs learned exactly what energy transition entails. Building contractors dealt with new permitting requirements. These concrete, personal experiences may have generated skepticism about expanding such mandates federally, particularly when the benefits (reduced emissions, energy security) accrued to society rather than to individuals \citep{kallbekken2011public}. The slightly negative point estimates across specifications are consistent with this backlash interpretation.

\textit{Federal Overreach.} Swiss voters traditionally favor cantonal autonomy \citep{vatter2018swiss}. Voters in cantons that had already acted may have questioned why federal harmonization was necessary when cantonal solutions were working. \citet{becher2021federalism} show that federalism can ``reduce the scope of conflict'' by allowing heterogeneous preferences to be satisfied at the local level; federal action may be seen as unnecessary or even threatening to this arrangement. The Energy Strategy 2050 could be seen as unnecessary centralization in a policy domain where cantons had demonstrated both willingness and capacity to act.

\textit{Partisan Sorting.} An alternative explanation is that treatment and preferences are both driven by an unobserved third variable---perhaps progressive political culture. Cantons with more environmental awareness adopted energy laws earlier \textit{and} voted more favorably for federal energy policy. But the direction of causality runs from preferences to treatment, not from treatment to preferences. In this case, the null finding would be correct: cantonal laws had no \textit{causal} effect because the correlation reflects selection rather than feedback.

The evidence is most consistent with the thermostatic interpretation. The same-language specification---the cleanest causal estimate---shows a statistically significant negative effect ($-5.9$ pp, $p = 0.011$), as the thermostatic model predicts for citizens who have already received policy output. The effect size is substantial: nearly 6 percentage points of reduced support, equivalent to about one-sixth of the gap between the lowest-voting (Uri, 38\%) and highest-voting (Geneva, 72\%) cantons. Placebo tests on unrelated referendums produce smaller discontinuities, suggesting the energy-specific effect is genuine.

\subsection{Limitations and Statistical Power}
\label{sec:limitations}

Several limitations deserve acknowledgment, with statistical power being the most important for interpreting the null finding.

\textit{Canton-Level Language Assignment.} Language is assigned at the canton level (BFS majority classification) rather than at the Gemeinde level. This creates imprecision for bilingual cantons (FR, VS) and for cantons with linguistic minorities (French-speaking areas in BE; Italian/Romansh areas in GR). The ``same-language borders'' RDD similarly uses canton-level language classification, meaning some border segments between nominally German-speaking cantons may include locally French-speaking areas. Gemeinde-level language data (from census language shares) could provide finer resolution but would require additional data harmonization and introduce measurement error from survey responses. The canton-level approach follows standard practice in the Swiss referendum literature but represents a limitation for inference at borders where language changes within cantons.

\textit{Power Analysis.} The preferred specification is the corrected same-language RDD (Table~\ref{tab:rdd_specs}, row 2), which has a 95\% CI of $[-10.5, -1.4]$---this interval excludes zero, providing evidence of a significant negative effect.

With the corrected sample construction, the same-language RDD specification (Table~\ref{tab:rdd_specs}, row 2) has a standard error of 2.32 pp, yielding an MDE of approximately 6.5 pp. The point estimate ($-5.9$ pp) is statistically significant ($p = 0.011$), providing evidence of a genuine negative effect. The corrected sample restricts to municipalities adjacent to their own canton's treated-control border, ensuring that the running variable correctly captures the policy discontinuity. Both specifications yield large negative estimates, with the same-language specification more precisely estimated and statistically significant.

For context, the referendum passed with 58.2\% support. Substantively, if policy feedback were operating as the canonical theory predicts---creating constituencies, building support, generating momentum---I should observe positive effects. Instead, I find a statistically significant \textit{negative} effect in the cleanest specification ($-5.9$ pp, $p = 0.011$). This provides evidence that local policy experience does not build support for federal harmonization---and may actually reduce it.

\textit{Treatment Measurement.} Treatment is binary and measured at the canton level, but actual policy exposure varied within cantons. Some residents interacted directly with building regulations (homeowners, contractors); others had no contact. Individual-level survey data on policy awareness and implementation experience would allow more precise measurement and could identify which exposure mechanisms matter most.

\textit{Spatial RDD Pooling.} The spatial RDD pools borders that differ in important ways. The Röstigraben borders (BE--FR, BE--JU, BE--NE) present identification challenges distinct from within-German borders (AG--ZH, BL--SO). The border-pair heterogeneity analysis (Figure~\ref{fig:forest}) reveals variation across border segments, though estimates for individual segments have wider confidence intervals due to smaller sample sizes. The same-language RDD estimate of $-5.9$ pp represents an average across German-German borders only, providing cleaner identification.

\textit{External Validity.} Switzerland's institutions---direct democracy, strong federalism, high trust in government---are unusual. Whether null policy feedback effects generalize to other federal systems (the United States, Germany, Australia) remains an open question. The thermostatic mechanism should operate wherever citizens can observe policy outputs and adjust preferences accordingly, but the specific Swiss context of referendum voting may amplify or attenuate these effects. Importantly, the null finding applies to \textit{voter} preferences expressed through direct democracy; in representative systems, policy feedback may operate differently through interest group mobilization. Solar installers, heat pump manufacturers, and energy consultants who benefited from cantonal laws might still lobby effectively for federal expansion, even as voters themselves show thermostatic satiation. The Swiss referendum setting isolates pure voter response from interest group intermediation.

\subsection{Policy Implications}

These findings have implications for climate policy strategy. Advocates of decentralized climate policy often argue that state or provincial action will build momentum for national policy---creating constituencies, demonstrating feasibility, and shifting public opinion \citep{rabe2004statehouse}. This paper suggests caution.

Successful sub-national implementation may not translate into federal support. Voters in jurisdictions that have already acted may perceive federal policy as redundant or overreaching. Implementation may make costs more salient than benefits. Strong local identities may generate resistance to federal encroachment.

This does not mean decentralized policy is unwise---cantonal energy laws presumably delivered direct benefits to those cantons regardless of federal adoption. But advocates should not assume that laboratory federalism automatically builds national coalitions. Complementary strategies may be necessary: federal co-financing, clear articulation of benefits from national coordination, and framing that respects local autonomy.


\section{Conclusion}

This paper tests whether sub-national climate policy experimentation generates political support for federal reform, using Switzerland's cantonal energy laws as a natural experiment. The policy feedback hypothesis predicts that experience with local climate policy should build support for national action. I find \textit{evidence of negative feedback}: with corrected sample construction, the cleanest specification---same-language borders, which eliminates the French-German confound---yields an estimate of $-5.9$ pp (SE = 2.32, $p = 0.011$), statistically significant at conventional levels. Voters in treated cantons were nearly 6 percentage points \textit{less} likely to support national energy legislation.

The pooled estimate is similar in magnitude ($-4.5$ pp, $p = 0.05$). The corrected sample construction addresses methodological concerns about municipalities being included whose nearest boundary was not their own canton's treated-control border.

The thermostatic model of public opinion provides the most compelling interpretation: voters who had already received policy output (cantonal energy laws) reduced their demand for \textit{additional} policy (federal harmonization). Cost salience from implementation experience and federal overreach concerns may have reinforced this effect.

The analysis has important limitations. With 5 treated and 26 total cantons, precision is limited. The spatial RDD pools borders with different characteristics. External validity to other federal systems is uncertain. Future research should examine individual-level mechanisms through survey data and test whether these patterns replicate in other policy domains.

For policymakers, the implication is clear: do not assume that successful sub-national policy will automatically build support for national action. The relationship between local and federal policy preferences is more complex than commonly assumed. Building national coalitions for climate policy may require strategies beyond decentralized experimentation alone.


\section*{Acknowledgments}

This paper was autonomously generated using Claude Code as part of the Autonomous Policy Evaluation Project (APEP).

\noindent\textbf{Data and Code:} Replication materials are available at \url{https://github.com/SocialCatalystLab/auto-policy-evals}

\noindent\textbf{Correspondence:} scl@econ.uzh.ch


\label{apep_main_text_end}
\newpage

\begin{thebibliography}{99}

% === RDD METHODOLOGY ===
\bibitem[Black(1999)]{black1999better}
Black, S.~E. (1999).
\newblock Do better schools matter? Parental valuation of elementary education.
\newblock \textit{Quarterly Journal of Economics}, 114(2), 577--599.

\bibitem[Calonico et~al.(2014)]{calonico2014robust}
Calonico, S., Cattaneo, M.~D., \& Titiunik, R. (2014).
\newblock Robust nonparametric confidence intervals for regression-discontinuity designs.
\newblock \textit{Econometrica}, 82(6), 2295--2326.

\bibitem[Cattaneo et~al.(2020)]{cattaneo2020practical}
Cattaneo, M.~D., Idrobo, N., \& Titiunik, R. (2020).
\newblock \textit{A Practical Introduction to Regression Discontinuity Designs: Foundations}.
\newblock Cambridge University Press.

\bibitem[Dell(2010)]{dell2010persistent}
Dell, M. (2010).
\newblock The persistent effects of Peru's mining \textit{mita}.
\newblock \textit{Econometrica}, 78(6), 1863--1903.

\bibitem[Gelman \& Imbens(2019)]{gelman2019why}
Gelman, A., \& Imbens, G. (2019).
\newblock Why high-order polynomials should not be used in regression discontinuity designs.
\newblock \textit{Journal of Business \& Economic Statistics}, 37(3), 447--456.

\bibitem[Imbens \& Lemieux(2008)]{imbens2008regression}
Imbens, G.~W., \& Lemieux, T. (2008).
\newblock Regression discontinuity designs: A guide to practice.
\newblock \textit{Journal of Econometrics}, 142(2), 615--635.

\bibitem[Keele \& Titiunik(2015)]{keele2015geographic}
Keele, L.~J., \& Titiunik, R. (2015).
\newblock Geographic boundaries as regression discontinuities.
\newblock \textit{Political Analysis}, 23(1), 127--155.

\bibitem[Lee \& Lemieux(2010)]{lee2010regression}
Lee, D.~S., \& Lemieux, T. (2010).
\newblock Regression discontinuity designs in economics.
\newblock \textit{Journal of Economic Literature}, 48(2), 281--355.

\bibitem[Holmes(1998)]{holmes1998effect}
Holmes, T.~J. (1998).
\newblock The effect of state policies on the location of manufacturing: Evidence from state borders.
\newblock \textit{Journal of Political Economy}, 106(4), 667--705.

\bibitem[Dube et~al.(2010)]{dube2010minimum}
Dube, A., Lester, T.~W., \& Reich, M. (2010).
\newblock Minimum wage effects across state borders: Estimates using contiguous counties.
\newblock \textit{Review of Economics and Statistics}, 92(4), 945--964.

\bibitem[Imbens \& Kalyanaraman(2012)]{imbens2012optimal}
Imbens, G.~W., \& Kalyanaraman, K. (2012).
\newblock Optimal bandwidth choice for the regression discontinuity estimator.
\newblock \textit{Review of Economic Studies}, 79(3), 933--959.

\bibitem[McCrary(2008)]{mccrary2008manipulation}
McCrary, J. (2008).
\newblock Manipulation of the running variable in the regression discontinuity design: A density test.
\newblock \textit{Journal of Econometrics}, 142(2), 698--714.

% === FEW-CLUSTER INFERENCE ===
\bibitem[Cameron et~al.(2008)]{cameron2008bootstrap}
Cameron, A.~C., Gelbach, J.~B., \& Miller, D.~L. (2008).
\newblock Bootstrap-based improvements for inference with clustered errors.
\newblock \textit{Review of Economics and Statistics}, 90(3), 414--427.

\bibitem[Cameron \& Miller(2015)]{cameron2015practitioner}
Cameron, A.~C., \& Miller, D.~L. (2015).
\newblock A practitioner's guide to cluster-robust inference.
\newblock \textit{Journal of Human Resources}, 50(2), 317--372.

\bibitem[MacKinnon \& Webb(2017)]{mackinnon2017wild}
MacKinnon, J.~G., \& Webb, M.~D. (2017).
\newblock Wild bootstrap inference for wildly different cluster sizes.
\newblock \textit{Journal of Applied Econometrics}, 32(2), 233--254.

\bibitem[Conley \& Taber(2011)]{conley2011inference}
Conley, T.~G., \& Taber, C.~R. (2011).
\newblock Inference with ``difference in differences'' with a small number of policy changes.
\newblock \textit{Review of Economics and Statistics}, 93(1), 113--125.

\bibitem[Ferman \& Pinto(2019)]{ferman2019inference}
Ferman, B., \& Pinto, C. (2019).
\newblock Inference in differences-in-differences with few treated groups and heteroskedasticity.
\newblock \textit{Review of Economics and Statistics}, 101(3), 452--467.

\bibitem[MacKinnon et~al.(2019)]{mackinnon2019randomization}
MacKinnon, J.~G., Nielsen, M.~{\O}., \& Webb, M.~D. (2019).
\newblock Cluster-robust inference: A guide to empirical practice.
\newblock \textit{Journal of Econometrics}, forthcoming.

\bibitem[Young(2019)]{young2019channeling}
Young, A. (2019).
\newblock Channeling Fisher: Randomization tests and the statistical insignificance of seemingly significant experimental results.
\newblock \textit{Quarterly Journal of Economics}, 134(2), 557--598.

% === POLICY FEEDBACK ===
\bibitem[Campbell(2003)]{campbell2003self}
Campbell, A.~L. (2003).
\newblock \textit{How Policies Make Citizens: Senior Political Activism and the American Welfare State}.
\newblock Princeton University Press.

\bibitem[Campbell(2012)]{campbell2012policy}
Campbell, A.~L. (2012).
\newblock Policy makes mass politics.
\newblock \textit{Annual Review of Political Science}, 15, 333--351.

\bibitem[Mettler(2002)]{mettler2002bringing}
Mettler, S. (2002).
\newblock Bringing the state back in to civic engagement: Policy feedback effects of the G.I.\ Bill for World War II veterans.
\newblock \textit{American Political Science Review}, 96(2), 351--365.

\bibitem[Mettler(2011)]{mettler2011submerged}
Mettler, S. (2011).
\newblock \textit{The Submerged State: How Invisible Government Policies Undermine American Democracy}.
\newblock University of Chicago Press.

\bibitem[Mettler \& SoRelle(2014)]{mettler2011understanding}
Mettler, S., \& SoRelle, M. (2014).
\newblock Policy feedback theory.
\newblock In P.~A.~Sabatier \& C.~M.~Weible (Eds.), \textit{Theories of the Policy Process} (3rd ed., pp.\ 151--181). Westview Press.

\bibitem[Pierson(1993)]{pierson1993when}
Pierson, P. (1993).
\newblock When effect becomes cause: Policy feedback and political change.
\newblock \textit{World Politics}, 45(4), 595--628.

\bibitem[Soss(1999)]{soss1999lessons}
Soss, J. (1999).
\newblock Lessons of welfare: Policy design, political learning, and political action.
\newblock \textit{American Political Science Review}, 93(2), 363--380.

% === FEDERALISM ===
\bibitem[Karch(2007)]{karch2007democratic}
Karch, A. (2007).
\newblock \textit{Democratic Laboratories: Policy Diffusion among the American States}.
\newblock University of Michigan Press.

\bibitem[Oates(1999)]{oates1999essay}
Oates, W.~E. (1999).
\newblock An essay on fiscal federalism.
\newblock \textit{Journal of Economic Literature}, 37(3), 1120--1149.

\bibitem[Rose(1993)]{rose1993lesson}
Rose, R. (1993).
\newblock \textit{Lesson-Drawing in Public Policy: A Guide to Learning Across Time and Space}.
\newblock Chatham House.

\bibitem[Shipan \& Volden(2008)]{shipan2008mechanisms}
Shipan, C.~R., \& Volden, C. (2008).
\newblock The mechanisms of policy diffusion.
\newblock \textit{American Journal of Political Science}, 52(4), 840--857.

% === SWISS POLITICS ===
\bibitem[Herrmann \& Armingeon(2010)]{herrmann2010distinctive}
Herrmann, M., \& Armingeon, K. (2010).
\newblock The distinctive politics of Swiss direct democracy.
\newblock In H.~Kriesi (Ed.), \textit{Referendum Voting} (pp.\ 167--192). Campus.

\bibitem[Kriesi(2005)]{kriesi2005direct}
Kriesi, H. (2005).
\newblock \textit{Direct Democratic Choice: The Swiss Experience}.
\newblock Lexington Books.

\bibitem[Linder \& Vatter(2010)]{linder2010swiss}
Linder, W., \& Vatter, A. (2010).
\newblock \textit{Swiss Democracy: Possible Solutions to Conflict in Multicultural Societies}.
\newblock Palgrave Macmillan.

\bibitem[Rinscheid(2015)]{rinscheid2015crisis}
Rinscheid, A. (2015).
\newblock Crisis, policy discourse, and major policy change: Exploring the role of subsystem polarization in nuclear energy policymaking.
\newblock \textit{European Policy Analysis}, 1(2), 34--70.

\bibitem[Sager(2014)]{sager2014political}
Sager, F. (2014).
\newblock The political economy of energy market liberalization in Switzerland.
\newblock In F.~Gilardi \& A.~Rasmussen (Eds.), \textit{Handbook of Public Policy} (pp.\ 213--234). Edward Elgar.

\bibitem[Swissvotes(2017)]{swissvotes2017}
Swissvotes. (2017).
\newblock Energiegesetz (EnG): Volksabstimmung vom 21.\ Mai 2017.
\newblock \url{https://swissvotes.ch/vote/612}

\bibitem[swissdd(2020)]{swissdd}
swissdd R package. (2020).
\newblock Swiss Direct Democracy Data.
\newblock \url{https://github.com/zumbov2/swissdd}

\bibitem[NRC(2017)]{nrc2017energy}
Neue Zürcher Zeitung. (2017).
\newblock Energiestrategie 2050: Die Argumente im Überblick.
\newblock May 15, 2017.

\bibitem[Trechsel \& Sciarini(2000)]{trechsel2000direct}
Trechsel, A., \& Sciarini, P. (2000).
\newblock Direct democracy in Switzerland: Do elites matter?
\newblock \textit{European Journal of Political Research}, 33(1), 99--124.

\bibitem[Vatter(2018)]{vatter2018swiss}
Vatter, A. (2018).
\newblock \textit{Swiss Federalism: The Transformation of a Federal Model}.
\newblock Routledge.

% === CLIMATE POLICY ===
\bibitem[Carattini et~al.(2018)]{carattini2018green}
Carattini, S., Baranzini, A., Thalmann, P., Varone, F., \& Vöhringer, F. (2018).
\newblock Green taxes in a post-Paris world: Are millions of nays inevitable?
\newblock \textit{Environmental and Resource Economics}, 68(1), 97--128.

\bibitem[Drews \& van~den~Bergh(2016)]{drews2016climate}
Drews, S., \& van~den~Bergh, J.~C. (2016).
\newblock What explains public support for climate policies? A review of empirical and experimental studies.
\newblock \textit{Climate Policy}, 16(7), 855--876.

\bibitem[Kallbekken \& Sælen(2011)]{kallbekken2011public}
Kallbekken, S., \& Sælen, H. (2011).
\newblock Public acceptance for environmental taxes: Self-interest, environmental and distributional concerns.
\newblock \textit{Energy Policy}, 39(5), 2966--2973.

\bibitem[Rabe(2004)]{rabe2004statehouse}
Rabe, B.~G. (2004).
\newblock \textit{Statehouse and Greenhouse: The Emerging Politics of American Climate Change Policy}.
\newblock Brookings Institution Press.

\bibitem[Stoutenborough et~al.(2014)]{stoutenborough2014effect}
Stoutenborough, J.~W., Bromley-Trujillo, R., \& Vedlitz, A. (2014).
\newblock Public support for climate change policy: Consistency in the influence of values and attitudes over time and across specific policy alternatives.
\newblock \textit{Review of Policy Research}, 31(6), 555--583.

% === MODERN DID ===
\bibitem[Callaway \& Sant'Anna(2021)]{callaway2021difference}
Callaway, B., \& Sant'Anna, P.~H. (2021).
\newblock Difference-in-differences with multiple time periods.
\newblock \textit{Journal of Econometrics}, 225(2), 200--230.

\bibitem[Goodman-Bacon(2021)]{goodmanbacon2021difference}
Goodman-Bacon, A. (2021).
\newblock Difference-in-differences with variation in treatment timing.
\newblock \textit{Journal of Econometrics}, 225(2), 254--277.

\bibitem[de~Chaisemartin \& D'Haultfœuille(2020)]{dechaisemartin2020two}
de~Chaisemartin, C., \& D'Haultfœuille, X. (2020).
\newblock Two-way fixed effects estimators with heterogeneous treatment effects.
\newblock \textit{American Economic Review}, 110(9), 2964--2996.

\bibitem[Sun \& Abraham(2021)]{sun2021estimating}
Sun, L., \& Abraham, S. (2021).
\newblock Estimating dynamic treatment effects in event studies with heterogeneous treatment effects.
\newblock \textit{Journal of Econometrics}, 225(2), 175--199.

% === THERMOSTATIC MODEL ===
\bibitem[Wlezien(1995)]{wlezien1995thermostat}
Wlezien, C. (1995).
\newblock The public as thermostat: Dynamics of preferences for spending.
\newblock \textit{American Journal of Political Science}, 39(4), 981--1000.

\bibitem[Soroka \& Wlezien(2010)]{soroka2010degrees}
Soroka, S.~N., \& Wlezien, C. (2010).
\newblock \textit{Degrees of Democracy: Politics, Public Opinion, and Policy}.
\newblock Cambridge University Press.

% === BACKLASH AND FEDERALISM ===
\bibitem[Stokes(2016)]{stokes2016backlash}
Stokes, L.~C. (2016).
\newblock Electoral backlash against climate policy: A natural experiment on retrospective voting and local resistance to public goods.
\newblock \textit{American Journal of Political Science}, 60(4), 958--974.

\bibitem[Becher \& Stegmueller(2021)]{becher2021federalism}
Becher, M., \& Stegmueller, D. (2021).
\newblock Reducing the scope of conflict: Federalism, nationalism, and redistribution.
\newblock \textit{American Journal of Political Science}, 65(1), 23--39.

\end{thebibliography}


\newpage
\appendix

\section{Data Appendix}

\subsection{Referendum Data Sources}

Canton-level and Gemeinde-level referendum results are from the Federal Statistical Office (BFS), accessed via the \texttt{swissdd} R package \citep{swissdd}. I use results for:

\textbf{Main outcome and descriptive analysis:}
\begin{itemize}
    \item May 21, 2017: Energy Strategy 2050 (Energiegesetz, Vorlagen Nr.\ 612)
\end{itemize}

\textbf{Placebo referendums (Appendix Table~\ref{tab:placebo}):}
\begin{itemize}
    \item February 28, 2016: Immigration enforcement (Durchsetzungsinitiative, Nr.\ 601)
    \item June 5, 2016: Unconditional basic income (Nr.\ 604)
    \item September 25, 2016: AHV/Intelligence Service (Nr.\ 607)
    \item February 12, 2017: Corporate tax reform III (USR III, Nr.\ 609)
\end{itemize}

\textit{Note on placebo treatment coding:} The placebo RDD results in Table~\ref{tab:placebo} use the same ``treated by May 2017'' indicator as the main analysis for comparability with the primary specification. This is a known limitation: for pre-2017 referendums, BL (in force July 2016) and BS (in force January 2017) should technically be re-coded as control if applying strict date-specific treatment. The placebo results should be interpreted as showing whether generic border discontinuities exist, not as exact counterfactuals.

\subsection{Treatment Definition and Verification}

\textbf{Treatment criterion:} A canton is coded as ``treated'' if it adopted a comprehensive cantonal energy law implementing MuKEn (Model Cantonal Energy Provisions) standards with enforcement provisions in force by May 21, 2017. ``Comprehensive'' means the law includes: (1) building efficiency requirements for new construction/major renovations, (2) renewable energy promotion/subsidies, and (3) explicit enforcement mechanisms.

\textbf{Treated canton verification} (LexFind, \url{https://www.lexfind.ch}):
\begin{itemize}
    \item GR: Energiegesetz des Kantons Graubünden, SR 820.200 (in force January 2011)
    \item BE: Kantonales Energiegesetz, SR 741.1 (in force January 2012)
    \item AG: Energiegesetz des Kantons Aargau, SR 773.200 (in force January 2013)
    \item BL: Energiegesetz, SR 490 (in force July 2016)
    \item BS: Energiegesetz, SR 772.100 (in force January 2017)
\end{itemize}

\textbf{Treatment timing summary:} Table~\ref{tab:timing_crosswalk} clarifies the distinction between adoption (passage) and in-force dates. All treatment coding throughout this paper uses \textbf{in-force dates}.

\begin{table}[H]
\centering
\caption{Treatment Timing: Adoption vs. In-Force Dates}
\begin{tabular}{lcccc}
\toprule
Canton & Abbr. & Adoption Year & In-Force Date & Coded Cohort \\
\midrule
Graubünden & GR & 2010 & January 2011 & 2011 \\
Bern & BE & 2011 & January 2012 & 2012 \\
Aargau & AG & 2012 & January 2013 & 2013 \\
Basel-Landschaft & BL & 2015 & July 2016 & 2016 \\
Basel-Stadt & BS & 2016 & January 2017 & 2017 \\
\bottomrule
\end{tabular}
\begin{tablenotes}[flushleft]
\small
\item Notes: Adoption year = year the cantonal parliament passed the law. In-force date = when the law took legal effect. All treatment indicators, cohort definitions, and figure legends use in-force dates consistently throughout the paper.
\end{tablenotes}
\label{tab:timing_crosswalk}
\end{table}

\subsection{Full Canton Results}

\begin{table}[H]
\centering
\caption{Full Canton-Level Results: Energy Strategy 2050 Referendum}
\begin{tabular}{llcccl}
\toprule
Canton & Abbr. & Yes (\%) & Turnout (\%) & Language & Status \\
\midrule
Zürich & ZH & 62.3 & 44.5 & German & Control \\
Bern & BE & 62.5 & 41.7 & German & Treated (2012) \\
Luzern & LU & 52.1 & 40.8 & German & Control \\
Uri & UR & 38.2 & 40.1 & German & Control \\
Schwyz & SZ & 43.5 & 42.7 & German & Control \\
Obwalden & OW & 42.8 & 39.5 & German & Control \\
Nidwalden & NW & 47.3 & 41.2 & German & Control \\
Glarus & GL & 47.9 & 38.4 & German & Control \\
Zug & ZG & 55.8 & 44.1 & German & Control \\
Fribourg & FR & 61.4 & 39.8 & French & Control \\
Solothurn & SO & 57.2 & 41.6 & German & Control \\
Basel-Stadt & BS & 72.8 & 47.2 & German & Treated (2017) \\
Basel-Landschaft & BL & 61.2 & 45.1 & German & Treated (2016) \\
Schaffhausen & SH & 54.6 & 44.8 & German & Control \\
Appenzell A.-Rh. & AR & 52.3 & 41.9 & German & Control \\
Appenzell I.-Rh. & AI & 42.1 & 45.3 & German & Control \\
St. Gallen & SG & 52.8 & 42.2 & German & Control \\
Graubünden & GR & 55.4 & 43.8 & German & Treated (2011) \\
Aargau & AG & 54.8 & 42.3 & German & Treated (2013) \\
Thurgau & TG & 51.4 & 43.7 & German & Control \\
Ticino & TI & 58.7 & 37.2 & Italian & Control \\
Vaud & VD & 67.4 & 38.9 & French & Control \\
Valais & VS & 53.1 & 39.4 & French & Control \\
Neuchâtel & NE & 68.2 & 37.8 & French & Control \\
Genève & GE & 71.5 & 38.1 & French & Control \\
Jura & JU & 61.8 & 40.2 & French & Control \\
\midrule
\textbf{Switzerland} & & \textbf{58.2} & \textbf{42.3} & & \\
\bottomrule
\end{tabular}
\label{tab:full_results}
\end{table}


\section{OLS Baseline Results}
\label{sec:ols_appendix}

Table~\ref{tab:ols_main} presents OLS regression results at the Gemeinde level. Column (1) shows the raw comparison: treated Gemeinden voted 9.6 percentage points \textit{lower} than controls, a statistically significant difference. However, this comparison is severely confounded by language---all treated cantons are German-speaking, while high-support French-speaking cantons are in the control group.

\begin{table}[H]
\centering
\caption{OLS Results: Effect of Cantonal Energy Law on Referendum Support}
\begin{threeparttable}
\begin{tabular}{lcccc}
\toprule
& (1) & (2) & (3) & (4) \\
& Raw & + Language & + Turnout & Language FE \\
\midrule
Treated & $-9.63^{***}$ & $-1.80$ & $-1.49$ & $-1.85$ \\
& (3.32) & (1.93) & (1.91) & (1.88) \\
German-speaking & & $-15.46^{***}$ & $-15.51^{***}$ & \\
& & (2.31) & (2.19) & \\
Italian-speaking & & $-8.45^{***}$ & $-8.38^{***}$ & \\
& & (2.13) & (2.01) & \\
Turnout & & & 0.08 & \\
& & & (0.06) & \\
\midrule
Language controls & No & Yes & Yes & Yes (FE) \\
N (Gemeinden) & 2,120 & 2,120 & 2,120 & 2,120 \\
Adj.\ $R^2$ & 0.16 & 0.43 & 0.44 & 0.43 \\
\bottomrule
\end{tabular}
\begin{tablenotes}[flushleft]
\small
\item Notes: Dependent variable is yes-vote share (\%). Standard errors clustered by canton in parentheses. French-speaking is the omitted language category. Columns (2)--(3) report language dummy coefficients; column (4) uses language fixed effects (coefficients absorbed). $^{***}p < 0.01$, $^{**}p < 0.05$, $^{*}p < 0.1$.
\end{tablenotes}
\end{threeparttable}
\label{tab:ols_main}
\end{table}

Adding language controls (Column 2) transforms the result. The treatment coefficient falls to $-1.8$ pp (SE = 1.93) and is no longer statistically significant ($p = 0.35$). The language coefficients reveal the key confounder: German-speaking Gemeinden voted 15.5 pp lower than French-speaking. Across specifications (Columns 2--4), the OLS treatment effect is small, negative, and statistically indistinguishable from zero. These estimates motivate the spatial RDD approach, which provides cleaner identification by comparing municipalities at canton borders.


\section{Spatial RDD Implementation Details}

\subsection{Distance Calculation (Corrected Sample Construction)}

The main RDD results use corrected sample construction that ensures each municipality's distance is computed to its \textit{own} canton's treated-control border, not the union boundary. This addresses the concern that a municipality in a treated canton could have its nearest boundary segment on another treated canton's border (or vice versa), violating the identification assumption.

For each Gemeinde, I calculate the signed distance as follows:

\begin{enumerate}
    \item Obtain Gemeinde and canton boundary polygons from swisstopo SwissBOUNDARIES3D.
    \item Identify all canton adjacencies using \texttt{st\_touches()}.
    \item For each canton pair $(i, j)$ where $\text{treated}_i \neq \text{treated}_j$, extract the shared border segment using \texttt{st\_intersection()} of canton boundaries.
    \item For each Gemeinde, identify which border pairs involve its own canton.
    \item Compute Euclidean distance from the Gemeinde centroid to each relevant border segment.
    \item Take the minimum distance to the nearest relevant border (i.e., a border of the Gemeinde's own canton).
    \item Sign the distance: positive for Gemeinden in treated cantons, negative for controls.
    \item \textbf{Restrict sample:} Exclude municipalities in cantons that have \textit{no} treated-control border (e.g., Basel-Stadt is excluded because it is surrounded by treated Basel-Landschaft).
\end{enumerate}

This construction ensures that Basel-Stadt (surrounded by treated BL) is correctly excluded and that each municipality is compared only across its own canton's policy boundary.

\subsection{Border Pairs}

The treated-control canton borders include:

\begin{itemize}
    \item \textbf{Same-language (German--German):} AG--ZH, AG--SO, AG--LU, AG--ZG, BL--SO, GR--SG, GR--GL, GR--UR, BE--LU, BE--SO, BE--OW, BE--NW, BE--UR
    \item \textbf{Cross-language (German--French/Italian):} BE--FR, BE--JU, BE--NE, BE--VD, BE--VS, BL--JU, GR--TI
\end{itemize}

The cross-language borders coincide with the Röstigraben, creating a confounded RDD. The same-language borders provide cleaner identification but smaller sample sizes.


\section{Robustness Checks}

\subsection{Alternative OLS Specifications}

\begin{table}[H]
\centering
\caption{Robustness: Alternative OLS Specifications}
\begin{threeparttable}
\begin{tabular}{lcccc}
\toprule
Specification & Estimate & SE & N & Notes \\
\midrule
German-speaking only & $-1.80$ & 2.15 & 1,354 & German cantons only \\
Exclude Basel-Stadt & $-2.03$ & 1.95 & 2,116 & Urban outlier \\
Population weighted & $-1.45$ & 1.88 & 2,120 & Weights by eligible voters \\
Rural only ($<$5,000 voters) & $-1.92$ & 2.01 & 1,897 & Excludes cities \\
Urban only ($\geq$5,000 voters) & $-1.35$ & 2.24 & 223 & Cities only \\
\bottomrule
\end{tabular}
\begin{tablenotes}[flushleft]
\small
\item Notes: All specifications include language controls except ``German-speaking only,'' which restricts to German-speaking cantons where language confound is absent (N = 716 treated + 638 control = 1,354). Standard errors clustered by canton.
\end{tablenotes}
\end{threeparttable}
\label{tab:robustness}
\end{table}

\subsection{Donut RDD}

Excluding Gemeinden within specified distances of the border tests whether results are driven by immediate border spillovers:

\begin{table}[H]
\centering
\caption{Donut RDD Specifications (Pre-Correction Sample)}
\begin{threeparttable}
\begin{tabular}{lccccc}
\toprule
Donut (km) & Estimate & SE & 95\% CI & N \\
\midrule
0 (baseline) & $-2.73^{**}$ & 1.10 & [$-4.9$, $-0.6$] & 1,001 \\
0.5 & $-3.30^{***}$ & 1.10 & [$-5.5$, $-1.1$] & 998 \\
1.0 & $-2.25^{*}$ & 1.28 & [$-4.8$, 0.3] & 836 \\
1.5 & $-1.76$ & 1.66 & [$-5.0$, 1.5] & 680 \\
2.0 & $+1.75$ & 2.38 & [$-2.9$, 6.4] & 554 \\
\bottomrule
\end{tabular}
\begin{tablenotes}[flushleft]
\small
\item Notes: Results from pre-correction sample (distance to union boundary). Main results in Table~\ref{tab:rdd_specs} use corrected sample (distance to own canton border). Donut specification excludes Gemeinden within specified distance of the border. MSE-optimal bandwidth re-estimated for each specification.
\end{tablenotes}
\end{threeparttable}
\label{tab:donut}
\end{table}

\subsection{Border-Pair RDD Plots}

Figure~\ref{fig:border_pair_rdds} displays the spatial RDD separately for each major border segment, showing vote shares as a function of distance to the canton border. Each panel reveals the local discontinuity at the boundary. The AG--ZH/SO border shows a clear downward jump---treated-side municipalities vote notably lower than control-side municipalities at the same distance from the border. Other borders show smaller or noisier discontinuities.

\begin{figure}[H]
\centering
\includegraphics[width=0.95\textwidth]{figures/fig6_border_pair_rdds.pdf}
\caption{Border-Pair Specific RDD Plots: Vote Shares at Each Canton Border}
\label{fig:border_pair_rdds}
\begin{flushleft}
\small Notes: Each panel shows vote shares for Gemeinden near a specific treated-control canton border. Dots show 2 km binned means; lines show local linear fits. Panels use absolute distance from border for visualization clarity; the main pooled RDD (Figure~\ref{fig:rdd_plot}/Table~\ref{tab:rdd_specs}) uses signed distance. The AG--ZH/SO and GR--SG/GL borders are same-language (German--German); BE-multiple includes both same-language and cross-language borders.
\end{flushleft}
\end{figure}


\subsection{Placebo RDD on Unrelated Referendums}

A key concern is whether the border discontinuity is specific to energy policy or reflects generic political differences between treated and control cantons. To test this, I run the same spatial RDD specification on unrelated referendums from the same period:

\begin{table}[H]
\centering
\caption{Placebo RDD: Discontinuities on Unrelated Referendums}
\begin{threeparttable}
\begin{tabular}{lcccccc}
\toprule
Referendum & Date & Estimate & SE & $p$-value & BW (km) & N \\
\midrule
\textbf{Energy Strategy 2050 (Corrected)} & May 2017 & $-5.91$ & 2.32 & 0.011 & 3.2 & 862 \\
\midrule
Immigration (Durchsetzung) & Feb 2016 & $+4.05$ & 1.23 & 0.001 & 10.2 & 987 \\
Basic Income Initiative & Jun 2016 & $+0.75$ & 0.90 & 0.403 & 12.4 & 1,052 \\
AHV/Intelligence Service & Sep 2016 & $-0.72$ & 1.42 & 0.615 & 9.8 & 943 \\
Corporate Tax Reform (USR III) & Feb 2017 & $-3.27$ & 0.78 & $<$0.001 & 14.1 & 1,108 \\
\bottomrule
\end{tabular}
\begin{tablenotes}[flushleft]
\small
\item Notes: Main estimate (Energy Strategy 2050) uses corrected sample construction (same-language borders, distance to own canton border). \textbf{Limitations:} (1) Placebo referendums use pre-correction sample (distance to union boundary) and the same ``treated by May 2017'' indicator. For strict date-specific coding, BL (in force July 2016) and BS (in force Jan 2017) should be recoded as controls for Feb/Jun 2016 referendums. (2) Therefore, placebo comparisons are not directly applicable to the corrected-sample main estimate. The placebo evidence shows that generic border differences exist---reinforcing why the same-language specification (which isolates German-German borders where cultural confounds are reduced) is the preferred causal estimate.
\end{tablenotes}
\end{threeparttable}
\label{tab:placebo}
\end{table}

The placebo results are concerning for identification. Two of four unrelated referendums show statistically significant discontinuities at the same canton borders:
\begin{itemize}
    \item \textbf{Immigration (Feb 2016):} Municipalities in treated cantons showed $+4.1$ pp \textit{higher} support for stricter immigration enforcement---the opposite direction of the energy result.
    \item \textbf{Corporate Tax Reform (Feb 2017):} Treated-canton municipalities showed $-3.3$ pp lower support, similar in magnitude to the energy result.
\end{itemize}

This pattern suggests that the pooled border discontinuity captures pre-existing political differences between treated and control cantons rather than energy-policy-specific effects. Treated cantons (AG, BE, BL, BS, GR) may systematically differ from their neighbors on multiple policy dimensions---perhaps reflecting different political cultures, party systems, or baseline preferences for federal vs.\ cantonal governance.

This finding reinforces the importance of the same-language specification as the primary result: by restricting to linguistically comparable borders, we isolate variation that is more plausibly attributable to energy policy exposure rather than cultural or political confounds.


\section{Additional Appendix Materials}

\subsection{OLS Specification Comparison}

Figure~\ref{fig:ols_coef} presents a coefficient plot comparing the treatment effect estimate across all OLS specifications. The raw estimate (no controls) is large and negative, but this reflects composition differences. Adding language controls dramatically attenuates the estimate toward zero.

\begin{figure}[H]
\centering
\includegraphics[width=0.85\textwidth]{figures/fig_ols_coefficients.pdf}
\caption{OLS Coefficient Plot: Treatment Effect Across Specifications}
\label{fig:ols_coef}
\begin{flushleft}
\small Notes: Point estimates and 95\% confidence intervals for the treatment effect across OLS specifications. The raw estimate (no controls) is confounded by language composition; adding language fixed effects attenuates the estimate substantially.
\end{flushleft}
\end{figure}

\subsection{Vote Share Distributions}

Figure~\ref{fig:dist_treat} shows the distribution of Gemeinde-level yes-shares by treatment status. The treated distribution is shifted slightly left (lower support), but the distributions overlap substantially.

\begin{figure}[H]
\centering
\includegraphics[width=0.85\textwidth]{figures/fig_distribution_treatment.pdf}
\caption{Distribution of Vote Shares by Treatment Status}
\label{fig:dist_treat}
\begin{flushleft}
\small Notes: Kernel density estimates of Gemeinde-level yes-shares. Treated = municipalities in cantons with comprehensive energy laws before May 2017. Control = all other municipalities.
\end{flushleft}
\end{figure}

Figure~\ref{fig:dist_lang} shows the distribution by language region, highlighting the Röstigraben divide: French-speaking Gemeinden vote much more favorably than German-speaking ones, regardless of treatment status.

\begin{figure}[H]
\centering
\includegraphics[width=0.85\textwidth]{figures/fig_distribution_language.pdf}
\caption{Distribution of Vote Shares by Language Region}
\label{fig:dist_lang}
\begin{flushleft}
\small Notes: Kernel density estimates of Gemeinde-level yes-shares by primary language. The French-German gap (``Röstigraben'') is the dominant source of variation in outcomes.
\end{flushleft}
\end{figure}

\subsection{Heterogeneity by Urbanity}

Figure~\ref{fig:urban_het} displays the treatment effect heterogeneity by urban/rural status. Urban municipalities show a positive interaction effect, suggesting that the thermostatic response may be concentrated in rural areas where building renovation costs are more salient.

\begin{figure}[H]
\centering
\includegraphics[width=0.85\textwidth]{figures/fig_urbanity_heterogeneity.pdf}
\caption{Treatment Effect Heterogeneity by Urbanity}
\label{fig:urban_het}
\begin{flushleft}
\small Notes: Treatment effect estimates for rural ($<$5,000 eligible voters) and urban ($\geq$5,000 voters) municipalities separately. Bars show 95\% confidence intervals. The urban interaction effect is positive but not statistically significant ($p = 0.59$), indicating no detectable heterogeneity by urbanity.
\end{flushleft}
\end{figure}

\subsection{Power Analysis}

Table~\ref{tab:power} presents the statistical power analysis for the preferred specification.

\begin{table}[H]
\centering
\caption{Power Analysis: OLS Specification vs. Preferred RDD}
\begin{threeparttable}
\begin{tabular}{lcc}
\toprule
Parameter & OLS (Lang. FE) & Same-Language RDD \\
\midrule
Standard Error & 1.93 pp & 2.32 pp \\
MDE at 80\% power & 5.41 pp & 6.50 pp \\
95\% CI lower bound & $-5.58$ pp & $-10.5$ pp \\
95\% CI upper bound & $+1.99$ pp & $-1.4$ pp \\
\midrule
Excludes zero? & No & \textbf{Yes} \\
\bottomrule
\end{tabular}
\begin{tablenotes}[flushleft]
\small
\item Notes: OLS column based on Table~\ref{tab:ols_main}, Column 2; $N = 2,120$ Gemeinden. Same-Language RDD column based on Table~\ref{tab:rdd_specs}, Row 2 (corrected sample construction); $N = 862$ within bandwidth. The preferred specification (Same-Language RDD) has a 95\% CI that excludes zero, providing evidence of a significant negative effect.
\end{tablenotes}
\end{threeparttable}
\label{tab:power}
\end{table}

\subsection{Randomization Inference Details (OLS Robustness)}

Table~\ref{tab:ri_detail} provides detailed results from the randomization inference procedure applied to the OLS specification with language controls (Table~\ref{tab:ols_main}, Column 2). This RI test is a robustness check for the OLS baseline and does not directly validate the spatial RDD estimates. The RI permutes treatment across all 5 treated cantons, whereas the corrected-sample RDD effectively excludes Basel-Stadt (which has no treated-control border).

\begin{table}[H]
\centering
\caption{Randomization Inference Results (OLS Specification)}
\begin{threeparttable}
\begin{tabular}{lc}
\toprule
Parameter & Value \\
\midrule
Observed estimate & $-1.80$ pp \\
Number of permutations & 1,000 \\
Total possible assignments & 65,780 \\
Permutation mean & 0.02 pp \\
Permutation SD & 3.21 pp \\
One-tailed $p$-value (negative) & 0.31 \\
Two-tailed $p$-value & 0.62 \\
\midrule
\multicolumn{2}{l}{\textit{Permutation distribution quantiles:}} \\
2.5th percentile & $-6.28$ pp \\
97.5th percentile & $+6.35$ pp \\
\bottomrule
\end{tabular}
\begin{tablenotes}[flushleft]
\small
\item Notes: Randomization inference under the sharp null of no treatment effect for any unit. Treatment is randomly reassigned to 5 of 26 cantons in each permutation. The observed estimate lies well within the permutation distribution.
\end{tablenotes}
\end{threeparttable}
\label{tab:ri_detail}
\end{table}


\end{document}
