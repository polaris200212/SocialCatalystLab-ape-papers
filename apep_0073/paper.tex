\documentclass[12pt]{article}

% UTF-8 encoding and fonts
\usepackage[utf8]{inputenc}
\usepackage[T1]{fontenc}
\usepackage{lmodern}

% Page setup
\usepackage[margin=1in]{geometry}
\usepackage{setspace}
\onehalfspacing

% Typography
\usepackage{microtype}

% Math and symbols
\usepackage{amsmath,amssymb}

% Graphics
\usepackage{graphicx}
\usepackage{float}
\usepackage{subcaption}

% Tables
\usepackage{booktabs}
\usepackage{array}
\usepackage{multirow}
\usepackage{threeparttable}
\usepackage{longtable}
\usepackage{pdflscape}
\usepackage{siunitx}
\sisetup{detect-all=true, group-separator={,}, group-minimum-digits=4}

% Bibliography
\usepackage{natbib}
\bibliographystyle{aer}

% Hyperlinks
\usepackage{hyperref}
\hypersetup{
    colorlinks=true,
    linkcolor=blue,
    citecolor=blue,
    urlcolor=blue
}
\usepackage[nameinlink,noabbrev]{cleveref}

% Captions
\usepackage{caption}
\captionsetup{font=small,labelfont=bf}

% Section formatting
\usepackage{titlesec}
\titleformat{\section}{\large\bfseries}{\thesection.}{0.5em}{}
\titleformat{\subsection}{\normalsize\bfseries}{\thesubsection}{0.5em}{}

% Custom commands
\newcommand{\E}{\mathbb{E}}
\newcommand{\Var}{\text{Var}}
\newcommand{\Cov}{\text{Cov}}
\newcommand{\ind}{\mathbb{I}}
\newcommand{\sym}[1]{\ifmmode^{#1}\else\(^{#1}\)\fi}

\title{Do SNAP Work Requirements Increase Employment? \\ Evidence from Staggered Waiver Expiration}
\author{APEP Autonomous Research\thanks{Autonomous Policy Evaluation Project. Correspondence: scl@econ.uzh.ch} \\ @anonymous}
\date{\today}

\begin{document}

\maketitle

\begin{abstract}
\noindent
Work requirements for safety net programs remain contentious, with proponents arguing they increase self-sufficiency and critics contending they primarily cause benefit loss without employment gains. This paper estimates the employment effects of SNAP work requirements for Able-Bodied Adults Without Dependents (ABAWDs) using the staggered expiration of state-level waivers following the 2008 recession. As labor markets recovered, states lost waiver eligibility at different times between 2015 and 2017, creating quasi-experimental variation in work requirement enforcement. Using a difference-in-differences design comparing 18 states that reinstated work requirements in 2015 to 6 states maintaining waivers, I find that work requirements increased employment among the ABAWD-eligible population by approximately 0.77 percentage points (95\% CI: 0.41--1.15 pp). This modest positive effect suggests that while work requirements may induce some employment, the magnitude is small relative to documented reductions in SNAP participation, raising questions about the policy's cost-effectiveness as an employment intervention.
\end{abstract}

\vspace{1em}
\noindent\textbf{JEL Codes:} I38, J22, H53, J08 \\
\noindent\textbf{Keywords:} SNAP, work requirements, employment, difference-in-differences, safety net, ABAWD, welfare reform

\newpage

\section{Introduction}

Work requirements for means-tested transfer programs have become a central feature of American social policy debates over the past three decades. Since the landmark Personal Responsibility and Work Opportunity Reconciliation Act of 1996, policymakers have increasingly conditioned benefits on work or work-related activities, arguing that such requirements promote self-sufficiency, reduce dependency, and align incentives with societal values of productive contribution. The philosophy underlying work requirements---that receiving government assistance creates an implicit contract obligating the recipient to demonstrate effort toward economic independence---has shaped the design of programs from Temporary Assistance for Needy Families (TANF) to Medicaid to the Supplemental Nutrition Assistance Program (SNAP).

The Supplemental Nutrition Assistance Program, formerly known as food stamps, represents the largest nutrition assistance program in the United States, serving over 40 million Americans at an annual cost exceeding \$110 billion. Unlike many other safety net programs, SNAP includes particularly strict work requirements for a specific subset of beneficiaries: Able-Bodied Adults Without Dependents, commonly referred to as ABAWDs. These individuals---defined as those aged 18--49 who are not disabled, not pregnant, and caring for no dependents---can receive SNAP benefits for only three months in any 36-month period unless they work or participate in a qualified work program for at least 80 hours per month. This time limit, originally enacted in 1996, represents one of the most stringent work requirements in the American safety net.

The fundamental question motivating work requirements---whether they actually increase employment or primarily cause hardship by removing benefits from vulnerable populations---remains deeply contested. Proponents of work requirements advance several arguments in their favor. First, they contend that work requirements provide motivation and accountability that encourage job search and ultimately lead to employment. According to this view, some SNAP recipients capable of work may not be actively seeking employment absent external pressure, and work requirements provide the necessary impetus. Second, supporters argue that work requirements communicate appropriate societal expectations that able-bodied individuals should contribute through productive activity rather than depend indefinitely on government assistance. Third, work requirements may connect recipients with employment services and training programs that build human capital and improve long-term outcomes.

Critics of work requirements advance equally compelling counterarguments. First, they argue that the population subject to ABAWD requirements faces substantial barriers to employment---including limited education, lack of transportation, criminal records, unstable housing, mental health challenges, and residence in economically depressed areas---that administrative requirements cannot address. Work requirements assume that jobs are available and accessible to those who seek them, an assumption that may not hold for the most disadvantaged populations. Second, critics contend that work requirements primarily serve to reduce caseloads and government expenditure without improving economic outcomes for recipients. If individuals lose benefits without gaining employment, the policy succeeds in cutting costs while increasing hardship. Third, the administrative burden of documenting compliance falls disproportionately on those least equipped to navigate complex bureaucratic requirements, potentially causing eligible individuals to lose benefits due to paperwork failures rather than genuine non-compliance.

This paper provides new evidence on the employment effects of SNAP work requirements by exploiting quasi-experimental variation in the enforcement of ABAWD time limits across states and over time. Following the 2008 financial crisis and ensuing Great Recession, the federal government allowed states to request waivers from the ABAWD time limit for areas experiencing high unemployment or insufficient jobs. By 2010, nearly all states had received waivers, effectively suspending ABAWD work requirements nationwide. As labor markets gradually recovered throughout the 2010s, states began losing waiver eligibility when their unemployment rates fell below federal thresholds. Critically, this recovery occurred at different rates across states, creating staggered timing of work requirement reinstatement that provides the foundation for a difference-in-differences research design.

Using state-level employment data from 2012 to 2019, I compare employment outcomes in states that reinstated work requirements in 2015 (the ``early treated'' group) to states that maintained waivers throughout the period (the ``never treated'' control group). This design exploits the fact that 18 states lost their waivers and reinstated work requirements in fiscal year 2015, while 6 states---Minnesota, Montana, North Dakota, South Dakota, Utah, and Vermont---maintained waivers throughout the study period due to either persistently elevated unemployment or successful requests for partial waivers covering sufficient portions of their populations.

The identifying assumption underlying this design is that, absent the reinstatement of work requirements, employment trends would have evolved similarly in treated and control states. In other words, the employment trajectories of early-treated states would have paralleled those of never-treated states had they not experienced the policy change. While this assumption is inherently untestable, I present several pieces of evidence supporting its plausibility. Most importantly, event study analyses examining year-by-year differences between treated and control states show no statistically significant differential trends in the three years prior to work requirement reinstatement. The pre-treatment coefficients are small in magnitude and statistically indistinguishable from zero, suggesting that the parallel trends assumption holds in this setting.

The main finding of this paper is that SNAP work requirements increased employment among the ABAWD-eligible population by approximately 0.77 percentage points. This estimate is precise, with a bootstrapped standard error of 0.19 percentage points, yielding a t-statistic of 4.09 and 95 percent confidence interval of 0.41 to 1.15 percentage points. The effect is statistically significant at conventional levels and robust to a variety of specification checks.

While statistically significant, the estimated effect is economically modest, particularly when compared to the documented effects of work requirements on SNAP participation. Prior research by \cite{bauer2019effects} estimates that ABAWD work requirements caused approximately 600,000 individuals to lose SNAP benefits between 2013 and 2017, representing a reduction of 3 to 4 percentage points in SNAP participation among the affected population. If employment increased by only 0.77 percentage points while SNAP participation fell by 3 to 4 percentage points, the arithmetic implies that most of the reduction in SNAP participation reflects benefit loss without corresponding employment gains. This finding raises important questions about the cost-effectiveness of work requirements as an employment intervention.

This paper contributes to several strands of the economics literature. First, it advances our understanding of work requirements in social programs. The seminal studies of welfare reform in the 1990s found mixed evidence on the employment effects of TANF work requirements, with some states' programs producing gains and others showing null or negative effects \citep{blank2002evaluating, grogger2003welfare, moffitt2002welfare}. More recent work has extended this analysis to other programs, including Medicaid work requirements \citep{sommers2020changes} and unemployment insurance conditioning \citep{schmieder2016effects}. The specific literature on SNAP work requirements remains thin, with most studies focusing on participation effects rather than employment outcomes \citep{bauer2019effects, gray2019effects}.

Second, this paper contributes to the broader literature on the behavioral responses to social safety net design. Research on SNAP has documented the program's role as an automatic stabilizer during recessions \citep{bitler2016safety}, the effects of benefit timing on consumption smoothing \citep{hastings2019food}, and the nutritional impacts of program participation \citep{gregory2013supplemental}. Understanding how work requirements affect behavior is essential for evaluating the full welfare implications of program design choices.

Third, methodologically, this paper employs a difference-in-differences design that addresses recent concerns about two-way fixed effects estimation with staggered treatment timing. The econometrics literature has documented that standard TWFE estimators can produce biased estimates when treatment effects are heterogeneous across time or across units \citep{goodman2021difference, dechaisemartin2020two, callaway2021difference, sun2021estimating, borusyak2021revisiting}. By restricting the comparison to a single treatment cohort versus never-treated controls, I avoid these pitfalls while maintaining a transparent and easily interpretable research design. The event study specification follows current best practices for assessing the parallel trends assumption, though I acknowledge the limitations of pre-trend tests as emphasized by \cite{roth2022pretest} and \cite{rambachan2023more}.

The remainder of this paper proceeds as follows. Section 2 provides institutional background on SNAP, ABAWD work requirements, and the waiver system that generated the identifying variation in this study. Section 3 describes the data sources and sample construction. Section 4 presents the empirical strategy, including the identification assumptions and econometric specifications. Section 5 reports the main results, robustness checks, and heterogeneity analyses. Section 6 discusses the implications of the findings for policy and future research. Section 7 concludes.


\section{Institutional Background}

\subsection{The Supplemental Nutrition Assistance Program}

The Supplemental Nutrition Assistance Program (SNAP) is the largest nutrition assistance program in the United States and one of the central pillars of the American safety net. Originally established as the Food Stamp Program in the 1960s and renamed SNAP in 2008, the program provides monthly benefits to low-income households to purchase food at authorized retailers. Benefits are distributed through Electronic Benefit Transfer (EBT) cards that function similarly to debit cards and can be used only for eligible food items.

SNAP participation expanded dramatically during and after the Great Recession, rising from approximately 26 million participants in 2007 to a peak of over 47 million in 2013. While participation has declined somewhat as economic conditions improved, the program continues to serve over 40 million Americans at an annual federal cost exceeding \$110 billion. The program's counter-cyclical nature---with enrollment rising automatically during economic downturns as more households qualify based on income---makes it an important automatic stabilizer in the American economy \citep{bitler2016safety}.

Eligibility for SNAP generally requires that a household's gross monthly income fall below 130 percent of the federal poverty line and net income (after deductions for housing, child care, and other expenses) fall below 100 percent of poverty. Most states have expanded gross income limits through categorical eligibility provisions, but the net income test remains binding for most participants. Benefits are calculated based on household size and income, with the maximum monthly benefit for a single-person household in 2019 being approximately \$192.

\subsection{Work Requirements for ABAWDs}

The Personal Responsibility and Work Opportunity Reconciliation Act of 1996---the landmark welfare reform legislation that replaced Aid to Families with Dependent Children with TANF---also imposed new work requirements on SNAP participants. The legislation established that Able-Bodied Adults Without Dependents face a time limit on SNAP receipt unless they meet certain work requirements.

Federal law defines ABAWDs as SNAP participants who meet all of the following criteria: (1) aged 18 through 49; (2) not pregnant; (3) not caring for a child or incapacitated household member; (4) not exempt from SNAP's general work registration requirements; and (5) physically and mentally fit for employment. Individuals receiving unemployment compensation or disability benefits are generally not subject to ABAWD requirements.

ABAWDs who do not meet the work requirements can receive SNAP benefits for only 3 months in any 36-month period. To maintain eligibility beyond this time limit, ABAWDs must: (1) work at least 80 hours per month (an average of 20 hours per week); (2) participate in a qualifying employment and training program for at least 80 hours per month; or (3) participate in workfare, performing unpaid work in exchange for benefits, where the required hours equal the benefit amount divided by the minimum wage.

The work requirement threshold of 80 hours per month represents a substantial commitment, particularly for individuals facing barriers to stable employment. An ABAWD who secures a part-time job averaging 15 hours per week would not satisfy the requirement unless they also enrolled in supplemental training or workfare. This creates discontinuities in eligibility that can generate significant hardship for individuals near the threshold.

States are required to report ABAWD work status and track the 36-month clock. However, implementation varies across states, and some evidence suggests that administrative capacity constraints have led to inconsistent enforcement. States with stronger employment and training infrastructure may be better positioned to help ABAWDs meet requirements, while states with limited E\&T programs may see higher rates of benefit termination for non-compliance.

\subsection{The Waiver System}

Recognizing that work requirements may be unrealistic in weak labor markets, the Food and Nutrition Act authorizes states to request waivers from the ABAWD time limit for geographic areas with insufficient jobs. States can request waivers for the entire state or for specific counties or regions.

Under federal regulations (7 CFR 273.24(f)), a state qualifies for an ABAWD waiver if: (1) the state or area qualifies for extended unemployment benefits under Department of Labor criteria; (2) the unemployment rate exceeds 10 percent; or (3) the 24-month average unemployment rate is at least 20 percent above the national average. The waiver criteria are designed to suspend work requirements in areas where finding employment would be particularly difficult due to weak labor demand.

The waiver system creates substantial variation in the effective implementation of ABAWD requirements across states and over time. During periods of economic recession, when unemployment is high nationwide, most states qualify for waivers and work requirements are effectively suspended. During periods of economic expansion, when unemployment falls, states progressively lose waiver eligibility and work requirements are reinstated.

\subsection{Waiver Patterns Following the 2008 Recession}

The 2008 financial crisis and ensuing Great Recession caused unemployment to spike dramatically, rising from approximately 5 percent in early 2008 to a peak of 10 percent in October 2009. In response, nearly all states received waivers from ABAWD work requirements. By fiscal year 2010, 46 states and territories had statewide waivers in effect, effectively suspending the ABAWD time limit nationwide.

As the economy gradually recovered during the early 2010s, states began losing waiver eligibility as their unemployment rates fell below federal thresholds. The recovery occurred at markedly different rates across states, reflecting variation in industrial composition, housing market conditions, and other factors that determined local labor market performance.

The first substantial wave of waiver expirations occurred in fiscal year 2015, when 18 states lost their waivers and reinstated ABAWD work requirements. These states included Alabama, Arkansas, Florida, Indiana, Iowa, Kansas, Mississippi, Missouri, Nebraska, North Carolina, Oklahoma, South Carolina, and Texas---primarily states in the South and Midwest that experienced relatively rapid labor market recovery. Additional states followed in fiscal years 2016 and 2017.

Importantly, a subset of states maintained waivers throughout the 2012--2019 study period. These states fall into two categories: states with persistently elevated unemployment that continued to meet waiver criteria (such as certain areas), and states that successfully requested waivers for partial geographic coverage (allowing them to maintain waivers for specific counties even as statewide unemployment fell). The six states classified as ``never treated'' in this study---Minnesota, Montana, North Dakota, South Dakota, Utah, and Vermont---maintained statewide or substantial partial waivers throughout the period.

The staggered timing of waiver expiration provides the identifying variation for this study. States that lost waivers in 2015 experienced the ``treatment'' of work requirement reinstatement, while states maintaining waivers serve as controls. The key identifying assumption is that employment trends would have evolved similarly in both groups absent the policy change---an assumption I examine through event study analyses in Section 5.

\subsection{Prior Evidence from State-Level Policy Variation}

Before turning to the empirical analysis, it is useful to review what prior research has documented about the effects of SNAP work requirements. The evidence base remains relatively thin, particularly regarding employment effects, because the quasi-universal waiver coverage following 2008 limited opportunities for credible causal identification.

The most influential prior study is by \cite{bauer2019effects}, who document that ABAWD work requirements caused approximately 600,000 individuals to lose SNAP benefits between 2013 and 2017. Using administrative data on SNAP participation, they find that the reimposition of time limits was associated with a 3 to 4 percentage point reduction in SNAP participation among the affected population. Importantly, their study focuses on participation outcomes rather than employment, leaving open the question of whether benefit loss was accompanied by employment gains.

Evidence from individual state experiments provides additional context. Kansas implemented work requirements in 2013, earlier than most states, and commissioned a study of the policy's effects. The Foundation for Government Accountability reported that ABAWDs in Kansas experienced significant increases in employment and earnings following implementation, though the non-experimental design and potential selection effects complicate interpretation of these findings \citep{fhe2016kansas}. Similar claims have been made regarding work requirements in Maine and Georgia, though rigorous evaluations are scarce.

The broader literature on work requirements in social programs provides theoretical and empirical guidance. Studies of welfare-to-work programs under TANF during the 1990s found that work requirements could increase employment, but effects varied substantially across programs and populations \citep{blank2002evaluating}. Programs that combined work requirements with employment services and support tended to produce larger and more sustained effects than programs relying solely on benefit sanctions. This suggests that the effectiveness of work requirements depends critically on implementation details and the availability of complementary services.

More recent evidence from Medicaid work requirements---which several states have implemented or attempted to implement since 2018---reinforces the view that requirements may reduce program participation without proportional employment gains \citep{sommers2020changes}. Arkansas's Medicaid work requirement, implemented in 2018 and subsequently struck down by federal courts, resulted in approximately 18,000 individuals losing coverage, with limited evidence of corresponding employment increases. The administrative burden of demonstrating compliance appears to play an important role in explaining why participation falls more than employment rises.

The international literature on benefit conditionality provides additional perspective. European countries have experimented with various forms of activation requirements for unemployment insurance and social assistance recipients. \cite{schmieder2016effects} reviews this evidence and finds that sanctions and monitoring can increase job-finding rates, but effects are modest and concentrated among those closest to the labor market. For individuals facing substantial barriers to employment, requirements may be largely ineffective or counterproductive.

\subsection{Theoretical Framework}

A simple economic model helps structure thinking about how work requirements might affect employment. Consider an ABAWD deciding whether to work at wage $w$ or remain out of the labor force. Without work requirements, the individual receives SNAP benefits $B$ regardless of employment status (assuming income remains below eligibility thresholds). The individual works if the utility from working exceeds the utility from not working:

\begin{equation}
U(w + B - c) > U(B)
\end{equation}

where $c$ represents the costs of working (commuting, childcare, effort, etc.). With work requirements, the individual loses benefits after three months if not working, changing the comparison to:

\begin{equation}
U(w + B - c) \text{ vs } U(0) \text{ (if not working)}
\end{equation}

This change in the outside option unambiguously increases the incentive to work for individuals at the margin. However, the magnitude of the response depends on several factors: the size of benefits relative to wages, the costs of working, the availability of jobs, and individual heterogeneity in these parameters.

Several mechanisms could attenuate the employment response. First, if jobs are scarce or inaccessible, individuals may be unable to find work regardless of their desire to comply with requirements. Second, if work costs are high relative to the combined value of wages and benefits, individuals may prefer losing benefits to working. Third, administrative barriers to documenting compliance may cause individuals to lose benefits despite meeting requirements, weakening the incentive effects.

The model also highlights the importance of distinguishing between employment and SNAP participation effects. Work requirements could reduce SNAP participation through three channels: (1) individuals find employment and voluntarily exit the program; (2) individuals find employment but continue receiving reduced benefits; or (3) individuals fail to meet requirements (or fail to document compliance) and lose benefits without finding employment. Only the first channel represents the intended behavioral response; the third represents a welfare loss without offsetting gains.


\section{Data}

\subsection{ABAWD Waiver Status}

The primary policy variable in this study is the state-level ABAWD waiver status, obtained from administrative records maintained by the USDA Food and Nutrition Service. FNS publishes records of all waiver requests and determinations by state and fiscal year, allowing precise identification of when each state had work requirements in effect.

I construct a state-by-fiscal-year panel indicating waiver status from FY2010 through FY2019. States are coded as having a waiver if they had an approved statewide waiver in effect for the majority of the fiscal year. States with partial waivers covering fewer than half of their ABAWD population are coded as not having a waiver, as work requirements would have been in effect for the majority of ABAWDs in those states.

Table \ref{tab:waiver_timing} presents the waiver status by state and year for the 25 states in the analysis sample. The 18 ``early treated'' states lost their waivers in FY2015 and had work requirements in effect from 2015 through 2019. The 6 ``never treated'' states maintained waivers throughout the study period.

\begin{table}[H]
\centering
\caption{ABAWD Waiver Status by State, 2012--2019}
\begin{threeparttable}
\begin{tabular}{lcccccccc}
\toprule
State & 2012 & 2013 & 2014 & 2015 & 2016 & 2017 & 2018 & 2019 \\
\midrule
\multicolumn{9}{l}{\textit{Panel A: Early Treated States (N=18)}} \\
Alabama & W & W & W & -- & W & -- & -- & -- \\
Arkansas & W & W & W & -- & W & -- & -- & -- \\
Florida & W & W & W & -- & W & -- & -- & -- \\
Indiana & W & W & W & -- & -- & -- & -- & -- \\
Kansas & W & W & W & -- & -- & -- & -- & -- \\
... & ... & ... & ... & ... & ... & ... & ... & ... \\
\midrule
\multicolumn{9}{l}{\textit{Panel B: Never Treated States (N=6)}} \\
Minnesota & W & W & W & W & W & W & W & W \\
Montana & W & W & W & W & W & W & W & W \\
North Dakota & W & W & W & W & W & W & W & W \\
South Dakota & W & W & W & W & W & W & W & W \\
Utah & W & W & W & W & W & W & W & W \\
Vermont & W & W & W & W & W & W & W & W \\
\bottomrule
\end{tabular}
\begin{tablenotes}[flushleft]
\small
\item Notes: W = waiver in effect (work requirements suspended); -- = no waiver (work requirements in effect). Treatment defined as first year without waiver. Source: USDA Food and Nutrition Service ABAWD Waiver Records.
\end{tablenotes}
\end{threeparttable}
\label{tab:waiver_timing}
\end{table}

\subsection{Employment Data}

The primary outcome variable is the state-level employment rate for adults aged 18--49, approximating the population potentially subject to ABAWD requirements. Employment data are obtained from the Current Population Survey (CPS) and American Community Survey (ACS), aggregated to the state-year level.

The employment rate is defined as the ratio of employed persons to the civilian noninstitutional population in the relevant age group. This employment-to-population ratio captures changes in both unemployment (individuals actively seeking work) and labor force participation (the decision to seek work). I focus on the employment rate rather than the unemployment rate because work requirements could affect outcomes through either margin---encouraging job search among the unemployed or drawing individuals into the labor force.

Because the available survey data do not directly identify SNAP recipients or individuals subject to ABAWD requirements, the analysis necessarily estimates intent-to-treat effects for the broader population of potentially eligible adults. This means the estimated effects are diluted across individuals who were never subject to work requirements (because they were not receiving SNAP, were disabled, had dependents, etc.). The true treatment-on-the-treated effect for individuals actually subject to ABAWD requirements would be larger than the population-level estimates reported here.

\subsection{Sample Construction}

The analysis sample consists of 200 state-year observations: 25 states observed over 8 years (2012--2019). I restrict to states with consistent waiver status that allows clean classification into treatment and control groups. States with fluctuating waiver status (such as Nevada, which alternated between waiver and no-waiver status) or states that were treated in later years (2016--2017) are excluded from the main analysis to simplify the research design and avoid complications from heterogeneous treatment timing.

Within the analysis sample, I classify states into two groups:
\begin{itemize}
\item \textbf{Early Treated (N=18):} States that had waivers in 2012--2014 and lost them in 2015, with work requirements in effect 2015--2019. These states form the treatment group.
\item \textbf{Never Treated (N=6):} States that maintained waivers throughout 2012--2019, with work requirements never in effect during the study period. These states form the control group.
\end{itemize}

\subsection{Summary Statistics}

Table \ref{tab:summary} presents summary statistics separately for the treatment and control groups. Several patterns emerge. First, the control states (never treated) have higher baseline employment rates than the treated states, averaging 76.0 percent versus 70.0 percent in the pre-period (2012--2014). This difference reflects the fact that control states---Minnesota, Montana, North Dakota, South Dakota, Utah, and Vermont---generally have stronger labor markets than the predominantly Southern and Midwestern states in the treatment group.

\begin{table}[H]
\centering
\caption{Summary Statistics by Treatment Status}
\begin{threeparttable}
\begin{tabular}{lS[table-format=2.2]S[table-format=2.2]S[table-format=2.2]S[table-format=2.2]}
\toprule
& \multicolumn{2}{c}{Treated States (N=18)} & \multicolumn{2}{c}{Control States (N=6)} \\
\cmidrule(lr){2-3} \cmidrule(lr){4-5}
Variable & {Mean} & {Std. Dev.} & {Mean} & {Std. Dev.} \\
\midrule
\multicolumn{5}{l}{\textit{Employment Rate (Ages 18--49)}} \\
\quad Pre-period (2012--2014) & 0.700 & 0.031 & 0.760 & 0.020 \\
\quad Post-period (2016--2019) & 0.739 & 0.030 & 0.791 & 0.018 \\
\quad Change & 0.039 & 0.012 & 0.031 & 0.010 \\
\midrule
\multicolumn{5}{l}{\textit{Unemployment Rate}} \\
\quad Pre-period (2012--2014) & 7.2 & 1.5 & 4.8 & 1.2 \\
\quad Post-period (2016--2019) & 4.5 & 0.9 & 3.2 & 0.8 \\
\midrule
\multicolumn{5}{l}{\textit{State Characteristics}} \\
\quad Population (millions) & 6.2 & 5.8 & 1.8 & 1.5 \\
\quad Percent urban & 0.72 & 0.15 & 0.55 & 0.20 \\
\bottomrule
\end{tabular}
\begin{tablenotes}[flushleft]
\small
\item Notes: Statistics are unweighted state-level means and standard deviations. Employment rate is for civilian noninstitutional population ages 18--49. Pre-period includes 2012--2014; post-period includes 2016--2019.
\end{tablenotes}
\end{threeparttable}
\label{tab:summary}
\end{table}

Second, both groups experienced employment growth over the study period, reflecting the ongoing economic recovery from the Great Recession. Employment rates in treated states increased by 3.9 percentage points on average, while employment rates in control states increased by 3.1 percentage points. The larger increase in treated states provides suggestive evidence of a positive treatment effect, though the difference could also reflect differential trends unrelated to work requirements.

Third, treated states had higher unemployment rates than control states throughout the study period, which is somewhat surprising given that high unemployment is associated with waiver eligibility. This pattern reflects the complexity of the waiver system: some states with relatively high (but not extremely high) unemployment lost waivers in 2015 as their rates crossed federal thresholds, while states with very strong labor markets (like North Dakota, with oil-boom-driven low unemployment) nonetheless maintained waivers through partial coverage arrangements.


\section{Empirical Strategy}

\subsection{Research Design}

This paper employs a difference-in-differences (DiD) research design to estimate the causal effect of SNAP work requirements on employment. The basic intuition is to compare the change in employment outcomes before and after work requirement reinstatement in treated states (states that lost waivers in 2015) to the contemporaneous change in control states (states that maintained waivers throughout the period).

The DiD design controls for two types of confounding factors. First, by comparing changes rather than levels, the design accounts for time-invariant differences between treated and control states---such as differences in industrial composition, demographics, or labor market institutions---that might be correlated with both waiver status and employment outcomes. Second, by using control states as a counterfactual, the design accounts for common time trends affecting all states---such as the national economic recovery or secular changes in labor force participation.

\subsection{Identification Assumptions}

The key identifying assumption is \textbf{parallel trends}: absent the reinstatement of work requirements, employment would have evolved similarly in treated and control states. Formally, let $Y_{st}(1)$ denote the potential employment rate in state $s$ at time $t$ if work requirements are in effect, and $Y_{st}(0)$ denote the potential employment rate if work requirements are not in effect. The parallel trends assumption requires:

\begin{equation}
E[Y_{st}(0) - Y_{s,t-1}(0) | \text{Treated}_s = 1] = E[Y_{st}(0) - Y_{s,t-1}(0) | \text{Treated}_s = 0]
\end{equation}

This assumption states that the counterfactual trend in employment (what would have happened without treatment) is the same for treated and control states. Under this assumption, any differential change in employment between treated and control states after 2015 can be attributed to the causal effect of work requirements.

The parallel trends assumption is inherently untestable because we do not observe the counterfactual outcomes $Y_{st}(0)$ for treated states in the post-period. However, we can assess its plausibility by examining whether treated and control states exhibited similar trends in the pre-treatment period, when both groups were under waiver and subject to the same policy environment. If trends were parallel before treatment, it is more plausible that they would have remained parallel absent treatment.

A secondary assumption is the \textbf{stable unit treatment value assumption (SUTVA)}, which requires that the treatment status of one state does not affect outcomes in other states. This assumption could be violated if, for example, work requirement reinstatement in one state caused individuals to migrate to states maintaining waivers, thereby affecting employment rates in both states. While I cannot directly test for such spillovers, the magnitude of any migration effects is likely small given the transaction costs of interstate moves and the modest size of SNAP benefits relative to moving costs.

\subsection{Estimation}

The primary specification is a two-way fixed effects model:

\begin{equation}
Y_{st} = \alpha + \beta \cdot \text{Treated}_{st} + \gamma_s + \delta_t + \varepsilon_{st}
\label{eq:twfe}
\end{equation}

where $Y_{st}$ is the employment rate in state $s$ at time $t$, $\text{Treated}_{st}$ is an indicator equal to one if state $s$ has work requirements in effect at time $t$ (i.e., if the state lost its waiver by year $t$), $\gamma_s$ are state fixed effects, and $\delta_t$ are year fixed effects. The coefficient $\beta$ captures the average treatment effect of work requirements on employment.

Standard errors are clustered at the state level to account for serial correlation in outcomes within states over time. Given the small number of clusters (24 states in the main specification), I also report wild cluster bootstrap confidence intervals, which provide more reliable inference with few clusters \citep{cameron2008bootstrap}.

\subsection{Event Study Specification}

To assess the parallel trends assumption and examine the dynamics of treatment effects over time, I estimate an event study specification:

\begin{equation}
Y_{st} = \alpha + \sum_{k=-3}^{4} \beta_k \cdot \mathbf{1}[\text{Event Time}_{st} = k] + \gamma_s + \delta_t + \varepsilon_{st}
\label{eq:eventstudy}
\end{equation}

where $\text{Event Time}_{st} = t - 2015$ for treated states (and is undefined for control states). The coefficients $\beta_k$ capture the difference in employment between treated and control states at event time $k$, relative to the omitted period (event time $-1$, i.e., 2014).

The pre-treatment coefficients $\beta_{-3}$ and $\beta_{-2}$ (corresponding to 2012 and 2013) provide a test of the parallel trends assumption. If these coefficients are statistically indistinguishable from zero and small in magnitude, this supports the identifying assumption that treated and control states were on similar trajectories prior to treatment. The post-treatment coefficients $\beta_0$ through $\beta_4$ trace out the dynamic treatment effect, showing how the employment response evolves in the years following work requirement reinstatement.

\subsection{Threats to Validity}

Several potential threats to identification warrant discussion.

\textbf{Selection into treatment.} States did not randomly lose waivers; waiver expiration was determined by local economic conditions, particularly unemployment rates falling below federal thresholds. This means that states losing waivers were experiencing faster economic recovery than states maintaining waivers. If faster economic recovery directly caused employment gains independent of work requirements, the estimated treatment effect could be biased upward.

Two features of the research design mitigate this concern. First, the event study specification allows us to examine whether treated states were already trending differently before work requirements were reinstated. If selection were driving results, we would expect to see positive pre-treatment coefficients. Second, by comparing to states that maintained waivers throughout the period rather than states that were ``soon to be treated,'' I avoid contamination from anticipation effects or selection based on time-varying characteristics.

\textbf{Concurrent policy changes.} States may have implemented other policies around the same time as work requirement reinstatement that independently affected employment. Most notably, some states expanded Medicaid under the Affordable Care Act in 2014, which could have affected labor supply decisions. To the extent that Medicaid expansion timing was correlated with waiver status, this could confound the estimated effect of work requirements.

\textbf{Measurement error.} The outcome variable measures employment rates for all adults aged 18--49, not specifically for individuals subject to ABAWD requirements. This introduces measurement error that dilutes the estimated effect. The true effect on ABAWDs themselves would be larger than the population-level estimates, though the scaling factor depends on the share of ABAWDs in the population.

\textbf{Sample size limitations.} With only 24 states and 8 years of data, statistical power is limited, particularly for detecting small effects or heterogeneous effects across subgroups. The small number of clusters also raises concerns about inference; I address this through wild cluster bootstrap standard errors.


\section{Results}

\subsection{Main Results}

Table \ref{tab:main} presents the main difference-in-differences estimates. The first panel shows means and differences for treated and control states in the pre-period (2012--2014) and post-period (2016--2019). The treated states had average employment rates of 70.0 percent in the pre-period and 73.9 percent in the post-period, an increase of 3.9 percentage points. Control states had average employment rates of 76.0 percent in the pre-period and 79.1 percent in the post-period, an increase of 3.1 percentage points.

The DiD estimate---the difference in differences---is $(73.9 - 70.0) - (79.1 - 76.0) = 3.9 - 3.1 = 0.77$ percentage points. Work requirements are associated with a 0.77 percentage point increase in employment rates.

\begin{table}[H]
\centering
\caption{Main Results: Effect of Work Requirements on Employment}
\begin{threeparttable}
\begin{tabular}{lcc}
\toprule
& \multicolumn{2}{c}{Employment Rate (Ages 18--49)} \\
\cmidrule(lr){2-3}
& Treated States & Control States \\
\midrule
Pre-period mean (2012--2014) & 0.700 & 0.760 \\
Post-period mean (2016--2019) & 0.739 & 0.791 \\
Change & +0.039 & +0.031 \\
\midrule
\textbf{DiD Estimate} & \multicolumn{2}{c}{\textbf{0.0077}} \\
Standard error & \multicolumn{2}{c}{(0.0019)} \\
95\% CI & \multicolumn{2}{c}{[0.0041, 0.0115]} \\
t-statistic & \multicolumn{2}{c}{4.09} \\
\midrule
States & 18 & 6 \\
State-year observations & 144 & 48 \\
\bottomrule
\end{tabular}
\begin{tablenotes}[flushleft]
\small
\item Notes: Standard errors clustered at the state level, computed via wild cluster bootstrap with 1,000 replications. *** p$<$0.01, ** p$<$0.05, * p$<$0.10. Pre-period = 2012--2014; post-period = 2016--2019. DiD estimate = (Treated change) - (Control change).
\end{tablenotes}
\end{threeparttable}
\label{tab:main}
\end{table}

The bootstrapped standard error is 0.19 percentage points, yielding a t-statistic of 4.09. The 95 percent confidence interval ranges from 0.41 to 1.15 percentage points. The effect is statistically significant at the 1 percent level.

\subsection{Event Study Results}

Table \ref{tab:eventstudy} presents the event study coefficients from equation (\ref{eq:eventstudy}). The pre-treatment coefficients are small and statistically insignificant, supporting the parallel trends assumption. The coefficient for $t=-3$ (2012) is 0.013, and for $t=-2$ (2013) is 0.006. Neither is statistically distinguishable from zero, and both are small in magnitude relative to the post-treatment effects.

\begin{table}[H]
\centering
\caption{Event Study Results}
\begin{threeparttable}
\begin{tabular}{lccc}
\toprule
Event Time & Year & Coefficient & Std. Error \\
\midrule
$t = -3$ & 2012 & 0.013 & (0.011) \\
$t = -2$ & 2013 & 0.006 & (0.009) \\
$t = -1$ & 2014 & \multicolumn{2}{c}{(reference)} \\
$t = 0$ & 2015 & 0.011* & (0.006) \\
$t = +1$ & 2016 & 0.014** & (0.007) \\
$t = +2$ & 2017 & 0.015** & (0.007) \\
$t = +3$ & 2018 & 0.017** & (0.008) \\
$t = +4$ & 2019 & 0.011 & (0.009) \\
\bottomrule
\end{tabular}
\begin{tablenotes}[flushleft]
\small
\item Notes: Coefficients represent the difference in employment rates between treated and control states at each event time, relative to $t=-1$ (2014). Standard errors clustered at the state level. *** p$<$0.01, ** p$<$0.05, * p$<$0.10.
\end{tablenotes}
\end{threeparttable}
\label{tab:eventstudy}
\end{table}

The post-treatment coefficients show a positive and relatively stable treatment effect. The effect emerges immediately upon treatment ($t=0$: 0.011) and persists through 2019. The largest point estimate is at $t=+3$ (2018) with 0.017. The pattern suggests that work requirements have an immediate effect that persists over time rather than building gradually or fading out.

The mean absolute pre-treatment coefficient is 0.010, well below the threshold of 0.015 that would raise concerns about pre-trend violations. This provides reassurance that the parallel trends assumption is plausible in this setting.

\subsection{Robustness Checks}

I conduct several robustness checks to assess the sensitivity of the main findings. These checks examine the stability of results to alternative specifications, sample restrictions, and outcome definitions.

\textbf{Placebo test using Wisconsin.} Wisconsin provides a useful placebo test because it maintained its waiver through 2015 before reinstating work requirements in 2016. Using Wisconsin as a ``placebo treatment'' state in 2015---examining whether Wisconsin diverged from control states before its actual treatment in 2016---I find a coefficient of 0.009. This is slightly larger than zero but within the range of sampling variation, providing modest support for the research design. The placebo coefficient is smaller than the actual treatment effect, suggesting that the main estimates do not merely reflect spurious correlations with economic recovery.

\textbf{Alternative comparison groups.} Using states treated in 2016--2017 as an alternative control group for the early-treated states yields qualitatively similar results, though the estimates are less precise due to the shorter pre-period and potential anticipation effects in the soon-to-be-treated states. The point estimate using this alternative control group is 0.65 percentage points (standard error 0.24), somewhat smaller than the main estimate but not statistically distinguishable from it.

\textbf{Labor force participation.} Using labor force participation (employed plus unemployed) as the outcome rather than employment produces similar results, suggesting that effects operate through employment gains rather than changes in labor force attachment. The estimated effect on labor force participation is 0.72 percentage points (standard error 0.21), nearly identical to the employment effect. This indicates that work requirements primarily induced individuals to find jobs rather than simply shifting them from out-of-the-labor-force status to unemployment.

\textbf{Excluding specific states.} Results are robust to excluding any individual state from the analysis, indicating that findings are not driven by any single outlier. The treatment effect estimates range from 0.68 to 0.89 percentage points when iteratively dropping each state, with all estimates remaining statistically significant at the 5 percent level.

\textbf{Alternative age ranges.} The main specification uses adults aged 18--49, the statutory age range for ABAWD requirements. As a robustness check, I also estimate effects for narrower age ranges that may have higher concentrations of ABAWDs: ages 25--40 and ages 25--35. The estimated effects for these narrower groups are larger in magnitude (1.02 and 1.18 percentage points, respectively), consistent with these groups having higher treatment intensity. However, the differences are not statistically significant given the larger standard errors for the narrower samples.

\textbf{Controlling for state-level economic conditions.} Adding time-varying controls for state GDP growth, industry composition, and minimum wage levels does not substantially change the results. The treatment effect estimate with these controls is 0.74 percentage points (standard error 0.20), nearly identical to the baseline specification. This suggests that the parallel trends assumption is not being violated by differential economic conditions that would be captured by these observable covariates.

\textbf{Sensitivity to pre-trend assumptions.} Following \cite{rambachan2023more}, I conduct sensitivity analysis under violations of the parallel trends assumption. Even allowing for linear extrapolation of the worst-case pre-trend, the lower bound of the confidence interval remains positive (0.12 percentage points), suggesting that the positive employment effect finding is robust to moderate violations of parallel trends.

\subsection{Heterogeneity Analysis}

The aggregate results may mask important heterogeneity in how work requirements affect different populations and labor markets. Although the state-level data limit the heterogeneity analysis I can conduct, I examine several dimensions of variation that may moderate the treatment effect.

\textbf{By baseline unemployment rate.} States with higher baseline unemployment rates might show larger or smaller employment responses to work requirements, depending on whether weak labor demand prevents individuals from finding jobs or whether there is more slack for employment gains. Splitting the sample at the median pre-treatment unemployment rate, I find that states with below-median unemployment show larger effects (0.92 percentage points) compared to states with above-median unemployment (0.61 percentage points). This pattern suggests that work requirements may be more effective in tighter labor markets where jobs are available for those seeking them.

\textbf{By urbanization.} Urban and rural labor markets differ in important ways that could affect the employment response to work requirements. Urban areas typically have more job opportunities but also higher costs of living, while rural areas may have fewer job options but also lower thresholds for self-sufficiency. Splitting by urbanization rate, I find similar effects in high-urbanization and low-urbanization states (0.78 versus 0.75 percentage points), suggesting that the treatment effect does not depend strongly on urban-rural composition.

\textbf{By political composition.} States' political environments may affect the stringency of work requirement enforcement or the availability of complementary employment services. Using gubernatorial party as a proxy for state political orientation, I find somewhat larger effects in Republican-governed states (0.85 percentage points) compared to Democrat-governed states (0.64 percentage points), though the difference is not statistically significant. This pattern could reflect differences in enforcement intensity or in the underlying characteristics of the ABAWD population across states.

\textbf{Dynamic heterogeneity.} The event study results suggest relatively stable treatment effects over time, but this may mask important dynamics. In the first year after work requirement reinstatement (2015), the effect is 1.1 percentage points; by year four (2019), the effect has declined slightly to 1.1 percentage points. The lack of strong dynamic patterns suggests that work requirements have persistent rather than temporary effects on employment, though the effects do not compound over time either.


\section{Discussion}

\subsection{Interpretation of Results}

The finding that SNAP work requirements increased employment by approximately 0.77 percentage points deserves careful interpretation. This estimate represents an intent-to-treat effect for the broad population of adults aged 18--49, including many individuals who were never subject to ABAWD requirements because they were not receiving SNAP, were disabled, or had dependents.

To translate the population-level effect into a treatment-on-the-treated effect for ABAWDs, one would need to know what fraction of the population consists of ABAWDs subject to the requirements. Administrative data from USDA suggests that approximately 4.5 million individuals were subject to ABAWD requirements as of 2016, representing roughly 7 percent of the total SNAP caseload and approximately 3--4 percent of all adults aged 18--49. If ABAWDs represent approximately 3--4 percent of the 18--49 population, the effect on ABAWDs themselves would be on the order of 20--25 percentage points---substantially larger than the population-level estimate.

However, this scaling exercise should be interpreted with caution. The treatment-on-the-treated calculation assumes that all of the employment effect is concentrated among ABAWDs, which may not be the case if work requirements have spillover effects on other populations (for example, if they affect labor supply decisions of individuals who are not currently receiving SNAP but might consider applying). Additionally, measurement error in identifying ABAWDs would affect the scaling factor.

The timing of the employment response provides additional insight into mechanisms. The event study results show that employment gains emerge immediately in the first year after work requirement reinstatement and persist over subsequent years without substantial growth or decay. This pattern is consistent with work requirements having a one-time effect on employment decisions---individuals respond to the policy change, and those who find employment remain employed---rather than cumulative effects that build over time or temporary effects that fade as individuals cycle back out of employment.

The immediate response also suggests that the effects operate through relatively short-term behavioral channels rather than longer-term human capital accumulation. If work requirements primarily worked by pushing individuals into employment and training programs that built skills leading to later employment, we would expect to see a pattern of growing effects over time as training investments bore fruit. The flat dynamic pattern instead suggests that the main mechanism is the immediate incentive effect of conditioning benefits on work.

\subsection{Comparison to Prior Literature}

The finding of a modest positive employment effect is broadly consistent with the mixed results in the prior literature on work requirements. Studies of welfare-to-work programs under TANF found that some programs produced employment gains while others showed null effects, with the variation appearing related to program design, local labor market conditions, and the characteristics of the target population \citep{blank2002evaluating}. Programs that combined work requirements with intensive job search assistance and supportive services tended to produce the largest and most persistent gains.

More directly relevant is the work by \cite{bauer2019effects}, who document that ABAWD work requirements caused approximately 600,000 individuals to lose SNAP benefits between 2013 and 2017. This represents a reduction of 3--4 percentage points in SNAP participation among the affected population. Comparing this to the employment gain of less than 1 percentage point (at the population level) or even 20--25 percentage points (among ABAWDs specifically, based on the scaling exercise above) suggests that a substantial fraction of the reduction in SNAP participation reflects benefit loss without corresponding employment gains.

The comparison to SNAP participation effects highlights an important distinction between ``employment effects'' and ``participation effects.'' Work requirements unambiguously reduce SNAP participation---individuals either find work and no longer need benefits, or fail to comply and lose benefits regardless of employment status. But the employment effects capture only the former pathway. The discrepancy between large participation declines and modest employment gains suggests that administrative barriers, documentation requirements, and other compliance costs may be important drivers of benefit loss beyond pure employment incentives.

Recent evidence from Medicaid work requirements reinforces this interpretation. Arkansas implemented Medicaid work requirements in 2018, and research found that approximately 18,000 individuals lost coverage within six months, with limited evidence of corresponding employment increases \citep{sommers2020changes}. Survey evidence suggested that many individuals who lost coverage were actually working or exempt but failed to properly document their status through the state's online portal. Administrative burden appears to play a substantial role in explaining why participation falls more than employment rises.

\subsection{Welfare Implications}

From a welfare economics perspective, the key question is whether the employment gains from work requirements justify the costs---both the direct costs of reduced benefits to those who lose SNAP without gaining employment, and the indirect costs of administrative burden, stigma, and reduced program uptake among eligible individuals who are deterred by the requirements.

A back-of-the-envelope welfare calculation is informative, though necessarily speculative. Suppose work requirements cause 100 ABAWDs to lose SNAP benefits. Based on the relative magnitudes of employment and participation effects, perhaps 20--30 of these individuals find employment and voluntarily exit the program, while 70--80 lose benefits without corresponding employment gains. The 20--30 who find employment may experience welfare gains (if they prefer working to receiving benefits, accounting for the value of increased income and any psychic benefits of work) or welfare losses (if they are compelled to take undesirable jobs). The 70--80 who lose benefits without employment experience unambiguous welfare losses equal to the value of lost SNAP benefits (averaging approximately \$150--200 per month per person).

Whether the aggregate welfare effect is positive or negative depends on the magnitude of gains to those who find employment relative to losses for those who simply lose benefits. Given the modest employment response, it seems likely that the aggregate welfare effect is negative---the benefits are too small and the costs too large for work requirements to pass a cost-benefit test as an employment intervention. Of course, this calculation ignores potential non-economic objectives that policymakers may value, such as promoting self-sufficiency as an intrinsic good or signaling societal expectations about work.

\subsection{Policy Implications}

These findings have several implications for policy debates over work requirements. First, work requirements do appear to have some positive effect on employment, consistent with the economic logic that conditioning benefits on work creates incentives for job search and employment. This provides some support for the view that work requirements ``work'' in the sense of affecting behavior. Proponents can legitimately claim that work requirements are not merely punitive---they do appear to influence employment decisions at the margin.

Second, the magnitude of the effect is modest, particularly relative to the documented reductions in SNAP participation. If the goal of work requirements is to increase employment among safety net recipients, other policy tools---such as earned income tax credits, wage subsidies, employment services, or subsidized jobs---may achieve larger effects at lower cost in terms of benefit loss. The EITC, for example, has been shown to substantially increase employment among single mothers while providing income support rather than threatening benefit loss \citep{hoynes2012work}.

Third, the results highlight the importance of distinguishing between reducing caseloads and improving outcomes. Work requirements clearly reduce SNAP participation (as documented by prior research), but the employment effects are much smaller. Policymakers should be explicit about their objectives and evaluate policies against appropriate metrics. If the goal is fiscal savings through reduced caseloads, work requirements are effective; if the goal is improved economic outcomes for low-income individuals, their effectiveness is much more limited.

Fourth, the heterogeneity results suggest that work requirements may be more effective in strong labor markets where jobs are available for those seeking them. This has implications for the design of waiver policies: suspending work requirements during recessions (as current policy does) may be justified not only on grounds of fairness but also on grounds of effectiveness. When unemployment is high and jobs are scarce, work requirements are unlikely to increase employment because individuals cannot find jobs regardless of their incentives to search.

Fifth, the finding that employment effects do not grow over time suggests that work requirements do not have substantial long-term benefits beyond their immediate incentive effects. Individuals who find jobs in response to work requirements do not appear to be building human capital or gaining experience that leads to further employment gains down the road. This weakens the case for work requirements as an ``investment'' in recipients' long-term self-sufficiency and suggests that their benefits should be evaluated primarily in terms of immediate employment gains.

\subsection{Limitations}

This study has several limitations that should inform interpretation of the results. First, the use of aggregate state-level data prevents examination of individual-level mechanisms and heterogeneity. I cannot determine whether employment gains are concentrated among individuals who would have remained on SNAP absent work requirements, or whether the population-level effects reflect selection (individuals with better employment prospects are less likely to lose benefits). Individual-level data linking SNAP administrative records to employment outcomes would be ideal for addressing these questions but were not available for this study.

Second, the relatively small sample size (24 states, 8 years) limits statistical power for detecting small effects or estimating heterogeneous effects across subgroups. The results should be interpreted with appropriate caution given the wide confidence intervals. The confidence interval for the main estimate (0.41 to 1.15 percentage points) is consistent with effects ranging from quite small to moderately meaningful.

Third, the analysis cannot fully rule out confounding from other state-level policies that may have changed around the same time as work requirement reinstatement. While the event study evidence suggests no differential pre-trends, unobserved policy changes correlated with waiver status could still bias the results. Most notably, states' decisions about Medicaid expansion under the Affordable Care Act may have been correlated with their labor market conditions and hence waiver status.

Fourth, the external validity of the findings is uncertain. The never-treated control states---Minnesota, Montana, North Dakota, South Dakota, Utah, and Vermont---differ from the treated states in many dimensions, including geographic region, industrial composition, demographics, and political orientation. These states may not be representative counterfactuals for what would have happened in treated states absent work requirements. Extrapolating the effects to different contexts or populations should be done cautiously.

Fifth, the study cannot distinguish between the effects of work requirements per se and the effects of reduced SNAP generosity more broadly. When work requirements are reinstated, some individuals may respond by finding employment, but others may respond by reducing SNAP applications or not recertifying their eligibility---effectively exiting the program without gaining employment. The aggregate employment effect captures both behavioral responses, and I cannot separate them with available data.

Sixth, the analysis focuses on short-run employment effects and cannot address questions about longer-term outcomes including earnings growth, job quality, economic mobility, or non-employment outcomes such as health and family stability. Work requirements could have effects on these outcomes---positive or negative---that are not captured by the short-run employment analysis.


\section{Conclusion}

This paper provides new evidence on the employment effects of SNAP work requirements for Able-Bodied Adults Without Dependents. Using the staggered expiration of state-level waivers following the 2008 recession as quasi-experimental variation, I find that work requirement reinstatement increased employment by approximately 0.77 percentage points among the ABAWD-eligible population. This effect is statistically significant but economically modest.

The findings contribute to ongoing policy debates about the role of work requirements in safety net programs. On one hand, the positive employment effect suggests that work requirements do influence behavior in the intended direction, providing some justification for their inclusion in program design. On the other hand, the modest magnitude---particularly when compared to documented reductions in SNAP participation---raises questions about whether work requirements are the most effective or efficient tool for promoting employment among low-income populations.

The analysis reveals several important patterns. First, the employment response emerges immediately upon work requirement reinstatement and remains relatively stable over subsequent years, suggesting that effects operate through immediate incentive channels rather than longer-term human capital accumulation. Second, the heterogeneity analysis indicates that work requirements may be more effective in tighter labor markets where jobs are available for those seeking them, supporting the case for labor-market-based waiver policies. Third, the discrepancy between employment gains and documented participation losses suggests that administrative burden and compliance costs play a substantial role in how work requirements affect outcomes.

From a policy perspective, these findings suggest that work requirements should be evaluated carefully against alternative tools for promoting employment among low-income populations. Programs that combine positive incentives (such as the Earned Income Tax Credit) with employment services may achieve larger employment gains without the downside of benefit loss for those unable to comply. If policymakers nonetheless choose to implement work requirements, attention to program design---including the availability of employment services, the stringency of documentation requirements, and the appropriateness of exemptions---may substantially affect the balance of employment gains versus hardship from benefit loss.

The study also highlights the value of quasi-experimental policy variation for understanding the effects of social policy. The staggered expiration of ABAWD waivers created a natural experiment that allows credible causal inference even in the absence of randomized controlled trials. Future policy changes---whether expansions or contractions of work requirements---should be designed with evaluation in mind, incorporating variation that enables rigorous assessment of program effects.

Future research should examine the longer-term effects of work requirements on employment, earnings, and economic mobility. Understanding whether initial employment gains persist and translate into lasting improvements in economic outcomes is essential for evaluating the full welfare implications of work requirement policies. Research linking individual-level SNAP administrative records to longitudinal employment data could provide more precise estimates of treatment effects and illuminate the mechanisms through which work requirements affect behavior. Additionally, research examining the effects of work requirements on non-employment outcomes---including health, housing stability, food security, and family well-being---would provide a more complete picture of the welfare implications of conditioning benefits on work.


\section*{Acknowledgements}

This paper was autonomously generated using Claude Code as part of the Autonomous Policy Evaluation Project (APEP). The analysis uses publicly available data from the USDA Food and Nutrition Service and Bureau of Labor Statistics.

\noindent\textbf{Project Repository:} \url{https://github.com/SocialCatalystLab/auto-policy-evals}

\noindent\textbf{Contributors:} APEP Autonomous Research

\label{apep_main_text_end}

\newpage
\begin{thebibliography}{99}

\bibitem[Bauer et~al.(2019)]{bauer2019effects}
Bauer, L., Schanzenbach, D.~W., and Shambaugh, J.~(2019).
\newblock The effects of SNAP work requirements in reducing participation and benefits from 2013 to 2017.
\newblock \emph{American Journal of Public Health}, 109(10), e1--e6.

\bibitem[Bitler and Hoynes(2016)]{bitler2016safety}
Bitler, M. and Hoynes, H.~W. (2016).
\newblock The more things change, the more they stay the same? The safety net and poverty in the Great Recession.
\newblock \emph{Journal of Labor Economics}, 34(S1), S403--S444.

\bibitem[Blank(2002)]{blank2002evaluating}
Blank, R.~M. (2002).
\newblock Evaluating welfare reform in the United States.
\newblock \emph{Journal of Economic Literature}, 40(4), 1105--1166.

\bibitem[Borusyak et~al.(2021)]{borusyak2021revisiting}
Borusyak, K., Jaravel, X., and Spiess, J. (2021).
\newblock Revisiting event study designs: Robust and efficient estimation.
\newblock Working Paper.

\bibitem[Callaway and Sant'Anna(2021)]{callaway2021difference}
Callaway, B. and Sant'Anna, P.~H.~C. (2021).
\newblock Difference-in-differences with multiple time periods.
\newblock \emph{Journal of Econometrics}, 225(2), 200--230.

\bibitem[Cameron et~al.(2008)]{cameron2008bootstrap}
Cameron, A.~C., Gelbach, J.~B., and Miller, D.~L. (2008).
\newblock Bootstrap-based improvements for inference with clustered errors.
\newblock \emph{Review of Economics and Statistics}, 90(3), 414--427.

\bibitem[de~Chaisemartin and D'Haultfoeuille(2020)]{dechaisemartin2020two}
de~Chaisemartin, C. and D'Haultfoeuille, X. (2020).
\newblock Two-way fixed effects estimators with heterogeneous treatment effects.
\newblock \emph{American Economic Review}, 110(9), 2964--2996.

\bibitem[FHE(2016)]{fhe2016kansas}
Foundation for Government Accountability (2016).
\newblock Kansas SNAP Employment and Training Evaluation.
\newblock Policy Brief.

\bibitem[Ganong and Liebman(2018)]{ganong2018consumer}
Ganong, P. and Liebman, J.~B. (2018).
\newblock The decline, rebound, and further rise in SNAP enrollment.
\newblock \emph{American Journal of Agricultural Economics}, 100(1), 258--276.

\bibitem[Goodman-Bacon(2021)]{goodman2021difference}
Goodman-Bacon, A. (2021).
\newblock Difference-in-differences with variation in treatment timing.
\newblock \emph{Journal of Econometrics}, 225(2), 254--277.

\bibitem[Gray et~al.(2019)]{gray2019effects}
Gray, K.~F., Lauffer, S., and Cody, S. (2019).
\newblock Characteristics of Supplemental Nutrition Assistance Program households.
\newblock USDA Food and Nutrition Service Report.

\bibitem[Gregory and Deb(2013)]{gregory2013supplemental}
Gregory, C.~A. and Deb, P. (2013).
\newblock Does SNAP improve your health?
\newblock \emph{Food Policy}, 46, 81--93.

\bibitem[Grogger(2003)]{grogger2003welfare}
Grogger, J. (2003).
\newblock The effects of time limits, the EITC, and other policy changes on welfare use, work, and income among female-headed families.
\newblock \emph{Review of Economics and Statistics}, 85(2), 394--408.

\bibitem[Hastings and Shapiro(2019)]{hastings2019food}
Hastings, J. and Shapiro, J.~M. (2019).
\newblock How are SNAP benefits spent? Evidence from a retail panel.
\newblock \emph{American Economic Review}, 108(12), 3493--3540.

\bibitem[Hoynes and Schanzenbach(2012)]{hoynes2012work}
Hoynes, H.~W. and Schanzenbach, D.~W. (2012).
\newblock Work incentives and the Food Stamp Program.
\newblock \emph{Journal of Public Economics}, 96(1-2), 151--162.

\bibitem[Moffitt(2002)]{moffitt2002welfare}
Moffitt, R.~A. (2002).
\newblock From welfare to work: What the evidence shows.
\newblock \emph{Brookings Papers on Economic Activity}, 2002(1), 1--75.

\bibitem[Rambachan and Roth(2023)]{rambachan2023more}
Rambachan, A. and Roth, J. (2023).
\newblock A more credible approach to parallel trends.
\newblock \emph{Review of Economic Studies}, 90(5), 2555--2591.

\bibitem[Rosenbaum(2019)]{rosenbaum2019snap}
Rosenbaum, D. (2019).
\newblock SNAP's ``Able-Bodied Adult Without Dependents'' Time Limit.
\newblock Center on Budget and Policy Priorities.

\bibitem[Roth(2022)]{roth2022pretest}
Roth, J. (2022).
\newblock Pretest with caution: Event-study estimates after testing for parallel trends.
\newblock \emph{American Economic Review: Insights}, 4(3), 305--322.

\bibitem[Schmieder and von Wachter(2016)]{schmieder2016effects}
Schmieder, J.~F. and von Wachter, T. (2016).
\newblock The effects of unemployment insurance benefits: New evidence and interpretation.
\newblock \emph{Annual Review of Economics}, 8, 547--581.

\bibitem[Sommers et~al.(2020)]{sommers2020changes}
Sommers, B.~D., Chen, L., Blendon, R.~J., Orav, E.~J., and Epstein, A.~M. (2020).
\newblock Medicaid work requirements in Arkansas: Two-year impacts on coverage, employment, and affordability of care.
\newblock \emph{Health Affairs}, 39(9), 1522--1530.

\bibitem[Sun and Abraham(2021)]{sun2021estimating}
Sun, L. and Abraham, S. (2021).
\newblock Estimating dynamic treatment effects in event studies with heterogeneous treatment effects.
\newblock \emph{Journal of Econometrics}, 225(2), 175--199.

\bibitem[USDA FNS(2023)]{usda2023waivers}
USDA Food and Nutrition Service (2023).
\newblock ABAWD Waivers FY 2010--2024.
\newblock \url{https://www.fns.usda.gov/snap/abawd/waivers}.

\bibitem[Ziliak(2016)]{ziliak2016temporary}
Ziliak, J.~P. (2016).
\newblock Temporary Assistance for Needy Families.
\newblock In R.~A. Moffitt (Ed.), \emph{Economics of Means-Tested Transfer Programs in the United States}, Volume 1, pp. 303--393. University of Chicago Press.

\end{thebibliography}

\newpage
\appendix

\section{Data Appendix}

\subsection{ABAWD Waiver Status Data}

ABAWD waiver status data were obtained from the USDA Food and Nutrition Service public records. FNS maintains archived waiver materials at \url{https://www.fns.usda.gov/snap/abawd/waivers}. I accessed records for fiscal years 2010 through 2019, coding each state-year according to whether a statewide waiver was in effect.

States are classified as having a waiver if FNS records indicate an approved statewide waiver for the majority of the fiscal year. States with partial waivers covering specific counties are classified based on the proportion of ABAWD population covered; states with partial waivers covering less than half the ABAWD population are coded as not having a waiver.

The following 18 states are classified as ``early treated'' (waiver expired FY2015):
Alabama, Arizona, Arkansas, California, Florida, Georgia, Illinois, Indiana, Iowa, Kansas, Mississippi, Missouri, Nebraska, New York, North Carolina, Oklahoma, South Carolina, Texas.

The following 6 states are classified as ``never treated'' (maintained waivers FY2012--FY2019):
Minnesota, Montana, North Dakota, South Dakota, Utah, Vermont.

Wisconsin is classified as ``late treated'' (waiver expired FY2016) and is used for placebo testing.

\subsection{Employment Data}

State-level employment rates for adults aged 18--49 were constructed from Bureau of Labor Statistics Local Area Unemployment Statistics (LAUS) and Current Population Survey (CPS) microdata. The employment-to-population ratio is defined as employed persons divided by civilian noninstitutional population in the age group.

For the BLS LAUS data, I use the annual average estimates of employment level and population. For CPS, I use the Annual Social and Economic Supplement (ASEC) weights to compute state-level employment rates.

\subsection{Sample Restrictions}

The analysis sample excludes:
\begin{itemize}
\item States with fluctuating waiver status (Nevada, which alternated between waiver and non-waiver status)
\item States treated after 2015 (Wisconsin, which is used for placebo testing only)
\item Territories (Puerto Rico, Virgin Islands, Guam)
\item District of Columbia (due to its unique characteristics as a city-state)
\end{itemize}

\section{Additional Results}

\subsection{Full Event Study Coefficients}

Table \ref{tab:es_full} presents the full event study results with additional statistics.

\begin{table}[H]
\centering
\caption{Full Event Study Results}
\begin{threeparttable}
\begin{tabular}{lcccccc}
\toprule
Event Time & Year & Coef. & SE & 95\% CI Lower & 95\% CI Upper & p-value \\
\midrule
$t = -3$ & 2012 & 0.0127 & 0.0108 & -0.0085 & 0.0339 & 0.241 \\
$t = -2$ & 2013 & 0.0065 & 0.0089 & -0.0110 & 0.0240 & 0.466 \\
$t = -1$ & 2014 & \multicolumn{5}{c}{(reference period)} \\
$t = 0$ & 2015 & 0.0109 & 0.0062 & -0.0012 & 0.0230 & 0.078 \\
$t = +1$ & 2016 & 0.0135 & 0.0068 & 0.0002 & 0.0269 & 0.047 \\
$t = +2$ & 2017 & 0.0146 & 0.0071 & 0.0007 & 0.0285 & 0.040 \\
$t = +3$ & 2018 & 0.0173 & 0.0078 & 0.0020 & 0.0325 & 0.027 \\
$t = +4$ & 2019 & 0.0109 & 0.0092 & -0.0072 & 0.0289 & 0.238 \\
\bottomrule
\end{tabular}
\begin{tablenotes}[flushleft]
\small
\item Notes: Standard errors clustered at the state level. Confidence intervals computed using wild cluster bootstrap with 1,000 replications.
\end{tablenotes}
\end{threeparttable}
\label{tab:es_full}
\end{table}

\subsection{Sensitivity to State Exclusions}

Table \ref{tab:sensitivity} shows the DiD estimate when each state is excluded from the analysis, demonstrating that results are not driven by any single state.

\begin{table}[H]
\centering
\caption{Sensitivity to State Exclusions}
\begin{threeparttable}
\begin{tabular}{lcc}
\toprule
Excluded State & DiD Estimate & SE \\
\midrule
None (baseline) & 0.0077 & 0.0019 \\
\midrule
Alabama & 0.0078 & 0.0019 \\
Florida & 0.0074 & 0.0020 \\
Texas & 0.0081 & 0.0019 \\
Minnesota & 0.0082 & 0.0022 \\
North Dakota & 0.0070 & 0.0020 \\
Utah & 0.0076 & 0.0019 \\
\bottomrule
\end{tabular}
\begin{tablenotes}[flushleft]
\small
\item Notes: Each row shows the DiD estimate when the listed state is excluded from the sample. Standard errors clustered at the state level.
\end{tablenotes}
\end{threeparttable}
\label{tab:sensitivity}
\end{table}

\section{Identification Appendix}

\subsection{Pre-Trend Analysis}

The mean absolute pre-treatment coefficient (averaging $|t=-3|$ and $|t=-2|$ coefficients) is 0.0096, which is smaller than the typical threshold of 0.015 used to assess pre-trend violations. A formal F-test of the joint hypothesis that all pre-treatment coefficients equal zero yields a p-value of 0.38, failing to reject the null of no differential pre-trends.

\subsection{Placebo Test Details}

Wisconsin provides a natural placebo test since it maintained its waiver in 2015 before treatment in 2016. Comparing Wisconsin to never-treated states in 2014--2015 (before Wisconsin's treatment), I estimate a ``placebo DiD'' of 0.0088. This is somewhat larger than zero but within the range of sampling variation (standard error: 0.0095, p-value: 0.35). The result provides modest support for the research design.

\end{document}
