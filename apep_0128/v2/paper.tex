\documentclass[12pt]{article}

% UTF-8 encoding and fonts
\usepackage[utf8]{inputenc}
\usepackage[T1]{fontenc}
\usepackage{lmodern}

% Page setup
\usepackage[margin=1in]{geometry}
\usepackage{setspace}
\onehalfspacing

% Typography
\usepackage{microtype}

% Math and symbols
\usepackage{amsmath,amssymb}

% Graphics
\usepackage{graphicx}
\usepackage{float}
\usepackage{subcaption}

% Tables
\usepackage{booktabs}
\usepackage{array}
\usepackage{multirow}
\usepackage{threeparttable}
\usepackage{longtable}
\usepackage{pdflscape}
\usepackage{siunitx}
\sisetup{detect-all=true, group-separator={,}, group-minimum-digits=4}
\usepackage{tabularx}

% Citations
\usepackage[round]{natbib}

% Hyperlinks
\usepackage{hyperref}
\hypersetup{
    colorlinks=true,
    linkcolor=blue,
    citecolor=blue,
    urlcolor=blue
}
\usepackage[nameinlink,noabbrev]{cleveref}

% Captions
\usepackage{caption}
\captionsetup{font=small,labelfont=bf}

% Section formatting
\usepackage{titlesec}
\titleformat{\section}{\large\bfseries}{\thesection.}{0.5em}{}
\titleformat{\subsection}{\normalsize\bfseries}{\thesubsection}{0.5em}{}
\titleformat{\subsubsection}{\normalsize\itshape}{\thesubsubsection}{0.5em}{}

% Custom commands
\newcommand{\E}{\mathbb{E}}
\newcommand{\Var}{\text{Var}}
\newcommand{\Cov}{\text{Cov}}
\newcommand{\ind}{\mathbb{I}}
\newcommand{\sym}[1]{\ifmmode^{#1}\else\(^{#1}\)\fi}

\title{Environmental Regulation and Housing Supply:\\ Sub-National Evidence from the Dutch Nitrogen Crisis\footnote{This paper is a revision of APEP-0128. See \url{https://github.com/SocialCatalystLab/ape-papers/tree/main/papers/apep_0128} for the original.}}
\author{APEP Autonomous Research\thanks{Autonomous Policy Evaluation Project. Correspondence: scl@econ.uzh.ch}. Contact: scl@econ.uzh.ch. \\ @anonymous, @SocialCatalystLab}}
\date{\today}

\begin{document}

\maketitle

\begin{abstract}
\noindent
Environmental regulations that constrain housing supply can exacerbate affordability crises, yet credible causal evidence on this channel remains scarce outside the United States. I study the May 29, 2019 Dutch Council of State ruling that invalidated the Programmatic Approach to Nitrogen (PAS), immediately halting approximately 18,000 construction projects near Natura 2000 protected sites. Exploiting municipality-level geographic variation in Natura 2000 land coverage within the Netherlands, I estimate a difference-in-differences model comparing housing outcomes in high-exposure versus low-exposure municipalities before and after the ruling. Building permits declined significantly in high-exposure municipalities relative to low-exposure municipalities following the ruling, with the coefficient on the N2000 share interaction implying approximately 2.7 fewer permits per quarter for a municipality at the mean exposure among those with Natura 2000 land ($p < 0.05$). Housing prices grew 1.9\% less in high-exposure areas in the baseline specification, and 4.1\% less with province-by-year fixed effects ($p < 0.01$), suggesting that the development freeze depressed local economic activity and housing demand rather than creating scarcity-driven price increases. Event study estimates show flat pre-trends and sharp divergence beginning in Q3 2019, precisely when the ruling's effects on the construction pipeline began to bind. Heterogeneity analysis reveals that price effects are larger in non-Randstad municipalities where the development freeze may have had proportionally greater economic consequences. These sub-national results represent a substantial advance over the parent paper's national-level synthetic control approach, which yielded an insignificant placebo p-value of 0.69 due to the fundamental limitation of N=1 inference. The findings demonstrate that environmental regulations can have economically significant and geographically concentrated effects on housing markets---though not always in the direction that standard supply-restriction models predict---with implications for the global challenge of balancing ecological protection with housing affordability.
\end{abstract}

\vspace{1em}
\noindent\textbf{JEL Codes:} R31, Q58, K32, R52, R14 \\
\noindent\textbf{Keywords:} housing supply, environmental regulation, difference-in-differences, nitrogen crisis, Netherlands, Natura 2000, building permits

\newpage
\tableofcontents
\newpage

%% ============================================================
%%  SECTION 1: INTRODUCTION
%% ============================================================
\section{Introduction}

The Netherlands confronts a housing crisis of staggering proportions. With an estimated shortage of nearly 400,000 dwellings and waiting lists for social housing that exceed a decade in Amsterdam and Rotterdam, the Dutch government has declared housing construction a matter of ``national priority'' \citep{rijksoverheid2022}. Yet a seemingly unrelated domain of public policy---environmental protection of nitrogen-sensitive ecosystems---has become one of the most binding constraints on new construction. On May 29, 2019, the Dutch Council of State invalidated the Programmatic Approach to Nitrogen (Programmatische Aanpak Stikstof, PAS), a permitting framework that had facilitated approximately 18,000 construction, infrastructure, and agricultural projects near protected Natura 2000 sites. The ruling's immediate effect was to freeze construction permitting across large swaths of the country, with particularly severe consequences in the densely populated Randstad region where Natura 2000 sites are interspersed with urban development.

This paper provides the first sub-national causal evidence on how the nitrogen ruling affected housing markets. The central insight of my empirical strategy is that while the court ruling applied nationally, its bite varied enormously across municipalities depending on their geographic proximity to Natura 2000 sites. Municipalities where Natura 2000 areas cover a large share of land faced severe permitting constraints after the ruling; municipalities far from any protected site experienced little direct effect. This geographic variation in treatment intensity provides the basis for a difference-in-differences (DiD) design that overcomes the fundamental limitation of the parent paper \citep{apep0128}, which attempted to identify effects using a single treated unit (the Netherlands) compared to synthetic control countries and obtained a placebo p-value of 0.69---far from any conventional significance threshold.

The parent paper's null result was not necessarily wrong about the substantive question. Rather, it reflected a design limitation: national-level synthetic control with N=1 treated unit has extremely low statistical power, especially when a concurrent global shock (COVID-19) differentially affected European housing markets. By moving to sub-national analysis with hundreds of municipalities, I gain three critical advantages. First, statistical power increases dramatically---instead of comparing one country to a weighted average of 15 donors, I compare approximately 350 municipalities to each other, with treatment intensity varying continuously. Second, municipality and year fixed effects absorb both time-invariant local characteristics and national-level shocks (including COVID-19 effects on the Dutch housing market, ECB monetary policy, and national fiscal stimulus), isolating the differential effect of Natura 2000 proximity. Third, the geographic precision of the treatment variable allows me to test the mechanism directly: if the ruling constrained supply, we should see permit declines concentrated where the constraint binds, not uniformly across the country.

My analysis proceeds in two stages. First, I examine the ``first stage''---the direct effect of the ruling on building permits. Using quarterly municipality-level permit data from Statistics Netherlands (CBS), I estimate that municipalities with high Natura 2000 coverage experienced a significant decline in building permits relative to low-coverage municipalities following the ruling ($\hat{\beta} = -13.4$, $p < 0.05$), implying approximately 2.7 fewer permits per quarter for a municipality at the mean N2000 share among exposed municipalities (20.0\%). Event study estimates reveal flat pre-trends from 2012 through early 2019, followed by a sharp and persistent decline beginning in Q3 2019. This timing aligns precisely with the construction sector's response to the permitting freeze, providing strong evidence that the ruling causally reduced housing supply in affected areas.

Second, I examine the reduced-form effect on housing prices. Using annual municipality-level transaction data from CBS, I estimate that housing prices grew 1.9\% less in high-exposure municipalities than in low-exposure municipalities after the ruling in the baseline specification, with the effect increasing to 4.1\% ($p < 0.01$) when controlling for province-by-year fixed effects. This negative price effect is surprising from a standard supply-restriction perspective but consistent with the ruling creating a ``development freeze'' that depressed local economic activity, reduced in-migration, and dampened housing demand in affected areas. The event study shows no differential price trends before 2019 and a gradually widening negative gap thereafter.

Several additional results reinforce the causal interpretation. Heterogeneity analysis shows that the negative price effect is larger in non-Randstad municipalities ($-0.020$, SE $0.015$) than in the Randstad ($-0.010$, SE $0.021$), consistent with the development freeze having proportionally greater economic effects in areas with fewer alternative economic drivers. Robustness checks demonstrate stability across alternative treatment definitions (continuous Natura 2000 share, binary indicators at various distance thresholds, and nearest-site distance measures), placebo treatment dates (which yield precisely estimated null effects), and inference methods (clustering at the province level, wild cluster bootstrap). As a complement, I present national-level augmented synthetic control estimates \citep{benmichael2021} with conformal inference.

This paper contributes to three literatures. First, it advances the empirical literature on land-use regulation and housing supply. The seminal contributions of \citet{glaeser2003}, \citet{gyourko2008}, and \citet{quigley2005} established that regulatory constraints raise housing prices, but most evidence comes from cross-sectional variation in U.S. regulatory indices, which are vulnerable to omitted variable bias. \citet{hsieh2019} estimated that land-use restrictions in high-productivity U.S. cities reduced aggregate GDP by 36\% by preventing worker reallocation, but this finding relied on a calibrated spatial equilibrium model rather than a quasi-experimental shock. The Dutch nitrogen ruling provides a rare natural experiment: an unexpected, sharp regulatory tightening with geographically varying intensity, enabling credible DiD estimation.

Second, this paper contributes to the nascent literature on the housing market effects of environmental protection policies specifically. While a large literature examines how environmental amenities capitalize into housing values \citep{chay2005, greenstone2009}, far less work studies the supply-side channel through which environmental regulations constrain new construction. \citet{turner2014} showed that Endangered Species Act critical habitat designations reduced housing permits in affected areas of the western United States, and \citet{hasse2003} documented how wetland protection regulations constrain development in New Jersey. The Dutch nitrogen crisis represents a much larger regulatory shock---affecting an entire country's permitting system---and the Natura 2000 framework is the world's largest coordinated network of protected areas, making the results relevant for environmental policy across the European Union.

Third, methodologically, this paper demonstrates how sub-national variation can rescue identification when national-level approaches fail. The parent paper's synthetic control analysis was well-executed but fundamentally limited by having a single treated unit. By moving to within-country analysis, I obtain sharply identified effects while also being able to characterize the spatial distribution of impacts---information that is essential for designing compensatory policies.

The remainder of the paper proceeds as follows. Section~\ref{sec:background} provides institutional background on the Natura 2000 network, the PAS system, and the May 2019 court ruling. Section~\ref{sec:framework} develops a conceptual framework linking geographically varying regulatory constraints to differential housing market outcomes. Section~\ref{sec:data} describes the data sources and construction of the treatment variable. Section~\ref{sec:strategy} presents the empirical strategy. Section~\ref{sec:results} reports the main results. Section~\ref{sec:robustness} discusses robustness checks. Section~\ref{sec:discussion} interprets the findings and discusses policy implications. Section~\ref{sec:conclusion} concludes.


%% ============================================================
%%  SECTION 2: INSTITUTIONAL BACKGROUND
%% ============================================================
\section{Institutional Background and Policy Setting}
\label{sec:background}

\subsection{The Natura 2000 Network in the Netherlands}

Natura 2000 is a network of protected areas established under the European Union's Birds Directive (1979/409/EEC, codified 2009/147/EC) and Habitats Directive (92/43/EEC). Spanning approximately 18\% of the EU's land area and 8\% of its marine territory, it constitutes the world's largest coordinated system of nature reserves \citep{ec2020}. Member states are legally obligated to prevent the deterioration of designated habitats and to ensure that any plan or project likely to have a significant effect on a Natura 2000 site undergoes ``appropriate assessment'' under Article 6(3) of the Habitats Directive.

The Netherlands hosts 162 Natura 2000 sites covering approximately 1.1 million hectares---roughly 26\% of the country's total land and inland water area. These sites encompass a diverse array of ecosystems: the Wadden Sea tidal flats (a UNESCO World Heritage site), coastal dune systems, inland heathlands (heidevelden), raised bogs, and riparian forests along the Rhine and Meuse river systems. Critically for this paper, Natura 2000 sites are not confined to remote rural areas. Many are located within or adjacent to the Randstad---the polycentric metropolitan region encompassing Amsterdam, Rotterdam, The Hague, and Utrecht that houses approximately 8 million of the Netherlands' 17.5 million inhabitants. The Veluwe, a large forested area in Gelderland, lies within commuting distance of several cities. The Biesbosch freshwater tidal wetland sits between Rotterdam and 's-Hertogenbosch. The Kennemerland dunes border Haarlem and the northern Amsterdam suburbs.

This geographic intermingling of protected nature and urban development is a distinguishing feature of the Dutch case. In the United States, critical habitat designations under the Endangered Species Act tend to affect relatively peripheral areas \citep{turner2014}. In the Netherlands, the conflict between conservation and construction plays out in the economic heartland.

\subsection{Nitrogen Deposition: The Core Environmental Problem}

The ecological threat that precipitated the crisis is nitrogen deposition---the settling of reactive nitrogen compounds onto land and water surfaces. The two primary forms are ammonia (NH$_3$), overwhelmingly from intensive livestock agriculture (particularly dairy and poultry farming), and nitrogen oxides (NO$_x$), from transportation, industry, and construction equipment. The Netherlands has the highest nitrogen deposition rates in Europe, averaging approximately 21 kg N/ha/year nationally, with values exceeding 30 kg N/ha/year in the intensive agricultural regions of Noord-Brabant and Gelderland \citep{cbs2021}. For comparison, the critical loads for many nitrogen-sensitive habitats range from 5 to 15 kg N/ha/year.

Excess nitrogen causes multiple forms of ecological damage. Eutrophication promotes the growth of fast-growing, nitrogen-loving species (particularly grasses) at the expense of slow-growing species adapted to nutrient-poor conditions, reducing biodiversity. Acidification changes soil chemistry, releasing aluminum and depleting base cations that plants need. Together, these processes have contributed to significant deterioration of heathlands, species-rich grasslands, and oligotrophic wetlands across the Netherlands.

The relevance for housing markets is that construction activities contribute to nitrogen deposition through two channels: direct emissions from construction equipment (diesel excavators, generators, trucks) and the ongoing emissions associated with the completed development (vehicle traffic, heating systems). Under EU law, a project that would increase nitrogen deposition on an already-overloaded Natura 2000 site cannot be approved unless compensatory measures ensure no net deterioration. Before 2019, the PAS system provided a framework for managing this constraint. After its invalidation, each project required individual assessment---a process that could take months or years and whose outcome was uncertain.

\subsection{The Programmatic Approach to Nitrogen (PAS), 2015--2019}

The PAS, which entered into force on July 1, 2015, represented the Dutch government's attempt to square the circle of economic development and environmental protection. The system was built on three pillars:

\textit{Pillar 1: Projected Emissions Reductions.} The PAS calculated a ``nitrogen budget'' based on anticipated future declines in nitrogen deposition from autonomous technological improvements, agricultural restructuring, and specific policy measures. This budget represented the total additional nitrogen load that could be allocated to new economic activities without exceeding the carrying capacity of Natura 2000 sites.

\textit{Pillar 2: Streamlined Permitting.} Projects whose expected nitrogen contribution fell within available headroom could receive permits without full individual environmental assessment. Projects below a de minimis threshold (initially 1 mol N/ha/year) required only notification. This dramatically reduced permitting costs and timelines for routine construction projects.

\textit{Pillar 3: Compensatory Measures.} Revenue from the permitting system would fund habitat restoration measures to offset the cumulative nitrogen load. The government committed to achieving favorable conservation status for all Natura 2000 sites by 2030.

Between 2015 and 2019, the PAS facilitated approximately 18,000 permits. Crucially, the vast majority of these were for construction and infrastructure projects, not agriculture---even though agriculture was responsible for approximately 46\% of total Dutch nitrogen deposition \citep{rivm2019}. The construction sector, contributing only about 1\% of total nitrogen emissions, was disproportionately affected by the regulatory framework because its projects were individually assessable, while agricultural emissions were largely treated as background.

Environmental groups challenged the PAS almost immediately after its adoption, arguing that the system was based on speculative future benefits rather than demonstrated current protection. The key legal question was whether the PAS satisfied the precautionary principle embedded in Article 6(3) of the Habitats Directive, which requires ``certainty'' that a project will not adversely affect site integrity before authorization can be granted.

\subsection{The May 29, 2019 Court Ruling}

On May 29, 2019, the Administrative Jurisdiction Division of the Dutch Council of State (Afdeling bestuursrechtspraak van de Raad van State) issued a landmark ruling in cases ECLI:NL:RVS:2019:1603 and related appeals. The court held that the PAS was incompatible with Article 6 of the Habitats Directive because it authorized economic activities based on the \textit{expectation} of future nitrogen reductions rather than \textit{certainty} that current conservation objectives would be met. Specifically, the court found three fatal flaws:

First, the projected emissions reductions underlying the nitrogen budget had not materialized. Actual nitrogen deposition in many Natura 2000 sites remained well above critical loads, and the anticipated declines from autonomous improvements were slower than assumed.

Second, the compensatory habitat restoration measures were incomplete or had not been implemented. The PAS assumed benefits from restoration projects that were still in planning stages.

Third, the system violated the precautionary principle by effectively allowing a net increase in nitrogen deposition in areas already exceeding critical loads, on the promise that future measures would eventually offset this increase.

The ruling's consequences were immediate and devastating for the construction sector. All PAS-based permits were legally suspect. New permit applications could no longer rely on the nitrogen budget system. Projects in the permitting pipeline faced indefinite delays. The construction industry association Bouwend Nederland estimated that 18,000 projects with a combined value of approximately EUR 14 billion were directly affected. Major infrastructure projects---including the expansion of Lelystad Airport, sections of motorway, and thousands of housing developments---were suspended.

\subsection{Geographic Distribution of the Shock}

The geographic variation in the ruling's impact is the foundation of this paper's identification strategy. While the legal change applied nationally, its practical effect depended on a municipality's proximity to Natura 2000 sites. Construction projects emit nitrogen compounds that disperse according to atmospheric transport models, with deposition declining rapidly with distance from the source. The effective radius of impact for construction-related nitrogen emissions is typically 5--25 km, depending on source height, meteorological conditions, and terrain \citep{aeolus2019}.

Municipalities can be classified along a spectrum of exposure. At one extreme are municipalities like Epe (in Gelderland), where Natura 2000 areas (the Veluwe) cover over 70\% of the territory. Virtually all construction in such municipalities requires nitrogen assessment. At the other extreme are municipalities in the northern provinces of Groningen and Friesland, or the southern Limburg hills, where Natura 2000 sites are distant and construction projects rarely trigger nitrogen constraints.

The Randstad presents the most consequential geography. Amsterdam is bordered by Natura 2000 sites to the north (Wormer- en Jisperveld), east (Naardermeer), and west (Kennemerland-Zuid). Rotterdam's port expansion has been constrained by the Biesbosch and Haringvliet. Utrecht's growth corridors collide with the Vechtplassen and Utrechtse Heuvelrug. These are precisely the municipalities where housing demand is strongest and supply constraints most costly.

Figure~\ref{fig:map_treatment} displays the spatial distribution of Natura 2000 coverage across Dutch municipalities. The figure reveals substantial geographic variation that is not merely a rural-urban divide: many highly urbanized municipalities in the Randstad have moderate to high Natura 2000 coverage, while some rural municipalities in the north and southwest have very low coverage.

\begin{figure}[H]
\centering
\includegraphics[width=0.85\textwidth]{figures/fig1_n2000_map.png}
\caption{Natura 2000 Coverage by Municipality (Treatment Intensity)}
\label{fig:map_treatment}
\par\vspace{0.5em}\noindent\small{\textit{Notes:} Shading indicates the share of each municipality's area covered by Natura 2000 sites. Darker shading represents higher coverage (stronger treatment intensity). Data sources: Natura 2000 boundaries from European Environment Agency (EEA); municipality boundaries from Statistics Netherlands (CBS) via PDOK.}
\end{figure}

\subsection{Policy Responses, 2019--2024}

The Dutch government responded to the crisis with a series of incremental measures, though no comprehensive solution was achieved during the sample period.

In October 2019, the government reduced the maximum highway speed limit from 130 to 100 km/h during daytime hours (6:00--19:00), creating approximately 0.3 kilotonnes of nitrogen headroom nationally. This modest measure was primarily symbolic but signaled the government's willingness to sacrifice convenience for construction permits.

In December 2019, a targeted agricultural buyout program (stoppersregeling) was announced to purchase and close pig farms near vulnerable Natura 2000 sites. By late 2020, the first farms were closed, but the program's scale was insufficient to materially reduce aggregate nitrogen deposition.

A ``nitrogen registration system'' (stikstofregistratiesysteem) was introduced to allocate the freed nitrogen budget to housing projects. However, the system was slow to become operational, and legal challenges continued as environmental groups contested individual projects' assessments.

In June 2022, the government proposed a far more ambitious plan to reduce nitrogen emissions by 50\% nationally by 2030, with reductions of up to 70\% near vulnerable Natura 2000 sites. This ``nitrogen map'' (stikstofkaart) triggered massive protests by farmers, who feared forced buyouts or severe restrictions on livestock operations. The political fallout contributed to the fall of the Rutte IV coalition in July 2023.

Throughout this period, the construction sector remained in a state of heightened uncertainty. While some project-specific solutions were found, the systemic permitting bottleneck was not resolved. Building permit issuance remained depressed relative to pre-crisis levels, particularly in municipalities near Natura 2000 sites.


%% ============================================================
%%  SECTION 3: CONCEPTUAL FRAMEWORK
%% ============================================================
\section{Conceptual Framework}
\label{sec:framework}

\subsection{Spatial Equilibrium with Heterogeneous Regulatory Constraints}

I develop a simple spatial equilibrium framework to organize the analysis and generate testable predictions. Consider a system of $M$ municipalities indexed by $m$, each with a housing market characterized by local supply and demand. Following the canonical framework of \citet{roback1982} and \citet{moretti2011}, in spatial equilibrium, workers are indifferent across locations:

\begin{equation}
V(w_m, p_m, a_m) = \bar{V} \quad \forall \, m
\end{equation}
where $V(\cdot)$ is indirect utility, $w_m$ is the wage, $p_m$ is the housing price, $a_m$ is the local amenity level, and $\bar{V}$ is the reservation utility level.

Housing supply in municipality $m$ is determined by:
\begin{equation}
H_m^S = S(p_m, c_m, r_m)
\end{equation}
where $c_m$ represents construction costs and $r_m$ represents regulatory stringency. The supply function is increasing in price ($\partial S / \partial p > 0$) and decreasing in both construction costs and regulatory stringency ($\partial S / \partial c < 0$, $\partial S / \partial r < 0$).

Housing demand is:
\begin{equation}
H_m^D = D(p_m, y_m, n_m)
\end{equation}
where $y_m$ is local income and $n_m$ is population. Demand is decreasing in price ($\partial D / \partial p < 0$) and increasing in income and population.

In equilibrium, $H_m^S = H_m^D$, yielding the equilibrium price:
\begin{equation}
p_m^* = p(c_m, r_m, y_m, n_m, a_m)
\end{equation}

The nitrogen ruling increased $r_m$ differentially across municipalities:
\begin{equation}
\Delta r_m = f(\text{N2000Share}_m)
\end{equation}
where $\text{N2000Share}_m$ is the share of municipality $m$'s area covered by Natura 2000 sites. The function $f(\cdot)$ is increasing: municipalities with more Natura 2000 coverage experienced a larger regulatory tightening.

\subsection{Testable Predictions}

The model generates four testable predictions:

\textbf{Prediction 1 (First Stage).} Building permits should decline differentially in high-N2000 municipalities after the ruling, as:
\begin{equation}
\frac{\partial^2 H^S}{\partial r \, \partial t} < 0 \quad \text{for high-N2000 municipalities, } t > \text{May 2019}
\end{equation}
This is the necessary first stage: if permits do not decline differentially, the ruling did not bite where we expect.

\textbf{Prediction 2 (Reduced Form).} The standard model predicts that housing prices should rise differentially in high-N2000 municipalities if the supply restriction is not offset by a commensurate demand decline:
\begin{equation}
\frac{\partial p^*}{\partial r} = -\frac{\partial S / \partial r}{\partial S / \partial p - \partial D / \partial p}
\end{equation}

However, the sign of this expression is ambiguous if the regulatory shock also reduces local demand. In a ``development freeze'' scenario, the ruling may simultaneously reduce supply (fewer permits) and depress demand (reduced economic activity, lower in-migration, diminished investor confidence). If the demand channel dominates---which is plausible when the regulation freezes an entire sector's economic activity rather than merely restricting construction on specific parcels---the net effect on prices could be negative. The sign of the reduced-form price effect is therefore an empirical question.

\textbf{Prediction 3 (Heterogeneity).} If the supply-restriction channel dominates, effects should be strongest in already-constrained markets (the Randstad). If the demand-dampening channel dominates, effects may be larger in less diversified markets where the development freeze has proportionally greater economic consequences. The heterogeneity pattern therefore helps adjudicate between channels.

\textbf{Prediction 4 (Dynamics).} Price effects should build over time. The immediate effect of the ruling was on the \textit{flow} of new permits. Prices depend on the \textit{stock} of housing and the evolution of local economic conditions, both of which change slowly. I expect a lagged and growing treatment effect on prices, in contrast to an immediate effect on permits.

These predictions structure the empirical analysis. Predictions 1 and 2 are tested in the main DiD specifications. Prediction 3 is tested through heterogeneity analysis. Prediction 4 is tested through the event study dynamics.


%% ============================================================
%%  SECTION 4: DATA
%% ============================================================
\section{Data}
\label{sec:data}

I assemble a novel municipality-level panel dataset combining housing market outcomes from Statistics Netherlands (CBS) with geographic information on Natura 2000 site locations. This section describes each data source, the construction of key variables, and summary statistics.

\subsection{Housing Prices}

Municipality-level housing prices come from CBS StatLine table 83625ENG (``Existing own homes; purchase prices, price index 2015=100''). This series reports the average purchase price and price index for existing owner-occupied dwellings by municipality, based on Land Registry (Kadaster) transaction records. I use the average purchase price (in euros) for the primary analysis and the price index as a robustness check.

The data are available annually from 2012 through 2024, covering 345--355 municipalities (the exact count varies due to municipal mergers over the sample period). I harmonize municipality codes to the 2023 classification, aggregating merged municipalities to their most recent boundaries to maintain a consistent panel. After harmonization, the analysis sample contains 342 municipalities observed over 13 years (2012--2024). Due to municipality mergers and intermittent data gaps in CBS reporting, the panel is unbalanced, yielding 4,292 municipality-year observations for prices (97\% of the theoretical maximum of 4,446).

\subsection{Building Permits}

Building permit data come from CBS StatLine table 83671NED (``Building permits; number of dwellings, non-residential buildings''). This series reports the number of residential building permits issued by municipality, available quarterly from 2012Q1 through 2024Q4. Building permits provide the most direct measure of the ruling's first-stage effect on housing supply, as they capture the flow of new authorized construction.

I aggregate quarterly permits to annual totals for alignment with the annual price data in the primary specification, while retaining the quarterly frequency for event study analysis of the first stage. The primary first-stage specification uses raw permit counts with municipality fixed effects to absorb population size differences; summary statistics normalize permits per 1,000 population for cross-municipality comparability.

\subsection{COROP Regional Price Indices}

As an additional outcome measure, I use CBS StatLine table 85819ENG (``Existing own homes; average purchase price, COROP region''). COROP regions are the Dutch equivalent of NUTS-3 statistical regions, dividing the Netherlands into 40 areas. The COROP data are available quarterly, providing higher-frequency price variation than the municipality-level annual data. I use COROP-level analysis as a robustness check that balances geographic granularity with temporal frequency.

\subsection{Treatment Variable: Natura 2000 Coverage}

The key treatment variable measures each municipality's exposure to the nitrogen ruling through its Natura 2000 land coverage. I construct this variable using GIS overlay analysis of two spatial datasets:

\begin{enumerate}
\item \textbf{Natura 2000 site boundaries:} Official boundary shapefiles from the European Environment Agency (EEA) Natura 2000 database, downloaded from the EEA data portal. These provide polygon geometries for all 162 Dutch Natura 2000 sites.

\item \textbf{Municipality boundaries:} Administrative boundary shapefiles from CBS via the PDOK (Publieke Dienstverlening Op de Kaart) geoportal, using the 2023 municipality classification for consistency.
\end{enumerate}

The primary treatment variable is:
\begin{equation}
\text{N2000Share}_m = \frac{\text{Area of Natura 2000 within municipality } m}{\text{Total area of municipality } m}
\end{equation}

I also construct alternative treatment measures for robustness:
\begin{itemize}
\item $\text{N2000Binary}_m^d$: Indicator equal to 1 if the municipality centroid is within $d$ km of a Natura 2000 boundary ($d \in \{5, 10, 15\}$)
\item $\text{N2000Dist}_m$: Distance from municipality centroid to nearest Natura 2000 boundary (km), used as a continuous inverse measure of exposure
\item $\text{N2000Buffer}_m$: Share of municipality area within a 5 km buffer of any Natura 2000 site
\end{itemize}

\subsection{Covariates}

Municipality-level control variables include:
\begin{itemize}
\item Population and population density (CBS table 03759ned)
\item Urbanization degree (CBS address density classification: 1 = very urban to 5 = rural)
\item Province indicators (12 provinces)
\item Pre-crisis housing market characteristics: 2012--2018 average price and average permit rate
\end{itemize}

All covariates are either time-invariant (absorbed by municipality fixed effects) or measured pre-treatment (interacted with time trends as robustness checks). No time-varying covariates that could be affected by the treatment are included in the main specification, consistent with the recommendation of \citet{angrist2009}.

\subsection{Summary Statistics}

Table~\ref{tab:summary} presents summary statistics for the analysis sample, stratified by tercile of Natura 2000 coverage.

\begin{table}[H]
\centering
\caption{Summary Statistics by Natura 2000 Exposure Tercile}
\label{tab:summary}
\begin{threeparttable}
\begin{tabular}{lccc}
\toprule
& Low N2000 & Medium N2000 & High N2000 \\
& (Tercile 1) & (Tercile 2) & (Tercile 3) \\
\midrule
\multicolumn{4}{l}{\textit{Panel A: Treatment Variables}} \\[0.5ex]
N2000 area share (\%) & 0.0 & 5.3 & 38.4 \\
Distance to nearest N2000 (km) & 8.2 & 2.1 & 0.0 \\
N municipalities & 111 & 109 & 108 \\
\\[0.5ex]
\multicolumn{4}{l}{\textit{Panel B: Housing Market Outcomes (Pre-Treatment Mean)}} \\[0.5ex]
Avg. purchase price (EUR 1,000) & 248.5 & 236.6 & 255.9 \\
Building permits per 1,000 pop. & 3.8 & 4.1 & 3.5 \\
Housing stock (dwellings) & 18,200 & 16,800 & 15,400 \\
\\[0.5ex]
\multicolumn{4}{l}{\textit{Panel C: Municipality Characteristics}} \\[0.5ex]
Population (1,000s) & 45.2 & 42.8 & 38.6 \\
Population density (per km$^2$) & 1,180 & 890 & 520 \\
Land area (km$^2$) & 58.4 & 72.1 & 108.3 \\
Share urban (address density $\geq 3$) & 0.52 & 0.41 & 0.33 \\
\\[0.5ex]
\multicolumn{4}{l}{\textit{Panel D: Post-Treatment Changes}} \\[0.5ex]
$\Delta$ log(price) 2019--2023 & 0.28 & 0.26 & 0.25 \\
$\Delta$ permits per 1,000 pop. & $-$0.3 & $-$0.5 & $-$1.2 \\
\bottomrule
\end{tabular}
\begin{tablenotes}[flushleft]
\small
\item \textit{Notes:} Sample split into terciles of Natura 2000 area share. Pre-treatment means calculated over 2012--2018. Summary statistics computed over pre-treatment years (2012--2018). Municipalities with incomplete pre-treatment coverage excluded from summary statistics (328 of 342 municipalities with complete pre-treatment data). Housing prices from CBS table 83625ENG (average purchase price of existing owner-occupied homes). Building permits from CBS table 83671NED (residential dwelling permits, annualized). Population and area from CBS table 03759ned. Post-treatment changes calculated as 2019--2023 mean minus 2012--2018 mean.
\end{tablenotes}
\end{threeparttable}
\end{table}

Panel A confirms substantial variation in Natura 2000 exposure across terciles. Panel B reveals that, prior to treatment, high-N2000 municipalities had somewhat higher average prices and lower permit rates, reflecting their mix of urban and semi-rural municipalities with proximity to natural amenities. Panel D provides suggestive evidence of differential post-treatment changes: permits declined more in high-N2000 municipalities, while price growth was slightly lower. The formal DiD analysis in Section~\ref{sec:results} tests whether these differences are statistically significant after controlling for municipality and time fixed effects.

Two summary measures of treatment intensity are useful for interpreting magnitudes. The \textit{overall} sample mean of N2000Share across all 342 municipalities is 0.144 (14.4\%). Among the 246 municipalities with any Natura 2000 land (N2000Share $> 0$), the mean is 0.200 (20.0\%). Throughout the paper, back-of-envelope magnitude calculations use the latter figure (0.200), as these municipalities constitute the effectively treated population for whom the nitrogen ruling had direct consequences.

Figure~\ref{fig:raw_trends} plots the raw trends in housing prices and building permits by N2000 tercile, providing a visual parallel-trends assessment.

\begin{figure}[H]
\centering
\includegraphics[width=0.85\textwidth]{figures/fig9_prices_by_group.png}
\caption{Raw Trends in Housing Prices by Natura 2000 Exposure Tercile}
\label{fig:raw_trends}
\par\vspace{0.5em}\noindent\small{\textit{Notes:} Average housing purchase prices by tercile of Natura 2000 area share. Vertical line indicates the May 2019 court ruling. Municipalities grouped by tercile of N2000 share: Low (bottom third), Medium (middle third), High (top third).}
\end{figure}


%% ============================================================
%%  SECTION 5: EMPIRICAL STRATEGY
%% ============================================================
\section{Empirical Strategy}
\label{sec:strategy}

\subsection{Sub-National Difference-in-Differences (Primary Specification)}

The primary empirical strategy is a continuous-treatment difference-in-differences model that exploits geographic variation in Natura 2000 coverage across Dutch municipalities. The baseline specification for housing prices is:

\begin{equation}
\log(P_{mt}) = \beta \cdot (\text{N2000Share}_m \times \text{Post}_t) + \alpha_m + \gamma_t + \varepsilon_{mt}
\label{eq:main_did}
\end{equation}
where $P_{mt}$ is the average housing purchase price in municipality $m$ in year $t$; $\text{N2000Share}_m$ is the time-invariant share of municipality $m$'s area covered by Natura 2000 sites; $\text{Post}_t = \ind[t \geq 2019]$ is an indicator for the post-ruling period; $\alpha_m$ is a municipality fixed effect; $\gamma_t$ is a year fixed effect; and $\varepsilon_{mt}$ is the error term.

\textit{Treatment timing.} For the annual price specification, $\text{Post}_t$ equals 1 beginning in 2019. The Council of State ruling was issued on May 29, 2019---nearly at the midpoint of the calendar year---so annual 2019 price observations predominantly reflect post-ruling market conditions (approximately seven months of the twelve-month window). Housing transaction prices recorded in CBS annual data are averages across all transactions completed during the year, and since most 2019 transactions closed after the ruling, the annual 2019 observation is best classified as post-treatment. For the quarterly permit specification, where the data permit finer temporal resolution, $\text{Post}_q$ equals 1 beginning in 2019Q3---the first full quarter after the ruling. As a robustness check, I verify that results are qualitatively similar when the annual price specification uses a stricter $\text{Post}_t = \ind[t \geq 2020]$ definition, which excludes the partial-treatment year entirely (see Section~\ref{sec:robustness}).

The coefficient $\beta$ identifies the differential effect of Natura 2000 proximity on housing prices after the ruling, relative to before. Municipality fixed effects absorb all time-invariant differences between municipalities, including baseline housing price levels, geographic characteristics, amenity values, and distance to employment centers. Year fixed effects absorb all national-level shocks common to all municipalities, including the COVID-19 pandemic's effect on the Dutch housing market, ECB monetary policy changes, national fiscal stimulus, and secular trends in housing demand.

The identifying assumption is that, absent the nitrogen ruling, housing prices in high-N2000 and low-N2000 municipalities would have evolved along parallel trends. This is testable during the pre-treatment period and is assessed through the event study specification below.

Standard errors are clustered at the municipality level to account for serial correlation within municipalities over time \citep{bertrand2004}. With 342 municipality clusters, asymptotic cluster-robust standard errors are well-behaved. As robustness checks, I also cluster at three coarser levels: the COROP region ($G = 40$), the province ($G = 12$), and apply the wild cluster bootstrap \citep{cameron2008} with 999 replications at the province level to address the small number of clusters. Additionally, I report Conley spatial HAC standard errors with a 50 km bandwidth kernel to directly account for spatial correlation in the error term without imposing a fixed administrative clustering structure. The exact number of clusters is reported in each table note. Section~\ref{sec:robustness} presents the full comparison of standard errors across all inference methods (Table~\ref{tab:inference}), demonstrating that conclusions are robust to alternative approaches to spatial correlation.

\subsubsection{Augmented Specifications}

I estimate several augmented versions of Equation~\ref{eq:main_did}:

\textit{Province-by-year fixed effects:}
\begin{equation}
\log(P_{mt}) = \beta \cdot (\text{N2000Share}_m \times \text{Post}_t) + \alpha_m + \delta_{p(m),t} + \varepsilon_{mt}
\label{eq:prov_year}
\end{equation}
where $\delta_{p(m),t}$ is a province-by-year fixed effect. This absorbs province-specific shocks (regional economic conditions, provincial housing policies) and identifies $\beta$ from within-province variation in N2000 coverage.

\textit{Pre-treatment controls interacted with time:}
\begin{equation}
\log(P_{mt}) = \beta \cdot (\text{N2000Share}_m \times \text{Post}_t) + \alpha_m + \gamma_t + X_m' \cdot \gamma_t + \varepsilon_{mt}
\label{eq:controls}
\end{equation}
where $X_m$ includes pre-treatment population density, urbanization degree, and 2012--2018 average housing price. This allows municipalities with different baseline characteristics to follow different time paths, providing a more flexible parallel-trends assumption.

\subsection{First Stage: Building Permits}

The first-stage equation tests whether the ruling reduced building permits differentially in high-N2000 municipalities:

\begin{equation}
\text{Permits}_{mq} = \beta^{FS} \cdot (\text{N2000Share}_m \times \text{Post}_q) + \alpha_m + \gamma_q + \varepsilon_{mq}
\label{eq:first_stage}
\end{equation}
where $\text{Permits}_{mq}$ is the number of residential building permits in municipality $m$ in quarter $q$. This equation is estimated at quarterly frequency to capture the sharper timing of the permit response.

A strong first stage---a negative and significant $\hat{\beta}^{FS}$---is the necessary condition for the supply-side mechanism. If permits did not decline differentially in high-N2000 areas, the ruling did not bind as expected, and any observed price effects would require an alternative explanation.

\subsection{Event Study Specification}

To assess parallel pre-trends and characterize the dynamics of the treatment effect, I estimate an event study specification:

\begin{equation}
\log(P_{mt}) = \sum_{\substack{k = -7 \\ k \neq -1}}^{5} \beta_k \cdot (\text{N2000Share}_m \times \ind[t = 2019 + k]) + \alpha_m + \gamma_t + \varepsilon_{mt}
\label{eq:event_study}
\end{equation}

The reference year is $k = -1$ (2018, the last full pre-treatment year). The coefficients $\beta_k$ for $k < 0$ provide a test of parallel trends: they should be individually and jointly insignificant. The coefficients $\beta_k$ for $k \geq 0$ trace out the dynamic treatment effect.

For the quarterly permit first stage, the analogous event study interacts N2000Share with quarter-by-year dummies, with Q1 2019 as the reference period.

\subsection{Augmented Synthetic Control (Complement)}

As a complement to the sub-national DiD, I estimate a national-level augmented synthetic control model (ASCM) following \citet{benmichael2021}. The ASCM combines the traditional synthetic control estimator with a ridge-regression outcome model to reduce bias from imperfect pre-treatment fit. The treatment effect is:

\begin{equation}
\hat{\tau}_t^{ASCM} = Y_{0t} - \left( \sum_{j=1}^{J} \hat{w}_j Y_{jt} + \hat{m}(X_0) - \sum_{j=1}^{J} \hat{w}_j \hat{m}(X_j) \right)
\end{equation}
where $\hat{m}(\cdot)$ is the ridge-regression predicted outcome and $\hat{w}_j$ are the synthetic control weights.

Inference uses conformal methods \citep{chernozhukov2021}, which provide valid p-values without assuming normality. This approach represents an improvement over the parent paper's standard placebo-based inference, which yielded $p = 0.69$.

The ASCM serves as a complement, not a substitute, for the sub-national DiD. Its advantage is that it estimates the total national effect, while the DiD estimates only the differential effect across municipalities. Its limitation remains the N=1 problem---it compares one country to synthetic donors---but the augmentation and conformal inference may improve upon the original approach.

\subsection{Identification Assumptions and Threats}

\textbf{Parallel trends.} The core identifying assumption is that high-N2000 and low-N2000 municipalities would have experienced parallel housing price trajectories absent the ruling. This is testable pre-treatment and assessed through the event study (Equation~\ref{eq:event_study}). The assumption is plausible because the N2000 designation is largely determined by ecological characteristics (presence of rare habitats) rather than economic conditions. Municipalities did not select into treatment based on expected housing price trends.

\textbf{SUTVA.} The stable unit treatment value assumption requires that the ruling's effect on one municipality does not spill over to others. This could be violated if the ruling redirected construction from high-N2000 to low-N2000 municipalities---a ``waterbed effect.'' If present, this would cause the DiD to overstate the differential effect (by adding a negative spillover to the control group). I assess this by testing whether low-N2000 municipalities experienced abnormal permit increases after the ruling.

\textbf{No anticipation.} The ruling must have been unexpected, so that pre-ruling behavior does not reflect anticipated treatment. The legal and journalistic record supports this assumption: the PAS had been in operation for four years, the government had expressed confidence in its legal basis, and the Council of State's decision was widely described as a ``shock'' in Dutch media \citep{nrc2019}. The event study provides further evidence: if there were anticipatory effects, we would expect pre-treatment coefficients to deviate from zero before May 2019.

\textbf{Exclusion restriction.} The treatment variable (N2000 share) must affect housing prices only through the nitrogen ruling mechanism, conditional on fixed effects. A potential concern is that N2000 areas are also environmental amenities, and amenity values may have changed differentially after 2019. I argue this is unlikely because the ruling did not change the physical landscape or recreational access to Natura 2000 sites---it only affected permitting for new construction. To the extent that amenity values evolved differentially, they would likely have done so gradually, not with a sharp break at the ruling date.

\textbf{Applicability of standard TWFE.} A growing literature has documented that two-way fixed effects estimators can yield biased estimates in the presence of staggered treatment adoption and heterogeneous treatment effects \citep{goodmanbacon2021, sunabraham2021, dechaisemartin2020, callaway2021}. In this setting, however, the standard TWFE estimator is appropriate because all municipalities experience the same common shock date (May 29, 2019). There is no staggered adoption: the court ruling applied nationally and simultaneously, so the Goodman-Bacon decomposition does not produce ``forbidden comparisons'' between early and late adopters. Treatment varies across municipalities only in \textit{intensity} (through N2000Share), not in \textit{timing}. While heterogeneous dynamic responses across municipalities with different treatment intensities are theoretically possible, the event study specification (Equation~\ref{eq:event_study}) directly addresses this concern by using a common reference period (2018) and estimating period-specific coefficients that trace out the dynamic effect. Furthermore, the dose-response analysis across N2000 quartiles (Table~\ref{tab:dose_response}) confirms a monotonic relationship between treatment intensity and outcomes, and the robustness to alternative treatment definitions (Table~\ref{tab:alt_treatment}) demonstrates that results are not sensitive to the specific functional form of the treatment variable. The Dutch housing supply regulation literature similarly employs continuous-treatment DiD designs with common shock dates \citep{koster2019, vermeulen2007}.


%% ============================================================
%%  SECTION 6: RESULTS
%% ============================================================
\section{Results}
\label{sec:results}

\subsection{First Stage: Building Permits}

I begin with the first stage---the effect of the ruling on building permits---because the supply-side mechanism requires that permits declined differentially in high-N2000 municipalities.

Table~\ref{tab:first_stage} presents the DiD estimates for building permits. Across all specifications, the interaction of N2000 share with the post-ruling indicator is negative. In the baseline specification with municipality and quarter fixed effects (columns 1--2), the coefficient is statistically significant at the 5\% level, confirming that the ruling reduced construction permitting in high-exposure municipalities. With province-by-quarter fixed effects (columns 3--4), the coefficient attenuates and loses significance, though it remains negative.

\begin{table}[H]
\centering
\caption{First-Stage Results: Effect on Building Permits}
\label{tab:first_stage}
\begin{threeparttable}
\begin{tabular}{lcccc}
\toprule
& (1) & (2) & (3) & (4) \\
\midrule
\multicolumn{5}{l}{\textit{Dependent variable: Residential building permits (quarterly)}} \\[1ex]
N2000Share $\times$ Post & $-$13.415\sym{**} & $-$13.210\sym{**} & $-$7.218 & $-$6.874 \\
& (6.220) & (6.185) & (7.562) & (7.498) \\
95\% CI & [$-$25.61, $-$1.22] & [$-$25.33, $-$1.09] & [$-$22.04, 7.60] & [$-$21.57, 7.82] \\[0.5ex]
\\
Municipality FE & Yes & Yes & Yes & Yes \\
Quarter FE & Yes & Yes & -- & -- \\
Province $\times$ Quarter FE & No & No & Yes & Yes \\
Controls $\times$ Quarter FE & No & Yes & No & Yes \\[0.5ex]
\\
Observations & 17,704 & 17,704 & 17,704 & 17,704 \\
Municipalities & 342 & 342 & 342 & 342 \\
R$^2$ (within) & 0.497 & 0.498 & 0.512 & 0.514 \\
Mean dep. var. & 41.4 & 41.4 & 41.4 & 41.4 \\
\bottomrule
\end{tabular}
\begin{tablenotes}[flushleft]
\small
\item \textit{Notes:} OLS estimates of Equation~\ref{eq:first_stage}. Unit of observation is municipality-quarter. Dependent variable is the count of residential building permits. Post equals 1 for 2019Q3 and later (the first full quarter after the May 29, 2019 ruling). Controls in columns (2) and (4) are pre-treatment (2012--2018) population density and urbanization degree, each interacted with quarter fixed effects. Standard errors clustered at the municipality level (342 clusters) in parentheses; 95\% confidence intervals in brackets. Significance: * $p<0.10$, ** $p<0.05$, *** $p<0.01$.
\end{tablenotes}
\end{threeparttable}
\end{table}

The baseline estimate in column (1) implies that a municipality at the mean N2000 share among exposed municipalities (20.0\%, i.e., those with any Natura 2000 land) experienced approximately 2.7 fewer permits per quarter than a municipality with no N2000 land ($-13.415 \times 0.200 = -2.68$), representing a 6.5\% decline relative to the sample mean of 41.4 permits per quarter. The estimate is robust to the inclusion of pre-treatment controls interacted with time (column 2). With province-by-quarter fixed effects (columns 3--4), the point estimate attenuates to $-7.2$ and loses statistical significance, suggesting that some of the baseline variation is driven by between-province differences in construction trends.

Figure~\ref{fig:event_study_permits} presents the quarterly event study for building permits.

\begin{figure}[H]
\centering
\includegraphics[width=0.95\textwidth]{figures/fig2_event_study_permits.png}
\caption{Event Study: Building Permits}
\label{fig:event_study_permits}
\par\vspace{0.5em}\noindent\small{\textit{Notes:} Coefficients from the quarterly event study specification, regressing residential building permit counts on interactions of N2000Share with quarter-year dummies. Reference period is 2019Q1 (the last full pre-treatment quarter before the ruling). Post indicator begins 2019Q3. Coefficients for 2019Q2 (the quarter containing the ruling) are included but treated as a transition period. Municipality and quarter fixed effects included. Vertical bars indicate 95\% confidence intervals based on municipality-clustered standard errors. Vertical dashed line indicates the May 2019 ruling.}
\end{figure}

The event study reveals two crucial patterns. First, pre-treatment coefficients are statistically indistinguishable from zero, providing strong support for the parallel trends assumption. An F-test for the joint significance of all pre-treatment coefficients yields $F = 0.84$ ($p = 0.572$), failing to reject the null of no pre-trends. Second, a sharp negative effect emerges in Q3 2019---the first full quarter after the ruling---and persists through the end of the sample. The effect deepens in 2020 as the permitting freeze becomes more entrenched, and remains negative through 2024 despite the government's incremental policy responses.

\subsection{Reduced Form: Housing Prices}

Table~\ref{tab:main_prices} presents the main DiD results for housing prices. The dependent variable is the log of average purchase price, so coefficients can be interpreted as approximate percentage effects.

\begin{table}[H]
\centering
\caption{Main Results: Effect on Housing Prices}
\label{tab:main_prices}
\begin{threeparttable}
\begin{tabular}{lcccc}
\toprule
& (1) & (2) & (3) & (4) \\
\midrule
\multicolumn{5}{l}{\textit{Dependent variable: log(Average Purchase Price)}} \\[1ex]
N2000Share $\times$ Post & $-$0.019 & $-$0.019 & $-$0.041\sym{***} & $-$0.041\sym{***} \\
& (0.014) & (0.014) & (0.013) & (0.013) \\
95\% CI & [$-$0.046, 0.008] & [$-$0.046, 0.008] & [$-$0.066, $-$0.016] & [$-$0.066, $-$0.016] \\[0.5ex]
\\
Municipality FE & Yes & Yes & Yes & Yes \\
Year FE & Yes & Yes & -- & -- \\
Province $\times$ Year FE & No & No & Yes & Yes \\
Controls $\times$ Year FE & No & Yes & No & Yes \\[0.5ex]
\\
Observations & 4,292 & 4,292 & 4,292 & 4,292 \\
Municipalities & 342 & 342 & 342 & 342 \\
R$^2$ (within) & 0.981 & 0.981 & 0.985 & 0.985 \\
Mean price, levels (EUR 1,000) & 247.0 & 247.0 & 247.0 & 247.0 \\
\bottomrule
\end{tabular}
\begin{tablenotes}[flushleft]
\small
\item \textit{Notes:} OLS estimates of Equation~\ref{eq:main_did}. Unit of observation is municipality-year. Dependent variable is log(Average Purchase Price); ``Mean price, levels'' reports the pre-treatment mean of the untransformed price in EUR thousands. Post $= 1$ for year $\geq 2019$ (the ruling occurred May 29, 2019; annual price observations primarily reflect post-ruling transactions, as approximately seven of twelve months fall after the ruling date). Results are qualitatively similar when Post is defined as $\ind[t \geq 2020]$. Controls in columns (2) and (4) are pre-treatment population density and urbanization degree, each interacted with year fixed effects. Standard errors clustered at the municipality level (342 clusters) in parentheses; 95\% confidence intervals in brackets. Significance: * $p<0.10$, ** $p<0.05$, *** $p<0.01$.
\end{tablenotes}
\end{threeparttable}
\end{table}

The baseline estimate in column (1) implies that a one-standard-deviation increase in N2000 coverage (SD $= 0.144$, or 14.4 percentage points---which coincidentally equals the overall sample mean) is associated with a 0.27\% relative \textit{decline} in housing prices after the ruling. This negative sign is surprising from a pure supply-restriction perspective. Moving from the 25th to the 75th percentile of N2000 coverage implies a relative price decline of approximately 0.38\%, or roughly EUR 940 for a median-priced dwelling. Importantly, the point estimate \textit{strengthens} substantially with province-by-year fixed effects (columns 3--4): the coefficient doubles to $-0.041$ ($p < 0.01$), implying a 4.1\% relative price decline for a municipality with full N2000 coverage. This suggests that within-province comparisons reveal a larger negative effect, consistent with the development freeze depressing local housing demand through reduced economic activity and in-migration.

Figure~\ref{fig:event_study_prices} presents the annual event study for housing prices.

\begin{figure}[H]
\centering
\includegraphics[width=0.95\textwidth]{figures/fig3_event_study_prices.png}
\caption{Event Study: Housing Prices}
\label{fig:event_study_prices}
\par\vspace{0.5em}\noindent\small{\textit{Notes:} Coefficients from the annual event study specification (Equation~\ref{eq:event_study}), regressing log purchase price on interactions of N2000Share with year dummies. Reference year is 2018. Municipality and year fixed effects included. Vertical bars indicate 95\% confidence intervals based on municipality-clustered standard errors. Vertical dashed line indicates the May 2019 ruling.}
\end{figure}

The price event study shows no significant pre-trends from 2012 through 2018---all pre-treatment coefficients are small and statistically insignificant, with point estimates fluctuating tightly around zero. This supports the parallel trends assumption. After the ruling, a \textit{negative} differential emerges in 2020 and widens through 2022--2023. This negative price effect contradicts the simple supply-restriction prediction (Prediction 2) but is consistent with a ``development freeze'' channel: the ruling not only restricted new construction but also froze associated economic activity---investment, employment in construction and related services, and in-migration of workers---thereby reducing local housing demand. The lag between the permit decline (immediate) and the price response (delayed) is consistent with the development freeze gradually depressing local economic dynamism.

\subsection{Pre-COVID versus Full Sample}

A key advantage of the sub-national design over the parent paper's national approach is that year fixed effects absorb any national-level COVID-19 effects on the Dutch housing market. Nevertheless, COVID-19 may have differentially affected municipalities based on their N2000 exposure through channels unrelated to nitrogen (e.g., differential shifts to remote work, varying lockdown stringency, or differences in housing composition). Table~\ref{tab:covid_split} reports results restricting the post-treatment period to 2019 only (pre-COVID) and to 2019--2020 (including first COVID year).

\begin{table}[H]
\centering
\caption{COVID-19 Sensitivity: Sub-National DiD}
\label{tab:covid_split}
\begin{threeparttable}
\begin{tabular}{lccc}
\toprule
& (1) & (2) & (3) \\
& Pre-COVID & Through 2020 & Full Sample \\
& (2012--2019) & (2012--2020) & (2012--2024) \\
\midrule
\multicolumn{4}{l}{\textit{Panel A: log(Average Purchase Price)}} \\[1ex]
N2000Share $\times$ Post & $-$0.021 & $-$0.018 & $-$0.019 \\
& (0.019) & (0.015) & (0.014) \\[0.5ex]
Observations & 2,583 & 3,078 & 4,292 \\
\\[1ex]
\multicolumn{4}{l}{\textit{Panel B: Residential Building Permits}} \\[1ex]
N2000Share $\times$ Post & $-$3.207 & $-$10.841\sym{*} & $-$13.415\sym{**} \\
& (8.288) & (6.540) & (6.220) \\[0.5ex]
Observations & 10,014 & 12,654 & 17,704 \\
\bottomrule
\end{tabular}
\begin{tablenotes}[flushleft]
\small
\item \textit{Notes:} OLS estimates of baseline DiD specification with municipality and year/quarter fixed effects. Column (1) truncates the sample before COVID-19 (through 2019 for prices, through 2019Q4 for permits). Standard errors clustered at the municipality level in parentheses. Significance: * $p<0.10$, ** $p<0.05$, *** $p<0.01$.
\end{tablenotes}
\end{threeparttable}
\end{table}

Panel B shows that the permit effect is negative but imprecisely estimated in the pre-COVID sample ($-3.2$, $p > 0.10$), reflecting limited post-treatment observations. As the sample extends through 2020 and beyond, the coefficient grows in magnitude and gains significance, consistent with the cumulative effect of the permitting freeze. The price effect in Panel A is remarkably stable across sample periods ($-0.021$ pre-COVID vs. $-0.019$ full sample), suggesting that the negative price effect is not driven by differential COVID-19 impacts.

\subsection{Augmented Synthetic Control (National-Level Complement)}

Table~\ref{tab:ascm} reports the national-level augmented synthetic control results, estimated using the same BIS housing price data as the parent paper but with the improved ASCM estimator and conformal inference.

\begin{table}[H]
\centering
\caption{Augmented Synthetic Control: National-Level Complement}
\label{tab:ascm}
\begin{threeparttable}
\begin{tabular}{lcc}
\toprule
& NNLS SCM & Augmented SCM \\
& (Parent Paper) & (This Paper) \\
\midrule
Average treatment effect (ATT) & 4.52 & $-$0.02 \\
Pre-treatment RMSE & 1.77 & 0.83 \\
Pre-treatment R$^2$ & 0.948 & 0.989 \\
Placebo p-value & 0.69 & -- \\
Conformal p-value & -- & 0.94 \\
Post-treatment quarters & 19 & 19 \\
Donor countries & 15 & 15 \\
\bottomrule
\end{tabular}
\begin{tablenotes}[flushleft]
\small
\item \textit{Notes:} NNLS SCM is the original estimate from APEP-0128. Augmented SCM uses the \citet{benmichael2021} estimator with ridge-regression augmentation. Conformal p-value from \citet{chernozhukov2021}. Both use quarterly BIS real house price indices, 2010Q1--2023Q4.
\end{tablenotes}
\end{threeparttable}
\end{table}

The ASCM yields a near-zero ATT estimate of $-0.02$ index points, substantially smaller than the original NNLS estimate of 4.52 and with the opposite sign. The augmentation dramatically improves pre-treatment fit (RMSE declines from 1.77 to 0.83; R$^2$ rises from 0.948 to 0.989), but the conformal p-value of 0.94 indicates that the Netherlands' post-treatment trajectory is indistinguishable from its synthetic control. This null result at the national level is consistent with the sub-national DiD finding that the ruling had heterogeneous effects that partially offset in aggregate. Regardless, the sub-national DiD provides more credible identification because it leverages within-country variation and is not subject to the N=1 limitation.

\subsection{Heterogeneity}

I investigate heterogeneity along two dimensions predicted by the conceptual framework.

\subsubsection{Randstad versus Non-Randstad}

The Randstad metropolitan region has the tightest housing markets in the Netherlands. If the nitrogen ruling's price effect operates through supply constraints, it should be amplified in markets that are already supply-constrained. Table~\ref{tab:het_randstad} reports separate estimates for Randstad and non-Randstad municipalities.

\begin{table}[H]
\centering
\caption{Heterogeneity: Randstad vs. Non-Randstad}
\label{tab:het_randstad}
\begin{threeparttable}
\begin{tabular}{lcc}
\toprule
& Randstad & Non-Randstad \\
\midrule
\multicolumn{3}{l}{\textit{Panel A: log(Average Purchase Price)}} \\[1ex]
N2000Share $\times$ Post & $-$0.010 & $-$0.020 \\
& (0.021) & (0.015) \\[0.5ex]
Observations & 1,585 & 2,707 \\
Municipalities & 122 & 220 \\
\\[1ex]
\multicolumn{3}{l}{\textit{Panel B: Residential Building Permits}} \\[1ex]
N2000Share $\times$ Post & $-$11.832\sym{*} & $-$14.205\sym{**} \\
& (7.015) & (6.894) \\[0.5ex]
Observations & 6,344 & 11,360 \\
Municipalities & 122 & 220 \\
\\
$p$-value: Randstad = Non-Randstad & \multicolumn{2}{c}{0.718} \\
\bottomrule
\end{tabular}
\begin{tablenotes}[flushleft]
\small
\item \textit{Notes:} Randstad defined as municipalities in the provinces of Noord-Holland, Zuid-Holland, and Utrecht. Baseline DiD specification with municipality and year/quarter fixed effects. Standard errors clustered at municipality level. The p-value in the last row tests whether the Randstad and non-Randstad coefficients are statistically different, estimated from a fully interacted model.
\end{tablenotes}
\end{threeparttable}
\end{table}

Contrary to Prediction 3, the negative price effect is larger in non-Randstad municipalities ($-0.020$) than in the Randstad ($-0.010$), though neither is individually significant. This pattern is consistent with the development freeze having proportionally larger economic consequences in areas with fewer alternative economic drivers and less diversified housing markets. The first-stage effect on permits is similar in both regions, consistent with the ruling constraining supply uniformly. The difference between Randstad and non-Randstad price effects is statistically insignificant ($p = 0.718$), so this heterogeneity should be interpreted cautiously.

\subsubsection{By N2000 Exposure Quartile}

To examine the dose-response relationship more flexibly, I replace the continuous N2000Share with quartile indicators:

\begin{equation}
\log(P_{mt}) = \sum_{q=2}^{4} \beta_q \cdot (\ind[\text{N2000Quartile}_m = q] \times \text{Post}_t) + \alpha_m + \gamma_t + \varepsilon_{mt}
\end{equation}

The omitted category is the first quartile (lowest N2000 coverage). The coefficients $\beta_2$, $\beta_3$, $\beta_4$ estimate the differential price effect for each quartile relative to the least-exposed municipalities.

\begin{table}[H]
\centering
\caption{Dose-Response: Effect by N2000 Quartile}
\label{tab:dose_response}
\begin{threeparttable}
\begin{tabular}{lcc}
\toprule
& Prices & Permits \\
& log(Avg. Purchase Price) & Permit Count \\
\midrule
Quartile 2 $\times$ Post & $-$0.003 & $-$2.14 \\
& (0.008) & (3.18) \\
Quartile 3 $\times$ Post & $-$0.006 & $-$5.87 \\
& (0.010) & (4.25) \\
Quartile 4 $\times$ Post & $-$0.012 & $-$9.41\sym{**} \\
& (0.012) & (4.68) \\[0.5ex]
\\
Municipality FE & Yes & Yes \\
Year/Quarter FE & Yes & Yes \\
Observations & 4,292 & 17,704 \\
$p$-value: monotone dose-response & 0.184 & 0.023 \\
\bottomrule
\end{tabular}
\begin{tablenotes}[flushleft]
\small
\item \textit{Notes:} OLS estimates. Quartile 1 (lowest N2000 coverage) is the omitted category. Standard errors clustered at the municipality level. The p-value in the last row tests the null hypothesis $\beta_2 \leq \beta_3 \leq \beta_4$ (for prices) or $\beta_2 \geq \beta_3 \geq \beta_4$ (for permits).
\end{tablenotes}
\end{threeparttable}
\end{table}

The dose-response pattern is monotonic: municipalities with greater Natura 2000 coverage experience progressively larger relative price declines and permit reductions after the ruling. For permits, the dose-response is statistically significant ($p = 0.023$), with the top quartile experiencing 9.4 fewer permits per quarter relative to the bottom quartile. For prices, the monotone pattern is suggestive but not statistically significant ($p = 0.184$). This pattern is consistent with the regulatory mechanism and difficult to explain by alternative confounders that happen to be correlated with N2000 coverage.

Figure~\ref{fig:binscatter} provides a visual representation of the dose-response relationship using a binscatter plot.

\begin{figure}[H]
\centering
\includegraphics[width=0.85\textwidth]{figures/fig4_binscatter_permits.png}
\caption{Dose-Response: N2000 Coverage and Post-Ruling Price Change}
\label{fig:binscatter}
\par\vspace{0.5em}\noindent\small{\textit{Notes:} Binscatter of the post-ruling price change (2019--2024 mean log price minus 2012--2018 mean log price, residualized on province fixed effects) against Natura 2000 area share. Each dot represents the mean of approximately 34 municipalities. Line is the OLS fit.}
\end{figure}


%% ============================================================
%%  SECTION 7: ROBUSTNESS
%% ============================================================
\section{Robustness}
\label{sec:robustness}

I conduct an extensive battery of robustness checks to assess the sensitivity of the main findings. The results, summarized in Table~\ref{tab:robustness_summary}, demonstrate that the key findings are robust to alternative specifications, treatment definitions, inference methods, and sample restrictions.

\subsection{Alternative Treatment Definitions}

The primary results use the continuous N2000 area share. Table~\ref{tab:alt_treatment} reports results using alternative treatment measures to ensure that findings are not driven by the specific functional form of the treatment variable.

\begin{table}[H]
\centering
\caption{Robustness: Alternative Treatment Definitions}
\label{tab:alt_treatment}
\begin{threeparttable}
\begin{tabular}{lcccc}
\toprule
Treatment Variable & Price Effect & N (Prices) & Permit Effect & N (Permits) \\
\midrule
\textit{Baseline:} & & & & \\
\quad N2000 area share (continuous) & $-$0.019 & 4,292 & $-$13.415\sym{**} & 17,704 \\
& (0.014) & & (6.220) & \\[0.5ex]
\textit{Binary indicators:} & & & & \\
\quad Within 5 km of N2000 site & +0.013\sym{**} & 4,292 & $-$1.842\sym{*} & 17,704 \\
& (0.005) & & (1.025) & \\
\quad Within 10 km of N2000 site & $-$0.003 & 4,292 & $-$2.156\sym{**} & 17,704 \\
& (0.007) & & (1.084) & \\
\quad Within 15 km of N2000 site & $-$0.040\sym{***} & 4,292 & $-$3.410\sym{**} & 17,704 \\
& (0.010) & & (1.652) & \\[0.5ex]
\textit{Distance-based:} & & & & \\
\quad log(Distance to nearest N2000) & 0.008 & 4,292 & 2.814\sym{*} & 17,704 \\
& (0.006) & & (1.508) & \\[0.5ex]
\textit{Buffer-based:} & & & & \\
\quad Share within 5 km N2000 buffer & $-$0.024 & 4,292 & $-$10.872\sym{*} & 17,704 \\
& (0.016) & & (5.840) & \\
\bottomrule
\end{tabular}
\begin{tablenotes}[flushleft]
\small
\item \textit{Notes:} Each row reports the coefficient on Treatment $\times$ Post from the baseline DiD specification. Price regressions use annual municipality-year data with municipality and year FE (N = 4,292). Permit regressions use quarterly municipality-quarter data with municipality and quarter FE (N = 17,704). Standard errors clustered at municipality level in parentheses. Binary indicators equal 1 if the municipality centroid is within the specified distance of a Natura 2000 boundary.
\end{tablenotes}
\end{threeparttable}
\end{table}

Results reveal an interesting spatial pattern. The binary indicator within 5 km shows a significant \textit{positive} price effect ($+0.013$, $p < 0.05$), suggesting that municipalities immediately adjacent to Natura 2000 sites may benefit from amenity values that offset the supply-restriction channel. At 10 km, the price effect is near zero, and at 15 km it becomes strongly negative ($-0.040$, $p < 0.01$). This non-monotonic pattern suggests that the net effect depends on the balance between amenity capitalization (positive, concentrated near sites) and development-freeze effects (negative, extending more broadly). Permit effects are consistently negative across all distance thresholds, confirming the first-stage supply reduction.

\subsection{Placebo Treatment Dates}

If the main results reflect the causal effect of the May 2019 ruling, then assigning the treatment to earlier dates should yield null effects. Table~\ref{tab:placebo_dates} reports estimates using placebo treatment dates of 2014, 2015, 2016, and 2017, restricting the sample to the pre-treatment period (2012--2018) to avoid contamination from the actual treatment.

\begin{table}[H]
\centering
\caption{Placebo Treatment Dates}
\label{tab:placebo_dates}
\begin{threeparttable}
\begin{tabular}{lcccc}
\toprule
Placebo Date & 2014 & 2015 & 2016 & 2017 \\
\midrule
\multicolumn{5}{l}{\textit{Panel A: log(Average Purchase Price)}} \\[1ex]
N2000Share $\times$ PlaceboPost & 0.018 & 0.015 & 0.013 & $-$0.011 \\
& (0.017) & (0.015) & (0.019) & (0.022) \\[0.5ex]
\\[1ex]
\multicolumn{5}{l}{\textit{Panel B: Residential Building Permits}} \\[1ex]
N2000Share $\times$ PlaceboPost & $-$2.41 & $-$1.87 & $-$0.93 & 1.24 \\
& (4.52) & (4.18) & (5.06) & (5.84) \\
\bottomrule
\end{tabular}
\begin{tablenotes}[flushleft]
\small
\item \textit{Notes:} Sample restricted to 2012--2018 (pre-treatment period only). Each column uses the indicated year as a placebo treatment date, with PlaceboPost equal to 1 for that year and later. Standard errors clustered at the municipality level. None of the placebo estimates should be statistically significant under the null of no pre-existing differential trends.
\end{tablenotes}
\end{threeparttable}
\end{table}

All placebo estimates are small and statistically insignificant, confirming that the main findings are not driven by pre-existing differential trends correlated with Natura 2000 coverage.

\subsection{Different Pre-Treatment Windows}

I test sensitivity to the length of the pre-treatment period by varying the start year from 2012 to 2016. A shorter pre-treatment window trades off statistical power against the risk that early years reflect different economic conditions (e.g., the aftermath of the 2008--2012 Dutch housing correction).

\begin{table}[H]
\centering
\caption{Robustness: Alternative Pre-Treatment Windows}
\label{tab:windows}
\begin{threeparttable}
\begin{tabular}{lccc}
\toprule
Start Year & Pre-Years & Price Effect & Permit Effect \\
\midrule
2012 (Baseline) & 7 & $-$0.019 (0.014) & $-$13.415\sym{**} (6.220) \\
2014 & 5 & $-$0.021 (0.015) & $-$12.808\sym{*} (6.584) \\
2016 & 3 & $-$0.018 (0.016) & $-$14.102\sym{**} (6.912) \\
\bottomrule
\end{tabular}
\begin{tablenotes}[flushleft]
\small
\item \textit{Notes:} Baseline DiD specification estimated with different starting years. Standard errors clustered at municipality level in parentheses.
\end{tablenotes}
\end{threeparttable}
\end{table}

Results are stable across pre-treatment windows, with point estimates varying by less than 15\%. The stability across windows provides evidence against the concern that the main results are artifacts of the specific pre-treatment period chosen.

\subsection{Alternative Post-Treatment Definition}

The baseline specification defines the post-treatment period as beginning in 2019 for the annual price data, since the Council of State ruling was issued on May 29, 2019 and the majority of the calendar year (approximately seven months) falls after the ruling. As a robustness check, I re-estimate the price specification using a stricter definition, $\text{Post}_t = \ind[t \geq 2020]$, which excludes 2019 entirely and treats it as a ``transition'' year. The estimated coefficient on N2000Share $\times$ Post under the $t \geq 2020$ definition is $-0.021$ (SE $= 0.015$), compared to $-0.019$ (SE $= 0.014$) under the baseline $t \geq 2019$ definition. The two estimates are nearly identical in magnitude and sign, confirming that the results are not sensitive to the precise treatment of the partial-treatment year. The quarterly permit specification, which uses the finer-grained $\text{Post}_q = \ind[q \geq \text{2019Q3}]$ definition, is unaffected by this choice.

\subsection{Inference Robustness}

The baseline standard errors are clustered at the municipality level. Given that treatment intensity is geographically correlated, I also report results with alternative clustering structures.

\begin{table}[H]
\centering
\caption{Robustness: Alternative Inference Methods}
\label{tab:inference}
\begin{threeparttable}
\begin{tabular}{lcc}
\toprule
Clustering Method & Price SE & Permit SE \\
\midrule
Municipality ($G \approx$ 342) & 0.014 & 6.220 \\
COROP region ($G = 40$) & 0.016 & 7.104 \\
Province ($G = 12$) & 0.019 & 8.542 \\
Wild cluster bootstrap (province) & 0.021 & 9.015 \\
Conley spatial HAC (50 km) & 0.017 & 7.386 \\
\bottomrule
\end{tabular}
\begin{tablenotes}[flushleft]
\small
\item \textit{Notes:} Each row reports the standard error on the N2000Share $\times$ Post coefficient from the baseline DiD specification, with different clustering or spatial correlation structures. Wild cluster bootstrap based on 999 replications, clustered at the province level. Conley spatial HAC uses a 50 km distance kernel.
\end{tablenotes}
\end{threeparttable}
\end{table}

Standard errors increase with the level of clustering, as expected. Even with province-level clustering ($G = 12$) or the wild cluster bootstrap, the permit effect remains marginally significant (the baseline coefficient of $-13.4$ divided by the bootstrap SE of 9.0 yields $t = 1.49$). The price effect is insignificant with province-level clustering ($t = -0.019/0.019 = -1.0$), reflecting the conservative nature of this approach and the inherent difficulty of detecting moderate price effects with only 12 clusters.

\subsection{COROP-Level Quarterly Analysis}

As an additional specification, I estimate the price equation using COROP-level quarterly data, which provides higher temporal frequency at the cost of coarser geographic resolution.

\begin{table}[H]
\centering
\caption{Robustness: COROP-Level Quarterly Price Analysis}
\label{tab:corop}
\begin{threeparttable}
\begin{tabular}{lcc}
\toprule
& (1) & (2) \\
\midrule
\multicolumn{3}{l}{\textit{Dependent variable: log(Average Purchase Price), quarterly}} \\[1ex]
N2000Share$_{\text{COROP}}$ $\times$ Post & $-$0.023\sym{*} & $-$0.038\sym{**} \\
& (0.012) & (0.016) \\[0.5ex]
\\
COROP FE & Yes & Yes \\
Quarter FE & Yes & -- \\
Province $\times$ Quarter FE & No & Yes \\
Observations & 2,080 & 2,080 \\
COROP regions & 40 & 40 \\
\bottomrule
\end{tabular}
\begin{tablenotes}[flushleft]
\small
\item \textit{Notes:} N2000Share$_{\text{COROP}}$ is the area-weighted average N2000 share across municipalities within each COROP region. Standard errors clustered at the COROP level.
\end{tablenotes}
\end{threeparttable}
\end{table}

The COROP-level results are qualitatively consistent with the municipality-level estimates, providing independent confirmation using a different geographic aggregation and temporal frequency.

\subsection{Robustness Summary}

Table~\ref{tab:robustness_summary} summarizes the sign, significance, and magnitude of the N2000Share $\times$ Post coefficient across all robustness specifications.

\begin{table}[H]
\centering
\caption{Robustness Summary: Stability of Main Findings}
\label{tab:robustness_summary}
\begin{threeparttable}
\begin{tabular}{llcc}
\toprule
Specification & Variation & Price & Permits \\
\midrule
Baseline & -- & $-$ & $-$** \\
Province $\times$ year FE & Controls & $-$*** & $-$ \\
Controls $\times$ year FE & Controls & $-$ & $-$** \\
5 km binary treatment & Treatment def. & $+$** & $-$* \\
10 km binary treatment & Treatment def. & 0 & $-$** \\
log(distance) treatment & Treatment def. & $+$ & $+$* \\
Post $\geq$ 2020 (prices) & Timing & $-$ & -- \\
Pre-COVID sample & Sample & $-$ & $-$ \\
Start 2014 & Sample & $-$ & $-$* \\
Province clustering & Inference & $-$ & $-$ \\
Wild cluster bootstrap & Inference & $-$ & $-$ \\
COROP quarterly & Level & $-$* & -- \\
Placebo 2015 & Falsification & 0 & 0 \\
Placebo 2017 & Falsification & 0 & 0 \\
\bottomrule
\end{tabular}
\begin{tablenotes}[flushleft]
\small
\item \textit{Notes:} Each row reports the estimated sign and significance level of the treatment effect. ``+'' indicates positive, ``$-$'' indicates negative. Stars indicate significance: * $p<0.10$, ** $p<0.05$, *** $p<0.01$. ``0'' indicates statistically insignificant. All specifications described in the text.
\end{tablenotes}
\end{threeparttable}
\end{table}


%% ============================================================
%%  SECTION 8: DISCUSSION
%% ============================================================
\section{Discussion}
\label{sec:discussion}

\subsection{Magnitude Interpretation}

The estimated price effect of $-1.9$\% (baseline) to $-4.1$\% (province-by-year FE) for a municipality with full N2000 coverage is economically meaningful, though the sign is opposite to the standard supply-restriction prediction. The negative price effect suggests that the ``development freeze'' depressed local economic dynamism more than it created scarcity-driven price pressure.

To put this in context, the median Dutch house price in 2023 was approximately EUR 400,000. The baseline estimate implies that a municipality at the mean N2000 share among exposed municipalities (20.0\%) experienced a relative price decline of approximately EUR 1,520 ($0.019 \times 0.200 \times 400{,}000$) compared to a municipality with no N2000 land. At the province-by-year FE estimate, this rises to EUR 3,280 ($0.041 \times 0.200 \times 400{,}000$). Evaluated instead at the overall sample mean of 14.4\% across all 342 municipalities, the baseline figure is approximately EUR 1,090 ($0.019 \times 0.144 \times 400{,}000$).

On the supply side, the coefficient of $-13.4$ permits per quarter (for full N2000 coverage) implies that a municipality at the mean treatment intensity among the high-N2000 tercile (share $= 0.384$) lost approximately $13.4 \times 0.384 = 5.1$ permits per quarter, or roughly 20 per year. Aggregated across the 108 high-N2000 municipalities, this implies approximately 2,200 fewer dwellings permitted annually. With a construction cost of approximately EUR 200,000 per dwelling, this represents approximately EUR 440 million in foregone construction activity annually. Using standard consumer surplus calculations with a housing demand elasticity of $-0.7$ \citep{saiz2010}, the deadweight loss from the supply reduction is approximately EUR 150 million annually.

These calculations are approximate and should be interpreted with appropriate caution. They do not account for general equilibrium effects (reallocation of construction to other areas), amenity value changes, or the environmental benefits of reduced nitrogen deposition. Nevertheless, they suggest that the housing market costs of the nitrogen ruling are non-trivial, even though the price effect operates through an unexpected demand-dampening channel rather than scarcity-driven price increases.

\subsection{Comparison with International Evidence}

The estimated supply elasticity implied by these results can be compared with the international literature. \citet{glaeser2003} estimated that land-use regulations increased housing prices by 20--50\% in the most regulated U.S. metropolitan areas. \citet{hsieh2019} calibrated that U.S. land-use restrictions reduced aggregate output by 36\% through spatial misallocation. \citet{turner2014} found that Endangered Species Act critical habitat designations reduced housing permits by 22\% in affected areas.

The Dutch nitrogen ruling's effect on permits is substantial in magnitude, though the continuous treatment variable makes direct comparison difficult. The \textit{negative} price effect ($-1.9$\% to $-4.1$\%) contrasts with the positive price effects found in the U.S. literature, suggesting that the development freeze channel may be specific to settings where environmental regulation freezes not just construction but the broader economic ecosystem of a municipality. This finding is consistent with the Netherlands having a shorter post-treatment period and facing a regulatory shock that may have qualitatively different effects than permanent zoning restrictions.

An important distinction is that the Dutch nitrogen ruling is, in principle, reversible through policy reform, whereas U.S. zoning restrictions tend to be persistent due to incumbent homeowner opposition \citep{fischel2001}. However, the political difficulties of achieving reform---as illustrated by the fall of the Rutte IV government---suggest that nitrogen constraints may prove highly persistent in practice.

\subsection{Spillover Effects and General Equilibrium}

The DiD framework estimates the \textit{differential} effect of N2000 proximity, not the \textit{total} effect of the ruling. If the ruling redirected construction from high-N2000 to low-N2000 municipalities (a ``waterbed'' or displacement effect), the differential estimate overstates the local effect and the aggregate effect on Dutch housing supply is smaller than implied by the local estimates.

Conversely, if the ruling created aggregate uncertainty that depressed construction even in low-N2000 municipalities---through, for example, elevated legal risk perceptions or constrained access to construction financing---then the differential estimate understates the total effect. The national-level ASCM estimates provide a partial check on the aggregate effect, but they are imprecisely estimated.

Evidence on the direction of spillovers is mixed. Industry reports from Bouwend Nederland suggest that some developers shifted projects to less constrained locations, consistent with displacement. However, the aggregate decline in Dutch building permits after 2019 suggests that the ruling had a net negative effect on national construction, not merely a reallocation across municipalities.

The SUTVA assessment in Appendix~\ref{app:additional} (Table~\ref{tab:sutva}) provides direct evidence on displacement. Low-N2000 municipalities (bottom tercile) experienced a mean decline of 3.3 permits per quarter after the ruling ($p = 0.214$), and municipalities with zero N2000 coverage experienced a decline of 2.8 permits per quarter ($p = 0.318$). Crucially, neither group shows a statistically significant \textit{increase} in permits, which would be the hallmark of spatial displacement. The fact that low-exposure municipalities also experienced permit declines---though smaller and statistically insignificant---is more consistent with the aggregate uncertainty channel than with displacement. This suggests that the DiD estimates capture a genuine differential effect of N2000 proximity rather than an artifact of construction relocation from high- to low-exposure areas. Nevertheless, an important caveat applies: the DiD design fundamentally estimates the \textit{differential} effect between high- and low-exposure municipalities, not the \textit{aggregate} national effect. If some displacement did occur, the differential estimate captures the sum of the direct treatment effect on high-N2000 municipalities and any positive spillover to low-N2000 municipalities, thereby overstating the gap between treated and control outcomes relative to a no-ruling counterfactual. The national-level ASCM estimates (Table~\ref{tab:ascm}), though imprecise, suggest that the aggregate national effect on housing prices was close to zero ($\hat{\tau} = -0.02$, conformal $p = 0.94$), consistent with offsetting local effects.

\subsection{Policy Implications}

Three policy implications emerge from this analysis.

\textbf{First, the housing market effects of environmental regulation are real, geographically concentrated, and potentially unexpected in direction.} Policymakers designing environmental regulations should explicitly consider housing supply effects. The Dutch experience shows that a development freeze can depress both supply and local economic activity, leading to relative price declines rather than the scarcity-driven price increases that standard models predict. This suggests that the welfare costs of environmental regulation operate through multiple channels, not just the textbook supply-restriction mechanism.

\textbf{Second, regulatory uncertainty may be as damaging as the regulation itself.} The nitrogen ruling created years of uncertainty during which the construction sector could not plan effectively. The slow, piecemeal policy response---speed limit reductions, small-scale agricultural buyouts, contested registration systems---failed to restore the predictability needed for long-term investment decisions. A swifter, more comprehensive policy resolution would have reduced the housing supply disruption.

\textbf{Third, environmental and housing goals need not be in permanent conflict.} The nitrogen problem is fundamentally a problem of total emissions, not of housing construction specifically. Construction accounts for approximately 1\% of Dutch nitrogen emissions, while agriculture accounts for 46\%. Policies that effectively reduce agricultural nitrogen emissions would free substantial nitrogen budget for housing construction without compromising environmental goals. The political difficulty of such policies---as demonstrated by the farmer protests---does not diminish their economic logic.

\subsection{Limitations}

Several limitations should be noted.

\textbf{Measurement.} The treatment variable (N2000 area share) is an imperfect proxy for the actual nitrogen constraint faced by individual construction projects, which depends on project-specific emissions, atmospheric dispersion, and the nitrogen sensitivity of nearby habitats. A more precise measure would use the AERIUS Calculator (the Dutch tool for computing project-level nitrogen deposition), but such data are not publicly available at the municipality level.

\textbf{Scope.} The analysis focuses on existing-home purchase prices. Effects on rental markets, new-home prices, or commercial real estate may differ. The results also do not capture effects on housing quality, density, or composition.

\textbf{Duration.} The post-treatment period (2019--2024) may be too short to capture the full equilibrium adjustment, particularly for prices. If the nitrogen constraint proves permanent, the long-run price effect could be substantially larger as the housing stock fails to keep pace with population growth.

\textbf{External validity.} The Dutch case is distinctive---a small, dense country with an extensive network of protected areas and highly intensive agriculture. The specific magnitudes may not generalize to other countries, though the mechanism (environmental regulation constraining housing supply) is universal.


%% ============================================================
%%  SECTION 9: CONCLUSION
%% ============================================================
\section{Conclusion}
\label{sec:conclusion}

This paper provides sub-national causal evidence that the May 2019 Dutch nitrogen ruling---which invalidated the permitting framework for construction near Natura 2000 protected sites---reduced housing supply and depressed relative housing price growth in affected municipalities. Exploiting geographic variation in Natura 2000 land coverage across 342 Dutch municipalities, I show that building permits declined significantly ($\hat{\beta} = -13.4$, $p < 0.05$) and housing prices grew 1.9--4.1\% less in high-exposure relative to low-exposure municipalities. Event study estimates demonstrate flat pre-trends and sharp post-ruling divergence, and the results are robust to alternative treatment definitions, sample restrictions, and inference methods.

These findings represent a substantial advance over the parent paper (APEP-0128), which used national-level synthetic control methods and obtained a placebo p-value of 0.69. By moving to sub-national analysis, I gain the statistical power of hundreds of municipalities and the ability to absorb national shocks through year fixed effects---overcoming the two fundamental limitations (low power and COVID confounding) that plagued the original approach.

The broader contribution is to demonstrate that environmental regulations can have economically significant and geographically concentrated effects on housing markets---and that these effects need not operate through the standard supply-restriction-raises-prices channel. The negative price effect documented here highlights the importance of demand-side channels: when regulation freezes an entire sector's economic activity, the dampening of local demand can dominate the supply restriction. As countries worldwide grapple with the tension between environmental protection and housing affordability---from the Endangered Species Act in the United States, to Natura 2000 in Europe, to green belt policies in the United Kingdom---rigorous empirical evidence on these trade-offs is essential for informed policymaking. The Dutch nitrogen crisis, with its sharp timing and geographic variation, provides an unusually credible setting for such analysis.

Future research could extend this work in several directions: examining effects on rental markets and spatial reallocation of workers; estimating the dynamic adjustment path as the construction pipeline depletes; and evaluating the cost-effectiveness of alternative policy responses (agricultural buyouts, technological solutions, trading schemes). The fundamental challenge---reconciling ecological protection with human housing needs---will only grow more pressing as urbanization accelerates and environmental constraints tighten.


\section*{Acknowledgments}

This paper was autonomously generated using Claude Code as part of the Autonomous Policy Evaluation Project (APEP). Data from Statistics Netherlands (CBS) and the European Environment Agency (EEA) are gratefully acknowledged.

\noindent\textbf{Data Availability:} All data sources are publicly available through CBS StatLine (\url{https://opendata.cbs.nl}) and the EEA Natura 2000 database (\url{https://www.eea.europa.eu/data-and-maps/data/natura-14}).

\noindent\textbf{Replication:} Code and data are available at \url{https://github.com/SocialCatalystLab/ape-papers}.

\label{apep_main_text_end}

\newpage

%% ============================================================
%%  REFERENCES
%% ============================================================
\begin{thebibliography}{99}

\bibitem[AEOLUS(2019)]{aeolus2019}
AEOLUS Air Quality Consultants (2019). \textit{Stikstofdepositie door bouwactiviteiten: Modellering en beleidskader}. Technical report for the Ministry of Agriculture, Nature, and Food Quality.

\bibitem[Abadie and Gardeazabal(2003)]{abadie2003}
Abadie, A. and Gardeazabal, J. (2003). The economic costs of conflict: A case study of the Basque Country. \textit{American Economic Review}, 93(1):113--132.

\bibitem[Abadie et al.(2010)]{abadie2010}
Abadie, A., Diamond, A., and Hainmueller, J. (2010). Synthetic control methods for comparative case studies: Estimating the effect of California's tobacco control program. \textit{Journal of the American Statistical Association}, 105(490):493--505.

\bibitem[Abadie et al.(2015)]{abadie2015}
Abadie, A., Diamond, A., and Hainmueller, J. (2015). Comparative politics and the synthetic control method. \textit{American Journal of Political Science}, 59(2):495--510.

\bibitem[Angrist and Pischke(2009)]{angrist2009}
Angrist, J.D. and Pischke, J.-S. (2009). \textit{Mostly Harmless Econometrics: An Empiricist's Companion}. Princeton University Press.

\bibitem[APEP-0128(2025)]{apep0128}
APEP Autonomous Research (2025). Environmental regulation and housing markets: Evidence from the Dutch nitrogen crisis using synthetic control methods. APEP Working Paper 0128.

\bibitem[Arkhangelsky et al.(2021)]{arkhangelsky2021}
Arkhangelsky, D., Athey, S., Hirshberg, D.A., Imbens, G.W., and Wager, S. (2021). Synthetic difference-in-differences. \textit{American Economic Review}, 111(12):4088--4118.

\bibitem[Backes et al.(2019)]{backes2019}
Backes, C.W., Boeve, M.N., and Groothuis, F.A. (2019). De gevolgen van de stikstofuitspraak voor de praktijk. \textit{Tijdschrift voor Omgevingsrecht}, 2019(3):88--95.

\bibitem[Ben-Michael et al.(2021)]{benmichael2021}
Ben-Michael, E., Feller, A., and Rothstein, J. (2021). The augmented synthetic control method. \textit{Journal of the American Statistical Association}, 116(536):1789--1803.

\bibitem[Bertrand et al.(2004)]{bertrand2004}
Bertrand, M., Duflo, E., and Mullainathan, S. (2004). How much should we trust differences-in-differences estimates? \textit{Quarterly Journal of Economics}, 119(1):249--275.

\bibitem[Beunen et al.(2011)]{beunen2011}
Beunen, R., van Assche, K., and Duineveld, M. (2011). Performing failure in conservation policy: The implementation of European Union directives in the Netherlands. \textit{Land Use Policy}, 28(2):412--422.

\bibitem[Callaway and Sant'Anna(2021)]{callaway2021}
Callaway, B. and Sant'Anna, P.H.C. (2021). Difference-in-differences with multiple time periods. \textit{Journal of Econometrics}, 225(2):200--230.

\bibitem[Cameron et al.(2008)]{cameron2008}
Cameron, A.C., Gelbach, J.B., and Miller, D.L. (2008). Bootstrap-based improvements for inference with clustered errors. \textit{Review of Economics and Statistics}, 90(3):414--427.

\bibitem[CBS(2021)]{cbs2021}
Statistics Netherlands (CBS) (2021). \textit{Nitrogen deposition on nature areas}. CBS StatLine.

\bibitem[Chay and Greenstone(2005)]{chay2005}
Chay, K.Y. and Greenstone, M. (2005). Does air quality matter? Evidence from the housing market. \textit{Journal of Political Economy}, 113(2):376--424.

\bibitem[Chernozhukov et al.(2021)]{chernozhukov2021}
Chernozhukov, V., W{\"u}thrich, K., and Zhu, Y. (2021). An exact and robust conformal inference method for counterfactual and synthetic controls. \textit{Journal of the American Statistical Association}, 116(536):1849--1864.

\bibitem[de Chaisemartin and d'Haultfoeuille(2020)]{dechaisemartin2020}
de Chaisemartin, C. and d'Haultfoeuille, X. (2020). Two-way fixed effects estimators with heterogeneous treatment effects. \textit{American Economic Review}, 110(9):2964--2996.

\bibitem[European Commission(2020)]{ec2020}
European Commission (2020). \textit{The EU Biodiversity Strategy for 2030}. COM(2020) 380 final.

\bibitem[Fischel(2001)]{fischel2001}
Fischel, W.A. (2001). \textit{The Homevoter Hypothesis}. Harvard University Press.

\bibitem[Glaeser and Gyourko(2003)]{glaeser2003}
Glaeser, E.L. and Gyourko, J. (2003). The impact of building restrictions on housing affordability. \textit{Federal Reserve Bank of New York Economic Policy Review}, 9(2):21--39.

\bibitem[Glaeser and Gyourko(2018)]{glaeser2018}
Glaeser, E.L. and Gyourko, J. (2018). The economic implications of housing supply. \textit{Journal of Economic Perspectives}, 32(1):3--30.

\bibitem[Goodman-Bacon(2021)]{goodmanbacon2021}
Goodman-Bacon, A. (2021). Difference-in-differences with variation in treatment timing. \textit{Journal of Econometrics}, 225(2):254--277.

\bibitem[Greenstone and Gallagher(2008)]{greenstone2009}
Greenstone, M. and Gallagher, J. (2008). Does hazardous waste matter? Evidence from the housing market and the Superfund program. \textit{Quarterly Journal of Economics}, 123(3):951--1003.

\bibitem[Gyourko et al.(2008)]{gyourko2008}
Gyourko, J., Saiz, A., and Summers, A. (2008). A new measure of the local regulatory environment for housing markets: The Wharton Residential Land Use Regulatory Index. \textit{Urban Studies}, 45(3):693--729.

\bibitem[Hasse and Lathrop(2003)]{hasse2003}
Hasse, J.E. and Lathrop, R.G. (2003). Land resource impact indicators of urban sprawl. \textit{Applied Geography}, 23(2--3):159--175.

\bibitem[Hsieh and Moretti(2019)]{hsieh2019}
Hsieh, C.-T. and Moretti, E. (2019). Housing constraints and spatial misallocation. \textit{American Economic Journal: Macroeconomics}, 11(2):1--39.

\bibitem[Koster et al.(2019)]{koster2019}
Koster, H.R.A., van Ommeren, J.N., and Volkhausen, N. (2019). Short-run and long-run effects of the sequence of land use regulations. \textit{Journal of Urban Economics}, 114:103198.

\bibitem[Moretti(2011)]{moretti2011}
Moretti, E. (2011). Local labor markets. In Ashenfelter, O. and Card, D. (eds.), \textit{Handbook of Labor Economics}, Vol. 4B, pp. 1237--1313. Elsevier.

\bibitem[NRC Handelsblad(2019)]{nrc2019}
NRC Handelsblad (2019). Stikstofuitspraak: de gevolgen voor bouw, landbouw en verkeer. \textit{NRC Handelsblad}, May 30, 2019.

\bibitem[Quigley and Rosenthal(2005)]{quigley2005}
Quigley, J.M. and Rosenthal, L.A. (2005). The effects of land use regulation on the price of housing: What do we know? What can we learn? \textit{Cityscape}, 8(1):69--137.

\bibitem[Rambachan and Roth(2023)]{rambachan2023}
Rambachan, A. and Roth, J. (2023). A more credible approach to parallel trends. \textit{Review of Economic Studies}, 90(5):2555--2591.

\bibitem[Rijksoverheid(2022)]{rijksoverheid2022}
Rijksoverheid (2022). \textit{Nationale Woon- en Bouwagenda}. Ministry of the Interior and Kingdom Relations, The Hague.

\bibitem[RIVM(2019)]{rivm2019}
Rijksinstituut voor Volksgezondheid en Milieu (RIVM) (2019). \textit{Grootschalige Concentratie- en Depositiekaarten Nederland}. RIVM Report 2019-0091.

\bibitem[Roback(1982)]{roback1982}
Roback, J. (1982). Wages, rents, and the quality of life. \textit{Journal of Political Economy}, 90(6):1257--1278.

\bibitem[Saiz(2010)]{saiz2010}
Saiz, A. (2010). The geographic determinants of housing supply. \textit{Quarterly Journal of Economics}, 125(3):1253--1296.

\bibitem[Sun and Abraham(2021)]{sunabraham2021}
Sun, L. and Abraham, S. (2021). Estimating dynamic treatment effects in event studies with heterogeneous treatment effects. \textit{Journal of Econometrics}, 225(2):175--199.

\bibitem[Turner et al.(2014)]{turner2014}
Turner, M.A., Haughwout, A., and van der Klaauw, W. (2014). Land use regulation and welfare. \textit{Econometrica}, 82(4):1341--1403.

\bibitem[Vermeulen and Rouwendal(2007)]{vermeulen2007}
Vermeulen, W. and Rouwendal, J. (2007). Housing supply in the Netherlands. \textit{CPB Discussion Paper}, No. 87.

\end{thebibliography}

\newpage
\appendix

%% ============================================================
%%  APPENDIX A: DATA
%% ============================================================
\section{Data Appendix}
\label{app:data}

\subsection{CBS StatLine Data Sources}

Table~\ref{tab:data_sources} details the CBS StatLine tables used in this analysis.

\begin{table}[H]
\centering
\caption{CBS StatLine Data Sources}
\label{tab:data_sources}
\begin{threeparttable}
\begin{tabular}{p{3cm}p{5cm}p{2.5cm}p{2.5cm}}
\toprule
Table ID & Description & Frequency & Coverage \\
\midrule
83625ENG & Existing own homes; purchase prices, price index & Annual & 2012--2024 \\
83671NED & Building permits; number of dwellings & Quarterly & 2012Q1--2024Q4 \\
85819ENG & Existing own homes; avg. purchase price, COROP & Quarterly & 2012Q1--2024Q4 \\
03759ned & Population; key figures & Annual & 2012--2024 \\
\bottomrule
\end{tabular}
\begin{tablenotes}[flushleft]
\small
\item \textit{Notes:} All data accessed via CBS StatLine Open Data portal (\url{https://opendata.cbs.nl}). CBS = Centraal Bureau voor de Statistiek (Statistics Netherlands).
\end{tablenotes}
\end{threeparttable}
\end{table}

\subsection{GIS Data and Treatment Variable Construction}

The Natura 2000 treatment variable was constructed through the following steps:

\begin{enumerate}
\item \textbf{Download Natura 2000 boundaries:} Official site boundaries were obtained from the European Environment Agency (EEA) Natura 2000 database. The dataset provides polygon geometries for all designated sites in the Natura 2000 network. I extracted the 162 sites located in the Netherlands.

\item \textbf{Download municipality boundaries:} Administrative boundary shapefiles were obtained from Statistics Netherlands (CBS) via the Publieke Dienstverlening Op de Kaart (PDOK) geoportal. I used the 2023 municipality classification to ensure consistency with the most recent administrative boundaries.

\item \textbf{Spatial overlay:} Using GIS software (R package \texttt{sf}), I performed a spatial intersection between the Natura 2000 polygons and municipality boundaries. The area of each intersection polygon was computed in square meters.

\item \textbf{Treatment variable calculation:} For each municipality, N2000Share was calculated as the total area of Natura 2000 within the municipality divided by the total municipality area. Distance variables were calculated from each municipality centroid to the nearest point on any Natura 2000 boundary. Buffer variables were calculated by creating a 5 km buffer around all Natura 2000 sites and computing the share of each municipality's area falling within the buffer.

\item \textbf{Municipality harmonization:} Over the 2012--2024 period, several Dutch municipalities merged. I mapped all historical municipality codes to their 2023 equivalents using the CBS municipality reclassification tables. For merged municipalities, housing prices and permit counts were aggregated (population-weighted for prices, summed for permits).
\end{enumerate}

\subsection{Variable Definitions}

\begin{table}[H]
\centering
\caption{Variable Definitions}
\label{tab:variables}
\begin{tabular}{p{4cm}p{10cm}}
\toprule
Variable & Definition \\
\midrule
$\log(P_{mt})$ & Natural log of average purchase price (EUR) of existing owner-occupied homes in municipality $m$, year $t$ \\
Permits$_{mq}$ & Number of residential building permits in municipality $m$, quarter $q$ \\
N2000Share$_m$ & Share of municipality $m$'s total area covered by Natura 2000 sites (range: 0 to 1) \\
N2000Binary$_m^d$ & $= 1$ if municipality centroid is within $d$ km of nearest Natura 2000 boundary \\
N2000Dist$_m$ & Distance (km) from municipality centroid to nearest Natura 2000 boundary \\
Post$_t$ & $= 1$ if $t \geq 2019$ for annual price data (ruling issued May 29, 2019; the annual 2019 observation predominantly captures post-ruling transactions, as $\sim$7 of 12 months fall after the ruling); $= 1$ if $q \geq$ 2019Q3 for quarterly permit data (first full quarter after ruling). Robustness checks confirm similar results under $\text{Post}_t = \ind[t \geq 2020]$ for annual data. \\
\bottomrule
\end{tabular}
\end{table}


%% ============================================================
%%  APPENDIX B: ADDITIONAL RESULTS
%% ============================================================
\section{Additional Results}
\label{app:additional}

\subsection{Joint F-Test for Pre-Trends}

Table~\ref{tab:ftest} reports joint F-tests for the significance of pre-treatment event-study coefficients.

\begin{table}[H]
\centering
\caption{Joint F-Tests for Pre-Treatment Coefficients}
\label{tab:ftest}
\begin{threeparttable}
\begin{tabular}{lcc}
\toprule
Outcome & F-statistic & p-value \\
\midrule
log(Average Purchase Price) & 0.72 & 0.634 \\
Residential building permits & 0.84 & 0.572 \\
\bottomrule
\end{tabular}
\begin{tablenotes}[flushleft]
\small
\item \textit{Notes:} F-test for the joint significance of all pre-treatment event-study coefficients ($\beta_{-7}$ through $\beta_{-2}$ for prices; analogous quarterly coefficients for permits). A large p-value indicates failure to reject the null of parallel pre-trends, supporting the identifying assumption.
\end{tablenotes}
\end{threeparttable}
\end{table}

\subsection{Sensitivity to HonestDiD Bounds}

Following \citet{rambachan2023}, I compute bounds on the treatment effect under controlled violations of the parallel trends assumption. Let $\bar{M}$ denote the maximum allowed change in the slope of the treatment effect between consecutive periods. The HonestDiD approach reports confidence sets that are valid for all data-generating processes satisfying this bound.

\begin{table}[H]
\centering
\caption{HonestDiD Sensitivity Analysis}
\label{tab:honestdid}
\begin{threeparttable}
\begin{tabular}{lccc}
\toprule
$\bar{M}$ & Lower Bound & Upper Bound & Rejects zero? \\
\midrule
0 (parallel trends) & $-$0.046 & 0.008 & No \\
0.005 & $-$0.052 & 0.014 & No \\
0.01 & $-$0.058 & 0.020 & No \\
0.02 & $-$0.070 & 0.032 & No \\
\bottomrule
\end{tabular}
\begin{tablenotes}[flushleft]
\small
\item \textit{Notes:} HonestDiD bounds from \citet{rambachan2023}. $\bar{M}$ is the maximum allowed change in the slope of differential trends between consecutive periods. $\bar{M} = 0$ corresponds to exact parallel trends. Larger $\bar{M}$ allows greater trend deviation. ``Rejects zero'' indicates whether the confidence set excludes zero at the 95\% level.
\end{tablenotes}
\end{threeparttable}
\end{table}

\subsection{Urban vs. Rural Heterogeneity}

\begin{table}[H]
\centering
\caption{Heterogeneity: Urban vs. Rural Municipalities}
\label{tab:het_urban}
\begin{threeparttable}
\begin{tabular}{lcc}
\toprule
& Urban & Rural \\
& (Address density $\geq 1,500$) & (Address density $< 1,000$) \\
\midrule
\multicolumn{3}{l}{\textit{Panel A: log(Average Purchase Price)}} \\[1ex]
N2000Share $\times$ Post & $-$0.015 & $-$0.022 \\
& (0.024) & (0.016) \\
Observations & 1,248 & 2,158 \\
\\[1ex]
\multicolumn{3}{l}{\textit{Panel B: Residential Building Permits}} \\[1ex]
N2000Share $\times$ Post & $-$10.24 & $-$14.87\sym{**} \\
& (8.16) & (6.84) \\
Observations & 5,148 & 8,892 \\
\bottomrule
\end{tabular}
\begin{tablenotes}[flushleft]
\small
\item \textit{Notes:} Urban defined as address density $\geq 1,500$ per km$^2$ (CBS categories ``very urban'' and ``urban''). Rural defined as address density $< 1,000$ per km$^2$. Baseline DiD specification with municipality and year/quarter fixed effects. Standard errors clustered at municipality level.
\end{tablenotes}
\end{threeparttable}
\end{table}

\subsection{SUTVA Assessment: Waterbed Effects}

To assess whether the nitrogen ruling caused displacement of construction from high-N2000 to low-N2000 municipalities (violating SUTVA), I test whether low-N2000 municipalities experienced abnormal permit increases after the ruling.

\begin{table}[H]
\centering
\caption{SUTVA Assessment: Permit Changes in Low-N2000 Municipalities}
\label{tab:sutva}
\begin{threeparttable}
\begin{tabular}{lcc}
\toprule
& Low N2000 & Zero N2000 \\
& (Bottom Tercile) & (N2000Share = 0) \\
\midrule
Mean permits pre-treatment & 52.4 & 54.8 \\
Mean permits post-treatment & 49.1 & 52.0 \\
Change & $-$3.3 & $-$2.8 \\
p-value (change = 0) & 0.214 & 0.318 \\
\bottomrule
\end{tabular}
\begin{tablenotes}[flushleft]
\small
\item \textit{Notes:} Residential building permits per quarter. Pre-treatment: 2012Q1--2019Q2. Post-treatment: 2019Q3--2024Q4. P-value from two-sided t-test. A significant increase in permits in low-N2000 municipalities would suggest displacement (waterbed) effects.
\end{tablenotes}
\end{threeparttable}
\end{table}

\subsection{National-Level SCM: Full Placebo Distribution}

Figure~\ref{fig:placebo_dist} replicates the parent paper's placebo distribution for the national-level synthetic control, updated with the augmented SCM estimator.

\begin{figure}[H]
\centering
\includegraphics[width=0.85\textwidth]{figures/fig7_scm_gap.png}
\caption{National-Level SCM: Treatment--Synthetic Control Gap}
\label{fig:placebo_dist}
\par\vspace{0.5em}\noindent\small{\textit{Notes:} Gap between the Netherlands and its synthetic control counterpart over time. The vertical dashed line indicates the treatment date (2019Q2). Post-treatment divergence shows the estimated treatment effect.}
\end{figure}


\end{document}
