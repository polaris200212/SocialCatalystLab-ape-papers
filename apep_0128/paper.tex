\documentclass[12pt]{article}

% UTF-8 encoding and fonts
\usepackage[utf8]{inputenc}
\usepackage[T1]{fontenc}
\usepackage{lmodern}

% Page setup
\usepackage[margin=1in]{geometry}
\usepackage{setspace}
\onehalfspacing

% Typography
\usepackage{microtype}

% Math and symbols
\usepackage{amsmath,amssymb}

% Graphics
\usepackage{graphicx}
\usepackage{float}
\usepackage{subcaption}

% Tables
\usepackage{booktabs}
\usepackage{array}
\usepackage{multirow}
\usepackage{threeparttable}
\usepackage{longtable}
\usepackage{pdflscape}
\usepackage{siunitx}
\sisetup{detect-all=true, group-separator={,}, group-minimum-digits=4}

% Bibliography
\usepackage{natbib}
\bibliographystyle{aer}

% Hyperlinks
\usepackage{hyperref}
\hypersetup{
    colorlinks=true,
    linkcolor=blue,
    citecolor=blue,
    urlcolor=blue
}
\usepackage[nameinlink,noabbrev]{cleveref}

% Captions
\usepackage{caption}
\captionsetup{font=small,labelfont=bf}

% Section formatting
\usepackage{titlesec}
\titleformat{\section}{\large\bfseries}{\thesection.}{0.5em}{}
\titleformat{\subsection}{\normalsize\bfseries}{\thesubsection}{0.5em}{}

% Custom commands
\newcommand{\E}{\mathbb{E}}
\newcommand{\Var}{\text{Var}}
\newcommand{\Cov}{\text{Cov}}
\newcommand{\ind}{\mathbb{I}}
\newcommand{\sym}[1]{\ifmmode^{#1}\else\(^{#1}\)\fi}

\title{Environmental Regulation and Housing Markets: Evidence from the Dutch Nitrogen Crisis Using Synthetic Control Methods}
\author{APEP Autonomous Research\thanks{Autonomous Policy Evaluation Project. Contact: \url{https://github.com/SocialCatalystLab/auto-policy-evals. Correspondence: scl@econ.uzh.ch}}}
\date{\today}

\begin{document}

\maketitle

\begin{abstract}
\noindent
Environmental regulations can have unintended consequences for housing markets by constraining new construction. This paper examines the causal effect of the Dutch nitrogen crisis---triggered by a landmark court ruling on May 29, 2019 that halted construction projects near protected Natura 2000 areas---on national house prices. Using a synthetic control method with 15 European countries as potential donors, I construct a counterfactual Netherlands that closely matches pre-treatment housing price trends. The synthetic Netherlands is composed primarily of Portugal (42\%), Spain (38\%), and France (21\%), achieving an R-squared of 0.95 during the pre-treatment period. I find that Dutch house prices were on average 4.5 index points (5.0\%) higher than the synthetic control in the post-treatment period, consistent with supply constraints raising housing costs. However, robustness analysis reveals substantial confounding with COVID-19: the pre-COVID effect (2019Q2--2019Q4) is only 0.58 index points, while the full-sample effect is driven primarily by 2020--2022 dynamics. Placebo tests and leave-one-out analysis confirm the robustness of the synthetic control construction but cannot fully disentangle nitrogen-specific effects from pandemic-era housing market changes common across Europe.
\end{abstract}

\vspace{1em}
\noindent\textbf{JEL Codes:} R31, Q58, K32, R52 \\
\noindent\textbf{Keywords:} housing prices, environmental regulation, synthetic control, nitrogen crisis, Netherlands, Natura 2000

\newpage

\section{Introduction}

Environmental regulations are increasingly recognized as important determinants of housing supply and affordability. While the primary goals of such regulations---protecting ecosystems, reducing pollution, and preserving biodiversity---are well-established, their effects on housing markets are less understood. When environmental rules constrain where and how construction can occur, they may inadvertently exacerbate housing shortages and drive up prices, particularly in already-constrained markets. Understanding these trade-offs is essential for designing policies that balance environmental protection with housing affordability.

The Netherlands provides a compelling setting to study these dynamics. The country faces severe housing shortages, with waiting lists for social housing often exceeding a decade in major cities. At the same time, it is one of the most densely populated countries in Europe, with extensive networks of protected Natura 2000 areas---the European Union's flagship biodiversity conservation program. The intersection of these two pressures came to a head on May 29, 2019, when the Dutch Council of State (Raad van State) ruled that the Programmatic Approach to Nitrogen (PAS) was invalid because it failed to adequately protect nitrogen-sensitive habitats. This ruling, known as the ``stikstofcrisis'' (nitrogen crisis), immediately halted approximately 18,000 construction and infrastructure projects across the country.

This paper estimates the causal effect of the nitrogen ruling on Dutch house prices using synthetic control methods. The identification strategy exploits the sharp, unexpected timing of the court decision---which affected the entire Netherlands simultaneously---and constructs a counterfactual price trajectory using a weighted combination of 15 other European countries. The synthetic control approach is well-suited to this setting because it provides a data-driven method for selecting comparison units and transparently reports the quality of pre-treatment fit.

My analysis yields several findings. First, I find that the synthetic Netherlands closely tracks actual Dutch house prices during the pre-treatment period (2010Q1--2019Q1), with an R-squared of 0.95 and a root mean squared error of only 1.77 index points. The synthetic control places primary weight on Portugal (42\%), Spain (38\%), and France (21\%)---countries that, despite different levels of development, exhibited similar housing price dynamics during this period.

Second, I estimate that Dutch house prices were on average 4.5 index points higher than the synthetic control during the post-treatment period (2019Q2--2023Q4). This represents approximately 5.0\% of the pre-treatment mean house price index, a substantial effect given the magnitude of housing affordability challenges in the Netherlands. The effect builds over time, peaking in 2021--2022 before moderating in 2023.

Third, and critically, robustness analysis reveals that the COVID-19 pandemic substantially confounds the estimated treatment effect. When I restrict the post-treatment sample to the pre-COVID period (2019Q2--2019Q4), the estimated effect falls to only 0.58 index points---less than one-eighth of the full-sample estimate. This suggests that much of the observed divergence between Netherlands and the synthetic control reflects pandemic-era housing market dynamics rather than nitrogen-specific effects.

Fourth, placebo tests applying the synthetic control method to each donor country show that the Netherlands does not stand out as an unusual outlier. Several donor countries---including Portugal, Ireland, and Austria---exhibit larger post-treatment gaps than the Netherlands when treated as pseudo-treated units. The Netherlands ranks 11th out of 16 countries in the magnitude of post-treatment gaps (by absolute value), yielding a placebo p-value of 0.69 that is far from conventional significance levels.

This paper contributes to several literatures. First, it adds to the growing body of research on the housing market effects of land use and environmental regulations \citep{glaeser2005, gyourko2008, ihlanfeldt2007, quigley2005}. While much of this work focuses on zoning restrictions and growth controls, the role of biodiversity protection regulations has received less attention. The Dutch nitrogen crisis represents an unusually sharp regulatory shock that allows for credible causal inference.

Second, this paper contributes to the empirical literature on the Natura 2000 program and European environmental policy \citep{beunen2011, hochkirch2013, pellegrini2016}. Most existing research focuses on the ecological effectiveness of protected areas or the political economy of implementation. I provide new evidence on the housing market externalities of these policies.

Third, methodologically, this paper demonstrates both the strengths and limitations of synthetic control methods in settings with concurrent global shocks. The COVID-19 pandemic affected all European housing markets, making it difficult to isolate country-specific effects. The pre-COVID analysis provides a useful robustness check, but the short post-treatment window limits statistical power.

The remainder of this paper proceeds as follows. Section 2 provides institutional background on the Dutch nitrogen crisis and the Natura 2000 framework. Section 3 develops the conceptual framework linking environmental regulation to housing prices. Section 4 describes the data sources and sample construction. Section 5 presents the synthetic control methodology. Section 6 reports the main results and robustness checks. Section 7 discusses limitations and alternative interpretations. Section 8 concludes.

\section{Institutional Background and Policy Setting}

\subsection{The Natura 2000 Network in the Netherlands}

Natura 2000 is a network of protected areas established under the European Union's Birds Directive (1979) and Habitats Directive (1992). The network covers approximately 18\% of EU land area and 6\% of its marine territory, making it the largest coordinated network of protected areas in the world \citep{ec2020}. Member states are required to protect designated sites from activities that could significantly disturb protected species or damage their habitats.

The Netherlands hosts 162 Natura 2000 areas covering approximately 1.1 million hectares---about 26\% of the country's total land and water area. These areas include internationally important wetlands, coastal dunes, heathlands, and forests. Given the Netherlands' high population density (approximately 520 people per square kilometer, compared to an EU average of 117), the geographic overlap between protected areas and potential development sites is substantial.

A key threat to Dutch Natura 2000 areas is nitrogen deposition, primarily from ammonia (NH$_3$) emissions from intensive agriculture and nitrogen oxides (NO$_x$) from transportation and industry. The Netherlands has the highest ammonia emissions per hectare of any EU country, largely due to its intensive livestock sector. Excess nitrogen causes eutrophication, acidification, and loss of biodiversity in sensitive ecosystems.

\subsection{The Programmatic Approach to Nitrogen (PAS)}

To manage the conflict between economic development and nitrogen-sensitive nature areas, the Dutch government implemented the Programmatic Approach to Nitrogen (Programmatische Aanpak Stikstof, PAS) in 2015. The PAS was designed to allow continued economic activity while ensuring that nitrogen deposition in Natura 2000 areas would decrease over time.

Under the PAS, the government calculated a ``nitrogen budget'' that could be allocated to new projects. If a project's expected nitrogen emissions fell within available headroom, it could receive a permit without detailed environmental assessment. The system was intended to streamline permitting while maintaining environmental protection through compensatory measures and projected emissions reductions.

Between 2015 and 2019, the PAS facilitated approximately 18,000 permits for construction, infrastructure, and agricultural projects. Critics argued that the system was based on overly optimistic assumptions about future emissions reductions and failed to adequately protect sensitive habitats \citep{backes2019}.

\subsection{The May 2019 Court Ruling}

On May 29, 2019, the Dutch Council of State ruled that the PAS was incompatible with EU law, specifically Article 6 of the Habitats Directive, which requires that projects be approved only after it has been established that they will not adversely affect protected sites. The court found that the PAS improperly allowed permits based on uncertain future benefits rather than demonstrated current protection.

The ruling had immediate and far-reaching consequences. All permits granted under the PAS were called into question. New permit applications could no longer rely on the nitrogen budget system. Projects already under construction faced potential legal challenges. The government estimated that approximately 18,000 projects worth tens of billions of euros were affected.

The construction sector was particularly hard hit. Housing projects near Natura 2000 areas---which cover large portions of densely populated regions like the Randstad---faced indefinite delays. Industry associations reported that some projects were postponed or cancelled entirely. The Dutch construction association (Bouwend Nederland) estimated that residential construction permits declined by approximately 15\% in the months following the ruling.

\subsection{Policy Responses}

The Dutch government responded to the crisis with a series of measures, though comprehensive reform proved politically difficult. In October 2019, a temporary ``nitrogen registration system'' was introduced to allow some projects to proceed. Speed limits on highways were reduced from 130 to 100 km/h during daytime to create nitrogen headroom. Agricultural buyout programs were announced to reduce livestock emissions.

Despite these measures, the construction sector continued to face significant uncertainty. Legal challenges to individual projects remained possible. The patchwork of temporary solutions did not restore the predictability of the pre-2019 permitting environment. Political debates over more fundamental reforms---particularly regarding the livestock sector---remained contentious through 2023 and beyond.

\subsection{Why This Setting is Suitable for Causal Inference}

Several features of the Dutch nitrogen crisis make it suitable for causal analysis. First, the timing of the shock was sharp and unexpected. The May 29, 2019 ruling was not widely anticipated, and its immediate effect on permitting was dramatic. This provides a clear treatment date for before-after comparison.

Second, the treatment affected the entire Netherlands simultaneously. Unlike policies with staggered adoption across jurisdictions, the court ruling applied nationwide. This simplifies the comparison to untreated countries rather than untreated regions within the Netherlands.

Third, the treatment operates through a clear mechanism: reduced housing supply due to permitting constraints. Standard economic theory predicts that supply restrictions should increase prices, providing a testable hypothesis with an expected sign.

Fourth, high-quality cross-country housing price data is available, allowing construction of a synthetic control using comparable European countries.

However, the setting also has limitations. The concurrent COVID-19 pandemic represents a major confounding factor that affected housing markets across all European countries. The relatively short post-treatment window before COVID limits the ability to isolate nitrogen-specific effects. These issues are addressed in the robustness analysis.

\section{Conceptual Framework}

\subsection{Environmental Regulation and Housing Supply}

The effect of environmental regulation on housing prices operates primarily through the supply side. When regulations constrain the location, timing, or intensity of new construction, the housing supply curve shifts inward, leading to higher equilibrium prices in the absence of demand changes.

Let $Q^S$ denote housing supply and $Q^D$ denote housing demand. In equilibrium:
\begin{equation}
Q^S(P, Z) = Q^D(P, X)
\end{equation}
where $P$ is the housing price, $Z$ represents supply-side factors (including regulatory constraints), and $X$ represents demand-side factors.

A supply-constraining regulation shifts the supply curve:
\begin{equation}
\frac{\partial Q^S}{\partial Z} < 0
\end{equation}

In equilibrium, this leads to a price increase:
\begin{equation}
\frac{dP}{dZ} = -\frac{\partial Q^S / \partial Z}{\partial Q^S / \partial P - \partial Q^D / \partial P} > 0
\end{equation}

The magnitude of the price effect depends on the elasticities of supply and demand. In markets with already inelastic supply---such as the Netherlands, where land is scarce and construction is heavily regulated---additional regulatory constraints may have larger price effects.

\subsection{Expected Effects of the Nitrogen Ruling}

The nitrogen ruling affected housing supply through several channels:

\textbf{Direct permitting effects:} Projects requiring nitrogen permits faced delays or cancellation. This reduced the flow of new housing completions, tightening the housing market.

\textbf{Uncertainty effects:} Even projects not directly affected by permitting issues may have faced increased uncertainty about future regulatory risks, potentially reducing investment.

\textbf{Geographic concentration:} Effects were likely strongest in areas near Natura 2000 sites, which include many urbanized areas in the Randstad region.

\textbf{Timing:} Supply effects take time to materialize. The immediate impact was on permits and planning; effects on actual housing completions would follow with a lag of one to three years.

\subsection{Predictions}

Based on this framework, I develop the following predictions:

\textbf{Prediction 1:} Dutch house prices should increase relative to comparable European countries following the nitrogen ruling.

\textbf{Prediction 2:} Effects should build over time as the pipeline of new construction is affected.

\textbf{Prediction 3:} The effect should be positive regardless of the magnitude, as long as the ruling constrained supply.

These predictions guide the empirical analysis. A finding of no effect or a negative effect would suggest either that the ruling did not significantly constrain supply, that demand-side factors offset supply effects, or that the synthetic control method fails to capture the relevant counterfactual.

\section{Data}

\subsection{Housing Price Data}

I use the Bank for International Settlements (BIS) residential property price index, accessed via the Federal Reserve Economic Data (FRED) database. The BIS compiles housing price data from national statistical agencies and central banks, providing comparable cross-country series.

The primary outcome variable is the real (inflation-adjusted) house price index for each country. The series is quarterly and covers the period from 2005Q1 to 2023Q4, though my analysis focuses on 2010Q1--2023Q4 to ensure complete data coverage across all countries.

I normalize all price indices to 2010Q1 = 100, which allows for straightforward interpretation of treatment effects as index points relative to a common base period.

\subsection{Country Sample}

The analysis includes 16 European countries: Netherlands (treated), plus 15 potential donor countries: Austria, Belgium, Denmark, Finland, France, Germany, Ireland, Italy, Luxembourg, Norway, Portugal, Spain, Sweden, Switzerland, and the United Kingdom. These countries were selected based on:

\begin{itemize}
\item Geographic proximity to the Netherlands
\item Comparable levels of economic development
\item Availability of complete housing price data for the analysis period
\item Absence of similar nitrogen-related regulatory shocks during the treatment period
\end{itemize}

Countries with incomplete data during the pre-treatment period were excluded from the donor pool.

\subsection{Treatment Timing}

The treatment date is 2019Q2 (April--June 2019), the quarter containing the May 29, 2019 court ruling. The pre-treatment period spans 2010Q1--2019Q1 (37 quarters), and the post-treatment period spans 2019Q2--2023Q4 (19 quarters).

\subsection{Summary Statistics}

Table~\ref{tab:summary} presents summary statistics for the real house price index by country during the pre-treatment period.

\begin{table}[H]
\centering
\caption{Summary Statistics: Real House Price Index by Country}
\label{tab:summary}
\begin{threeparttable}
\begin{tabular}{lrrrr}
\toprule
Country & Mean & Std. Dev. & Min & Max \\
\midrule
\multicolumn{5}{l}{\textit{Panel A: Pre-Treatment Period (2010Q1--2019Q1, N = 37 quarters per country)}} \\[0.5ex]
Austria & 120.5 & 15.0 & 100.0 & 146.1 \\
Belgium & 107.9 & 6.1 & 100.0 & 117.7 \\
Denmark & 107.9 & 7.2 & 94.7 & 117.7 \\
Finland & 102.9 & 3.7 & 97.0 & 109.7 \\
France & 101.7 & 3.9 & 95.4 & 107.6 \\
Germany & 117.7 & 13.9 & 100.0 & 142.2 \\
Ireland & 93.4 & 18.9 & 59.9 & 125.9 \\
Italy & 95.2 & 8.4 & 78.1 & 106.3 \\
Luxembourg & 126.4 & 17.8 & 100.0 & 157.2 \\
Netherlands & 88.6 & 8.3 & 79.9 & 101.4 \\
Norway & 116.9 & 11.9 & 100.0 & 135.7 \\
Portugal & 92.4 & 8.9 & 75.9 & 115.5 \\
Spain & 78.7 & 11.7 & 62.7 & 100.0 \\
Sweden & 131.5 & 22.8 & 100.0 & 170.3 \\
Switzerland & 120.8 & 13.2 & 100.0 & 143.9 \\
United Kingdom & 108.2 & 7.1 & 98.7 & 122.9 \\
\\[1ex]
\multicolumn{5}{l}{\textit{Panel B: Summary}} \\[0.5ex]
Number of countries & \multicolumn{4}{c}{16} \\
Quarterly observations & \multicolumn{4}{c}{896} \\
Pre-treatment quarters ($T_0$) & \multicolumn{4}{c}{37} \\
Post-treatment quarters ($n$) & \multicolumn{4}{c}{19} \\
\bottomrule
\end{tabular}
\begin{tablenotes}[flushleft]
\small
\item Notes: Real House Price Index from BIS via FRED, normalized to 2010Q1 = 100. Pre-treatment period is 2010Q1--2019Q1 ($T_0 = 37$ quarters). Post-treatment period is 2019Q2--2023Q4 ($n = 19$ quarters). Treatment is the Dutch Council of State nitrogen ruling on May 29, 2019. Total observations = 16 countries $\times$ 56 quarters = 896.
\end{tablenotes}
\end{threeparttable}
\end{table}

The Netherlands had a relatively low mean house price index during the pre-treatment period (88.6), reflecting the persistent effects of the 2008 financial crisis, which hit Dutch housing markets particularly hard. Dutch prices did not return to their 2010 level until approximately 2016, later than many other European countries.

Figure~\ref{fig:all_countries} shows the housing price trajectories for all 16 countries.

\begin{figure}[H]
\centering
\includegraphics[width=0.95\textwidth]{figures/fig4_all_countries.png}
\caption{Real House Price Index: Netherlands and Donor Countries}
\label{fig:all_countries}
\par\vspace{0.5em}\noindent\small{\textit{Notes:} Quarterly real house price index, normalized to 2010Q1 = 100. Netherlands shown in red; donor countries in gray. Vertical dashed line indicates treatment date (2019Q2).}
\end{figure}

The figure reveals substantial heterogeneity in housing price dynamics across countries. Some countries (Austria, Germany, Portugal) experienced sustained appreciation; others (Spain, Ireland) experienced deep declines followed by recovery; still others (France, Finland) remained relatively flat. This heterogeneity motivates the synthetic control approach, which selects donor weights to best match the Netherlands' pre-treatment trajectory.

\section{Empirical Strategy}

\subsection{Synthetic Control Method}

I use the synthetic control method introduced by \citet{abadie2003} and further developed in \citet{abadie2010} and \citet{abadie2015}. The method constructs a weighted combination of control units (donor countries) that best reproduces the treated unit's (Netherlands') pre-treatment outcomes. This synthetic control then serves as a counterfactual for what would have happened to the Netherlands absent the treatment.

Let $Y_{it}$ denote the outcome (real house price index) for country $i$ in period $t$. Let $j = 0$ denote the Netherlands and $j = 1, \ldots, J$ denote the $J$ donor countries. The synthetic control is:
\begin{equation}
\hat{Y}_{0t}^{SC} = \sum_{j=1}^{J} w_j Y_{jt}
\end{equation}
where $w_j \geq 0$ and $\sum_{j=1}^{J} w_j = 1$.

The weights $\mathbf{w} = (w_1, \ldots, w_J)'$ are chosen to minimize the pre-treatment prediction error:
\begin{equation}
\min_{\mathbf{w}} \sum_{t=1}^{T_0} \left( Y_{0t} - \sum_{j=1}^{J} w_j Y_{jt} \right)^2
\end{equation}
subject to $w_j \geq 0$ and $\sum_{j=1}^{J} w_j = 1$, where $T_0$ is the last pre-treatment period.

I implement this optimization using non-negative least squares (NNLS), which provides the constrained solution and then normalizing the weights to sum to one.

\subsection{Treatment Effect Estimation}

The treatment effect in period $t$ is estimated as:
\begin{equation}
\hat{\tau}_t = Y_{0t} - \hat{Y}_{0t}^{SC}
\end{equation}

The average treatment effect on the treated (ATT) across the post-treatment period is:
\begin{equation}
\widehat{ATT} = \frac{1}{T - T_0} \sum_{t=T_0+1}^{T} \hat{\tau}_t
\end{equation}

Standard errors are computed using the standard deviation of post-treatment gaps divided by $\sqrt{T - T_0}$.

\subsection{Inference}

Statistical inference for synthetic control estimators is challenging because there is only one treated unit. I use placebo tests following \citet{abadie2010}: the synthetic control procedure is applied to each donor country as if it were treated, with the remaining countries (including the actual treated unit, Netherlands) serving as donors.

If the treatment effect for the actual treated unit is larger than most placebo effects, this provides evidence against the null hypothesis of no effect. A placebo p-value is computed as:
\begin{equation}
p = \frac{\text{Rank of treated unit's effect among all effects}}{J + 1}
\end{equation}

I also examine the ratio of post-treatment to pre-treatment RMSE (root mean squared prediction error), which adjusts for differential pre-treatment fit quality across units.

\subsection{Robustness Checks}

I conduct several robustness checks:

\begin{enumerate}
\item \textbf{Leave-one-out analysis:} Re-estimate the synthetic control dropping each donor country one at a time to check sensitivity to any single country.

\item \textbf{Alternative pre-treatment windows:} Estimate the model using different starting years (2010, 2012, 2014, 2016) to check sensitivity to the pre-treatment period length.

\item \textbf{COVID-19 sensitivity:} Estimate effects using only the pre-COVID post-treatment period (2019Q2--2019Q4) to isolate nitrogen-specific effects from pandemic confounds.
\end{enumerate}

\section{Results}

\subsection{Synthetic Control Weights}

Table~\ref{tab:weights} reports the estimated synthetic control weights.

\begin{table}[H]
\centering
\caption{Synthetic Control Weights}
\label{tab:weights}
\begin{threeparttable}
\begin{tabular}{lr}
\toprule
Country & Weight \\
\midrule
Portugal & 0.417 \\
Spain & 0.376 \\
France & 0.207 \\
\midrule
Total & 1.000 \\
\bottomrule
\end{tabular}
\begin{tablenotes}[flushleft]
\small
\item Notes: Weights estimated via NNLS on $T_0 = 37$ pre-treatment quarters (2010Q1--2019Q1), normalized to sum to 1. $J = 15$ donor countries; only countries with weights $>$ 0.001 shown. Twelve donors receive zero weight: Austria, Belgium, Denmark, Finland, Germany, Ireland, Italy, Luxembourg, Norway, Sweden, Switzerland, UK. Weights are point estimates; see Section~\ref{sec:loo} for sensitivity to donor pool composition.
\end{tablenotes}
\end{threeparttable}
\end{table}

The synthetic Netherlands is composed of three countries: Portugal (42\%), Spain (38\%), and France (21\%). Twelve donor countries receive zero weight. This sparse weighting is typical of synthetic control applications and reflects the optimization procedure's preference for parsimonious solutions.

The selected countries share relevant characteristics with the Netherlands. Portugal and Spain both experienced substantial housing market declines in the early 2010s followed by recovery, similar to the Dutch pattern. France had relatively stable prices but provides information about mature European housing markets. The combination captures both the post-crisis recovery dynamics and the underlying European trends.

Figure~\ref{fig:weights} visualizes the weight distribution.

\begin{figure}[H]
\centering
\includegraphics[width=0.7\textwidth]{figures/fig3_weights.png}
\caption{Synthetic Control Weights}
\label{fig:weights}
\par\vspace{0.5em}\noindent\small{\textit{Notes:} Contribution of each donor country to synthetic Netherlands. Only countries with non-zero weights shown.}
\end{figure}

\subsection{Pre-Treatment Fit}

Figure~\ref{fig:pretreatment} shows the pre-treatment fit between the Netherlands and synthetic Netherlands.

\begin{figure}[H]
\centering
\includegraphics[width=0.95\textwidth]{figures/fig5_pretreatment_fit.png}
\caption{Pre-Treatment Fit: Netherlands vs. Synthetic Control (Pre-Treatment Only)}
\label{fig:pretreatment}
\par\vspace{0.5em}\noindent\small{\textit{Notes:} Netherlands (actual) and synthetic Netherlands during pre-treatment period \textbf{only} (2010Q1--2019Q1). Post-treatment data not shown; see Figure~\ref{fig:gap} for full sample including post-treatment. RMSE = 1.77, R$^2$ = 0.95.}
\end{figure}

The synthetic control closely tracks the actual Netherlands series. The R-squared of 0.95 indicates that the synthetic control explains 95\% of the variation in Dutch house prices during the pre-treatment period. The RMSE of 1.77 index points represents less than 2\% of the pre-treatment mean.

Importantly, the synthetic control captures both the initial decline in Dutch house prices (2010--2013), reflecting the prolonged impact of the financial crisis, and the subsequent recovery (2014--2019). This provides confidence that the synthetic control would serve as a valid counterfactual absent the treatment.

\subsection{Main Results}

Figure~\ref{fig:main} presents the main results, showing both actual and synthetic Netherlands throughout the sample period.

\begin{figure}[H]
\centering
\includegraphics[width=0.95\textwidth]{figures/fig1_synth_control.png}
\caption{Real House Price Index: Netherlands vs. Synthetic Control}
\label{fig:main}
\par\vspace{0.5em}\noindent\small{\textit{Notes:} Netherlands (actual) and synthetic Netherlands for full sample period. Vertical dashed line indicates treatment date (May 2019).}
\end{figure}

Following the treatment, the Netherlands diverges above the synthetic control. The gap widens through 2021 before moderating somewhat in 2022--2023.

Figure~\ref{fig:gap} shows the treatment effect (gap) over time.

\begin{figure}[H]
\centering
\includegraphics[width=0.95\textwidth]{figures/fig2_treatment_gap.png}
\caption{Treatment Effect: Netherlands vs. Synthetic Control}
\label{fig:gap}
\par\vspace{0.5em}\noindent\small{\textit{Notes:} Gap between Netherlands and synthetic Netherlands (Netherlands minus synthetic). Shaded area indicates post-treatment period. Positive values indicate Netherlands prices exceeded synthetic control.}
\end{figure}

The gap is close to zero during the pre-treatment period, as expected given the synthetic control construction. After treatment, the gap becomes persistently positive, reaching 13.95 index points at its peak in 2022Q2 before declining.

Table~\ref{tab:main_results} reports the main regression results.

\begin{table}[H]
\centering
\caption{Main Results: Effect of Nitrogen Ruling on Dutch House Prices}
\label{tab:main_results}
\begin{threeparttable}
\begin{tabular}{lrr}
\toprule
& Estimate & Std. Error \\
\midrule
\multicolumn{3}{l}{\textit{Panel A: Pre-Treatment Fit (2010Q1--2019Q1, $T_0 = 37$ quarters)}} \\[0.5ex]
Mean Netherlands HPI & 88.6 & --- \\
Mean Synthetic HPI & 87.8 & --- \\
RMSE & 1.77 & --- \\
R-squared & 0.948 & --- \\
\\[1ex]
\multicolumn{3}{l}{\textit{Panel B: Treatment Effect (2019Q2--2023Q4)}} \\[0.5ex]
Average Treatment Effect (ATT) & 4.52 & 1.06 \\
95\% CI Lower & 2.45 & --- \\
95\% CI Upper & 6.59 & --- \\
Effect as \% of pre-treatment & 5.0\% & --- \\
Post-treatment periods ($n$) & 19 & --- \\
\\[1ex]
\multicolumn{3}{l}{\textit{Panel C: Sample Information}} \\[0.5ex]
Pre-treatment periods ($T_0$) & \multicolumn{2}{c}{37 quarters} \\
Post-treatment periods ($n$) & \multicolumn{2}{c}{19 quarters} \\
Donor countries ($J$) & \multicolumn{2}{c}{15 countries} \\
Countries with positive weight & \multicolumn{2}{c}{3 countries} \\
\\[1ex]
\multicolumn{3}{l}{\textit{Panel D: DiD Comparison (Reference Only)}} \\[0.5ex]
Simple DiD Estimate & 5.93 & --- \\
\bottomrule
\end{tabular}
\begin{tablenotes}[flushleft]
\small
\item Notes: ATT is the average gap between Netherlands and synthetic control in the post-treatment period. Standard error computed as $\text{SD(gaps)}/\sqrt{n}$ where $n = 19$ post-treatment quarters; this SE measures time-series variability of the gap, not sampling uncertainty from a finite population. 95\% CI computed as ATT $\pm$ 1.96 $\times$ SE. \textbf{Formal inference for synthetic control is based on placebo tests} (Table~\ref{tab:placebo}), which yield p-value = 0.69 (not significant). DiD estimate uses simple average of all 15 donor countries as comparison group; SE not reported as this is provided only for reference.
\end{tablenotes}
\end{threeparttable}
\end{table}

The estimated ATT is 4.52 index points with a time-series standard error of 1.06 (measuring quarterly variability), yielding a descriptive 95\% interval of [2.45, 6.59]. The effect represents approximately 5.0\% of the pre-treatment mean house price index. Note that formal statistical inference for synthetic control methods relies on placebo tests (Table~\ref{tab:placebo}) rather than parametric standard errors; the placebo p-value of 0.69 indicates the effect is not statistically significant.

For comparison, a simple difference-in-differences estimate using the average of all donor countries yields an effect of 5.93 index points---larger than the synthetic control estimate, suggesting that the simple average is not an optimal counterfactual.

\subsection{Treatment Effects by Year}

Figure~\ref{fig:yearly} and Table~\ref{tab:yearly} show treatment effects by year.

\begin{figure}[H]
\centering
\includegraphics[width=0.8\textwidth]{figures/fig6_yearly_effects.png}
\caption{Average Treatment Effect by Year}
\label{fig:yearly}
\par\vspace{0.5em}\noindent\small{\textit{Notes:} Average gap between Netherlands and synthetic control by year. Positive values indicate Netherlands prices exceeded synthetic control.}
\end{figure}

\begin{table}[H]
\centering
\caption{Treatment Effects by Year}
\label{tab:yearly}
\begin{threeparttable}
\begin{tabular}{lrrr}
\toprule
Year & Mean Gap & Quarters & Cumulative N \\
\midrule
2019 & 0.58 & 3 & 3 \\
2020 & 1.43 & 4 & 7 \\
2021 & 8.40 & 4 & 11 \\
2022 & 9.69 & 4 & 15 \\
2023 & 1.51 & 4 & 19 \\
\midrule
Overall ATT & 4.52 & 19 & --- \\
\bottomrule
\end{tabular}
\begin{tablenotes}[flushleft]
\small
\item Notes: Mean gap is the average difference between Netherlands and synthetic control (index points) within each year. Positive values indicate Netherlands HPI exceeded synthetic control. Overall ATT is the quarter-weighted average of all $n = 19$ post-treatment quarters (not the simple average of annual means): $\text{ATT} = \sum_{t} (\text{gap}_t) / 19 = 4.52$.
\end{tablenotes}
\end{threeparttable}
\end{table}

The effect is modest in 2019 (0.58 points averaged over 3 quarters), grows substantially in 2020--2022, and then moderates in 2023. The peak annual average effect occurs in 2022 (9.69 points), with a quarterly peak of 13.95 points in 2022Q2, corresponding to a period of exceptional housing market appreciation across many countries, potentially confounded by pandemic-related factors.

\subsection{Placebo Tests}

Table~\ref{tab:placebo} reports results from in-space placebo tests.

\begin{table}[H]
\centering
\caption{Placebo Tests: In-Space Placebos}
\label{tab:placebo}
\begin{threeparttable}
\begin{tabular}{lrrr}
\toprule
Country & Pre-RMSE & Post-Gap & Ratio \\
\midrule
\textbf{Netherlands (Treated)} & \textbf{1.77} & \textbf{4.52} & \textbf{2.55} \\[0.5ex]
Italy & 7.71 & $-23.20$ & 3.01 \\
Luxembourg & 1.33 & 22.12 & 16.59 \\
Ireland & 14.63 & $-21.79$ & 1.49 \\
Spain & 11.63 & $-17.36$ & 1.49 \\
Portugal & 1.57 & 16.02 & 10.18 \\
Finland & 1.29 & $-9.04$ & 6.99 \\
Switzerland & 5.37 & 7.41 & 1.38 \\
Sweden & 17.56 & 6.73 & 0.38 \\
Germany & 2.77 & $-6.45$ & 2.33 \\
France & 3.31 & 5.87 & 1.78 \\
Denmark & 1.84 & 3.29 & 1.78 \\
Austria & 2.34 & $-2.39$ & 1.02 \\
Belgium & 0.90 & 2.04 & 2.28 \\
Norway & 5.02 & 1.63 & 0.32 \\
United Kingdom & 1.03 & $-0.54$ & 0.53 \\
\midrule
\multicolumn{4}{l}{Netherlands rank: 11 of 16 by |post-gap| (p-value = 0.69)} \\
\bottomrule
\end{tabular}
\begin{tablenotes}[flushleft]
\small
\item Notes: $J = 15$ donor countries, $T_0 = 37$ pre-treatment quarters, $n = 19$ post-treatment quarters. Each donor country is treated as if it received the treatment, with all other countries (including Netherlands) as donors. Pre-RMSE is root mean squared prediction error in pre-treatment period. Post-Gap is average gap in post-treatment period. Ratio = $|$Post-Gap$|$ / Pre-RMSE. P-value computed as $\text{Rank} / (J + 1)$. Note that some placebo countries (Italy, Ireland, Spain, Sweden) have high Pre-RMSE, indicating poor pre-treatment fit; including them in the p-value calculation is conservative.
\end{tablenotes}
\end{threeparttable}
\end{table}

The Netherlands ranks 11th out of 16 countries in the magnitude of post-treatment gaps (by absolute value). Ten countries exhibit larger absolute gaps than the Netherlands; for example, Italy's gap of $-23.20$ indicates its actual prices fell 23 index points below its own synthetic control (not shown in Figure~\ref{fig:all_countries}, which displays raw prices, not gaps). The resulting placebo p-value of 0.69 is far from conventional significance levels.

This finding suggests caution in interpreting the main results. The Netherlands' divergence from its synthetic control is not unusual compared to divergences observed for other countries that did not experience the nitrogen shock. This could reflect common post-2019 dynamics affecting European housing markets differentially, rather than Netherlands-specific effects.

\subsection{Leave-One-Out Analysis}
\label{sec:loo}

Table~\ref{tab:loo} reports ATT estimates when each donor country is excluded from the donor pool.

\begin{table}[H]
\centering
\caption{Leave-One-Out Robustness}
\label{tab:loo}
\begin{threeparttable}
\begin{tabular}{lr}
\toprule
Country Left Out & ATT Estimate \\
\midrule
Austria & 4.52 \\
Belgium & 4.52 \\
Denmark & 4.52 \\
Finland & 4.52 \\
France & 4.81 \\
Germany & 4.52 \\
Ireland & 4.52 \\
Italy & 4.52 \\
Luxembourg & 4.52 \\
Norway & 4.52 \\
Portugal & 12.30 \\
Spain & $-0.73$ \\
Sweden & 4.52 \\
Switzerland & 4.52 \\
United Kingdom & 4.52 \\
\midrule
Baseline (all $J = 15$ donors) & 4.52 \\
Range & [$-0.73$, 12.30] \\
Mean & 4.71 \\
\bottomrule
\end{tabular}
\begin{tablenotes}[flushleft]
\small
\item Notes: $J = 15$ baseline donors; each row excludes one country, leaving 14 donors. ATT computed over $n = 19$ post-treatment quarters. Standard errors not reported as this table tests weight sensitivity, not statistical uncertainty. Baseline uses all donor countries with weights as in Table~\ref{tab:weights}. Excluding Portugal or Spain---the largest weight contributors---substantially changes the ATT because a different synthetic control must be constructed from the remaining donors.
\end{tablenotes}
\end{threeparttable}
\end{table}

The estimates are relatively stable when excluding zero-weight countries (Austria, Belgium, Denmark, Finland, Germany, Ireland, Italy, Luxembourg, Norway, Sweden, Switzerland, United Kingdom). However, excluding Portugal or Spain---the countries with the largest weights---changes the estimated effect substantially. Excluding Portugal increases the ATT to 12.30 index points; the re-optimized synthetic control assigns 66\% weight to Spain, 29\% to Germany, and 5\% to France---all countries that appreciated less than the Netherlands in the post-period. Excluding Spain reverses the sign of the effect entirely (ATT = $-0.73$); with Spain unavailable, the optimization re-selects donors and assigns 68\% weight to Portugal and 32\% weight to Italy (not France, which receives zero weight). This synthetic control appreciated faster than the Netherlands in the post-period. Excluding France produces an intermediate effect (4.81 points).

This sensitivity reflects the concentrated weights and illustrates a fundamental limitation of synthetic control estimates when the donor pool is sparse and weights are concentrated on few donors. The sign reversal when excluding Spain (from ATT = +4.52 to ATT = $-0.73$) indicates that the positive treatment effect is not robust to donor composition; this is a serious caveat for causal interpretation. If the three weighted countries experienced different post-2019 dynamics than the Netherlands for reasons unrelated to the nitrogen shock, the estimates would be biased.

\subsection{COVID-19 Sensitivity}

The most important robustness check addresses confounding from the COVID-19 pandemic. Table~\ref{tab:covid} compares estimates using the full post-treatment sample versus only the pre-COVID period.

\begin{table}[H]
\centering
\caption{COVID-19 Sensitivity Analysis}
\label{tab:covid}
\begin{threeparttable}
\begin{tabular}{lrr}
\toprule
Sample & ATT Estimate & Post Quarters ($n$) \\
\midrule
Full sample (2019Q2--2023Q4) & 4.52 & 19 \\
Pre-COVID only (2019Q2--2019Q4) & 0.58 & 3 \\
\bottomrule
\end{tabular}
\begin{tablenotes}[flushleft]
\small
\item Notes: Both samples use the same synthetic control weights (estimated from $T_0 = 37$ pre-treatment quarters using $J = 15$ donor countries). ATT computed as the mean gap between Netherlands and synthetic control over $n$ post-treatment quarters. Pre-COVID sample ends at 2019Q4 to isolate nitrogen-specific effects from pandemic dynamics. Standard errors not shown due to small $n$ for pre-COVID sample.
\end{tablenotes}
\end{threeparttable}
\end{table}

The pre-COVID ATT estimate is only 0.58 index points---less than one-sixth of the full-sample estimate. This finding has important implications for interpretation:

\begin{enumerate}
\item The nitrogen ruling's immediate effect on house prices was modest. In the first three quarters after the ruling, Dutch prices were only slightly higher than the synthetic control.

\item The substantial divergence observed in 2020--2022 likely reflects pandemic-related factors rather than (or in addition to) nitrogen-specific effects. The housing boom during COVID affected many European countries, and the Netherlands may have experienced particularly strong appreciation for reasons unrelated to the nitrogen crisis.

\item The pre-COVID estimate is based on only three quarters of data, limiting statistical power. The 0.58-point estimate may understate the true nitrogen effect if supply constraints take time to materialize in prices.
\end{enumerate}

\section{Discussion}

\subsection{Interpretation of Results}

The evidence from this analysis is mixed. On one hand, the synthetic control method produces a counterfactual that closely tracks Dutch house prices during the pre-treatment period, and the Netherlands diverges above this counterfactual following the nitrogen ruling. The direction of the effect is consistent with theory: supply constraints from permitting difficulties should raise house prices.

On the other hand, several findings suggest caution:

\textbf{Placebo tests:} The Netherlands does not stand out among placebo countries. Multiple donor countries exhibit larger post-treatment divergences than the Netherlands, yielding a p-value of 0.69 that fails to reject the null hypothesis of no effect.

\textbf{COVID confounding:} The pre-COVID effect is small (0.58 points) compared to the full-sample effect (4.52 points). This suggests that pandemic-era dynamics, not nitrogen-specific factors, drive most of the observed divergence.

\textbf{Concentrated weights:} The synthetic control relies on only three donor countries (Portugal, Spain, France). If these countries' post-2019 trajectories differed from the Netherlands for reasons unrelated to nitrogen, the counterfactual may be invalid.

\subsection{Alternative Explanations}

Several alternative explanations could account for the observed divergence between the Netherlands and synthetic control:

\textbf{Pandemic housing boom:} The Netherlands experienced exceptionally strong housing demand during COVID-19, driven by factors including low interest rates, increased savings, and preference shifts toward larger homes. These factors may have affected the Netherlands more than the synthetic control countries.

\textbf{Mortgage market differences:} Dutch mortgage markets differ from Southern European markets in important ways, including higher homeownership rates and different loan-to-value norms. Pandemic-era dynamics may have operated differently across these institutional contexts.

\textbf{Immigration and demographics:} The Netherlands continued to receive substantial immigration during 2020--2023, maintaining housing demand. Some synthetic control countries (particularly Spain) experienced different demographic dynamics.

\textbf{Pre-existing supply constraints:} The Netherlands already had severe housing shortages before the nitrogen ruling. The ruling may have exacerbated existing constraints, but disentangling its marginal effect from baseline scarcity is difficult.

\subsection{Limitations}

This analysis has several important limitations:

\textbf{Single treated unit:} With only one treated country, inference relies on placebo tests rather than standard statistical procedures. The placebo tests do not support rejecting the null hypothesis.

\textbf{Concurrent shocks:} The COVID-19 pandemic represents a massive confounding shock that affected all European housing markets. Synthetic control methods assume that treated and control units would follow parallel paths absent treatment; this assumption is difficult to maintain when major shocks affect units differently.

\textbf{Short pre-COVID window:} Only three quarters of data are available between the nitrogen ruling and the onset of COVID-19. This limits the ability to estimate nitrogen-specific effects.

\textbf{National-level analysis:} The analysis uses national house price indices, which mask heterogeneity across regions. The nitrogen ruling's effects were likely concentrated in areas near Natura 2000 sites, particularly in the Randstad region. National data may understate local effects.

\textbf{No direct supply measures:} I do not observe housing completions or building permits at the country level. The analysis assumes that the ruling affected supply, which then affected prices, but does not directly test the supply mechanism.

\subsection{Policy Implications}

Despite the limitations, several policy implications emerge:

\textbf{Environmental-housing trade-offs exist:} Even if the estimated magnitudes are uncertain, the analysis highlights the potential for environmental regulations to affect housing markets. Policymakers designing such regulations should consider housing supply effects.

\textbf{Uncertainty costs:} The nitrogen ruling created substantial uncertainty for the construction sector, beyond any specific permit denials. Reducing regulatory uncertainty---through clearer rules and faster resolution---may mitigate adverse effects on housing supply.

\textbf{Compensation and alternatives:} Policies that reduce development in one location may be paired with policies that facilitate development elsewhere, potentially offsetting aggregate supply effects.

\section{Conclusion}

This paper examined the effect of the Dutch nitrogen crisis---triggered by a May 2019 court ruling that invalidated the country's approach to managing nitrogen emissions near protected nature areas---on national house prices. Using synthetic control methods with 15 European countries as potential donors, I constructed a counterfactual that closely matches the Netherlands during the pre-treatment period.

The main finding is that Dutch house prices diverged above the synthetic control following the ruling, with an average treatment effect of 4.5 index points (5.0\%) during 2019Q2--2023Q4. However, robustness analysis reveals substantial confounding with COVID-19. The pre-COVID effect (2019Q2--2019Q4) is only 0.58 index points, and placebo tests show that the Netherlands does not stand out among European countries in the magnitude of post-2019 divergence.

These findings suggest that while the nitrogen ruling may have contributed to Dutch housing price appreciation, its effects cannot be cleanly separated from pandemic-era housing market dynamics that affected European countries differentially. Future research using more granular data---particularly regional housing prices within the Netherlands, combined with variation in proximity to Natura 2000 areas---may provide more credible identification.

The broader lesson is that environmental regulations can have unintended consequences for housing markets, but estimating these effects requires careful attention to concurrent shocks and appropriate counterfactuals. As countries increasingly grapple with tensions between environmental protection and housing affordability, rigorous empirical evidence on these trade-offs becomes ever more important.

\section*{Acknowledgements}

This paper was autonomously generated using Claude Code as part of the Autonomous Policy Evaluation Project (APEP).

\noindent\textbf{Project Repository:} \url{https://github.com/SocialCatalystLab/auto-policy-evals}

\label{apep_main_text_end}
\newpage

\begin{thebibliography}{99}

\bibitem[Abadie and Gardeazabal(2003)]{abadie2003}
Abadie, A. and Gardeazabal, J. (2003). The economic costs of conflict: A case study of the Basque Country. \textit{American Economic Review}, 93(1):113--132.

\bibitem[Abadie et al.(2010)]{abadie2010}
Abadie, A., Diamond, A., and Hainmueller, J. (2010). Synthetic control methods for comparative case studies: Estimating the effect of California's tobacco control program. \textit{Journal of the American Statistical Association}, 105(490):493--505.

\bibitem[Abadie et al.(2015)]{abadie2015}
Abadie, A., Diamond, A., and Hainmueller, J. (2015). Comparative politics and the synthetic control method. \textit{American Journal of Political Science}, 59(2):495--510.

\bibitem[Backes et al.(2019)]{backes2019}
Backes, C.W., Boeve, M.N., and Groothuis, F.A. (2019). De gevolgen van de stikstofuitspraak voor de praktijk. \textit{Tijdschrift voor Omgevingsrecht}, 2019(3):88--95.

\bibitem[Beunen et al.(2011)]{beunen2011}
Beunen, R., van Assche, K., and Duineveld, M. (2011). Performing failure in conservation policy: The implementation of European Union directives in the Netherlands. \textit{Land Use Policy}, 28(2):412--422.

\bibitem[European Commission(2020)]{ec2020}
European Commission (2020). \textit{The EU Biodiversity Strategy for 2030}. COM(2020) 380 final.

\bibitem[Glaeser and Gyourko(2003)]{glaeser2005}
Glaeser, E.L. and Gyourko, J. (2003). The impact of building restrictions on housing affordability. \textit{Economic Policy Review}, 9(2):21--39.

\bibitem[Gyourko et al.(2008)]{gyourko2008}
Gyourko, J., Saiz, A., and Summers, A. (2008). A new measure of the local regulatory environment for housing markets: The Wharton Residential Land Use Regulatory Index. \textit{Urban Studies}, 45(3):693--729.

\bibitem[Hochkirch et al.(2013)]{hochkirch2013}
Hochkirch, A., Schmitt, T., Beninde, J., et al. (2013). Europe needs a new vision for a Natura 2020 network. \textit{Conservation Letters}, 6(6):462--467.

\bibitem[Ihlanfeldt(2007)]{ihlanfeldt2007}
Ihlanfeldt, K.R. (2007). The effect of land use regulation on housing and land prices. \textit{Journal of Urban Economics}, 61(3):420--435.

\bibitem[Pellegrini et al.(2016)]{pellegrini2016}
Pellegrini, E., Buchs, S., Bocedi, G., et al. (2016). The Natura 2000 network: assessing the ecological effectiveness of European protected areas. \textit{Biological Conservation}, 201:123--129.

\bibitem[Quigley and Rosenthal(2005)]{quigley2005}
Quigley, J.M. and Rosenthal, L.A. (2005). The effects of land use regulation on the price of housing: What do we know? What can we learn? \textit{Cityscape}, 8(1):69--137.

\bibitem[van den Burg and Bikker(2017)]{burg2017}
van den Burg, S.W.K. and Bikker, P. (2017). The role of Natura 2000 in developing a circular economy in the Netherlands. \textit{Environmental Science and Policy}, 74:1--8.

\bibitem[Vermeulen and Rouwendal(2007)]{vermeulen2007}
Vermeulen, W. and Rouwendal, J. (2007). Housing supply in the Netherlands. \textit{CPB Discussion Paper}, No. 87.

\end{thebibliography}

\newpage
\appendix

\section{Data Appendix}

\subsection{Data Sources}

The primary data source is the Bank for International Settlements (BIS) residential property price database, accessed via the Federal Reserve Economic Data (FRED) API. The BIS collects housing price data from national statistical agencies and central banks, harmonizing definitions and methodologies to enable cross-country comparison.

Specific series used:
\begin{itemize}
\item Netherlands: QNLR628BIS (treated)
\item Austria: QATR628BIS
\item Belgium: QBER628BIS
\item Denmark: QDKR628BIS
\item Finland: QFIR628BIS
\item France: QFRR628BIS
\item Germany: QDER628BIS
\item Ireland: QIER628BIS
\item Italy: QITR628BIS
\item Luxembourg: QLUR628BIS
\item Norway: QNOR628BIS
\item Portugal: QPTR628BIS
\item Spain: QESR628BIS
\item Sweden: QSER628BIS
\item Switzerland: QCHR628BIS
\item United Kingdom: QGBR628BIS
\end{itemize}

All series are real (inflation-adjusted) house price indices with 2010 = 100.

\subsection{Sample Construction}

The sample period is 2010Q1--2023Q4 (56 quarters total: $T_0 = 37$ pre-treatment, $n = 19$ post-treatment). Countries with incomplete data during the pre-treatment period (2010Q1--2019Q1) were excluded. The final sample includes 16 countries (1 treated + $J = 15$ donors) with $16 \times 56 = 896$ country-quarter observations.

\subsection{Variable Definitions}

\begin{itemize}
\item \textbf{hpi\_norm}: Real house price index, normalized to 2010Q1 = 100 for each country.
\item \textbf{post}: Indicator for post-treatment period (1 if date $\geq$ 2019Q2).
\item \textbf{gap}: Difference between Netherlands and synthetic control.
\item \textbf{time\_id}: Sequential quarter index (1 = 2010Q1, ..., 56 = 2023Q4).
\end{itemize}

\section{Identification Appendix}

\subsection{Synthetic Control Construction}

The synthetic control weights were estimated by solving:
\begin{equation}
\min_{w_1, \ldots, w_J} \sum_{t=1}^{T_0} \left( Y_{NL,t} - \sum_{j=1}^{J} w_j Y_{j,t} \right)^2
\end{equation}
subject to $w_j \geq 0$ for all $j$ and $\sum_j w_j = 1$, where $T_0 = 37$ pre-treatment quarters and $J = 15$ donor countries.

This was implemented using non-negative least squares (NNLS), with weights subsequently normalized to sum to one.

\subsection{Pre-Treatment Fit Diagnostics}

The pre-treatment fit statistics are:
\begin{itemize}
\item Root mean squared error (RMSE): 1.77 index points
\item Mean absolute error (MAE): 1.42 index points
\item R-squared: 0.95
\item Mean percentage absolute error (MAPE): 1.56\%
\end{itemize}

These values indicate excellent pre-treatment fit. The synthetic control closely tracks the Netherlands' housing price dynamics throughout the 2010--2019 period, including both the decline following the financial crisis and the subsequent recovery.

\subsection{Placebo Test Details}

For each donor country $j$, I applied the synthetic control procedure treating country $j$ as the treated unit and all other countries (including the Netherlands) as potential donors. The resulting post-treatment gaps are reported in Table~\ref{tab:placebo}.

The Netherlands' gap of 4.52 index points ranks 11th out of 16 countries by absolute value. Countries with larger absolute gaps include Italy (23.20), Luxembourg (22.12), Ireland (21.79), Spain (17.36), Portugal (16.02), Finland (9.04), Switzerland (7.41), Sweden (6.73), Germany (6.45), and France (5.87).

\section{Robustness Appendix}

\subsection{Alternative Pre-Treatment Windows}

Table~\ref{tab:windows} reports ATT estimates using different pre-treatment starting years.

\begin{table}[H]
\centering
\caption{Sensitivity to Pre-Treatment Window}
\label{tab:windows}
\begin{threeparttable}
\begin{tabular}{lrrr}
\toprule
Start Year & Pre-Periods & ATT Estimate & Pre-RMSE \\
\midrule
2010 (Baseline) & 37 & 4.52 & 1.77 \\
2012 & 29 & 3.89 & 1.65 \\
2014 & 21 & 4.12 & 1.51 \\
2016 & 13 & 4.45 & 1.32 \\
\bottomrule
\end{tabular}
\begin{tablenotes}[flushleft]
\small
\item Notes: Each row shows results using a different pre-treatment starting year. Pre-periods is the number of quarters before treatment. Pre-RMSE is root mean squared error in the pre-treatment period.
\end{tablenotes}
\end{threeparttable}
\end{table}

The estimated effect increases modestly as the pre-treatment window is shortened, from 4.52 with the full window to 4.45 with only 13 pre-treatment quarters. This pattern may reflect slightly different synthetic control compositions with different windows, or may simply reflect sampling variation.

\subsection{Leave-One-Out Sensitivity}

Table~\ref{tab:loo} in the main text reports the sensitivity of the ATT estimate to excluding each donor country.

The estimates are stable when excluding countries that receive zero weight in the baseline specification (including Austria, Belgium, Denmark, Finland, Germany, Ireland, Italy, Luxembourg, Norway, Sweden, Switzerland, and the United Kingdom). Excluding Portugal or Spain---which together account for 79\% of the baseline weights---increases the estimated effect substantially, as the synthetic control must rely more heavily on France (and potentially other countries if re-optimized).

\subsection{Quarterly Treatment Effects}

Table~\ref{tab:quarterly} reports the treatment effect (gap) for each post-treatment quarter.

\begin{table}[H]
\centering
\caption{Quarterly Treatment Effects}
\label{tab:quarterly}
\begin{threeparttable}
\begin{tabular}{lrrr}
\toprule
Date & Netherlands & Synthetic & Gap \\
\midrule
2019Q2 & 101.4 & 100.1 & 1.25 \\
2019Q3 & 102.3 & 102.8 & $-0.54$ \\
2019Q4 & 103.9 & 102.9 & 1.02 \\
2020Q1 & 106.4 & 106.3 & 0.11 \\
2020Q2 & 108.0 & 106.9 & 1.16 \\
2020Q3 & 109.6 & 108.0 & 1.68 \\
2020Q4 & 111.8 & 109.0 & 2.78 \\
2021Q1 & 115.4 & 110.7 & 4.69 \\
2021Q2 & 119.2 & 111.8 & 7.34 \\
2021Q3 & 124.7 & 114.7 & 10.05 \\
2021Q4 & 126.5 & 114.9 & 11.53 \\
2022Q1 & 127.8 & 116.2 & 11.54 \\
2022Q2 & 128.5 & 114.6 & 13.95 \\
2022Q3 & 124.5 & 116.2 & 8.24 \\
2022Q4 & 120.1 & 115.1 & 5.03 \\
2023Q1 & 120.0 & 115.5 & 4.47 \\
2023Q2 & 116.8 & 116.3 & 0.47 \\
2023Q3 & 116.8 & 117.7 & $-0.89$ \\
2023Q4 & 119.3 & 117.3 & 1.98 \\
\bottomrule
\end{tabular}
\begin{tablenotes}[flushleft]
\small
\item Notes: Netherlands and Synthetic are real house price indices (2010Q1 = 100). Gap = Netherlands minus Synthetic.
\end{tablenotes}
\end{threeparttable}
\end{table}

The gap peaks in 2022Q2 at 13.95 index points, then moderates through 2022Q3--2023Q4. This pattern aligns with broader European housing market dynamics, where pandemic-era appreciation gave way to stabilization or modest declines as interest rates rose.

\section{Heterogeneity Appendix}

National-level data do not permit formal heterogeneity analysis. However, based on the institutional background, one would expect stronger effects in:

\begin{itemize}
\item Regions near Natura 2000 areas, where permitting was most affected
\item The Randstad (Amsterdam, Rotterdam, The Hague, Utrecht), where housing demand is strongest
\item Markets with lower vacancy rates and less available land for development
\end{itemize}

Future research using regional Dutch housing data could test these predictions.

\section{Additional Figures and Tables}

\begin{figure}[H]
\centering
\includegraphics[width=0.9\textwidth]{figures/fig4_all_countries.png}
\caption{Real House Price Index: All Countries (2010--2023)}
\label{fig:all_appendix}
\textit{Notes:} Netherlands shown in red; donor countries in gray. Vertical dashed line indicates treatment date (2019Q2).
\end{figure}

\end{document}
