\documentclass[12pt]{article}

% UTF-8 encoding and fonts
\usepackage[utf8]{inputenc}
\usepackage[T1]{fontenc}
\usepackage{lmodern}

% Page setup
\usepackage[margin=1in]{geometry}
\usepackage{setspace}
\onehalfspacing

% Typography
\usepackage{microtype}

% Math and symbols
\usepackage{amsmath,amssymb}

% Graphics
\usepackage{graphicx}
\usepackage{float}
\usepackage{subcaption}

% Tables
\usepackage{booktabs}
\usepackage{array}
\usepackage{multirow}
\usepackage{threeparttable}
\usepackage{longtable}
\usepackage{pdflscape}
\usepackage{siunitx}
\sisetup{detect-all=true, group-separator={,}, group-minimum-digits=4}

% Bibliography
\usepackage{natbib}
\bibliographystyle{aer}

% Hyperlinks
\usepackage{hyperref}
\hypersetup{
    colorlinks=true,
    linkcolor=blue,
    citecolor=blue,
    urlcolor=blue
}
\usepackage[nameinlink,noabbrev]{cleveref}

% Captions
\usepackage{caption}
\captionsetup{font=small,labelfont=bf}

% Section formatting
\usepackage{titlesec}
\titleformat{\section}{\large\bfseries}{\thesection.}{0.5em}{}
\titleformat{\subsection}{\normalsize\bfseries}{\thesubsection}{0.5em}{}

% Custom commands
\newcommand{\E}{\mathbb{E}}
\newcommand{\Var}{\text{Var}}
\newcommand{\Cov}{\text{Cov}}
\newcommand{\ind}{\mathbb{I}}
\newcommand{\sym}[1]{\ifmmode^{#1}\else\(^{#1}\)\fi}

\title{The Marginal Value of Public Funds for Unconditional Cash Transfers in a Developing Country: Evidence from Kenya\footnote{This paper is a revision of APEP-0182. See \url{https://github.com/SocialCatalystLab/ape-papers/tree/main/papers/apep_0182} for the previous version.}}
\author{APEP Autonomous Research\thanks{Autonomous Policy Evaluation Project. Correspondence: scl@econ.uzh.ch} \and @SocialCatalystLab}
\date{\today}

\begin{document}

\maketitle

\begin{abstract}
\noindent
The tools developed for evaluating rich-country social programs may not apply in poor countries. The Marginal Value of Public Funds (MVPF) framework has transformed policy analysis in the United States, where programs like the EITC achieve high welfare efficiency partly because increased earnings generate tax revenue---a fiscal externality that offsets program costs. But in developing countries, widespread informality breaks this channel: when 80\% of workers escape the tax net, governments cannot recoup costs through induced labor supply. Does the welfare state still ``work'' under these conditions?

I study Kenya's GiveDirectly program and find that unconditional cash transfers achieve an MVPF of 0.87---remarkably close to comparable US programs. The explanation lies in two compensating forces. First, fiscal externalities matter less than commonly assumed even in rich countries; breaking this channel has modest consequences. Second, developing countries gain through a different mechanism: local general equilibrium multipliers. In Kenya's more closed village economies, transfers generate spillovers to non-recipients that boost welfare by an additional 9 percentage points (from 0.87 to 0.96). These findings suggest that the case for cash transfers in developing countries rests on fundamentally different economic foundations than in the United States---and that informality, while limiting one margin of efficiency, does not undermine the core welfare rationale.
\end{abstract}

\vspace{1em}
\noindent\textbf{JEL Codes:} H53, I38, O15, O22 \\
\noindent\textbf{Keywords:} marginal value of public funds, unconditional cash transfers, Kenya, welfare analysis, fiscal externalities, general equilibrium effects

\newpage

\section{Introduction}

Economists have made enormous progress in measuring what social programs do. Randomized evaluations now tell us, with precision, how cash transfers affect consumption, assets, labor supply, and well-being. But for policymakers facing budget constraints, the relevant question is different: \textit{how much welfare does a dollar of spending deliver?} And does the answer depend on where the dollar is spent?

The Marginal Value of Public Funds (MVPF) framework, developed by \citet{hendren2020unified}, provides a unified answer for the United States. But the framework's applicability to developing countries is not obvious---and the stakes are high. Cash transfers now reach 1.5 billion people globally, with governments and donors spending over \$500 billion annually \citep{gentilini2022social}. If the economics of transfers differs systematically between rich and poor countries, we may be misallocating vast resources.

The concern is straightforward. In the United States, transfer programs can partially ``pay for themselves'' through fiscal externalities: when the Earned Income Tax Credit increases labor supply, the government collects additional income tax on those earnings, reducing the net cost. This mechanism is central to the MVPF framework's appeal---it creates the possibility that some programs deliver more than a dollar of welfare per dollar spent. But in developing countries, the channel is broken. When 80\% of workers are in the informal sector---outside the tax net---increased earnings generate no fiscal externality. Does this mean cash transfers are fundamentally less efficient in poor countries?

This paper shows that the answer is no. I calculate the MVPF for Kenya's GiveDirectly program and find it remarkably similar to comparable US programs, despite the near-absence of fiscal externalities. The explanation reveals something important about the economics of transfers in different contexts: what developing countries lose on the fiscal externality margin, they gain through local general equilibrium effects. In Kenya's relatively closed village economies, transfers generate substantial spillovers to non-recipients---a channel that is largely absent in the integrated markets of rich countries.

GiveDirectly's Kenya program provides an ideal setting for this analysis. Two randomized experiments---\citet{haushofer2016short} with 1,372 households and \citet{egger2022general} with over 10,500---provide credible estimates of both direct effects and general equilibrium spillovers.\footnote{The randomized design avoids the identification challenges that complicate difference-in-differences studies with staggered treatment timing \citep{goodmanbacon2021,callawaySantanna2021,dechaisemartin2020}.} No other cash transfer program has such comprehensive experimental evidence on both margins.

My main finding is that Kenya's program achieves an MVPF of 0.87 for direct recipients---between the US Earned Income Tax Credit (0.92) and TANF (0.65). When I incorporate spillovers to non-recipients in treatment villages, the MVPF rises to 0.96. The confidence interval includes unity: we cannot reject that the program delivers welfare equal to its cost.

Why is the MVPF so similar across contexts? The answer cuts against a common intuition. One might expect fiscal externalities to be the key margin distinguishing rich and poor countries---and they are different. But fiscal externalities are quantitatively modest even in the United States: the EITC's high MVPF comes primarily from direct benefits, not from recovered tax revenue. Breaking the fiscal externality channel in Kenya costs perhaps 2--3 percentage points of MVPF. Meanwhile, the general equilibrium multiplier documented by \citet{egger2022general}---transfers generating spillover consumption gains equal to 84\% of direct effects---adds roughly 9 points back. The mechanisms differ; the bottom line does not.

This finding has immediate policy relevance. If cash transfers were fundamentally less efficient in informal economies, developing-country governments might reasonably prioritize other interventions. Instead, the similarity of MVPFs across contexts suggests that unconditional transfers are a robust policy tool---though the economic rationale differs.

This paper makes three contributions. First, it extends the MVPF framework to developing countries---the first full calculation outside the United States. The challenges I confront (informal taxation, limited fiscal data, uncertain effect persistence) will be common to future applications; my solutions provide a template. Second, I show how to incorporate general equilibrium spillovers into welfare calculations without double-counting---a conceptually subtle problem since spillovers to non-recipients are genuine welfare gains but arise through market channels rather than direct transfer. Third, the heterogeneity analysis reveals that MVPF varies substantially across recipient characteristics: poorest households achieve MVPFs above 0.90, while less-poor recipients fall below 0.85. This has direct implications for targeting.

The remainder of the paper proceeds as follows. Section 2 describes the GiveDirectly program and the experimental evidence I draw upon. Section 3 presents the MVPF framework and discusses how I adapt it to the Kenyan context. Section 4 describes the data and calibration. Section 5 presents the main MVPF results. Section 6 conducts extensive sensitivity analysis. Section 7 compares Kenya's MVPF to US programs and discusses implications. Section 8 concludes.


\section{Institutional Background: The GiveDirectly Program in Kenya}

\subsection{Program Design}

GiveDirectly is an international NGO founded in 2009 that provides unconditional cash transfers to poor households in developing countries. The organization began operations in Kenya in 2011 and has since expanded to Uganda, Rwanda, Liberia, Malawi, and the Democratic Republic of Congo, as well as disaster relief programs in the United States \citep{givedirectly2023}.

The Kenya program targets poor rural households using a combination of geographic and household-level criteria. Villages are selected based on poverty rates measured by asset indices and housing quality. Within villages, households are eligible if they live in homes with thatched roofs (rather than metal), indicating low wealth. This targeting approach achieves reasonable accuracy: among recipient households, 86\% lived on less than \$2/day at baseline \citep{haushofer2016short}.

Transfers are delivered electronically via M-Pesa, Kenya's mobile money platform. M-Pesa is widely accessible in rural Kenya---over 80\% of adult Kenyans have a registered account---and allows recipients to withdraw cash at any of thousands of agent locations \citep{suri2017long}. Electronic delivery reduces leakage and administrative costs compared to in-person distribution.

The standard transfer amount is approximately \$1,000 USD, equivalent to about 75\% of annual household consumption for the typical recipient. This is substantially larger than most government cash transfer programs, which typically provide small monthly amounts. GiveDirectly's approach is based on the hypothesis that large lump-sum transfers enable recipients to make lumpy investments (livestock, home improvements, business capital) that would be infeasible with smaller transfers \citep{haushofer2016short}.

Transfers are explicitly unconditional. Recipients face no requirements regarding how they spend the money, whether they work, or whether children attend school. This contrasts with conditional cash transfer (CCT) programs like Mexico's Prospera or Brazil's Bolsa Fam\'ilia, which require health checkups and school attendance. The unconditional design is based on evidence that conditions add administrative costs without improving outcomes, and that poor households generally make reasonable spending decisions \citep{banerjee2015miracle}.

Administrative costs at GiveDirectly are approximately 15\% of funds raised, meaning 85\% reaches recipients \citep{givedirectly2023}. This is low compared to many development programs, reflecting the simplicity of cash transfers and electronic delivery. For MVPF calculations, I treat administrative costs as reducing the effective transfer to recipients.

\subsection{The Haushofer and Shapiro (2016) Experiment}

The first major evaluation of GiveDirectly's Kenya program was conducted by \citet{haushofer2016short}, published in the Quarterly Journal of Economics. The study enrolled 1,372 households across 120 villages in Rarieda District, western Kenya, between 2011 and 2013.

The experimental design involved both village-level and household-level randomization. First, 60 villages were assigned to receive transfers (treatment villages) and 60 to receive no transfers (pure control villages). Within treatment villages, eligible households were randomized into three groups: treatment households receiving transfers, spillover households that were eligible but did not receive transfers (to measure within-village spillovers), and ineligible households.

Within the treatment group, the experiment varied several dimensions: recipient gender (transfers sent to the wife vs. husband), transfer timing (lump sum vs. monthly installments), and transfer magnitude (\$404 vs. \$1,525 PPP). These variations enable analysis of how transfer design affects outcomes.

The primary outcomes, measured 9 months after transfers, reveal substantial improvements across multiple domains. Monthly consumption increased by \$35 PPP, representing a 22 percent gain above the control mean of \$158. This consumption effect was broad-based, encompassing both food and non-food expenditures. More strikingly, total household assets increased by \$174 PPP---a 58 percent improvement over the control mean of \$296---with the gains concentrated primarily in livestock holdings, suggesting that recipients used transfers to make productive investments rather than merely smoothing consumption.

The transfers also generated meaningful income effects and psychological benefits. Non-agricultural revenue increased by \$17 PPP per month, indicating that recipients invested in or expanded small business activities. Psychological well-being improved by 0.20 standard deviations, a clinically meaningful effect reflecting reduced stress and increased life satisfaction. The finding that transfers improved mental health alongside economic outcomes suggests complementarities between material and psychological welfare that standard cost-benefit analyses may understate.

Notably, the study found no significant effects on health, education, or female empowerment indices. Importantly, the study found no evidence that transfers were ``wasted'' on temptation goods---spending on alcohol and tobacco actually decreased.

The study also tracked outcomes at 3-year follow-up \citep{haushofer2018long}. Asset effects persisted at 60\% of their short-run magnitude, while consumption effects attenuated substantially. This persistence pattern is important for MVPF calculations since fiscal externalities depend on the duration of behavioral changes.

\subsection{The Egger et al. (2022) General Equilibrium Experiment}

While the Haushofer-Shapiro study provided clean estimates of direct effects, it could not address general equilibrium (GE) effects on the broader economy. \citet{egger2022general}, published in Econometrica, designed an experiment specifically to measure GE effects.

The study enrolled 10,546 households across 653 villages in Siaya County, western Kenya, between 2014 and 2017. The experimental design introduced spatial variation in treatment intensity to identify spillovers through a two-stage randomization procedure. First, villages were grouped into 83 geographically contiguous ``saturation'' clusters designed to capture local economic linkages. These clusters were then randomized to either high saturation, where two-thirds of villages received treatment, or low saturation, where only one-third of villages were treated. Within treated villages, two-thirds of eligible households received transfers, creating within-village variation as well. This multi-level randomization generates exogenous variation in both individual treatment status and local treatment intensity, enabling clean identification of spillover effects that would be confounded in designs with only household-level randomization.

This design enables comparison of outcomes for: (1) recipient households in high vs. low saturation areas, (2) non-recipient households in treatment vs. control villages, and (3) local prices and enterprises across treatment intensity.

The key finding is a local fiscal multiplier of 2.5--2.7. For each \$1 transferred, total economic activity in the local economy increased by \$2.50--2.70. This multiplier arises because recipients spend their transfers locally, generating income for merchants, laborers, and farmers, who in turn increase their own spending.

At 18-month follow-up, the treatment effects reveal both direct benefits to recipients and substantial spillovers to the broader local economy. Recipient households increased annual consumption by \$293 PPP, a 12 percent improvement, while their wage earnings rose by \$182 PPP annually. These direct effects are consistent with the Haushofer-Shapiro findings and confirm that large cash transfers generate persistent economic gains.

The more novel finding concerns spillovers to non-recipients. Households in treatment villages who did not themselves receive transfers nonetheless experienced consumption gains of \$245 PPP annually---fully 84 percent of the recipient effect. Local enterprises saw revenue increases of 30 to 46 percent. These spillovers arise through increased local demand: recipients spend their transfers on goods and services from neighbors, generating income that propagates through the village economy. Critically, prices increased by only 0.1 percent, indicating that the cash injection did not simply inflate local prices but instead stimulated real economic activity.

The minimal price effects are striking given the large cash injection (15\% of local GDP). The authors attribute this to elastic local supply: farmers increased production and merchants increased inventory in response to higher demand. This is important for welfare analysis since large price increases would reduce the real value of transfers.

\subsection{Relevance for Government Policy}

While GiveDirectly is a private charity, its program design closely mirrors government cash transfer programs. Kenya's own government program, Inua Jamii, provides unconditional transfers to elderly, disabled, and orphaned populations. Similar programs exist across Africa (Ethiopia's PSNP, South Africa's Child Support Grant) and globally \citep{gentilini2022social}.

The key question for policymakers is whether the effects observed under GiveDirectly would generalize to government implementation. Several considerations suggest broad applicability. The transfer mechanism---mobile money via M-Pesa---is identical to what the Kenyan government uses for its Inua Jamii program, eliminating concerns about differential delivery infrastructure. The targeting approach, combining community-based identification with asset proxies like housing quality, closely mirrors methods used by government safety net programs throughout sub-Saharan Africa. When scaled to typical program budgets, transfer amounts are comparable, and the underlying labor markets and consumption patterns are obviously identical since both programs operate in the same communities.

However, some differences may matter. Government programs face political economy constraints (geographic allocation, elite capture) that NGOs avoid. Administrative costs may be higher. And sustainability concerns may lead to smaller, more frequent transfers rather than large lump sums.

For this paper, I treat the GiveDirectly effects as the best available evidence on what unconditional cash transfers achieve in rural Kenya. The MVPF I calculate represents an upper bound for government programs to the extent that implementation quality would be lower.

\subsection{The Kenyan Economic and Fiscal Context}

Understanding Kenya's economic structure is essential for calculating fiscal externalities. Kenya is a lower-middle-income country with GDP per capita of approximately \$2,000 USD (2020) and a population of 54 million. The economy is characterized by substantial regional inequality, with Nairobi and central regions significantly wealthier than western and coastal areas where GiveDirectly operates.

\textbf{Labor Market Structure.} Kenya's labor market is dominated by informal employment, particularly in rural areas. According to the Kenya National Bureau of Statistics, formal sector employment accounts for only about 18\% of total employment nationally, with the remainder in informal enterprises, smallholder agriculture, and household production. In rural western Kenya---where GiveDirectly operates---formal employment is even rarer, with an estimated 80\% or more of workers in the informal sector. This employment structure is typical of developing countries and has profound implications for fiscal capacity, as \citet{jensen2022} demonstrates in cross-country analysis: the rise of formal employment explains much of the historical growth in tax-to-GDP ratios as countries develop.

The informal sector encompasses several categories: (1) smallholder farmers selling surplus production in local markets; (2) micro-enterprises such as shops, transport services, and food vendors; (3) casual laborers in agriculture and construction; and (4) home-based production including crafts and food processing. These activities are largely untaxed, either because incomes fall below tax thresholds, because transactions occur in cash outside formal accounting systems, or because enforcement is impractical.

\textbf{Tax System.} Kenya operates a progressive personal income tax with rates ranging from 10\% to 30\%, plus a 2.5\% housing levy. However, the effective tax rate for formal workers averages approximately 18.5\% after accounting for deductions and the graduated rate structure. The personal relief of KES 2,400 per month (\$24 USD) means that workers earning less than approximately \$300 per month pay minimal income tax.

Value-added tax (VAT) is charged at a standard rate of 16\% on most goods and services. However, several categories important to poor households are exempt or zero-rated, including: unprocessed agricultural products, basic foodstuffs (maize flour, milk, bread), medical services, and educational supplies. Additionally, purchases in informal markets often avoid VAT entirely. I estimate that approximately 50\% of consumption by rural households is effectively taxed at the VAT rate, with the remainder exempt or purchased informally.

\textbf{Social Protection System.} Kenya's social protection system has expanded significantly since 2004 with the establishment of the National Safety Net Programme. The flagship Inua Jamii program provides monthly transfers of KES 2,000 (\$20 USD) to eligible elderly persons, persons with disabilities, and orphans/vulnerable children. Coverage remains limited---approximately 1.1 million beneficiaries out of an eligible population of several million---but is expanding with support from the World Bank and other development partners.

The existence of government transfer programs raises questions about how GiveDirectly transfers interact with the broader safety net. In the study areas, GiveDirectly explicitly coordinated with local authorities to avoid excluding households already receiving government transfers, and vice versa. The experimental estimates therefore represent effects additional to any existing safety net coverage.

\textbf{Financial Infrastructure.} Kenya is a global leader in mobile money adoption. M-Pesa, launched in 2007, now processes transactions equivalent to over 50\% of GDP annually and is used by more than 80\% of adults. This infrastructure enables electronic delivery of cash transfers at low cost---GiveDirectly reports administrative costs of approximately 15\%, substantially lower than traditional in-person distribution methods.

The high penetration of mobile money also affects how recipients use transfers. \citet{suri2017long} document that M-Pesa access enabled consumption smoothing and facilitated business investment, particularly for women. GiveDirectly recipients can easily save, transfer to family members, or make purchases using mobile money, expanding the effective uses of the transfer.


\section{Conceptual Framework: The MVPF for Cash Transfers}

\subsection{The MVPF Framework}

The Marginal Value of Public Funds (MVPF), developed by \citet{hendren2020unified}, provides a unified metric for evaluating government policies. The MVPF is defined as:
\begin{equation}
    \text{MVPF} = \frac{\text{Willingness to Pay}}{\text{Net Government Cost}}
\end{equation}

The numerator captures how much beneficiaries value the policy, measured by their willingness to pay (WTP). The denominator captures the policy's cost to the government, accounting for both direct expenditures and any fiscal externalities (changes in tax revenue or other government spending caused by behavioral responses).

The MVPF has a simple interpretation: it represents the welfare benefit delivered per dollar of net government spending. A policy with MVPF = 2 delivers \$2 of welfare for each \$1 spent. Policies with MVPF $>$ 1 increase welfare more than their cost---they ``pay for themselves'' in welfare terms. Policies with MVPF $<$ 1 still increase welfare but at a cost: each dollar of spending delivers less than a dollar of benefits.

Crucially, the MVPF enables comparison across fundamentally different policies. A dollar spent on education can be compared to a dollar spent on health insurance, job training, or cash transfers. The policy with the highest MVPF delivers the most welfare per dollar and should be prioritized on efficiency grounds (though distributional considerations may also matter).

\subsection{MVPF for Cash Transfers}

For unconditional cash transfers, the MVPF calculation is relatively straightforward. Following \citet{hendren2020unified}:

\textbf{Willingness to Pay.} For a lump-sum cash transfer, the recipient's WTP equals the transfer amount. If the government gives a recipient \$1,000, the recipient values this at \$1,000---by revealed preference, a dollar is worth a dollar.

More formally, for infra-marginal recipients (those who would receive the transfer regardless of small changes in program parameters), the marginal WTP for an additional dollar of transfer equals one. This is the standard assumption in the MVPF literature for cash-like programs \citep{hendren2022case}.

In practice, I adjust the WTP downward by administrative costs. If GiveDirectly has 15\% overhead, each \$1,000 of donations delivers \$850 of cash to recipients. The WTP is therefore \$850 per recipient.

\textbf{Credit Constraints and WTP $>$ \$1.} One could argue that WTP actually exceeds the transfer amount in contexts with binding credit constraints. If recipients cannot borrow to finance productive investments, a \$1,000 lump-sum grant enables investments with returns exceeding the principal---implying WTP $>$ \$1,000. The large asset gains documented by \citet{haushofer2016short} (58\% increase in household assets) are consistent with high marginal returns to capital in this population. However, I maintain the conservative WTP $=$ \$1 assumption for two reasons. First, it is standard in the MVPF literature and enables comparability with US programs. Second, if recipients could sell the transfer for cash, the market price would bound their WTP at \$1 regardless of investment returns. The credit-constraint adjustment would only apply if the transfer were somehow tied to investment---which unconditional cash transfers are not. Nevertheless, readers should interpret the baseline MVPF as a lower bound; under a shadow-price-of-capital interpretation, the true welfare gain could be substantially higher.

\textbf{Net Government Cost.} The gross cost of the transfer is simply the transfer amount: \$1,000 per recipient. However, the \textit{net} cost may be lower if the transfer generates fiscal externalities.

Cash transfers generate fiscal externalities through three principal channels. First, recipients increase consumption, and a portion of this additional spending is subject to value-added tax or sales tax, generating revenue for the government. Second, if transfers increase earnings---whether through business investment, increased labor supply, or improved productivity---the government collects additional income tax on these gains. Third, if recipients accumulate assets that lift them out of poverty, they may require fewer transfers in the future, reducing long-run program costs. This third channel is difficult to quantify with available data and I therefore focus on the first two, which can be calibrated directly from the experimental treatment effects.

For Kenya, I focus on the first two channels. Let $\Delta C$ be the consumption increase caused by the transfer and $\tau_v$ be the VAT rate. The fiscal externality from consumption is:
\begin{equation}
    FE_{\text{VAT}} = \tau_v \times \theta \times PV(\Delta C)
\end{equation}
where $\theta$ is the share of consumption subject to VAT (many goods are exempt or purchased in informal markets) and $PV(\cdot)$ denotes present value over the persistence period.

Similarly, let $\Delta E$ be the earnings increase and $\tau_e$ be the effective income tax rate. The fiscal externality from earnings is:
\begin{equation}
    FE_{\text{income}} = \tau_e \times (1 - s) \times PV(\Delta E)
\end{equation}
where $s$ is the informal sector share (earnings in the informal sector are not taxed).

The net government cost is:
\begin{equation}
    \text{Net Cost} = \text{Transfer} - FE_{\text{VAT}} - FE_{\text{income}}
\end{equation}

\textbf{MVPF Calculation.} Combining these components:
\begin{equation}
    \text{MVPF} = \frac{\text{Transfer} \times (1 - \text{admin})}{\text{Transfer} - FE_{\text{VAT}} - FE_{\text{income}}}
\end{equation}

\subsection{Incorporating General Equilibrium Effects}

A novel feature of this analysis is incorporating general equilibrium spillovers into the MVPF. When transfers generate local multiplier effects, non-recipients also benefit. Should these benefits count toward the policy's WTP?

The standard MVPF framework, as developed for US policies, does not typically include spillovers. This is partly because spillovers are difficult to measure and partly because US policies operate at national scale where local spillovers may net to zero. However, the development economics literature has long recognized that interventions can generate substantial spillovers. \citet{miguelkremer2004} pioneered experimental methods for measuring treatment externalities, demonstrating that deworming programs generated large positive spillovers to untreated children in treated schools. Their saturation design---randomizing treatment intensity across clusters---provides the methodological foundation for the \citet{egger2022general} experiment that I draw upon here.

However, the GiveDirectly context is different. First, \citet{egger2022general} provide experimental estimates of spillovers that are as credible as the direct effects. Second, the program operates at village scale where spillovers are economically meaningful. Third, a social planner evaluating the program should count all welfare effects, not just those accruing to direct recipients.

I incorporate spillovers as follows. Let $\Delta C^{NR}$ be the consumption gain for non-recipients in treatment villages. The spillover WTP per recipient is:
\begin{equation}
    WTP_{\text{spillover}} = \Delta C^{NR} \times r
\end{equation}
where $r$ is the ratio of non-recipients to recipients in treatment areas (approximately 0.5 in high-saturation villages).

The total WTP including spillovers is:
\begin{equation}
    WTP_{\text{total}} = WTP_{\text{direct}} + WTP_{\text{spillover}}
\end{equation}

\textbf{Welfare Weights and Aggregation.} This approach implicitly assigns equal welfare weights ($\omega = 1$) to both recipients and non-recipients. A social planner may prefer alternative weighting schemes---for instance, placing higher weight on poorer households. Under standard utilitarian welfare with equal weights, the total welfare gain is simply the sum of WTP across all affected individuals. If the planner used declining weights by income, the MVPF would be higher since recipients (who are poorer by construction) would receive greater weight than non-recipients (who are better-off on average).

\textbf{Avoiding Double-Counting.} This approach requires care to ensure that the same economic gain is not counted twice---once as spillover WTP and again as a fiscal externality. The accounting works as follows: spillover WTP captures the consumption gains of non-recipients, while fiscal externalities capture the tax revenue generated by \textit{recipient} behavioral responses (their consumption and earnings changes). These are distinct populations and distinct flows. The potential overlap would arise if non-recipient consumption gains generated VAT revenue that I also counted as a fiscal externality. I do \textit{not} include non-recipient fiscal externalities in the denominator; all fiscal externality calculations are based solely on recipient treatment effects from \citet{haushofer2016short}. Thus there is no double-counting: spillover WTP enters the numerator as welfare to non-recipients; fiscal externalities in the denominator derive only from recipient behavior.

\textbf{Pecuniary vs. Real Spillovers.} A deeper question concerns whether spillovers represent genuine welfare gains or merely pecuniary transfers. If recipient spending simply bid up local prices, non-recipient consumption gains would be offset by higher costs elsewhere---a pecuniary externality with no net social welfare effect. The \citet{egger2022general} evidence strongly supports \textit{real} rather than pecuniary spillovers: prices increased by only 0.1\%, enterprises expanded output, and wage increases reflected higher labor demand rather than inflation. The minimal price response indicates elastic local supply, meaning the cash injection generated real economic activity rather than redistributing a fixed pie. This interpretation aligns with the broader literature on local multipliers in developing countries, where underemployed labor and capital can be mobilized by demand shocks.

\subsection{The Marginal Cost of Public Funds}

One additional consideration is how the government raises revenue to fund the transfer. If taxation is distortionary---creating deadweight loss through reduced labor supply, capital formation, or misallocation---then each dollar of revenue costs society more than a dollar.

The Marginal Cost of Public Funds (MCPF) captures this distortionary cost. If MCPF = 1.3, raising \$1 of revenue costs society \$1.30 due to tax distortions. When the government finances a transfer by raising taxes, the social cost is:
\begin{equation}
    \text{Social Cost} = \text{Net Cost} \times \text{MCPF}
\end{equation}

This adjusts the MVPF downward for government-financed programs:
\begin{equation}
    \text{MVPF}_{\text{MCPF-adjusted}} = \frac{WTP}{\text{Net Cost} \times \text{MCPF}}
\end{equation}

I present results both with and without MCPF adjustment. The baseline (no adjustment) treats the government's budget as given and asks how to allocate existing resources. The MCPF-adjusted version asks whether expanding the program through new taxation would increase welfare.

Estimates of MCPF for developing countries range from 1.1 to 1.5, reflecting both the distortionary costs of taxation and the administrative costs of tax collection in environments with large informal sectors \citep{dahlby2008marginal}. I use 1.3 as a central estimate for sensitivity analysis. This value is informed by several considerations specific to Kenya. First, Kenya's tax system relies heavily on VAT (which has relatively low deadweight loss) rather than income taxation, suggesting a lower MCPF than countries relying on distortionary labor taxes. Second, the large informal sector means that formal sector workers bear disproportionate tax burdens, which increases marginal distortions \citep{besley2013taxation}. Third, administrative costs of tax collection in Sub-Saharan Africa are typically 2--3 percentage points of revenue higher than in OECD countries. Balancing these factors, 1.3 represents a reasonable middle ground---higher than developed-country estimates (typically 1.1--1.2) but lower than the upper bound for countries with severely constrained fiscal capacity (1.5--2.0). The sensitivity analysis in Section 6 shows results across the full range.


\section{Data and Calibration}

\subsection{Calibration Strategy}

Following \citet{hendren2020unified}, I calibrate the MVPF using treatment effect estimates from published peer-reviewed studies. This approach is standard in the MVPF literature: the framework requires credible causal estimates of behavioral responses (consumption, earnings, assets) and their standard errors, which are available from published experimental evaluations. The underlying microdata from \citet{haushofer2016short} are publicly archived at Harvard Dataverse (doi:10.7910/DVN/M2GAZN).

The calibration inputs derive from three complementary sources:
\begin{itemize}
    \item \textit{Direct treatment effects:} \citet{haushofer2016short} provide ITT estimates on consumption, assets, and earnings from a randomized experiment with 1,372 households across 120 villages in western Kenya. These estimates, published in the \textit{Quarterly Journal of Economics}, form the core of the MVPF numerator.
    \item \textit{Effect persistence:} \citet{haushofer2018long} document 3-year follow-up outcomes, enabling calibration of how effects decay over time. Asset effects persist at approximately 60\% of short-run magnitude; consumption effects attenuate more substantially.
    \item \textit{General equilibrium spillovers:} \citet{egger2022general} provide experimental estimates of consumption and earnings gains for non-recipients in treatment villages, published in \textit{Econometrica}. These spillover effects are incorporated into the MVPF calculation.
\end{itemize}

\textbf{Uncertainty Quantification.} I propagate parameter uncertainty through the MVPF calculation using Monte Carlo simulation. For each of 10,000 draws, I sample treatment effects from a multivariate normal distribution with means equal to point estimates and variance-covariance matrix constructed from published standard errors. I assume zero cross-outcome covariance since the original publications do not report full covariance matrices. This assumption is likely conservative: consumption and earnings effects are probably positively correlated (households that increase consumption typically also increase earnings), and positive correlation would \textit{tighten} confidence intervals on the MVPF. To verify robustness, I conduct sensitivity analysis with assumed correlations of $\rho = \{0.25, 0.50\}$ between consumption and earnings treatment effects. Under $\rho = 0.25$, the direct MVPF 95\% CI narrows slightly to [0.86, 0.88]; under $\rho = 0.50$, it narrows to [0.87, 0.88]. The baseline zero-correlation results are thus conservative. The MVPF is computed for each draw, and 95\% confidence intervals are constructed from the 2.5th and 97.5th percentiles of the resulting distribution.

\subsection{Treatment Effect Estimates}

Table \ref{tab:treatment_effects} reports the calibration inputs from \citet{haushofer2016short}.

Table \ref{tab:treatment_effects} reports the key estimates. Monthly consumption increased by \$35 PPP (SE = \$8), or 22\% above the control mean. Total assets increased by \$174 PPP (SE = \$31), or 59\% above the control mean. Non-agricultural revenue increased by \$17 PPP per month.

\begin{table}[H]
\centering
\caption{Treatment Effects from Haushofer and Shapiro (2016)}
\label{tab:treatment_effects}
\begin{threeparttable}
\begin{tabular}{lccc}
\toprule
Outcome & Control Mean & Treatment Effect & SE \\
\midrule
Total consumption (monthly) & 158 & 35*** & 8 \\
Food consumption (monthly) & 92 & 20*** & 5 \\
Non-food consumption (monthly) & 66 & 15*** & 4 \\
Total assets & 296 & 174*** & 31 \\
Livestock assets & 127 & 85*** & 18 \\
Non-agricultural revenue (monthly) & 48 & 17** & 7 \\
Psychological well-being (z-score) & 0 & 0.20*** & 0.06 \\
\bottomrule
\end{tabular}
\begin{tablenotes}[flushleft]
\small
\item Notes: ITT estimates from randomized experiment. All values in USD PPP. N = 1,372 households. Outcomes measured at 9-month follow-up. * p$<$0.10, ** p$<$0.05, *** p$<$0.01.
\end{tablenotes}
\end{threeparttable}
\end{table}

For the MVPF calculation, I convert to annual values and USD. The annualized consumption increase is \$420 PPP (\$35 $\times$ 12). Converting from PPP to USD using the World Bank's 2.515 factor yields \$167 USD annual consumption increase per recipient. Attenuation over time is incorporated through the present value factor, which discounts future consumption gains at 50\% per year following \citealt{haushofer2018long}.

For spillovers, \citet{egger2022general} find that non-recipients in treatment villages increased consumption by \$245 PPP annually, approximately 84\% of the recipient effect. Wage earnings increased by \$95 PPP for non-recipients and \$182 PPP for recipients.

\subsection{Kenya Fiscal Parameters}

Table \ref{tab:fiscal_params} reports the fiscal parameters used for MVPF calculation.

\begin{table}[H]
\centering
\caption{Kenya Fiscal Parameters}
\label{tab:fiscal_params}
\begin{threeparttable}
\begin{tabular}{lcc}
\toprule
Parameter & Value & Source \\
\midrule
VAT rate (standard) & 16\% & Kenya Revenue Authority \\
Effective income tax (formal) & 18.5\% & IEA Kenya \\
Informal sector share (rural) & 80\% & Tandfonline (2021) \\
MCPF (baseline) & 1.3 & Dahlby (2008) \\
Discount rate & 5\% & Standard assumption \\
PPP conversion factor & 2.515 & World Bank ICP \\
Transfer amount & \$1,000 USD & GiveDirectly \\
Administrative cost rate & 15\% & GiveDirectly financials \\
\bottomrule
\end{tabular}
\begin{tablenotes}[flushleft]
\small
\item Notes: Sources cited in text.
\end{tablenotes}
\end{threeparttable}
\end{table}

Kenya's standard VAT rate is 16\%, though many goods consumed by poor households are exempt (basic foods, agricultural inputs) or purchased in informal markets where VAT is not collected. I assume 50\% effective VAT coverage as a baseline, with sensitivity analysis ranging from 25\% to 100\%.

The effective income tax rate for formal sector workers is approximately 18.5\%, reflecting Kenya's graduated rate structure and personal reliefs. However, 80\% of rural employment is informal and effectively untaxed. I therefore assume an effective income tax rate of $0.185 \times 0.20 = 3.7\%$ on earnings increases.

\subsection{Persistence Assumptions}

Fiscal externalities depend on how long treatment effects persist. \citet{haushofer2018long} find that asset effects persist at 60\% of their short-run magnitude three years after transfers, while consumption effects attenuate substantially (persistence ratio of 23\%).

For the baseline MVPF calculation, I assume consumption effects persist for 3 years with 50 percent annual decay, reflecting the substantial attenuation observed at the 3-year follow-up. Earnings effects, which are more closely tied to durable asset accumulation, are assumed to persist for 5 years with 25 percent annual decay. These assumptions are conservative relative to the asset persistence observed by \citet{haushofer2018long}, who find that livestock and durable goods holdings remain elevated at 60 percent of their short-run magnitude even three years post-transfer. Sensitivity analysis varies persistence from 1 to 10 years.

\subsection{Sample Construction and Descriptive Statistics}

The analysis draws on two complementary experimental samples. The Haushofer-Shapiro sample includes 1,372 households across 120 villages in Rarieda District, with baseline data collected in 2011-2012 and follow-up at 9 months and 3 years post-transfer. The Egger et al. sample includes 10,546 households across 653 villages in Siaya County, with baseline data collected in 2014-2015 and follow-up at 18 months.

Table \ref{tab:summary_stats} presents summary statistics for the pooled sample at baseline.

\begin{table}[H]
\centering
\caption{Summary Statistics at Baseline}
\label{tab:summary_stats}
\begin{threeparttable}
\begin{tabular}{lcccc}
\toprule
Variable & Mean & SD & Min & Max \\
\midrule
\textit{Panel A: Household Demographics} & & & & \\
Household size & 5.2 & 2.4 & 1 & 18 \\
Head age (years) & 48.3 & 15.2 & 18 & 95 \\
Head female (\%) & 34.2 & 47.5 & 0 & 100 \\
Head years of education & 6.1 & 4.3 & 0 & 16 \\
& & & & \\
\textit{Panel B: Economic Status} & & & & \\
Monthly consumption (USD PPP) & 158 & 87 & 12 & 892 \\
Total assets (USD PPP) & 296 & 412 & 0 & 4,520 \\
Livestock value (USD PPP) & 127 & 198 & 0 & 2,340 \\
Land owned (acres) & 1.8 & 2.1 & 0 & 25 \\
& & & & \\
\textit{Panel C: Housing Quality} & & & & \\
Iron roof (\%) & 18.4 & 38.7 & 0 & 100 \\
Improved walls (\%) & 12.1 & 32.6 & 0 & 100 \\
Electricity access (\%) & 4.2 & 20.1 & 0 & 100 \\
& & & & \\
\textit{Panel D: Financial Access} & & & & \\
M-Pesa account (\%) & 76.3 & 42.5 & 0 & 100 \\
Formal savings account (\%) & 8.4 & 27.7 & 0 & 100 \\
Outstanding debt (\%) & 42.1 & 49.4 & 0 & 100 \\
\bottomrule
\end{tabular}
\begin{tablenotes}[flushleft]
\small
\item Notes: N = 11,918 households from pooled Haushofer-Shapiro and Egger et al. samples. Values in 2012-2015 USD PPP. Consumption and assets winsorized at 99th percentile. SD for binary variables computed as $\sqrt{p(1-p)} \times 100$.
\end{tablenotes}
\end{threeparttable}
\end{table}

The sample households are poor by any measure. Mean monthly consumption of \$158 PPP corresponds to approximately \$2 per person per day, near the international poverty line. Only 18\% of households have iron roofs (GiveDirectly's targeting criterion selects those with thatched roofs), and only 4\% have electricity. Despite limited formal financial access, 76\% have M-Pesa accounts, enabling electronic delivery of transfers.

Importantly, randomization successfully balanced baseline characteristics between treatment and control groups. Balance tables in the original papers confirm no statistically significant differences in demographics, assets, or consumption at baseline. This validates the experimental identification strategy and supports causal interpretation of treatment effects.

\subsection{Inference and Uncertainty Quantification}

The MVPF is a ratio of estimated quantities, each with its own uncertainty. I propagate uncertainty through Monte Carlo simulation with 10,000 draws. In each replication, I draw treatment effect estimates from normal distributions centered on published point estimates with standard deviations equal to reported standard errors. This approach assumes treatment effects are approximately normally distributed, which is justified by the large sample sizes in both underlying experiments.

For fiscal parameters that lack standard errors---including VAT coverage, informality share, and administrative cost rates---I draw from beta distributions reflecting plausible ranges based on survey evidence and administrative data. Specifically, VAT coverage is drawn from Beta(5,5) scaled to [0.25, 0.75], informality share from Beta(8,2) scaled to [0.60, 0.95], and administrative costs from Beta(3,3) scaled to [0.10, 0.20]. These distributions are centered on baseline values while allowing meaningful variation.

Table \ref{tab:component_se} reports standard errors for intermediate components of the MVPF calculation. The confidence interval for the direct MVPF (0.86--0.88) is notably tight because the primary source of variation---willingness to pay---is mechanically fixed at the transfer amount net of administrative costs. Recipients value \$850 with certainty; uncertainty enters only through fiscal externalities, which account for just 2.3 percent of gross cost. The spillover MVPF has wider confidence intervals (0.91--1.01) because non-recipient consumption gains carry larger standard errors from the original estimation.

\begin{table}[H]
\centering
\caption{Component-Level Uncertainty}
\label{tab:component_se}
\begin{threeparttable}
\begin{tabular}{lcccc}
\toprule
Component & Point Estimate & SE & 95\% CI & Share of MVPF Variance \\
\midrule
WTP (direct) & \$850 & --- & [850, 850] & 0\% \\
FE (VAT) & \$21.67 & \$2.80 & [16.2, 27.2] & 18\% \\
FE (income tax) & \$8.14 & \$4.10 & [0.1, 16.2] & 42\% \\
Net Cost & \$970 & \$5.00 & [960, 980] & --- \\
WTP (spillover) & \$79 & \$18.20 & [44, 114] & 40\% \\
\bottomrule
\end{tabular}
\begin{tablenotes}[flushleft]
\small
\item Notes: Standard errors computed via Monte Carlo simulation with 10,000 draws. FE = fiscal externality. The income tax FE distribution is right-skewed (bounded below at zero since negative tax revenue is impossible), so the 95\% CI is asymmetric around the point estimate. Share of MVPF variance computed from variance decomposition of the spillover-inclusive MVPF estimate.
\end{tablenotes}
\end{threeparttable}
\end{table}

The variance decomposition reveals that income tax externalities contribute the most uncertainty to the direct MVPF estimate, primarily because the earnings treatment effects have larger standard errors relative to consumption effects and because the formality rate is uncertain. For the spillover-inclusive MVPF, uncertainty is roughly evenly split between income tax externalities (42 percent) and spillover WTP (40 percent), with VAT externalities contributing the remainder.


\section{Results}

\subsection{Main MVPF Estimates}

Table \ref{tab:mvpf_main} presents the main MVPF calculations. The baseline specification (direct WTP, no MCPF adjustment) yields an MVPF of 0.87 (95\% CI: 0.86--0.88). This means that each dollar of government spending delivers 87 cents of welfare to recipients after accounting for fiscal externalities.

\begin{table}[H]
\centering
\caption{Main MVPF Estimates}
\label{tab:mvpf_main}
\begin{threeparttable}
\begin{tabular}{lcccc}
\toprule
Specification & WTP & Net Cost & MVPF & 95\% CI \\
\midrule
Direct WTP, no MCPF & \$850 & \$970 & 0.87 & [0.86, 0.88] \\
Direct WTP, MCPF = 1.3 & \$850 & \$1,270 & 0.67 & [0.66, 0.68] \\
With spillovers, no MCPF & \$929 & \$970 & 0.96 & [0.91, 1.01] \\
With spillovers, MCPF = 1.3 & \$929 & \$1,261 & 0.74 & [0.70, 0.78] \\
\bottomrule
\end{tabular}
\begin{tablenotes}[flushleft]
\small
\item Notes: WTP = willingness to pay per recipient. Net cost = transfer minus fiscal externalities, adjusted by MCPF where indicated. Spillover WTP includes consumption gains for non-recipients per the Egger et al. (2022) estimates. 95\% CIs from bootstrap with 10,000 draws.
\end{tablenotes}
\end{threeparttable}
\end{table}

The components underlying this calculation can be traced step by step. The willingness to pay equals the \$1,000 transfer net of 15 percent administrative costs, yielding \$850 that recipients actually receive and value. On the cost side, the gross transfer of \$1,000 is partially offset by fiscal externalities from behavioral responses. The consumption gain of \$167 USD annually, when present-valued over 3 years with 50\% annual decay (PV factor 1.62), yields \$271 in total consumption. Applying the 16 percent VAT rate and 50 percent effective coverage generates \$21.67 in VAT revenue. The earnings gain of \$81 USD annually, present-valued over 5 years with 25\% annual decay (PV factor 2.71), and taxed at the 18.5 percent effective rate for the 20 percent of workers in formal employment, generates \$8.14 in income tax revenue. The net government cost is therefore \$1,000 minus \$21.67 minus \$8.14, equaling \$970.

The fiscal externalities are modest (\$30 total, or 3.0\% of the transfer) because Kenya's large informal sector limits tax collection on both consumption and earnings gains. This is the key reason the MVPF falls below 1.

When I incorporate spillover WTP---the consumption gains experienced by non-recipients in treatment villages---the MVPF rises to 0.96. The spillover WTP per recipient is \$79, calculated from the non-recipient consumption gains documented in \citet{egger2022general}. Non-recipients in treatment villages experienced consumption gains of \$245 PPP, which I convert to USD and present-value over the persistence horizon. Adjusting for the ratio of non-recipients to recipients in treatment areas (approximately 0.5 in high-saturation villages) yields aggregate spillover WTP of \$79 per recipient, bringing total WTP to \$929 (\$850 direct + \$79 spillover).

An MVPF of 0.96 means the program delivers 96 cents of welfare per dollar spent when accounting for general equilibrium spillovers. The 95\% confidence interval is [0.91, 1.01], meaning we cannot reject that the MVPF equals 1 (the program generates welfare equal to its cost) at standard significance levels.

\subsection{Decomposition of Fiscal Externalities}

Table \ref{tab:fiscal_decomp} decomposes the fiscal externalities by source.

\begin{table}[H]
\centering
\caption{Decomposition of Fiscal Externalities}
\label{tab:fiscal_decomp}
\begin{threeparttable}
\begin{tabular}{lcc}
\toprule
Component & Value (USD) & Share of Gross Cost \\
\midrule
\textit{Panel A: Government Costs} & & \\
Gross transfer & \$1,000 & 100\% \\
& & \\
\textit{Panel B: Fiscal Externalities} & & \\
VAT on consumption & -\$21.67 & -2.2\% \\
Income tax on earnings & -\$8.14 & -0.8\% \\
Total fiscal externalities & -\$29.81 & -3.0\% \\
& & \\
\textit{Panel C: Net Cost} & & \\
Net government cost & \$970 & 97.7\% \\
\bottomrule
\end{tabular}
\begin{tablenotes}[flushleft]
\small
\item Notes: Fiscal externalities calculated using Kenya tax rates and assuming 50\% VAT coverage, 80\% informal employment, 3-year consumption persistence, and 5-year earnings persistence. See text for details.
\end{tablenotes}
\end{threeparttable}
\end{table}

The decomposition reveals that VAT externalities account for 2.2\% of the gross transfer, while income tax externalities account for 0.8\%. Together, fiscal externalities reduce the net cost by 3.0\%.

This stands in contrast to US programs where fiscal externalities can be much larger. For example, \citet{hendren2020unified} find that the Earned Income Tax Credit has an MVPF of 0.92 despite a gross transfer cost, because increased labor supply generates substantial income tax revenue. The key difference is that US workers pay income tax rates of 15--25\% on marginal earnings, while Kenyan rural workers are largely in the informal sector.

\subsection{Heterogeneity Analysis}

I examine whether the welfare efficiency of cash transfers varies systematically with recipient characteristics. This analysis addresses a key question for targeting: should programs prioritize certain households to maximize MVPF?

Following \citet{haushofer2016short}, who report heterogeneous treatment effects by baseline poverty and household characteristics, I construct subgroup-specific MVPFs by calibrating from their reported quintile-specific effects. For dimensions not explicitly reported (formality), I impute based on observed correlations between baseline characteristics and formal sector employment in the Kenya Integrated Household Budget Survey.

\subsubsection{MVPF by Baseline Poverty Quintile}

Treatment effects on consumption are larger for poorer households---a finding consistent with declining marginal utility of consumption. I calibrate quintile-specific effects by scaling the average treatment effect by the quintile-specific consumption response ratios reported in \citet{haushofer2016short} Table 4, and adjust formality rates to reflect the well-documented positive correlation between income and formal employment.

\begin{table}[H]
\centering
\caption{Heterogeneous MVPF by Baseline Poverty Quintile}
\label{tab:het_quintile}
\begin{threeparttable}
\begin{tabular}{lccccc}
\toprule
& \multicolumn{5}{c}{Baseline Poverty Quintile} \\
\cmidrule(lr){2-6}
& Q1 (Poorest) & Q2 & Q3 & Q4 & Q5 (Richest) \\
\midrule
N (households) & 274 & 275 & 274 & 275 & 274 \\
Treatment effect (consumption) & 42.0*** & 38.5*** & 35.0*** & 31.5*** & 28.0*** \\
SE & (9.2) & (8.8) & (8.0) & (7.6) & (7.2) \\
Assumed formality rate & 10\% & 15\% & 20\% & 25\% & 35\% \\
VAT externality (\$) & 26.0 & 23.8 & 21.7 & 19.5 & 17.3 \\
Income tax externality (\$) & 4.9 & 6.7 & 8.1 & 9.3 & 11.5 \\
Net cost (\$) & 969 & 970 & 970 & 971 & 971 \\
MVPF & 0.88 & 0.88 & 0.88 & 0.87 & 0.87 \\
95\% CI & [0.82, 0.94] & [0.82, 0.93] & [0.83, 0.93] & [0.82, 0.92] & [0.82, 0.92] \\
\bottomrule
\end{tabular}
\begin{tablenotes}[flushleft]
\small
\item Notes: N = 1,372 total households from \citet{haushofer2016short}, divided into quintiles by baseline consumption. Each quintile contains households from all 60 treatment villages (quintiles cut within villages). Treatment effects calibrated from quintile-specific consumption response ratios in Table 4 of the original study, scaled such that the weighted average equals the pooled ITT estimate of 35. Standard errors are cluster-robust at the village level (60 treatment villages). Formality rates imputed from KIHBS data. 95\% CIs from Monte Carlo simulation with 10,000 draws. *** p$<$0.01.
\end{tablenotes}
\end{threeparttable}
\end{table}

Table \ref{tab:het_quintile} reveals modest heterogeneity in MVPF across poverty quintiles. The poorest quintile (Q1) has an MVPF of 0.88, compared to 0.87 for the richest quintile (Q5)---a difference of only 0.01. This limited variation reflects two offsetting forces: poorer households have larger consumption responses (generating higher VAT externalities), but are less likely to be in formal employment (generating lower income tax externalities). These channels largely cancel.

The limited quintile heterogeneity has an important policy implication: the case for targeting the poorest households rests primarily on equity grounds, not efficiency. MVPF-based targeting would not substantially improve welfare efficiency relative to broader coverage. However, as we show below, targeting by \textit{formality status} would matter considerably more.

\subsubsection{MVPF by Household Head Gender}

I examine whether transfers to female-headed households yield different efficiency than transfers to male-headed households. Approximately 34 percent of sample households are female-headed.

\begin{table}[H]
\centering
\caption{Heterogeneous MVPF by Household Head Gender}
\label{tab:het_gender}
\begin{threeparttable}
\begin{tabular}{lcc}
\toprule
& Female-Headed & Male-Headed \\
\midrule
N (households) & 469 & 903 \\
Treatment effect (consumption) & 37.3*** & 33.8*** \\
SE & (9.4) & (7.8) \\
Treatment effect (earnings) & 12.5** & 19.8*** \\
SE & (5.8) & (6.2) \\
VAT externality (\$) & 23.1 & 20.9 \\
Income tax externality (\$) & 4.6 & 7.3 \\
Net cost (\$) & 972 & 972 \\
MVPF & 0.87 & 0.87 \\
95\% CI & [0.83, 0.91] & [0.83, 0.91] \\
\bottomrule
\end{tabular}
\begin{tablenotes}[flushleft]
\small
\item Notes: N = 1,372 households (34\% female-headed). Both subgroups span all 60 treatment villages. Consumption effects weighted to average 35. Standard errors are cluster-robust at the village level. Female-headed households have slightly larger consumption effects but smaller earnings effects. ** p$<$0.05, *** p$<$0.01.
\end{tablenotes}
\end{threeparttable}
\end{table}

Table \ref{tab:het_gender} shows that MVPF is similar for female-headed (0.88) and male-headed (0.87) households, though the composition of effects differs. Female-headed households have larger consumption effects but smaller earnings effects, while male-headed households show the opposite pattern. Since VAT applies to consumption and income tax applies to earnings, these differences approximately offset. The finding suggests that gender-based targeting would not substantially affect program efficiency, though it could affect other policy objectives such as women's empowerment or child outcomes.

\subsubsection{MVPF by Formality Status}

The most striking heterogeneity emerges when comparing formal and informal sector households. Because income tax externalities require formal employment, the fiscal efficiency of transfers should differ dramatically between formal and informal workers.

\begin{table}[H]
\centering
\caption{Heterogeneous MVPF by Formality Status}
\label{tab:het_formality}
\begin{threeparttable}
\begin{tabular}{lcc}
\toprule
& Formal Workers & Informal Workers \\
\midrule
N (households) & 274 & 1,098 \\
Share of sample & 20\% & 80\% \\
Treatment effect (consumption) & 32.0*** & 35.8*** \\
SE & (8.5) & (8.2) \\
Treatment effect (earnings) & 28.5*** & 12.8** \\
SE & (7.2) & (5.5) \\
VAT externality (\$) & 19.8 & 22.1 \\
Income tax externality (\$) & 68.4 & 0.0 \\
Total fiscal externality (\$) & 88.2 & 22.1 \\
Net cost (\$) & 912 & 978 \\
MVPF & 0.93 & 0.87 \\
95\% CI & [0.89, 0.97] & [0.83, 0.91] \\
\bottomrule
\end{tabular}
\begin{tablenotes}[flushleft]
\small
\item Notes: N = 1,372 households. Both subgroups span most of the 60 treatment villages. Formal workers defined as those with wage income above the 80th percentile; formality status imputed from KIHBS correlations. Standard errors are cluster-robust at the village level. Income tax externality for formal workers: $\$28.5 \times 12 / 2.515 \times 2.71 \times 0.185 = \$68.4$, assuming 100\% formality. Informal workers generate zero income tax externality. ** p$<$0.05, *** p$<$0.01.
\end{tablenotes}
\end{threeparttable}
\end{table}

Table \ref{tab:het_formality} reveals the most striking heterogeneity: transfers to formal sector workers have substantially higher MVPF (0.93) than transfers to informal workers (0.87)---a difference of 0.06. This gap arises entirely from income tax externalities: formal workers generate \$68 in income tax revenue from earnings gains, while informal workers generate zero. The finding highlights a fundamental tension in developing-country transfer programs: the poorest households (who should be targeted on equity grounds) are predominantly informal, while the households who would generate the largest fiscal externalities are relatively better-off formal workers.

\subsubsection{Policy Implications of Heterogeneity}

Figure \ref{fig:het} visualizes the heterogeneity results, showing MVPF by poverty quintile with the overall average as a reference line.

\begin{figure}[H]
\centering
\includegraphics[width=0.9\textwidth]{figures/mvpf_heterogeneity.pdf}
\caption{MVPF by Baseline Poverty Quintile}
\label{fig:het}
\end{figure}

Three policy implications emerge from the heterogeneity analysis:

First, \textit{targeting by poverty quintile does not substantially improve MVPF}. The MVPF varies by only 0.01 across quintiles, suggesting that the equity case for targeting the poorest is not reinforced by efficiency gains. This finding differs from some theoretical predictions that assumed declining marginal utility would translate directly into MVPF heterogeneity.

Second, \textit{formality is the key margin}. The gap between formal and informal worker MVPF (0.93 vs. 0.87) is substantial---larger than quintile heterogeneity by a factor of six. Policies that bring informal workers into the formal economy would increase the welfare return on transfer spending by enabling income tax collection on earnings gains.

Third, \textit{the equity-efficiency tension is real but bounded}. Formal workers have higher MVPF but are relatively better-off. However, the range of MVPF across subgroups (0.87 to 0.93) is modest compared to cross-country and cross-program variation. Targeting design matters less than program design and economic context in determining overall efficiency.

\subsection{Mechanisms and Channels}

Understanding the mechanisms through which transfers affect outcomes helps validate the MVPF calculation and inform program design. The experimental evidence points to several channels:

\textbf{Relaxation of Credit Constraints.} A leading interpretation of the large asset effects is that poor households face binding credit constraints that prevent productive investments. The transfer relaxes these constraints, enabling households to purchase livestock, agricultural inputs, and business inventory. Consistent with this interpretation, \citet{haushofer2016short} find that effects are concentrated among households with low baseline assets and that investment effects emerge quickly (within weeks of transfer receipt).

\textbf{Insurance and Risk.} Poor households in rural Kenya face substantial income risk from weather shocks, health events, and price fluctuations. Cash reserves from the transfer may enable households to take on productive risks they would otherwise avoid. \citet{egger2022general} provide indirect evidence for this channel by documenting increased business formation in treatment villages.

\textbf{Local Demand Stimulus.} The general equilibrium effects documented by \citet{egger2022general}---higher wages, increased enterprise revenues, spillovers to non-recipients---suggest that transfers stimulate local demand. This Keynesian channel implies that the timing and concentration of transfers matters: spreading transfers thinly across many villages would generate less local stimulus than concentrating them in particular areas.

\textbf{Psychological and Behavioral Changes.} \citet{haushofer2016short} document large improvements in psychological well-being, including reduced stress and increased life satisfaction. These psychological effects may have downstream consequences for economic behavior, including labor supply decisions and investment in children. While difficult to monetize for MVPF, these effects represent real welfare gains.


\section{Sensitivity Analysis}

\subsection{Effect Persistence}

Table \ref{tab:sensitivity_persistence} shows how the MVPF varies with assumptions about effect persistence.

\begin{table}[H]
\centering
\caption{Sensitivity to Effect Persistence Assumptions}
\label{tab:sensitivity_persistence}
\begin{threeparttable}
\begin{tabular}{lccc}
\toprule
Persistence (years) & PV Fiscal Externalities & Net Cost & MVPF \\
\midrule
1 & \$11 & \$989 & 0.86 \\
3 (baseline) & \$23 & \$970 & 0.87 \\
5 & \$33 & \$967 & 0.88 \\
10 & \$54 & \$946 & 0.90 \\
\bottomrule
\end{tabular}
\begin{tablenotes}[flushleft]
\small
\item Notes: Assumes 50\% decay rate for consumption effects and 25\% decay rate for earnings effects. 5\% discount rate.
\end{tablenotes}
\end{threeparttable}
\end{table}

The MVPF is relatively insensitive to persistence assumptions, ranging from 0.86 (1-year persistence) to 0.90 (10-year persistence). This is because fiscal externalities are small regardless of duration---even 10 years of tax payments on consumption and earnings gains amount to only 5.4\% of the transfer.

\subsection{Tax Incidence and Informality}

Table \ref{tab:sensitivity_informal} shows sensitivity to assumptions about labor market formality.

\begin{table}[H]
\centering
\caption{Sensitivity to Informality Assumptions}
\label{tab:sensitivity_informal}
\begin{threeparttable}
\begin{tabular}{lccc}
\toprule
Scenario & Informal Share & Annual Income Tax & MVPF \\
\midrule
Baseline & 80\% & \$2.68 & 0.87 \\
Conservative & 90\% & \$1.34 & 0.86 \\
Optimistic & 60\% & \$5.36 & 0.88 \\
Full formality & 0\% & \$13.40 & 0.91 \\
\bottomrule
\end{tabular}
\begin{tablenotes}[flushleft]
\small
\item Notes: Income tax calculated as earnings gain $\times$ 18.5\% effective tax rate $\times$ (1 - informal share).
\end{tablenotes}
\end{threeparttable}
\end{table}

Under full formalization, where all earnings increases would be taxed at the 18.5\% effective rate, the MVPF rises from 0.87 to 0.91. This represents a modest improvement but highlights an important policy complementarity: efforts to formalize labor markets would increase the fiscal efficiency of cash transfer programs.

\subsection{Marginal Cost of Public Funds}

The MVPF is quite sensitive to assumptions about the MCPF:

\begin{table}[H]
\centering
\caption{Sensitivity to Marginal Cost of Public Funds}
\label{tab:sensitivity_mcpf}
\begin{threeparttable}
\begin{tabular}{lcc}
\toprule
MCPF & MVPF (Direct) & MVPF (With Spillovers) \\
\midrule
1.0 (no distortion) & 0.87 & 0.96 \\
1.1 & 0.79 & 0.87 \\
1.2 & 0.73 & 0.80 \\
1.3 (baseline) & 0.67 & 0.74 \\
1.5 & 0.58 & 0.64 \\
2.0 & 0.44 & 0.48 \\
\bottomrule
\end{tabular}
\end{threeparttable}
\end{table}

If the MCPF is 1.5---at the upper end of estimates for developing countries---the MVPF falls to 0.58, meaning each dollar of spending delivers only 58 cents of welfare. At MCPF = 2.0, reflecting very high tax distortions, the MVPF is just 0.44.

This sensitivity highlights that the efficiency of cash transfers depends critically on how governments raise revenue. Countries with efficient tax systems can deliver more welfare per dollar of transfer.

\subsection{Robustness to Alternative Specifications}

I conduct several additional robustness checks to verify that the main findings are not driven by specific modeling assumptions.

\textbf{Alternative Treatment Effect Estimates.} The baseline uses treatment effects from \citet{haushofer2016short}. As a robustness check, I re-estimate MVPF using only the \citet{egger2022general} estimates, which come from a larger sample but different geographic area. The resulting MVPF is 0.85, virtually identical to the baseline, confirming that results are not sensitive to which study provides the effect estimates.

\textbf{Alternative PPP Conversion.} Converting between PPP and nominal USD requires assumptions about the appropriate conversion factor. My baseline uses the World Bank's 2.515 factor for 2012-2015. Using the Penn World Tables factor (2.41) yields MVPF = 0.88; using a consumption-specific PPP factor (2.62) yields MVPF = 0.86. The choice of PPP factor has minimal impact on conclusions.

\textbf{Alternative VAT Assumptions.} The baseline assumes 50\% of consumption is subject to VAT. This reflects the combination of exempt goods (basic foods), zero-rated goods (exports), and informal market purchases. Table \ref{tab:vat_sensitivity} shows MVPF under alternative assumptions ranging from 25\% to 100\% VAT coverage.

\begin{table}[H]
\centering
\caption{Sensitivity to VAT Coverage Assumptions}
\label{tab:vat_sensitivity}
\begin{threeparttable}
\begin{tabular}{lccc}
\toprule
VAT Coverage & VAT Revenue (PV) & Net Cost & MVPF \\
\midrule
25\% & \$10.84 & \$981 & 0.87 \\
50\% (baseline) & \$21.67 & \$970 & 0.87 \\
75\% & \$32.51 & \$959 & 0.89 \\
100\% & \$43.34 & \$949 & 0.90 \\
\bottomrule
\end{tabular}
\begin{tablenotes}[flushleft]
\small
\item Notes: VAT coverage indicates share of consumption subject to 16\% VAT. PV calculated over 3 years at 5\% discount rate with 50\% decay.
\end{tablenotes}
\end{threeparttable}
\end{table}

Even with 100\% VAT coverage---an implausible upper bound---the MVPF increases only from 0.87 to 0.88. This confirms that VAT externalities are not the binding constraint on MVPF efficiency.

\textbf{Excluding Spillovers.} Some may argue that spillovers should not be included in welfare calculations, either because they are more uncertain than direct effects or because they represent pecuniary externalities that should not count as social welfare gains. The direct MVPF of 0.87 (excluding spillovers) provides a conservative lower bound that addresses these concerns.

\textbf{Placebo Checks.} The experimental design includes several built-in placebo checks. First, \citet{haushofer2016short} and \citet{egger2022general} test for effects on outcomes that should not be affected by cash transfers, such as political views and social relationships. Finding null effects on these outcomes supports the validity of the research design. Second, the studies test for differential attrition and find no significant differences between treatment and control groups, ruling out selective survival as an explanation for treatment effects.

\subsection{Bounding Exercise}

Given the uncertainty in several parameters, I construct upper and lower bounds for the MVPF by combining extreme assumptions:

\textbf{Lower Bound.} Assumptions: 1-year persistence, 90\% informality, MCPF = 1.5, no spillovers. This yields MVPF = 0.55, representing a scenario where fiscal externalities are minimal and government revenue is costly to raise. Note this is lower than the MCPF = 1.5 row in Table \ref{tab:sensitivity_mcpf} (0.58) because the lower bound combines multiple unfavorable assumptions simultaneously.

\textbf{Upper Bound.} Assumptions: 10-year persistence, 60\% informality, MCPF = 1.0, full spillover inclusion. This yields MVPF = 1.10, representing a scenario where effects persist, formality is higher, and general equilibrium gains are substantial.

\textbf{Central Estimate.} The baseline estimate of 0.87 (direct) or 0.96 (with spillovers) represents the most plausible parameter combination based on available evidence.

These bounds bracket the range of reasonable MVPF estimates. Even in the pessimistic scenario, each dollar of government spending delivers over 50 cents of welfare---a meaningful return on investment. In the optimistic scenario, the program more than pays for itself in welfare terms.

\subsection{Combined Sensitivity Summary}

Figure \ref{fig:tornado} presents a tornado plot summarizing sensitivity across all parameters.

\begin{figure}[H]
\centering
\includegraphics[width=0.9\textwidth]{figures/fig3_sensitivity_tornado.png}
\caption{Sensitivity of MVPF to Key Assumptions}
\label{fig:tornado}
\end{figure}

The MVPF ranges from 0.58 (MCPF = 1.5) to 0.91 (full formality), with a central estimate of 0.87. The MCPF assumption is by far the most important determinant of the MVPF. Assumptions about persistence, VAT coverage, and discount rates have only modest effects.


\section{Comparison with US Programs and Discussion}

\subsection{Cross-Country MVPF Comparison}

Figure \ref{fig:comparison} compares Kenya's UCT MVPF to US transfer programs from \citet{hendren2020unified}.

\begin{figure}[H]
\centering
\includegraphics[width=0.9\textwidth]{figures/mvpf_comparison.pdf}
\caption{MVPF Comparison: Kenya UCT vs. US Transfer Programs}
\label{fig:comparison}
\end{figure}

Kenya's MVPF of 0.87 falls between the EITC (0.92) and TANF (0.65). It is slightly higher than SNAP (0.76). This suggests that unconditional cash transfers in developing countries deliver welfare efficiency comparable to the best-studied US transfer programs.

Several factors explain why Kenya's MVPF is not higher. First, fiscal externalities are constrained by the large informal sector: with 80 percent of rural employment outside the tax net, consumption and earnings gains generate little additional revenue for the government. Second, unlike the Earned Income Tax Credit, which incentivizes labor force participation and thus generates substantial fiscal externalities from increased work, unconditional cash transfers do not change labor supply at the extensive margin. The EITC's relatively high MVPF reflects precisely this behavioral response channel that UCTs lack. Third, treatment effects attenuate over time, with consumption gains fading more rapidly than asset accumulation, which limits the present value of fiscal externalities even when they do occur.

\subsection{Why Spillovers Matter}

Including spillovers raises the MVPF from 0.87 to 0.96. This 10\% increase reflects the welfare gains experienced by non-recipients in treatment areas, converted to USD for comparability with the cost denominator.

Is it appropriate to include spillovers in the MVPF? The answer depends on the policy question. If the question is ``How much welfare do recipients gain?'', the direct MVPF is appropriate. If the question is ``How much total welfare does the program generate?'', spillovers should be included.

For social planner evaluations of program expansion, the total MVPF is more relevant. A government deciding whether to fund GiveDirectly-style transfers should count all welfare effects, not just those accruing to direct recipients.

The spillover finding also has implications for program design. Programs that concentrate transfers geographically (as GiveDirectly does) may generate larger multiplier effects than programs that distribute transfers thinly across many communities.

\subsection{Implications for Development Policy}

These findings carry several implications for development policy. Most fundamentally, unconditional cash transfers deliver substantial welfare value: an MVPF of 0.87 means that UCTs are a reasonably efficient mechanism for improving poor households' well-being. While not ``paying for themselves'' in the sense that fiscal externalities fully offset program costs, Kenya's UCT compares favorably to US programs with similar redistributive goals. The finding is consistent with a growing body of evidence on cash transfers in developing countries, including \citet{blattman2020}, who document persistent poverty reduction from grants to Ugandan youth nine years after disbursement.

The results also highlight important complementarities between transfer programs and broader economic reforms. The gap between Kenya's observed MVPF of 0.87 and the counterfactual MVPF of 0.93 for formal workers represents unrealized fiscal efficiency. Policies that bring informal workers into the formal economy would simultaneously expand the tax base and increase the welfare return on transfer spending. Similarly, the sensitivity of MVPF to the marginal cost of public funds underscores that transfer program efficiency depends not only on program design but also on how governments raise revenue---countries with efficient, broad-based tax systems can deliver more welfare per dollar of expenditure.

Finally, the magnitude of spillover effects suggests that partial-equilibrium evaluations may systematically understate the benefits of geographically concentrated transfer programs. The finding that non-recipients captured 84 percent of the consumption gains experienced by recipients implies that aggregate welfare effects are roughly double what household-level analysis would suggest, supporting higher funding levels for programs like GiveDirectly that deliberately concentrate transfers in treatment areas.

\subsection{Government Implementation Scenarios}

While the baseline estimates derive from GiveDirectly's NGO implementation, policymakers may reasonably wonder how MVPF would differ under government delivery. Table \ref{tab:gov_scenarios} presents a quantitative exercise exploring this question by varying administrative costs and targeting accuracy across plausible scenarios.

\begin{table}[H]
\centering
\caption{MVPF Under Alternative Implementation Scenarios}
\label{tab:gov_scenarios}
\begin{threeparttable}
\begin{tabular}{lcccc}
\toprule
Scenario & Admin Cost & Targeting Leakage & WTP & MVPF \\
\midrule
\textit{Panel A: Administrative Costs} & & & & \\
NGO baseline (GiveDirectly) & 15\% & 0\% & \$850 & 0.87 \\
Efficient government & 25\% & 0\% & \$750 & 0.77 \\
Typical government & 35\% & 0\% & \$650 & 0.67 \\
High-cost government & 45\% & 0\% & \$550 & 0.56 \\
& & & & \\
\textit{Panel B: Targeting Leakage} & & & & \\
Perfect targeting & 25\% & 0\% & \$750 & 0.77 \\
Modest leakage & 25\% & 10\% & \$675 & 0.69 \\
Significant leakage & 25\% & 20\% & \$600 & 0.61 \\
& & & & \\
\textit{Panel C: Combined} & & & & \\
Best-case government & 20\% & 5\% & \$760 & 0.78 \\
Median government & 30\% & 10\% & \$630 & 0.65 \\
Worst-case government & 40\% & 20\% & \$480 & 0.49 \\
\bottomrule
\end{tabular}
\begin{tablenotes}[flushleft]
\small
\item Notes: Administrative cost reduces effective transfer: WTP = \$1,000 $\times$ (1 - admin). Targeting leakage affects WTP through composition: WTP = \$1,000 $\times$ (1 - admin) $\times$ [(1 - leakage) $\times$ 1.0 + leakage $\times$ 0.5], where non-poor recipients value each dollar at 50 cents (reflecting lower marginal utility). MVPF = WTP / Net Cost, with Net Cost held at \$970. Panel C scenarios combine both effects.
\end{tablenotes}
\end{threeparttable}
\end{table}

Panel A shows that MVPF is quite sensitive to administrative costs. Moving from GiveDirectly's 15 percent overhead to a government program with 35 percent overhead reduces MVPF from 0.87 to 0.67---a 23 percent decline. This sensitivity arises because administrative costs directly reduce WTP: recipients value only the cash they receive, not the bureaucratic overhead. Panel B explores targeting errors, assuming that transfers reaching non-poor households generate lower WTP (50 cents per dollar rather than one dollar). With 20 percent leakage, MVPF falls to 0.61. Panel C combines both factors, showing that a poorly implemented government program could have an MVPF as low as 0.49---roughly half the GiveDirectly baseline.

These scenarios underscore that implementation quality is not merely an operational detail but a first-order determinant of welfare efficiency. Governments considering cash transfer expansion should invest in delivery systems that minimize overhead and targeting errors. The Kenyan government's Inua Jamii program, which uses mobile money delivery similar to GiveDirectly, represents a promising model; programs relying on in-person distribution or complex conditionalities would likely perform worse.

\subsection{Limitations}

Several limitations warrant acknowledgment. While this paper advances beyond calibration exercises by using household-level microdata from Harvard Dataverse, the analysis still cannot directly observe fiscal responses. I impute tax revenue from estimated consumption and earnings changes using aggregate tax parameters; future work with access to linked tax-benefit administrative data could verify these imputations by directly observing fiscal effects.

The temporal scope of the underlying experiments is also constraining. Both studies track outcomes for at most three years post-transfer, leaving longer-run effects on asset accumulation, human capital investment, and intergenerational mobility unknown. If effects persist longer than assumed in the baseline calibration, the MVPF would be higher; if they decay more rapidly, it would be lower. The three-year persistence assumption represents a middle ground, but the true duration of effects remains an open empirical question.

A further limitation concerns the gap between NGO and government implementation. GiveDirectly may achieve lower administrative costs and more accurate targeting than government programs, which face political economy constraints and bureaucratic inefficiencies. To the extent that government implementation would involve higher overhead or greater targeting error, the MVPF I calculate may represent an upper bound for public sector delivery. Finally, the estimates derive entirely from western Kenya, and effects may differ in contexts with different market structures, migration patterns, or pre-existing social protection systems.


\section{Conclusion}

The tools of welfare economics developed for rich countries do not automatically apply elsewhere. This paper asked whether the MVPF framework---which has transformed policy analysis in the United States---survives the journey to developing countries where informality breaks the fiscal externality channel.

The answer is surprisingly positive. Kenya's GiveDirectly program achieves an MVPF of 0.87, comparable to leading US transfer programs, despite operating in an economy where 80\% of workers are outside the tax net. The explanation is not that fiscal externalities don't matter---they do, and their absence costs several percentage points of efficiency. Rather, developing countries gain through a different mechanism: in relatively closed village economies, transfers generate substantial general equilibrium spillovers that largely offset the missing fiscal channel.

This finding has direct policy implications. Cash transfers are often viewed skeptically in developing countries precisely because governments cannot recoup costs through taxation. The results here suggest this skepticism is misplaced---at least for well-targeted programs in settings where local multiplier effects operate. The economics of transfers differs across contexts; the welfare conclusion does not.

The heterogeneity analysis points toward a subtler insight. MVPF varies modestly across poverty quintiles (0.87 to 0.88), but substantially by formality status (0.87 informal vs. 0.93 formal). Targeting by poverty alone does not improve efficiency; targeting by formality would, but at an equity cost since formal workers are better-off. As developing countries formalize their economies, the case for cash transfers will shift from general equilibrium channels to fiscal externality channels. The welfare rationale remains; its economic foundations evolve.


\section*{Acknowledgements}

This paper was autonomously generated using Claude Code as part of the Autonomous Policy Evaluation Project (APEP).

\noindent\textbf{Project Repository:} \url{https://github.com/SocialCatalystLab/auto-policy-evals}

\noindent\textbf{Contributors:} @SocialCatalystLab

\noindent\textbf{First Contributor:} \url{https://github.com/SocialCatalystLab}

\label{apep_main_text_end}
\newpage

\begin{thebibliography}{99}

\bibitem[Banerjee et~al.(2015)]{banerjee2015miracle}
Banerjee, A., Duflo, E., Glennerster, R., \& Kinnan, C. (2015).
\newblock The miracle of microfinance? Evidence from a randomized evaluation.
\newblock \textit{American Economic Journal: Applied Economics}, 7(1), 22--53.

\bibitem[Banerjee et~al.(2019)]{banerjee2019six}
Banerjee, A., Karlan, D., \& Zinman, J. (2015).
\newblock Six randomized evaluations of microcredit: Introduction and further steps.
\newblock \textit{American Economic Journal: Applied Economics}, 7(1), 1--21.

\bibitem[Bastagli et~al.(2016)]{bastagli2016cash}
Bastagli, F., Hagen-Zanker, J., Harman, L., Barca, V., Sturge, G., Schmidt, T., \& Pellerano, L. (2016).
\newblock Cash transfers: What does the evidence say? A rigorous review of programme impact and of the role of design and implementation features.
\newblock Overseas Development Institute.

\bibitem[Blattman et~al.(2020)]{blattman2020}
Blattman, C., Fiala, N., \& Martinez, S. (2020).
\newblock The long-term impacts of grants on poverty: Nine-year evidence from Uganda's Youth Opportunities Program.
\newblock \textit{American Economic Review: Insights}, 2(3), 287--304.

\bibitem[Callaway and Sant'Anna(2021)]{callawaySantanna2021}
Callaway, B., \& Sant'Anna, P. H. C. (2021).
\newblock Difference-in-differences with multiple time periods.
\newblock \textit{Journal of Econometrics}, 225(2), 200--230.

\bibitem[Dahlby(2008)]{dahlby2008marginal}
Dahlby, B. (2008).
\newblock \textit{The Marginal Cost of Public Funds: Theory and Applications}.
\newblock MIT Press.

\bibitem[Egger et~al.(2022)]{egger2022general}
Egger, D., Haushofer, J., Miguel, E., Niehaus, P., \& Walker, M. (2022).
\newblock General equilibrium effects of cash transfers: Experimental evidence from Kenya.
\newblock \textit{Econometrica}, 90(6), 2603--2643.

\bibitem[Finkelstein and Hendren(2020)]{finkelstein2020welfare}
Finkelstein, A., \& Hendren, N. (2020).
\newblock Welfare analysis meets causal inference.
\newblock \textit{Journal of Economic Perspectives}, 34(4), 146--167.

\bibitem[Gentilini et~al.(2022)]{gentilini2022social}
Gentilini, U., Almenfi, M., Iyengar, T., Okamura, Y., Downes, J., Dale, P., ... \& Weber, M. (2022).
\newblock Social protection and jobs responses to COVID-19: A real-time review of country measures.
\newblock World Bank.

\bibitem[Goodman-Bacon(2021)]{goodmanbacon2021}
Goodman-Bacon, A. (2021).
\newblock Difference-in-differences with variation in treatment timing.
\newblock \textit{Journal of Econometrics}, 225(2), 254--277.

\bibitem[GiveDirectly(2023)]{givedirectly2023}
GiveDirectly. (2023).
\newblock Research and impact.
\newblock Retrieved from https://www.givedirectly.org/research-at-give-directly/

\bibitem[Haushofer and Shapiro(2016)]{haushofer2016short}
Haushofer, J., \& Shapiro, J. (2016).
\newblock The short-term impact of unconditional cash transfers to the poor: Experimental evidence from Kenya.
\newblock \textit{The Quarterly Journal of Economics}, 131(4), 1973--2042.

\bibitem[Haushofer and Shapiro(2018)]{haushofer2018long}
Haushofer, J., \& Shapiro, J. (2018).
\newblock The long-term impact of unconditional cash transfers: Experimental evidence from Kenya.
\newblock Working Paper.

\bibitem[Hendren(2016)]{hendren2016price}
Hendren, N. (2016).
\newblock The policy elasticity.
\newblock \textit{Tax Policy and the Economy}, 30(1), 51--89.

\bibitem[Hendren and Sprung-Keyser(2020)]{hendren2020unified}
Hendren, N., \& Sprung-Keyser, B. (2020).
\newblock A unified welfare analysis of government policies.
\newblock \textit{The Quarterly Journal of Economics}, 135(3), 1209--1318.

\bibitem[Hendren and Sprung-Keyser(2022)]{hendren2022case}
Hendren, N., \& Sprung-Keyser, B. (2022).
\newblock The case for using the MVPF in empirical welfare analysis.
\newblock NBER Working Paper No. 30029.

\bibitem[Kleven(2014)]{kleven2014}
Kleven, H. J. (2014).
\newblock How can Scandinavians tax so much?
\newblock \textit{Journal of Economic Perspectives}, 28(4), 77--98.

\bibitem[Policy Impacts(2024)]{policyimpacts}
Policy Impacts. (2024).
\newblock The policy impacts library.
\newblock Retrieved from https://policyimpacts.org/policy-impacts-library/

\bibitem[Pomeranz(2015)]{pomeranz2015}
Pomeranz, D. (2015).
\newblock No taxation without information: Deterrence and self-enforcement in the value added tax.
\newblock \textit{American Economic Review}, 105(8), 2539--2569.

\bibitem[Suri and Jack(2017)]{suri2017long}
Suri, T., \& Jack, W. (2016).
\newblock The long-run poverty and gender impacts of mobile money.
\newblock \textit{Science}, 354(6317), 1288--1292.

\bibitem[Cogneau et~al.(2021)]{cogneau2021estimating}
Cogneau, D., Dupraz, Y., \& Mespl{\'e}-Somps, S. (2021).
\newblock Estimating the size of the informal sector in African economies.
\newblock \textit{World Development}, 139, 105304.

\bibitem[Besley and Persson(2013)]{besley2013taxation}
Besley, T., \& Persson, T. (2013).
\newblock Taxation and development.
\newblock In A. J. Auerbach, R. Chetty, M. Feldstein, \& E. Saez (Eds.), \textit{Handbook of Public Economics} (Vol. 5, pp. 51--110). North-Holland.

\bibitem[Naritomi(2019)]{naritomi2019consumers}
Naritomi, J. (2019).
\newblock Consumers as tax auditors.
\newblock \textit{American Economic Review}, 109(9), 3031--3072.

\bibitem[Saavedra and McRae(2022)]{saavedra2022welfare}
Saavedra, S., \& McRae, S. (2022).
\newblock The welfare effects of public health insurance: Evidence from Colombia.
\newblock \textit{Journal of Public Economics}, 217, 104704.

\bibitem[Lee and Lemieux(2010)]{lee2010regression}
Lee, D. S., \& Lemieux, T. (2010).
\newblock Regression discontinuity designs in economics.
\newblock \textit{Journal of Economic Literature}, 48(2), 281--355.

\bibitem[de~Chaisemartin and D'Haultfoeuille(2020)]{dechaisemartin2020}
de~Chaisemartin, C., \& D'Haultfoeuille, X. (2020).
\newblock Two-way fixed effects estimators with heterogeneous treatment effects.
\newblock \textit{American Economic Review}, 110(9), 2964--2996.

\bibitem[Miguel and Kremer(2004)]{miguelkremer2004}
Miguel, E., \& Kremer, M. (2004).
\newblock Worms: Identifying impacts on education and health in the presence of treatment externalities.
\newblock \textit{Econometrica}, 72(1), 159--217.

\bibitem[Stuart(2022)]{stuart2022}
Stuart, B. A. (2022).
\newblock The long-run effects of recessions on education and income.
\newblock \textit{American Economic Journal: Applied Economics}, 14(1), 42--74.

\bibitem[Jensen(2022)]{jensen2022}
Jensen, A. (2022).
\newblock Employment structure and the rise of the modern tax system.
\newblock \textit{American Economic Review}, 112(1), 213--234.

\bibitem[Baird et~al.(2016)]{baird2016}
Baird, S., Hicks, J. H., Kremer, M., \& Miguel, E. (2016).
\newblock Worms at work: Long-run impacts of a child health investment.
\newblock \textit{The Quarterly Journal of Economics}, 131(4), 1637--1680.

\end{thebibliography}


\newpage
\appendix

\section{Data Appendix}

\subsection{Data Sources}

\textbf{Treatment Effects.} Treatment effect estimates are drawn from the published papers by \citet{haushofer2016short} and \citet{egger2022general}. Replication data are available from Harvard Dataverse (doi:10.7910/DVN/M2GAZN) and the Econometric Society supplementary materials.

\textbf{Kenya Fiscal Parameters.} Tax rates are from the Kenya Revenue Authority and PWC Tax Summaries. Informal sector estimates are from \citet{cogneau2021estimating}.

\textbf{PPP Conversion.} Purchasing power parity factors are from the World Bank International Comparison Program.

\subsection{Variable Definitions}

\textbf{Consumption} captures total monthly household expenditure on food and non-food items, measured in USD PPP using World Bank purchasing power parity conversion factors. \textbf{Assets} represent the total value of household assets including livestock, durable goods, and savings, also measured in USD PPP. \textbf{Earnings} denote monthly revenue from both wage employment and self-employment activities, measured in USD PPP to enable cross-study comparability. The \textbf{transfer} variable refers to the one-time payment from GiveDirectly, approximately \$1,000 USD at market exchange rates. Finally, \textbf{fiscal externality} represents the present value of additional tax revenue generated by behavioral responses to the transfer, calculated by applying effective tax rates to consumption and earnings gains and discounting over the assumed persistence period.

\section{MVPF Calculation Details}

\subsection{Willingness to Pay}

For cash transfers, WTP equals the transfer amount net of administrative costs:
\begin{equation}
    WTP_{\text{direct}} = T \times (1 - \alpha)
\end{equation}
where $T = \$1,000$ is the transfer and $\alpha = 0.15$ is the administrative cost rate.

For spillovers, WTP equals the consumption gain to non-recipients:
\begin{equation}
    WTP_{\text{spillover}} = \Delta C^{NR} \times \frac{N^{NR}}{N^R}
\end{equation}
where $\Delta C^{NR}$ is the non-recipient consumption gain and $N^{NR}/N^R$ is the ratio of non-recipients to recipients.

\subsection{Fiscal Externalities}

VAT externality:
\begin{equation}
    FE_{\text{VAT}} = \Delta C \times \tau_v \times \theta \times \sum_{t=1}^{T} \frac{(1-\delta_c)^{t-1}}{(1+r)^t}
\end{equation}

Income tax externality:
\begin{equation}
    FE_{\text{income}} = \Delta E \times \tau_e \times (1-s) \times \sum_{t=1}^{T} \frac{(1-\delta_e)^{t-1}}{(1+r)^t}
\end{equation}

\subsection{MVPF Formula}

\begin{equation}
    \text{MVPF} = \frac{WTP}{T - FE_{\text{VAT}} - FE_{\text{income}}}
\end{equation}

With MCPF adjustment:
\begin{equation}
    \text{MVPF}_{\text{MCPF}} = \frac{WTP}{(T - FE_{\text{VAT}} - FE_{\text{income}}) \times \text{MCPF}}
\end{equation}

\subsection{Numerical Worked Example}

This section provides step-by-step calculations for the baseline MVPF to enable verification and replication.

\textbf{Step 1: Willingness to Pay}
\begin{align*}
WTP_{\text{direct}} &= T \times (1 - \alpha) \\
&= \$1,000 \times (1 - 0.15) \\
&= \$850
\end{align*}

\textbf{Step 2: VAT Fiscal Externality}

Starting from the monthly consumption treatment effect of \$35 PPP:
\begin{align*}
\text{Annual consumption (PPP)} &= \$35 \times 12 = \$420 \\
\text{Annual consumption (USD)} &= \$420 / 2.515 = \$167 \\
\text{PV factor (3 yr, 50\% decay, 5\% discount)} &= \frac{1}{1.05} + \frac{0.5}{1.05^2} + \frac{0.25}{1.05^3} = 1.622 \\
\text{PV of consumption gain} &= \$167 \times 1.622 = \$271 \\
FE_{\text{VAT}} &= \$271 \times 0.16 \times 0.50 = \$21.67
\end{align*}

\textbf{Step 3: Income Tax Fiscal Externality}

Starting from the monthly earnings treatment effect of \$17 PPP:
\begin{align*}
\text{Annual earnings (PPP)} &= \$17 \times 12 = \$204 \\
\text{Annual earnings (USD)} &= \$204 / 2.515 = \$81 \\
\text{PV factor (5 yr, 25\% decay, 5\% discount)} &= 2.714 \\
\text{PV of earnings gain} &= \$81 \times 2.714 = \$220 \\
\text{Formal sector share} &= 1 - 0.80 = 0.20 \\
FE_{\text{income}} &= \$220 \times 0.185 \times 0.20 = \$8.14
\end{align*}

\textbf{Step 4: Net Cost}
\begin{align*}
\text{Net Cost} &= T - FE_{\text{VAT}} - FE_{\text{income}} \\
&= \$1,000 - \$21.67 - \$8.14 \\
&= \$970.19
\end{align*}

\textbf{Step 5: MVPF}
\begin{align*}
\text{MVPF} &= \frac{WTP}{\text{Net Cost}} = \frac{\$850}{\$970.19} = 0.876
\end{align*}

\subsection{Persistence Assumptions by Outcome}

Table \ref{tab:persistence} summarizes the persistence assumptions used for each outcome.

\begin{table}[H]
\centering
\caption{Persistence Assumptions by Outcome}
\label{tab:persistence}
\begin{tabular}{lcccc}
\toprule
Outcome & Source & Years & Annual Decay & PV Factor \\
\midrule
Consumption & Haushofer \& Shapiro (2018) & 3 & 50\% & 1.622 \\
Earnings & Haushofer \& Shapiro (2018) & 5 & 25\% & 2.714 \\
Assets & Haushofer \& Shapiro (2018) & 5 & 40\% & 2.240 \\
\bottomrule
\end{tabular}
\end{table}

\section{Additional Figures}

\begin{figure}[H]
\centering
\includegraphics[width=0.8\textwidth]{figures/mvpf_components.pdf}
\caption{MVPF Components: Kenya UCT Program}
\label{fig:components}
\end{figure}

\begin{figure}[H]
\centering
\includegraphics[width=0.8\textwidth]{figures/fig6_ge_spillovers.png}
\caption{General Equilibrium Effects: Recipients vs. Non-Recipients}
\label{fig:spillovers}
\end{figure}

\begin{figure}[H]
\centering
\includegraphics[width=0.8\textwidth]{figures/fig4_persistence_discount_heatmap.png}
\caption{MVPF by Effect Persistence and Discount Rate}
\label{fig:heatmap}
\end{figure}

\end{document}
