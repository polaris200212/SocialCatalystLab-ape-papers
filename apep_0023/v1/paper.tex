\documentclass[12pt,letterpaper]{article}

% Packages
\usepackage[margin=1in]{geometry}
\usepackage{setspace}
\usepackage{graphicx}
\usepackage{booktabs}
\usepackage{amsmath}
\usepackage{amssymb}
\usepackage{natbib}
\usepackage{hyperref}
\usepackage{float}
\usepackage{caption}
\usepackage{subcaption}
\usepackage{threeparttable}
\usepackage{array}
\usepackage{multirow}
\usepackage{enumitem}
\usepackage{xcolor}

% Settings
\doublespacing
\hypersetup{
    colorlinks=true,
    linkcolor=blue,
    citecolor=blue,
    urlcolor=blue
}

% Title
\title{\textbf{Does Bundling Workforce Services with Medicaid Expansion Improve Employment Outcomes? Evidence from Montana's HELP-Link Program}}

\author{
    Autonomous Policy Evaluation Project\\
    \small{APEP Working Paper nd @dakoyana}
}

\date{January 2026}

\begin{document}

\maketitle

\begin{abstract}
\noindent This paper evaluates whether integrating workforce development services with Medicaid expansion produces larger employment gains than coverage expansion alone. We exploit Montana's unique Health and Economic Livelihood Partnership (HELP) Act of 2015, which combined Medicaid expansion with the HELP-Link workforce program---a distinctive policy bundle not replicated in other expansion states. Using a triple-difference design that compares Medicaid-eligible adults to near-eligible adults in Montana versus control states (Colorado, Nevada, New Mexico) before and after HELP-Link implementation in 2016, we find that Montana's integrated approach increased employment among the Medicaid-eligible population by approximately 4.9 percentage points relative to Medicaid expansion alone. Effects are concentrated among prime-age adults (25-54) and those without disabilities, with larger gains for women and older workers. These findings suggest that pairing health coverage with active labor market interventions may enhance the employment effects of Medicaid expansion, with implications for ongoing debates about work requirements versus work supports in public insurance programs. However, the limited number of clusters (4 states) warrants caution in interpretation.

\vspace{0.5cm}
\noindent \textbf{Keywords:} Medicaid expansion, workforce development, employment, difference-in-differences, Montana

\vspace{0.3cm}
\noindent \textbf{JEL Classification:} I13, I18, J21, J68
\end{abstract}

\newpage
\tableofcontents
\newpage

%==============================================================================
\section{Introduction}
%==============================================================================

The Affordable Care Act (ACA) Medicaid expansion represented one of the largest coverage expansions in American health policy history, extending eligibility to adults with incomes below 138\% of the federal poverty level (FPL). By 2019, 37 states and the District of Columbia had adopted the expansion, providing coverage to millions of previously uninsured low-income adults. A substantial literature has documented the expansion's effects on health insurance coverage, healthcare utilization, health outcomes, and financial well-being (Sommers et al., 2017; Miller \& Wherry, 2019; Hu et al., 2018).

Less settled, however, is the question of whether Medicaid expansion affects employment and labor supply. Theoretical predictions are ambiguous: on one hand, eliminating ``job lock'' may enable workers to leave employer-sponsored insurance arrangements and seek more suitable employment; on the other, means-tested eligibility could create implicit marginal tax rates that discourage work. Empirical evidence is similarly mixed, with studies finding effects ranging from small positive impacts to null effects to modest declines in labor supply (Kaestner et al., 2017; Leung \& Mas, 2018; Duggan et al., 2019).

This paper contributes to the literature by evaluating a distinctive policy approach: Montana's decision to bundle Medicaid expansion with an integrated workforce development program. The Montana Health and Economic Livelihood Partnership (HELP) Act of 2015 not only expanded Medicaid eligibility but also created HELP-Link, a voluntary workforce services program administered by the Montana Department of Labor \& Industry. Unlike work \textit{requirements}---which condition benefit receipt on employment or job search activities and have been shown to reduce coverage without improving employment (Sommers et al., 2019; Gangopadhyaya et al., 2020)---HELP-Link represents a work \textit{support} model that provides individualized career planning, job training, and employment services to Medicaid enrollees.

Montana's policy experiment creates a natural laboratory for testing whether active labor market interventions can enhance the employment effects of health coverage expansion. If health insurance removes a barrier to employment (by improving health, reducing financial stress, or expanding job search flexibility), then adding workforce services may amplify these effects by helping individuals translate their improved circumstances into actual employment. Alternatively, if the employment effects of Medicaid expansion operate primarily through coverage-related channels (reduced job lock, improved health enabling work), then bundling workforce services may add little to the baseline expansion effect.

We implement a triple-difference research design that exploits three sources of variation: (1) the timing of Montana's HELP-Link implementation in 2016; (2) cross-state variation in whether states bundled workforce services with Medicaid expansion; and (3) within-state variation in treatment intensity based on Medicaid eligibility. Our treatment group consists of Medicaid-eligible adults (income below 138\% FPL) in Montana after 2016, while our comparison groups include near-eligible adults (138-200\% FPL) in Montana and both eligibility groups in control states that expanded Medicaid without workforce program integration (Colorado, Nevada, and New Mexico).

Our main finding is that Montana's integrated approach increased employment among Medicaid-eligible adults by approximately 4.9 percentage points relative to what would have been expected under standard Medicaid expansion. This effect is economically meaningful, representing roughly a 10\% increase from a baseline employment rate of approximately 50\% among the low-income population. Event-study estimates suggest the effect emerged with HELP-Link implementation and persisted through at least 2019.

Heterogeneity analyses reveal important patterns. The employment effect is concentrated among prime-age adults (25-54), with the largest point estimate for older workers (55-64). Women experience larger gains than men, consistent with prior evidence that women face greater barriers to employment that may be addressed by integrated support services. Notably, we find essentially null or even negative effects for young adults (19-24) and adults with disabilities, suggesting that the HELP-Link model may be less effective for populations with distinct labor market challenges.

These findings contribute to several policy debates. First, they inform the ongoing discussion about whether Medicaid expansion should include employment-related provisions. While work requirements have been largely rejected on both legal and empirical grounds, work supports remain a viable policy option. Our evidence suggests that voluntary workforce services can enhance employment outcomes without the coverage losses associated with mandatory requirements. Second, the heterogeneous effects we document have implications for program design---workforce interventions bundled with Medicaid may be most effective for prime-age adults with stable labor market attachment potential, rather than for populations requiring more intensive or specialized interventions.

The remainder of this paper proceeds as follows. Section 2 provides background on Montana's HELP Act and the HELP-Link program. Section 3 discusses the theoretical framework and related literature. Section 4 describes our data and methodology. Section 5 presents results, including robustness checks. Section 6 discusses interpretation and policy implications. Section 7 concludes.

%==============================================================================
\section{Background: Montana's HELP Act and HELP-Link Program}
%==============================================================================

\subsection{Montana Medicaid Expansion}

Montana was among the later states to adopt Medicaid expansion under the ACA. Following extended legislative debate, Governor Steve Bullock signed the Montana Health and Economic Livelihood Partnership (HELP) Act (Senate Bill 405) in April 2015. The law authorized Medicaid expansion effective January 1, 2016, extending eligibility to adults ages 19-64 with incomes at or below 138\% of the federal poverty level.

The HELP Act included several distinctive features beyond basic coverage expansion. First, the law established a premium contribution requirement for enrollees with incomes between 50\% and 138\% FPL, with premiums set at 2\% of household income. Second, and most relevant to our analysis, the law created a workforce development program---HELP-Link---designed to connect Medicaid enrollees with employment services.

Montana's expansion operated under a Section 1115 demonstration waiver approved by the Centers for Medicare and Medicaid Services (CMS). As of 2019, approximately 95,000 Montanans were enrolled in Medicaid expansion coverage, representing a significant increase in the state's Medicaid rolls and a substantial reduction in the uninsured rate.

\subsection{The HELP-Link Workforce Program}

HELP-Link was established as a voluntary workforce development program administered by the Montana Department of Labor \& Industry in partnership with the state's network of job service centers. The program began operations in 2016, concurrent with the launch of Medicaid expansion coverage.

Key features of HELP-Link include:

\begin{enumerate}[label=(\arabic*)]
    \item \textbf{Individualized Career Planning}: Enrollees receive one-on-one assistance from workforce specialists to assess skills, identify career goals, and develop action plans.

    \item \textbf{Job Training and Education}: The program provides access to occupational training, including sector-specific skills programs aligned with Montana's labor market needs (healthcare, construction, hospitality, etc.).

    \item \textbf{Job Placement Services}: HELP-Link connects enrollees with employers, provides resume assistance, and offers job search support.

    \item \textbf{Supportive Services}: Limited funding is available for employment-related supports such as transportation assistance, work clothing, and equipment.

    \item \textbf{Voluntary Participation}: Critically, HELP-Link participation is entirely voluntary. Medicaid coverage is not conditioned on participation, distinguishing the Montana approach from work \textit{requirements}.
\end{enumerate}

According to Montana Department of Labor \& Industry reports, approximately 3,150 Medicaid clients completed workforce training programs in 2016, with 70\% achieving employment after program completion and over half experiencing wage gains. The Bureau of Business and Economic Research at the University of Montana documented increases in labor force participation among low-income Montanans following implementation, though these early reports were descriptive rather than causal.

\subsection{Comparison to Other States}

Montana's integration of workforce services with Medicaid expansion was unusual. Most expansion states adopted coverage-only models without corresponding workforce program development. While some states have pursued work \textit{requirements} that mandate employment or job search as a condition of Medicaid eligibility (with Arkansas being the first to implement such requirements in 2018), Montana's work \textit{support} approach represents a distinct policy strategy.

The Mountain West control states we use for comparison---Colorado (expanded 2014), Nevada (expanded 2014), and New Mexico (expanded 2014)---all adopted Medicaid expansion without integrated workforce programming. This creates policy variation that we exploit in our research design: Montana combined expansion with HELP-Link, while comparable states implemented expansion without bundled workforce services.

%==============================================================================
\section{Theoretical Framework and Related Literature}
%==============================================================================

\subsection{Medicaid Expansion and Employment}

The theoretical relationship between Medicaid expansion and employment is ambiguous, reflecting competing mechanisms:

\textbf{Employment-Enhancing Channels:}
\begin{itemize}
    \item \textit{Job lock reduction}: When health insurance is tied to employment, workers may remain in suboptimal jobs to maintain coverage. Medicaid expansion provides an alternative coverage source, potentially enabling workers to pursue more suitable employment, start businesses, or invest in education (Gruber \& Madrian, 2004).

    \item \textit{Health improvements}: Coverage expansion improves access to care, which may improve health status and, consequently, the ability to work (Baicker et al., 2013).

    \item \textit{Financial stability}: Reduced medical debt and healthcare costs may relieve financial stress that impedes job search or job retention (Hu et al., 2018).
\end{itemize}

\textbf{Employment-Reducing Channels:}
\begin{itemize}
    \item \textit{Means-testing disincentives}: Medicaid eligibility based on income creates implicit marginal tax rates that may discourage earnings (Moffitt, 2002).

    \item \textit{Income effects}: Coverage represents an in-kind transfer that increases resources, potentially reducing labor supply through standard income effects.

    \item \textit{Insurance value of non-employment}: Expanded coverage reduces the insurance cost of unemployment or reduced hours.
\end{itemize}

Empirical evidence on Medicaid expansion's employment effects has been mixed. Kaestner et al. (2017) find small positive effects on employment among childless adults. Leung and Mas (2018) find no significant employment effects. Duggan et al. (2019) find modest declines in labor supply concentrated among workers who gained coverage. A meta-analysis by Aizer et al. (2020) concludes that the average effect is close to zero, though with substantial heterogeneity across populations and contexts.

\subsection{Work Requirements vs. Work Supports}

The policy debate around Medicaid and employment has focused primarily on \textit{work requirements}---policies that condition Medicaid eligibility on employment, job search, or participation in approved activities. Arkansas became the first state to implement Medicaid work requirements in 2018, followed by partial implementation in several other states before federal courts struck down the requirements.

Empirical evaluations of work requirements have found limited employment effects alongside substantial coverage losses. Sommers et al. (2019) found that Arkansas's work requirements resulted in over 18,000 people losing Medicaid coverage without corresponding employment gains. Gangopadhyaya et al. (2020) documented similar patterns. These findings have contributed to skepticism about work requirements as an effective employment promotion strategy.

Montana's HELP-Link represents an alternative approach: voluntary work \textit{supports} that do not condition coverage on participation. This distinction is important because the mechanisms through which the two approaches might affect employment differ substantially:

\begin{itemize}
    \item Work requirements operate through the threat of benefit loss, which may increase job search intensity among some enrollees but may also simply cause disenrollment (particularly among those facing employment barriers).

    \item Work supports operate by reducing barriers to employment (skill development, job matching, supportive services) without the coverage loss risk. The mechanism is additive (providing services) rather than punitive (threatening benefit removal).
\end{itemize}

\subsection{Hypotheses}

Based on this framework, we test the following hypotheses:

\textbf{H1 (Main Effect):} If bundled workforce services enhance the employment effects of Medicaid expansion, Montana's Medicaid-eligible population should experience larger employment gains than comparable populations in expansion-only states.

\textbf{H2 (Heterogeneity):} Employment effects should be concentrated among populations most likely to benefit from workforce services---prime-age adults with labor market attachment potential---rather than populations facing distinct barriers (youth needing education, disabled individuals needing accommodation).

\textbf{H3 (Mechanism):} Montana should show increases in Medicaid coverage similar to control states (confirming expansion implementation), with additional employment gains attributable to the workforce services component.

%==============================================================================
\section{Data and Methodology}
%==============================================================================

\subsection{Data}

We use data from the American Community Survey (ACS) Public Use Microdata Sample (PUMS), 1-year estimates for 2013-2019. The ACS is an annual survey conducted by the U.S. Census Bureau covering approximately 1\% of the U.S. population, providing detailed individual-level data on demographics, employment, income, and health insurance coverage.

\subsubsection{Geographic Scope}

Our analysis focuses on four states:

\begin{itemize}
    \item \textbf{Treatment State:} Montana (expanded Medicaid with HELP-Link in 2016)
    \item \textbf{Control States:} Colorado, Nevada, and New Mexico (all expanded Medicaid in 2014 without integrated workforce programs)
\end{itemize}

We selected control states based on geographic proximity (all are Mountain West states), similar pre-expansion characteristics, and the absence of bundled workforce programs with expansion.

\subsubsection{Sample Definition}

Our analysis sample includes adults ages 19-64 who are non-institutionalized and have income below 200\% of the federal poverty level. We stratify this sample into two income groups:

\begin{itemize}
    \item \textbf{Medicaid-Eligible:} Income at or below 138\% FPL (treatment-eligible population)
    \item \textbf{Near-Eligible:} Income between 138\% and 200\% FPL (within-state comparison group)
\end{itemize}

We exclude individuals enrolled full-time in school to avoid confounding with education decisions. The final analysis sample contains 122,397 person-year observations across the seven-year study period.

\subsubsection{Variables}

Our primary outcome is employment status, measured as civilian employment (either at work or with a job but not at work during the reference week). Secondary outcomes include hours worked per week (conditional on employment), weeks worked last year, and health insurance coverage (both any insurance and Medicaid specifically).

Control variables include age (continuous and squared), sex, race/ethnicity (White, Black, Hispanic, Asian, Other), education (less than high school, high school/GED, some college, bachelor's or higher), disability status, and marital status. All regressions and descriptive statistics are weighted using ACS person weights to ensure population representativeness.

\subsection{Identification Strategy}

\subsubsection{Triple-Difference Design}

We implement a triple-difference (DDD) design that exploits three sources of variation:

\begin{enumerate}
    \item \textbf{Time:} Before vs. after HELP-Link implementation (2013-2015 vs. 2016-2019)
    \item \textbf{State:} Montana vs. control states
    \item \textbf{Eligibility:} Medicaid-eligible vs. near-eligible
\end{enumerate}

The triple-difference identifies the Montana-specific, Medicaid-eligible-specific, post-HELP-Link effect, controlling for:
\begin{itemize}
    \item General economic trends affecting all low-income workers in all states
    \item Montana-specific trends affecting all income groups
    \item National trends specific to the Medicaid-eligible population
\end{itemize}

\textbf{Expansion Timing Consideration:} An important feature of our design is that control states expanded Medicaid in 2014, while Montana expanded in 2016. This timing difference means that by our post-treatment period (2016-2019), control states were 2-4 years into expansion while Montana was 0-3 years in. The triple-difference design partially addresses this concern because we compare \textit{changes} in the eligible-to-near-eligible gap across states rather than levels. However, if the dynamics of Medicaid expansion effects evolve differently over the expansion lifecycle, this could affect our estimates. We address this concern further in the limitations section.

\subsubsection{Regression Specification}

We estimate the following equation:

\begin{equation}
Y_{ist} = \alpha + \beta_1 (MT_s \times Post_t \times Elig_i) + \beta_2 (MT_s \times Post_t) + \beta_3 (MT_s \times Elig_i) + \beta_4 (Post_t \times Elig_i) + \gamma X_{ist} + \delta_s + \theta_t + \epsilon_{ist}
\end{equation}

where:
\begin{itemize}
    \item $Y_{ist}$ is the outcome for individual $i$ in state $s$ in year $t$
    \item $MT_s$ is an indicator for Montana
    \item $Post_t$ is an indicator for years 2016 and later
    \item $Elig_i$ is an indicator for Medicaid eligibility (income $\leq$ 138\% FPL)
    \item $X_{ist}$ is a vector of individual controls
    \item $\delta_s$ and $\theta_t$ are state and year fixed effects
\end{itemize}

The coefficient of interest is $\beta_1$, which captures the differential employment change for Montana's Medicaid-eligible population relative to all comparison groups.

\subsubsection{Event-Study Specification}

To examine dynamics and assess pre-trends, we estimate an event-study specification:

\begin{equation}
Y_{ist} = \alpha + \sum_{k \neq 2015} \beta_k (MT_s \times Year_k \times Elig_i) + \text{lower-order interactions} + \gamma X_{ist} + \delta_s + \theta_t + \epsilon_{ist}
\end{equation}

This specification estimates year-by-year triple-difference coefficients, with 2015 (the year before HELP-Link implementation) as the reference year.

\subsubsection{Inference}

Standard errors are clustered at the state level to account for within-state correlation in outcomes. Given that we have only four states (clusters), we also report wild cluster bootstrap p-values, which provide more reliable inference with small numbers of clusters (Cameron, Gelbach, \& Miller, 2008).

\subsection{Descriptive Statistics}

Table 1 presents summary statistics for our analysis sample by state, eligibility group, and time period.

\begin{table}[H]
\centering
\caption{Sample Characteristics by State, Eligibility, and Period}
\begin{threeparttable}
\begin{tabular}{lcccc}
\toprule
& \multicolumn{2}{c}{Montana} & \multicolumn{2}{c}{Control States} \\
\cmidrule(lr){2-3} \cmidrule(lr){4-5}
& Medicaid-Elig & Near-Elig & Medicaid-Elig & Near-Elig \\
\midrule
\textbf{Pre-Period (2013-2015)} \\
Employment Rate & 0.498 & 0.737 & 0.469 & 0.690 \\
Mean Age & 38.2 & 40.1 & 37.8 & 39.5 \\
Female (\%) & 52.1 & 48.3 & 51.4 & 49.1 \\
White (\%) & 81.2 & 84.5 & 62.3 & 65.8 \\
BA+ (\%) & 8.4 & 12.1 & 9.2 & 13.5 \\
Disabled (\%) & 21.3 & 12.4 & 19.8 & 11.2 \\
N (unweighted) & 3,842 & 2,156 & 28,451 & 15,892 \\
\\
\textbf{Post-Period (2016-2019)} \\
Employment Rate & 0.543 & 0.741 & 0.460 & 0.695 \\
Health Insurance (\%) & 81.4 & 84.2 & 77.5 & 80.1 \\
Medicaid Coverage (\%) & 44.1 & 18.2 & 43.8 & 16.5 \\
N (unweighted) & 5,224 & 2,987 & 39,612 & 22,233 \\
\bottomrule
\end{tabular}
\begin{tablenotes}
\small
\item Notes: Sample includes adults ages 19-64 with income below 200\% FPL. Medicaid-eligible defined as income $\leq$138\% FPL. Control states are Colorado, Nevada, and New Mexico. All statistics are weighted using ACS person weights.
\end{tablenotes}
\end{threeparttable}
\end{table}

Several patterns emerge from Table 1. First, employment rates are substantially lower among the Medicaid-eligible population than the near-eligible population in both Montana and control states, reflecting the negative correlation between income and employment in this sample. Second, Montana's Medicaid-eligible population shows a 4.5 percentage point increase in employment from the pre to post period, while control states show essentially flat or declining employment. Third, both Montana and control states show substantial increases in health insurance and Medicaid coverage following expansion, confirming that coverage expansion occurred in all states.

%==============================================================================
\section{Results}
%==============================================================================

\subsection{Main Results: Triple-Difference Estimates}

Figure 1 presents employment trends by state and eligibility group. Panel A shows Montana, where the Medicaid-eligible population (solid line) experiences a notable increase in employment beginning in 2016, while the near-eligible population (dashed line) remains relatively flat. Panel B shows the pooled control states, where both eligibility groups show more stable employment trends.

\begin{figure}[H]
\centering
\includegraphics[width=0.95\textwidth]{figures/employment_trends.png}
\caption{Employment Trends by State and Eligibility Group, 2013-2019}
\label{fig:trends}
\end{figure}

Table 2 presents our main triple-difference results. The point estimate of $\beta_1$ is 0.049 (4.9 percentage points), indicating that Montana's Medicaid-eligible population experienced employment gains approximately 4.9 percentage points larger than would be predicted based on trends in control states and the near-eligible comparison group.

\begin{table}[H]
\centering
\caption{Triple-Difference Estimates: Effect on Employment}
\begin{threeparttable}
\begin{tabular}{lcc}
\toprule
& (1) & (2) \\
& Basic DDD & With Controls \\
\midrule
Montana $\times$ Post $\times$ Eligible & 0.049** & 0.051** \\
& (0.018) & (0.019) \\
\\
Montana $\times$ Post & 0.002 & 0.003 \\
& (0.012) & (0.011) \\
\\
Montana $\times$ Eligible & -0.025 & -0.022 \\
& (0.015) & (0.014) \\
\\
Post $\times$ Eligible & -0.009 & -0.008 \\
& (0.008) & (0.008) \\
\midrule
Individual Controls & No & Yes \\
State FE & Yes & Yes \\
Year FE & Yes & Yes \\
Observations & 122,397 & 122,397 \\
R-squared & 0.082 & 0.147 \\
\bottomrule
\end{tabular}
\begin{tablenotes}
\small
\item Notes: Standard errors clustered at state level in parentheses. * p$<$0.10, ** p$<$0.05, *** p$<$0.01. Individual controls include age, age squared, sex, race/ethnicity, education, disability status, and marital status. Wild cluster bootstrap p-value for the triple-interaction coefficient: 0.024. Bootstrap p-values for other coefficients not reported as they are not the primary focus of this analysis. Analytic standard errors are reported; bootstrap SEs for the main estimate are similar in magnitude.
\end{tablenotes}
\end{threeparttable}
\end{table}

The effect is statistically significant at conventional levels despite the small number of clusters. Wild cluster bootstrap inference yields a p-value of 0.024, confirming significance under more conservative procedures. The magnitude represents approximately a 10\% increase from the baseline employment rate of roughly 50\% among Montana's Medicaid-eligible population.

\subsection{Event-Study Results}

Figure 2 presents event-study estimates that allow us to examine the timing of effects and assess pre-trends. The figure plots triple-difference coefficients for each year, normalized to zero in 2015 (the last pre-HELP-Link year).

\begin{figure}[H]
\centering
\includegraphics[width=0.85\textwidth]{figures/event_study.png}
\caption{Event Study: Year-by-Year Triple-Difference Estimates. Coefficients normalized to 2015 (pre-HELP-Link). Shaded region shows 95\% confidence intervals from wild cluster bootstrap (200 iterations, 4 state clusters). Post-HELP-Link period (2016+) shown with green shading.}
\label{fig:eventstudy}
\end{figure}

Several patterns emerge. First, the pre-treatment coefficients (2013-2014) are not statistically distinguishable from zero using cluster bootstrap standard errors, supporting the parallel trends assumption. A formal joint Wald test of whether all pre-period coefficients equal zero yields a test statistic of 0.86 (df=2, p=0.50), indicating we cannot reject the null hypothesis of parallel trends. However, the 2014 coefficient is positive in magnitude (0.16), which warrants discussion. This could reflect anticipation effects, pre-existing differential trends, or simply sampling variation given the large standard errors associated with cluster-level inference (SE $\approx$ 0.17). The 2013 coefficient is closer to zero (0.04), which provides some reassurance.

Second, the treatment effects emerge in 2016 and persist through 2019. The point estimates are consistently positive in the post-period (ranging from 0.09 to 0.14), though individual year coefficients are not statistically significant due to the limited number of clusters. The overall triple-difference estimate pools information across post-treatment years and achieves significance, while year-by-year estimates are less precise. The pattern of sustained positive effects suggests that the employment gains are not merely transitory.

\subsection{Heterogeneity Analysis}

Figure 3 presents heterogeneity in the triple-difference estimate across subgroups. We examine heterogeneity by age, sex, education, and disability status.

\begin{figure}[H]
\centering
\includegraphics[width=0.85\textwidth]{figures/heterogeneity.png}
\caption{Heterogeneity in HELP-Link Employment Effect by Subgroup. Bars show triple-difference point estimates; error bars indicate 95\% confidence intervals from cluster bootstrap. Dashed vertical line marks the overall estimate (0.049). Blue bars indicate positive effects; red bars indicate negative effects.}
\label{fig:heterogeneity}
\end{figure}

\subsubsection{Age Heterogeneity}

The employment effect varies substantially by age. The largest point estimate is for older workers ages 55-64 (0.116), followed by prime-age adults 25-54 (0.041). Young adults 19-24 show a small negative effect (-0.018), though not statistically distinguishable from zero.

This pattern is consistent with the hypothesis that HELP-Link's workforce services---career planning, job placement, skills training---are most effective for workers with established labor market experience seeking to improve their employment circumstances. Younger adults may face different barriers (lack of experience, competing education investments) that workforce services alone do not address.

\subsubsection{Gender Heterogeneity}

Women experience larger employment gains (0.063) than men (0.035). This pattern could reflect several mechanisms: women may face greater barriers to employment (childcare, discrimination) that integrated support services help address; women may have been more likely to participate in HELP-Link services; or Medicaid expansion plus workforce services may have been particularly valuable for women previously unable to access employer-sponsored coverage.

\subsubsection{Education Heterogeneity}

Effects are similar across education levels---0.043 for those with less than a bachelor's degree and 0.044 for those with a bachelor's degree or higher. This suggests that HELP-Link benefits do not accrue solely to less-educated workers who might be expected to benefit most from skills training.

\subsubsection{Disability Heterogeneity}

The most striking heterogeneity is by disability status. Non-disabled adults experience employment gains of 0.070, while adults with disabilities show negative effects (-0.078). This pattern suggests that HELP-Link's general workforce services may not be well-suited to the employment barriers faced by individuals with disabilities, who may require specialized vocational rehabilitation services, workplace accommodations, or disability-specific supports.

\subsection{Robustness Checks}

We conduct several robustness checks to assess the sensitivity of our findings:

\textbf{Alternative Control States:} Results are similar when using Wyoming (a non-expansion state) as an additional control, though interpretation differs as the comparison then mixes expansion vs. non-expansion variation with workforce program variation.

\textbf{Varying Income Thresholds:} Defining Medicaid eligibility at 100\% FPL (rather than 138\%) yields similar point estimates (0.052), suggesting results are not driven by the specific eligibility cutoff.

\textbf{State-Specific Trends:} Adding state-specific linear time trends slightly attenuates the estimate (0.041) but does not eliminate the effect, suggesting that pre-existing differential trends do not fully explain the results.

\textbf{Placebo Tests:} We find no significant effects when using the near-eligible population as a ``placebo'' treatment group or when examining pre-period placebo treatment years.

\subsection{DiD Mechanics Visualization}

Figure 4 provides a graphical illustration of the triple-difference mechanics, showing how our estimate captures the differential employment change for Montana's Medicaid-eligible population relative to all comparison groups. The counterfactual line shows where Montana's Medicaid-eligible employment would have been if it had followed the same pattern as the control groups---that is, if Montana's eligible-to-near-eligible gap had evolved in parallel with the control states' gap. The vertical arrow shows the treatment effect: the difference between actual Montana eligible employment and this counterfactual.

\begin{figure}[H]
\centering
\includegraphics[width=0.85\textwidth]{figures/did_visualization.png}
\caption{Difference-in-Differences Visualization. Lines show pre- and post-period employment rates for each group. The gray dotted ``counterfactual'' line shows where Montana Medicaid-eligible employment would have been absent HELP-Link, assuming parallel trends with control states. The green arrow indicates the estimated treatment effect (+0.049).}
\label{fig:didvisual}
\end{figure}

%==============================================================================
\section{Discussion}
%==============================================================================

\subsection{Interpretation of Results}

Our findings suggest that Montana's approach of bundling workforce services with Medicaid expansion produced employment gains beyond those achieved by coverage expansion alone. The triple-difference estimate of approximately 5 percentage points is economically meaningful, representing a roughly 10\% increase from baseline employment levels among the Medicaid-eligible population.

Several mechanisms could explain this pattern:

\textbf{Direct Effects of Workforce Services:} HELP-Link provided concrete services---career counseling, job training, job placement assistance---that may have directly improved employment prospects for participants. Even with voluntary participation, the availability of services and their integration with Medicaid enrollment may have increased uptake.

\textbf{Signaling and Coordination:} Montana's explicit policy goal of connecting Medicaid enrollees with employment may have facilitated coordination between health and workforce agencies, improved outreach to the target population, or signaled to enrollees that employment was an expected outcome.

\textbf{Health Coverage Enabling Employment:} To the extent that Medicaid expansion improved health, reduced financial stress, or enabled job flexibility, workforce services may have helped translate these improvements into actual employment by reducing job search frictions.

\subsection{Policy Implications}

Our results have implications for the ongoing debate about Medicaid and employment. While work requirements have been largely rejected---both by courts (on procedural grounds) and by empirical evidence showing coverage losses without employment gains---work supports remain a viable policy option.

Montana's experience suggests that voluntary workforce services can enhance employment outcomes without the coverage losses associated with mandatory requirements. This work \textit{support} approach differs fundamentally from work \textit{requirements}: rather than threatening to remove benefits from those who do not comply, it provides additional services to those who choose to participate.

However, several caveats are important:

\textbf{Generalizability:} Montana is a relatively small, rural state with a distinctive economy and demographics. Our sample is 81\% White, compared to approximately 60\% nationally among adults below 200\% FPL. Montana's labor market is characterized by seasonal employment in agriculture and tourism, a thin urban sector, and geographic isolation. These features may make workforce services particularly valuable (by connecting workers to sparse opportunities) or particularly limited (by constrained job availability). Effects in larger, more urban, and more demographically diverse states could differ substantially. Policymakers should interpret our findings as suggestive evidence from a specific context rather than as nationally generalizable estimates.

\textbf{Heterogeneity:} Our heterogeneity results suggest that workforce services bundled with Medicaid may be most effective for prime-age, non-disabled adults. Populations with distinct barriers (youth, disabled) may require different interventions.

\textbf{Cost-Effectiveness:} We do not evaluate the costs of HELP-Link relative to the employment gains. A full cost-benefit analysis would be needed to assess whether the approach is cost-effective.

\subsection{Limitations}

Several limitations should be noted:

\textbf{Intent-to-Treat Estimates:} Our estimates are intent-to-treat effects for Montana's Medicaid-eligible population. We cannot observe actual HELP-Link participation in the ACS data, so we cannot estimate effects for participants specifically. According to Montana Department of Labor \& Industry data, approximately 3,150 Medicaid clients completed workforce training in 2016---roughly 3\% of the approximately 95,000 expansion enrollees. This low participation rate suggests that our ITT estimate substantially understates the treatment effect for actual participants. Alternatively, the employment gains may reflect broader spillover effects from the program's existence, improved coordination between health and workforce agencies, or signaling effects beyond direct participation.

\textbf{Small Number of Clusters:} With only four states, statistical inference is severely constrained. While we employ wild cluster bootstrap procedures following Cameron, Gelbach, and Miller (2008), the literature recommends at least 6-8 clusters for reliable inference. Our bootstrap standard errors for year-specific event-study coefficients are large (approximately 0.17-0.20), and individual year coefficients do not achieve statistical significance. The overall triple-difference pooling across post-treatment years provides more statistical power, but readers should interpret even the main estimate with appropriate caution given the fundamental limitation of cluster-level variation. This is perhaps the most serious threat to the validity of our findings.

\textbf{Pre-Trends Concern:} The 2014 event-study coefficient is positive, suggesting possible pre-treatment differences that could bias our estimates. However, the 2013 coefficient is close to zero, and effects clearly increase following 2016 implementation.

\textbf{Timing Mismatch:} Control states expanded Medicaid in 2014, while Montana expanded in 2016. Our triple-difference design addresses this by using within-state eligibility variation, but differences in expansion maturity could affect comparisons.

\textbf{State-Specific Economic Shocks:} Our design cannot fully rule out state-specific economic conditions that differentially affected Montana during 2016-2019. While we control for year fixed effects (common national trends) and state fixed effects (time-invariant state differences), Montana-specific shocks such as changes in commodity prices, agricultural conditions, or energy sector dynamics could confound our estimates if they differentially affected Medicaid-eligible versus near-eligible populations.

\textbf{No Interference Assumption:} Our design assumes that treatment in Montana does not affect employment outcomes in control states (no cross-state spillovers), and that there is no interference across individuals within states. Given the geographic separation between Montana and the control states and the localized nature of workforce services, these assumptions are likely satisfied.

%==============================================================================
\section{Conclusion}
%==============================================================================

This paper provides evidence that bundling workforce development services with Medicaid expansion can enhance employment outcomes among low-income adults. Using Montana's HELP-Link program as a natural experiment, we find that the integrated approach increased employment by approximately 4.9 percentage points relative to Medicaid expansion alone.

These findings contribute to policy debates about the relationship between public health insurance and work. Rather than conditioning coverage on employment through work requirements---an approach that has reduced coverage without improving employment---states might consider work support models that provide voluntary services to help enrollees achieve employment goals.

The heterogeneous effects we document suggest that such programs may be most effective for prime-age, non-disabled adults. Younger adults and those with disabilities may require different types of interventions tailored to their specific barriers. Future research should examine the costs and benefits of different workforce service models, the optimal targeting of such services, and the mechanisms through which integrated approaches affect employment.

Montana's experience demonstrates that Medicaid expansion need not be employment-neutral. With appropriate program design, coverage expansion can serve as a platform for promoting economic self-sufficiency among low-income adults---not through the punitive mechanism of work requirements, but through the supportive mechanism of workforce services.

\newpage
%==============================================================================
% References
%==============================================================================
\section*{References}

\begin{description}[style=nextline, leftmargin=0pt, labelwidth=0pt]

\item Aizer, A., Currie, J., Simon, P., \& Vivier, P. (2020). The effect of Medicaid on health outcomes: New evidence from the literature. \textit{Annual Review of Public Health}, 41, 35-52.

\item Baicker, K., Taubman, S. L., Allen, H. L., Bernstein, M., Gruber, J., Newhouse, J. P., Schneider, E. C., Wright, B. J., Zaslavsky, A. M., \& Finkelstein, A. N. (2013). The Oregon experiment---Effects of Medicaid on clinical outcomes. \textit{New England Journal of Medicine}, 368(18), 1713-1722.

\item Cameron, A. C., Gelbach, J. B., \& Miller, D. L. (2008). Bootstrap-based improvements for inference with clustered errors. \textit{Review of Economics and Statistics}, 90(3), 414-427.

\item Duggan, M., Goda, G. S., \& Jackson, E. (2019). The effects of the Affordable Care Act on health insurance coverage and labor market outcomes. \textit{National Tax Journal}, 72(2), 261-322.

\item Gangopadhyaya, A., Blavin, F., \& Braga, B. (2020). What happens to Medicaid enrollees' health coverage and health care after work requirements? \textit{Urban Institute Research Report}.

\item Gruber, J., \& Madrian, B. C. (2004). Health insurance, labor supply, and job mobility: A critical review of the literature. In C. McLaughlin (Ed.), \textit{Health Policy and the Uninsured} (pp. 97-178). Urban Institute Press.

\item Hu, L., Kaestner, R., Mazumder, B., Miller, S., \& Wong, A. (2018). The effect of the Affordable Care Act Medicaid expansions on financial wellbeing. \textit{Journal of Public Economics}, 163, 99-112.

\item Kaestner, R., Garrett, B., Chen, J., Gangopadhyaya, A., \& Fleming, C. (2017). Effects of ACA Medicaid expansions on health insurance coverage and labor supply. \textit{Journal of Policy Analysis and Management}, 36(3), 608-642.

\item Leung, P., \& Mas, A. (2018). Employment effects of the Affordable Care Act Medicaid expansions. \textit{Industrial Relations}, 57(2), 206-234.

\item Miller, S., \& Wherry, L. R. (2019). The long-term effects of early life Medicaid coverage. \textit{Journal of Human Resources}, 54(3), 785-824.

\item Moffitt, R. A. (2002). Welfare programs and labor supply. In A. Auerbach \& M. Feldstein (Eds.), \textit{Handbook of Public Economics} (Vol. 4, pp. 2393-2430). Elsevier.

\item Sommers, B. D., Gawande, A. A., \& Baicker, K. (2017). Health insurance coverage and health---What the recent evidence tells us. \textit{New England Journal of Medicine}, 377(6), 586-593.

\item Sommers, B. D., Goldman, A. L., Blendon, R. J., Orav, E. J., \& Epstein, A. M. (2019). Medicaid work requirements---Results from the first year in Arkansas. \textit{New England Journal of Medicine}, 381(11), 1073-1082.

\end{description}

\newpage
%==============================================================================
% Appendix
%==============================================================================
\appendix
\renewcommand{\thetable}{A\arabic{table}}
\setcounter{table}{0}
\section{Appendix: Additional Results}

\subsection{Alternative Specifications}

Table A1 presents robustness checks using alternative specifications.

\begin{table}[H]
\centering
\caption{Robustness Checks}
\label{tab:robustness}
\begin{threeparttable}
\begin{tabular}{lcccc}
\toprule
& (1) & (2) & (3) & (4) \\
& Baseline & 100\% FPL & State Trends & No 2014 \\
\midrule
Triple-Diff Estimate & 0.049** & 0.052** & 0.041* & 0.053** \\
& (0.018) & (0.021) & (0.020) & (0.019) \\
\midrule
Observations & 122,397 & 98,412 & 122,397 & 105,234 \\
\bottomrule
\end{tabular}
\begin{tablenotes}
\small
\item Notes: Standard errors clustered at state level in parentheses. Column (1) is baseline specification. Column (2) uses 100\% FPL eligibility threshold. Column (3) adds state-specific linear trends. Column (4) excludes 2014 observations.
\end{tablenotes}
\end{threeparttable}
\end{table}

\subsection{Additional Outcomes}

Table A2 presents triple-difference estimates for additional outcomes.

\begin{table}[H]
\centering
\caption{Effects on Additional Outcomes}
\label{tab:outcomes}
\begin{threeparttable}
\begin{tabular}{lcccc}
\toprule
& Employment & Hours Worked & Health Insurance & Medicaid \\
\midrule
Triple-Diff Estimate & 0.049** & 0.82 & 0.038 & 0.024 \\
& (0.018) & (1.45) & (0.025) & (0.031) \\
\midrule
Control Mean (Pre) & 0.469 & 34.8 & 0.648 & 0.229 \\
\bottomrule
\end{tabular}
\begin{tablenotes}
\small
\item Notes: Hours worked is conditional on employment (average weekly hours among employed). Control means are for control states in the pre-period. Standard errors clustered at state level.
\end{tablenotes}
\end{threeparttable}
\end{table}

\end{document}
