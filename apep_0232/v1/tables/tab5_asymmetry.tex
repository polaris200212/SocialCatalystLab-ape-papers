
\begin{table}[t]
\centering
\caption{Asymmetric Monetary Transmission: Tightening vs. Easing}
\label{tab:asymmetry}
\begin{threeparttable}
\begin{tabular}{@{}lccccl@{}}
\toprule
& \multicolumn{2}{c}{Tightening} & \multicolumn{2}{c}{Easing} & \\
Horizon & $\hat{\gamma}_{tight}^h$ & (SE) & $\hat{\gamma}_{ease}^h$ & (SE) & N \\
\midrule
$h = 12$ & 0.4109 & (0.5741) & -0.0575 & (0.3722) & 15,708 \\
$h = 24$ & 0.1744 & (0.5727) & 0.6220 & (0.5230) & 15,096 \\
$h = 36$ & 1.1021 & (0.8008) & 0.5577 & (0.6657) & 14,484 \\
\bottomrule
\end{tabular}
\begin{tablenotes}[flushleft]
\small
\item \textit{Notes:} Tightening shocks are BRW $> 0$; easing shocks are BRW $< 0$. Each is interacted separately with the standardized HtM share. HANK predicts that tightening effects are larger in absolute value for high-HtM states (asymmetric amplification). State and year-month FE; Driscoll-Kraay SEs. *, **, *** denote significance at 10\%, 5\%, 1\%.
\end{tablenotes}
\end{threeparttable}
\end{table}

