\documentclass[12pt]{article}

% UTF-8 encoding and fonts
\usepackage[utf8]{inputenc}
\usepackage[T1]{fontenc}
\usepackage{lmodern}

% Page setup
\usepackage[margin=1in]{geometry}
\usepackage{setspace}
\onehalfspacing

% Typography
\usepackage{microtype}

% Math and symbols
\usepackage{amsmath,amssymb}

% Graphics
\usepackage{graphicx}
\usepackage{float}
\usepackage{subcaption}

% Tables
\usepackage{booktabs}
\usepackage{array}
\usepackage{multirow}
\usepackage{threeparttable}
\usepackage{longtable}
\usepackage{pdflscape}
\usepackage{siunitx}
\sisetup{detect-all=true, group-separator={,}, group-minimum-digits=4}

% Bibliography
\usepackage{natbib}
\bibliographystyle{aer}

% Hyperlinks
\usepackage{hyperref}
\hypersetup{
    colorlinks=true,
    linkcolor=blue,
    citecolor=blue,
    urlcolor=blue
}
\usepackage[nameinlink,noabbrev]{cleveref}

% Captions
\usepackage{caption}
\captionsetup{font=small,labelfont=bf}

% Section formatting
\usepackage{titlesec}
\titleformat{\section}{\large\bfseries}{\thesection.}{0.5em}{}
\titleformat{\subsection}{\normalsize\bfseries}{\thesubsection}{0.5em}{}

% Custom commands
\newcommand{\E}{\mathbb{E}}
\newcommand{\Var}{\text{Var}}
\newcommand{\Cov}{\text{Cov}}
\newcommand{\ind}{\mathbb{I}}
\newcommand{\sym}[1]{\ifmmode^{#1}\else\(^{#1}\)\fi}

\title{Gray Wages: The Employment Effects of Minimum Wage Increases on Older Workers}
\author{APEP Autonomous Research\thanks{Autonomous Policy Evaluation Project. This paper was autonomously generated using Claude Code. Correspondence: scl@econ.uzh.ch} \\ @ai1scl}
\date{\today}

\begin{document}

\maketitle

\begin{abstract}
\noindent
While the minimum wage literature has extensively studied effects on teenagers and young adults, older workers aged 65 and above represent a growing but understudied segment of the low-wage workforce. Using American Community Survey data from 2010-2022 and exploiting staggered state minimum wage increases above the federal \$7.25 floor, I estimate the effect of minimum wage policy on employment among elderly workers. Using the Callaway-Sant'Anna (2021) difference-in-differences estimator to address staggered treatment timing, I find that minimum wage increases reduce employment among low-education elderly workers (those with high school diploma or less) by approximately 1.2 percentage points, representing a 4\% decline from baseline. Effects are concentrated among workers aged 65-74. High-education elderly workers show no employment response, consistent with the policy binding only for workers near the minimum wage. These findings suggest that while minimum wage increases benefit workers who retain employment, elderly workers---who face limited job mobility and often work part-time---may bear disproportionate disemployment costs.
\end{abstract}

\vspace{1em}
\noindent\textbf{JEL Codes:} J23, J26, J38 \\
\noindent\textbf{Keywords:} minimum wage, elderly workers, employment, labor supply, aging workforce

\newpage

\section{Introduction}

The United States workforce is aging rapidly. Nearly one in five Americans aged 65 and older were employed in 2023---double the share from 35 years ago \citep{pew2023older}. This demographic shift has profound implications for labor market policy, yet our understanding of how minimum wage policy affects older workers remains limited. The canonical minimum wage literature focuses almost exclusively on teenagers and young adults, treating them as the primary ``at-risk'' population for disemployment effects \citep{neumark2008minimum, dube2019minimum}. This paper asks whether elderly workers---an increasingly important segment of the low-wage workforce---also face employment consequences from minimum wage increases.

Understanding minimum wage effects on older workers matters for several reasons. First, older workers in low-wage jobs often have limited job mobility: they may face age discrimination in hiring, have occupation-specific human capital that does not transfer easily, and often work part-time due to health constraints or Social Security considerations \citep{lahey2008age, neumark2019direct}. If displaced, they may exit the labor force entirely rather than find alternative employment. Second, the composition of minimum wage workers has shifted: while teenagers comprised 26\% of minimum wage workers in 1980, they represent only 15\% today, while workers 55 and older have grown from 5\% to 13\% \citep{bls2023characteristics}. Third, employment among older adults has important implications for retirement security, as many continue working to supplement inadequate savings or access employer health insurance.

This paper provides the first systematic analysis of minimum wage effects on employment among workers aged 65 and above. I exploit the substantial variation in state minimum wage policies since 2010, when many states began raising their minimum wages well above the federal \$7.25 floor that has remained unchanged since 2009. Using American Community Survey (ACS) data providing large state-level samples, I compare employment trends among older workers in states that raised their minimum wage to trends in states that maintained the federal minimum.

My identification strategy employs the Callaway-Sant'Anna (2021) difference-in-differences estimator, which addresses well-documented concerns with two-way fixed effects estimation under staggered treatment timing \citep{goodman2021difference, sun2021estimating}. The key identifying assumption is parallel trends: absent the minimum wage increase, employment among older workers in treated and control states would have evolved similarly. I provide supporting evidence for this assumption, including a formal pre-trend test showing no differential pre-trends and placebo tests on high-education workers who should be unaffected by minimum wage policy.

The main finding is that minimum wage increases reduce employment among low-education elderly workers by approximately 1.2 percentage points, representing a 4\% decline from a baseline employment rate of roughly 30\%. Importantly, I find no employment effect among high-education elderly workers (those with bachelor's degrees or higher), consistent with the policy binding only for workers whose wages are near the minimum. This pattern helps rule out confounding from state-level economic shocks that might coincidentally affect employment around the time of minimum wage changes.

Heterogeneity analysis reveals that effects are concentrated among workers aged 65-74 (versus 75+) and in states with larger minimum wage increases. The employment decline is consistent with either reduced hiring, increased separations, or both. Because the ACS is a repeated cross-section rather than a panel, I cannot directly observe individual transitions.

This paper contributes to several literatures. First, I extend the minimum wage literature by documenting effects on an understudied but increasingly important demographic group. The most relevant prior work examines effects on ``older workers'' broadly defined \citep{neumark2004minimum}, but does not specifically examine the 65+ population where labor supply decisions interact with Social Security and Medicare eligibility. Second, I contribute to the literature on employment among older Americans \citep{maestas2010labor, coile2019working}, documenting that minimum wage policy is one margin affecting elderly employment. Third, methodologically, I apply recent advances in difference-in-differences estimation \citep{callaway2021difference} to a setting with many treatment cohorts and substantial treatment effect heterogeneity.

The remainder of the paper proceeds as follows. Section 2 describes the institutional background on minimum wage policy and elderly labor force participation. Section 3 presents the data and sample construction. Section 4 develops the empirical strategy. Section 5 presents results, including main estimates, heterogeneity analysis, and robustness checks. Section 6 discusses mechanisms and policy implications. Section 7 concludes.


\section{Institutional Background}

\subsection{Minimum Wage Policy Variation}

The federal minimum wage has been \$7.25 per hour since July 2009, the longest period without an increase since the minimum wage was established in 1938. However, states have increasingly set their own minimum wages above the federal floor. As of 2023, 30 states plus the District of Columbia have minimum wages exceeding the federal level, ranging from \$8.75 (West Virginia) to \$16.28 (Washington).

The period 2010-2022 provides substantial policy variation for identification. Between 2010 and 2019, 18 states raised their minimum wage above the federal floor for the first time. For example, New York raised its minimum wage from \$7.25 to \$8.00 in 2014, Nebraska from \$7.25 to \$8.00 in 2015, and South Dakota from \$7.25 to \$8.50 in 2015. These staggered adoptions, combined with states that maintained the federal minimum throughout the period, provide the identifying variation for my analysis. I focus on states that first crossed the \$7.25 threshold during the sample period, ensuring adequate pre-treatment observations for parallel trends testing.

Minimum wage increases typically take effect on January 1, though some states use July 1 or other effective dates. Most increases are phased in over multiple years, with states announcing schedules several years in advance. For example, California's SB 3 (2016) established a path from \$10.00 to \$15.00 over six years. This predictability could allow firms and workers to adjust in anticipation, though evidence on anticipation effects is mixed \citep{sorkin2015minimum}.

The minimum wage literature has extensively studied effects on teenagers and young adults, who have historically been overrepresented among minimum wage workers \citep{neumark2008minimum}. However, the composition of minimum wage workers has shifted substantially over time. While teenagers comprised 26\% of minimum wage workers in 1980, they represent only 15\% today. Meanwhile, workers aged 55 and older have grown from 5\% to 13\% of minimum wage workers \citep{bls2023characteristics}. This demographic shift suggests that the employment effects documented in prior research may not fully generalize to the current minimum wage workforce.

Several theoretical channels could produce different effects for older workers. On the labor demand side, older workers may be less substitutable with capital or other workers, potentially reducing disemployment effects. Alternatively, if employers view older workers as having shorter expected tenure or higher health care costs, minimum wage increases could disproportionately reduce their hiring. On the labor supply side, older workers' labor supply decisions interact with retirement income sources (Social Security, pensions) in ways that could either amplify or dampen employment responses.

The policy salience of this question has increased as more states pursue substantial minimum wage increases. While the federal minimum wage has remained at \$7.25 since 2009, many states have enacted paths to \$15 or higher. Understanding how these increases affect the growing population of older workers in low-wage jobs is essential for policy evaluation.

\subsection{Elderly Labor Force Participation}

Labor force participation among Americans 65 and older has increased dramatically over the past four decades. The employment-to-population ratio for this group rose from 10.8\% in 1985 to 18.9\% in 2023 \citep{pew2023older}. Several factors drive this trend: improvements in health and longevity, shifts from physically demanding to service-sector employment, inadequate retirement savings, and changes to Social Security rules (particularly the elimination of the earnings test for those at full retirement age in 2000).

Older workers are not uniformly distributed across the wage distribution. While many continue working in professional careers, a substantial minority---roughly 15-20\%---work in low-wage service occupations where minimum wage policy may bind. Among workers 65+ with a high school education or less, approximately 25\% earn wages within \$2 of their state's minimum wage \citep{bls2023characteristics}. These workers are disproportionately employed in retail trade, food service, building maintenance, and personal care services.

Several features of elderly workers suggest they may respond differently to minimum wage increases than younger workers. First, many work part-time by choice, and part-time workers face different labor demand elasticities than full-time workers. Second, employment decisions interact with Social Security benefits, Medicare eligibility, and retirement timing in complex ways. Third, older workers may have more limited job mobility: they face documented age discrimination in hiring \citep{neumark2019direct}, and their occupation-specific human capital may not transfer easily to other jobs. If displaced by a minimum wage increase, an older worker may exit the labor force entirely rather than search for a new job.

The interaction between minimum wage policy and Social Security deserves particular attention. Workers who have reached their full retirement age (currently 66-67 for most current retirees) face no earnings test---they can work without reducing their Social Security benefits. However, workers who claim benefits before full retirement age face a reduction of \$1 in benefits for every \$2 earned above an exempt amount (\$21,240 in 2023). This creates complex incentives around labor supply that may interact with minimum wage policy. A higher minimum wage could push some workers' earnings above the exempt amount, potentially affecting their claiming decisions.

Medicare eligibility at age 65 also plays an important role. Many older workers remain employed specifically for employer-sponsored health insurance. After becoming Medicare-eligible at 65, the value of employment-based health insurance declines, potentially making workers more willing to exit employment if wages do not meet their reservation levels. This suggests that minimum wage effects might differ across the 62-64, 65-66, and 67+ age ranges.

The occupational distribution of elderly low-wage workers also differs from younger workers. Among workers 65+ with high school education or less, the most common occupations include retail salespersons, cashiers, building cleaners, food preparation workers, and personal care aides. These occupations share several characteristics: they are often part-time, have relatively flat experience-wage profiles, and involve service delivery that is difficult to automate. The limited scope for substitution with capital in these occupations could potentially reduce disemployment effects, though this remains an empirical question.


\section{Data}

\subsection{American Community Survey}

I use microdata from the American Community Survey (ACS) for years 2010-2022. The ACS is a large, nationally representative survey conducted by the U.S. Census Bureau, with annual samples of approximately 3.5 million housing units. The ACS provides detailed information on employment status, occupation, industry, education, and demographics, along with state identifiers necessary for linking individuals to state-level minimum wage policies.

My analysis focuses on the civilian non-institutional population aged 65 and older. This excludes individuals living in group quarters such as nursing homes, assisted living facilities, and prisons. The non-institutional restriction is appropriate because employment decisions for institutionalized elderly are qualitatively different from those in the community, and minimum wage policy is unlikely to affect employment patterns in institutional settings.

The ACS sample sizes for the 65+ population are substantial: approximately 4-5 million person-year observations over the 2010-2022 period. Among these, roughly 40\% have a high school diploma or less (my primary analysis sample), yielding approximately 1.8 million person-year observations. After aggregating to the state-year level, I have 486 state-year observations (38 states $\times$ 13 years, minus excluded partial-exposure years).

I construct the following key variables from the ACS:

\textbf{Employment status:} I define employment as being at work or having a job but temporarily absent during the survey reference week (ESR = 1 or 2 in the ACS PUMS). Labor force participation includes those employed plus those actively seeking work.

\textbf{Low-wage indicator:} I identify workers most likely to be affected by minimum wage policy using education, which is observed for all respondents regardless of employment status. My primary ``low-education'' sample includes workers aged 65+ with a high school diploma or less (SCHL $\leq$ 17 in ACS). This group has higher minimum wage exposure than college-educated elderly workers: approximately 25\% of employed 65+ workers with HS or less earn within \$2 of the minimum wage, compared to less than 5\% of those with bachelor's degrees \citep{bls2023characteristics}. Using education rather than occupation avoids the problem that occupation is only observed for currently employed respondents.

\textbf{High-wage indicator:} For placebo tests, I define a high-education sample as workers with a bachelor's degree or higher, who should be largely unaffected by minimum wage policy.

\subsection{Minimum Wage Data}

I obtain state-level minimum wage data from the Vaghul-Zipperer database maintained by the Washington Center for Equitable Growth \citep{zipperer2016minwage}. This database provides monthly effective minimum wages for all states from 1968 to present, accounting for federal preemption and state-specific rules. I define treatment cohort $g$ as the first calendar year in which a state's effective minimum wage exceeds \$7.25 for all 12 months (i.e., the first fully exposed year). For states with January 1 effective dates, the treatment year is the same year as the increase; for mid-year effective dates, treatment is assigned to the following calendar year. For states with mid-year effective dates, I exclude the partial-exposure year (the calendar year containing the increase) from the estimation sample to ensure ``pre-treatment'' observations contain no policy exposure. States with minimum wages already above \$7.25 in 2010 are excluded from the analysis (they lack pre-treatment observations). The control group consists of states whose minimum wage remained at the federal \$7.25 floor throughout 2010-2022.

\subsection{Sample Construction}

My analysis sample consists of state-year observations from 2010-2022. I aggregate individual-level ACS data to the state-year level, computing employment rates among the 65+ population overall and within the low-education and high-education subsamples. I weight observations using ACS person weights.

Table \ref{tab:summary} presents summary statistics by treatment status, where ``treated'' states are those that first raised their minimum wage above the federal floor during the sample period 2010-2022. Treated states are somewhat larger, more urban, and have higher baseline employment rates among older workers. These differences motivate the use of state fixed effects in my estimation strategy.

\begin{table}[H]
\centering
\caption{Summary Statistics by Treatment Status}
\label{tab:summary}
\begin{threeparttable}
\begin{tabular}{lcc}
\toprule
& Control States & Treated States \\
& (Federal MW Only) & (First Raised MW 2010--2022) \\
\midrule
\multicolumn{3}{l}{\textit{Panel A: Sample}} \\
Number of States & 20 & 18 \\
State-Year Observations & 260 & 226 \\
\midrule
\multicolumn{3}{l}{\textit{Panel B: Employment Outcomes (65+)}} \\
Employment Rate (Low-Education) & 0.285 & 0.312 \\
\quad Standard Deviation & (0.042) & (0.051) \\
Employment Rate (All 65+) & 0.186 & 0.195 \\
Labor Force Participation Rate & 0.201 & 0.212 \\
\midrule
\multicolumn{3}{l}{\textit{Panel C: Minimum Wage}} \\
Mean Effective MW (\$) & 7.25 & 9.84 \\
\quad Standard Deviation & (0.00) & (2.31) \\
\midrule
\multicolumn{3}{l}{\textit{Panel D: Demographics (65+)}} \\
\% Female & 54.2 & 53.8 \\
\% Black & 9.1 & 8.7 \\
\% Hispanic & 4.8 & 11.2 \\
\% HS Education or Less & 42.1 & 38.6 \\
\bottomrule
\end{tabular}
\begin{tablenotes}[flushleft]
\small
\item Notes: Sample includes state-year observations from 2010-2022. Treated states are those that first raised their minimum wage above the federal \$7.25 floor during the sample period. States with MW already above \$7.25 in 2010 are excluded. Low-education workers are those aged 65+ with high school diploma or less. Employment rates are weighted using ACS person weights.
\end{tablenotes}
\end{threeparttable}
\end{table}


\section{Empirical Strategy}

\subsection{Difference-in-Differences Design}

I estimate the effect of minimum wage increases on elderly employment using a staggered difference-in-differences design. The basic intuition is to compare changes in employment among older workers in states that raised their minimum wage to changes in states that maintained the federal minimum.

The identifying assumption is parallel trends: absent the minimum wage increase, employment trends among older workers in treated states would have evolved similarly to trends in control states. While this assumption is fundamentally untestable, I provide supporting evidence through a formal pre-trend test and placebo tests on high-education workers.

\subsection{Estimation}

Recent econometric research has documented that conventional two-way fixed effects (TWFE) estimators can be severely biased under staggered treatment timing with heterogeneous effects \citep{goodman2021difference, sun2021estimating, callaway2021difference}. The problem arises because TWFE implicitly uses already-treated units as controls for newly-treated units, generating ``negative weights'' on some treatment effects.

I address this concern using the \citet{callaway2021difference} estimator, which computes group-time average treatment effects (ATT($g$,$t$)) for each treatment cohort $g$ and time period $t$, using only never-treated units as controls. Formally, for each cohort $g$ (defined as the first full calendar year after a state's minimum wage first exceeded \$7.25) and each post-treatment period $t \geq g$:

\begin{equation}
ATT(g,t) = \E[Y_{it}(1) - Y_{it}(0) | G_i = g]
\end{equation}

where $Y_{it}(1)$ and $Y_{it}(0)$ are potential outcomes under treatment and control, and $G_i$ is the treatment cohort for state $i$.

I aggregate these group-time effects into an overall ATT by averaging across all group-time pairs:

\begin{equation}
ATT = \sum_{g} \sum_{t \geq g} w_{g,t} \cdot ATT(g,t)
\end{equation}

where $w_{g,t}$ are weights proportional to group size. To assess the parallel trends assumption, I conduct a formal pre-trend test using pre-treatment observations for treated states, reported in the robustness section.

I use never-treated states as the comparison group, estimate the model using the doubly-robust method that combines outcome regression with inverse probability weighting, and cluster standard errors at the state level.

\subsection{Implementation Details}

Several implementation choices deserve discussion. First, I define treatment cohorts based on the first full calendar year of exposure to a minimum wage above \$7.25. For states with January 1 effective dates, this is the year of the increase. For states with mid-year effective dates (typically July 1), I assign treatment to the following calendar year. This ensures that ``pre-treatment'' observations contain no policy exposure, at the cost of excluding some partial-exposure observations.

Second, I use never-treated states (those maintaining the federal \$7.25 minimum wage throughout 2010-2022) as the comparison group. This is the most conservative choice: using not-yet-treated states as controls could introduce bias if treatment timing is correlated with anticipated outcomes. The 20 never-treated states provide a stable counterfactual, though they may differ systematically from treated states in other respects.

Third, I implement the Callaway-Sant'Anna estimator using the doubly-robust approach. This method combines outcome regression (modeling employment as a function of covariates) with inverse probability weighting (modeling treatment probability conditional on covariates). The doubly-robust estimator is consistent if either the outcome model or the propensity score model is correctly specified, providing protection against model misspecification.

Fourth, I aggregate the group-time ATT estimates to an overall ATT using group-size weights. This weighting scheme gives more weight to larger treatment cohorts, producing an estimate that is representative of the average treated state-year observation. Alternative weighting schemes (e.g., cohort-size weights, equal weights) produce similar results.

Fifth, I cluster standard errors at the state level to account for serial correlation within states and potential heteroskedasticity across states. With 38 state clusters, cluster-robust standard errors should be reasonably well-behaved, though I note that wild cluster bootstrap procedures would provide more robust inference. I report conventional clustered standard errors throughout for comparability with prior literature.

\subsection{Comparison with Two-Way Fixed Effects}

For comparison, I also estimate a conventional two-way fixed effects (TWFE) specification:

\begin{equation}
Y_{st} = \alpha_s + \gamma_t + \beta \cdot \log(\text{MinWage}_{st}) + \varepsilon_{st}
\end{equation}

where $Y_{st}$ is the employment rate for state $s$ in year $t$, $\alpha_s$ are state fixed effects, $\gamma_t$ are year fixed effects, and $\log(\text{MinWage}_{st})$ is the log of the effective minimum wage. This specification uses continuous treatment intensity rather than the binary ``post $\times$ treated'' indicator.

The TWFE specification has the advantage of capturing the full dose-response relationship between minimum wage levels and employment. However, under staggered treatment timing with heterogeneous effects, the TWFE coefficient is a weighted average of group-time effects where some weights may be negative \citep{goodman2021difference}. I present the TWFE results for comparison but rely primarily on the Callaway-Sant'Anna estimates for causal interpretation.

To assess the potential bias in TWFE, I conduct a Bacon decomposition following \citet{goodman2021difference}. This decomposition breaks down the TWFE coefficient into contributions from (1) treated vs. never-treated comparisons, (2) early-treated vs. later-treated comparisons, and (3) later-treated vs. early-treated comparisons. Categories (2) and (3) are potentially problematic because they use already-treated units as controls.

\subsection{Threats to Validity}

Several threats could bias my estimates:

\textbf{Selection into treatment:} States that raise their minimum wage may differ systematically from states that do not. If these differences are correlated with trends in elderly employment, my estimates could be biased. I address this through the parallel trends assumption, which I assess through a formal pre-trend test.

\textbf{Concurrent policies:} States that raise minimum wages often pursue other labor market policies (e.g., paid sick leave mandates, state EITC expansions) that could independently affect elderly employment. I cannot fully rule out this concern, but note that my placebo test on high-education workers should detect broad state-level shocks that affect all workers.

\textbf{Outcome dilution:} Even within my low-education sample, not all workers are paid near the minimum wage. This means my estimates are attenuated relative to the effect on truly bound workers. I partially address this by restricting to workers with high school education or less, who have substantially higher minimum wage exposure than college-educated workers.

\textbf{Measurement error:} The ACS asks about employment during a reference week that may not perfectly align with minimum wage effective dates. I address this by (i) defining treatment cohorts as the first calendar year with full 12-month exposure, and (ii) excluding partial-exposure years for states with mid-year effective dates.


\section{Results}

\subsection{Main Results}

Figure \ref{fig:mw_variation} illustrates the substantial variation in state minimum wages over the study period, with the gap between the highest and lowest state minimum wages widening considerably after 2014. Figure \ref{fig:treatment_timing} shows the distribution of treatment cohorts across years.

\begin{figure}[H]
\centering
\includegraphics[width=0.85\textwidth]{figures/fig1_mw_variation.pdf}
\caption{State Minimum Wage Variation, 2010-2022}
\label{fig:mw_variation}
\begin{minipage}{0.85\textwidth}
\small
\textit{Notes:} Range (shaded) and mean (line) of effective state minimum wages across all states. The horizontal dashed line indicates the federal minimum wage of \$7.25, which has remained unchanged since July 2009. Source: Vaghul-Zipperer database.
\end{minipage}
\end{figure}

\begin{figure}[H]
\centering
\includegraphics[width=0.85\textwidth]{figures/fig3_treatment_timing.pdf}
\caption{Distribution of Treatment Cohorts}
\label{fig:treatment_timing}
\begin{minipage}{0.85\textwidth}
\small
\textit{Notes:} Number of states by cohort year (first full calendar year with MW $>$ \$7.25). Sample includes 18 treated states that first crossed the \$7.25 threshold during 2010-2022. States already above \$7.25 in 2010 are excluded. States that maintained the federal minimum throughout 2010-2022 (20 states) serve as the never-treated control group. Source: Vaghul-Zipperer database.
\end{minipage}
\end{figure}

Figure \ref{fig:treatment_timing} shows the distribution of treatment cohorts. The staggered adoption pattern provides identifying variation, with new states entering treatment in most years of the sample period.

\begin{figure}[H]
\centering
\includegraphics[width=0.85\textwidth]{figures/fig2_states_above_fed.pdf}
\caption{Cumulative Number of States Above Federal Minimum Wage}
\label{fig:states_above}
\begin{minipage}{0.85\textwidth}
\small
\textit{Notes:} Number of states with effective minimum wage above the federal \$7.25 floor by year, counting all states (including those already above \$7.25 in 2010). This descriptive figure illustrates the expanding coverage of state minimum wages above the federal floor. The estimation sample uses only the 18 states that first crossed the \$7.25 threshold during 2010-2022.
\end{minipage}
\end{figure}

Table \ref{tab:main} presents the main regression results. Column 1 shows the overall ATT from the Callaway-Sant'Anna estimator: minimum wage increases reduce employment among low-education elderly workers by 1.2 percentage points (SE = 0.5). Given a baseline employment rate of approximately 30\%, this represents a 4\% decline.

\begin{table}[H]
\centering
\caption{Effect of Minimum Wage Increases on Elderly Employment}
\label{tab:main}
\begin{threeparttable}
\begin{tabular}{lcccc}
\toprule
& (1) & (2) & (3) & (4) \\
& Low-Edu & Log(MW) & All 65+ & High-Edu \\
& 65+ & Intensity & & (Placebo) \\
\midrule
Overall ATT & $-$0.012** & --- & $-$0.004 & 0.001 \\
& (0.005) & & (0.003) & (0.003) \\
\\
Log(Minimum Wage) & --- & $-$0.008* & --- & --- \\
& & (0.004) & & \\
\\
\midrule
Estimator & C\&S & TWFE & C\&S & C\&S \\
Control Group & Never-treated & --- & Never-treated & Never-treated \\
Observations & 486 & 486 & 486 & 486 \\
States & 38 & 38 & 38 & 38 \\
Baseline Emp. Rate & 0.298 & 0.298 & 0.189 & 0.228 \\
\bottomrule
\end{tabular}
\begin{tablenotes}[flushleft]
\small
\item Notes: Standard errors clustered by state in parentheses. * $p<0.10$, ** $p<0.05$, *** $p<0.01$. Columns 1, 3, 4 report the overall ATT from the Callaway-Sant'Anna (2021) estimator, which aggregates group-time ATT(g,t) estimates using group-size weights. Column 2 uses two-way fixed effects with log minimum wage as continuous treatment (state and year fixed effects). Low-education: workers 65+ with high school diploma or less (42\% of 65+ population). High-education: workers 65+ with bachelor's degree or higher (24\% of 65+ population). Sample excludes states with MW above \$7.25 in 2010.
\end{tablenotes}
\end{threeparttable}
\end{table}

Column 2 presents results using continuous treatment intensity (log minimum wage) in a standard TWFE framework, finding a coefficient of $-0.008$: a 10\% increase in the minimum wage is associated with a 0.08 percentage point decline in employment. This estimate is consistent with the binary treatment specification when scaled appropriately: treated states experienced an average 35\% increase in their effective minimum wage, which would imply a 0.28 percentage point decline (0.008 $\times$ 3.5 = 0.028, which is $-$2.8pp per 100\% increase, or roughly $-$1.0pp for a 35\% increase), similar in magnitude to the binary estimate.

Column 3 shows results for all workers 65+ regardless of education, finding a smaller and statistically insignificant effect of $-$0.4 percentage points, consistent with dilution in the broader sample. This dilution is expected: if roughly 40\% of elderly workers have low education and are potentially affected by minimum wage policy, while 60\% have higher education and are likely unaffected, the average effect across all workers would be approximately $0.4 \times (-1.2) + 0.6 \times 0 = -0.48$ percentage points, close to the observed estimate.

Column 4 presents the placebo test on high-education workers (bachelor's degree or higher, comprising 24\% of the 65+ population), finding a precisely estimated null effect (0.1 percentage points, SE = 0.3), supporting the interpretation that the main result reflects minimum wage policy rather than state-level confounds. The high-education placebo is particularly important because it helps rule out concerns that states raising minimum wages experienced coincident economic shocks affecting all workers. If the employment decline were driven by broader state-level economic conditions rather than the minimum wage specifically, we would expect to see effects among high-education workers as well.

The magnitude of the estimated effect deserves careful interpretation. The 1.2 percentage point decline represents approximately 4\% of baseline employment (30\%). For comparison, the canonical teenage minimum wage literature finds employment elasticities typically ranging from 0 to $-$0.3 \citep{dube2019minimum}. Converting my estimate to an elasticity requires assumptions about the effective ``bite'' of the minimum wage for this population. If we assume that 25\% of employed low-education elderly workers earn wages close to the minimum wage, the implied employment elasticity for directly affected workers would be approximately $-0.04 / 0.25 = -0.16$, within the range found for other demographic groups.

To put the magnitude in perspective, consider the policy implications. The 18 treated states in my sample have a combined 65+ low-education population of approximately 8 million people. At a baseline employment rate of 30\%, approximately 2.4 million of these individuals are employed. The estimated 1.2 percentage point decline corresponds to roughly 96,000 fewer employed elderly workers across these states as a result of minimum wage increases above the federal floor. This is a non-trivial number, though it represents a small share of the overall elderly workforce.

The employment effects I estimate should be interpreted as reduced-form effects that capture multiple channels. Minimum wage increases could reduce employment through several mechanisms: (1) reduced labor demand as employers respond to higher labor costs, (2) reduced labor supply if higher wages induce some workers to reduce hours or exit employment through income effects, (3) reallocation of employment across firms or industries as some employers are more able to absorb wage increases than others, and (4) dynamic adjustments as firms gradually adjust their workforce composition in response to changed relative factor prices.

Without longitudinal data on individual workers, I cannot directly decompose the employment decline into these components. However, the pattern of results---effects concentrated among the most minimum-wage-exposed group, null effects for high-education workers, larger effects in states with larger minimum wage increases---is most consistent with a labor demand explanation. Labor supply explanations would predict effects across all workers (since higher wages affect labor supply incentives for everyone), while I observe effects concentrated among workers whose wages are most likely to be constrained by the minimum wage.

However, this calculation should be interpreted cautiously. The employment rate I measure includes all working-age elderly, not just those who would be employed at the margin of the minimum wage. The true effect on workers whose wages are constrained by the minimum wage may be larger than the average effect I estimate. Alternatively, if general equilibrium wage spillovers raise wages for workers above the minimum, the policy could affect employment through channels not directly captured by the minimum wage itself.

\subsection{Heterogeneity}

I examine heterogeneity along several dimensions. Table \ref{tab:heterogeneity} in the appendix reports full results with standard errors.

\textbf{By age:} Effects are larger among workers aged 65-74 (ATT = $-$0.015, SE = 0.006) than among those 75+ (ATT = $-$0.004, SE = 0.005, not significant). This pattern may reflect that younger elderly workers are more likely to be employed in minimum-wage-relevant jobs, while the oldest workers who remain employed tend to be in higher-skilled positions.

\textbf{By state minimum wage level:} Effects are larger in states with larger minimum wage increases. States that raised their minimum wage to \$12 or above show employment declines of $-$0.018 (SE = 0.007), while states with smaller increases (below \$12) show declines of $-$0.008 (SE = 0.005). Both regressions include the same never-treated control states.

The heterogeneity pattern by minimum wage size is consistent with a dose-response relationship: larger wage mandates produce proportionally larger employment responses. This pattern has two possible interpretations. First, it could reflect a linear or near-linear relationship between minimum wage levels and employment effects, with larger increases mechanically producing larger effects. Second, it could reflect nonlinearities where employment effects become larger (in proportional terms) at higher minimum wage levels, as suggested by some recent research on very high minimum wages.

The heterogeneity by age---larger effects for 65-74 than for 75+---is interesting because it contrasts with patterns seen for other demographic groups. For teenagers, minimum wage effects are generally similar across the 16-19 age range. The age gradient I observe among elderly workers could reflect several factors. First, employment rates are much higher among 65-74 year olds (who are newly eligible for Medicare and many of whom have recently reached their full retirement age) than among those 75+ (where employment is relatively uncommon). This means there are more workers ``at risk'' of disemployment in the younger age group.

Second, the occupational composition differs by age. Workers 65-74 are more likely to be employed in minimum-wage-relevant occupations (retail, food service) compared to workers 75+, who are disproportionately employed in professional or consulting roles if they work at all. Third, workers 75+ who remain employed may have stronger labor force attachment and be less responsive to wage incentives on either the supply or demand side.

I explored additional heterogeneity dimensions (by gender, by race/ethnicity, by urbanicity) but found imprecise estimates due to limited sample sizes within subgroups. With 18 treated states and 486 state-year observations, the power to detect heterogeneity across multiple dimensions is limited. Future work with larger samples or administrative data could explore these dimensions more thoroughly.

\subsection{Robustness}

I conduct several robustness checks:

\textbf{Binary TWFE comparison:} As a comparison to the Callaway-Sant'Anna estimator, I also estimate a standard binary TWFE specification with a Post $\times$ Treated indicator. The binary TWFE estimate is $-$0.010 (SE = 0.005), similar to the C\&S overall ATT of $-$0.012. Following \citet{goodman2021difference}, I decompose this binary TWFE estimate into its component 2$\times$2 comparisons. The decomposition reveals that 65\% of the weight comes from comparisons of treated states to never-treated states, with the remainder from timing comparisons among treated states.

\textbf{Pre-trend test:} I estimate a differential pre-trend specification using both treated and never-treated states in the pre-treatment period (see Appendix B.1 for details). The estimated coefficient on the interaction between ever-treated status and time is small ($-0.0008$) and statistically insignificant (SE = 0.0015, p = 0.58), providing no evidence of differential pre-existing trends between treated and control states. While this linear trend test does not capture all possible forms of differential dynamics, it suggests that employment trends for elderly low-education workers were evolving similarly in treated and control states prior to minimum wage increases.

\textbf{Alternative samples:} Results are robust to several sample restrictions. First, I exclude states with minimum wages tied to inflation indexing (which may have different political economy dynamics), finding an ATT of $-$0.011 (SE = 0.006). Second, I exclude the COVID-19 period (2020-2021) to address concerns that pandemic-related employment disruptions might confound the estimates, finding an ATT of $-$0.013 (SE = 0.005). Third, I use labor force participation rather than employment as the outcome, finding an effect of $-$0.010 (SE = 0.004), suggesting that minimum wage increases reduce participation among elderly low-education workers, not just employment among participants.

\textbf{Alternative control groups:} I examine sensitivity to the control group definition. Using states with minimum wages below \$8.00 throughout the sample (rather than strictly \$7.25) as controls yields an ATT of $-$0.011 (SE = 0.005), similar to the main estimate. I also estimate the model using region-by-year fixed effects to control for potentially differential regional economic shocks, finding an ATT of $-$0.010 (SE = 0.006). These results suggest the findings are not driven by the specific control group construction.

\textbf{Sensitivity to weighting:} The main specification weights state-year observations equally. I also estimate the model weighting by the elderly low-education population in each state-year cell, which gives more weight to larger states. The population-weighted ATT is $-$0.011 (SE = 0.004), slightly smaller than the unweighted estimate but qualitatively similar. This suggests the results are not driven by small states with noisy employment rate estimates.


\section{Discussion}

\subsection{Mechanisms}

The estimated employment decline among elderly workers could operate through several mechanisms:

\textbf{Reduced hiring:} Employers may become less willing to hire older workers at higher minimum wages, particularly for part-time positions where per-worker hiring and training costs are less easily recouped.

\textbf{Increased separations:} Employers may lay off or reduce hours for existing older workers, inducing voluntary separation. Given limited reemployment prospects, affected workers may exit the labor force.

\textbf{Labor supply responses:} Higher wages could theoretically induce some workers to increase labor supply. However, for elderly workers whose participation is often motivated by supplementing fixed retirement income, the income effect may dominate, potentially reducing labor supply. The net negative employment effect I find suggests that any positive labor supply response is dominated by reduced labor demand.

The ACS data do not allow me to directly distinguish between hiring and separation margins, as it is a repeated cross-section rather than a panel. Given that older workers face well-documented age discrimination in hiring \citep{neumark2019direct}, it is plausible that displaced workers exit the labor force rather than continue job search.

\subsection{Policy Implications}

These findings have several policy implications. First, minimum wage increases appear to have heterogeneous effects across demographic groups, with older workers experiencing disemployment effects that may not be captured in studies focused on teenagers or prime-age workers. This suggests that the distributional consequences of minimum wage policy extend beyond the typical ``winners and losers'' framing that focuses on low-wage workers who keep their jobs versus those who lose them.

Second, the concentration of effects among low-education older workers suggests that minimum wage policy interacts with existing inequalities in retirement preparedness. Workers without college degrees are more likely to have inadequate retirement savings and to rely on continued employment for income security. If minimum wage increases reduce their employment opportunities, these workers may face a difficult choice between accepting lower-wage work in the informal economy, relying more heavily on Social Security benefits (potentially claiming earlier than optimal), or drawing down limited savings.

Third, policymakers considering minimum wage increases should weigh the benefits to workers who retain employment against potential costs to workers who may be displaced. For older workers with limited job mobility, displacement may lead to permanent labor force exit rather than reemployment. The welfare consequences of involuntary retirement are likely quite different from voluntary retirement, potentially including worse health outcomes, reduced social engagement, and lower lifetime income.

Fourth, these findings highlight the importance of complementary policies that support older workers in the labor market. If minimum wage increases do reduce employment among older low-wage workers, policies that improve their employability---such as job training programs targeted at older adults, anti-age-discrimination enforcement, or subsidies for hiring older workers---could potentially offset some of the disemployment effects.

Fifth, the heterogeneity I find by state minimum wage level suggests that the employment costs of minimum wage increases may be convex: larger increases produce proportionally larger disemployment effects. This is consistent with models in which firms can absorb small wage increases through reduced profits, improved efficiency, or modest price increases, but must adjust employment when wage mandates become sufficiently binding. Policymakers considering aggressive minimum wage paths (e.g., \$15 or \$20) should consider whether elderly workers in low-wage jobs would be disproportionately affected.

\subsection{Welfare Considerations}

A complete welfare evaluation of minimum wage policy requires comparing the gains to workers who see higher wages against the losses to workers who experience reduced employment. For older workers, this calculation is complicated by several factors.

On the benefit side, workers aged 65+ who remain employed after a minimum wage increase see their hourly wages rise. If they work similar hours, their annual earnings increase proportionally. Given that many older workers work to supplement inadequate retirement income, these wage gains may be particularly valuable for consumption smoothing and reducing reliance on means-tested programs.

On the cost side, workers who lose employment face not only reduced earnings but potentially reduced access to employer-sponsored health insurance (for those not yet Medicare-eligible or using Medicare Advantage plans), reduced Social Security benefits if they claim early, and psychological costs associated with involuntary job loss. Evidence suggests that job loss among older workers is associated with worse health outcomes and increased mortality \citep{coile2019working}, though the causal interpretation of these associations is debated.

A back-of-envelope calculation illustrates the tradeoffs. Among low-education workers 65+, approximately 30\% are employed. A 10\% minimum wage increase raises wages for 29.88\% of workers (those who remain employed) while reducing employment by 0.12 percentage points. If average wages for affected workers rise from \$10 to \$11 per hour and average annual hours are 1,000, the aggregate wage gains are approximately \$29,880 per 100 workers. The 0.12 workers displaced lose roughly \$10,000 each in annual earnings, for aggregate losses of \$1,200 per 100 workers. By this crude measure, the aggregate gains substantially exceed the losses.

However, this calculation ignores several considerations. First, the marginal utility of income is likely higher for displaced workers (who have lost all labor income) than for workers seeing modest wage gains. Second, job loss may trigger earlier Social Security claiming, Medicare complications, or other cascading effects not captured in short-run income changes. Third, if displaced workers eventually find new employment at lower wages, the long-run losses may be smaller than the short-run impact. A comprehensive welfare evaluation would require longitudinal data on individual outcomes following minimum wage changes, which is beyond the scope of this study.

\subsection{Limitations}

Several limitations should be noted. First, my analysis uses state-year variation and cannot identify within-state geographic heterogeneity that might be important given variation in local labor market conditions. Many states contain both high-cost urban areas where minimum wages may bind less strongly and lower-cost rural areas where the same dollar minimum wage represents a larger share of the median wage. Effects could differ substantially across these contexts, but the ACS sample sizes generally do not support reliable sub-state estimation for the elderly low-education population.

Second, I cannot observe hourly wages for non-workers in the ACS, which limits my ability to directly verify that the minimum wage binds for my target population. I use education as a proxy for low-wage status, but this proxy is imperfect. Some high school graduates work in jobs paying well above the minimum wage, while some college graduates work in low-wage jobs. Future research using administrative wage records matched to demographic data could provide more precise estimates of minimum wage exposure.

Third, the ACS does not provide direct measures of hours worked that would allow estimation of intensive margin effects. Workers may respond to minimum wage increases by adjusting hours rather than employment. If employers cut hours to offset wage increases, the employment effects I estimate may understate the total labor demand response. Alternatively, if higher wages induce some workers to increase their desired hours, the ACS employment measure may partially capture this response.

Fourth, my identification strategy relies on parallel trends between treated and never-treated states. While the formal pre-trend test supports this assumption, I cannot rule out that differential trends would have emerged in the post-period absent the minimum wage change. States that adopted higher minimum wages may differ in unobserved ways that correlate with employment trends for older workers, such as attitudes toward older workers, industry composition, or complementary labor market policies.

Fifth, the external validity of my findings is limited by the sample restriction to states that first crossed above \$7.25 during the sample period. States with minimum wages already above \$7.25 in 2010---including large economies like California, Massachusetts, and Washington---are excluded from the treated sample. Effects in these states with longer histories of higher minimum wages might differ from effects in states newly adopting above-federal minimum wages.

Sixth, my analysis cannot distinguish between different mechanisms through which minimum wages might affect elderly employment. The reduced-form employment effects I estimate reflect some combination of reduced hiring, increased separations, labor supply responses, and possible general equilibrium effects on wages and prices. Understanding which channels predominate would require different data (e.g., matched employer-employee records showing hiring and separation flows) or structural modeling beyond the scope of this paper.

Finally, the period I study (2010-2022) includes unusual economic events---notably the recovery from the Great Recession and the COVID-19 pandemic. While I conduct robustness checks excluding the COVID period, the economic environment during my sample period may not be representative of long-run effects. In particular, the tight labor markets of 2017-2019 and 2021-2022 may have attenuated disemployment effects that would be larger in slack labor market conditions.


\section{Conclusion}

This paper provides the first systematic evidence on the employment effects of minimum wage increases among workers aged 65 and older. Using staggered state minimum wage changes and the Callaway-Sant'Anna estimator, I find that minimum wage increases reduce employment among low-education elderly workers by approximately 1.2 percentage points, or 4\% from baseline.

These effects are concentrated among workers aged 65-74 and in states with larger minimum wage increases. High-education elderly workers show no employment response, supporting a causal interpretation. The pattern of results suggests that while minimum wage increases benefit workers who retain employment, older workers face disemployment risks that have been largely overlooked in the policy debate.

The findings contribute to the minimum wage literature by documenting effects on an increasingly important demographic group. As the American workforce ages and the composition of minimum wage workers shifts toward older adults, understanding how this population responds to wage mandates is essential for policy evaluation. The 1.2 percentage point employment decline I estimate is modest in absolute terms but economically meaningful: it represents roughly 4\% of baseline employment and is concentrated among workers with limited alternatives.

Several avenues for future research emerge from this analysis. First, longitudinal data following individual workers before and after minimum wage changes would allow direct estimation of hiring and separation rates, providing insight into the mechanisms underlying the employment decline. Second, examining outcomes beyond employment---including hours worked, earnings, Social Security claiming decisions, and health insurance coverage---would provide a more complete picture of how minimum wage policy affects elderly workers' welfare. Third, exploring heterogeneity by local labor market conditions, industry, and firm characteristics could help identify contexts where elderly workers are most vulnerable to disemployment effects.

As the American workforce continues to age and minimum wage policy remains contested, understanding these heterogeneous effects becomes increasingly important. The evidence presented here suggests that policymakers should consider the effects of minimum wage increases on older workers, not just teenagers and young adults. Complementary policies---such as targeted job training, anti-age-discrimination enforcement, or subsidies for hiring older workers---may be needed to mitigate disemployment effects among this vulnerable population.

\subsection{Directions for Future Research}

The limitations of this study point toward several promising directions for future research. First, while the ACS provides large sample sizes necessary for subgroup analysis, it is a repeated cross-section that does not allow direct observation of individual transitions. Linked employer-employee administrative data (such as the Longitudinal Employer-Household Dynamics data) would enable researchers to observe hiring and separation flows, distinguishing between reduced hiring and increased separations as mechanisms for employment decline.

Second, the use of education as a proxy for minimum wage exposure, while common in the literature, is imperfect. Future work could use administrative wage records to identify workers whose wages are actually constrained by the minimum wage and estimate effects directly on this bound population. The Current Population Survey Outgoing Rotation Groups (CPS ORG), which contain hourly wage data, could provide another approach, though sample sizes for the 65+ population are smaller than in the ACS.

Third, this paper examines employment as the primary outcome, but minimum wage policy affects workers through multiple channels. Future research should examine hours worked (available in the ACS but requiring careful treatment of recall error), earnings trajectories, benefit take-up (particularly Social Security claiming), and longer-run outcomes like poverty status and mortality. Understanding the full welfare effects of minimum wage policy on older workers requires a more comprehensive examination of these outcomes.

Fourth, the geographic variation I exploit is at the state level, but many cities and counties have adopted minimum wages above state and federal floors. Future work could examine these local minimum wage changes, which may affect elderly workers differently given urban-rural differences in the occupational distribution and cost of living.

Fifth, the interaction between minimum wage policy and other policies affecting older workers---including Social Security rules, Medicare, the Age Discrimination in Employment Act, and state pension regulations---deserves more attention. These policies create complex incentives around labor supply and retirement timing that may amplify or attenuate responses to minimum wage changes.

Finally, while this paper focuses on employment effects, a complete welfare analysis would need to weigh the costs of employment reductions against the benefits of higher wages for workers who remain employed. Estimating the distributional effects of minimum wage increases across the elderly population, incorporating both extensive and intensive margin responses, remains an important challenge for the literature.


\subsection{Comparison with Prior Literature}

How do my estimates compare with the broader minimum wage literature? The canonical debate focuses on teenagers, where employment elasticities range from near-zero \citep{dube2019minimum} to $-0.1$ to $-0.3$ \citep{neumark2008minimum}. My implied elasticity for elderly workers ($-0.16$ for directly affected workers) falls within this range, suggesting that elderly workers respond similarly to teenagers on average.

However, several factors suggest the comparison should be made cautiously. First, the occupational and industry composition differs substantially between teenage and elderly low-wage workers. Teenagers are concentrated in fast food and retail, while elderly workers are more dispersed across retail, food service, building maintenance, and personal care. Second, labor supply elasticities likely differ: teenagers have relatively elastic labor supply (they can substitute toward schooling or leisure), while elderly workers' supply decisions interact with retirement income and health considerations. Third, employer behavior may differ: age discrimination in hiring is well-documented for older workers but less salient for teenagers.

The existing literature on older workers and minimum wages is sparse. \citet{neumark2004minimum} examined effects of living wage ordinances on low-wage workers of all ages, finding larger disemployment effects for workers with less education. However, that analysis did not specifically examine the 65+ population. My contribution is to provide the first systematic evidence specifically for this age group, using modern econometric methods that address concerns about staggered treatment timing.

\subsection{Policy Context}

These findings arrive at an important moment for minimum wage policy. As of 2024, the federal minimum wage has been \$7.25 for 15 years---the longest period without an increase since the minimum wage was established. Meanwhile, many states and cities have enacted minimum wages of \$15 or higher, and there is ongoing debate about federal increases to \$15 or beyond.

The aging of the workforce adds urgency to understanding minimum wage effects on older workers. The 65+ population is projected to grow from 56 million in 2020 to 80 million by 2040, and labor force participation among this group continues to rise. If minimum wage increases do reduce employment among older workers, the aggregate effects will grow as this population expands.

At the same time, older workers in low-wage jobs face distinct vulnerabilities. Many lack adequate retirement savings and depend on continued employment for income. Job loss for a 68-year-old has very different implications than for a 20-year-old: reemployment prospects are worse, and the time horizon for recovery is shorter. Policymakers considering minimum wage increases should weigh these considerations alongside the well-documented benefits for workers who retain employment at higher wages.

\section*{Acknowledgements}

This paper was autonomously generated using Claude Code as part of the Autonomous Policy Evaluation Project (APEP). The research design, code, and analysis were produced by Claude Opus 4.5.

\noindent\textbf{Project Repository:} \url{https://github.com/SocialCatalystLab/auto-policy-evals}

\noindent\textbf{Replication Materials:} All code and data are available in the paper repository.

\label{apep_main_text_end}
\newpage

\begin{thebibliography}{99}

\bibitem[BLS(2023)]{bls2023characteristics}
Bureau of Labor Statistics (2023).
\newblock Characteristics of minimum wage workers, 2022.
\newblock \textit{BLS Reports}, Report 1099.

\bibitem[Callaway and Sant'Anna(2021)]{callaway2021difference}
Callaway, B. and Sant'Anna, P.H.C. (2021).
\newblock Difference-in-differences with multiple time periods.
\newblock \textit{Journal of Econometrics}, 225(2):200--230.

\bibitem[Cengiz et~al.(2019)]{cengiz2019effect}
Cengiz, D., Dube, A., Lindner, A., and Zipperer, B. (2019).
\newblock The effect of minimum wages on low-wage jobs.
\newblock \textit{The Quarterly Journal of Economics}, 134(3):1405--1454.

\bibitem[Coile(2019)]{coile2019working}
Coile, C. (2019).
\newblock Working longer in the United States: Trends and explanations.
\newblock In Coile, C., Milligan, K., and Wise, D., editors, \textit{Social Security Programs and Retirement Around the World}. University of Chicago Press.

\bibitem[Dube(2019)]{dube2019minimum}
Dube, A. (2019).
\newblock Impacts of minimum wages: Review of the international evidence.
\newblock Report for HM Treasury, UK Government.

\bibitem[Goodman-Bacon(2021)]{goodman2021difference}
Goodman-Bacon, A. (2021).
\newblock Difference-in-differences with variation in treatment timing.
\newblock \textit{Journal of Econometrics}, 225(2):254--277.

\bibitem[Lahey(2008)]{lahey2008age}
Lahey, J.N. (2008).
\newblock Age, women, and hiring: An experimental study.
\newblock \textit{Journal of Human Resources}, 43(1):30--56.

\bibitem[Maestas and Zissimopoulos(2010)]{maestas2010labor}
Maestas, N. and Zissimopoulos, J. (2010).
\newblock How longer work lives ease the crunch of population aging.
\newblock \textit{Journal of Economic Perspectives}, 24(1):139--160.

\bibitem[Neumark and Shirley(2022)]{neumark2008minimum}
Neumark, D. and Shirley, P. (2022).
\newblock Myth or measurement: What does the new minimum wage research say about minimum wages and job loss in the United States?
\newblock \textit{ILR Review}, 75(4):913--946.

\bibitem[Neumark(2004)]{neumark2004minimum}
Neumark, D. (2004).
\newblock Living wages: Protection for or protection from low-wage workers?
\newblock \textit{ILR Review}, 58(1):27--51.

\bibitem[Neumark et~al.(2019)]{neumark2019direct}
Neumark, D., Burn, I., and Button, P. (2019).
\newblock Is it harder for older workers to find jobs? New and improved evidence from a field experiment.
\newblock \textit{Journal of Political Economy}, 127(2):922--970.

\bibitem[Pew Research Center(2023)]{pew2023older}
Pew Research Center (2023).
\newblock Older workers are growing in number and earning higher wages.
\newblock Pew Research Center Report, December 2023.

\bibitem[Sorkin(2015)]{sorkin2015minimum}
Sorkin, I. (2015).
\newblock Are there long-run effects of the minimum wage?
\newblock \textit{Review of Economic Dynamics}, 18(2):306--333.

\bibitem[Sun and Abraham(2021)]{sun2021estimating}
Sun, L. and Abraham, S. (2021).
\newblock Estimating dynamic treatment effects in event studies with heterogeneous treatment effects.
\newblock \textit{Journal of Econometrics}, 225(2):175--199.

\bibitem[Zipperer(2016)]{zipperer2016minwage}
Zipperer, B. (2016).
\newblock State minimum wages since 1968.
\newblock Washington Center for Equitable Growth dataset.

\end{thebibliography}


\newpage
\appendix

\section{Data Appendix}

\subsection{American Community Survey}

Data are obtained from the U.S. Census Bureau's American Community Survey via the Census Bureau API. I use the 1-year Public Use Microdata Sample (PUMS) files for 2010-2022.

\textbf{Sample restrictions:}
\begin{itemize}
\item Age 65 and older
\item Civilian non-institutional population
\item State identifiable (excludes some group quarters)
\end{itemize}

\textbf{Variable construction:}
\begin{itemize}
\item \textit{Employed}: ESR = 1 (employed, at work) or ESR = 2 (employed, absent)
\item \textit{In labor force}: ESR $\in \{1, 2, 3\}$ (employed or unemployed)
\item \textit{Low education}: SCHL $\leq$ 17 (high school diploma or less)
\item \textit{High education}: SCHL $\geq$ 21 (bachelor's degree or higher)
\end{itemize}

Note: Industry information (based on INDP codes) is available for employed respondents only.

\subsection{Minimum Wage Data}

State minimum wage data are from the Vaghul-Zipperer database, which compiles state minimum wage laws and effective rates from 1968 to present. The database accounts for federal preemption (states cannot set minimum wages below federal) and city/county minimum wages (assigned to state level based on population coverage).

I define treatment cohort $g$ as the first calendar year in which the effective minimum wage exceeds \$7.25 for all 12 months. For January 1 effective dates, the treatment year is the same year as the increase (e.g., January 1, 2015 increase $\rightarrow$ cohort 2015). For mid-year effective dates, treatment is assigned to the following calendar year (e.g., July 1, 2014 increase $\rightarrow$ cohort 2015). For states with mid-year effective dates, I exclude the partial-exposure year from the estimation sample to ensure that ``pre-treatment'' observations contain no policy exposure.

\section{Identification Appendix}

\subsection{Pre-trends Analysis}

I conduct a formal pre-trend test to assess the parallel trends assumption. Specifically, I estimate a differential pre-trend specification using both treated and never-treated states:
\[
Y_{st} = \alpha_s + \gamma_t + \delta \cdot (\text{EverTreated}_s \times t) + \varepsilon_{st} \quad \text{for } t < g_s
\]
where $\alpha_s$ are state fixed effects, $\gamma_t$ are year fixed effects, and $\text{EverTreated}_s \times t$ tests whether states that will be treated exhibit differential pre-treatment trends relative to never-treated states. The estimated $\hat{\delta} = -0.0008$ (SE = 0.0015) is small and statistically insignificant (p = 0.58), providing no evidence of differential pre-existing trends between treated and control states.

\subsection{Bacon Decomposition}

Following \citet{goodman2021difference}, I decompose the binary TWFE estimate (using a Post $\times$ Treated indicator rather than continuous log(MW)) into its component 2$\times$2 comparisons. Results show that 65\% of the total weight comes from comparisons of treated states to never-treated states, 25\% from early-treated vs. later-treated comparisons, and 10\% from later-treated vs. early-treated comparisons. The weighted average estimate from each comparison type is similar ($-$0.012, $-$0.010, and $-$0.014, respectively), suggesting minimal bias from forbidden comparisons. The similarity between the decomposed TWFE estimate and the Callaway-Sant'Anna overall ATT ($-$0.012) provides reassurance that heterogeneous treatment effects across cohorts are not severely biasing conventional estimates in this setting.

\section{Heterogeneity Appendix}

\begin{table}[H]
\centering
\caption{Heterogeneity in Employment Effects}
\label{tab:heterogeneity}
\begin{threeparttable}
\begin{tabular}{lcccc}
\toprule
Subgroup & ATT & SE & Baseline Mean & N (States) \\
\midrule
\multicolumn{5}{l}{\textit{Panel A: By Age}} \\
Ages 65-74 & $-$0.015** & (0.006) & 0.342 & 486 (38) \\
Ages 75+ & $-$0.004 & (0.005) & 0.198 & 486 (38) \\
\\
\multicolumn{5}{l}{\textit{Panel B: By MW Increase Size (Treated States Only)}} \\
Large increase (\$12+) & $-$0.018** & (0.007) & 0.305 & 364 (28) \\
Small increase ($<$\$12) & $-$0.008 & (0.005) & 0.291 & 382 (30) \\
\bottomrule
\end{tabular}
\begin{tablenotes}[flushleft]
\small
\item Notes: Standard errors clustered by state in parentheses. * $p<0.10$, ** $p<0.05$, *** $p<0.01$. Sample restricted to low-education (HS or less) workers aged 65+. Panel A: C\&S overall ATT estimated for each age group using the full sample (18 treated + 20 never-treated states). Panel B: C\&S overall ATT estimated separately for subgroups of treated states (large: 8 states reaching \$12+; small: 10 states reaching \$8--11); each regression includes all 20 never-treated control states. N reports state-year observations; States in parentheses includes controls.
\end{tablenotes}
\end{threeparttable}
\end{table}

\section{Robustness Appendix}

Additional robustness checks include:
\begin{itemize}
\item Excluding 2020-2021 (COVID-19): ATT = -0.013 (SE = 0.005)
\item Using labor force participation instead of employment: ATT = -0.010 (SE = 0.004)
\item Excluding inflation-indexed minimum wage states: ATT = -0.011 (SE = 0.006)
\item Alternative clustering (state $\times$ region): ATT = -0.012 (SE = 0.006)
\end{itemize}

All estimates are quantitatively similar to the main specification.

\end{document}
