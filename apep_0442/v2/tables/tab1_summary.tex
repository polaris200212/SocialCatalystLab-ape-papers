\begin{table}[htbp]
\centering
\caption{Summary Statistics: Civil War Veterans in the 1910 Census (1.4\% Oversampled)}
\label{tab:summary}
\begin{tabular}{lcccc}
\hline\hline
 & \multicolumn{3}{c}{Union Veterans} & Confederate \\
\cmidrule(lr){2-4} \cmidrule(lr){5-5}
 & All & Age $<$ 62 & Age $\geq$ 62 & All \\
\hline
N & 3,830 & 220 & 3,610 & 1,491 \\
Mean Age & 69.2 & 57.2 & 69.9 & 68.9 \\
\hline
\multicolumn{5}{l}{\textit{Panel A: Outcome Variables}} \\
Labor Force Participation & 0.540 & 0.805 & 0.524 & 0.723 \\
Has Occupation & 0.554 & 0.823 & 0.538 & 0.726 \\
Farm Occupation & 0.247 & 0.277 & 0.245 & 0.512 \\
Manual Labor & 0.088 & 0.177 & 0.082 & 0.043 \\
Owns Home & 0.652 & 0.559 & 0.658 & 0.685 \\
Household Head & 0.774 & 0.836 & 0.771 & 0.818 \\
\hline
\multicolumn{5}{l}{\textit{Panel B: Pre-Determined Covariates}} \\
Literate & 0.897 & 0.859 & 0.899 & 0.849 \\
Native Born & 0.761 & 0.714 & 0.764 & 0.925 \\
Married & 0.704 & 0.768 & 0.700 & 0.754 \\
Urban & 0.336 & 0.409 & 0.332 & 0.183 \\
White & 0.936 & 0.868 & 0.940 & 0.943 \\
\hline\hline
\multicolumn{5}{p{0.9\textwidth}}{\footnotesize \textit{Notes:}
Summary statistics for Union and Confederate Civil War veterans in the
IPUMS 1910 1.4\% oversampled census (us1910l). Veterans identified via the VETCIVWR variable.
Panel A reports outcome variables; Panel B reports pre-determined covariates.
LFP = labor force participation rate.} \\
\end{tabular}
\end{table}
