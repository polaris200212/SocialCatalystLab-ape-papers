\begin{table}[htbp]
\centering
\caption{Effect of Pension Eligibility on Labor Force Participation: RDD at Age 62}
\label{tab:main_rdd}
\begin{tabular}{lcccc}
\hline\hline
 & (1) & (2) & (3) & (4) \\
 & Linear & Quadratic & Bias-Corrected & Robust \\
\hline
RD Estimate & 0.163 & 0.186 & 0.186 & 0.186 \\
Std. Error & (0.108) & (0.139) & (0.144) & (0.144) \\
$p$-value & 0.130 & 0.182 & 0.195 & 0.195 \\
95\% CI & [-0.048, 0.375] & [-0.087, 0.458] & [-0.096, 0.469] & [-0.096, 0.469] \\
\hline
Bandwidth (left/right) & 4.6 / 4.6 & 7.6 / 7.6 & 4.6 / 4.6 & 4.6 / 4.6 \\
Eff. N (left + right) & 116 + 1082 & 155 + 1903 & 116 + 1082 & 116 + 1082 \\
Total N & 3,666 & 3,666 & 3,666 & 3,666 \\
Kernel & Triangular & Triangular & Triangular & Triangular \\
\hline\hline
\multicolumn{5}{p{0.9\textwidth}}{\footnotesize \textit{Notes:}
Sharp RDD estimates of the effect of crossing the age 62 pension eligibility
threshold on labor force participation among Union Civil War veterans.
Column (1): local linear with MSE-optimal bandwidth.
Column (2): local quadratic.
Column (3): bias-corrected estimate with robust standard errors.
Column (4): robust bias-corrected (same as Column 3; shown for completeness).
Columns (3)--(4) use the same bandwidth as Column (1).
The bias-corrected estimate adjusts the coefficient for estimation bias; robust
standard errors account for additional variability from the bias correction.
Running variable: age. Cutoff: 62.
Full Union veteran sample: $N = 3,666$.} \\
\end{tabular}
\end{table}
