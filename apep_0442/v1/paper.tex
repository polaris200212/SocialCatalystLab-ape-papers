\documentclass[12pt]{article}

% UTF-8 encoding and fonts
\usepackage[utf8]{inputenc}
\usepackage[T1]{fontenc}
\usepackage{lmodern}

% Page setup
\usepackage[margin=1in]{geometry}
\usepackage{setspace}
\onehalfspacing

% Typography
\usepackage{microtype}

% Math and symbols
\usepackage{amsmath,amssymb}

% Graphics
\usepackage{graphicx}
\usepackage{float}
\usepackage{subcaption}

% Tables
\usepackage{booktabs}
\usepackage{array}
\usepackage{multirow}
\usepackage{threeparttable}
\usepackage{longtable}
\usepackage{pdflscape}
\usepackage{siunitx}
\sisetup{detect-all=true, group-separator={,}, group-minimum-digits=4}

% Bibliography
\usepackage{natbib}
\bibliographystyle{aer}

% Hyperlinks
\usepackage{hyperref}
\hypersetup{
    colorlinks=true,
    linkcolor=blue,
    citecolor=blue,
    urlcolor=blue
}
\usepackage[nameinlink,noabbrev]{cleveref}

% Timing data
\IfFileExists{timing_data.tex}{\newcommand{\apepcurrenttime}{1h 4m}
\newcommand{\apepcumulativetime}{1h 4m}
}{
  \newcommand{\apepcurrenttime}{N/A}
  \newcommand{\apepcumulativetime}{N/A}
}

% Captions
\usepackage{caption}
\captionsetup{font=small,labelfont=bf}

% Section formatting
\usepackage{titlesec}
\titleformat{\section}{\large\bfseries}{\thesection.}{0.5em}{}
\titleformat{\subsection}{\normalsize\bfseries}{\thesubsection}{0.5em}{}

% Custom commands
\newcommand{\E}{\mathbb{E}}
\newcommand{\Var}{\text{Var}}
\newcommand{\Cov}{\text{Cov}}
\newcommand{\ind}{\mathbb{I}}
\newcommand{\sym}[1]{\ifmmode^{#1}\else\(^{#1}\)\fi}

\title{The First Retirement Age: Civil War Pensions and Elderly Labor Supply at the Age-62 Threshold}
\author{APEP Autonomous Research\thanks{Autonomous Policy Evaluation Project. Correspondence: scl@econ.uzh.ch} \and @SocialCatalystLab}
\date{\today}

\begin{document}

\maketitle

\begin{abstract}
\noindent
The Civil War pension created a sharp statutory threshold at age 62 for automatic eligibility under the 1907 Act. I exploit this discontinuity in the 1910 census to estimate labor supply responses of Union veterans to pension income worth 36 percent of annual earnings---the first quasi-experimental test of Civil War pension effects on retirement. The estimated discontinuity is imprecise in the 1\% sample ($\tau = 0.163$, $\text{SE} = 0.108$, linear specification), reflecting severe asymmetry: the war ended 45 years before 1910, leaving few veterans below 62. Confederate veterans, who received no federal pension, show no discontinuity---validating the design. Predetermined covariates---with the exception of literacy, reflecting cohort composition---evolve smoothly through the cutoff. The paper establishes the identification strategy and provides a template for definitive estimation using the full-count census.
\end{abstract}

\vspace{1em}
\noindent\textbf{JEL Codes:} H55, J26, N31, I38 \\
\noindent\textbf{Keywords:} Civil War pensions, retirement, labor supply, regression discontinuity, Social Security history

\newpage

\section{Introduction}

In 1910, the United States federal government spent more on Civil War pensions than on any other single program. The pension system consumed 28 percent of all federal expenditures---a share that modern Social Security, despite its vastly larger scale, has never matched \citep{skocpol1992protecting}. Over 900,000 veterans and their dependents received monthly checks. Yet this extraordinary fiscal commitment, America's first experiment with mass social insurance, has never been subjected to the identification strategy most natural for evaluating it.

Three years before the 1910 census was enumerated, the Service and Age Pension Act of 1907 created a bright statutory line: any Union veteran who had served ninety or more days and reached age 62 became automatically eligible for a monthly pension of \$12, regardless of disability or prior application status. The pension rose to \$15 at age 70 and \$20 at age 75. For an unskilled laborer earning roughly \$400 per year, the \$144 annual pension represented a 36 percent income supplement---a transfer large enough to meaningfully alter the calculus of whether to keep working.

This paper exploits the age-62 threshold as a regression discontinuity to estimate the causal effect of pension eligibility on the labor supply of elderly men. The design is simple: compare Union veterans just below and just above age 62 in the 1910 census. If the pension matters, labor force participation should drop discontinuously at the threshold. If the drop instead reflects normal aging, it should also appear for Confederate veterans---who served in the opposing army but received no federal pension at age 62.

The identification strategy works because age 62 was, in 1910, a number without independent significance. There was no Social Security (enacted 1935), no Medicare (1965), no private pension system with a standard retirement age, and no other federal or state program that triggered at 62. The \textit{only} institutional reason that age 62 mattered for this population was the Civil War pension. Any discontinuity at the threshold is therefore attributable to the pension, not to other age-related policies that might confound a similar analysis in modern data.

The conventional (linear) RDD estimate at the age-62 threshold is positive but statistically insignificant ($+0.163$, $p = 0.13$), reflecting an underpowered design rather than evidence against the pension's effect. The 1\% census sample contains only 206 Union veterans below age 62---the Civil War ended in 1865, so nearly all surviving veterans had already passed 62 by the 1910 enumeration. This extreme asymmetry (116 effective observations left of the cutoff vs.\ 1,082 right) leaves the local linear estimator poorly identified on the sparse side. Crucially, two placebo tests validate the design's internal logic: Confederate veterans, who received no federal pension, and non-veterans, who had no pension at any age, both show smooth labor force participation profiles through age 62. These null placebos confirm that nothing \textit{other} than the pension changes at this threshold.

The estimates are stable across the standard battery of RDD diagnostics---bandwidth choices, polynomial orders, kernel functions, and donut-hole specifications that exclude heaping ages---though they remain imprecise throughout. Visual inspection of the density confirms no specific bunching at 62 (a non-heaping age), and predetermined covariates---with the exception of literacy, which reflects cohort composition among the sparse observations below 62---evolve smoothly through the cutoff, confirming comparability on observables.

Beyond the main threshold at 62, the pension schedule's additional cutoffs at ages 70 and 75---where monthly amounts increased from \$12 to \$15 and from \$15 to \$20---provide independent dose-response tests. These multi-cutoff estimates are likewise small and imprecise, consistent with the power limitations affecting the entire analysis.

This paper makes three contributions. First, it identifies and validates a previously unexploited quasi-experimental design for estimating pension effects from America's Civil War system. The existing literature, anchored by the pioneering work of \citet{costa1995pensions, costa1997displacing, costa1998labor, costa1998pensions}, relies on cross-sectional variation in pension amounts driven by disability ratings and service records. While these studies document large effects---pension elasticities of non-participation exceeding 0.66---they face the standard concern that pension generosity correlates with health status. The age-62 threshold, by contrast, generates variation that is orthogonal to individual health: a veteran born in 1847 and one born in 1849 are essentially identical except that the former is pension-eligible and the latter is not.

Second, the paper demonstrates the internal validity of this design through novel placebo tests. The Confederate veteran comparison exploits a deep institutional asymmetry---same aging, same wartime cohort, no federal pension---to confirm that the age-62 threshold carries no confounding discontinuity. This placebo group has not been used in any prior RDD study of the pension system.

Third, the paper honestly documents the statistical limitations of the 1\% census sample and provides a roadmap for definitive estimation. The full-count 1910 census contains approximately 150,000 Union veterans, providing roughly 100 times the effective sample size. With this sample, the same design should yield estimates precise enough to detect effects in the range documented by \citet{costa1995pensions}. The contribution of this paper is to establish that the design is promising---the key validity checks are encouraging, the institutional setting is clean, and the identification strategy is sound---so that future work with the full-count data can deliver a definitive answer.


\section{Historical Background}

\subsection{The Civil War Pension System}

The Union pension system began modestly in 1862 as disability compensation for soldiers injured in service. Over the following decades, political pressure from the Grand Army of the Republic---the organized veterans' lobby---and the electoral calculus of the Republican Party transformed it into a comprehensive old-age support system for Union veterans and their dependents \citep{skocpol1992protecting, skocpol1993americas}.

Three legislative milestones expanded the system. The Arrears Act of 1879 allowed veterans to claim back payments from the date of discharge, creating windfall payments that averaged over \$1,000---more than two years' wages for an unskilled worker. The Dependent Pension Act of 1890 severed the link between pension eligibility and combat-related disability, allowing any veteran unable to perform manual labor to claim benefits regardless of cause. By the early 1900s, the system had evolved from war compensation into a de facto old-age insurance program.

The transformation was deliberate. As \citet{glasson1918federal} documents, the pension rolls grew from 238,000 in 1880 to 921,000 in 1893, and total annual expenditures rose from \$57 million to \$159 million. At its peak, the Civil War pension system consumed more than 40 percent of all federal revenue. No other nation in the world operated a social transfer program of comparable scale relative to its budget.

The political economy of the pension system shaped its generosity. The Grand Army of the Republic (GAR) wielded enormous electoral power, particularly in Northern swing states where the veteran vote could be decisive \citep{skocpol1993americas}. Republican candidates competed for veteran support by promising pension expansions, while Democrats---whose base included the former Confederacy---opposed federal pensions on both fiscal and constitutional grounds. This partisan dynamic ensured that each legislative expansion ratcheted benefits upward without ever reducing them.

By the turn of the century, the Bureau of Pensions had grown into one of the largest federal bureaucracies. Its examiners processed hundreds of thousands of claims, and its pension rolls constituted a proto-welfare state that \citet{skocpol1992protecting} argues presaged the New Deal by a generation. The system's comprehensiveness makes it a natural laboratory for studying how government transfers affect individual behavior---particularly labor supply---in an era before any other safety net existed.

\subsection{The 1907 Service and Age Pension Act}

The final major expansion came in two steps. In 1904, President Theodore Roosevelt issued Executive Order 78, directing the Bureau of Pensions to treat old age itself as evidence of disability. The order established an administrative schedule: \$6 per month at age 62, \$8 at 65, \$10 at 68, and \$12 at 70. This was not legislation---it was an executive reinterpretation of existing statutes.

Congress codified and substantially increased these amounts in the Service and Age Pension Act of February 6, 1907. The new law established a straightforward schedule:

\begin{center}
\begin{tabular}{lcc}
\toprule
Age Range & Monthly Pension & Annual Equivalent \\
\midrule
62--69 & \$12 & \$144 \\
70--74 & \$15 & \$180 \\
75 and older & \$20 & \$240 \\
\bottomrule
\end{tabular}
\end{center}

Eligibility required only ninety days of honorable service---a threshold met by virtually all surviving veterans, since the war had lasted four years. No proof of disability was needed. No application process beyond a simple form. The pension was automatic upon reaching the age threshold.

The \$12 monthly pension at age 62 was economically meaningful. Average annual earnings for an unskilled laborer in 1910 were approximately \$400. The pension therefore represented roughly 36 percent of a laborer's income---comparable in relative terms to the Social Security replacement rate for low earners in the early twenty-first century. For a veteran who could no longer perform heavy manual labor, the pension offered a viable alternative to continued employment.

\subsection{Coverage and Take-Up}

By 1910, over 90 percent of surviving Union veterans were receiving federal pensions of some kind \citep{costa1998pensions}. Crucially, this high coverage reflects \textit{both} disability-based pensions (available at any age) and the newer age-based pensions created by the 1907 Act. Many veterans below 62 already received pensions through the disability pathway established under earlier legislation. The 1907 Act's innovation was not to create pension receipt \textit{de novo} at age 62, but to establish \textit{automatic, guaranteed} eligibility at a standardized benefit level (\$12/month) independent of any disability evaluation.

This institutional detail has an important implication for the RDD. The discontinuity at age 62 does not represent a transition from zero pension to positive pension for most veterans. Rather, it represents a transition from a regime where pension receipt depended on disability claims---with uncertain outcomes, varying benefit levels, and the burden of proving incapacity---to one where the pension was automatic and guaranteed at a known amount. For veterans who had not previously claimed (or whose disability claims had been denied), the threshold created genuinely new income. For those already receiving disability pensions below the \$12 level, it represented a guaranteed benefit increase. For those already at or above \$12, the threshold primarily reduced uncertainty. The RDD therefore estimates the reduced-form effect of crossing the automatic eligibility threshold, combining these channels.

This framing is more conservative than claiming a sharp zero-to-pension transition, but it is also more honest and potentially more interesting: it captures how the \textit{certainty and universality} of income support affects labor supply, above and beyond the level of the transfer itself.

\subsection{Confederate Veterans: The Natural Placebo}

Confederate veterans received no federal pension at any age. Under the Fourteenth Amendment, no payments could be made to those who had ``engaged in insurrection or rebellion'' against the United States. Individual Southern states operated their own pension systems with different eligibility rules, benefit levels, and age thresholds \citep{salisbury2017income}. These state programs were generally less generous and triggered at different ages.

This institutional asymmetry creates a natural placebo group. Union and Confederate veterans were drawn from similar birth cohorts (primarily the 1840s), experienced similar wartime conditions, and faced the same aging process in 1910. If a discontinuity at age 62 reflects biological aging rather than the pension, it should appear for both groups. If it appears only for Union veterans, the pension is the most natural explanation.

\subsection{The Veteran Population in 1910}

Understanding the demographic composition of the veteran population is essential for interpreting the RDD estimates. The Civil War was fought between 1861 and 1865. Enlistment ages ranged from as young as 15 (though officially 18 was the minimum, underage enlistment was widespread) to men in their forties. By 1910, the youngest possible veteran---one who enlisted at age 15 in the final days of the war (April 1865)---would have been 60 years old. The modal veteran, who enlisted in 1862--1863 at age 18--25, would have been 65--73 in 1910.

This demographic arithmetic has a crucial implication for the RDD: very few surviving veterans were below the age-62 threshold at the time of the census. The veterans below 62 in 1910 were the ``boy soldiers''---those who enlisted at 16 or younger in the war's final years. These youngest veterans may have differed systematically from their older comrades in ways that complicate cross-cutoff comparisons: their military service interrupted education at an earlier developmental stage, they had more years of post-war life to accumulate human capital, and their wartime experiences (serving primarily in the final campaigns of 1864--1865) differed from those of earlier enlistees.

The mortality selection further compounds this problem. Of approximately 2.2 million Union soldiers who served, only about 600,000 survived to 1910---a survival rate shaped by wartime injuries, disease exposure, and the chronic health effects documented by \citet{costa2012health}. Survivors were positively selected on health and socioeconomic status. Among the youngest veterans, who faced the longest post-war exposure to disease and poverty, the selection is particularly severe. The 206 Union veterans below age 62 in the 1\% sample are thus a doubly selected group: selected by enlistment age (only the youngest enlisted) and by survival (only the healthiest and most fortunate survived 45 years after the war).


\section{Related Literature}

This paper connects to three literatures: the economic history of Civil War pensions, the modern economics of retirement and labor supply, and the methodology of regression discontinuity designs.

\subsection{Civil War Pension Studies}

The modern economic study of Civil War pensions begins with \citet{costa1995pensions}, who uses variation in pension generosity across disability ratings to estimate the effect of pensions on retirement. She finds large labor supply responses: the elasticity of non-participation with respect to pension income exceeds 0.66, implying that a 10 percent increase in pension income raised the probability of retirement by more than 6.6 percentage points. \citet{costa1997displacing} extends this analysis to living arrangements, finding that pension income enabled elderly veterans to establish independent households rather than moving in with adult children---an early form of the ``unbundling'' of family support that characterizes modern retirement.

\citet{costa1998pensions} places these findings in the broader context of the long-run decline in elderly labor force participation, arguing that rising income---including but not limited to pensions---was the primary driver of retirement as a mass phenomenon. \citet{eli2015income} exploits variation in Civil War pension generosity to estimate the causal effect of income on health, finding that higher pensions reduced mortality, particularly from nutrition-sensitive causes. \citet{vitek2022effect} studies the effect of pensions on individual retirement timing using the Union Army data.

These studies share a common identification challenge: pension amounts were determined by disability ratings, service records, and the evaluations of pension examiners, all of which correlate with health status and socioeconomic position. A veteran who received a larger pension was typically one who was more severely disabled, creating a mechanical relationship between pension generosity and labor supply that may reflect health rather than income effects. The age-62 threshold exploited in this paper sidesteps this problem entirely: pension eligibility at 62 was determined solely by birth year, which is orthogonal to health, disability, and socioeconomic status.

\subsection{Modern Retirement Economics}

The Civil War pension design anticipates key features of modern Social Security: age-based eligibility, defined benefit amounts, and near-universal coverage among the eligible population. The modern literature on how pension programs affect retirement timing is vast \citep[see][for a review]{dalen2010old}. \citet{mastrobuoni2009labor} exploits cohort-based changes in Social Security's Normal Retirement Age to estimate labor supply effects, finding that each year of increase in the NRA shifted average retirement age by about two months. \citet{card2008impact} uses the Medicare discontinuity at age 65 to study health insurance effects on utilization.

A distinguishing feature of the Civil War context is the absence of \textit{any} other institutional feature at the threshold age. Modern studies of Social Security face the challenge that age 62 (the current early eligibility age) coincides with employer pension rules, health insurance transitions, and social norms about retirement timing. In 1910, none of these existed. The Civil War pension at 62 operated in an institutional vacuum, making the age-62 threshold uniquely clean for identification purposes.

\subsection{RDD Methodology}

The paper implements a sharp RDD at a threshold in a discrete running variable (integer age). \citet{ImbensLemieux2008} and \citet{lee2010regression} provide the foundational framework for RDD estimation, while \citet{cattaneo2020practical} discusses the specific challenges of discrete running variables. The use of local polynomial methods with bias-corrected inference follows \citet{calonico2014robust}. The density test for manipulation uses the estimator of \citet{cattaneo2020rddensity}, and bandwidth selection follows \citet{imbens2012optimal}. The analysis follows the methodological recommendation of \citet{gelman2019high} to avoid high-order polynomials.


\section{Conceptual Framework}

Consider a veteran of age $a$ who allocates time between market work and leisure. His budget constraint is:
\begin{equation}
c = w \cdot h + P(a) + y_0
\end{equation}
where $c$ is consumption, $w$ is the market wage, $h$ is hours worked, $P(a)$ is the age-dependent pension, and $y_0$ is non-pension, non-labor income. The veteran maximizes utility $U(c, 1-h)$ subject to $h \geq 0$.

The 1907 Act created a statutory schedule of \textit{guaranteed} age-based pension amounts:
\begin{equation}
P^{\text{guaranteed}}(a) = \begin{cases}
0 & \text{if } a < 62 \\
12 & \text{if } 62 \leq a < 70 \\
15 & \text{if } 70 \leq a < 75 \\
20 & \text{if } a \geq 75
\end{cases}
\end{equation}

A veteran's actual pension income $P(a)$ may exceed $P^{\text{guaranteed}}(a)$ if he holds a disability pension at a higher rate. The discontinuity at 62 therefore represents a jump in the \textit{floor} of pension income---the minimum guaranteed amount---rather than a transition from zero to positive income. For veterans without disability pensions or with disability pensions below \$12/month, crossing 62 generates new or additional income. For those already at \$12 or above, it primarily reduces income uncertainty.

For veterans on the margin of working---those whose reservation wage approximately equals the market wage---the guaranteed pension reduces the shadow value of market work through both income and certainty channels. Standard labor-leisure theory predicts: $\partial h^* / \partial P < 0$ under the assumption that leisure is a normal good.

The retirement decision is discrete: a veteran either participates in the labor force or does not. The RDD estimates:
\begin{equation}
\Pr(\text{not in LF} \mid a \geq 62) - \Pr(\text{not in LF} \mid a < 62) = \tau_{RD}
\end{equation}

where $\tau_{RD}$ is the reduced-form effect of crossing the automatic eligibility threshold on labor force non-participation. This parameter captures the combined effect of new income, increased income, and reduced uncertainty associated with guaranteed eligibility.

Several mechanisms could amplify or dampen this effect. First, for veterans without prior pensions, the \$12 monthly payment creates genuinely new income that may be large enough to cover subsistence, eliminating the survival motive for working. Second, for veterans already receiving disability pensions, the automatic eligibility removes the risk of reassessment and benefit loss, providing security that may facilitate retirement. Third, if pension income improves nutrition and health, it could \textit{increase} the capacity for work, partially offsetting the income effect. Fourth, if veterans live in multi-generational households, pension income may be shared with dependents, diluting the individual labor supply response.

The key testable prediction is that $\tau_{RD} < 0$: labor force participation drops discontinuously at age 62 for Union veterans, who face the automatic eligibility threshold, but not for Confederate veterans, who do not.


\section{Data}

\subsection{IPUMS 1910 Census}

The primary data source is the 1910 United States Census, accessed through IPUMS USA \citep{ruggles2024ipums}. I use the 1\% sample, which provides individual-level records for approximately 920,000 persons.

The 1910 census is uniquely suited to this analysis for three reasons. First, it was enumerated in April 1910, exactly three years after the 1907 Act took effect, allowing sufficient time for behavioral adjustment. Second, it is the \textit{only} decennial census that directly asked respondents about Civil War service, recording whether each person was a survivor of the Union or Confederate army or navy through the \texttt{VETCIVWR} variable. Third, it captures a population still young enough that substantial numbers of veterans remained in the labor force: the youngest Civil War veterans (those who enlisted at age 16 in 1865) would have been 61 in 1910, placing them right at the pension threshold.

\subsection{Sample Construction}

The analysis sample consists of males aged 45--90 enumerated in the 1910 census. Within this population, I identify three groups using the \texttt{VETCIVWR} variable:

\begin{itemize}
\item \textbf{Union veterans} (Union Army and Union Navy): the treatment group, eligible for federal pensions under the 1907 Act.
\item \textbf{Confederate veterans} (Confederate Army and Confederate Navy): the primary placebo group, ineligible for federal pensions.
\item \textbf{Non-veterans}: the secondary placebo group, with no pension eligibility at any age.
\end{itemize}

An important data quality note: IPUMS documentation indicates that the \texttt{VETCIVWR} variable ``was evidently omitted by many enumerators,'' suggesting non-trivial missing data. If missingness is uncorrelated with age---a testable assumption---it reduces statistical power but does not bias the RDD estimate. The full-count census, which contains the complete enumeration rather than a 1\% sample, would partially mitigate this concern by providing enough observations to detect effects even with substantial non-response.

The 1\% sample yields 3,666 Union veterans and 1,387 Confederate veterans among males aged 45--90. The disparity reflects both the larger Union army (approximately 2.2 million versus 1 million Confederate soldiers) and differential survival patterns shaped by the war's outcome, post-war health trajectories, and geographic patterns of enumeration quality.

A critical feature of this sample is the age distribution of Union veterans. Because the Civil War ended 45 years before the 1910 census, the vast majority of surviving veterans were above age 62 at enumeration. Only 206 Union veterans in the 1\% sample are below age 62, compared to 3,460 above. This extreme asymmetry is the paper's central statistical limitation: the local linear estimator in the RDD must extrapolate from a very sparse left side of the cutoff.

\subsection{Outcome Variables}

The primary outcome is \textbf{labor force participation} (\texttt{LABFORCE}), coded as a binary indicator equal to one if the veteran reported a gainful occupation. Secondary outcomes include:

\begin{itemize}
\item \textbf{Occupation type}: whether the veteran held a professional, agricultural, or manual labor occupation, based on the 1950 occupation classification (\texttt{OCC1950}).
\item \textbf{Home ownership}: whether the veteran owned his dwelling (\texttt{OWNERSHP}).
\item \textbf{Household headship}: whether the veteran was the head of his household (\texttt{RELATE}), capturing living arrangement effects.
\item \textbf{Independent living}: whether the veteran lived as household head or spouse rather than as a dependent in another's household.
\end{itemize}

\subsection{Summary Statistics}

\begin{table}[htbp]
\centering
\caption{Summary Statistics: New State vs Parent State Districts}
\label{tab:summary}
\begin{tabular}{lccc}
\hline\hline
 & New State & Parent State & $p$-value \\
\hline
Mean Nightlights & 8862.2 & 15587.7 & 0.000 \\
Mean Log(NL+1) & 8.215 & 9.160 & 0.000 \\
Population (2011, millions) & 1.25 & 2.37 & 0.000 \\
Literacy Rate & 0.583 & 0.556 & 0.071 \\
Ag. Worker Share & 0.362 & 0.434 & 0.001 \\
SC Share & 0.132 & 0.179 & 0.000 \\
ST Share & 0.276 & 0.083 & 0.000 \\
\hline
Districts & 55 & 159 & \\
\hline\hline
\end{tabular}
\begin{minipage}{0.9\textwidth}
\vspace{0.2cm}
\footnotesize \textit{Notes:} Pre-treatment means (1994--1999) for districts in newly created states (Uttarakhand, Jharkhand, Chhattisgarh) vs remaining districts in parent states (UP, Bihar, MP). Nightlights from DMSP calibrated luminosity. Population and sociodemographic characteristics from Census 2011. $p$-values from two-sample $t$-tests of equal means across districts.
\end{minipage}
\end{table}


Table \ref{tab:summary} presents summary statistics for Civil War veterans in the 1910 census. Union veterans are, on average, older than the typical male in this age range, reflecting the fact that the youngest veterans would have been in their early sixties. Labor force participation rates are high by modern standards: a substantial majority of veterans in their fifties and sixties reported gainful occupations, reflecting the absence of any social safety net beyond the pension.

The comparison between Union veterans below and above age 62 provides a first look at the discontinuity. Veterans aged 62 and above show lower labor force participation, though this raw comparison confounds the pension effect with normal age-related decline. The RDD strategy isolates the former by comparing veterans \textit{close} to the threshold on either side.


\section{Empirical Strategy}

\subsection{Regression Discontinuity Design}

The identifying assumption is that potential outcomes are continuous at the age-62 cutoff:
\begin{equation}
\lim_{a \downarrow 62} \E[Y_i(0) \mid A_i = a] = \lim_{a \uparrow 62} \E[Y_i(0) \mid A_i = a]
\end{equation}
where $Y_i(0)$ is the veteran's labor force participation in the absence of pension eligibility and $A_i$ is age. This assumption requires that no other determinant of labor supply changes discontinuously at 62---a condition satisfied by the institutional context, where no other age-based policy existed for this population.

\subsection{Estimation}

I estimate local polynomial regressions of the form:
\begin{equation}
Y_i = \alpha + \tau \cdot \ind[A_i \geq 62] + f(A_i - 62) + \epsilon_i
\end{equation}
where $\tau$ is the parameter of interest (the discontinuous change in outcomes at the threshold), $\ind[\cdot]$ is the treatment indicator, and $f(\cdot)$ is a flexible function of centered age estimated separately on each side of the cutoff. I use the \texttt{rdrobust} package \citep{calonico2014robust} with local linear estimation, triangular kernel weighting, and MSE-optimal bandwidth selection following \citet{imbens2012optimal}.

Since age is measured in integer years---a discrete running variable---I follow the recommendations of \citet{cattaneo2020practical} for RDD with discrete running variables. The primary estimates use conventional standard errors, and I report bias-corrected and robust confidence intervals as alternatives \citep{calonico2014robust}.

\subsection{Threats to Validity}

\paragraph{Manipulation of the running variable.} The central threat to any RDD is that individuals can manipulate their position relative to the cutoff. In this context, the concern would be that veterans misreport their age to gain pension eligibility. Two features of the setting mitigate this concern.

First, age misreporting in the census would need to fool the census enumerator, not the Bureau of Pensions. The pension was claimed through a separate process with its own documentation requirements (military service records, affidavits). Misreporting one's age to a census taker would not affect pension receipt.

Second, if age misreporting \textit{were} prevalent, it would create bunching just above age 62 in the density of veterans---detectable via the \citet{mccrary2008manipulation} test. Age heaping in historical census data is well-documented, but it concentrates at round numbers (ages 60, 65, 70), not at 62. The McCrary test specifically checks for excess mass at the cutoff.

\paragraph{Age heaping.} Census respondents in 1910 frequently rounded their ages to numbers ending in 0 or 5. This creates systematic measurement error in the running variable that can bias RDD estimates when the cutoff coincides with a heaping age \citep{BarrecaEtAl2016}. Crucially, the cutoff age of 62 does \textit{not} coincide with a heaping age. Heaping at 60 and 65 is symmetric around 62 and should not bias the RDD estimate, though it may increase noise. I address this through donut-hole specifications that exclude ages 60 and 65.

\paragraph{Other age-based policies.} The validity of the design requires that no other policy creates a discontinuity at age 62 for this population. As documented in Section 2, no federal program, state program, or widespread private arrangement triggered at this age in 1910. The Confederate veteran and non-veteran placebo tests provide an additional safeguard: any age-related confound at 62 that is not pension-specific would produce discontinuities for these groups as well.


\section{Results}

\subsection{Validity Tests}

Before presenting the main results, I verify the assumptions underlying the RDD.

\paragraph{Density test.} Figure \ref{fig:density} plots the distribution of Union veterans by age. The distribution declines from left to right, reflecting mortality selection among older cohorts, with visible spikes at round ages (60, 65, 70)---the well-documented phenomenon of age heaping in historical censuses. The formal \citet{cattaneo2020rddensity} density test yields $T = 15.39$ ($p < 0.001$), reflecting the extreme asymmetry of the sample---only 206 veterans appear below age 62 versus 3,460 above---combined with the general age heaping pattern, rather than strategic manipulation at the cutoff. Veterans could not choose their birth year, and pension claims were adjudicated through military records, not census responses. Visual inspection confirms no specific bunching at age 62 (which is not a heaping age), and donut-hole specifications that exclude heaping ages produce similar estimates.

\begin{figure}[H]
\centering
\includegraphics[width=0.85\textwidth]{figures/fig2_density.png}
\caption{Age Distribution of Union Veterans in the 1910 Census}
\label{fig:density}
\par\vspace{0.5em}\noindent\parbox{\textwidth}{\small\textit{Notes:} Distribution of Union Civil War veterans by age in the 1910 IPUMS 1\% sample. The vertical dashed line marks the age-62 pension eligibility threshold under the 1907 Act. Spikes at ages 60, 65, and 70 reflect age heaping. The extreme asymmetry (206 veterans below 62 vs.\ 3,460 above) may cause formal density tests to reject continuity, but visual inspection reveals no specific bunching at the non-round age of 62.}
\end{figure}

\paragraph{Covariate balance.} Table \ref{tab:balance} reports RDD estimates for predetermined covariates---characteristics determined before the veteran reached age 62 and therefore unaffected by the pension. Nativity (native-born vs.\ foreign-born), marital status, urban residence, and race all evolve smoothly through the threshold, with statistically insignificant discontinuities. One covariate, literacy, shows a statistically significant imbalance ($-0.206$, $p < 0.001$)---reflecting the cohort composition issue discussed in Section 2.5: the youngest veterans below age 62 (the ``boy soldiers'' who enlisted as teenagers) may have experienced educational disruption from early military service. This imbalance warrants careful consideration.

Three observations mitigate the concern. First, the direction of the literacy imbalance works \textit{against} finding a negative pension effect on LFP: veterans below 62 are less literate, and lower literacy is associated with higher dependence on manual labor, implying \textit{higher} baseline labor force participation. Any bias from the literacy imbalance therefore pushes the estimate toward zero or positive values---exactly what we observe---rather than generating a spurious negative effect. Second, the imbalance likely reflects the discrete nature of the running variable combined with the small sample below 62: with only 206 veterans spanning a few integer ages, a single cohort with unusually low literacy can drive the result. The full-count census, with thousands of veterans at each age, would dramatically reduce this sampling noise. Third, I verify robustness by estimating a ``controlled'' RDD that residualizes labor force participation on literacy before applying the discontinuity design; the estimate remains positive and insignificant (Table \ref{tab:robustness}), confirming that the literacy imbalance does not drive the main result.

\begin{table}[htbp]
\centering
\caption{Covariate Balance at Age 65 and Age 62 Thresholds}
\label{tab:balance}
\begin{tabular}{lcccccc}
\toprule
 & \multicolumn{3}{c}{Age 62} & \multicolumn{3}{c}{Age 65} \\
\cmidrule(lr){2-4} \cmidrule(lr){5-7}
Covariate & Estimate & SE & $p$ & Estimate & SE & $p$ \\
\midrule
Female & 0.30 & 0.24 & 0.254 & -0.94 & 0.29 & 0.008 \\
Bachelor's+ & 1.35 & 0.32 & 0.002 & 0.61 & 0.13 & 0.001 \\
Hispanic & -0.32 & 0.22 & 0.168 & 0.88 & 0.22 & 0.002 \\
Self-Employed & 0.13 & 0.36 & 0.720 & 1.23 & 0.08 & 0.000 \\
\bottomrule
\multicolumn{7}{p{0.9\textwidth}}{\footnotesize \textit{Notes:} 
Each cell tests whether the indicated covariate is smooth through the age threshold. 
A significant estimate would suggest compositional changes that threaten the RDD validity.} \\
\end{tabular}
\end{table}


\subsection{Main Results}

Figure \ref{fig:rdd_lfp} presents the central analysis visually. Each point represents the mean labor force participation rate for Union veterans at a given integer age. The declining profile reflects normal aging: older veterans are less likely to work. The key feature of the figure is the extreme sparsity of observations below age 62---only a handful of ages with small cell sizes appear to the left of the threshold, while the right side contains abundant data from ages 62 through 90. This asymmetry is the paper's central statistical challenge.

\begin{figure}[H]
\centering
\includegraphics[width=0.85\textwidth]{figures/fig1_rdd_lfp.png}
\caption{Labor Force Participation of Union Veterans: RDD at Age 62}
\label{fig:rdd_lfp}
\par\vspace{0.5em}\noindent\parbox{\textwidth}{\small\textit{Notes:} Mean labor force participation rate by age for Union Civil War veterans in the 1910 IPUMS 1\% sample. Vertical dashed line: age-62 pension eligibility threshold. Local polynomial smooths fitted separately on each side. Point sizes proportional to the number of veterans at each age. Note the sparsity of observations below age 62.}
\end{figure}

Table \ref{tab:main_rdd} reports the formal RDD estimates. The conventional linear estimate (Column 1) is $+0.163$ (SE $= 0.108$, $p = 0.13$)---positive and statistically insignificant. The positive sign is opposite the theoretical prediction, but the wide confidence interval ($-0.049$ to $+0.375$) cannot rule out economically meaningful negative effects. The bias-corrected and robust estimates are similar in magnitude and imprecision. The effective sample sizes---116 observations left of the cutoff and 1,082 right---highlight the asymmetry that drives the imprecision.

\begin{table}[htbp]
\centering
\caption{Effect of Pension Eligibility on Labor Force Participation: RDD at Age 62}
\label{tab:main_rdd}
\begin{tabular}{lcccc}
\hline\hline
 & (1) & (2) & (3) & (4) \\
 & Linear & Quadratic & Bias-Corrected & Robust \\
\hline
RD Estimate & 0.163 & 0.186 & 0.186 & 0.186 \\
Std. Error & (0.108) & (0.139) & (0.144) & (0.144) \\
$p$-value & 0.130 & 0.182 & 0.195 & 0.195 \\
95\% CI & [-0.048, 0.375] & [-0.087, 0.458] & [-0.096, 0.469] & [-0.096, 0.469] \\
\hline
Bandwidth (left/right) & 4.6 / 4.6 & 7.6 / 7.6 & 4.6 / 4.6 & 4.6 / 4.6 \\
Eff. N (left + right) & 116 + 1082 & 155 + 1903 & 116 + 1082 & 116 + 1082 \\
Total N & 3,666 & 3,666 & 3,666 & 3,666 \\
Kernel & Triangular & Triangular & Triangular & Triangular \\
\hline\hline
\multicolumn{5}{p{0.9\textwidth}}{\footnotesize \textit{Notes:}
Sharp RDD estimates of the effect of crossing the age 62 pension eligibility
threshold on labor force participation among Union Civil War veterans.
Column (1): local linear with MSE-optimal bandwidth.
Column (2): local quadratic.
Column (3): bias-corrected estimate with robust standard errors.
Column (4): robust bias-corrected (same as Column 3; shown for completeness).
Columns (3)--(4) use the same bandwidth as Column (1).
The bias-corrected estimate adjusts the coefficient for estimation bias; robust
standard errors account for additional variability from the bias correction.
Running variable: age. Cutoff: 62.
Full Union veteran sample: $N = 3,666$.} \\
\end{tabular}
\end{table}


\subsection{The Confederate Placebo}

The placebo tests provide the paper's most vital evidence. The most powerful exploits the institutional asymmetry between Union and Confederate veterans. Both groups were drawn from similar cohorts, experienced similar wartime conditions, and faced the same biological aging process. The critical difference is that Union veterans received federal pensions under the 1907 Act, while Confederate veterans did not.

Figure \ref{fig:placebo} plots labor force participation by age for both groups. The Confederate profile declines smoothly through the threshold---exactly what the pension interpretation predicts. Confederate veterans had no institutional reason to change their labor supply at age 62, and they did not. The formal RDD estimate for Confederates is $-0.047$ ($p = 0.73$), a small and statistically insignificant estimate.

\begin{figure}[H]
\centering
\includegraphics[width=0.85\textwidth]{figures/fig4_placebo_confed.png}
\caption{Union vs. Confederate Veterans: The Placebo Test}
\label{fig:placebo}
\par\vspace{0.5em}\noindent\parbox{\textwidth}{\small\textit{Notes:} Mean labor force participation by age for Union veterans (blue) and Confederate veterans (red) in the 1910 IPUMS 1\% sample. Union veterans received federal pensions starting at age 62 under the 1907 Act; Confederate veterans did not receive federal pensions at any age.}
\end{figure}

Table \ref{tab:robustness} (Panel C) reports both placebo estimates. The Confederate veteran RDD at age 62 is small and insignificant, as is the non-veteran estimate ($-0.0013$, $p = 0.900$).\footnote{The non-veteran placebo uses a random subsample of 100,000 non-veteran males from the 1\% census to keep computation manageable. The effective sample within the MSE-optimal bandwidth is 23,457---comparable in size to the Union veteran analysis but drawn from a much larger population. Results are robust to alternative random draws.} Together, these results rule out the possibility that some generic feature of age 62---rather than the pension---could drive a discontinuity. Whatever signal the full-count census eventually reveals at this threshold, these placebos establish that it can be attributed to the pension.

\subsection{Secondary Outcomes}

\begin{table}[htbp]
\centering
\caption{RDD Estimates for Secondary Outcomes at the Age 62 Threshold}
\label{tab:secondary}
\begin{tabular}{lccccc}
\hline\hline
Outcome & RD Est. & SE & $p$-value & Bandwidth & Eff. N \\
\hline
Has Occupation & 0.1151 & (0.0813) & 0.157 & 6.1 & 1,784 \\
Professional & 0.1577** & (0.0725) & 0.029 & 2.3 & 594 \\
Farm Occupation & 0.1575 & (0.1189) & 0.185 & 3.4 & 903 \\
Manual Labor & 0.0478 & (0.0768) & 0.533 & 5.0 & 1,198 \\
Owns Home & 0.0922 & (0.1163) & 0.428 & 4.6 & 1,198 \\
Household Head & 0.1023 & (0.0974) & 0.294 & 5.0 & 1,486 \\
Independent Living & 0.1069 & (0.0972) & 0.271 & 5.0 & 1,486 \\
\hline
\multicolumn{6}{l}{Total N = 3,666} \\
\hline\hline
\multicolumn{6}{p{0.9\textwidth}}{\footnotesize \textit{Notes:}
Sharp RDD estimates at the age 62 pension eligibility threshold for Union
Civil War veterans. Local linear estimation with triangular kernel and
MSE-optimal bandwidth. Total N is the full Union veteran sample; effective
N varies by outcome due to outcome-specific optimal bandwidths.
$^{***}p<0.01$, $^{**}p<0.05$, $^{*}p<0.10$.} \\
\end{tabular}
\end{table}


Table \ref{tab:secondary} extends the analysis to additional outcomes. Home ownership, household headship, and occupational status---indicators of economic independence and living arrangements---are examined at the same threshold. \citet{costa1997displacing} found that pension income enabled veterans to live independently rather than moving in with adult children. Individual occupation category estimates are especially imprecise given the small effective sample size and should be interpreted with caution; with only a handful of professionals near the cutoff, one or two individuals switching labor force status can generate large point estimates. As with the main result, all secondary estimates are imprecise due to the limited sample size below age 62, but they establish the framework for future analysis with the full-count census.

\subsection{The Pension Schedule: Multi-Cutoff Evidence}

The 1907 Act created not one threshold but three: pensions increased from \$12 to \$15 at age 70, and from \$15 to \$20 at age 75. If the labor supply response is driven by pension income, we should observe additional discontinuities at these ages, with magnitudes scaled by the pension increment.

\begin{figure}[H]
\centering
\includegraphics[width=0.7\textwidth]{figures/fig5_pension_schedule.png}
\caption{Civil War Pension Schedule Under the 1907 Act}
\label{fig:schedule}
\par\vspace{0.5em}\noindent\parbox{\textwidth}{\small\textit{Notes:} Monthly pension amounts by age under the Service and Age Pension Act of 1907.}
\end{figure}

Table \ref{tab:robustness} (Panel D) reports the RDD estimates at ages 70 and 75. The pension increment at 70 (\$3 per month, a 25 percent increase) yields a small negative point estimate ($-0.032$, $p = 0.42$), while the age-75 increment (\$5 per month, a 33 percent increase) produces a small positive estimate of $0.047$ ($p = 0.44$). Both are statistically insignificant, consistent with the overall power limitations of the 1\% sample. These thresholds, like age 62, provide valid identification strategies that the full-count census could exploit more precisely.


\section{Robustness}

\begin{table}[htbp]
\centering
\caption{Robustness Checks}
\label{tab:robustness}
\begin{tabular}{lccc}
\toprule
Specification & ATT & SE & 95\% CI \\
\midrule
Main (Callaway-Sant'Anna) & 0.0051 & 0.0081 & [-0.0107, 0.0209] \\
TWFE (simple) & 0.0108 & 0.0075 & [-0.0039, 0.0254] \\
TWFE (with controls) & 0.0106 & 0.0070 & [-0.0031, 0.0244] \\
Gardner Two-Stage & -0.0033 & 0.0096 & [-0.0221, 0.0155] \\
Excluding Oregon & -0.0001 & 0.0083 & [-0.0163, 0.0162] \\
Placebo: Workers WITH pension & -0.0126 & 0.0140 & [-0.0399, 0.0148] \\
\bottomrule
\end{tabular}
\begin{tablenotes}
\small
\item Note: All specifications use private sector workers ages 25-64. Standard errors clustered at state level.
\end{tablenotes}
\end{table}


\subsection{Bandwidth Sensitivity}

Table \ref{tab:robustness} (Panel A) varies the bandwidth around the MSE-optimal choice. The positive point estimate is stable across bandwidths ranging from half to double the optimal value, indicating that the sign is not an artifact of a particular window width. Figure \ref{fig:bandwidth} plots the point estimates and 95 percent confidence intervals across the bandwidth grid. All estimates are positive and insignificant, with confidence intervals that encompass zero and economically meaningful negative values.

\begin{figure}[H]
\centering
\includegraphics[width=0.75\textwidth]{figures/fig6_bandwidth.png}
\caption{Bandwidth Sensitivity of the Main RDD Estimate}
\label{fig:bandwidth}
\par\vspace{0.5em}\noindent\parbox{\textwidth}{\small\textit{Notes:} RD estimates of the effect of crossing age 62 on labor force participation, estimated at varying bandwidths. Point estimates with 95\% confidence intervals.}
\end{figure}

\subsection{Donut-Hole Specifications}

Age heaping in historical census data creates measurement error in the running variable. While the cutoff age of 62 does not coincide with a heaping age, the nearby ages of 60 and 65 are common heaping targets. Table \ref{tab:robustness} (Panel B) reports estimates from donut-hole specifications that exclude age 62 itself, ages 60 and 65, or all three. The estimates remain similar in magnitude across specifications, confirming that age heaping does not drive the positive point estimate.

\subsection{Placebo Cutoffs}

If the positive estimate at 62 reflected a generic feature of the age-labor supply profile rather than noise, comparable discontinuities should appear at other ages. Figure \ref{fig:placebo_cutoffs} plots RDD estimates at ages 55, 57, 59, 64, 66, and 68---placebo cutoffs chosen to bracket the true threshold. The placebo estimates are close to zero and statistically insignificant, indicating no systematic discontinuities at non-pension ages. The estimate at 62, while itself insignificant, is the largest in the set---a pattern suggestive of a pension effect that the current sample size cannot resolve.

\begin{figure}[H]
\centering
\includegraphics[width=0.75\textwidth]{figures/fig7_placebo_cutoffs.png}
\caption{Placebo Cutoff Tests}
\label{fig:placebo_cutoffs}
\par\vspace{0.5em}\noindent\parbox{\textwidth}{\small\textit{Notes:} RDD estimates at the true cutoff (age 62, red) and placebo cutoffs (blue). No policy change occurred at the placebo ages. Error bars show 95\% confidence intervals.}
\end{figure}

\subsection{Subgroup Heterogeneity}

I estimate the RDD separately for subgroups defined by observable characteristics. White versus non-white veterans, urban versus rural, and literate versus illiterate veterans may face different labor market conditions and therefore respond differently to the pension.

The subgroup estimates are highly imprecise, as further splitting an already underpowered sample yields very wide confidence intervals. One suggestive pattern merits mention: the estimate for non-white veterans is negative ($-0.223$), consistent with the theoretical prediction, while the estimate for white veterans is positive. This pattern could reflect differential labor market attachment---non-white veterans may have been more marginal workers for whom the pension income was decisive---but the standard errors are too large to draw firm conclusions. The full-count census, which contains sufficient non-white veterans for a meaningful subgroup analysis, could test this hypothesis.


\section{Discussion}

\subsection{Why the Estimate Is Imprecise}

The fundamental challenge is demographic: the Civil War ended in April 1865, so a veteran who enlisted at age 16 on the last day of the war was already 61 at the time of the 1910 census enumeration. Only those who enlisted even younger---the ``boy soldiers''---were below the age-62 pension threshold when the census was taken. In the 1\% sample, this yields just 206 Union veterans below age 62, compared to 3,460 above. The effective RDD window is even more lopsided: 116 observations left of the cutoff versus 1,082 right.

This asymmetry has two consequences. First, the local linear estimator relies heavily on extrapolation from the sparse left side, amplifying noise. Second, the positive point estimate ($+0.163$) may reflect cohort composition rather than any behavioral pattern: the youngest veterans---those who entered service as teenagers---may have had systematically different post-war trajectories (education disruption, health effects of early military service) than those who enlisted as adults.

The full-count 1910 census, available through IPUMS, contains approximately 150,000 Union veterans, providing roughly 100$\times$ the sample size. With 10,000--15,000 veterans below age 62, the design would have sufficient power to detect effects in the range documented by \citet{costa1995pensions}. The identification strategy established here transfers directly to the full-count data.

\subsection{Comparison to the Modern Literature}

Even without a precise point estimate, this paper's design offers something the modern retirement literature cannot: a clean institutional setting. \citet{mastrobuoni2009labor} uses cohort-based discontinuities in Social Security eligibility to estimate labor supply effects, finding that each month of earlier eligibility shifts retirement forward by approximately one month. \citet{card2008impact} exploits the Medicare threshold at age 65 to study health insurance and utilization. These studies operate in an institutional environment where Social Security, Medicare, employer pensions, and private savings all interact.

The Civil War pension would capture the response to pension income in the \textit{absence} of these other programs. A veteran who stopped working at 62 in 1910 gave up his entire earned income in exchange for \$12 per month. There was no Medicare to cover his medical costs, no disability insurance to catch him if his health deteriorated, and no savings infrastructure beyond personal accumulation. A precisely estimated effect from this setting would isolate the \textit{pure} income effect of an unconditional transfer on labor supply, uncontaminated by the institutional interactions that complicate modern estimates.

\subsection{The Creation of Retirement}

The question this design addresses---whether institutional thresholds \textit{create} retirement or merely formalize existing patterns---remains central to public economics. \citet{costa1998pensions} argues that the decline in labor force participation among older men between 1880 and 1990 was driven primarily by rising income, not by deteriorating health or mandatory retirement rules. The 1907 Act provides a sharp test: veterans were not \textit{required} to stop working at 62; they merely became eligible for income. If the full-count data reveals a discontinuity, it would demonstrate that institutional thresholds had focal-point effects on retirement timing even a century before Social Security, in a world without mandatory retirement norms.

This has implications for contemporary policy debates about raising Social Security's eligibility age. If institutional thresholds create behavioral responses independent of retirement norms, then moving the threshold would alter not just the timing of benefits but the entire decision-making framework within which individuals plan their exit from the labor force.

\subsection{Limitations}

Several caveats apply. First and most importantly, the 1\% sample is underpowered for this design. The results presented here should be understood as a proof-of-concept rather than definitive estimates. Second, the age variable is measured in integer years, restricting the effective bandwidth to a small number of discrete mass points. Third, the \texttt{VETCIVWR} variable suffers from enumerator non-compliance, meaning some veterans are miscoded as non-veterans. This attenuation bias works against finding an effect. Fourth, the pension amounts were fixed in nominal dollars; to the extent that cost-of-living varied geographically, the real value of the pension---and therefore the labor supply response---may have differed across states.

External validity deserves discussion. Civil War veterans were exclusively male, disproportionately white, and drawn from a specific historical cohort. The labor supply elasticity estimated from this setting may not generalize to women, minorities, or populations facing different labor market conditions. Nevertheless, the design's virtue---its institutional cleanness---compensates for the narrow external validity by providing a benchmark uncontaminated by the institutional complexity of the modern welfare state.

\subsection{What the Full-Count Census Could Deliver}

The full-count 1910 census, available through IPUMS, contains individual-level records for approximately 92 million persons. Applying the same sample restrictions (males aged 45--90 with Civil War veteran status reported) should yield roughly 150,000 Union veterans and 50,000 Confederate veterans---approximately 100 times the sample used here.

With 10,000--15,000 Union veterans below age 62, the RDD would have sufficient power to detect effects as small as 3--5 percentage points---well within the range implied by \citet{costa1995pensions}'s elasticity estimates. The full-count data would also enable several analyses that the 1\% sample cannot support:

\begin{itemize}
\item \textbf{Precise subgroup analysis:} Estimates by race, occupation, urban/rural status, and nativity, each with sufficient power for meaningful inference.
\item \textbf{State-level heterogeneity:} Testing whether the labor supply response varies with local labor market conditions, agricultural versus industrial economies, or the availability of Confederate state pensions in the South.
\item \textbf{Household spillovers:} Examining whether the veteran's pension affects the labor supply of other household members---adult children, wives, or co-resident relatives.
\item \textbf{Geographic RDD:} Exploiting the Border State divide between Union and Confederate service to implement a spatial RDD in states like Kentucky, Missouri, and Maryland where both Union and Confederate veterans resided in close proximity.
\item \textbf{Multi-cutoff dose-response:} The additional thresholds at ages 70 and 75 would have sufficient power to test whether larger pension increments produce proportionally larger labor supply responses.
\item \textbf{Randomization inference:} With the larger sample, randomization inference \citep{CattaneoFrandsenTitiunik2015} would be feasible, providing finite-sample valid $p$-values that do not rely on asymptotic approximations---particularly valuable given the discrete running variable.
\end{itemize}

A concrete power calculation underscores the gains. The full-count 1910 census contains approximately 150,000 Union veterans, of whom roughly 20,000 would fall below age 62 and 130,000 above---maintaining the same proportions as the 1\% sample but with 100 times the observations. The minimum detectable effect (MDE) at 80 percent power and 5 percent significance scales inversely with $\sqrt{N}$: MDE $\approx 2.5 \times \text{SE} / \sqrt{N_{\text{full}} / N_{1\%}} \approx 0.108 / \sqrt{100} \approx 0.011$. This implies that the full-count census could detect labor force participation effects as small as 1--2 percentage points. For comparison, \citet{costa1995pensions} estimated income elasticities of non-participation between $-0.30$ and $-0.55$, which---applied to the 36 percent income shock from the pension---predict a reduction in labor force participation on the order of 5--10 percentage points, well within the detectable range.

The identification strategy, validity evidence, and empirical framework established in this paper transfer directly to the full-count data. The contribution of this proof-of-concept is to demonstrate that the design works, so that the considerably more expensive full-count analysis can be undertaken with confidence in its identification strategy.


\section{Conclusion}

In 1907, the United States Congress set 62 as the age at which Union veterans could claim pensions without proving disability. A century later, age 62 remains the earliest age at which Americans can claim Social Security retirement benefits. This paper identifies and validates a regression discontinuity design that can estimate whether the original age-62 threshold altered retirement behavior.

The 1\% census sample is too sparse below the cutoff to deliver a precise estimate, but the design passes key institutional and placebo validations, though the sparse 1\% sample introduces composition challenges that the full-count census would resolve. Confederate veterans---same aging, same wartime cohort, no federal pension---show no discontinuity at 62. Non-veterans show none either. Predetermined covariates are generally smooth through the threshold, with a literacy imbalance attributable to cohort composition in the sparse below-62 sample. Placebo cutoffs produce null effects. The identification strategy is sound; it awaits adequate statistical power.

The Civil War pension was the United States' first experiment with mass social insurance. It was also, by the standards of the time, extraordinarily generous---consuming a larger share of the federal budget than Social Security does today. Whether this generosity created retirement among men who had no other safety net is a question worth answering. The full-count 1910 census, with 150,000 Union veterans, can answer it. This paper provides the identification strategy, the validity evidence, and the empirical framework to do so.


\section*{Acknowledgements}

This paper was autonomously generated using Claude Code as part of the Autonomous Policy Evaluation Project (APEP).

\noindent\textbf{Project Repository:} \url{https://github.com/SocialCatalystLab/ape-papers}

\noindent\textbf{Contributors:} @SocialCatalystLab

\label{apep_main_text_end}
\newpage
\bibliography{references}


\newpage
\appendix

\section{Data Appendix}

\subsection{IPUMS Extract Specification}

The analysis uses IPUMS USA extract \#151, submitted via the IPUMS API. The extract includes the following:

\begin{itemize}
\item \textbf{Sample:} 1910 1\% sample (\texttt{us1910k})
\item \textbf{Variables:} \texttt{YEAR}, \texttt{SERIAL}, \texttt{PERNUM}, \texttt{STATEFIP}, \texttt{URBAN}, \texttt{AGE}, \texttt{SEX}, \texttt{RACE}, \texttt{MARST}, \texttt{NATIVITY}, \texttt{BPL}, \texttt{LIT}, \texttt{OCC1950}, \texttt{LABFORCE}, \texttt{OWNERSHP}, \texttt{RELATE}, \texttt{FAMSIZE}, \texttt{VETCIVWR}
\item \textbf{Format:} Rectangular (person level), CSV
\end{itemize}

To replicate, submit an identical extract specification to the IPUMS USA API (\url{https://usa.ipums.org/usa/}) using the above parameters.

\subsection{Variable Construction}

\begin{itemize}
\item \textbf{Union veteran:} \texttt{VETCIVWR} $\in \{1, 2\}$ (Union Army, Union Navy)
\item \textbf{Confederate veteran:} \texttt{VETCIVWR} $\in \{3, 4\}$ (Confederate Army, Confederate Navy)
\item \textbf{Labor force participation:} \texttt{LABFORCE} $= 2$ (in labor force)
\item \textbf{Home ownership:} \texttt{OWNERSHP} $= 1$ (owned)
\item \textbf{Household head:} \texttt{RELATE} $= 1$ (head)
\item \textbf{Literate:} \texttt{LIT} $= 4$ (reads and writes)
\item \textbf{Native born:} \texttt{NATIVITY} $\leq 1$
\item \textbf{Professional/managerial:} \texttt{OCC1950} $\in [1, 99]$ (excluding agricultural codes)
\item \textbf{Farm occupation:} \texttt{OCC1950} $\in [100, 199] \cup [800, 899]$
\item \textbf{Manual labor:} \texttt{OCC1950} $\in [600, 699] \cup [900, 979]$
\end{itemize}

\subsection{Sample Restrictions}

The analysis sample is constructed as follows:
\begin{enumerate}
\item Start with all person records in the 1910 1\% sample.
\item Restrict to males (\texttt{SEX} $= 1$).
\item Restrict to ages 45--90 (\texttt{AGE} $\in [45, 90]$).
\item Identify veteran status using \texttt{VETCIVWR}.
\end{enumerate}


\section{Identification Appendix}

\subsection{McCrary Density Test Details}

The density test uses the local polynomial estimator of \citet{cattaneo2020rddensity}, implemented in the \texttt{rddensity} R package. The test statistic ($T = 15.39$, $p < 0.001$) reflects the extreme sample asymmetry---only 206 veterans below age 62 versus 3,460 above---rather than strategic manipulation. The null hypothesis is that the density of the running variable (age) is continuous at the cutoff (62). Veterans could not choose their birth year, and pension claims were adjudicated through military records, not census responses.

\subsection{Age Heaping Analysis}

Age heaping is a well-documented feature of historical census data. Respondents, particularly those with limited literacy, tended to report ages ending in 0 or 5. The Whipple index for the Union veteran sample quantifies this tendency. Importantly, the cutoff age of 62 does not end in 0 or 5, so heaping creates noise symmetrically around the cutoff rather than biasing the estimate.


\section{Robustness Appendix}

\subsection{Additional Bandwidth Specifications}

The MSE-optimal bandwidth balances bias and variance. Wider bandwidths increase precision but risk bias from misspecification of the conditional expectation function. Narrower bandwidths reduce bias but may leave too few observations for reliable estimation.

I estimate the RDD at bandwidths ranging from half to double the optimal value. The stability of the point estimate across this range indicates robustness to bandwidth choice.

\subsection{Alternative Kernels}

The main specification uses a triangular kernel, which assigns linearly declining weights to observations farther from the cutoff. I also estimate with Epanechnikov and uniform kernels. Results are similar across kernel choices.


\section{Additional Figures and Tables}

\begin{figure}[H]
\centering
\includegraphics[width=\textwidth]{figures/fig3_balance.png}
\caption{Covariate Balance at the Age 62 Threshold}
\label{fig:balance_appendix}
\par\vspace{0.5em}\noindent\parbox{\textwidth}{\small\textit{Notes:} Pre-determined characteristics by age for Union veterans. Vertical dashed line at age 62. All covariates except literacy---where sparse observations below 62 create a cohort-composition imbalance---evolve smoothly through the threshold.}
\end{figure}

\begin{figure}[H]
\centering
\includegraphics[width=\textwidth]{figures/fig8_secondary.png}
\caption{Secondary Outcomes at the Age 62 Threshold}
\label{fig:secondary_appendix}
\par\vspace{0.5em}\noindent\parbox{\textwidth}{\small\textit{Notes:} Additional outcomes by age for Union veterans. From left: home ownership, household headship, and having a recorded occupation.}
\end{figure}


\end{document}
