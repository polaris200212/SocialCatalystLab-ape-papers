\begin{table}[htbp]
\centering
\caption{Alternative Outcomes, Heterogeneity, and Additional RDD Specifications}
\label{tab:alt_outcomes}
\small
\begin{tabular}{lccccc}
\toprule
 & Estimate & Robust SE & $p$-value & N(left) & N(right) \\
\midrule
\multicolumn{6}{l}{\textit{Panel A: Alternative Outcomes}} \\
Total Clean Energy (all years) & -39.921 & 15.638 & 0.011 & 12 & 9 \\
Pre-IRA Clean Energy (placebo) & -33.934 & 15.608 & 0.03 & 10 & 9 \\
N Proposed Generators & -1.01 & 0.816 & 0.216 & 28 & 15 \\
\midrule
\multicolumn{6}{l}{\textit{Panel B: Heterogeneity by Area Type}} \\
MSA Areas & -7.447 & 5.259 & 0.157 & 18 & 9 \\
Non-MSA Areas & -0.894 & 1.097 & 0.415 & 24 & 8 \\
\midrule
\multicolumn{6}{l}{\textit{Panel C: Additional Specifications}} \\
Bivariate RDD (OLS) & -2.262 & 1.401 & 0.115 & \multicolumn{2}{c}{N = 43} \\
Unemployment Margin RDD & -0.495 & 3.42 & 0.885 & 8 & 12 \\
\bottomrule
\end{tabular}
\begin{minipage}{0.95\textwidth}
\vspace{0.3em}
\footnotesize
\textit{Notes:} Panel A reports RDD estimates for alternative outcome variables. Total clean energy includes all operational generators (not just post-IRA). Pre-IRA clean energy is a placebo test using generators with operating years before 2023 (prior to IRA implementation). N proposed generators counts EIA Form 860 proposed projects. Panel B splits the sample by metropolitan (MSA) vs.\ non-metropolitan (non-MSA) areas. Panel C reports the bivariate RDD (OLS, both FF employment and unemployment thresholds, full sample) and the unemployment margin RDD (among areas with FF employment $\geq 0.17\%$). All estimates except the bivariate RDD use robust bias-corrected inference from \texttt{rdrobust}.
\end{minipage}
\end{table}
