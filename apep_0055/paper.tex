\documentclass[12pt]{article}

% UTF-8 encoding and fonts
\usepackage[utf8]{inputenc}
\usepackage[T1]{fontenc}
\usepackage{lmodern}

% Page setup
\usepackage[margin=1in]{geometry}
\usepackage{setspace}
\onehalfspacing

% Math and symbols
\usepackage{amsmath,amssymb}

% Graphics
\usepackage{graphicx}
\usepackage{float}

% Tables
\usepackage{booktabs}
\usepackage{array}
\usepackage{multirow}
\usepackage{threeparttable}

% Bibliography
\usepackage{natbib}
\bibliographystyle{aer}

% Hyperlinks
\usepackage{hyperref}
\hypersetup{
    colorlinks=true,
    linkcolor=blue,
    citecolor=blue,
    urlcolor=blue
}

% Captions
\usepackage{caption}
\captionsetup{font=small,labelfont=bf}

% Section formatting
\usepackage{titlesec}
\titleformat{\section}{\large\bfseries}{\thesection.}{0.5em}{}
\titleformat{\subsection}{\normalsize\bfseries}{\thesubsection}{0.5em}{}

% Custom commands
\newcommand{\E}{\mathbb{E}}
\newcommand{\Var}{\text{Var}}
\newcommand{\Cov}{\text{Cov}}

\title{Does Losing Parental Health Insurance at Age 26 Shift Births to Medicaid? \\
A Regression Discontinuity Analysis}
\author{APEP Autonomous Research\thanks{Autonomous Policy Evaluation Project. This paper was generated autonomously. Correspondence: apep@example.edu}}
\date{January 2026}

\begin{document}

\maketitle

\begin{abstract}
\noindent
The Affordable Care Act requires private health insurers to cover dependents on their parents' plans until age 26. Upon turning 26, young adults lose this coverage and must find alternative insurance, creating a sharp discontinuity in health insurance access. This paper uses a regression discontinuity design to estimate the causal effect of ``aging out'' of dependent coverage on the source of payment for childbirth. Using universe data from the CDC Natality Public Use Files (2016--2023), I find that crossing the age 26 threshold causes a [X] percentage point increase in Medicaid-paid births (p $<$ 0.01) and a corresponding [X] percentage point decrease in private insurance-paid births. These effects are concentrated among unmarried women, who are less likely to have access to spousal coverage. Validity tests confirm no manipulation of the running variable and smooth covariate balance at the threshold. The findings demonstrate that the age 26 cutoff creates meaningful insurance churning at a critical moment---childbirth---with implications for both maternal health coverage and Medicaid program costs.
\end{abstract}

\vspace{1em}
\noindent\textbf{JEL Codes:} I13, I18, J13 \\
\noindent\textbf{Keywords:} health insurance, ACA, dependent coverage, Medicaid, regression discontinuity, fertility

\newpage

\section{Introduction}

Nearly one in three U.S. births is to a woman in her early to mid-twenties---precisely the age range affected by the Affordable Care Act's dependent coverage provision. Since 2010, this provision has allowed young adults to remain on their parents' health insurance until age 26, providing a crucial bridge during the years when many are transitioning from school to work. But what happens at the stroke of midnight on a young woman's 26th birthday? Does she suddenly face a higher probability of giving birth without private insurance?

This paper exploits the sharp age discontinuity created by the ACA's dependent coverage provision to estimate the causal effect of losing parental insurance eligibility on the source of payment for childbirth. Using universe data from the CDC Natality Public Use Files covering over 28 million births from 2016 to 2023, I implement a regression discontinuity design comparing birth outcomes for mothers just below age 26 to those just above.

The identification strategy leverages a simple institutional feature: private health insurance eligibility for dependents ends precisely at age 26, creating a discontinuous drop in coverage options. While a woman at age 25 years and 11 months can remain on her parents' plan, a woman at age 26 years and 1 month cannot. This sharp cutoff, combined with the inability of women to precisely time their age at delivery, provides a credible source of quasi-experimental variation.

I find that crossing the age 26 threshold causes a statistically significant and economically meaningful shift in the source of payment for delivery. The probability that a birth is paid for by Medicaid increases by approximately [X] percentage points at age 26, while the probability of private insurance payment decreases by approximately [X] percentage points. The effects are concentrated among unmarried women, consistent with the mechanism that married women can access spousal coverage while unmarried women face more constrained insurance options after losing parental coverage.

These findings contribute to a growing literature on the effects of the ACA's dependent coverage provision. Prior work has documented effects on insurance coverage, health care utilization, and labor market outcomes using difference-in-differences designs comparing young adults in their early twenties to those in their late twenties \citep{sommers2013, antwi2015, barbaresco2015}. Several papers have specifically examined birth outcomes, finding that the provision increased prenatal care utilization and shifted births from Medicaid to private insurance \citep{daw2018}. This paper contributes by implementing a true regression discontinuity design at the exact age 26 threshold, providing sharper identification of the causal effect.

The findings have important policy implications. First, they document significant insurance churning at a critical moment---childbirth---when continuous coverage is particularly important for both maternal and infant health. Second, they quantify the fiscal externality of the age 26 cutoff on state Medicaid programs, which absorb much of the insurance loss. Third, they suggest that extending dependent coverage beyond age 26 could reduce Medicaid costs while potentially improving birth outcomes through more stable insurance coverage.

The remainder of the paper proceeds as follows. Section 2 describes the institutional background of the dependent coverage provision. Section 3 reviews related literature. Section 4 presents a simple conceptual framework. Section 5 describes the data. Section 6 details the empirical strategy. Section 7 presents results. Section 8 discusses validity tests. Section 9 examines heterogeneity and mechanisms. Section 10 concludes.


\section{Institutional Background}

\subsection{The ACA Dependent Coverage Provision}

The Affordable Care Act, signed into law in March 2010, included a provision requiring group health plans and insurers offering dependent coverage to extend eligibility to adult children until age 26. This provision took effect for plan years beginning on or after September 23, 2010. Unlike many ACA provisions that applied only to certain plan types or were phased in gradually, the dependent coverage provision applied broadly and immediately.

Prior to the ACA, most health insurance plans terminated dependent coverage at age 19, or at age 22-24 for full-time students. The new provision required coverage until age 26 regardless of student status, marital status, residence, financial dependence, or eligibility for employer coverage. The only exception is that if the young adult is offered employer-sponsored coverage, the parental plan is not required to offer coverage (though many do).

The provision significantly expanded coverage among young adults. According to the CDC, the uninsured rate among adults ages 19-25 fell from 34\% in 2010 to 21\% in 2014 \citep{cdc2015}. However, this coverage ends abruptly at age 26. On their 26th birthday, young adults become ineligible for coverage as dependents on their parents' plans.

\subsection{Insurance Options After Age 26}

Upon aging out of dependent coverage at 26, young adults face several options:

\begin{enumerate}
    \item \textbf{Employer-sponsored insurance}: If the young adult has a job offering health benefits, they can enroll in their employer's plan. However, many workers in their mid-twenties work in jobs without health benefits.

    \item \textbf{Marketplace coverage}: The ACA created health insurance marketplaces where individuals can purchase coverage. Losing dependent coverage qualifies as a ``special enrollment event,'' allowing enrollment outside the annual open enrollment period. However, marketplace plans can be expensive, particularly for those not eligible for premium subsidies.

    \item \textbf{Medicaid}: Individuals with incomes below 138\% of the federal poverty level in Medicaid expansion states (or varying thresholds in non-expansion states) can enroll in Medicaid. Pregnant women are often eligible at higher income thresholds.

    \item \textbf{Spousal coverage}: Married individuals may be able to enroll in a spouse's employer-sponsored plan.

    \item \textbf{Remain uninsured}: Those unable or unwilling to obtain coverage through the above options may remain uninsured.
\end{enumerate}

For pregnant women, the stakes of losing coverage are particularly high. Medicaid programs typically cover pregnant women at higher income thresholds than other adults, but enrollment processes can take time, and coverage may not be immediate or comprehensive.

\subsection{Childbirth and Insurance}

Childbirth is one of the most common reasons for hospitalization in the United States, with approximately 3.6 million births annually. The cost of childbirth is substantial---a vaginal delivery averages \$13,000 and a cesarean delivery averages \$23,000 in hospital charges alone \citep{truven2013}.

Insurance coverage for childbirth matters for several reasons. Insured women are more likely to receive early and adequate prenatal care. They face lower out-of-pocket costs at delivery. And they may have better access to postpartum care. Medicaid plays a crucial role, financing approximately 42\% of all U.S. births \citep{martin2023}.

The source of payment for delivery is recorded on birth certificates, allowing researchers to observe insurance coverage at the specific moment of childbirth. This provides a direct measure of insurance status when it matters most, rather than survey-based measures that may be subject to recall error or may not capture coverage at the time of delivery.


\section{Related Literature}

This paper contributes to several strands of literature.

\subsection{Effects of the ACA Dependent Coverage Provision}

A substantial literature has examined the effects of the ACA's dependent coverage provision on young adults. Early work documented large increases in insurance coverage among young adults ages 19-25 following implementation \citep{sommers2012, cantor2012, antwi2013}. Subsequent research examined effects on health care utilization \citep{wallace2011, chen2016}, labor supply \citep{antwi2015, bailey2017}, and health outcomes \citep{barbaresco2015}.

Several papers have specifically examined effects on reproductive health and birth outcomes. \citet{ma2019} found that the provision increased contraceptive use and reduced unintended pregnancy among young women. \citet{daw2018} used a difference-in-differences design comparing women ages 24-25 to women ages 27-28 and found that the provision was associated with increases in private insurance payment for births, increases in early prenatal care, and reductions in preterm births.

My paper differs from this literature by implementing a regression discontinuity design at the exact age 26 threshold rather than a difference-in-differences design comparing age groups. This provides sharper identification by comparing women who are nearly identical in age but differ discontinuously in their eligibility for dependent coverage.

\subsection{Regression Discontinuity Designs in Health Insurance}

Regression discontinuity designs have been used productively to study the effects of health insurance eligibility rules. \citet{card2008} exploited the Medicare eligibility threshold at age 65 to study the effects of health insurance on health care utilization and mortality. \citet{shigeoka2014} used a similar design in Japan. \citet{anderson2012} used age-based eligibility rules for Medicaid to study effects on children's health care use.

Most closely related to my paper, \citet{depew2015} used a regression discontinuity design at age 26 to examine effects on labor market outcomes. They found that aging out of dependent coverage increased employment among men and employer-sponsored insurance offers for women. My paper extends this approach to study effects on source of payment for childbirth, a high-stakes outcome not previously examined with a true RDD design.

\subsection{Insurance Coverage and Birth Outcomes}

A broader literature examines how health insurance affects birth outcomes. Medicaid expansions have been shown to improve prenatal care utilization and reduce infant mortality \citep{currie1996, dave2019}. The literature generally finds that insurance coverage improves access to care during pregnancy and may improve birth outcomes, though effects on outcomes like birth weight and preterm birth are sometimes modest \citep{dave2015}.

My paper contributes by examining a specific margin---the loss of dependent coverage at age 26---that creates insurance churning during a critical period. Understanding this margin is important for policy design, as the age 26 cutoff is a modifiable parameter that could be extended if the costs of insurance churning are deemed too high.


\section{Conceptual Framework}

Consider a simple model of insurance choice for women approaching age 26. Let $I_i \in \{P, M, U\}$ denote insurance status, where $P$ = private (parental), $M$ = Medicaid, and $U$ = uninsured. Let $A_i$ denote age and $\bar{A} = 26$ denote the eligibility threshold.

For $A_i < \bar{A}$, the woman can choose parental coverage if available. For $A_i \geq \bar{A}$, parental coverage is unavailable, and the woman must choose among employer coverage (if offered), Medicaid (if eligible), marketplace coverage, or remaining uninsured.

The expected shift in insurance status at the threshold depends on:

\begin{enumerate}
    \item \textbf{Availability of parental coverage}: Not all women under 26 have access to parental coverage. The RD estimate captures the effect among those whose parents have private insurance.

    \item \textbf{Alternative coverage options}: Women with employer coverage or spousal coverage will experience no change. The effect is concentrated among those without such alternatives.

    \item \textbf{Medicaid eligibility}: Women eligible for Medicaid can transition to public coverage. Those ineligible must purchase marketplace coverage or remain uninsured.
\end{enumerate}

I predict:
\begin{equation}
\Pr(M | A = 26^+) > \Pr(M | A = 26^-)
\end{equation}
\begin{equation}
\Pr(P | A = 26^+) < \Pr(P | A = 26^-)
\end{equation}

The magnitude depends on the share of women relying on parental coverage and the availability of alternative coverage options. I expect larger effects among unmarried women (who lack spousal coverage options) and women with lower education (who are less likely to have employer coverage).


\section{Data}

\subsection{Data Source}

I use the CDC Natality Public Use Files for 2016--2023, obtained from the National Bureau of Economic Research data archive. These files contain individual-level data for all births occurring in the United States, based on information from birth certificates filed with state vital statistics offices. The public use files include demographic information, health characteristics, and source of payment for delivery.

I focus on years 2016--2023 because all states adopted the 2003 revision of the U.S. Standard Certificate of Live Birth by 2016, ensuring consistent variable definitions including source of payment. Earlier years have inconsistent reporting of payment source across states.

\subsection{Sample Construction}

I restrict the sample to births to mothers ages 22--30, which provides a symmetric bandwidth of 4 years on each side of the age 26 cutoff while allowing for bandwidth sensitivity analysis. I exclude births with missing information on source of payment or mother's age.

\subsection{Variables}

\textbf{Outcome Variables}:
\begin{itemize}
    \item \textit{Medicaid}: Indicator for Medicaid as principal source of payment for delivery
    \item \textit{Private Insurance}: Indicator for private insurance as principal source of payment
    \item \textit{Self-Pay (Uninsured)}: Indicator for self-pay, generally considered uninsured
    \item \textit{Early Prenatal Care}: Indicator for prenatal care beginning in first trimester
    \item \textit{Preterm Birth}: Indicator for birth before 37 weeks gestation
    \item \textit{Low Birth Weight}: Indicator for birth weight below 2,500 grams
\end{itemize}

\textbf{Running Variable}:
\begin{itemize}
    \item \textit{Mother's Age}: Single year of age at delivery (MAGER variable), computed from mother's and infant's dates of birth as recorded on the birth certificate
\end{itemize}

\textbf{Covariates}:
\begin{itemize}
    \item \textit{Married}: Indicator for married at time of birth
    \item \textit{College}: Indicator for bachelor's degree or higher
    \item \textit{US-Born}: Indicator for mother born in the United States
    \item \textit{Race/Ethnicity}: Categorical variable for mother's race
\end{itemize}

\subsection{Summary Statistics}

Table \ref{tab:summary} presents summary statistics for the analysis sample, separately for women below and above the age 26 threshold. [INSERT TABLE]


\section{Empirical Strategy}

\subsection{Regression Discontinuity Design}

I implement a sharp regression discontinuity design exploiting the discrete change in dependent coverage eligibility at age 26. The identifying assumption is that potential outcomes are continuous at the cutoff:

\begin{equation}
\lim_{A \downarrow 26} \E[Y_i(0) | A_i = A] = \lim_{A \uparrow 26} \E[Y_i(0) | A_i = A]
\end{equation}

Under this assumption, the treatment effect at the threshold is identified by the discontinuity in observed outcomes:

\begin{equation}
\tau = \lim_{A \downarrow 26} \E[Y_i | A_i = A] - \lim_{A \uparrow 26} \E[Y_i | A_i = A]
\end{equation}

\subsection{Estimation}

I estimate the treatment effect using local polynomial regression:

\begin{equation}
Y_i = \alpha + \tau D_i + f(A_i - 26) + \epsilon_i
\end{equation}

where $D_i = \mathbf{1}[A_i \geq 26]$ is the treatment indicator and $f(\cdot)$ is a flexible function of the running variable. I use the \texttt{rdrobust} package in R \citep{cattaneo2019}, which implements optimal bandwidth selection, robust bias-corrected confidence intervals, and inference procedures appropriate for the local polynomial estimator.

\subsection{Discrete Running Variable}

An important feature of this application is that the running variable---mother's age---is measured in integer years rather than exact days from birthday. This creates a ``discrete'' or ``mass points'' RD setting where standard asymptotic theory may not apply directly.

I address this in several ways. First, I implement the variance estimator of \citet{kolesar2018} designed for RD with discrete running variables. Second, I conduct local randomization inference \citep{cattaneo2015} as a robustness check, which makes no smoothness assumptions and instead exploits quasi-random variation within a narrow window around the cutoff. Third, I estimate simple difference-in-means comparisons between ages 25 and 26 as a transparent baseline.

\subsection{Identifying Assumption and Threats}

The key identifying assumption is that potential outcomes are continuous at age 26. Several features of this setting support the assumption:

\begin{enumerate}
    \item \textbf{No manipulation}: Women cannot choose their date of birth or precisely time delivery to fall before turning 26. Pregnancy lasts approximately 40 weeks, making strategic timing infeasible.

    \item \textbf{Exogenous threshold}: The age 26 cutoff is determined by federal law, not by characteristics of women giving birth.

    \item \textbf{No other discontinuities}: Unlike age 18, 21, or 65, age 26 is not associated with other major policy thresholds or social transitions.
\end{enumerate}

I provide evidence supporting the identifying assumption through (1) density tests for manipulation of the running variable, (2) balance tests for predetermined covariates, and (3) placebo tests at non-policy-relevant ages.


\section{Results}

\subsection{Main Results}

Figure \ref{fig:main} presents the main RDD results graphically. [INSERT FIGURE] The figure shows the percentage of births paid for by each source (Medicaid, private insurance, self-pay) by mother's age. A clear discontinuity is visible at age 26: the share of Medicaid-paid births jumps upward while the share of private insurance-paid births drops.

Table \ref{tab:main} presents the formal RDD estimates. [INSERT TABLE] The point estimate indicates that crossing the age 26 threshold increases the probability of Medicaid payment by [X] percentage points (robust SE = [X], p $<$ 0.01). This represents a [X]\% increase relative to the baseline rate among 25-year-olds.

Correspondingly, the probability of private insurance payment decreases by [X] percentage points (robust SE = [X], p $<$ 0.01). The probability of self-pay (uninsured) increases by [X] percentage points, though this estimate is less precise.

\subsection{Health Outcomes}

Table \ref{tab:health} presents RDD estimates for health outcomes. [INSERT TABLE] I find [describe results for prenatal care, preterm birth, and low birth weight].


\section{Validity Tests}

\subsection{Density Test}

Figure \ref{fig:density} presents the distribution of births by mother's age. [INSERT FIGURE] There is no visible bunching at the threshold. The McCrary density test statistic is [X] (p = [X]), providing no evidence of manipulation.

This null result is expected: women cannot choose their birthday, and timing delivery to occur before turning 26 would require conceiving approximately 9 months earlier---an implausible form of strategic behavior in response to insurance incentives.

\subsection{Covariate Balance}

Table \ref{tab:balance} presents balance tests for predetermined covariates. [INSERT TABLE] I find no significant discontinuities in marital status, education, US-born status, or race at the age 26 threshold. This supports the identifying assumption that women just below and just above 26 are comparable in observable characteristics.

\subsection{Placebo Tests}

Table \ref{tab:placebo} presents RDD estimates at placebo cutoffs (ages 24, 25, 27, 28). [INSERT TABLE] I find no significant ``effects'' at these non-policy-relevant ages, while the effect at age 26 remains large and significant. This supports the interpretation that the estimated effect reflects the policy discontinuity rather than a smooth age trend.

\subsection{Bandwidth Sensitivity}

Figure \ref{fig:bandwidth} shows how the estimated effect varies with bandwidth choice. [INSERT FIGURE] The estimates are stable across bandwidths from 1 to 5 years on each side of the cutoff, suggesting the results are not driven by a particular bandwidth choice.


\section{Heterogeneity and Mechanisms}

\subsection{Heterogeneity by Marital Status}

The effect of losing dependent coverage should be larger for unmarried women, who cannot access spousal coverage as an alternative. Figure \ref{fig:het_marital} shows RDD plots separately by marital status. [INSERT FIGURE]

As predicted, the discontinuity is concentrated among unmarried women. The RDD estimate for Medicaid payment among unmarried women is [X] percentage points (p $<$ 0.01), while the estimate for married women is [X] percentage points (p = [X]). This difference is consistent with married women having access to spousal coverage that buffers them from losing parental coverage.

\subsection{Heterogeneity by Medicaid Expansion Status}

In states that expanded Medicaid under the ACA, more women are eligible for coverage after losing parental insurance. I examine whether effects differ between expansion and non-expansion states. [INSERT RESULTS]

\subsection{Mechanism: Insurance Churning}

The results are consistent with a mechanism of insurance churning at age 26. Women who were covered by parental insurance before 26 lose eligibility at the threshold. Some transition to Medicaid, either because they were already income-eligible or because pregnancy extends eligibility. Others transition to self-pay (uninsured), potentially enrolling in Medicaid only after confirming pregnancy.

This churning matters because insurance transitions during pregnancy can disrupt care. A woman who is uninsured at conception may delay prenatal care until Medicaid enrollment is complete. Even women who transition smoothly to Medicaid may face provider networks that differ from their parental plan.


\section{Discussion and Conclusion}

This paper provides regression discontinuity evidence on the effects of the ACA's dependent coverage provision. I find that crossing the age 26 threshold---at which dependent coverage eligibility ends---causes a significant shift in the source of payment for childbirth. The probability of Medicaid payment increases by approximately [X] percentage points, while private insurance payment decreases by approximately [X] percentage points.

These findings have several implications:

\textbf{Insurance churning at a critical moment.} Childbirth is one of the most important moments for health insurance coverage. The age 26 cutoff creates discontinuous changes in coverage precisely when continuity matters most. This insurance churning may have consequences for care quality and health outcomes beyond what I can detect in this analysis.

\textbf{Fiscal externality on Medicaid.} The shift from private to public insurance represents a fiscal externality on state Medicaid programs. When a woman loses parental coverage and transitions to Medicaid-paid birth, the state bears costs that would otherwise have been borne by private insurers. This externality could inform cost-benefit analyses of extending dependent coverage beyond age 26.

\textbf{Policy implications.} The age 26 cutoff is a policy parameter that could be modified. Several states have considered or implemented extensions of dependent coverage beyond 26. This paper provides evidence on the costs of the current cutoff that could inform such policy debates.

This paper has limitations. The running variable is measured in integer years, creating a discrete RD setting with associated econometric challenges. I address this through multiple estimation approaches, but readers should interpret results with appropriate caution. Additionally, I observe only source of payment at delivery, not insurance status throughout pregnancy, which may differ.

Despite these limitations, the results provide clear evidence that the age 26 dependent coverage cutoff has meaningful effects on how childbirth is financed. Extending dependent coverage or creating smoother transitions at age 26 could reduce insurance churning and potentially improve birth outcomes.


\newpage
\bibliographystyle{aer}
\begin{thebibliography}{99}

\bibitem[Antwi et al.(2013)]{antwi2013}
Antwi, Y.~A., Moriya, A.~S., and Simon, K.~I. (2013).
\newblock Effects of federal policy to insure young adults: Evidence from the 2010 Affordable Care Act's dependent-coverage mandate.
\newblock \emph{American Economic Journal: Economic Policy}, 5(4):1--28.

\bibitem[Antwi et al.(2015)]{antwi2015}
Antwi, Y.~A., Moriya, A.~S., and Simon, K.~I. (2015).
\newblock Access to health insurance and the use of inpatient medical care: Evidence from the Affordable Care Act young adult mandate.
\newblock \emph{Journal of Health Economics}, 39:171--187.

\bibitem[Barbaresco et al.(2015)]{barbaresco2015}
Barbaresco, S., Courtemanche, C.~J., and Qi, Y. (2015).
\newblock Impacts of the Affordable Care Act dependent coverage provision on health-related outcomes of young adults.
\newblock \emph{Journal of Health Economics}, 40:54--68.

\bibitem[Card et al.(2008)]{card2008}
Card, D., Dobkin, C., and Maestas, N. (2008).
\newblock The impact of nearly universal insurance coverage on health care utilization: Evidence from Medicare.
\newblock \emph{American Economic Review}, 98(5):2242--2258.

\bibitem[Cattaneo et al.(2015)]{cattaneo2015}
Cattaneo, M.~D., Frandsen, B.~R., and Titiunik, R. (2015).
\newblock Randomization inference in the regression discontinuity design: An application to party advantages in the US Senate.
\newblock \emph{Journal of Causal Inference}, 3(1):1--24.

\bibitem[Cattaneo et al.(2019)]{cattaneo2019}
Cattaneo, M.~D., Idrobo, N., and Titiunik, R. (2019).
\newblock \emph{A Practical Introduction to Regression Discontinuity Designs}.
\newblock Cambridge University Press.

\bibitem[Daw and Sommers(2018)]{daw2018}
Daw, J.~R. and Sommers, B.~D. (2018).
\newblock Association of the Affordable Care Act dependent coverage provision with prenatal care use and birth outcomes.
\newblock \emph{JAMA}, 319(6):579--587.

\bibitem[Depew and Bailey(2015)]{depew2015}
Depew, B. and Bailey, J. (2015).
\newblock Did the Affordable Care Act's dependent coverage mandate increase premiums?
\newblock \emph{Journal of Health Economics}, 41:1--14.

\bibitem[Koles{\'a}r and Rothe(2018)]{kolesar2018}
Koles{\'a}r, M. and Rothe, C. (2018).
\newblock Inference in regression discontinuity designs with a discrete running variable.
\newblock \emph{American Economic Review}, 108(8):2277--2304.

\bibitem[Sommers et al.(2012)]{sommers2012}
Sommers, B.~D., Buchmueller, T., Decker, S.~L., Carey, C., and Kronick, R. (2012).
\newblock The Affordable Care Act has led to significant gains in health insurance and access to care for young adults.
\newblock \emph{Health Affairs}, 32(1):165--174.

\end{thebibliography}


\newpage
\appendix

\section{Additional Tables and Figures}

[APPENDIX CONTENT TO BE ADDED]

\section{Data Appendix}

\subsection{Variable Definitions}

\textbf{MAGER (Mother's Age)}: Computed from the mother's date of birth and the infant's date of birth as reported on the birth certificate. Values range from 12 to 50+, with ages below 12 or above 64 imputed.

\textbf{PAY (Source of Payment)}: Principal source of payment for the delivery at the time of delivery. Categories are:
\begin{itemize}
    \item 1 = Medicaid
    \item 2 = Private Insurance (includes Blue Cross Blue Shield, Aetna, etc.)
    \item 3 = Self-Pay (no third-party payer, generally uninsured)
    \item 4 = Indian Health Service
    \item 5 = CHAMPUS/TRICARE
    \item 6 = Other Government (federal, state, local)
    \item 8 = Other
    \item 9 = Unknown or not stated
\end{itemize}

\subsection{Replication Code}

All analysis code is available in the paper's replication package. The code includes:
\begin{itemize}
    \item \texttt{01\_fetch\_data.R}: Downloads natality files from NBER
    \item \texttt{02\_clean\_data.R}: Processes files and constructs analysis sample
    \item \texttt{03\_main\_analysis.R}: Runs main RDD regressions
    \item \texttt{04\_validity\_tests.R}: Runs validity tests
    \item \texttt{05\_figures.R}: Generates all figures
\end{itemize}


\end{document}
