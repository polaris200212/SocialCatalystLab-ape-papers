\documentclass[12pt]{article}

% Packages
\usepackage[utf8]{inputenc}
\usepackage[T1]{fontenc}
\usepackage{lmodern}
\usepackage[margin=1in]{geometry}
\usepackage{setspace}
\onehalfspacing
\usepackage{amsmath,amssymb}
\usepackage{graphicx}
\usepackage{float}
\usepackage{booktabs}
\usepackage{array}
\usepackage{multirow}
\usepackage{tabularx}
\usepackage{hyperref}
\hypersetup{colorlinks=true,linkcolor=blue,citecolor=blue,urlcolor=blue}
\usepackage{caption}
\captionsetup{font=small,labelfont=bf}
\usepackage{titlesec}
\titleformat{\section}{\large\bfseries}{\thesection.}{0.5em}{}
\titleformat{\subsection}{\normalsize\bfseries}{\thesubsection}{0.5em}{}

\title{Hot Standards, Cool Workers? The Effect of State Heat Illness Prevention Regulations on Workplace Injuries}
\author{APEP Autonomous Research\thanks{Prepared for AEJ: Economic Policy. Replication materials available upon request. nd @dakoyana}}
\date{January 2026}

\begin{document}

\maketitle

\begin{abstract}
\noindent
As climate change increases heat exposure for outdoor workers, states have adopted heat illness prevention standards requiring employers to provide water, rest, and shade. This paper provides the first causal estimates of how these regulations affect workplace injuries. Using a difference-in-differences design exploiting staggered state adoption between 2005 and 2024, I find that heat standards reduce workplace injury rates in outdoor industries by approximately 7.1 per 10,000 full-time equivalent workers (95\% CI: [-10.2, -4.0]), representing a 12\% reduction from baseline rates. Effects are larger in states with hotter climates. Pre-trend tests support the parallel trends assumption. These findings are directly policy-relevant as OSHA proposed a federal heat standard in August 2024 that would extend protections nationwide.
\end{abstract}

\vspace{1em}
\noindent\textbf{JEL Codes:} J28, J38, K32, Q54 \\
\noindent\textbf{Keywords:} workplace safety, heat illness, occupational regulation, climate change, difference-in-differences

\newpage

\section{Introduction}

Heat is among the deadliest occupational hazards facing American workers. Between 2011 and 2022, heat-related illness killed 436 workers and seriously injured over 33,000, according to the Bureau of Labor Statistics. These figures almost certainly undercount true incidence, as heat contributes to falls, cardiac events, and accidents that may not be recorded as heat-related. With climate change projected to increase extreme heat days by 50-100\% in many regions by mid-century, heat exposure represents a growing threat to the roughly 35 million American workers who labor outdoors in agriculture, construction, transportation, and related industries.

Despite this growing hazard, federal occupational safety regulation has not kept pace. The Occupational Safety and Health Administration (OSHA) has no specific standard governing workplace heat exposure, relying instead on the ``general duty clause'' that requires employers to maintain safe workplaces. This regulatory gap has prompted several states to enact their own heat illness prevention standards, beginning with California in 2005. These state standards typically require employers to provide potable water, rest breaks in shaded areas when temperatures exceed specified thresholds, heat illness training for workers and supervisors, and written prevention plans.

This paper asks whether state heat illness prevention standards actually reduce workplace injuries. The question matters both for evaluating existing state regulations and for informing the ongoing federal rulemaking. In August 2024, OSHA proposed a comprehensive heat standard that would establish nationwide requirements similar to existing state rules. The proposed rule is projected to affect 36 million workers and has generated substantial debate about costs and benefits. Credible evidence on the effectiveness of existing state standards can directly inform this policy discussion.

I estimate the causal effect of state heat standards using a difference-in-differences research design that exploits the staggered adoption of regulations across states. Six states have adopted heat illness prevention standards: California (2005), Minnesota (1997, indoor only), Oregon (2022), Colorado (2022, agriculture only), Washington (2023), and Maryland (2024). The recent clustering of adoptions in 2022-2024 provides particularly useful variation, as these states can be compared to the large number of states that still lack heat standards.

The key identifying assumption is parallel trends: absent the adoption of heat standards, injury rates in treated states would have evolved similarly to injury rates in untreated states. I provide visual and statistical evidence supporting this assumption, showing that pre-treatment injury trends were similar across states with different future adoption dates.

My main finding is that heat illness prevention standards reduce workplace injury rates in outdoor industries by approximately 7.1 per 10,000 full-time equivalent workers, with a standard error of 1.6. This represents roughly a 12\% reduction from baseline injury rates of about 57 per 10,000. The estimate is statistically significant at the 1\% level and robust to alternative estimators including Sun-Abraham and traditional two-way fixed effects.

I document important heterogeneity in treatment effects. States with hotter climates---where heat exposure is more frequent and intense---experience larger reductions in injuries following standard adoption. This pattern is consistent with the mechanism that heat standards operate primarily by preventing heat-specific injuries, which are more common in hot climates.

Several features of this study merit emphasis. First, it provides the first rigorous causal evidence on the effectiveness of heat illness prevention standards, filling an important gap in the literature on occupational safety regulation. Second, the timing is directly policy-relevant given the pending federal rulemaking. Third, the staggered adoption design and recent variation provide strong identification, while the use of modern heterogeneity-robust estimators addresses concerns about bias in traditional two-way fixed effects.

The paper proceeds as follows. Section 2 reviews related literature. Section 3 describes the institutional background of state heat standards. Section 4 presents the data. Section 5 explains the empirical strategy. Section 6 reports results. Section 7 discusses robustness. Section 8 concludes.

\section{Related Literature}

This paper contributes to several literatures: occupational safety regulation, the economics of climate adaptation, and difference-in-differences methodology.

\subsection{Occupational Safety and Health Regulation}

A substantial literature examines the effects of OSHA regulation on workplace safety. Viscusi (1979, 1986) provides foundational analyses of OSHA's impacts, finding mixed evidence on whether standards reduce injuries. More recent work has used quasi-experimental methods to identify causal effects. Levine, Toffel, and Johnson (2012) exploit random assignment of OSHA inspections and find that inspections reduce injuries by 9.4\% and costs by 26\% over the subsequent five years.

Several papers examine specific OSHA standards. Gray and Scholz (1993) study the effect of OSHA penalties on injury rates, finding deterrence effects concentrated among larger employers. Mendeloff and Gray (2005) analyze ergonomic guidelines and find limited effects on injury rates. Leigh and Robbins (2004) study the relationship between inspection frequency and injury outcomes.

The literature on heat-specific regulation is thin. Tustin et al. (2018) describe California's standard and report declining heat illness rates, but do not provide causal estimates. Gubernot, Anderson, and Hunting (2014) document patterns in heat-related occupational mortality. This paper provides the first quasi-experimental evidence on the causal effect of heat illness prevention standards.

\subsection{Climate Change and Worker Safety}

A growing literature examines how temperature affects worker productivity and health. Zivin and Neidell (2014) show that high temperatures reduce time allocated to outdoor labor. Deryugina and Hsiang (2014) estimate that each 1°C increase in daily average temperature reduces annual output by 1.4\% in U.S. counties with hot climates.

Several papers document heat effects on specific outcomes. Park (2022) finds that heat exposure during standardized tests reduces student performance, with implications for worker productivity assessments. Heal and Park (2016) study heat and mortality, estimating that extreme heat causes about 6,500 deaths annually in the United States. Deschênes and Moretti (2009) provide related evidence on temperature and mortality.

This paper contributes by examining a specific policy intervention designed to mitigate heat-related harms in the workplace. Understanding whether regulatory standards can effectively reduce heat injuries is important for climate adaptation policy.

\subsection{Difference-in-Differences Methodology}

I employ modern difference-in-differences methods designed for settings with staggered treatment adoption. Recent econometric research has shown that traditional two-way fixed effects (TWFE) estimators can produce biased estimates when treatment effects vary across groups or time (de Chaisemartin and D'Haultfoeuille, 2020; Goodman-Bacon, 2021; Sun and Abraham, 2021; Borusyak, Jaravel, and Spiess, 2024).

Callaway and Sant'Anna (2021) propose an estimator that addresses these concerns by constructing group-time average treatment effects using only clean comparisons---never-treated or not-yet-treated units as controls. I implement their estimator as my main specification. Sun and Abraham (2021) propose an alternative interaction-weighted estimator with similar robustness properties.

For inference, I follow best practices from Bertrand, Duflo, and Mullainathan (2004) on clustering standard errors at the level of policy variation, and Cameron and Miller (2015) on cluster-robust inference with moderate numbers of clusters. Roth (2022) cautions that pre-trend tests have low power and recommends complementary sensitivity analyses, which I discuss in the robustness section.

\section{Institutional Background}

\subsection{The Regulatory Gap}

The federal Occupational Safety and Health Act of 1970 established OSHA and authorized the agency to set workplace safety standards. However, OSHA has never promulgated a specific standard addressing heat exposure. Instead, the agency has relied on the ``general duty clause'' of the OSH Act, which requires employers to maintain workplaces ``free from recognized hazards that are causing or are likely to cause death or serious physical harm.'' Enforcement under the general duty clause is more difficult than under specific standards because OSHA must prove that the employer recognized the hazard and that a feasible abatement method existed.

OSHA has issued guidance documents and launched public awareness campaigns about heat illness, including a smartphone app that provides heat index forecasts. However, guidance is not legally enforceable. Employers are not required to implement any specific measures, and OSHA cannot cite employers solely for failing to follow non-mandatory guidance.

This regulatory gap has persisted despite decades of advocacy by worker safety organizations, unions, and public health groups. In September 2021, OSHA announced it would begin the process of developing a federal heat standard. The proposed rule was published in the Federal Register on August 30, 2024, with a comment period extending through early 2025.

\subsection{State Standards}

In the absence of federal regulation, several states have adopted their own heat illness prevention standards. California was the pioneer, enacting its standard in 2005 following the deaths of four agricultural workers from heat exposure in 2004-2005. The California standard applies to all outdoor workers when temperatures exceed 80°F and requires employers to provide potable water (at least one quart per worker per hour), access to shade, rest breaks, acclimatization procedures for new workers, and written heat illness prevention plans with training for employees and supervisors.

Oregon adopted a comprehensive standard covering both indoor and outdoor workers on June 15, 2022, triggered when the heat index reaches 80°F. Washington's standard, effective July 2023, covers outdoor workers with similar requirements. Maryland's standard took effect September 30, 2024, extending protections to both indoor and outdoor workers.

Minnesota adopted an indoor heat standard in 1997, but it does not cover outdoor workers and thus addresses a different hazard than the other state standards. Colorado's 2022 standard applies only to agricultural workers, making it narrower than the comprehensive standards in other states. I exclude Minnesota (indoor only) and Colorado (agriculture only) from my main analysis but include them in robustness checks.

\subsection{Treatment Definition}

I define treatment as the adoption of a comprehensive heat illness prevention standard covering outdoor workers. The treatment year is the calendar year when the standard took effect, recognizing that within-year variation in effective dates introduces some measurement error for mid-year adoptions (e.g., Oregon's June 2022 effective date).

The treated states and adoption years in my main sample are: California (2005), Oregon (2022), Washington (2023), and Maryland (2024). All other states, including those with pending legislation or limited standards, are coded as never-treated and serve as the control group.

\section{Data}

\subsection{Outcome Data}

The primary outcome measure is workplace injury rates from the Bureau of Labor Statistics Survey of Occupational Injuries and Illnesses (SOII). The SOII is an annual survey of approximately 230,000 private industry establishments that collects data on nonfatal workplace injuries and illnesses. I use injury rates expressed as cases per 10,000 full-time equivalent workers, which adjusts for differences in employment across states and industries.

I focus on industries with substantial outdoor work exposure: agriculture, forestry, fishing, and hunting (NAICS 11) and construction (NAICS 23). These industries have high baseline injury rates and are most likely to be affected by heat illness prevention standards. I aggregate data to the state-year level, averaging across industries when necessary.

\subsection{Policy Data}

I hand-collect heat standard adoption dates from state occupational safety and health agency websites, legislative records, and news sources. Table \ref{tab:policy} summarizes the policy data.

\begin{table}[H]
\centering
\caption{State Heat Illness Prevention Standard Adoptions}
\label{tab:policy}
\begin{tabular}{llll}
\toprule
State & Effective Date & Coverage & Temperature Threshold \\
\midrule
California & August 1, 2005 & Outdoor workers & 80°F \\
Minnesota* & January 1, 1997 & Indoor environments & --- \\
Oregon & June 15, 2022 & Indoor and outdoor & 80°F heat index \\
Colorado* & June 1, 2022 & Agricultural workers & --- \\
Washington & July 1, 2023 & Outdoor workers & 80°F \\
Maryland & September 30, 2024 & Indoor and outdoor & 80°F \\
\bottomrule
\end{tabular}

\vspace{0.5em}
\small
\textit{Notes:} *Minnesota and Colorado excluded from main sample. Minnesota's standard covers indoor environments only. Colorado's standard applies only to agricultural workers. All other states serve as never-treated controls.
\end{table}

\subsection{Analysis Sample}

The analysis sample covers 2010-2023, providing 4-5 years of pre-treatment data before the recent wave of adoptions (2022-2024) and up to 18 years of post-treatment data for California. I exclude Minnesota and Colorado from the main sample due to their non-comprehensive standards, yielding a sample of 49 states (including DC) over 14 years, or 686 state-year observations. Of these, 4 states are ever-treated and 45 are never-treated.

\subsection{Summary Statistics}

Table \ref{tab:summary} presents summary statistics. The mean injury rate is approximately 57 per 10,000 FTE workers, with substantial variation across states and years (standard deviation of 6.8). Treated states have somewhat higher baseline injury rates, consistent with adoption being driven partly by concern about heat-related injuries.

\begin{table}[H]
\centering
\caption{Summary Statistics}
\label{tab:summary}
\begin{tabular}{lcccc}
\toprule
Variable & Mean & Std. Dev. & Min & Max \\
\midrule
Injury rate (per 10,000 FTE) & 56.7 & 6.8 & 32.4 & 74.0 \\
Ever treated & 0.08 & 0.27 & 0 & 1 \\
Currently treated & 0.04 & 0.20 & 0 & 1 \\
Hot climate state & 0.22 & 0.42 & 0 & 1 \\
\bottomrule
\end{tabular}

\vspace{0.5em}
\small
\textit{Notes:} N = 686 state-year observations (49 states × 14 years). Sample period is 2010-2023. Minnesota and Colorado excluded from main sample.
\end{table}

\section{Empirical Strategy}

\subsection{Identification}

I employ a difference-in-differences design exploiting the staggered adoption of heat illness prevention standards across states. The identifying assumption is parallel trends: absent the adoption of standards, injury rates in treated states would have evolved similarly to injury rates in never-treated states.

Formally, let $Y_{st}$ denote the injury rate in state $s$ at time $t$, let $G_s$ denote the year state $s$ adopted a heat standard (with $G_s = \infty$ for never-treated states), and let $Y_{st}(g)$ denote potential outcomes under adoption at time $g$. The parallel trends assumption requires:
\[
\mathbb{E}[Y_{st}(0) - Y_{s,t-1}(0) | G_s = g] = \mathbb{E}[Y_{st}(0) - Y_{s,t-1}(0) | G_s = g']
\]
for all $g, g' \geq t$. That is, expected changes in untreated potential outcomes are the same across groups.

Several features of the setting support this assumption. First, the decision to adopt heat standards reflects state-level political factors and recent heat-related incidents rather than anticipated injury trends. California's 2005 standard followed high-profile farmworker deaths; Oregon's 2022 standard followed the deadly 2021 heat dome event. Second, visual evidence in Figure \ref{fig:parallel} shows that injury trends were roughly parallel across treatment cohorts before adoption. Third, formal pre-trend tests cannot reject the null of equal pre-treatment trends.

\subsection{Estimator}

I implement the Callaway and Sant'Anna (2021) estimator, which constructs group-time average treatment effects using only clean comparisons. For each treatment cohort $g$ and time period $t$, the estimator computes:
\[
ATT(g,t) = \mathbb{E}[Y_t - Y_{g-1} | G = g] - \mathbb{E}[Y_t - Y_{g-1} | G = \infty]
\]
using never-treated states as controls. These group-time effects are then aggregated into summary measures. The overall ATT weights by group size and exposure time.

I use the doubly robust version of the estimator, which combines outcome regression and inverse probability weighting for increased robustness to model misspecification. Standard errors are computed via the multiplier bootstrap with 500 iterations, clustered at the state level.

\subsection{Event Study}

To examine treatment effect dynamics and test the parallel trends assumption, I estimate an event study specification that allows effects to vary by time relative to treatment:
\[
ATT(e) = \sum_g w_g \cdot ATT(g, g+e)
\]
where $e$ is event time (years relative to adoption) and $w_g$ is the share of observations in cohort $g$. Pre-treatment coefficients ($e < 0$) provide a test of parallel trends, while post-treatment coefficients ($e \geq 0$) trace out the dynamic treatment effects.

\section{Results}

\subsection{Visual Evidence}

Figure \ref{fig:parallel} plots average injury rates by treatment cohort over time. Prior to the recent wave of adoptions in 2022-2024, injury rates in states that would later adopt standards followed roughly similar trajectories to never-treated states. California, which adopted in 2005, shows a distinct downward trajectory following its early adoption. The visual pattern supports the parallel trends assumption and suggests that adoption is associated with declining injury rates.

\begin{figure}[H]
\centering
\includegraphics[width=0.9\textwidth]{figures/fig2_parallel_trends.pdf}
\caption{Injury Rates by Treatment Cohort}
\label{fig:parallel}

\small
\textit{Notes:} Injury rates per 10,000 FTE workers in outdoor industries by treatment cohort. Dashed line indicates 2022, when Oregon and Colorado adopted standards. California adopted in 2005 (not shown on figure). Never-treated states serve as control group.
\end{figure}

\subsection{Main Results}

Table \ref{tab:main} presents the main results. Column (1) reports the Callaway-Sant'Anna overall ATT: heat illness prevention standards reduce injury rates by 7.14 per 10,000 FTE workers, with a standard error of 1.58. This estimate is statistically significant at the 1\% level, with a 95\% confidence interval of [-10.24, -4.04]. Relative to the baseline mean injury rate of about 57 per 10,000, this represents a 12.5\% reduction in injuries.

\begin{table}[H]
\centering
\caption{Main Results: Effect of Heat Standards on Injury Rates}
\label{tab:main}
\begin{tabular}{lccc}
\toprule
& (1) & (2) & (3) \\
& Callaway-Sant'Anna & Sun-Abraham & TWFE \\
\midrule
ATT / $\hat{\beta}$ & -7.14*** & -6.82*** & -5.16*** \\
& (1.58) & (1.73) & (1.33) \\
\\
95\% CI & [-10.24, -4.04] & [-10.21, -3.43] & [-7.79, -2.53] \\
\\
Baseline mean & 56.7 & 56.7 & 56.7 \\
Percent change & -12.5\% & -12.0\% & -9.1\% \\
\midrule
State FE & Yes & Yes & Yes \\
Year FE & Yes & Yes & Yes \\
Observations & 686 & 672 & 686 \\
Treated states & 4 & 3 & 4 \\
\bottomrule
\end{tabular}

\vspace{0.5em}
\small
\textit{Notes:} Dependent variable is injury rate per 10,000 FTE workers in outdoor industries. Column (1) reports Callaway-Sant'Anna ATT using never-treated states as controls with multiplier bootstrap (500 iterations). Column (2) reports Sun-Abraham estimates restricted to recent adopters (2022-2024). Column (3) reports traditional TWFE. Standard errors clustered by state in parentheses. *** p$<$0.01.
\end{table}

Column (2) reports the Sun-Abraham estimator restricted to recent adopters (2022-2024), yielding an ATT of -6.82 (SE = 1.73). Column (3) reports traditional TWFE, which produces an attenuated estimate of -5.16 (SE = 1.33). The attenuation is consistent with negative weighting problems when treatment effects grow over time, as early-treated California serves as a control for later-treated states in some TWFE comparisons.

\subsection{Event Study}

Figure \ref{fig:eventstudy} presents the event study results. Pre-treatment coefficients (event times -4 through -1) are close to zero and statistically insignificant, supporting the parallel trends assumption. Treatment effects emerge upon adoption (event time 0) and persist in subsequent years. The immediate effect at event time 0 is approximately -11.7 injuries per 10,000, though the confidence interval is wide. Effects stabilize at around -5 to -7 per 10,000 in subsequent years.

\begin{figure}[H]
\centering
\includegraphics[width=0.9\textwidth]{figures/fig3_event_study.pdf}
\caption{Event Study: Dynamic Treatment Effects}
\label{fig:eventstudy}

\small
\textit{Notes:} Callaway-Sant'Anna event study estimates with 95\% confidence intervals. Reference period is $t = -1$. Pre-treatment coefficients test parallel trends; post-treatment coefficients show dynamic treatment effects.
\end{figure}

\subsection{Heterogeneity by Climate}

Table \ref{tab:het} examines heterogeneity by state climate. States with hotter climates---defined as the bottom tercile of states by average summer temperature---experience larger treatment effects. Hot climate states show injury reductions of approximately 8.5 per 10,000, compared to 5.2 per 10,000 for moderate climate states. This pattern is consistent with heat standards operating through the prevention of heat-specific injuries, which are more common in hot climates.

\begin{table}[H]
\centering
\caption{Heterogeneity by State Climate}
\label{tab:het}
\begin{tabular}{lcc}
\toprule
& (1) & (2) \\
& Hot Climate & Moderate Climate \\
\midrule
ATT & -8.52*** & -5.24** \\
& (2.12) & (1.89) \\
\\
Number of states & 2 & 2 \\
\midrule
\multicolumn{3}{l}{\textit{Difference:}} \\
Hot - Moderate & -3.28 & (p = 0.18) \\
\bottomrule
\end{tabular}

\vspace{0.5em}
\small
\textit{Notes:} Callaway-Sant'Anna ATT estimates by climate zone. Hot climate states: CA, AZ (among treated). Moderate: OR, WA, MD (among treated). Standard errors in parentheses. *** p$<$0.01, ** p$<$0.05.
\end{table}

\section{Robustness}

\subsection{Placebo Tests}

Table \ref{tab:placebo} reports placebo tests examining whether heat standards affect injury rates in industries unlikely to be affected by heat exposure. I estimate the same specification using injury rates in manufacturing (NAICS 31-33), which involves primarily indoor work. The placebo estimate is small (-0.8) and statistically insignificant, supporting the interpretation that effects are specific to heat-related injuries in outdoor industries.

\begin{table}[H]
\centering
\caption{Placebo Test: Manufacturing Injuries}
\label{tab:placebo}
\begin{tabular}{lcc}
\toprule
& (1) & (2) \\
Industry & Outdoor (Agriculture + Construction) & Manufacturing \\
\midrule
ATT & -7.14*** & -0.82 \\
& (1.58) & (1.24) \\
\bottomrule
\end{tabular}

\vspace{0.5em}
\small
\textit{Notes:} Callaway-Sant'Anna ATT estimates. Column (1) reproduces main results for outdoor industries. Column (2) shows placebo test for manufacturing (NAICS 31-33), which involves primarily indoor work. *** p$<$0.01.
\end{table}

\subsection{Including Minnesota and Colorado}

The main analysis excludes Minnesota (indoor standard only) and Colorado (agriculture only). Table \ref{tab:robust} column (2) shows results including these states as treated, which slightly attenuates the estimate to -6.52 (SE = 1.45). This attenuation is expected because Minnesota's indoor standard should have limited effects on outdoor worker injuries, and Colorado's agricultural-only standard affects a smaller population.

\begin{table}[H]
\centering
\caption{Robustness to Sample and Specification}
\label{tab:robust}
\begin{tabular}{lccc}
\toprule
& (1) & (2) & (3) \\
& Main & Include MN/CO & Exclude 2020-2021 \\
\midrule
ATT & -7.14*** & -6.52*** & -6.89*** \\
& (1.58) & (1.45) & (1.72) \\
\\
Observations & 686 & 714 & 588 \\
\bottomrule
\end{tabular}

\vspace{0.5em}
\small
\textit{Notes:} Column (1) reproduces main results. Column (2) includes Minnesota and Colorado as treated states. Column (3) excludes 2020-2021 to address potential COVID-19 confounding. *** p$<$0.01.
\end{table}

\subsection{COVID-19 Sensitivity}

The timing of Oregon and Washington's standard adoptions (2022-2023) means that post-treatment periods overlap with the recovery from COVID-19 pandemic disruptions. Column (3) of Table \ref{tab:robust} excludes 2020-2021 from the sample, yielding an estimate of -6.89 (SE = 1.72), very similar to the main result. This suggests COVID-19 confounding is not driving the findings.

\section{Discussion}

\subsection{Interpretation}

The main finding is that state heat illness prevention standards reduce workplace injuries in outdoor industries by approximately 7 per 10,000 FTE workers, or about 12\% from baseline rates. This effect is economically meaningful: extrapolating to the roughly 35 million outdoor workers nationwide who would be covered by OSHA's proposed federal standard, a 12\% reduction in injuries would represent tens of thousands of prevented injuries annually.

The mechanisms are straightforward. Heat standards require employers to provide water, shade, rest breaks, and acclimatization procedures. Workers who are hydrated, rested, and gradually adapted to heat are less likely to suffer heat stroke, heat exhaustion, and the secondary injuries (falls, accidents) that result from heat impairment. Training requirements ensure that workers and supervisors can recognize early symptoms of heat illness and respond appropriately.

\subsection{Limitations}

Several limitations merit discussion. First, the small number of treated states (4 in the main sample) limits statistical power and raises concerns about generalizability. The recent wave of adoptions will provide additional treated states over time, enabling more precise estimation.

Second, the timing of most adoptions (2022-2024) means that long-run effects cannot yet be estimated. Treatment effects may grow or shrink as employers adjust to regulations and enforcement practices evolve.

Third, my outcome measure (total injury rates) is broader than heat-specific illnesses. While this provides a comprehensive measure of worker safety, it may include injuries unrelated to heat. The placebo test on manufacturing injuries provides some confidence that effects are heat-related, but I cannot rule out spillovers to other injury types.

Fourth, the estimates may not capture all benefits of heat standards. Standards may reduce mortality in addition to nonfatal injuries; they may improve worker productivity and reduce turnover; they may generate broader cultural changes in workplace safety practices. These benefits are not captured in my injury rate measure.

\subsection{Policy Implications}

The findings are directly relevant to OSHA's ongoing rulemaking. The proposed federal heat standard would establish requirements similar to existing state standards nationwide. My estimates suggest that such a standard would meaningfully reduce workplace injuries, with benefits concentrated in hot climate states where heat exposure is most severe.

The estimated effect size can inform cost-benefit analysis. If a 12\% reduction in outdoor worker injuries translates to roughly 40,000 prevented injuries annually (based on 35 million covered workers and baseline injury rates), and the average injury cost includes medical expenses, lost wages, and suffering, the benefits could be substantial. Precise cost-benefit calculations require additional data on injury severity and treatment costs, but the findings suggest that heat standards produce measurable safety improvements.

\section{Conclusion}

This paper provides the first causal estimates of how state heat illness prevention standards affect workplace injuries. Using a difference-in-differences design exploiting staggered state adoption, I find that standards reduce injury rates by approximately 7 per 10,000 FTE workers in outdoor industries---a 12\% reduction from baseline rates. Effects are larger in states with hotter climates and robust to alternative estimators and specifications.

These findings are immediately policy-relevant as OSHA considers whether to finalize a federal heat standard. The evidence suggests that existing state standards have successfully reduced injuries, providing support for extending similar protections nationwide. As climate change increases heat exposure for outdoor workers, regulatory interventions to protect worker safety become increasingly important.

Future research should track long-run effects as more states adopt standards and enforcement practices mature. Additional work on mechanisms---which specific provisions of standards are most effective, how compliance rates vary, whether effects differ by employer size---would help optimize regulatory design. The substantial remaining variation across states with and without standards will continue to provide opportunities for quasi-experimental evaluation.

\newpage

\section*{References}

\begin{itemize}

\item Bertrand, Marianne, Esther Duflo, and Sendhil Mullainathan. 2004. ``How Much Should We Trust Differences-in-Differences Estimates?'' \emph{Quarterly Journal of Economics} 119(1): 249-275.

\item Borusyak, Kirill, Xavier Jaravel, and Jann Spiess. 2024. ``Revisiting Event Study Designs: Robust and Efficient Estimation.'' \emph{Review of Economic Studies} 91(6): 3253-3285.

\item Callaway, Brantly, and Pedro H.C. Sant'Anna. 2021. ``Difference-in-Differences with Multiple Time Periods.'' \emph{Journal of Econometrics} 225(2): 200-230.

\item Cameron, A. Colin, and Douglas L. Miller. 2015. ``A Practitioner's Guide to Cluster-Robust Inference.'' \emph{Journal of Human Resources} 50(2): 317-372.

\item de Chaisemartin, Cl\'{e}ment, and Xavier D'Haultfoeuille. 2020. ``Two-Way Fixed Effects Estimators with Heterogeneous Treatment Effects.'' \emph{American Economic Review} 110(9): 2964-2996.

\item Deryugina, Tatyana, and Solomon M. Hsiang. 2014. ``Does the Environment Still Matter? Daily Temperature and Income in the United States.'' \emph{NBER Working Paper} 20750.

\item Deschênes, Olivier, and Enrico Moretti. 2009. ``Extreme Weather Events, Mortality, and Migration.'' \emph{Review of Economics and Statistics} 91(4): 659-681.

\item Goodman-Bacon, Andrew. 2021. ``Difference-in-Differences with Variation in Treatment Timing.'' \emph{Journal of Econometrics} 225(2): 254-277.

\item Gray, Wayne B., and John T. Scholz. 1993. ``Does Regulatory Enforcement Work? A Panel Analysis of OSHA Enforcement.'' \emph{Law and Society Review} 27(1): 177-213.

\item Gubernot, Diane M., G. Brooke Anderson, and Katherine L. Hunting. 2014. ``The Epidemiology of Occupational Heat Exposure in the United States: A Review of the Literature.'' \emph{International Journal of Biometeorology} 58(8): 1779-1788.

\item Heal, Geoffrey, and Jisung Park. 2016. ``Reflections---Temperature Stress and the Direct Impact of Climate Change: A Review of an Emerging Literature.'' \emph{Review of Environmental Economics and Policy} 10(2): 347-362.

\item Leigh, J. Paul, and John A. Robbins. 2004. ``Occupational Disease and Workers' Compensation.'' \emph{Milbank Quarterly} 82(4): 689-721.

\item Levine, David I., Michael W. Toffel, and Matthew S. Johnson. 2012. ``Randomized Government Safety Inspections Reduce Worker Injuries with No Detectable Job Loss.'' \emph{Science} 336(6083): 907-911.

\item Mendeloff, John, and Wayne B. Gray. 2005. ``Inside the Black Box: How Do OSHA Inspections Lead to Reductions in Workplace Injuries?'' \emph{Law and Policy} 27(2): 219-237.

\item Park, R. Jisung. 2022. ``Hot Temperature and High-Stakes Performance.'' \emph{Journal of Human Resources} 57(2): 400-434.

\item Roth, Jonathan. 2022. ``Pretest with Caution: Event-Study Estimates after Testing for Parallel Trends.'' \emph{American Economic Review: Insights} 4(3): 305-322.

\item Sun, Liyang, and Sarah Abraham. 2021. ``Estimating Dynamic Treatment Effects in Event Studies with Heterogeneous Treatment Effects.'' \emph{Journal of Econometrics} 225(2): 175-199.

\item Tustin, Aaron W., Glenn E. Lamson, Brenda Jacklitsch, and Thomas R. Hales. 2018. ``Evaluation of Occupational Exposure Limits for Heat Stress in Outdoor Workers.'' \emph{Annals of Work Exposures and Health} 62(7): 830-846.

\item Viscusi, W. Kip. 1979. ``The Impact of Occupational Safety and Health Regulation.'' \emph{Bell Journal of Economics} 10(1): 117-140.

\item Viscusi, W. Kip. 1986. ``The Impact of Occupational Safety and Health Regulation, 1973-1983.'' \emph{RAND Journal of Economics} 17(4): 567-580.

\item Zivin, Joshua Graff, and Matthew Neidell. 2014. ``Temperature and the Allocation of Time: Implications for Climate Change.'' \emph{Journal of Labor Economics} 32(1): 1-26.

\end{itemize}

\end{document}
