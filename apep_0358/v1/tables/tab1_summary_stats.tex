\begin{table}[htbp]
\centering
\caption{Summary Statistics: Maternal Health Claims in Medicaid}
\label{tab:summary}
\small
\begin{tabular}{lccc}
\hline\hline
 & Pre-Treatment & Post-Treatment & Full Sample \\
 & (Jan 2018--Sept 2021) & (Treated states) & \\
\hline
\multicolumn{4}{l}{\textit{Panel A: Claims per state-month}} \\
Postpartum care (59430) & 182.8 (524.1) & 271.2 (700.4) & 226.7 (687.2) \\
Antepartum care (59425/26) & 1438.4 (6052.7) & 1244.8 (4876.8) & 1236.7 (5307.7) \\
Contraceptive services & 138.0 (600.0) & 135.6 (449.0) & 126.9 (516.3) \\
Delivery (594XX) & 222.2 (489.0) & 213.1 (475.3) & 221.5 (526.5) \\
[6pt]
\multicolumn{4}{l}{\textit{Panel B: Providers per state-month}} \\
Postpartum providers & 6.5 (17.6) & 9.3 (22.2) & 7.7 (21.7) \\
OB/GYN providers (any code) & 70.0 (101.8) & 73.4 (103.6) & 69.0 (102.3) \\
[6pt]
State-months & 2295 & 1268 & 4284 \\
States & 51 & 47 & 51 \\
\hline\hline
\end{tabular}
\begin{minipage}{0.95\linewidth}
\vspace{6pt}
\footnotesize
\textit{Notes:} Mean (standard deviation) reported. Pre-treatment period defined as January 2018 through September 2021 (before any state adopted the 12-month postpartum extension). Post-treatment includes only state-months where the extension is in effect. Claims and provider counts are at the state-month level. Data from T-MSIS Medicaid Provider Spending (2018--2024) linked to NPPES for provider identification.
\end{minipage}
\end{table}
