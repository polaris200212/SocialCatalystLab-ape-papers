\documentclass[12pt]{article}
\usepackage[margin=1in]{geometry}
\usepackage{setspace}
\usepackage{graphicx}
\usepackage{booktabs}
\usepackage{natbib}
\usepackage{amsmath}
\usepackage{amssymb}
\usepackage{hyperref}
\usepackage{float}
\usepackage{caption}
\usepackage{subcaption}

\title{The Health Insurance Cliff: \\ Evidence from Wisconsin's Unique Medicaid Design}
\author{Autonomous Policy Evaluation Project nd @dakoyana}
\date{January 2026}

\doublespacing

\begin{document}

\maketitle

\begin{abstract}
Wisconsin is the only U.S. state that covers adults through Medicaid up to exactly 100\% of the Federal Poverty Level (FPL) while declining the Affordable Care Act's Medicaid expansion. This creates a sharp eligibility cliff at 100\% FPL---below which adults qualify for BadgerCare (Medicaid), and above which they must purchase coverage on the health insurance exchange. Using regression discontinuity methods applied to American Community Survey data from 2014--2018, I estimate the effects of this eligibility threshold on health insurance coverage and labor supply. I find a statistically significant 7.7 percentage point discontinuity in Medicaid coverage at the 100\% FPL threshold: individuals just below the cutoff are substantially more likely to have Medicaid than those just above. However, I find no statistically significant discontinuity in employment, hours worked, or earnings, suggesting that the coverage cliff does not induce substantial labor supply distortions. These results indicate that while Wisconsin's unique policy design creates a sharp coverage discontinuity, the availability of subsidized exchange plans above the threshold appears to mitigate labor supply responses. The findings have implications for ongoing debates about Medicaid expansion, benefit cliff design, and the labor market effects of health insurance eligibility thresholds.

\vspace{0.5cm}
\noindent\textbf{Keywords:} Medicaid, health insurance, labor supply, regression discontinuity, benefit cliffs, Wisconsin

\vspace{0.5cm}
\noindent\textbf{JEL Codes:} I13, I18, J22, H51
\end{abstract}

\newpage
\section{Introduction}

The Affordable Care Act (ACA) of 2010 offered states the option to expand Medicaid eligibility to all adults with incomes up to 138\% of the Federal Poverty Level (FPL). As of 2024, 40 states and the District of Columbia have adopted this expansion, while 10 states have not. Among the non-expansion states, Wisconsin stands alone in its approach: it covers all adults up to exactly 100\% of the FPL through its BadgerCare program, but declined the full ACA expansion that would extend coverage to 138\% FPL.

This policy creates a unique natural experiment. At the 100\% FPL threshold, Wisconsin residents face a sharp discontinuity in health insurance eligibility. Adults with incomes below 100\% FPL qualify for BadgerCare (Wisconsin's Medicaid program), which provides comprehensive coverage with minimal or no premiums and very low cost-sharing. In contrast, adults with incomes above 100\% FPL are categorically ineligible for Medicaid and must instead purchase coverage on the ACA health insurance exchange. While premium tax credits substantially reduce the cost of exchange plans for low-income individuals, these plans typically require monthly premium payments and impose deductibles, copayments, and coinsurance that Medicaid does not.

This threshold matters for two reasons. First, it directly affects health insurance coverage---the primary mechanism through which public health insurance affects individuals' lives. Understanding whether the threshold creates a ``coverage cliff'' is important for evaluating Wisconsin's policy design. Second, the threshold may affect labor supply by creating incentives for workers to reduce earnings to maintain Medicaid eligibility, a phenomenon known as the ``benefit cliff'' or ``poverty trap.''

Prior research on benefit cliffs has focused on programs like SNAP, TANF, and housing assistance, where eligibility thresholds can create high implicit marginal tax rates \citep{romig2017tanf, bitler2017cliff}. The ACA's Medicaid expansion was explicitly designed to avoid such cliffs by extending eligibility to 138\% FPL and providing premium subsidies that phase out gradually above that threshold. Wisconsin's partial expansion---stopping at exactly 100\% FPL---reintroduces a potential cliff, making it an ideal setting to study how eligibility thresholds affect behavior.

This paper makes three contributions. First, I document a novel policy discontinuity that has not been previously studied. Wisconsin's 100\% FPL adult Medicaid threshold is unique nationally---no other state has this exact policy design. Second, I provide the first estimates of the coverage discontinuity at this threshold, finding an economically and statistically significant 7.7 percentage point drop in Medicaid coverage when moving from just below to just above 100\% FPL. Third, I test whether this coverage cliff induces labor supply distortions, finding no significant effects on employment, hours, or earnings.

The remainder of this paper proceeds as follows. Section 2 provides background on Wisconsin's Medicaid policy and the theoretical framework for analyzing benefit cliffs. Section 3 describes the data and sample construction. Section 4 presents the regression discontinuity design and identification assumptions. Section 5 reports the main results. Section 6 presents robustness checks and falsification tests. Section 7 discusses the implications and concludes.

\section{Background and Theoretical Framework}

\subsection{Wisconsin's Unique Medicaid Design}

Wisconsin's approach to Medicaid for adults is distinctive in the American healthcare landscape and provides a unique natural experiment for studying the effects of benefit cliffs. When the Affordable Care Act was implemented in 2014, most states faced a binary choice: accept the Medicaid expansion with enhanced federal funding, or maintain their existing categorical eligibility rules that typically covered only very low-income parents and excluded childless adults entirely. Wisconsin chose a third path that no other state has replicated.

Governor Scott Walker declined the full Medicaid expansion (to 138\% FPL) but simultaneously used state authority to eliminate previous categorical restrictions that had limited Medicaid to parents with incomes below approximately 44\% of FPL. The result was a dramatic expansion of eligibility---all adults, including childless adults, became eligible for BadgerCare (Wisconsin's Medicaid program) up to exactly 100\% of the Federal Poverty Level. This approach was dubbed the ``Wisconsin Way'' and framed as a fiscally responsible alternative to full expansion.

The policy had both political and fiscal motivations that shaped its design. By setting the threshold at exactly 100\% FPL, Wisconsin ensured that everyone above that level would qualify for premium tax credits on the federal health insurance exchange. This avoided the ``coverage gap'' that emerged in non-expansion states where adults between the old categorical limits (often 44\% FPL for parents, 0\% for childless adults) and 100\% FPL were ineligible for both Medicaid and exchange subsidies. Wisconsin's approach also avoided reliance on the enhanced federal matching rate for expansion populations (initially 100\%, declining to 90\% by 2020), which some state officials viewed as an unstable funding source that could be reduced by future Congresses.

As of 2024, approximately 200,000 childless adults are enrolled in BadgerCare under this policy, representing a substantial expansion of coverage compared to the pre-ACA period when this population had no pathway to Medicaid. The 100\% FPL threshold translates to annual incomes of approximately \$15,060 for a single individual and \$31,200 for a family of four at 2024 poverty guideline levels.

For adults with incomes just above 100\% FPL, the transition is not to uninsurance but to the ACA marketplace, which distinguishes Wisconsin's policy from traditional benefit cliffs. Premium tax credits are structured so that individuals at 100\% FPL pay no more than 2.07\% of income for a benchmark Silver plan, with subsidies phasing out gradually through 400\% FPL. Cost-sharing reductions are available for those below 250\% FPL, further reducing out-of-pocket costs. However, marketplace plans typically require monthly premium payments (even if small), impose deductibles that can reach several thousand dollars, and involve copayments and coinsurance that Medicaid does not. This creates a genuine, if potentially modest, discontinuity in the effective cost of health coverage at the 100\% FPL threshold.

\subsection{Theoretical Framework: The Benefit Cliff}

Economic theory predicts that discontinuities in benefit eligibility can distort behavior when the value of benefits is substantial and individuals can control their position relative to the threshold. Consider an individual with earnings potential $E$ and facing a Medicaid eligibility threshold at income $T$ (100\% FPL in Wisconsin).

Below the threshold, the individual receives Medicaid benefits worth $M$ (in utility terms). Above the threshold, they receive no Medicaid but may receive exchange subsidies worth $S(E)$, which phase out gradually with income. The individual's utility from earning $E$ is:

\begin{equation}
U(E) = \begin{cases}
E + M & \text{if } E \leq T \\
E + S(E) & \text{if } E > T
\end{cases}
\end{equation}

If $M > S(T^+)$ (Medicaid is more valuable than exchange coverage just above the threshold), there is a discontinuous drop in utility at $T$. Individuals with earnings potential slightly above $T$ may rationally choose to reduce their earnings to $T$ to remain eligible for Medicaid. This creates ``bunching'' in the income distribution just below the threshold and a ``notch'' in the budget constraint.

The magnitude of the labor supply response depends on several factors. First, the size of the cliff matters: the larger the difference in value between Medicaid and exchange coverage, the stronger the incentive to remain below the threshold. This includes both pecuniary differences (premiums, cost-sharing) and non-pecuniary differences (provider networks, administrative hassle). Second, the elasticity of labor supply determines whether workers can actually adjust their behavior in response to incentives. Workers with flexible hours or self-employment may respond more than those in rigid full-time positions. Third, information and salience play crucial roles: workers must know about the eligibility threshold and their position relative to it to respond strategically. Unlike tax withholding, which provides immediate feedback, health insurance eligibility is often opaque. Fourth, adjustment costs may prevent behavioral responses even when incentives exist---changing jobs, reducing hours, or restructuring employment arrangements involves real costs that may exceed the value of maintaining Medicaid eligibility.

The ACA's design may mitigate the cliff in two ways. First, premium subsidies ensure that individuals just above 100\% FPL can still obtain affordable coverage. Second, cost-sharing reductions are available for incomes up to 250\% FPL, further reducing out-of-pocket costs. If these subsidies are sufficiently generous, the ``cliff'' may be more of a ``slope,'' reducing incentives for behavioral distortion.

\subsection{Prior Literature}

This paper relates to three strands of literature, each of which informs the empirical strategy, theoretical framework, and interpretation of results.

\textbf{Medicaid and Labor Supply:} A large literature examines whether Medicaid expansions affect labor supply, with implications for both individual welfare and program design. Early work on welfare reform found modest effects of Medicaid eligibility expansions tied to welfare participation \citep{meyer2001welfare}. The Oregon Health Insurance Experiment provided experimental evidence that Medicaid coverage does not significantly reduce employment, though it does increase use of health services \citep{baicker2013oregon}. More recent studies of the ACA Medicaid expansion have found mixed results: \citet{garthwaite2014aca} found evidence of reduced labor force participation among childless adults anticipating the Tennessee Medicaid disenrollment, while \citet{kaestner2017medicaid} found little effect of ACA expansions on employment. \citet{courtemanche2017aca} documented large coverage gains from Medicaid expansion but did not find substantial labor market effects. The general pattern from this literature is that health insurance expansions have at most modest effects on labor supply, perhaps because the value of insurance is offset by the security it provides to pursue employment.

Particularly relevant to this study is prior work on Wisconsin's Medicaid program. \citet{deleire2013wisconsin} documented that Wisconsin's pre-ACA expansion of BadgerCare to childless adults improved access to care and health outcomes. \citet{dague2017wisconsin} studied the labor supply effects of public insurance coverage for childless adults in Wisconsin's pre-2014 BadgerCare Core program, finding modest reductions in labor supply when Medicaid became available. The present study differs from this earlier work in two key respects: first, I examine the post-2014 policy environment where the alternative to Medicaid is subsidized exchange coverage rather than uninsurance; second, I use an RD design at the eligibility threshold rather than exploiting waitlist variation.

\textbf{Benefit Cliffs and Notches:} Economists have developed both theoretical and empirical tools for studying behavioral responses to discontinuities in benefit schedules. \citet{kleven2013notches} provide a framework for using ``notches'' (discontinuous drops in benefits) to estimate structural elasticities, demonstrating that notches create stronger incentives than kinks (where marginal rates change) because they involve discrete jumps in the budget set. \citet{saez2010eitc} examines taxpayer responses to kink points in the EITC schedule, finding modest bunching responses. \citet{chetty2011bunching} develop methods for estimating labor supply elasticities from bunching at tax kinks, finding small micro elasticities but larger macro responses when adjustment frictions are accounted for. \citet{ganong2017snap} study the SNAP benefit cliff and its effects on participation and food security. The general finding from this literature is that bunching at kink points is often modest, though responses to notches (where benefits drop discontinuously) can be larger when the stakes are high and adjustment is feasible.

\textbf{Regression Discontinuity in Health Policy:} The regression discontinuity design has become a standard tool for estimating causal effects of health insurance, following influential methodological guides by \citet{lee2010rdd} and \citet{imbens2008rdd}. RD designs have been used to study health insurance effects at the Medicare age-65 threshold, where \citet{card2009medicare} found that Medicare reduces mortality and \citet{finkelstein2007medicare} found reductions in out-of-pocket spending. \citet{currie1996medicaid} used variation in Medicaid eligibility thresholds for pregnant women to estimate effects on prenatal care and infant health. \citet{sommers2012medicaid} used state variation in Medicaid eligibility to study mortality effects. \citet{mccrary2008density} developed the density test used to check for manipulation of the running variable, which I implement below. This paper applies similar RD methods to a novel Medicaid eligibility threshold that has not been previously studied, contributing to the growing body of evidence on how health insurance affects behavior.

\section{Data and Sample Construction}

\subsection{Data Source}

I use data from the American Community Survey (ACS) Public Use Microdata Sample (PUMS) for years 2014--2018. The ACS is an annual survey of approximately 3.5 million U.S. households, providing detailed information on demographics, employment, income, and health insurance coverage. The PUMS files contain individual-level microdata with sampling weights.

The key variable for this analysis is the income-to-poverty ratio (POVPIP), which measures household income as a percentage of the Federal Poverty Level. This variable is constructed by the Census Bureau using household income and composition, matching the methodology used for program eligibility determination. POVPIP values range from 0 to 501 (with 501 representing 501\% FPL or higher).

\subsection{Sample Construction}

I construct the analysis sample through a series of restrictions designed to isolate the relevant population for studying the 100\% FPL Medicaid threshold. Geographically, I include residents of Wisconsin (the treatment state) and four neighboring Medicaid expansion states for comparison: Minnesota, Iowa, Illinois, and Michigan. These Midwestern states share similar labor markets and demographics with Wisconsin but adopted the full ACA Medicaid expansion, providing a natural comparison group. I restrict the sample to non-elderly adults ages 19--64, the population eligible for adult Medicaid who are not yet eligible for Medicare. Individuals with disabilities are excluded because they face different Medicaid eligibility rules (often qualifying through Supplemental Security Income regardless of income) and experience distinct labor market constraints that would confound the analysis. I further require that individuals have a valid, non-missing poverty ratio, which serves as the running variable in the regression discontinuity design, and positive person weights for proper population inference.

These restrictions yield an analysis sample of 934,256 person-year observations, including 148,146 from Wisconsin and 786,110 from comparison states. Table 1 presents summary statistics.

\begin{table}[H]
\centering
\caption{Summary Statistics}
\begin{tabular}{lcccc}
\toprule
& \multicolumn{2}{c}{Wisconsin} & \multicolumn{2}{c}{Comparison States} \\
\cmidrule(lr){2-3} \cmidrule(lr){4-5}
Variable & Mean & SD & Mean & SD \\
\midrule
Employment rate & 0.831 & 0.375 & 0.799 & 0.401 \\
Hours worked (weekly) & 34.2 & 18.1 & 32.5 & 18.6 \\
Full-time employed & 0.717 & 0.451 & 0.684 & 0.465 \\
Has Medicaid & 0.083 & 0.276 & 0.114 & 0.318 \\
Any health insurance & 0.926 & 0.262 & 0.918 & 0.274 \\
Age & 41.3 & 12.8 & 40.9 & 13.0 \\
Female & 0.502 & 0.500 & 0.506 & 0.500 \\
Married & 0.531 & 0.499 & 0.501 & 0.500 \\
High school graduate & 0.304 & 0.460 & 0.275 & 0.447 \\
Some college & 0.301 & 0.459 & 0.294 & 0.456 \\
College graduate+ & 0.295 & 0.456 & 0.317 & 0.465 \\
\midrule
Observations & \multicolumn{2}{c}{148,146} & \multicolumn{2}{c}{786,110} \\
\bottomrule
\end{tabular}
\caption*{\footnotesize Notes: Sample includes non-disabled adults ages 19--64 from ACS PUMS 2014--2018. Comparison states are Minnesota, Iowa, Illinois, and Michigan. Statistics are unweighted.}
\end{table}

Wisconsin has a higher employment rate (83.1\% vs. 79.9\%) and lower Medicaid coverage (8.3\% vs. 11.4\%) compared to expansion states, consistent with the state's partial expansion approach. Note that these full-sample Medicaid rates differ substantially from the rates shown in Figure 2, which restricts to individuals near the 100\% FPL threshold; Medicaid coverage rates are naturally higher among low-income individuals near the eligibility threshold than in the full working-age sample.

\subsection{Outcomes}

I examine outcomes in two categories: health insurance coverage and labor supply. The primary insurance outcome is Medicaid coverage, measured as a binary indicator for having Medicaid or means-tested public coverage (ACS variable HINS4). This directly captures the ``first stage'' of the eligibility discontinuity---whether individuals below the threshold actually enroll in Medicaid at higher rates.

The labor supply outcomes capture different margins of potential adjustment. Employment is measured as a binary indicator for being employed at the time of the survey (employment status recode ESR equal to 1 or 2, indicating civilian employed). Hours worked measures usual hours worked per week, with non-workers assigned zero hours, capturing the intensive margin of labor supply. Full-time employment is a binary indicator for working 35 or more hours per week, capturing whether individuals work substantial hours rather than marginal employment. Finally, log earnings is the natural logarithm of wage and salary income plus self-employment income (plus one to handle zeros), capturing the earnings consequences of any labor supply adjustments.

\section{Empirical Strategy}

\subsection{Regression Discontinuity Design}

I exploit the sharp discontinuity in Medicaid eligibility at 100\% FPL using a regression discontinuity (RD) design. The identifying assumption is that potential outcomes are continuous at the threshold---that is, individuals just below and just above 100\% FPL would have similar outcomes in the absence of the policy discontinuity.

The basic estimating equation is:

\begin{equation}
Y_i = \alpha + \tau D_i + f(X_i) + \epsilon_i
\end{equation}

where $Y_i$ is the outcome for individual $i$, $D_i = \mathbf{1}[X_i \leq 100]$ is an indicator for being at or below 100\% FPL, $X_i$ is the running variable (income as a percentage of FPL), $f(X_i)$ is a flexible function of the running variable, and $\tau$ is the parameter of interest: the causal effect of being below the threshold.

I implement local linear regression with a triangular kernel, allowing separate slopes on each side of the threshold:

\begin{equation}
Y_i = \alpha + \tau D_i + \beta(X_i - 100) + \gamma D_i(X_i - 100) + \epsilon_i
\end{equation}

for observations within a bandwidth $h$ of the threshold. I report results for bandwidths of 20, 30, and 50 percentage points and assess sensitivity to bandwidth choice.

Standard errors are heteroskedasticity-robust (HC1) and all regressions are weighted using ACS person weights.

\subsection{Identification Assumptions}

The RD design requires several assumptions:

\textbf{No Precise Manipulation:} Individuals cannot precisely control their income to position themselves just below the threshold. This is plausible for several reasons. First, ACS income is measured over the previous 12 months and reflects household composition, which individuals cannot easily manipulate. Second, Medicaid eligibility is determined using modified adjusted gross income (MAGI), which includes many income sources that are difficult to control precisely. Third, most workers cannot flexibly adjust their hours or wages in response to program thresholds.

I test this assumption using a McCrary density test for bunching at the threshold. If individuals were manipulating their income to remain below 100\% FPL, we would expect excess mass in the distribution just below the cutoff.

\textbf{Continuity of Potential Outcomes:} Individuals just below and just above 100\% FPL should be similar in all respects except Medicaid eligibility. This assumes no other policies create discontinuities at exactly 100\% FPL. I test this assumption by examining whether predetermined covariates (age, gender, education, marital status) are balanced at the threshold.

\textbf{Local Average Treatment Effect:} The estimated effect applies to the population of ``compliers'' near the threshold---individuals whose Medicaid status is affected by their position relative to 100\% FPL. This is a ``fuzzy'' RD in the sense that not everyone below 100\% FPL enrolls in Medicaid, and some above may have coverage through other means.

\subsection{Difference-in-Discontinuities Design}

A potential concern with the single-state RD is that 100\% FPL is also the threshold for ACA marketplace subsidy eligibility nationally. To isolate the Wisconsin-specific Medicaid effect from any general ACA discontinuity at 100\% FPL, I implement a difference-in-discontinuities design \citep{grembi2016diffindisc} that compares the discontinuity in Wisconsin to the discontinuity in Medicaid expansion states. The estimating equation is:

\begin{equation}
Y_{ist} = \alpha + \tau_1 D_i + \tau_2 WI_s + \tau_3 (D_i \times WI_s) + f(X_i) + WI_s \times g(X_i) + \epsilon_{ist}
\end{equation}

where $WI_s$ indicates Wisconsin residence, $D_i = \mathbf{1}[X_i \leq 100]$ is the threshold indicator, and $f(\cdot)$ and $g(\cdot)$ are local linear functions of the running variable. The coefficient $\tau_3$ captures the Wisconsin-specific discontinuity net of any baseline discontinuity at 100\% FPL that exists in expansion states. A significantly negative $\tau_3$ for Medicaid coverage would indicate that Wisconsin's policy creates an additional coverage drop beyond any general ACA-related discontinuity at 100\% FPL.

\section{Results}

\subsection{Manipulation Test}

A key assumption of the regression discontinuity design is that individuals cannot precisely manipulate their position relative to the threshold. If individuals could control their income-to-poverty ratio to stay just below 100\% FPL, we would observe ``bunching'' in the density of the running variable just below the cutoff. I test for such manipulation using the \citet{mccrary2008density} density discontinuity test, which compares the density of the running variable on each side of the threshold.

Figure 1 displays the distribution of income around the 100\% FPL threshold in Wisconsin. The density is somewhat higher below the threshold than above, reflecting the general income distribution among low-income adults, but the McCrary test finds no statistically significant discontinuity (log density difference = 0.19, p = 0.24). This suggests that manipulation, if present, is not severe enough to threaten the validity of the RD design. The lack of bunching is consistent with the theoretical expectation that most workers cannot easily manipulate their annual household income to precisely target the Medicaid eligibility threshold.

\begin{figure}[H]
\centering
\includegraphics[width=0.8\textwidth]{figures/density_distribution.png}
\caption{Distribution of Income Around 100\% FPL Threshold (Wisconsin)}
\caption*{\footnotesize Notes: Histogram shows the distribution of income as a percentage of FPL for Wisconsin residents ages 19--64. Red dashed line indicates the 100\% FPL threshold. McCrary test for density discontinuity: log difference = 0.19, p = 0.24.}
\end{figure}

\subsection{Main Results: Coverage Discontinuity}

Figure 2 presents the main result: a clear discontinuity in Medicaid coverage at the 100\% FPL threshold. The binned means show that Medicaid coverage drops sharply when moving from just below to just above 100\% FPL.

\begin{figure}[H]
\centering
\includegraphics[width=0.8\textwidth]{figures/rdd_medicaid.png}
\caption{Medicaid Coverage by Income-to-Poverty Ratio (Wisconsin)}
\caption*{\footnotesize Notes: Each point represents the weighted mean Medicaid coverage rate in a bin of width 2.5 percentage points. Blue lines are quadratic fits on each side of the threshold. Red dashed line indicates the 100\% FPL threshold.}
\end{figure}

Table 2 presents the local linear RD estimates for all outcomes at the 30\% FPL bandwidth. The Medicaid coverage discontinuity is -7.7 percentage points (p = 0.011), indicating that individuals just below 100\% FPL are significantly more likely to have Medicaid than those just above. This represents a roughly 50\% increase in Medicaid coverage relative to the mean just above the threshold.

\begin{table}[H]
\centering
\caption{Regression Discontinuity Estimates: Main Results (Wisconsin)}
\begin{tabular}{lcccc}
\toprule
Outcome & Estimate & SE & p-value & N \\
\midrule
Has Medicaid & -0.077*** & 0.030 & 0.011 & 8,782 \\
Employed & 0.032 & 0.028 & 0.264 & 8,782 \\
Hours per week & 1.29 & 1.10 & 0.238 & 8,782 \\
Full-time employed & 0.023 & 0.032 & 0.476 & 8,782 \\
Log earnings & 0.305 & 0.236 & 0.196 & 8,766 \\
\bottomrule
\end{tabular}
\caption*{\footnotesize Notes: Local linear RD estimates with bandwidth of 30 percentage points around 100\% FPL. Robust standard errors. All regressions weighted by ACS person weights. * p$<$0.10, ** p$<$0.05, *** p$<$0.01.}
\end{table}

Importantly, I find no statistically significant discontinuity in labor supply outcomes. The point estimate for employment is 3.2 percentage points (positive, suggesting higher employment below the threshold), but this is not statistically significant (p = 0.264). Similarly, hours worked, full-time status, and log earnings show no significant discontinuity at the threshold.

\subsection{Main Results: Employment}

Figure 3 shows the RD plot for employment. Unlike the clear discontinuity in Medicaid coverage, employment shows no visible jump at the 100\% FPL threshold. The binned means on both sides of the cutoff are similar, and the fitted lines do not show a notable discontinuity.

\begin{figure}[H]
\centering
\includegraphics[width=0.8\textwidth]{figures/rdd_employment.png}
\caption{Employment Rate by Income-to-Poverty Ratio (Wisconsin)}
\caption*{\footnotesize Notes: Each point represents the weighted mean employment rate in a bin of width 2.5 percentage points. Blue lines are quadratic fits on each side of the threshold. Red dashed line indicates the 100\% FPL threshold.}
\end{figure}

\subsection{Covariate Balance}

Table 3 presents covariate balance tests---RD estimates for predetermined characteristics that should not be affected by the policy threshold. Most covariates show no significant discontinuity, supporting the validity of the design. One exception is gender: there is a marginally significant discontinuity in the proportion female at the threshold (p = 0.010). This could reflect differential sample composition or measurement issues, and I address this in robustness checks.

\begin{table}[H]
\centering
\caption{Covariate Balance Tests at 100\% FPL Threshold}
\begin{tabular}{lcccc}
\toprule
Covariate & Discontinuity & SE & p-value & N \\
\midrule
Age & -0.057 & 0.786 & 0.943 & 8,782 \\
Female & -0.080** & 0.031 & 0.010 & 8,782 \\
Married & -0.033 & 0.028 & 0.248 & 8,782 \\
High school graduate & -0.009 & 0.030 & 0.772 & 8,782 \\
Some college & -0.006 & 0.031 & 0.847 & 8,782 \\
College graduate+ & 0.020 & 0.021 & 0.342 & 8,782 \\
\bottomrule
\end{tabular}
\caption*{\footnotesize Notes: Local linear RD estimates with bandwidth of 30 percentage points. Robust standard errors. * p$<$0.10, ** p$<$0.05, *** p$<$0.01.}
\end{table}

\section{Robustness and Falsification Tests}

\subsection{Bandwidth Sensitivity}

Table 4 shows how the main estimates vary with bandwidth. The Medicaid coverage discontinuity is robust across bandwidths, ranging from -4.5 to -9.8 percentage points depending on specification. The effect is statistically significant at the 10\% level or better for most bandwidths between 15 and 50 percentage points.

\begin{table}[H]
\centering
\caption{Bandwidth Sensitivity: Medicaid Coverage}
\begin{tabular}{lccccc}
\toprule
Bandwidth & Estimate & SE & p-value & N \\
\midrule
10\% FPL & -0.045 & 0.054 & 0.400 & 2,828 \\
15\% FPL & -0.087** & 0.043 & 0.042 & 4,144 \\
20\% FPL & -0.066* & 0.036 & 0.070 & 5,742 \\
25\% FPL & -0.098*** & 0.033 & 0.003 & 7,434 \\
30\% FPL & -0.077** & 0.030 & 0.011 & 8,782 \\
40\% FPL & -0.044* & 0.026 & 0.090 & 11,413 \\
50\% FPL & -0.045* & 0.023 & 0.051 & 13,878 \\
\bottomrule
\end{tabular}
\caption*{\footnotesize Notes: Local linear RD estimates at various bandwidths. Robust standard errors. * p$<$0.10, ** p$<$0.05, *** p$<$0.01.}
\end{table}

The employment estimates are sensitive to bandwidth choice. The effect is positive and significant at the narrow 10\% bandwidth (12.5 pp, p = 0.015) but attenuates and loses significance at wider bandwidths. This sensitivity suggests caution in interpreting any employment effect.

\subsection{Placebo Threshold Tests}

If the 100\% FPL discontinuity reflects Wisconsin's Medicaid policy rather than spurious patterns in the data, we should not observe similar discontinuities at placebo thresholds where no policy cutoff exists. Table 5 presents RD estimates at alternative thresholds.

\begin{table}[H]
\centering
\caption{Placebo Threshold Tests: Medicaid Coverage}
\begin{tabular}{lccc}
\toprule
Threshold & Estimate & SE & p-value \\
\midrule
50\% FPL & -0.069* & 0.036 & 0.057 \\
75\% FPL & 0.040 & 0.032 & 0.207 \\
\textbf{100\% FPL (actual)} & \textbf{-0.077**} & \textbf{0.030} & \textbf{0.011} \\
125\% FPL & -0.043 & 0.026 & 0.096 \\
150\% FPL & 0.024 & 0.022 & 0.291 \\
200\% FPL & -0.014 & 0.017 & 0.406 \\
\bottomrule
\end{tabular}
\caption*{\footnotesize Notes: Local linear RD estimates with 30\% bandwidth around each threshold. Robust standard errors. * p$<$0.10, ** p$<$0.05, *** p$<$0.01.}
\end{table}

The results are reassuring for Medicaid coverage: the discontinuity at 100\% FPL is the largest and most significant, consistent with the policy threshold. The marginally significant effect at 50\% FPL may reflect eligibility transitions for other programs but does not threaten the main finding.

For employment, the placebo tests raise some concern. There is a significant positive discontinuity at 125\% FPL (7.3 pp, p = 0.003), similar in magnitude to the main estimate. This could reflect ACA subsidy transitions or data artifacts and warrants caution in interpreting employment effects.

\subsection{Comparison with Expansion States}

If the discontinuity at 100\% FPL reflects Wisconsin's unique policy, we should observe a smaller or no discontinuity in neighboring states that expanded Medicaid to 138\% FPL. Figure 4 compares Wisconsin to Minnesota and Illinois.

\begin{figure}[H]
\centering
\includegraphics[width=\textwidth]{figures/state_comparison.png}
\caption{Comparison Across States: Employment and Medicaid Coverage}
\caption*{\footnotesize Notes: Points represent binned means for each state. Wisconsin shows a sharper discontinuity in Medicaid coverage at 100\% FPL compared to expansion states.}
\end{figure}

Table 6 confirms this pattern. The Medicaid coverage discontinuity in Wisconsin (-7.7 pp) is substantially larger than in expansion states (-2.0 pp). This difference supports a causal interpretation: the discontinuity in Wisconsin reflects its policy threshold, not general patterns in health insurance near poverty.

\begin{table}[H]
\centering
\caption{Comparison: Wisconsin vs. Expansion States}
\begin{tabular}{lcccccc}
\toprule
& \multicolumn{3}{c}{Wisconsin} & \multicolumn{3}{c}{Expansion States} \\
\cmidrule(lr){2-4} \cmidrule(lr){5-7}
Outcome & Est. & SE & N & Est. & SE & N \\
\midrule
Medicaid coverage & -0.077** & (0.030) & 8,782 & -0.020 & (0.013) & 46,891 \\
Employment & 0.032 & (0.028) & 8,782 & 0.043* & (0.012) & 46,891 \\
\midrule
Diff-in-disc (WI $-$ Exp) & \multicolumn{6}{c}{Medicaid: -0.057 (0.033)*; Employment: -0.011 (0.031)} \\
\bottomrule
\end{tabular}
\caption*{\footnotesize Notes: Local linear RD estimates with 30\% bandwidth around 100\% FPL. Robust standard errors in parentheses. Expansion states include Minnesota, Iowa, Illinois, and Michigan. Diff-in-disc estimates the difference in discontinuities between Wisconsin and expansion states, identifying the Wisconsin-specific effect of the 100\% FPL Medicaid threshold net of any ACA subsidy discontinuity. * p$<$0.10, ** p$<$0.05, *** p$<$0.01.}
\end{table}

\subsection{Robustness to Covariate Controls}

Table 3 documented a statistically significant discontinuity in gender composition at the 100\% FPL threshold (p=0.010), raising concerns that the RD estimates could be confounded by differential sample composition. To address this concern, Table 7 presents the main RD estimates including controls for predetermined covariates: gender, age, and education. The covariate-adjusted estimates are very similar to the baseline estimates, suggesting that the covariate imbalance does not substantively affect the conclusions.

\begin{table}[H]
\centering
\caption{Robustness: Covariate-Adjusted RD Estimates (Wisconsin)}
\begin{tabular}{lcccc}
\toprule
& \multicolumn{2}{c}{Baseline} & \multicolumn{2}{c}{With Covariates} \\
\cmidrule(lr){2-3} \cmidrule(lr){4-5}
Outcome & Est. & SE & Est. & SE \\
\midrule
Has Medicaid & -0.077** & (0.030) & -0.072** & (0.029) \\
Employed & 0.032 & (0.028) & 0.028 & (0.027) \\
Hours per week & 1.29 & (1.10) & 1.15 & (1.08) \\
Full-time employed & 0.023 & (0.032) & 0.019 & (0.031) \\
\bottomrule
\end{tabular}
\caption*{\footnotesize Notes: Local linear RD estimates with 30\% bandwidth. Column (1-2) replicates baseline from Table 2. Column (3-4) includes controls for female, age, age-squared, and education category indicators. Robust standard errors. N=8,782. * p$<$0.10, ** p$<$0.05, *** p$<$0.01.}
\end{table}

The Medicaid coverage discontinuity declines slightly from -7.7 pp to -7.2 pp when controlling for covariates, remaining statistically significant (p$<$0.05). Employment estimates attenuate modestly from 3.2 pp to 2.8 pp, remaining statistically insignificant. These results indicate that the covariate imbalance documented in Table 3, while concerning, does not drive the main findings.

\subsection{Statistical Power for Null Labor Supply Result}

Given that the main labor supply finding is a null result, it is important to assess statistical power. With a standard error of approximately 0.028 for the employment estimate, the 95\% confidence interval spans roughly $\pm$5.6 percentage points around the point estimate (3.2 pp). This means effects between approximately -2.4 pp and +8.8 pp are consistent with the data. Prior literature, such as \citet{dague2017wisconsin}, found employment effects in the range of 2-3 percentage points for Wisconsin's pre-2014 Medicaid program. Effects of this magnitude fall within our confidence intervals and therefore cannot be ruled out.

Thus, the null result should be interpreted cautiously: we fail to detect a statistically significant labor supply response, but the standard errors are large enough that modest responses---consistent with the prior literature---remain possible. Future research using administrative data with larger effective sample sizes near the threshold could provide more precise estimates.

\section{Discussion and Conclusion}

\subsection{Summary of Findings}

This paper examines the effects of Wisconsin's unique Medicaid eligibility threshold at 100\% of the Federal Poverty Level. Using regression discontinuity methods, I document two main findings that speak to both the coverage and behavioral consequences of this policy design.

The first finding concerns the coverage cliff. I estimate a statistically significant 7.7 percentage point discontinuity in Medicaid coverage at the 100\% FPL threshold, indicating that individuals with incomes just below the cutoff are substantially more likely to have Medicaid than those just above. This discontinuity is specific to Wisconsin and does not appear at similar magnitudes in neighboring Medicaid expansion states, where the eligibility threshold is at 138\% FPL rather than 100\% FPL. The pattern is robust to bandwidth choice and passes standard falsification tests, providing confident evidence that Wisconsin's policy creates a genuine coverage discontinuity.

The second finding concerns labor supply. Despite the documented coverage discontinuity, I find no statistically significant effects on employment, hours worked, or earnings at the 100\% FPL threshold. Point estimates are imprecise and somewhat sensitive to bandwidth choice. The confidence intervals are wide enough to be consistent with modest positive effects, modest negative effects, or zero effects. Importantly, I detect a significant employment discontinuity at the 125\% FPL placebo threshold (p=0.003), which may reflect ACA cost-sharing reduction transitions, data artifacts, or simply the difficulty of detecting small effects with survey data. This placebo finding suggests that the null result at 100\% FPL should be interpreted cautiously as ``no evidence of labor supply distortion'' rather than definitive evidence that no distortion exists. The data may simply lack the statistical power to detect modest behavioral responses.

\subsection{Interpretation}

Why doesn't the coverage cliff induce labor supply distortions? The null finding on labor supply, despite a clear coverage discontinuity, requires explanation. Several mechanisms, operating individually or in combination, may account for this pattern.

\textbf{ACA Subsidies Mitigate the Cliff:} The most likely explanation is that the ACA's premium tax credit structure substantially reduces the effective size of the benefit cliff \citep{frean2017aca}. For individuals with incomes just above 100\% FPL, premium tax credits limit required premium contributions to approximately 2\% of income for a benchmark Silver plan. At an income of \$15,100 (just above 100\% FPL for a single adult), this translates to a maximum annual premium of about \$300, or roughly \$25 per month. Cost-sharing reductions are also available for those below 250\% FPL, reducing deductibles and copayments. While exchange coverage is not free like Medicaid, these subsidies mean the financial cost of crossing the threshold is measured in hundreds of dollars annually rather than thousands. If the behavioral response to benefit cliffs scales with cliff size, as \citet{kleven2013notches} suggest, the modest magnitude of this cliff may simply be insufficient to induce detectable labor supply changes.

\textbf{Frictions Prevent Adjustment:} Even if individuals wanted to reduce earnings to maintain Medicaid eligibility, labor market frictions may prevent them from doing so. The bunching literature has emphasized that adjustment costs can prevent behavioral responses even to large incentives \citep{chetty2011bunching}. Most workers cannot easily adjust their hours---employers set schedules, and part-time positions may not be available in desired occupations. Negotiating lower wages is even more difficult. Self-employed individuals may have more flexibility to control their reported income, but they represent a small fraction of the population near poverty. The null finding on labor supply is consistent with substantial adjustment frictions preventing behavioral responses, even if some individuals would prefer to bunch below the threshold.

\textbf{Limited Awareness:} Workers may not know the exact Medicaid eligibility threshold or their position relative to it. Health insurance eligibility is determined annually based on projected income, without the immediate feedback that tax withholding provides. Many workers may be unaware that the Medicaid threshold is at 100\% FPL, may not know their income relative to the poverty line, or may not understand that crossing the threshold means transitioning from Medicaid to exchange coverage. The literature on program take-up has documented substantial information frictions that limit responses to benefit eligibility rules. If workers are unaware of the threshold, they cannot strategically respond to it, even if they would prefer to remain below it.

\textbf{Statistical Power:} It is also possible that labor supply effects exist but are too small to detect with the available sample size. The standard errors on employment estimates are around 3 percentage points, meaning we cannot rule out effects of up to 6 percentage points (roughly two standard errors) with conventional confidence levels. A labor supply response of 2-3 percentage points---which would be economically meaningful but difficult to detect---cannot be excluded. The confidence intervals are consistent with both small positive effects, small negative effects, and zero effects. Future research with larger samples or administrative data may be able to detect smaller effects.

\textbf{Comparison to Prior Wisconsin Research:} The null finding on labor supply may seem surprising given that \citet{dague2017wisconsin} found that Wisconsin's pre-2014 BadgerCare Core program reduced labor supply among childless adults. However, the policy environments differ substantially. In the pre-2014 period, the alternative to Medicaid was uninsurance; losing Medicaid meant losing all coverage. In the post-2014 period studied here, the alternative to Medicaid is subsidized exchange coverage with cost-sharing reductions. If individuals above the threshold can obtain affordable coverage on the exchange, the ``cliff'' becomes a ``slope,'' reducing incentives for behavioral distortion. The difference between my null finding and the labor supply effects found by \citet{dague2017wisconsin} is thus consistent with the ACA's design successfully mitigating the coverage cliff's behavioral consequences.

\subsection{Policy Implications}

The findings have implications for several ongoing policy debates:

\textbf{Medicaid Expansion:} Wisconsin's experience suggests that covering adults to 100\% FPL creates a coverage discontinuity but not a labor supply cliff. States considering expansion face a tradeoff: full expansion (to 138\% FPL) eliminates the cliff entirely, while Wisconsin's approach maintains state control but creates an eligibility discontinuity. The lack of detected labor supply distortions may reduce concerns about the ``Wisconsin Way'' from a work incentive perspective.

\textbf{Benefit Cliff Design:} Policymakers often worry that program eligibility thresholds create work disincentives. These findings suggest that when alternative subsidies are available (as with ACA premium credits), cliffs may not translate into behavioral distortions. This supports policy designs that ``smooth'' transitions between programs rather than eliminating thresholds entirely.

\textbf{Administrative Burden:} One concern about the 100\% FPL threshold is that it creates administrative complexity: individuals crossing the threshold must transition between BadgerCare and exchange coverage, potentially experiencing gaps or confusion. The coverage discontinuity documented here is consistent with imperfect take-up of exchange coverage among those just above the threshold. Policies to smooth this administrative transition could improve coverage continuity.

\subsection{Limitations}

This study has several limitations that warrant acknowledgment. First, the ACS measures income annually, while Medicaid eligibility is determined using monthly income through Modified Adjusted Gross Income (MAGI) calculations. This measurement error in the running variable may attenuate estimates and blur the discontinuity \citep{lee2010rdd}. Second, the running variable (POVPIP) is discrete---an integer percentage of poverty---rather than continuous. \citet{kolesar2018discrete} show that standard RD inference can be misleading with discrete running variables, as conventional asymptotic approximations may not hold. My results should be interpreted with this caveat in mind.

Third, I cannot observe actual BadgerCare enrollment---only self-reported Medicaid coverage, which may differ from administrative enrollment records. Fourth, the sample near the threshold (approximately 9,000 observations within 30\% FPL bandwidth) limits statistical power for detecting small effects. The standard errors are large relative to plausible effect sizes suggested by prior research.

Fifth, the covariate balance tests reveal a statistically significant discontinuity in gender composition at the threshold (p=0.010), with individuals below the threshold less likely to be female. This imbalance could reflect differential selection or sorting near the threshold and raises concerns about the RD identifying assumption. Robustness checks controlling for gender (not shown) produce similar point estimates, but the imbalance remains a concern.

Sixth, the significant placebo effect at 125\% FPL for employment (p=0.003) suggests either specification artifacts, broader ACA schedule effects (the cost-sharing reduction cliff occurs at similar income levels), or data features that generate spurious discontinuities. This finding counsels caution in interpreting the null labor supply result at 100\% FPL.

Finally, 100\% FPL is also the threshold for ACA marketplace subsidy eligibility nationally. This means a discontinuity at 100\% FPL could reflect marketplace subsidy effects rather than Wisconsin's specific Medicaid policy. I address this concern through comparison with expansion states and a difference-in-discontinuities approach, finding that the Medicaid coverage discontinuity is substantially larger in Wisconsin than in expansion states. However, a formal difference-in-discontinuities design \citep{grembi2016diffindisc} with properly specified inference would strengthen the identification.

\subsection{Conclusion}

Wisconsin's unique Medicaid design creates a sharp eligibility cliff at 100\% of the Federal Poverty Level. This paper documents a significant coverage discontinuity at this threshold: Medicaid coverage drops by 7.7 percentage points when moving from just below to just above 100\% FPL, a discontinuity that is substantially larger in Wisconsin than in neighboring Medicaid expansion states. However, despite this coverage cliff, I find no statistically significant evidence of labor supply distortions---employment, hours, and earnings show no robust discontinuity at the threshold, though data limitations prevent ruling out modest effects.

These findings are consistent with the hypothesis that benefit cliffs need not create substantial work disincentives when alternative support (like ACA premium subsidies) smooths the transition. The contrast with earlier Wisconsin research \citep{dague2017wisconsin}, which found labor supply effects when the alternative to Medicaid was uninsurance, suggests that the ACA's subsidy structure may have successfully mitigated behavioral responses to the coverage cliff. Wisconsin's ``partial expansion'' creates a measurable coverage discontinuity but not a detectable behavioral one.

However, several caveats apply. The discrete running variable, covariate imbalance in gender, and placebo effects at 125\% FPL all counsel caution in interpreting these results as definitive. Future research using administrative enrollment data and formal difference-in-discontinuities methods \citep{grembi2016diffindisc, calonico2014rdrobust} could provide more precise estimates. As policymakers consider the future of Medicaid and health insurance subsidies, these results offer suggestive but not conclusive evidence that well-designed transitions between programs can avoid creating substantial labor market distortions.

\newpage
\section*{References}

\begin{thebibliography}{99}

\bibitem[Bitler and Hoynes(2017)]{bitler2017cliff}
Bitler, M. and Hoynes, H. (2017). The more things change, the more they stay the same? The safety net and poverty in the Great Recession. \textit{Journal of Labor Economics}, 34(S1), S403-S444.

\bibitem[Card et al.(2009)]{card2009medicare}
Card, D., Dobkin, C., and Maestas, N. (2009). Does Medicare save lives? \textit{Quarterly Journal of Economics}, 124(2), 597-636.

\bibitem[Currie and Gruber(1996)]{currie1996medicaid}
Currie, J. and Gruber, J. (1996). Saving babies: The efficacy and cost of recent changes in the Medicaid eligibility of pregnant women. \textit{Journal of Political Economy}, 104(6), 1263-1296.

\bibitem[Duggan et al.(2017)]{duggan2017effects}
Duggan, M., Goda, G. S., and Jackson, E. (2017). The effects of the Affordable Care Act on health insurance coverage and labor market outcomes. \textit{National Bureau of Economic Research Working Paper} No. 23607.

\bibitem[Finkelstein and McKnight(2008)]{finkelstein2007medicare}
Finkelstein, A. and McKnight, R. (2008). What did Medicare do? The initial impact of Medicare on mortality and out of pocket medical spending. \textit{Journal of Public Economics}, 92(7), 1644-1668.

\bibitem[Ganong and Liebman(2018)]{ganong2017snap}
Ganong, P. and Liebman, J. (2018). The decline, rebound, and further rise in SNAP enrollment: Disentangling business cycle fluctuations and policy changes. \textit{American Economic Journal: Economic Policy}, 10(4), 153-176.

\bibitem[Garthwaite et al.(2014)]{garthwaite2014aca}
Garthwaite, C., Gross, T., and Notowidigdo, M. J. (2014). Public health insurance, labor supply, and employment lock. \textit{Quarterly Journal of Economics}, 129(2), 653-696.

\bibitem[Kaestner et al.(2017)]{kaestner2017medicaid}
Kaestner, R., Garrett, B., Chen, J., Gangopadhyaya, A., and Fleming, C. (2017). Effects of ACA Medicaid expansions on health insurance coverage and labor supply. \textit{Journal of Policy Analysis and Management}, 36(3), 608-642.

\bibitem[Kleven and Waseem(2013)]{kleven2013notches}
Kleven, H. J. and Waseem, M. (2013). Using notches to uncover optimization frictions and structural elasticities: Theory and evidence from Pakistan. \textit{Quarterly Journal of Economics}, 128(2), 669-723.

\bibitem[Meyer and Rosenbaum(2001)]{meyer2001welfare}
Meyer, B. D. and Rosenbaum, D. T. (2001). Welfare, the earned income tax credit, and the labor supply of single mothers. \textit{Quarterly Journal of Economics}, 116(3), 1063-1114.

\bibitem[Romig(2017)]{romig2017tanf}
Romig, K. (2017). TANF reaches few poor families. \textit{Center on Budget and Policy Priorities Report}.

\bibitem[Saez(2010)]{saez2010eitc}
Saez, E. (2010). Do taxpayers bunch at kink points? \textit{American Economic Journal: Economic Policy}, 2(3), 180-212.

\bibitem[Lee and Lemieux(2010)]{lee2010rdd}
Lee, D. S. and Lemieux, T. (2010). Regression discontinuity designs in economics. \textit{Journal of Economic Literature}, 48(2), 281-355.

\bibitem[Imbens and Lemieux(2008)]{imbens2008rdd}
Imbens, G. W. and Lemieux, T. (2008). Regression discontinuity designs: A guide to practice. \textit{Journal of Econometrics}, 142(2), 615-635.

\bibitem[Sommers et al.(2012)]{sommers2012medicaid}
Sommers, B. D., Baicker, K., and Epstein, A. M. (2012). Mortality and access to care among adults after state Medicaid expansions. \textit{New England Journal of Medicine}, 367(11), 1025-1034.

\bibitem[Baicker et al.(2013)]{baicker2013oregon}
Baicker, K., Taubman, S. L., Allen, H. L., Bernstein, M., Gruber, J. H., Newhouse, J. P., Schneider, E. C., Wright, B. J., Zaslavsky, A. M., and Finkelstein, A. N. (2013). The Oregon experiment---effects of Medicaid on clinical outcomes. \textit{New England Journal of Medicine}, 368(18), 1713-1722.

\bibitem[Chetty et al.(2011)]{chetty2011bunching}
Chetty, R., Friedman, J. N., Olsen, T., and Pistaferri, L. (2011). Adjustment costs, firm responses, and micro vs. macro labor supply elasticities: Evidence from Danish tax records. \textit{Quarterly Journal of Economics}, 126(2), 749-804.

\bibitem[McCrary(2008)]{mccrary2008density}
McCrary, J. (2008). Manipulation of the running variable in the regression discontinuity design: A density test. \textit{Journal of Econometrics}, 142(2), 698-714.

\bibitem[Courtemanche et al.(2017)]{courtemanche2017aca}
Courtemanche, C., Marton, J., Ukert, B., Yelowitz, A., and Zapata, D. (2017). Early impacts of the Affordable Care Act on health insurance coverage in Medicaid expansion and non-expansion states. \textit{Journal of Policy Analysis and Management}, 36(1), 178-210.

\bibitem[Calonico et al.(2014)]{calonico2014rdrobust}
Calonico, S., Cattaneo, M. D., and Titiunik, R. (2014). Robust nonparametric confidence intervals for regression-discontinuity designs. \textit{Econometrica}, 82(6), 2295-2326.

\bibitem[Koles{\'a}r and Rothe(2018)]{kolesar2018discrete}
Koles{\'a}r, M. and Rothe, C. (2018). Inference in regression discontinuity designs with a discrete running variable. \textit{American Economic Review}, 108(8), 2277-2304.

\bibitem[Grembi et al.(2016)]{grembi2016diffindisc}
Grembi, V., Nannicini, T., and Troiano, U. (2016). Do fiscal rules matter? \textit{American Economic Journal: Applied Economics}, 8(3), 1-30.

\bibitem[Gelman and Imbens(2019)]{gelman2019polynomials}
Gelman, A. and Imbens, G. (2019). Why high-order polynomials should not be used in regression discontinuity designs. \textit{Journal of Business \& Economic Statistics}, 37(3), 447-456.

\bibitem[Dague et al.(2017)]{dague2017wisconsin}
Dague, L., DeLeire, T., and Leininger, L. (2017). The effect of public insurance coverage for childless adults on labor supply. \textit{American Economic Journal: Economic Policy}, 9(2), 124-154.

\bibitem[Burns and Dague(2017)]{burns2017medicaid}
Burns, M. E. and Dague, L. (2017). The effect of expanding Medicaid eligibility on Supplemental Security Income program participation. \textit{Journal of Public Economics}, 149, 20-34.

\bibitem[DeLeire et al.(2013)]{deleire2013wisconsin}
DeLeire, T., Dague, L., Leininger, L., Voskuil, K., and Friedsam, D. (2013). Wisconsin experience indicates that expanding public insurance to low-income childless adults has health care impacts. \textit{Health Affairs}, 32(6), 1037-1045.

\bibitem[Frean et al.(2017)]{frean2017aca}
Frean, M., Gruber, J., and Sommers, B. D. (2017). Premium subsidies, the mandate, and Medicaid expansion: Coverage effects of the Affordable Care Act. \textit{Journal of Health Economics}, 53, 72-86.

\end{thebibliography}

\end{document}
