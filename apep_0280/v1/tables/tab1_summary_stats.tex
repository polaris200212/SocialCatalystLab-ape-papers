\begin{table}[htbp]
\centering
\caption{Summary Statistics: Pre-Treatment Means by Treatment Status}
\label{tab:summ_stats}
\begin{tabular}{lccc}
\toprule
 & Treated (Pre-Ban) & Never Treated & Difference \\
\midrule
Smoking Rate & 0.208 & 0.223 & -0.015*** \\
Everyday Smoking & 0.313 & 0.259 & 0.055*** \\
Quit Attempt Rate & 0.562 & 0.571 & -0.009** \\
Pct Female & 0.516 & 0.515 & 0.001* \\
Pct College & 0.307 & 0.257 & 0.050*** \\
Pct High Income & 0.667 & 0.587 & 0.081*** \\
\midrule
N (state-years) & 312 & 483 & \\
\bottomrule
\end{tabular}
\begin{minipage}{0.9\textwidth}
\vspace{0.5em}
\footnotesize
\textit{Notes:} Pre-treatment means for states that eventually adopted comprehensive indoor smoking bans (``Treated'') and states that never adopted (``Never Treated''). Difference = Treated $-$ Never Treated; significance from two-sample $t$-test. N (state-years) reflects only pre-treatment observations: for treated states, years before ban adoption; for never-treated states, all available years. The full estimation panel contains 1,120 state-year observations (51 jurisdictions $\times$ 22 survey years, minus 2 state-years with insufficient data). Smoking Rate is the share of adults who currently smoke. Quit Attempt Rate is the share of ever-smokers who attempted to quit in the past 12 months. Pct College is the share with a bachelor's degree or higher. All means are survey-weighted. $^{***}p<0.01$, $^{**}p<0.05$, $^{*}p<0.10$. Source: BRFSS 1996--2004, 2006--2016, 2021--2022.
\end{minipage}
\end{table}
