\begin{table}[htbp]
\centering
\caption{Summary Statistics: Postpartum Women (Pre-Treatment, 2017--2019)}
\label{tab:summary}
\begin{tabular}{lcc}
\toprule
 & Treated States & Control States \\
\midrule
N & 68,857 & 33,287 \\
Medicaid (\%) & 31.0 & 26.0 \\
Uninsured (\%) & 10.0 & 15.3 \\
Employer Ins (\%) & 54.4 & 53.2 \\
Age & 30.2 & 29.8 \\
Married (\%) & 64.5 & 66.3 \\
White NH (\%) & 53.9 & 51.6 \\
Black NH (\%) & 15.3 & 12.5 \\
Hispanic (\%) & 20.1 & 26.4 \\
BA+ (\%) & 35.9 & 33.5 \\
Below 200\% FPL (\%) & 41.8 & 44.4 \\
\bottomrule
\end{tabular}
\begin{tablenotes}[flushleft]
\small
\item \textit{Notes:} Sample is women aged 18--44 who gave birth in the past 12 months.
Pre-treatment period is 2017--2019 (before PHE and policy adoption).
Statistics are weighted using ACS person weights.
Treated states are those that adopted the 12-month postpartum extension.
Control states are Arkansas and Wisconsin (never adopted as of 2023).
\end{tablenotes}
\end{table}

