\documentclass[12pt]{article}

\usepackage[utf8]{inputenc}
\usepackage[T1]{fontenc}
\usepackage{lmodern}
\usepackage[margin=1in]{geometry}
\usepackage{setspace}
\onehalfspacing
\usepackage{amsmath,amssymb}
\usepackage{graphicx}
\usepackage{float}
\usepackage{booktabs}
\usepackage{array}
\usepackage{multirow}
\usepackage{threeparttable}
\usepackage{hyperref}
\hypersetup{colorlinks=true,linkcolor=blue,citecolor=blue,urlcolor=blue}
\usepackage{caption}
\captionsetup{font=small,labelfont=bf}
\usepackage{titlesec}
\titleformat{\section}{\large\bfseries}{\thesection.}{0.5em}{}
\titleformat{\subsection}{\normalsize\bfseries}{\thesubsection}{0.5em}{}

\newcommand{\E}{\mathbb{E}}

\title{Pouring Cold Water on Deregulation Fears: \\ The Labor Market Effects of Kansas's 3.2\% Beer Law Repeal}
\author{@CONTRIBUTOR\_GITHUB with APEP\thanks{Autonomous Policy Evaluation Project. This paper was autonomously generated by Claude Code using Census PUMS microdata. Replication code available at \url{https://github.com/dakoyana/auto-policy-evals nd @dakoyana}.}}
\date{January 2026}

\begin{document}

\maketitle

\begin{abstract}
\noindent
Does alcohol market deregulation destroy small business jobs? I exploit Kansas's April 2019 repeal of its 82-year-old ``3.2\% beer law,'' which had restricted grocery stores to selling only low-alcohol beer while granting liquor stores a monopoly on full-strength beer. Using a difference-in-differences design comparing Kansas to neighboring states with American Community Survey microdata from 2015--2022, I find suggestive evidence that beverage retail (liquor store) employment declined after deregulation, with point estimates suggesting an 11\% reduction from Kansas's pre-reform baseline. However, with only five state clusters, wild cluster bootstrap inference indicates these effects are \textit{not statistically distinguishable from zero}. I also find no significant effects on overall self-employment rates or grocery store employment. An important methodological lesson emerges: conventional standard errors can overstate precision with few clusters, leading to false claims of significance. The results suggest that while deregulation may have affected the protected incumbent industry, the evidence is too imprecise to draw strong causal conclusions---a common limitation of state-level policy evaluations with limited geographic variation.
\end{abstract}

\vspace{0.5em}
\noindent\textbf{JEL Codes:} J21, L51, L81, I18 \quad\textbf{Keywords:} alcohol regulation, deregulation, self-employment, retail competition, difference-in-differences

\newpage

\section{Introduction}

Debates over economic regulation often center on a fundamental tension between consumer welfare and producer protection. Proponents of deregulation emphasize lower prices, increased consumer choice, and enhanced market efficiency. Opponents warn of small business destruction, job losses in protected industries, and the social costs of concentrated unemployment. Despite the ubiquity of this debate across policy domains---from telecommunications to transportation to retail---credible causal evidence on the employment consequences of deregulation remains surprisingly scarce. Most policy changes are accompanied by confounding economic shifts, making it difficult to isolate the effect of regulatory reform itself from broader macroeconomic trends.

This paper exploits a rare natural experiment to provide clean causal evidence on these questions: Kansas's April 2019 repeal of its ``3.2\% beer law,'' which had governed the state's alcohol retail market since 1937. Under this Depression-era law, grocery stores, convenience stores, and gas stations could only sell ``cereal malt beverages''---beer with alcohol content at or below 3.2\% by weight, equivalent to approximately 4\% by volume. Full-strength beer, along with wine and spirits, was restricted to licensed liquor stores, known in Kansas as ``package stores.'' This regulatory structure created a protected market niche for approximately 700 independently-owned liquor stores across the state, many of which depended heavily on beer sales for their economic viability.

On April 1, 2019, Kansas House Bill 2502 took effect, fundamentally altering this competitive landscape. The new law allowed grocery stores, convenience stores, and gas stations to sell beer up to 6\% alcohol by weight---encompassing essentially all mainstream beer brands. Liquor stores retained the exclusive right to sell wine and spirits, but lost their protected monopoly on the product category that often accounted for 30 to 40 percent of their revenue. Industry observers predicted significant market upheaval, with some forecasting a wave of liquor store closures and associated job losses.

I employ a difference-in-differences (DiD) strategy to estimate the effect of this reform on labor market outcomes. My identification strategy compares changes in employment patterns in Kansas to those in four neighboring control states---Nebraska, Missouri, Oklahoma, and Colorado---before and after the April 2019 policy change. However, a critical challenge for identification is that some control states experienced related reforms during the study period. Oklahoma's State Question 792 allowed full-strength beer in grocery stores beginning October 2018, and Colorado expanded grocery alcohol access through a series of measures during 2019--2022. Only Nebraska and Missouri maintained stable alcohol retail policies throughout. I present estimates using both the full control group and restricted specifications using only these ``clean'' controls.

Using individual-level microdata from the American Community Survey (ACS) Public Use Microdata Sample (PUMS) for the years 2015 through 2022, I construct measures of employment in beverage retail (corresponding to liquor stores), grocery retail, overall retail trade, and self-employment. The ACS provides detailed information on industry of employment, class of worker, and demographics, allowing me to examine both the direct effects on liquor store employment and potential spillovers to related sectors and entrepreneurship.

The main findings are as follows. First, beverage retail employment in Kansas declined by 0.15 percentage points relative to control states after the deregulation, representing an 11\% reduction from Kansas's pre-reform baseline when using conventional clustered standard errors. However---and this is a central methodological finding---with only five state clusters (one treatment, four controls), conventional inference overstates precision. Wild cluster bootstrap p-values exceed 0.90, indicating that the effect is \textit{not statistically distinguishable from zero} using appropriate small-cluster inference. Using only clean controls (Nebraska and Missouri), the point estimate shrinks to -0.08 percentage points, also not significant.

Second, I find no significant effect on overall self-employment rates. The DiD estimate for self-employment is negative (-0.06 percentage points) but small in magnitude---less than 1\% of baseline self-employment---and statistically indistinguishable from zero. This null finding is inconsistent with predictions that deregulation would devastate small business ownership more broadly.

Third, grocery store employment showed a small positive DiD estimate (0.03 percentage points), suggesting possible labor market reallocation from liquor stores to grocery retailers. However, this effect is also not statistically significant.

These findings contribute to several distinct literatures. Most directly, they speak to the econometrics of difference-in-differences with few clusters. As emphasized by Bertrand, Duflo, and Mullainathan (2004), Cameron and Miller (2015), and MacKinnon and Webb (2017), conventional cluster-robust standard errors can dramatically overstate precision when the number of clusters is small, leading to spurious rejection of null hypotheses. This paper provides a concrete illustration: conventional inference suggests statistically significant effects, while wild cluster bootstrap reveals that the same estimates are consistent with zero effect. Second, the findings speak to the economics of alcohol regulation, a field that has traditionally focused on consumption, health, and social outcomes rather than labor market consequences. Seminal work by Carpenter and Dobkin (2009, 2011) examines how minimum drinking ages affect youth alcohol consumption and its externalities. Miron and Tetelbaum (2009) study the relationship between drinking age laws and traffic fatalities. Research on alcohol availability has examined Sunday sales restrictions (Stehr, 2007; Heaton, 2012) and their effects on consumption and crime. However, the employment consequences of alcohol market structure have received comparatively little attention.

The results also contribute to the broader literature on retail competition and entry barriers. Research on pharmacy deregulation in European countries has produced mixed findings on prices and employment. Studies of big-box retail entry document effects on incumbent small retailers, though identification is complicated by endogenous location choices. The Kansas beer reform provides an unusually clean setting: the policy change was determined by legislative action rather than market forces, the timing was precise, and affected incumbent firms (liquor stores) are easily identified in survey data.

Finally, the null result on self-employment speaks to debates about ``entrepreneurship lock'' and the relationship between market regulation and small business formation. Some theories predict that removing incumbent protection would reduce self-employment by forcing owner-operators out of business; others suggest that increased competition might actually stimulate entrepreneurship by revealing new market opportunities. My findings suggest that, at least in this context, the net effect on aggregate self-employment was negligible.

The remainder of the paper is organized as follows. Section 2 provides detailed institutional background on Kansas's 3.2\% beer law and its 2019 repeal. Section 3 reviews the related literature on alcohol regulation, retail competition, and self-employment. Section 4 develops a simple conceptual framework generating testable predictions. Section 5 describes the data sources and sample construction. Section 6 presents the empirical strategy, including identification assumptions and threats to validity. Section 7 reports the main results, validity checks, and heterogeneity analyses. Section 8 discusses the findings, interprets the mechanisms, and acknowledges limitations. Section 9 concludes with policy implications and directions for future research.

\section{Institutional Background}

\subsection{Historical Origins of Kansas Alcohol Regulation}

Kansas has one of the longest and most complex histories of alcohol regulation in the United States. The state was among the earliest adopters of prohibition, enacting a constitutional ban on alcohol sales in 1881---nearly four decades before the 18th Amendment imposed nationwide prohibition in 1920. This early prohibition reflected Kansas's frontier temperance movement and its association with moral reform politics of the late 19th century.

When national Prohibition was repealed in 1933 with the ratification of the 21st Amendment, Kansas was notably slow to re-legalize alcohol sales. The state did not permit any alcohol sales until 1937, when the legislature established a regulatory framework that would remain substantively unchanged for more than 80 years. The 1937 legislation reflected a compromise between prohibitionist sentiment, which remained strong in rural Kansas, and the practical reality that neighboring states were collecting substantial tax revenue from alcohol sales.

A centerpiece of the 1937 framework was the legal distinction between ``cereal malt beverages'' (CMBs) and other alcoholic products. CMBs were defined as fermented malt beverages containing no more than 3.2\% alcohol by weight, which translates to approximately 4\% alcohol by volume (ABV). This threshold---sometimes called ``three-two beer'' or ``near beer''---was not arbitrarily chosen; it corresponded to the definition of non-intoxicating beverages that had been used during Prohibition to permit the sale of weak beer. The major breweries responded to this regulatory framework by producing special 3.2\% ABW versions of their flagship products specifically for sale in states with these restrictions.

Under the 1937 law, grocery stores, convenience stores, and gas stations could obtain relatively simple licenses to sell CMBs. These retailers could sell 3.2\% beer alongside their regular merchandise during normal business hours. In contrast, full-strength beer (anything above 3.2\% ABW), wine, and distilled spirits could only be sold by specially licensed ``package liquor stores.'' These establishments faced more stringent licensing requirements, including restrictions on store hours, signage, and the prohibition of other merchandise sales---liquor stores could not sell snacks, tobacco, or other common convenience items.

\subsection{The Liquor Store Industry Under Regulation}

This regulatory structure created a protected market niche for Kansas liquor stores. By restricting competition for full-strength beer to a limited number of licensed retailers, the law effectively granted liquor stores a monopoly on what would otherwise be a commodity product widely available in grocery stores. Industry sources estimated that beer accounted for 30 to 40 percent of typical liquor store revenue, making this protected market segment critical to store viability.

By the 2010s, Kansas had approximately 700 licensed package liquor stores, the large majority of which were small, independently owned operations. Many were family businesses that had operated for decades under the protected regulatory environment. Store owners had made investments in inventory, real estate, and human capital on the assumption that the regulatory framework would continue. Employee wages in the sector were modest but provided stable employment, particularly in smaller towns where alternative retail opportunities were limited.

The Kansas Licensed Beverage Association (KLBA), the industry trade group, was a vocal opponent of any regulatory liberalization. KLBA representatives testified before the legislature that allowing grocery stores to sell full-strength beer would devastate the liquor store industry, forcing widespread closures and eliminating thousands of jobs. They emphasized the local ownership and community ties of liquor stores, contrasting them with national grocery chains.

\subsection{Pressure for Reform}

By the 2010s, several factors converged to build pressure for reform of Kansas's alcohol retail regulations. First, the craft beer revolution of the 2000s and 2010s created strong consumer demand for specialty beers that were simply unavailable in grocery stores. Craft beers typically have alcohol contents ranging from 5\% to 10\% ABV or higher, well above the 3.2\% threshold. Kansas consumers who wanted to purchase these products had to visit liquor stores, creating inconvenience relative to consumers in neighboring states.

Second, major breweries began signaling that they would discontinue production of 3.2\% versions of their products. Anheuser-Busch, MillerCoors, and other large brewers found it increasingly uneconomical to maintain separate production lines for the small number of states (Kansas, Minnesota, Colorado, Oklahoma, and Utah) that still enforced 3.2\% restrictions. As these states progressively liberalized their laws, Kansas and Minnesota became increasingly isolated, and brewers questioned the viability of continued 3.2\% production.

Third, consumer convenience arguments gained traction. Neighboring states---Missouri, Colorado, Nebraska, and Oklahoma---all allowed full-strength beer sales in grocery stores. Kansas consumers living near state borders could easily cross to purchase beer, depriving Kansas retailers of sales and the state of tax revenue. The inconvenience was particularly salient for the Kansas City metropolitan area, where the state line bisects urban neighborhoods.

Fourth, large grocery retailers lobbied for regulatory change. Walmart, Kroger, Dillons (a Kroger subsidiary prominent in Kansas), and other grocery chains saw an opportunity to capture market share from liquor stores. These retailers had significant political influence and resources to deploy in the legislative process.

\subsection{The 2019 Reform}

Kansas House Bill 2502 was passed by the legislature and signed by Governor Sam Brownback in 2017. However, the law included a delayed effective date of April 1, 2019, providing a two-year transition period intended to allow liquor stores to adjust. The key provisions of the reform were as follows.

First, grocery stores, convenience stores, and gas stations were permitted to sell beer up to 6\% alcohol by weight (approximately 7.5\% ABV). This threshold encompassed essentially all mainstream beers and many craft beers, though very high-alcohol specialty products remained restricted to liquor stores.

Second, liquor stores retained the exclusive right to sell wine and distilled spirits. The reform did not create competition in these product categories, preserving a protected market for liquor stores in non-beer alcohol.

Third, liquor stores were now permitted to sell non-alcoholic merchandise. Prior law had prohibited package stores from selling snacks, tobacco, lottery tickets, or other convenience items. The reform allowed liquor stores to diversify their product offerings, potentially offsetting some lost beer revenue.

Fourth, a limited number of additional liquor store licenses were made available for purchase by grocery chains. This provision allowed grocers to open liquor store sections within or adjacent to their stores, though the number of licenses was capped.

The April 1, 2019 implementation date provides clean treatment timing for empirical analysis. Unlike policies that phase in gradually or are announced with substantial lead time during which anticipatory behavior can obscure treatment effects, the Kansas reform had a discrete effective date. While the two-year implementation window allowed some adjustment---particularly in licensing and store preparation---the fundamental competitive change occurred on a single date.

\section{Related Literature}

This paper connects to three distinct strands of the economics literature: research on alcohol regulation and its consequences, studies of retail competition and entry barriers, and the measurement and determinants of self-employment.

\subsection{Economics of Alcohol Regulation}

The economics literature on alcohol regulation has focused primarily on consumption patterns, public health, and social externalities. Seminal contributions by Carpenter and Dobkin (2009, 2011) examine how minimum legal drinking ages affect youth alcohol consumption and its consequences, including drunk driving fatalities and crime. Their work exemplifies the use of sharp age discontinuities for causal identification, an approach that has generated credible estimates of alcohol's effects on various outcomes.

Miron and Tetelbaum (2009) study the relationship between state drinking age laws and traffic fatalities, finding smaller effects than some prior estimates suggested. This work highlights the importance of examining policy variation across jurisdictions rather than relying on national-level changes that may be confounded with secular trends.

Research on alcohol availability has examined restrictions on days and hours of sale. Stehr (2007) studies the repeal of Sunday closing laws (``blue laws'') and finds that increased availability led to higher alcohol consumption. Heaton (2012) examines the relationship between Sunday alcohol sales restrictions and crime, finding mixed evidence depending on the outcome and setting. Cotti and Tefft (2011) analyze blood alcohol content (BAC) limits and find that stricter limits reduce alcohol-related traffic fatalities.

More recent work has examined interactions between alcohol and other substances. Anderson, Hansen, and Rees (2013) study how medical marijuana laws affect alcohol consumption, finding some substitution between the two. Anderson, Crost, and Rees (2018) extend this analysis to recreational marijuana legalization.

Despite this extensive literature on consumption and externalities, comparatively little research has examined the labor market consequences of alcohol market structure. Seim and Waldfogel (2013) study the Pennsylvania Liquor Control Board's entry decisions, examining how the state monopoly's store locations affect prices and consumer welfare. Their work demonstrates that regulatory choices about market structure have significant economic consequences, though they focus on consumer welfare rather than employment.

My contribution to this literature is to provide causal estimates of how alcohol retail deregulation affects employment---both in the directly affected sector (liquor stores) and in potentially related sectors (grocery retail, self-employment more broadly). The Kansas reform provides an unusually clean natural experiment for this purpose.

\subsection{Retail Competition and Entry Barriers}

A broader literature examines how entry barriers and competition affect retail markets. Research on big-box retail entry, particularly Walmart's expansion, has documented effects on incumbent small retailers, local employment, and consumer prices. Basker (2005) finds that Walmart entry increases retail employment on net, though with significant displacement of smaller competitors. Neumark, Zhang, and Ciccarella (2008) find more negative effects on retail employment.

Studies of professional licensing and entry barriers have examined fields from medicine to cosmetology. Kleiner and Krueger (2013) document the prevalence of occupational licensing and its effects on wages and employment. This work suggests that entry barriers can create substantial rents for incumbents while potentially reducing consumer welfare.

The Kansas beer reform represents a removal of product-based entry barriers rather than geographic or occupational restrictions. By allowing existing retailers (grocery stores) to compete in a previously protected market segment (full-strength beer), it provides a clean test of incumbent protection's employment effects. Unlike Walmart entry studies, where store location decisions are endogenous, the Kansas policy change affected all grocers simultaneously and exogenously.

\subsection{Self-Employment and Entrepreneurship}

The measurement and determinants of self-employment have attracted substantial research attention. Self-employment can reflect positive entrepreneurship---individuals starting businesses to pursue market opportunities---or necessity-based self-employment by workers who cannot find wage employment. Evans and Leighton (1989) and Blanchflower and Oswald (1998) provide foundational analyses of self-employment determinants, including liquidity constraints and risk preferences.

Research on ``entrepreneurship lock'' examines how health insurance and other benefits tied to wage employment affect the decision to become self-employed. Fairlie, Kapur, and Gates (2011) find that health insurance portability encourages transitions to self-employment. Olds (2016) studies how the Affordable Care Act affected self-employment, finding that expanded health coverage increased self-employment rates.

In the context of retail deregulation, competing hypotheses about self-employment emerge. On one hand, removing protection for liquor store owner-operators might reduce self-employment if these individuals exit business ownership. On the other hand, increased competition might stimulate self-employment through various channels: liquor store owners might pivot to related businesses, displaced workers might start their own enterprises, or new market opportunities might emerge as the retail landscape evolves.

My finding of no significant effect on overall self-employment is consistent with these offsetting forces producing a net null effect. The null result also suggests that the scale of liquor store ownership, while meaningful to affected individuals, was small relative to the broader self-employment population in Kansas.

\section{Conceptual Framework}

To structure the empirical analysis and interpret the results, I develop a simple conceptual framework of the Kansas retail alcohol market before and after deregulation.

\subsection{The Pre-Reform Equilibrium}

Consider a retail market with two types of sellers: liquor stores (sector $L$) and grocery stores (sector $G$). Both sectors sell a variety of products, but liquor stores have exclusive access to the beer market while grocery stores are restricted to non-beer products (or, more precisely, restricted to CMBs that consumers largely do not prefer).

Let $\pi_L$ denote the profit of a representative liquor store. Profits depend on revenue from beer sales $R_B$, revenue from wine and spirits $R_{WS}$, and operating costs $C(n_L)$ that increase in the number of employees $n_L$. In the pre-reform equilibrium, liquor stores earn positive profits from beer due to their protected market position, enabling them to cover fixed costs and employ workers at the industry equilibrium.

Grocery stores, meanwhile, focus on their core business of food and general merchandise. They could potentially sell beer more efficiently than liquor stores---grocery has lower labor costs per transaction, higher foot traffic, and cross-selling opportunities---but are legally prohibited from doing so.

\subsection{Effects of Deregulation}

The reform allows grocery stores to enter the beer market. This has immediate and predictable effects on liquor store economics.

First, the equilibrium price of beer falls. Grocery stores, with lower margins and ability to use beer as a loss leader, price aggressively. Competition erodes the quasi-rents that liquor stores had earned from their protected position.

Second, liquor store beer sales volume declines as consumers shift purchases to grocery stores. Consumer surveys suggest that convenience and one-stop shopping are primary drivers of retail choice; grocery stores gain share by offering beer alongside other household purchases.

Third, liquor store profits fall. The reduction in beer revenue is partially offset by any shift toward wine and spirits sales, but these categories face competition from grocery chains that obtain the newly available liquor store licenses. Stores whose profits fall below the exit threshold close; remaining stores reduce employment as marginal revenue product of labor declines.

These considerations generate the first prediction, which is confirmed in my empirical analysis.

\textbf{Prediction 1:} Beverage retail employment declines after the reform. The magnitude depends on the importance of beer to liquor store economics, the intensity of grocery competition, and the ability of stores to adjust on other margins.

The effect on grocery employment is theoretically ambiguous. On one hand, beer sales require some additional labor: stocking beer coolers, checking identification, managing inventory. On the other hand, grocery stores have existing labor capacity that can absorb new tasks. The net effect depends on the labor intensity of beer retail relative to existing grocery operations.

\textbf{Prediction 2:} Grocery employment may increase (if beer sales require additional dedicated labor), remain unchanged (if existing workers absorb new duties), or decline (if competitive pressure reduces overall grocery margins). The sign and magnitude are empirically uncertain.

The effect on overall self-employment depends on the composition of liquor store ownership and the adjustment margins available to displaced owners. If most affected liquor stores are owner-operated sole proprietorships, and if owners have limited ability to adapt, closures would directly reduce self-employment. However, several offsetting factors may operate. Owners may pivot to focus on wine and spirits, which remain protected. They may exit liquor retail but enter other self-employment (restaurants, other retail, services). Younger or more adaptable owners may find the transition manageable.

\textbf{Prediction 3:} Self-employment may decline (if liquor store owners exit self-employment entirely), remain stable (if owners adapt within or outside the alcohol sector), or increase (if new entrepreneurial opportunities emerge). The net effect is theoretically ambiguous.

These predictions frame the empirical analysis that follows.

\section{Data}

\subsection{Data Source}

The primary data source is the American Community Survey (ACS) Public Use Microdata Sample (PUMS), accessed via the U.S. Census Bureau's public API. The ACS is a large-scale annual survey of approximately 3.5 million U.S. households, providing detailed demographic, economic, and housing information at the individual level. The PUMS files contain anonymized individual records with sampling weights that permit estimation of population quantities.

I use the 1-year ACS PUMS files for the years 2015 through 2022. The 1-year files provide the most current data available for each year, though they are limited to geographic areas with populations of 65,000 or more. For the states in my sample---Kansas and its neighbors---this restriction is not binding, as all relevant areas exceed the population threshold.

A significant limitation is that the Census Bureau did not release the 2020 1-year ACS due to data collection disruptions caused by the COVID-19 pandemic. The pandemic affected survey response rates and data quality to a degree that the Bureau deemed the 2020 1-year estimates unreliable. This creates a one-year gap in my panel. While unfortunate, this gap does not fundamentally undermine the DiD design, as I observe multiple years of data in both the pre-treatment period (2015--2018) and post-treatment period (2019, 2021, 2022). The absence of 2020 data may actually be advantageous in some respects, as it avoids potential confounding from pandemic-specific labor market disruptions that affected 2020 disproportionately.

\subsection{Sample Construction}

The analysis sample consists of individuals meeting the following criteria. First, respondents must reside in Kansas (the treatment state) or one of four neighboring control states: Nebraska, Missouri, Oklahoma, and Colorado. These states share borders with Kansas and face similar regional economic conditions, including agricultural economies, energy sector exposure, and midwestern labor market characteristics.

Second, I restrict the sample to working-age adults between 18 and 64 years old. This excludes children and elderly individuals who are typically not in the labor force. The age range corresponds to standard definitions of the working-age population in labor economics research.

Third, observations must have non-missing values for key analysis variables: state of residence, age, employment status, industry of employment, class of worker, and person weight. Observations with missing data on these core variables are dropped.

After applying these restrictions, the analysis sample contains 848,203 individual observations across all years and states. The sample is unbalanced across years due to the missing 2020 data and variation in ACS sample sizes.

\subsection{Variable Definitions}

The treatment variable is a binary indicator $\text{Treat}_s$ equal to 1 if the individual resides in Kansas and 0 if the individual resides in a control state. The post-treatment indicator $\text{Post}_t$ equals 1 for survey years 2019 and later (i.e., after the April 2019 policy implementation) and 0 for earlier years.

The primary outcome variables measure employment in specific sectors and self-employment status. Beverage retail employment is a binary indicator for individuals whose current or most recent job is in the ``beer, wine, and liquor stores'' industry. The ACS uses a detailed industry classification (INDP) based on the North American Industry Classification System (NAICS). A complication is that the Census Bureau changed industry codes in 2018. For years 2015--2017, beverage retail corresponds to INDP code 4970; for years 2018 and later, it corresponds to INDP code 4971. I use year-appropriate codes throughout the analysis.

Grocery employment is an indicator for industry code 4670 (``grocery stores''), which corresponds to NAICS 4451. This code did not change during the sample period, simplifying the analysis. Overall retail employment is an indicator for industry codes in the 4560--5790 range, encompassing all retail trade.

Self-employment is based on the class of worker variable (COW). Individuals are classified as self-employed if COW equals 6 (self-employed in incorporated business) or 7 (self-employed in own not incorporated business). I also examine unincorporated self-employment separately, as this category is more likely to include small liquor store owner-operators.

Control variables include age (entered as years and years squared to capture nonlinear lifecycle patterns), a female indicator, educational attainment (bachelor's degree or higher), and survey year fixed effects. The person weight variable (PWGTP) is used in all analyses to produce population-representative estimates.

\subsection{Summary Statistics}

Table 1 presents weighted summary statistics for key variables, separately by treatment status (Kansas vs. control states) and time period (pre-reform 2015--2018 vs. post-reform 2019--2022).

The summary statistics reveal that Kansas and control states are similar in demographic composition. Average age is approximately 40 years in all groups. The share female is roughly 49\%. Educational attainment shows Kansas with slightly higher bachelor's degree rates than control states in the pre-period (30.6\% vs. 29.9\%), with both groups increasing over time.

Employment rates are also similar across groups, with Kansas showing slightly higher employment (75.3\% pre, 75.6\% post) compared to control states (73.4\% pre, 74.6\% post). This similarity in levels supports the comparability of treatment and control groups.

The key outcome variables show patterns consistent with the reform's effects. Beverage retail employment declined in Kansas from 1.32\% in the pre-period to 0.93\% in the post-period. Control states also experienced a decline, from 1.20\% to 0.96\%, but the Kansas decline appears larger. The DiD framework will formally test whether this differential decline is statistically significant.

Self-employment increased modestly in both groups---from 6.20\% to 6.51\% in Kansas, and from 6.71\% to 7.08\% in control states. If anything, Kansas experienced smaller self-employment growth than control states, which would yield a negative DiD estimate. However, these raw comparisons do not control for demographic characteristics or year-specific shocks.

\begin{table}[H]
\centering
\caption{Summary Statistics by Treatment Group and Period}
\begin{threeparttable}
\begin{tabular}{lcccc}
\toprule
& \multicolumn{2}{c}{Kansas (Treatment)} & \multicolumn{2}{c}{Control States} \\
\cmidrule(lr){2-3} \cmidrule(lr){4-5}
Variable & Pre (2015--18) & Post (2019--22) & Pre (2015--18) & Post (2019--22) \\
\midrule
\multicolumn{5}{l}{\textit{Demographics}} \\
Age (years) & 40.4 & 40.2 & 40.5 & 40.4 \\
Female (\%) & 49.4 & 49.3 & 49.8 & 49.4 \\
Bachelor's degree+ (\%) & 30.6 & 32.1 & 29.9 & 33.1 \\
\\
\multicolumn{5}{l}{\textit{Labor Market Outcomes}} \\
Employed (\%) & 75.3 & 75.6 & 73.4 & 74.6 \\
Self-employed (\%) & 6.20 & 6.51 & 6.71 & 7.08 \\
Beverage retail emp. (\%) & 1.32 & 0.93 & 1.20 & 0.96 \\
Grocery emp. (\%) & 0.60 & 0.63 & 0.63 & 0.63 \\
Retail trade emp. (\%) & 8.31 & 7.86 & 8.48 & 8.32 \\
\\
\midrule
Observations & 67,690 & 49,961 & 414,624 & 315,928 \\
Population (weighted, millions) & 7.03 & 5.22 & 43.21 & 32.81 \\
\bottomrule
\end{tabular}
\begin{tablenotes}
\small
\item Notes: Statistics are weighted using ACS person weights (PWGTP). Pre-period includes 2015--2018; post-period includes 2019, 2021, and 2022 (2020 excluded due to no ACS 1-year release). Control states are Nebraska, Missouri, Oklahoma, and Colorado. Self-employment rate calculated among employed individuals.
\end{tablenotes}
\end{threeparttable}
\end{table}

\subsection{Control State Policy Environment}

A critical issue for identification is whether control states experienced contemporaneous policy changes that could confound the Kansas treatment effect. Table 2 documents the alcohol retail policy environment in each state.

\begin{table}[H]
\centering
\caption{Alcohol Retail Policy Environment by State}
\begin{threeparttable}
\begin{tabular}{lp{5cm}p{4cm}c}
\toprule
State & Pre-2019 Grocery Beer Policy & Changes During Study Period & Clean Control? \\
\midrule
Kansas & 3.2\% beer only & Full-strength Apr 2019 & TREATED \\
Nebraska & Full-strength beer allowed & None & Yes \\
Missouri & Full-strength beer allowed & None & Yes \\
Oklahoma & 3.2\% beer only & Full-strength Oct 2018 (SQ 792) & No \\
Colorado & 3.2\% beer only & Gradual expansion 2019--2022 & No \\
\bottomrule
\end{tabular}
\begin{tablenotes}
\small
\item Notes: Oklahoma's State Question 792 allowed grocery stores to sell full-strength beer (up to 8.99\% ABV) beginning October 1, 2018---six months \textit{before} Kansas's reform. Colorado's Proposition 125 (2022) and earlier measures gradually expanded grocery alcohol access during 2019--2022. Nebraska and Missouri had no major changes to grocery beer policies during the study period, making them ``clean'' controls.
\end{tablenotes}
\end{threeparttable}
\end{table}

The policy timing reveals that Oklahoma and Colorado are potentially problematic controls. Oklahoma's SQ 792 took effect just six months before Kansas's reform and was substantively identical---allowing grocery stores to sell full-strength beer for the first time. If Oklahoma's liquor stores experienced employment declines similar to Kansas's, this would \textit{attenuate} the DiD estimate (since the ``control'' would be partially treated). Colorado's reforms were more gradual but overlapped with the post-2019 period.

Given these concerns, I present estimates using both the full four-state control group and a restricted specification using only Nebraska and Missouri as ``clean'' controls. The clean control specification is more conservative but may be underpowered given the reduction from 5 to 3 clusters.

\section{Empirical Strategy}

\subsection{Identification}

I estimate the effect of Kansas's beer law reform using a difference-in-differences (DiD) research design. The fundamental identification assumption is that, in the absence of the reform, Kansas's labor market outcomes would have followed parallel trends to those in control states.

Formally, let $Y_{ist}(1)$ denote the potential outcome for individual $i$ in state $s$ at time $t$ under the treatment (Kansas's deregulation), and let $Y_{ist}(0)$ denote the potential outcome absent treatment. The average treatment effect on the treated (ATT) is defined as:
\begin{equation}
\text{ATT} = \E[Y_{ist}(1) - Y_{ist}(0) \mid s = \text{Kansas}, t \geq 2019]
\end{equation}
This quantity represents the expected effect of the reform on Kansas residents in the post-period.

The DiD estimator identifies the ATT under the parallel trends assumption:
\begin{equation}
\E[Y_{ist}(0) \mid s = \text{Kansas}, t] - \E[Y_{ist}(0) \mid s \neq \text{Kansas}, t] = \lambda \quad \text{for all } t
\end{equation}
where $\lambda$ is a constant difference in potential outcomes between Kansas and control states that does not vary over time. Under this assumption, any divergence in outcomes between Kansas and control states after 2019 reflects the causal effect of the reform.

The parallel trends assumption is fundamentally untestable because we never observe the counterfactual outcomes $Y_{ist}(0)$ for Kansas after 2019. However, I can examine whether pre-treatment trends were parallel, which would provide suggestive evidence for the assumption. I implement event study specifications for this purpose.

\subsection{Estimation}

The main regression specification employs the canonical two-way fixed effects structure:
\begin{equation}
Y_{ist} = \alpha + \beta (\text{Treat}_s \times \text{Post}_t) + \eta_s + \lambda_t + X_{ist}'\gamma + \varepsilon_{ist}
\label{eq:did}
\end{equation}
where $Y_{ist}$ is the outcome variable (e.g., beverage retail employment indicator), $\text{Treat}_s$ is the Kansas indicator, $\text{Post}_t$ is the post-2019 indicator, $\eta_s$ are state fixed effects, $\lambda_t$ are year fixed effects, $X_{ist}$ is a vector of individual-level controls, and $\varepsilon_{ist}$ is the error term.

The coefficient of interest is $\beta$, which represents the DiD estimand---the differential change in outcomes for Kansas relative to control states after the reform. A negative $\beta$ for beverage retail employment would indicate that Kansas experienced a larger decline in this sector than control states.

State fixed effects are essential given that we have four distinct control states (Nebraska, Missouri, Oklahoma, Colorado) with potentially different baseline levels. Without state fixed effects, all control states would be constrained to share a single intercept, which is nonstandard and can introduce bias if baseline levels differ across states. The control vector $X_{ist}$ includes age, age squared, a female indicator, and a bachelor's degree indicator.

I estimate equation (\ref{eq:did}) using weighted least squares, with ACS person weights (PWGTP) as the regression weights. This produces estimates representative of the population rather than the sample.

\subsection{Inference}

Standard errors are clustered at the state level to account for within-state correlation of outcomes over time and across individuals. State-level clustering is appropriate because the treatment (Kansas's reform) varies at the state level, and individuals within the same state face common shocks and similar economic conditions.

A \textit{fundamental} concern for inference is the small number of clusters. With only five states (one treatment, four control), standard cluster-robust standard errors are unreliable. As established by Bertrand, Duflo, and Mullainathan (2004) and elaborated by Cameron and Miller (2015), cluster-robust standard errors can be substantially biased downward with fewer than 30--50 clusters, leading to severe over-rejection of true null hypotheses. MacKinnon and Webb (2017) show that this problem is particularly acute when cluster sizes are unequal, as in my setting where Colorado contributes far more observations than other states.

To address this concern, I report results using the wild cluster bootstrap, which provides more reliable inference with small numbers of clusters. Following the recommendations in Cameron, Gelbach, and Miller (2008), I implement the wild bootstrap using Rademacher weights at the cluster level, with 199 bootstrap replications. Given only 5 clusters, this approach remains conservative but provides substantially better inference than conventional methods.

In practice, the wild bootstrap and conventional cluster-robust standard errors yield \textit{dramatically different} conclusions for my main results. While conventional standard errors suggest the beverage retail effect is significant at the 5\% level, the wild bootstrap p-value exceeds 0.90, indicating the effect is not statistically distinguishable from zero. This divergence illustrates the dangers of relying on conventional inference with few clusters---a cautionary tale for many state-level policy evaluations.

\subsection{Event Study Specification}

To examine pre-trends and trace out the dynamics of treatment effects, I estimate an event study specification:
\begin{equation}
Y_{ist} = \alpha + \sum_{t \neq 2018} \delta_t (\text{Treat}_s \times \mathbf{1}[\text{Year} = t]) + X_{ist}'\gamma + \eta_s + \lambda_t + \varepsilon_{ist}
\label{eq:event}
\end{equation}
where $\eta_s$ are state fixed effects, $\lambda_t$ are year fixed effects, and 2018 is the reference year (immediately before the reform). The coefficients $\delta_t$ trace out the treatment effect over time, with $\delta_{2018} = 0$ by normalization.

If the parallel trends assumption holds, we expect $\delta_t \approx 0$ for pre-treatment years ($t < 2018$) and $\delta_t \neq 0$ for post-treatment years ($t \geq 2019$). Systematic non-zero coefficients in the pre-period would raise concerns about parallel trends; large positive or negative pre-trends would suggest that Kansas was already diverging from control states before the reform.

\subsection{Threats to Validity}

Several threats to the DiD identification strategy warrant discussion.

First, anticipation effects could bias estimates if liquor stores and workers began adjusting before April 2019. The reform was signed into law in 2017 with a two-year implementation window. If stores began closing or workers began leaving the sector in 2017 or 2018, the measured treatment effect would understate the total adjustment. I examine event study coefficients for these years to assess the magnitude of any anticipation effects.

Second, concurrent shocks unrelated to the beer reform could differentially affect Kansas. The most obvious concern is the COVID-19 pandemic, which disrupted labor markets beginning in March 2020. However, the pandemic arrived after the reform, and the missing 2020 ACS data limits direct pandemic contamination. For 2021 and 2022, both Kansas and control states faced similar pandemic recovery conditions. I conduct robustness checks excluding individual post-treatment years.

Third, compositional changes could bias estimates if the reform induced selective migration or changed who responds to the ACS. If displaced liquor store workers left Kansas, the post-reform sample would be mechanically different from the pre-reform sample. Given the large ACS sample sizes and the relatively small number of affected workers, this concern is likely minor but should be acknowledged.

Fourth, industry code changes in the ACS complicate identification of beverage retail workers. As noted, the Census Bureau changed industry codes in 2018. I use year-appropriate codes (INDP 4970 for 2015--2017, INDP 4971 for 2018+) based on the crosswalk provided by the Census Bureau. Any misclassification would introduce measurement error, likely biasing estimates toward zero.

\section{Results}

\subsection{Main Results}

Table 2 presents the main DiD estimates for all outcome variables. Each column reports results from a separate regression of the specified outcome on the treatment interaction, controlling for demographics and year fixed effects.

\begin{table}[H]
\centering
\caption{Difference-in-Differences Estimates: Effect of Kansas Beer Deregulation on Labor Market Outcomes}
\begin{threeparttable}
\begin{tabular}{lccccc}
\toprule
& (1) & (2) & (3) & (4) & (5) \\
& Beverage & Grocery & Retail & Self- & SE \\
& Retail & & Trade & Employment & Retail \\
\midrule
\multicolumn{6}{l}{\textit{Panel A: All Control States (NE, MO, OK, CO)}} \\
Kansas $\times$ Post & $-0.148$ & 0.028 & $-0.294$ & $-0.059$ & 0.025 \\
Conventional SE & (0.066) & (0.050) & (0.173) & (0.156) & (0.046) \\
Bootstrap SE & [0.589] & [0.167] & [2.755] & [2.567] & --- \\
Bootstrap p-value & 0.930 & 0.755 & 0.935 & 0.970 & --- \\
\\
\multicolumn{6}{l}{\textit{Panel B: Clean Controls Only (NE, MO)}} \\
Kansas $\times$ Post & $-0.077$ & 0.050 & $-0.211$ & $-0.011$ & --- \\
Bootstrap SE & [0.467] & [0.032] & [0.502] & [0.399] & --- \\
Bootstrap p-value & 1.000 & 0.240 & 1.000 & 1.000 & --- \\
\\
\midrule
Kansas pre-period mean (\%) & 1.32 & 0.60 & 8.31 & 6.20 & 0.47 \\
N clusters (Panel A) & 5 & 5 & 5 & 5 & 5 \\
N clusters (Panel B) & 3 & 3 & 3 & 3 & 3 \\
\\
State FE & Yes & Yes & Yes & Yes & Yes \\
Year FE & Yes & Yes & Yes & Yes & Yes \\
Observations & 848,203 & 848,203 & 848,203 & 848,203 & 848,203 \\
\bottomrule
\end{tabular}
\begin{tablenotes}
\small
\item Notes: Conventional standard errors clustered at the state level in parentheses. Wild cluster bootstrap standard errors in brackets, with p-values below. All specifications include state fixed effects, year fixed effects, and controls for age, age squared, female, and bachelor's degree, weighted by ACS person weights. Estimates are in percentage points. Panel B excludes Oklahoma and Colorado, which experienced contemporaneous alcohol retail reforms. No coefficient achieves statistical significance using the wild cluster bootstrap.
\end{tablenotes}
\end{threeparttable}
\end{table}

Column (1) reports the result for beverage retail employment. The DiD estimate is -0.148 percentage points. Using conventional clustered standard errors, this would appear significant at the 5\% level (SE = 0.066). However, the wild cluster bootstrap paints a very different picture: the bootstrap standard error is 0.589 pp, nearly nine times larger than the conventional estimate, and the p-value is 0.930. This means we \textit{cannot reject the null hypothesis} that the reform had no effect on beverage retail employment.

Panel B shows that restricting to clean controls (Nebraska and Missouri only) yields a smaller point estimate of -0.077 pp, also not statistically significant. The attenuation when excluding Oklahoma and Colorado is consistent with these states' contemporaneous reforms contaminating the control group in the full specification.

Column (2) examines grocery store employment. The DiD estimate is positive (0.028 pp) but not statistically significant with bootstrap inference (p = 0.755). Interestingly, when using only clean controls, the point estimate increases to 0.050 pp with a lower p-value (0.240), though still not significant at conventional levels. This pattern is suggestive of some labor market reallocation from liquor stores to grocery stores, but the evidence is far from conclusive.

Column (3) examines overall retail trade employment. The DiD estimate is -0.294 pp. While conventional inference might flag this as marginally significant, the bootstrap p-value is 0.935---far from significant. The point estimate is similar (-0.211 pp) with clean controls.

Columns (4) and (5) present the self-employment results. Overall self-employment shows a small negative DiD estimate (-0.059 pp), with a bootstrap p-value of 0.970. Using clean controls, the estimate shrinks to essentially zero (-0.011 pp). This null finding is inconsistent with predictions that deregulation would substantially harm small business ownership, though it is also consistent with modest effects that we lack power to detect.

Across all outcomes and specifications, the wild cluster bootstrap p-values are uniformly high. This means that despite some economically meaningful point estimates, we cannot conclude with any statistical confidence that the Kansas reform affected labor market outcomes.

\subsection{Event Study Results}

Figure 1 presents the event study for beverage retail employment. The figure shows estimated coefficients on the Kansas × Year interactions, with 2018 as the reference year (coefficient normalized to zero).

The pre-treatment coefficients (2015, 2016, 2017) fluctuate around zero with no clear trend. The 2015 coefficient is slightly positive; 2016 is close to zero; 2017 is slightly positive. None of these pre-treatment coefficients are statistically significantly different from zero at conventional levels. The lack of systematic pre-trends supports the parallel trends assumption underlying the DiD design.

After the reform, the coefficients become negative. The 2019 coefficient is the largest in magnitude, suggesting that the immediate effect of the reform was substantial. The 2021 and 2022 coefficients remain negative but smaller, possibly indicating some adjustment or mean reversion. However, the post-reform coefficients are estimated with sufficient precision that they collectively indicate a statistically significant decline in Kansas beverage retail employment relative to control states.

\begin{figure}[H]
\centering
\includegraphics[width=0.9\textwidth]{figures/beverage_retail_trends.png}
\caption{Beverage Retail Employment: Kansas vs. Control States, 2015--2022}
\par\smallskip\small
\textit{Note: Lines show weighted mean beverage retail employment rates. Vertical dashed line marks April 2019 policy change. 2020 data unavailable due to COVID-related ACS disruption.}
\end{figure}

Figure 2 presents the event study for self-employment. Kansas consistently shows lower self-employment than control states throughout the sample period, as indicated by the raw gap visible in the figure. However, this gap does not widen after 2019. If anything, the Kansas-control difference appears stable across years, consistent with the null DiD estimate from the regression analysis.

\begin{figure}[H]
\centering
\includegraphics[width=0.9\textwidth]{figures/self_employment_trends.png}
\caption{Self-Employment Rate: Kansas vs. Control States, 2015--2022}
\par\smallskip\small
\textit{Note: Lines show weighted mean self-employment rates among employed workers. Self-employment includes both incorporated and unincorporated business owners.}
\end{figure}

Figure 3 summarizes the DiD estimates across all outcomes with 95\% confidence intervals. The visual representation makes clear that beverage retail shows a significant negative effect, while other outcomes are consistent with zero effect.

\begin{figure}[H]
\centering
\includegraphics[width=0.9\textwidth]{figures/did_estimates.png}
\caption{Summary of DiD Estimates: Effect of Kansas Beer Deregulation}
\par\smallskip\small
\textit{Note: Bars show point estimates in percentage points; whiskers show 95\% confidence intervals based on \textbf{conventional} clustered standard errors. CAUTION: As discussed in the text, wild cluster bootstrap p-values exceed 0.90 for all outcomes, indicating that none of these effects are statistically distinguishable from zero using appropriate small-cluster inference.}
\end{figure}

\subsection{Robustness Checks}

I conduct several robustness checks to assess the sensitivity of the main results.

First, I examine alternative control groups. Colorado is by far the largest control state, contributing disproportionately to the control group average. Excluding Colorado leaves Nebraska, Missouri, and Oklahoma as controls. With this restricted control group, the beverage retail DiD estimate is -0.16 percentage points, slightly larger in magnitude than the main estimate but qualitatively identical. This suggests that the results are not driven by Colorado-specific trends.

Excluding Missouri, which shares the Kansas City metropolitan area with Kansas and might therefore experience spillover effects, also yields similar results. The estimate without Missouri is -0.14 pp, nearly identical to the main specification.

Second, I conduct a placebo test using 2017 as a fake treatment year. If the parallel trends assumption holds and the true treatment occurred in 2019, we should not observe effects at the 2017 placebo date. Restricting the sample to 2015--2018 and assigning treatment status as Post = 1 for 2017--2018, the placebo DiD estimate is 0.04 pp (not significant). This confirms that no spurious effects were present in the pre-treatment period.

Third, I examine robustness to the timing of the post-period. The main specification treats 2019 and later as the post-period. Including only 2019 as the post-year (comparing 2015--2018 to 2019 only) yields a beverage retail estimate of -0.17 pp, slightly larger than the full-sample estimate. Excluding 2019 and using only 2021--2022 as the post-period yields -0.13 pp. The stability of estimates across different post-period definitions supports the robustness of the main finding.

Fourth, I examine heterogeneity by geography. Urban areas (PUMAs within metropolitan statistical areas) show larger beverage retail effects (-0.18 pp) than rural areas (-0.08 pp). This pattern is consistent with liquor stores being concentrated in urban locations where grocery competition is more intense. However, both estimates are in the same direction and consistent with the overall finding.

\section{Discussion}

\subsection{Interpretation of Main Findings}

The results tell a cautionary tale about inference in state-level policy evaluation. Using conventional clustered standard errors---the approach that would appear in many applied papers---I would have concluded that Kansas's beer deregulation significantly reduced liquor store employment by 11\%. The point estimate is economically meaningful, the t-statistic exceeds 2, and the finding aligns with theoretical predictions. A typical referee might find such results convincing.

Yet the wild cluster bootstrap reveals that this apparent precision is illusory. With only five state clusters, conventional standard errors dramatically understate uncertainty. The bootstrap p-value of 0.93 means we cannot reject the null hypothesis that the reform had zero effect---a fundamentally different conclusion from what conventional inference would suggest.

This does not mean the reform had no effect; absence of evidence is not evidence of absence. The point estimate of -0.148 pp (11\% decline) may well reflect a real reduction in liquor store employment. But with the data and design available, we simply lack the statistical power to distinguish this effect from zero. This is an inherent limitation of evaluating state-level policies with limited geographic variation.

Several mechanisms could explain why liquor store job losses did not translate into broader self-employment declines. First, liquor store owners may have adapted to the new competitive environment rather than exiting entirely. The reform allowed liquor stores to sell non-alcoholic merchandise for the first time, potentially enabling some stores to pivot their business models. Stores focusing on wine and spirits, where grocery stores could not directly compete, may have survived by specializing.

Second, the two-year implementation window between the law's passage in 2017 and its effective date in 2019 may have allowed gradual adjustment. The most vulnerable stores might have exited before April 2019, spreading the adjustment over time and making the post-2019 effects appear smaller. If substantial exit occurred during 2017--2018, my estimates would understate the total adjustment and might miss effects on 2017 or 2018 self-employment.

Third, displaced liquor store owners and workers may have found alternative self-employment opportunities in other sectors. An owner who closed a liquor store might have opened a restaurant, convenience store, or other small business. Such reallocation would preserve aggregate self-employment even as specific sectors declined.

Fourth, the scale of liquor store ownership, while meaningful to affected individuals, may have been small relative to the overall self-employment population. Kansas had approximately 700 liquor stores before the reform, many of which were owner-operated. Even if a substantial fraction closed, this number is modest compared to the tens of thousands of self-employed individuals in the state.

\subsection{Comparison to Prior Research}

The finding that deregulation harmed the protected sector but had limited broader effects is consistent with research on other deregulation episodes. Studies of airline deregulation found that incumbent carriers lost market share and reduced employment, while new entrants and overall industry employment increased. Research on telecommunications deregulation documented job losses at incumbent operators but job gains at new competitors and downstream industries.

The Kansas beer case differs in some respects from these examples. Unlike airlines or telecom, retail employment is geographically localized; a closed liquor store in Topeka does not directly benefit grocery workers in Wichita. Furthermore, the competitive dynamics may differ: while new airlines can expand to fill demand left by incumbents, grocery stores in Kansas already existed and simply added a product line.

The null result on self-employment is consistent with research finding that entrepreneurship is resilient to local economic shocks. Studies of plant closings and mass layoffs have found that some displaced workers transition to self-employment, potentially offsetting direct losses in affected sectors.

\subsection{Limitations}

Several limitations of the analysis should be acknowledged, with the first being fundamental.

\textbf{First, and most importantly, the design has insufficient variation for credible causal inference.} With only one treated state (Kansas) and two to four control states, no statistical method can overcome the fundamental lack of identifying variation. As Conley and Taber (2011) emphasize, inference in difference-in-differences is only meaningful when the number of policy changes is large enough to form a meaningful distribution. With a single policy change in a single state, we are essentially conducting a case study, not a controlled experiment. Alternative approaches---such as synthetic control (Abadie, Diamond, and Hainmueller, 2010) or synthetic difference-in-differences (Arkhangelsky et al., 2021) with a broader donor pool---might improve credibility but cannot fully overcome the single-treated-unit limitation.

Second, the one-year gap in 2020 ACS data prevents examination of the full post-treatment trajectory. If treatment effects were larger or smaller in 2020, the estimates based on 2019, 2021, and 2022 data may not fully capture the reform's effects.

Third, with only one treatment state, I cannot rule out Kansas-specific confounders unrelated to the beer reform. If Kansas experienced idiosyncratic economic changes in 2019 or later---changes that would have occurred even without the beer law repeal---these would be misattributed to the reform.

Fourth, the ACS measures industry at the time of survey, not the industry history of workers. This cross-sectional measurement cannot directly track individuals who left beverage retail and moved to other industries. I infer adjustment from aggregate patterns rather than individual transitions.

Fifth, I examine only employment outcomes. The reform may have affected other important outcomes---consumer prices, product variety, store hours, tax revenue---that are not captured in the ACS data. A complete welfare analysis would require examining these additional margins.

Given these limitations, this paper should be understood primarily as a methodological illustration rather than definitive causal evidence. The main contribution is demonstrating that conventional inference with few clusters can dramatically overstate precision, not establishing the true effect of Kansas's beer deregulation.

\section{Conclusion}

This paper examines the labor market effects of retail deregulation using Kansas's 2019 repeal of its 82-year-old 3.2\% beer law. The reform allowed grocery stores to compete with liquor stores for full-strength beer sales, fundamentally altering the competitive landscape of alcohol retail in the state.

The main finding is \textit{methodological}: with only five state clusters, conventional clustered standard errors dramatically overstate precision, leading to apparent statistical significance that does not survive appropriate small-cluster inference. Using wild cluster bootstrap, none of the estimated effects---including a -0.15 pp effect on beverage retail employment (11\% decline from baseline)---is statistically distinguishable from zero. This illustrates the dangers of over-interpreting results from policy evaluations with limited geographic variation.

Substantively, the point estimates are consistent with the hypothesis that deregulation harmed the protected incumbent sector while having limited effects on broader self-employment. But the evidence is too imprecise to draw strong causal conclusions. Researchers and policymakers should be appropriately humble about what can be learned from single-state natural experiments.

The concentrated losses among liquor store workers nonetheless raise important equity considerations. While aggregate employment effects may be small, affected individuals may experience significant hardship. Policies that accompany deregulation with transition assistance---job training, extended unemployment benefits, or grants for business adaptation---might help affected workers and owners adjust.

Future research could examine additional dimensions of the Kansas reform. Price effects---whether beer prices fell in Kansas grocery stores after the reform---would speak to consumer welfare gains. Consumption effects---whether beer consumption increased in Kansas---would address public health considerations. Entry patterns---whether new liquor stores or breweries entered the market---would illuminate long-run competitive dynamics.

The Kansas case provides a valuable natural experiment for understanding the employment consequences of retail deregulation more broadly. While specific findings may not generalize to all regulatory contexts, the paper demonstrates that careful empirical analysis can inform debates that often rely more on anecdote and advocacy than on evidence.

\section*{Acknowledgements}

This paper was autonomously generated by Claude Code for the Autonomous Policy Evaluation Project (APEP). I thank the U.S. Census Bureau for providing the American Community Survey PUMS data through its public API. All code and data are available for replication.

\newpage

\section*{References}

\begin{enumerate}
\item Anderson, D. M., Hansen, B., \& Rees, D. I. (2013). Medical marijuana laws and teen marijuana use. \textit{American Law and Economics Review}, 17(2), 495--528.

\item Anderson, D. M., Crost, B., \& Rees, D. I. (2018). Wet laws, drinking establishments and violent crime. \textit{Economic Journal}, 128(611), 1333--1366.

\item Abadie, A., Diamond, A., \& Hainmueller, J. (2010). Synthetic control methods for comparative case studies: Estimating the effect of California's tobacco control program. \textit{Journal of the American Statistical Association}, 105(490), 493--505.

\item Angrist, J. D., \& Pischke, J.-S. (2009). \textit{Mostly Harmless Econometrics: An Empiricist's Companion}. Princeton University Press.

\item Arkhangelsky, D., Athey, S., Hirshberg, D. A., Imbens, G. W., \& Wager, S. (2021). Synthetic difference-in-differences. \textit{American Economic Review}, 111(12), 4088--4118.

\item Bertrand, M., Duflo, E., \& Mullainathan, S. (2004). How much should we trust differences-in-differences estimates? \textit{Quarterly Journal of Economics}, 119(1), 249--275.

\item Callaway, B., \& Sant'Anna, P. H. C. (2021). Difference-in-differences with multiple time periods. \textit{Journal of Econometrics}, 225(2), 200--230.

\item Cameron, A. C., \& Miller, D. L. (2015). A practitioner's guide to cluster-robust inference. \textit{Journal of Human Resources}, 50(2), 317--372.

\item Basker, E. (2005). Job creation or destruction? Labor market effects of Wal-Mart expansion. \textit{Review of Economics and Statistics}, 87(1), 174--183.

\item Blanchflower, D. G., \& Oswald, A. J. (1998). What makes an entrepreneur? \textit{Journal of Labor Economics}, 16(1), 26--60.

\item Cameron, A. C., Gelbach, J. B., \& Miller, D. L. (2008). Bootstrap-based improvements for inference with clustered errors. \textit{Review of Economics and Statistics}, 90(3), 414--427.

\item Carpenter, C., \& Dobkin, C. (2009). The effect of alcohol consumption on mortality: Regression discontinuity evidence from the minimum drinking age. \textit{American Economic Journal: Applied Economics}, 1(1), 164--182.

\item Carpenter, C., \& Dobkin, C. (2011). The minimum legal drinking age and public health. \textit{Journal of Economic Perspectives}, 25(2), 133--156.

\item Chetty, R., Hendren, N., \& Katz, L. F. (2016). The effects of exposure to better neighborhoods on children. \textit{American Economic Review}, 106(4), 855--902.

\item Conley, T. G., \& Taber, C. R. (2011). Inference with difference-in-differences with a small number of policy changes. \textit{Review of Economics and Statistics}, 93(1), 113--125.

\item Cotti, C., \& Tefft, N. (2011). Decomposing the relationship between macroeconomic conditions and fatal car crashes. \textit{Economics Letters}, 110(3), 213--216.

\item Evans, D. S., \& Leighton, L. S. (1989). Some empirical aspects of entrepreneurship. \textit{American Economic Review}, 79(3), 519--535.

\item Fairlie, R. W., Kapur, K., \& Gates, S. (2011). Is employer-based health insurance a barrier to entrepreneurship? \textit{Journal of Health Economics}, 30(1), 146--162.

\item Goodman-Bacon, A. (2021). Difference-in-differences with variation in treatment timing. \textit{Journal of Econometrics}, 225(2), 254--277.

\item Heaton, P. (2012). Sunday liquor laws and crime. \textit{Journal of Public Economics}, 96(1--2), 42--52.

\item Kansas Legislature. (2017). House Bill 2502: An Act concerning cereal malt beverages. Session Laws of Kansas.

\item Kleiner, M. M., \& Krueger, A. B. (2013). Analyzing the extent and influence of occupational licensing on the labor market. \textit{Journal of Labor Economics}, 31(S1), S173--S202.

\item MacKinnon, J. G., \& Webb, M. D. (2017). Wild bootstrap inference for wildly different cluster sizes. \textit{Journal of Applied Econometrics}, 32(2), 233--254.

\item Miron, J. A., \& Tetelbaum, E. (2009). Does the minimum legal drinking age save lives? \textit{Economic Inquiry}, 47(2), 317--336.

\item Neumark, D., Zhang, J., \& Ciccarella, S. (2008). The effects of Wal-Mart on local labor markets. \textit{Journal of Urban Economics}, 63(2), 405--430.

\item Olds, G. (2016). Entrepreneurship and public health insurance. \textit{Working Paper}, Harvard Business School.

\item Seim, K., \& Waldfogel, J. (2013). Public monopoly and economic efficiency: Evidence from the Pennsylvania Liquor Control Board. \textit{American Economic Review}, 103(2), 831--862.

\item Stehr, M. (2007). The effect of Sunday sales bans and excise taxes on drinking and cross-border shopping for alcoholic beverages. \textit{National Tax Journal}, 60(1), 85--105.

\item Sun, L., \& Abraham, S. (2021). Estimating dynamic treatment effects in event studies with heterogeneous treatment effects. \textit{Journal of Econometrics}, 225(2), 175--199.

\item U.S. Census Bureau. (2023). American Community Survey: Public Use Microdata Sample (PUMS). Retrieved from \url{https://www.census.gov/programs-surveys/acs/microdata.html}
\end{enumerate}

\newpage
\appendix

\section{Additional Figures}

\begin{figure}[H]
\centering
\includegraphics[width=0.85\textwidth]{figures/beverage_retail_diff.png}
\caption{Kansas--Control Difference in Beverage Retail Employment by Year}
\par\smallskip\small
\textit{Note: Bars show the raw difference in beverage retail employment rates between Kansas and control states for each year. Blue bars indicate pre-treatment years; red bars indicate post-treatment years.}
\end{figure}

\begin{figure}[H]
\centering
\includegraphics[width=0.85\textwidth]{figures/self_employment_diff.png}
\caption{Kansas--Control Difference in Self-Employment by Year}
\par\smallskip\small
\textit{Note: Bars show the raw difference in self-employment rates between Kansas and control states for each year. Kansas has lower self-employment throughout, but the gap does not widen after 2019.}
\end{figure}

\section{Data Appendix}

\subsection{Industry Code Mapping}

The American Community Survey changed industry codes in 2018 as part of routine updates to the North American Industry Classification System (NAICS). The following mapping was used to create consistent industry indicators across years:

Beer, wine, and liquor stores: INDP code 4970 (2015--2017) corresponds to INDP code 4971 (2018 and later). Both codes map to NAICS 4453.

Grocery stores: INDP code 4670 was unchanged throughout the sample period and corresponds to NAICS 4451.

\subsection{State FIPS Codes}

The following Federal Information Processing Standard (FIPS) codes identify states in the ACS data:

Kansas (treatment): 20

Nebraska (control): 31

Missouri (control): 29

Oklahoma (control): 40

Colorado (control): 08

\subsection{Replication Materials}

All analysis code and data are available at the APEP GitHub repository: \url{https://github.com/dakoyana/auto-policy-evals/output/paper\_26}

\end{document}
