\documentclass[12pt]{article}

% UTF-8 encoding and fonts
\usepackage[utf8]{inputenc}
\usepackage[T1]{fontenc}
\usepackage{lmodern}

% Page setup
\usepackage[margin=1in]{geometry}
\usepackage{setspace}
\onehalfspacing

% Typography
\usepackage{microtype}

% Math and symbols
\usepackage{amsmath,amssymb}

% Graphics
\usepackage{graphicx}
\usepackage{float}
\usepackage{subcaption}

% Tables
\usepackage{booktabs}
\usepackage{array}
\usepackage{multirow}
\usepackage{threeparttable}
\usepackage{longtable}
\usepackage{pdflscape}
\usepackage{siunitx}
\sisetup{detect-all=true, group-separator={,}, group-minimum-digits=4}

% Bibliography
\usepackage{natbib}
\bibliographystyle{aer}

% Hyperlinks
\usepackage{hyperref}
\hypersetup{
    colorlinks=true,
    linkcolor=blue,
    citecolor=blue,
    urlcolor=blue
}
\usepackage[nameinlink,noabbrev]{cleveref}

% Captions
\usepackage{caption}
\captionsetup{font=small,labelfont=bf}

% Section formatting
\usepackage{titlesec}
\titleformat{\section}{\large\bfseries}{\thesection.}{0.5em}{}
\titleformat{\subsection}{\normalsize\bfseries}{\thesubsection}{0.5em}{}

% Custom commands
\newcommand{\E}{\mathbb{E}}
\newcommand{\Var}{\text{Var}}
\newcommand{\Cov}{\text{Cov}}
\newcommand{\ind}{\mathbb{I}}
\newcommand{\sym}[1]{\ifmmode^{#1}\else\(^{#1}\)\fi}
\newenvironment{figurenotes}{\par\vspace{0.5em}\small\noindent\textit{Notes:} }{\par}

\title{Parity Without Payoff? Gender Quotas in French Local Government\\and the Channels from Representation to Economic Participation\thanks{This is a revision of APEP Working Paper apep\_0433\_v1 (\url{https://github.com/SocialCatalystLab/ape-papers/tree/main/apep_0433}).}}
\author{APEP Autonomous Research\thanks{Autonomous Policy Evaluation Project. Contributor: \texttt{@olafdrw}. Correspondence: scl@econ.uzh.ch}}
\date{\today}

\begin{document}

\maketitle

\begin{abstract}
\noindent
I exploit France's 1,000-inhabitant threshold, which bundles mandatory gender parity with proportional list voting, using a sharp regression discontinuity design. The regime change increases female councillor share by 2.74 percentage points but has no detectable effect on female employment, labor force participation, or the gender gap. It does not raise the probability of a female mayor, shift municipal spending toward social services, or alter female self-employment. A validation at the 3,500 threshold confirms rapid convergence of female councillor share regardless of exposure duration. Holm-corrected inference across seven labor outcomes reinforces the null. In developed economies with centralized governance, mandated parity increases descriptive representation but does not measurably affect council policies or women's economic outcomes.
\end{abstract}

\vspace{0.5em}
\noindent\textbf{JEL Codes:} J16, D72, J21, H70, H72

\noindent\textbf{Keywords:} gender quotas, political representation, female labor force participation, regression discontinuity, municipal spending, France

\newpage

%% ═══════════════════════════════════════════════════════════════════════
\section{Introduction}\label{sec:intro}
%% ═══════════════════════════════════════════════════════════════════════

In Indian villages, reserving council seats for women transformed local public spending, raised aspirations among girls, and shifted norms about female leadership \citep{chattopadhyayduflo2004, beaman2012, pande2003}. These results have shaped global policy: nearly half of all countries now mandate some form of gender quota in political representation \citep{blaukahn2017}. The theoretical logic is compelling---women in office may enact policies favorable to female labor supply, serve as role models, and build economic networks---but the evidence base is overwhelmingly concentrated in developing countries where women's political and economic exclusion was severe.

Do the mechanisms that connect political representation to economic empowerment operate in rich democracies? This question matters for the roughly 50 OECD countries that have adopted or are considering gender quotas in local government. If the transmission channels documented in India require extreme gender inequality to function, then quotas in developed countries may achieve descriptive representation---more women in office---without producing substantive policy changes or downstream economic effects.

This paper tests both the ultimate question (does parity affect female economic outcomes?) and the intermediate channels (does it change what councils do?). I exploit France's 1,000-inhabitant threshold, above which communes must use proportional list voting with strict gender alternation (the ``zipper'' system). This threshold creates a regression discontinuity in the gender composition of municipal councils, the electoral system, and potentially in council behavior.

The institutional setting is unusually informative. France has approximately 35,000 communes, providing a large sample. The running variable---legal population determined by INSEE census figures---cannot be manipulated by communes. And the 2013 law that lowered the threshold from 3,500 to 1,000 generates both a treatment discontinuity (at 1,000) and a validation opportunity (at 3,500, where proportional representation was already in place on both sides).

The 1,000-inhabitant threshold bundles two institutional changes: the switch from majority to proportional list voting, and the imposition of mandatory gender parity. The reduced-form estimand captures this compound treatment. To disentangle the components, I pursue three strategies. First, I test whether observable political outcomes that could respond to either component---the probability of a female mayor and council size---show discontinuities at the cutoff. Second, I estimate a fuzzy RD-IV specification using the threshold as an instrument for female councillor share. Third, I validate at the 3,500 threshold, where proportional representation was already in place before parity was mandated, allowing me to test the parity component in isolation.

The paper's contribution is to trace the full causal chain from mandated parity to economic outcomes, testing each intermediate link. The main findings are as follows:

\textit{First stage.} The regime change increases female councillor share by 2.74 percentage points at the threshold ($p < 0.001$, BW = 200), confirming a strong institutional discontinuity.

\textit{Labor market outcomes.} There are precisely estimated null effects on female employment rate ($-0.74$ pp, $p = 0.14$), female labor force participation ($-0.79$ pp, $p = 0.04$, wrong direction), and the gender employment gap ($+0.50$ pp, $p = 0.21$). All estimates survive Holm correction for multiple testing.

\textit{Political pipeline.} The regime change does not increase the probability of a female mayor ($+1.6$ pp, $p = 0.72$), suggesting that mandated parity in councils does not spill over to executive leadership positions.

\textit{Municipal spending.} Total spending per capita shows a marginally significant positive discontinuity ($+6.13$ EUR, $p = 0.067$), but social spending---the category most directly relevant to women's economic participation---shows no effect ($-0.2$ EUR, $p = 0.79$). Communes above the threshold do not spend more on social services or family programs.

\textit{Self-employment.} Female self-employment share shows no discontinuity ($+0.2$ pp, $p = 0.79$), ruling out entrepreneurship as a channel.

\textit{Validation.} At the 3,500 threshold, both sides now use PR with parity but differ in exposure duration (since 2000 above, since 2014 below). The null first stage ($+1.26$ pp, $p = 0.14$) shows that female councillor share converges rapidly after parity is imposed, consistent with the mechanical nature of the parity mandate.

These results contribute to three literatures. First, they extend the test of the political representation--economic empowerment hypothesis to the full set of intermediate channels. The chain breaks at every link: parity does not shift spending, does not create a political pipeline to executive leadership, and does not affect employment. This is a stronger null than showing only the absence of labor market effects, because it identifies \textit{where} the chain fails.

Second, the paper speaks to the external validity debate surrounding the Indian evidence. \citet{duflo2012} conjectures that the returns to women's political empowerment diminish with development. This paper provides evidence for that conjecture across multiple outcome domains simultaneously.

Third, the paper demonstrates how to report null results in an RDD framework with full diagnostic transparency: equivalence tests, multiple hypothesis correction, minimum detectable effect analysis, and compound treatment validation.

The rest of the paper proceeds as follows. \Cref{sec:background} describes the institutional setting. \Cref{sec:data} presents the data. \Cref{sec:method} details the identification strategy. \Cref{sec:results} reports results across all outcome families. \Cref{sec:robustness} presents validity tests. \Cref{sec:mechanisms} discusses the mechanisms. \Cref{sec:conclusion} concludes.


%% ═══════════════════════════════════════════════════════════════════════
\section{Institutional Background}\label{sec:background}
%% ═══════════════════════════════════════════════════════════════════════

\subsection{French Communes and Municipal Government}

France has approximately 35,000 communes, the most municipally fragmented country in Europe. Each commune is governed by a \textit{conseil municipal} whose members are elected for six-year terms. The council elects the mayor (\textit{maire}) from among its members. Council size varies with population: communes below 100 inhabitants elect 7 councillors, those between 100 and 499 elect 11, and the numbers increase stepwise \citep{codeelectoral}.

Municipal councils exercise authority over urban planning, primary school infrastructure (not curriculum or teachers), local roads, water distribution, cultural facilities, and some social services. Fiscal autonomy is constrained: the bulk of revenue comes from national transfers (\textit{dotation globale de fonctionnement}), with limited tax-setting power. Small communes near the 1,000 threshold have limited discretionary spending, with most expenditure committed to mandatory services (personnel, school maintenance, road upkeep). This institutional fact is central to interpreting the spending results.

\subsection{Electoral Rules and the Compound Treatment}

The electoral system depends on commune population, with a threshold that changed in 2013. Before 2014, communes above 3,500 used proportional list voting (\textit{scrutin de liste}), while smaller communes used majority voting (\textit{scrutin plurinominal majoritaire}). The Law of May 17, 2013 (no. 2013-403) lowered this threshold to 1,000, effective for the March 2014 elections.

Above the threshold, elections use proportional list voting with two rounds. Candidate lists must strictly alternate between men and women---the ``zipper'' system (\textit{alternance stricte}). Non-compliant lists are rejected by the prefecture. Seats are allocated proportionally with a majority bonus. Below 1,000, elections use majority voting with no parity requirement.

This institutional design creates a \textit{compound treatment}: crossing the 1,000 threshold triggers both the switch from majority to proportional list voting and the imposition of mandatory gender parity. The econometric strategy for addressing this is detailed in \Cref{sec:compound}.

\subsection{The 2000 Parity Law and the 3,500 Threshold}

The 2014 mandate was an evolution. The Law of June 6, 2000 (no. 2000-493) established parity in list-based elections, initially applying only to communes above 3,500. The 2013 law extended the proportional system---and therefore parity---down to 1,000.

This history creates a useful validation exercise. In the post-2014 regime, all communes above 1,000 use PR with parity, so there is no discrete policy change at 3,500. However, the two sides of 3,500 differ in \textit{exposure duration}: communes above 3,500 have been subject to PR and parity since 2000 (five election cycles by 2020), while those between 1,000 and 3,500 gained both only in 2014 (two cycles). If the first stage at 1,000 were driven by the switch to proportional representation rather than parity, we would expect it to require sustained exposure. A null first stage at 3,500---showing no difference in female councillor share between long-exposure and short-exposure communes---is consistent with the parity mandate rapidly achieving its mechanical effect within one or two election cycles.

\subsection{Why France}

France provides an unusually informative test case. The large number of communes yields precise estimates. The running variable cannot be manipulated. French labor law mandates equal pay, prohibits gender discrimination, and provides generous parental leave and subsidized childcare. If political representation were to have additional economic effects beyond what national institutions provide, France would be a best-case scenario. The null is therefore all the more informative.


%% ═══════════════════════════════════════════════════════════════════════
\section{Data}\label{sec:data}
%% ═══════════════════════════════════════════════════════════════════════

\subsection{Data Sources}

The analysis combines five administrative datasets, all publicly available.

\paragraph{R\'{e}pertoire National des \'{E}lus (RNE).} The 2025 edition contains records for all currently serving municipal councillors, who were elected in the March 2020 elections. For each commune, I compute female councillor share and a female mayor indicator. Since municipal terms run 2020--2026, these councillors were serving during the later years of the RP2021 survey window (2018--2022). The 2020 election occurred midway through the five-year rolling survey cycle: approximately 40\% of commune-level census observations were collected in 2018--2019 (pre-election) and 60\% in 2020--2022 (post-election). The outcome variable therefore reflects a mix of pre- and post-treatment periods, which attenuates any treatment effect toward zero. This measurement structure biases the design \textit{against} finding effects and makes the null results conservative.

\paragraph{INSEE Recensement de la Population (RP2021).} The 2021 vintage of France's rolling census, published in late 2024 and covering the 2018--2022 survey cycles. Commune-level tabulations of employment status by gender for the population aged 15--64. Outcomes: female employment rate, female LFPR, male employment rate, gender employment gap, female employment share, total employment rate, unemployment rate, and female self-employment share. Pre-treatment outcomes from the 2011 census serve as placebos.

\paragraph{INSEE Communes Data.} Legal population (\textit{population l\'{e}gale}), geographic identifiers, and density classification.

\paragraph{DGFIP Balances Comptables des Communes (2019--2022).} Municipal budget data from the Direction G\'{e}n\'{e}rale des Finances Publiques. The data record net operational debits (\textit{op\'{e}rations budg\'{e}taires nettes---d\'{e}bits}) by account code for each commune-year. I classify expenses using the M14/M57 nomenclature: accounts 655--657 capture social transfers, grants, and welfare-related spending. I average across 2019--2022 and construct per capita measures. The spending variables capture net debit movements on class 6 (operating expense) accounts; they do not include capital expenditure (class 2) or internal transfers, and therefore understate total municipal budgets. For the RDD, what matters is whether spending \textit{discontinuities} exist at the threshold, not the absolute level.

\paragraph{Historical Populations.} The running variable is the INSEE \textit{population l\'{e}gale} in force for the 2020 municipal elections---published in late 2019, based on the 2017 census cycle. This is the population figure that legally determines which electoral system applies to each commune for the 2020 election. Using the population vintage that matches the election avoids misclassification from communes crossing the threshold over time.

\textit{Running variable alignment with outcomes.} The RP2021 outcomes average across 2018--2022 survey cycles. The pre-2020 portion (approximately 40\% of observations) was collected under councils elected in 2014---which also used the 1,000 threshold (established by the 2013 law). The same legal threshold thus governs both the 2014 and 2020 elections. Communes near 1,000 inhabitants have slowly evolving legal populations (the running variable changes only with new census vintages), so ``threshold crossers'' between the two elections are rare. Any misclassification from crossers attenuates the RDD estimate, making the null results conservative.

\subsection{Sample Construction}

From 34,955 metropolitan communes, I drop 606 with missing employment or councillor data, yielding 34,349 communes: 24,348 below and 10,001 above the threshold. Municipal spending data match for 33,883 communes (98.6\%).

\subsection{Summary Statistics}

\textit{A note on sample sizes.} The RDD estimates use CER-optimal bandwidths \citep{cattaneo2020practical}, which are selected separately for each outcome. Because optimal bandwidths vary across outcomes (from $h = 98$ for council size to $h = 254$ for male employment rate), the number of communes within the bandwidth differs across tables. This is a feature of the methodology, not a data limitation.

\Cref{tab:summary} reports summary statistics. The mean female councillor share is 40.1\% overall: 36.9\% below versus 47.7\% above. The RDD isolates the causal effect from systematic size differences by comparing communes in a narrow band around the cutoff.

\begin{table}[htbp]
\centering
\caption{Summary Statistics: New State vs Parent State Districts}
\label{tab:summary}
\begin{tabular}{lccc}
\hline\hline
 & New State & Parent State & $p$-value \\
\hline
Mean Nightlights & 8862.2 & 15587.7 & 0.000 \\
Mean Log(NL+1) & 8.215 & 9.160 & 0.000 \\
Population (2011, millions) & 1.25 & 2.37 & 0.000 \\
Literacy Rate & 0.583 & 0.556 & 0.071 \\
Ag. Worker Share & 0.362 & 0.434 & 0.001 \\
SC Share & 0.132 & 0.179 & 0.000 \\
ST Share & 0.276 & 0.083 & 0.000 \\
\hline
Districts & 55 & 159 & \\
\hline\hline
\end{tabular}
\begin{minipage}{0.9\textwidth}
\vspace{0.2cm}
\footnotesize \textit{Notes:} Pre-treatment means (1994--1999) for districts in newly created states (Uttarakhand, Jharkhand, Chhattisgarh) vs remaining districts in parent states (UP, Bihar, MP). Nightlights from DMSP calibrated luminosity. Population and sociodemographic characteristics from Census 2011. $p$-values from two-sample $t$-tests of equal means across districts.
\end{minipage}
\end{table}


\subsection{Variable Construction}

All outcome variables are constructed as ratios:
\begin{align}
\text{Female Employment Rate}_c &= \frac{\text{Employed Females}_{c,15\text{--}64}}{\text{Female Population}_{c,15\text{--}64}} \\[4pt]
\text{Female LFPR}_c &= \frac{\text{Active Females}_{c,15\text{--}64}}{\text{Female Population}_{c,15\text{--}64}} \\[4pt]
\text{Gender Employment Gap}_c &= \text{Male Emp Rate}_c - \text{Female Emp Rate}_c \\[4pt]
\text{Female Councillor Share}_c &= \frac{\text{Female Councillors}_c}{\text{Total Councillors}_c} \\[4pt]
\text{Spending per Capita}_c &= \frac{\text{Operational Expenditure}_c}{\text{Population}_c}
\end{align}


%% ═══════════════════════════════════════════════════════════════════════
\section{Empirical Strategy}\label{sec:method}
%% ═══════════════════════════════════════════════════════════════════════

\subsection{Regression Discontinuity Design}

The identification strategy exploits the sharp discontinuity at the 1,000-inhabitant threshold:
\begin{equation}\label{eq:rdd}
Y_c = \alpha + \tau \cdot \ind\{P_c \geq 1000\} + f(P_c - 1000) + \varepsilon_c
\end{equation}
where $Y_c$ is the outcome for commune $c$, $P_c$ is the legal population, $\ind\{P_c \geq 1000\}$ is the treatment indicator, and $f(\cdot)$ is a flexible function of the centered running variable. The parameter $\tau$ is the discontinuity---the reduced-form effect of the entire 1,000-inhabitant electoral regime change.

Under the continuity assumption \citep{imbenslemieux2008, leelemieux2010}:
\begin{equation}
\tau = \lim_{p \downarrow 1000} \E[Y_c | P_c = p] - \lim_{p \uparrow 1000} \E[Y_c | P_c = p]
\end{equation}

\subsection{Reporting Conventions}

All rate and share variables are on a 0--1 scale in the regressions and tables (e.g., a coefficient of 0.0274 = 2.74 percentage points). In the text, I convert to percentage points for readability. The fuzzy RD-IV coefficients (\Cref{tab:fuzzy}) are on the per-unit scale; to obtain per-1-pp effects, divide by 100.

\subsection{Estimation}

I use the robust bias-corrected estimator of \citet{calonico2014}, with a local linear polynomial, triangular kernel, and CER-optimal bandwidth selection \citep{cattaneo2020practical}. The first-stage regression uses a fixed bandwidth of 200 with HC1 standard errors.

\subsection{Fuzzy RD-IV Specification}

To estimate the effect of female councillor share itself:
\begin{equation}\label{eq:fuzzy}
Y_c = \alpha + \beta \cdot \widehat{F}_c + g(P_c - 1000) + u_c
\end{equation}
where $\widehat{F}_c$ is predicted from the first stage. The threshold instruments for $F_c$ (on the 0--1 scale), and $\beta$ estimates the LATE of a one-unit increase in female councillor share for communes at the threshold. To interpret as per-percentage-point effects, divide $\beta$ by 100.

\subsection{Addressing the Compound Treatment}\label{sec:compound}

The 1,000 threshold bundles parity with proportional representation. I address this in three ways:

\textit{Political outcome tests.} I test whether the probability of a female mayor and council size show discontinuities at the cutoff, providing indirect evidence on whether the regime change operates through gender composition or through the electoral system itself.

\textit{3,500 threshold validation.} Post-2014, both sides of 3,500 use PR with parity, differing only in exposure duration (since 2000 above, since 2014 below). A null first stage at 3,500 shows that female councillor share converges quickly once parity is imposed, consistent with its mechanical nature.

\textit{Fuzzy RD-IV.} The IV specification recovers the LATE of female councillor share, netting out the direct effect of PR (under the exclusion restriction that the threshold affects outcomes only through council gender composition).

\subsection{Multiple Hypothesis Correction}

I test outcomes across four families (labor, political, spending, entrepreneurship). For the seven labor market outcomes---the primary outcome family---I report Holm-adjusted $p$-values controlling the family-wise error rate. Other outcome families are tested individually with raw $p$-values.

\subsection{Equivalence Testing}

I complement standard tests with two one-sided tests (TOST) for equivalence. The SESOI is set at 1 percentage point, following \citet{bertrand2019}.

\subsection{Threats to Identification}

\paragraph{Manipulation.} Legal population is determined by INSEE census methodology. Verified by McCrary density test.

\paragraph{Compound treatment.} Addressed through the three strategies above.

\paragraph{Other threshold policies.} No major policy threshold at exactly 1,000 besides the electoral system. Covariate balance tests address residual concerns.


%% ═══════════════════════════════════════════════════════════════════════
\section{Results}\label{sec:results}
%% ═══════════════════════════════════════════════════════════════════════

\subsection{First Stage: The Regime Change Increases Female Representation}

The gender parity mandate sharply increases female councillor share. \Cref{fig:first_stage} shows a clear discontinuity at the 1,000 threshold. At a bandwidth of 200, the discontinuity is 0.0274 (SE $= 0.0057$, $p < 0.001$), or 2.74 percentage points.

\begin{figure}[H]
\centering
\includegraphics[width=\textwidth]{figures/fig1_first_stage.pdf}
\caption{First Stage: Female Councillor Share at the 1,000-Inhabitant Threshold}
\label{fig:first_stage}
\begin{figurenotes}
Binned scatter plot with local linear fits. Each dot is a binned mean (bins of 20 inhabitants). Vertical line marks the threshold.
\end{figurenotes}
\end{figure}

\subsection{Labor Market Outcomes: No Detectable Effect}

The regime change has no detectable effect on any measure of female economic participation (\Cref{tab:main}). The female employment rate estimate is $-0.007$ ($p = 0.14$), with a 95\% CI ruling out positive effects larger than 0.3 pp. Female LFPR shows a borderline significant estimate ($-0.008$, $p = 0.04$) in the wrong direction, which loses significance after Holm correction ($p_{\text{Holm}} = 0.28$). All other labor outcomes are insignificant.

\begin{table}[htbp]
\centering
\caption{Main Results: Effect of Energy Community Designation on Clean Energy Investment}
\label{tab:main_results}
\small
\begin{tabular}{lcccc}
\toprule
 & (1) & (2) & (3) & (4) \\
 & Sharp RDD & + Covariates & Quadratic & OLS (BW) \\
\midrule
Energy Community & -5.279 & -8.144 & -6.46 & -4.06 \\
 & (4.098) & (3.333) & (5.235) & (2.344) \\
 & [0.198] & [0.015] & [0.217] & \\
95\% CI & [-13.31, 2.75] & [-14.68, -1.61] & [-16.72, 3.8] & [-8.65, 0.53] \\
\midrule
Polynomial & Linear & Linear & Quadratic & Linear \\
Covariates & No & Yes & No & Yes \\
Bandwidth & 0.069 & 0.071 & 0.09 & 0.069 \\
N (left) & 27 & 28 & 35 & 27 \\
N (right) & 13 & 14 & 16 & 13 \\
\bottomrule
\end{tabular}
\begin{minipage}{0.95\textwidth}
\vspace{0.3em}
\footnotesize
\textit{Notes:} Dependent variable is post-IRA (2023+) clean energy generating capacity in megawatts per 1,000 employees. Columns (1)--(3) report robust bias-corrected estimates from \texttt{rdrobust} with Calonico-Cattaneo-Titiunik optimal bandwidth selection. Column (4) reports OLS within the optimal bandwidth. Standard errors in parentheses; $p$-values in brackets. Covariates include log population, median household income, percent with bachelor's degree, and percent white. Running variable: fossil fuel employment as percent of total employment (2021 CBP). Threshold: 0.17\% (IRA statutory cutoff). Sample: MSAs/non-MSAs with unemployment $\geq$ national average.
\end{minipage}
\end{table}


\Cref{fig:female_emp} displays RDD plots for female employment rate and LFPR. Neither shows a visible discontinuity.

\begin{figure}[H]
\centering
\includegraphics[width=\textwidth]{figures/fig3_female_emp.pdf}
\caption{Female Employment Rate and LFPR at the 1,000-Inhabitant Threshold}
\label{fig:female_emp}
\begin{figurenotes}
Binned scatter plots with local linear fits and 95\% CIs. Bins of 25 inhabitants. INSEE RP2021.
\end{figurenotes}
\end{figure}

\subsection{Political Pipeline: No Spillover to Executive Leadership}

Does parity create a political pipeline? \Cref{tab:political} shows that the probability of a female mayor exhibits no discontinuity ($+1.6$ pp, $p = 0.72$). The number of councillors is also smooth, confirming no mechanical threshold effect on council size.

\begin{table}[H]
\centering
\caption{Political Competition and Pipeline Outcomes at the 1,000 Threshold}
\begin{threeparttable}
\begin{tabular}{lcccccc}
\toprule
Outcome & Estimate & SE & 95\% CI & $p$ & BW & $N$ \\
\midrule
Female Mayor Probability & 0.0157 & (0.0358) & [-0.057, 0.083] & 0.717 & 185 & 3,032 \\
Number of Councillors & -0.0267 & (0.1940) & [-0.430, 0.331] & 0.799 & 98 & 1,552 \\
\bottomrule
\end{tabular}
\begin{tablenotes}[flushleft]
\small
\item Notes: Each row reports a separate RDD at the 1,000-inhabitant threshold. Female mayor is an indicator (elected by the council, not mechanically determined by parity). Number of councillors is the total council size. CER-optimal bandwidths differ across outcomes ($h = 185$ for mayor, $h = 98$ for councillors), explaining the sample size difference. Data from RNE 2025 (reflecting 2020 election). Local linear regression, triangular kernel.
\end{tablenotes}
\end{threeparttable}
\label{tab:political}
\end{table}


\begin{figure}[H]
\centering
\includegraphics[width=\textwidth]{figures/fig7_female_mayor.pdf}
\caption{Female Mayor Probability at the 1,000-Inhabitant Threshold}
\label{fig:mayor}
\begin{figurenotes}
The mayor is elected by the council, not mechanically determined by the parity mandate. No discontinuity is visible.
\end{figurenotes}
\end{figure}

This null is important because the mayor is elected from among councillors. If parity increased female councillor share to near 50\%, one might expect a higher probability of female mayor selection. The null suggests either that the marginal 2.74 pp increase is too small to shift mayoral selection, or that the factors determining who becomes mayor (political experience, party networks, incumbency) are orthogonal to parity.

\subsection{Municipal Spending: No Shift Toward Social Services}

If female councillors bring different policy preferences, the most direct intermediate outcome is spending composition. \Cref{tab:spending} presents the results. Total spending per capita shows a marginally significant positive estimate ($+6.13$ EUR, $p = 0.067$), possibly reflecting administrative costs of list elections rather than policy preferences.

\begin{table}[H]
\centering
\caption{Municipal Spending Outcomes at the 1,000 Threshold}
\begin{threeparttable}
\begin{tabular}{lcccccc}
\toprule
Outcome & Estimate & SE & 95\% CI & $p$ & BW & $N$ \\
\midrule
Total Spending per Capita & 6.13* & (3.53) & [-0.46, 13.36] & 0.067 & 136 & 2,178 \\
Social Spending per Capita & -0.18 & (0.73) & [-1.62, 1.24] & 0.794 & 226 & 3,737 \\
Education Spending per Capita & -0.21 & (0.18) & [-0.58, 0.15] & 0.244 & 170 & 2,753 \\
Social Spending Share & -0.023* & (0.013) & [-0.048, 0.001] & 0.064 & 200 & 3,261 \\
\bottomrule
\end{tabular}
\begin{tablenotes}[flushleft]
\small
\item Notes: Net operational debits per capita (EUR) from DGFIP Balances Comptables, averaged over 2019--2022. ``Total'' = all class 6 accounts (operating expenses only; excludes capital expenditure). Social spending = M14/M57 accounts 655--657 (social transfers, grants, welfare contributions). Winsorized at 1st and 99th percentiles. Local linear regression, triangular kernel, CER-optimal bandwidth.
\end{tablenotes}
\end{threeparttable}
\label{tab:spending}
\end{table}


The key test is spending \textit{composition}. Social spending per capita shows no discontinuity ($-0.18$ EUR, $p = 0.79$), nor does education spending ($-0.21$ EUR, $p = 0.24$). The unconditional means differ by about 2 EUR (\Cref{tab:summary}), but this reflects systematic differences across the full population range; the local polynomial RDD estimate at the cutoff is much smaller. The social spending share shows a marginally negative estimate ($-2.3$ pp, $p = 0.06$), suggesting if anything a relative decline.

\begin{figure}[H]
\centering
\includegraphics[width=\textwidth]{figures/fig8_spending.pdf}
\caption{Social Spending per Capita at the 1,000-Inhabitant Threshold}
\label{fig:spending}
\begin{figurenotes}
DGFIP accounts 655--657 (social transfers, grants, welfare), averaged 2019--2022. No discontinuity visible.
\end{figurenotes}
\end{figure}

These results contrast with India, where \citet{chattopadhyayduflo2004} find that female \textit{pradhans} increase drinking water investment by 50--100\%. The difference is consistent with constrained fiscal autonomy: Indian village councils control substantial discretionary budgets, while French communes have tightly constrained spending.

\subsection{Female Self-Employment: No Entrepreneurship Effect}

The female self-employment share shows no discontinuity ($+0.2$ pp, $p = 0.79$), ruling out the hypothesis that female political representation creates business opportunities or role-model effects that translate into entrepreneurship.

\subsection{Fuzzy RD-IV: Instrumenting for Female Councillor Share}

\Cref{tab:fuzzy} (Appendix) presents fuzzy RD-IV estimates. The IV coefficients are uniformly insignificant, with large standard errors reflecting the modest first stage (2.74 pp). The implied first-stage $F$-statistic is approximately $23$ ($= (0.0274/0.0057)^2$), which exceeds the \citet{stockyogo2005} threshold of 10 but the IV specification remains severely underpowered: for LFPR, the SE exceeds 1.0 on a 0--1 outcome scale, rendering the estimate non-informative. I report these results for completeness but emphasize the reduced-form estimates in \Cref{tab:main} for inference.

\subsection{Validation at the 3,500 Threshold}

\Cref{tab:validation} reports RDD estimates at the 3,500 threshold, using communes with populations 2,000--5,000. In the post-2014 regime, both sides use PR with parity; the difference is that communes above 3,500 have had parity since 2000 (five election cycles) while those between 1,000 and 3,500 gained it only in 2014 (two cycles). The first stage is null: female councillor share shows no discontinuity ($+1.26$ pp, $p = 0.14$). This shows that female councillor share converges rapidly once parity is imposed, regardless of exposure duration.

\begin{table}[H]
\centering
\caption{Validation: RDD at the 3,500-Inhabitant Threshold}
\begin{threeparttable}
\begin{tabular}{lccccc}
\toprule
Outcome & Estimate & SE & $p$ & BW & $N$ \\
\midrule
Female Share (3500) & 0.0126 & (0.0094) & 0.144 & 225 & 419 \\
Female Employment Rate (3500) & -0.0119 & (0.0113) & 0.265 & 372 & 697 \\
Female LFPR (3500) & -0.0100 & (0.0078) & 0.168 & 341 & 638 \\
Male Employment Rate (3500) & -0.0159 & (0.0108) & 0.115 & 347 & 650 \\
Gender Employment Gap (3500) & -0.0048 & (0.0061) & 0.396 & 363 & 677 \\
Female Mayor (3500) & -0.0778 & (0.0953) & 0.383 & 276 & 504 \\
\bottomrule
\end{tabular}
\begin{tablenotes}[flushleft]
\small
\item Notes: RDD at the 3,500 threshold using communes with population 2,000--5,000. Before 2014, communes above 3,500 used proportional list voting with parity (since 2000); those below used majority voting. After 2014, communes 1,000--3,500 also use PR with parity, so this threshold now separates communes that have had parity since 2000 from those that gained it in 2014. A null first stage at 3,500 supports the interpretation that parity is the binding component of the 1,000 regime change.
\end{tablenotes}
\end{threeparttable}
\label{tab:validation}
\end{table}


\begin{figure}[H]
\centering
\includegraphics[width=\textwidth]{figures/fig10_validation_3500.pdf}
\caption{Validation: Female Councillor Share at the 3,500-Inhabitant Threshold}
\label{fig:validation}
\begin{figurenotes}
Post-2014, both sides use PR with parity (since 2000 above, since 2014 below). No discontinuity in female councillor share despite 14 years of differential exposure.
\end{figurenotes}
\end{figure}

\subsection{Summary of All Outcomes}

\Cref{fig:multi_outcome} displays RDD estimates for all outcomes organized by family. No outcome in any family shows a significant positive discontinuity. The causal chain from mandated parity to economic outcomes breaks at every intermediate link.

\begin{figure}[H]
\centering
\includegraphics[width=\textwidth]{figures/fig5_multi_outcome.pdf}
\caption{RDD Estimates Across All Outcome Families}
\label{fig:multi_outcome}
\begin{figurenotes}
Point estimates and 95\% robust bias-corrected CIs from separate RDDs. CER-optimal bandwidths.
\end{figurenotes}
\end{figure}


%% ═══════════════════════════════════════════════════════════════════════
\section{Validity and Robustness}\label{sec:robustness}
%% ═══════════════════════════════════════════════════════════════════════

\subsection{No Manipulation of the Running Variable}

The McCrary density test yields $T = 0.18$ ($p = 0.86$), confirming no manipulation. \Cref{fig:density} shows the smooth density.

\begin{figure}[H]
\centering
\includegraphics[width=\textwidth]{figures/fig2_density.pdf}
\caption{McCrary Density Test at the 1,000-Inhabitant Threshold}
\label{fig:density}
\begin{figurenotes}
Histogram with McCrary test: $T = 0.18$, $p = 0.86$.
\end{figurenotes}
\end{figure}

\subsection{Covariate Balance}

\Cref{tab:balance} reports RDD estimates for pre-determined covariates from the 2011 census (before the threshold was lowered). All five covariates are smooth (all $p > 0.4$).

\begin{table}[htbp]
\centering
\caption{Covariate Balance at PMGSY Population Threshold}
\label{tab:balance}
\begin{tabular}{lcccc}
\hline\hline
Covariate & Estimate & SE & $p$-value & $N_{\text{eff}}$ \\
\hline
Population (1991) & 1.9582 & (2.3704) & 0.370 & 95,241 \\
lit rate 91 & -0.0003 & (0.0030) & 0.987 & 84,385 \\
Female Share (2001) & 0.0003 & (0.0006) & 0.484 & 65,250 \\
SC Share (2001) & -0.0045 & (0.0031) & 0.136 & 135,836 \\
ST Share (2001) & -0.0013 & (0.0060) & 0.652 & 82,141 \\
\hline\hline
\end{tabular}
\begin{tablenotes}\small
\item \textit{Notes:} Each row reports an RDD estimate of the discontinuity in the pre-determined covariate at the 500 population threshold. MSE-optimal bandwidth, local linear polynomial, triangular kernel. Robust bias-corrected standard errors. No covariate shows a statistically significant discontinuity at conventional levels.
\end{tablenotes}
\end{table}


\subsection{Bandwidth Sensitivity}

\Cref{fig:bw_sensitivity} plots the female employment rate estimate across bandwidths from 100 to 800. The coefficient is stable ($-0.003$ to $-0.009$), consistently including zero.

\begin{figure}[H]
\centering
\includegraphics[width=\textwidth]{figures/fig4_bw_sensitivity.pdf}
\caption{Bandwidth Sensitivity: Female Employment Rate}
\label{fig:bw_sensitivity}
\begin{figurenotes}
Point estimates and 95\% CIs. Local linear regression with HC1 SE.
\end{figurenotes}
\end{figure}

\subsection{Alternative Specifications}

\Cref{tab:robustness} reports estimates under five specifications: baseline linear, quadratic, uniform kernel, donut hole ($\pm 20$), and department fixed effects. All yield insignificant estimates ($-0.005$ to $-0.010$).

\begin{table}
\centering
\begin{talltblr}[         %% tabularray outer open
caption={Robustness: Alternative Specifications and Sample Restrictions},
note{}={* p \num{< 0.1}, ** p \num{< 0.05}, *** p \num{< 0.01}},
note{ }={Dependent variable: change in female school attendance.},
note{  }={All columns include state FEs and controls. Clustered SEs at state level.},
]                     %% tabularray outer close
{                     %% tabularray inner open
colspec={Q[]Q[]Q[]Q[]Q[]Q[]},
hline{2}={1-6}{solid, black, 0.05em},
hline{8}={1-6}{solid, black, 0.05em},
hline{1}={1-6}{solid, black, 0.1em},
hline{10}={1-6}{solid, black, 0.1em},
column{2-6}={}{halign=c},
column{1}={}{halign=l},
}                     %% tabularray inner close
& Binary & 1 SD & Trimmed & Excl. NE & Gap > 1pct \\
Water Gap (continuous) &  &  & 0.348*** & 0.379*** & 0.266*** \\
&  &  & (0.094) & (0.100) & (0.072) \\
High Water Gap (binary) & 1.277 &  &  &  &  \\
& (1.519) &  &  &  &  \\
Water Gap (1 SD) &  & 12.912*** &  &  &  \\
&  & (3.481) &  &  &  \\
Observations & 626 & 626 & 626 & 541 & 477 \\
R-squared & 0.913 & 0.940 & 0.940 & 0.940 & 0.951 \\
\end{talltblr}
\end{table}


\subsection{Placebo Cutoffs}

\Cref{fig:placebo} shows estimates at placebo thresholds (500--2,000). None is significant, confirming the design does not generate false positives.

\begin{figure}[H]
\centering
\includegraphics[width=\textwidth]{figures/fig6_placebo.pdf}
\caption{Placebo Cutoff Tests for Female Employment Rate}
\label{fig:placebo}
\begin{figurenotes}
RDD estimates at placebo and true thresholds. No placebo is significant.
\end{figurenotes}
\end{figure}

\subsection{Pre-Treatment Placebo}

Using 2011 census outcomes, female employment rate shows no discontinuity ($-0.004$, $p = 0.41$), nor does LFPR ($-0.003$, $p = 0.53$). This rules out pre-existing trend differences.

\subsection{Equivalence Tests}

\Cref{tab:equivalence} reports TOST results. With SESOI = 1 pp, all three TOST $p$-values exceed 0.05 (ranging from 0.12 for the gender gap to 0.31 for female employment), meaning the design \textit{cannot} formally establish equivalence---the null of a meaningful effect ($|{\tau}| \geq 1$ pp) is not rejected. The binding constraint is the lower one-sided test: for female employment rate, the 95\% CI lower bound ($-1.8$ pp) extends beyond the $-1$ pp SESOI bound. This reflects the MDE of approximately 1.0--1.5 pp (\Cref{fig:mde}). The design rules out large effects (the 95\% CI for female employment rate excludes positive effects above 0.3 pp) but cannot demonstrate that effects are negligibly small.

\begin{table}[H]
\centering
\caption{Equivalence Tests (TOST): Can We Reject Meaningful Effects?}
\begin{threeparttable}
\begin{tabular}{lccccc}
\toprule
Outcome & Estimate & SE & SESOI & TOST $p$ & Equivalent? \\
\midrule
Female Employment Rate & -0.0074 & (0.0052) & $\pm$0.01 & 0.309 & No \\
Female LFPR & -0.0079 & (0.0040) & $\pm$0.01 & 0.300 & No \\
Gender Employment Gap & 0.0050 & (0.0042) & $\pm$0.01 & 0.117 & No \\
\bottomrule
\end{tabular}
\begin{tablenotes}[flushleft]
\small
\item Notes: Two one-sided test (TOST) procedure. SESOI (smallest effect of scientific interest) set at 1 percentage point (0.01 on the 0--1 scale), following \citet{bertrand2019} and \citet{blaukahn2017}. TOST $p$ = max of the two one-sided $p$-values. ``Equivalent'' = we reject that the true effect exceeds $\pm$1 pp at the 5\%\ level. None of the three outcomes achieve formal equivalence, because the 95\%\ CIs extend beyond $\pm$0.01 (e.g., Female Employment Rate CI lower bound $= -0.018$). The design is powered for effects $\geq$ 1.5 pp but not for formal equivalence at the 1 pp SESOI.
\end{tablenotes}
\end{threeparttable}
\label{tab:equivalence}
\end{table}


\subsection{Minimum Detectable Effects}

\begin{figure}[H]
\centering
\includegraphics[width=\textwidth]{figures/fig9_mde.pdf}
\caption{Minimum Detectable Effects vs.\ Literature Benchmarks}
\label{fig:mde}
\begin{figurenotes}
MDE at 80\% power, $\alpha = 0.05$. Horizontal lines show effects from the India and Norway literatures.
\end{figurenotes}
\end{figure}


%% ═══════════════════════════════════════════════════════════════════════
\section{Mechanisms and Interpretation}\label{sec:mechanisms}
%% ═══════════════════════════════════════════════════════════════════════

The expanded analysis allows a precise diagnosis of \textit{why} mandated parity does not affect women's economic outcomes.

\subsection{The Chain Breaks at Every Link}

The theoretical chain proceeds: more women in office $\rightarrow$ different spending priorities $\rightarrow$ reduced barriers to female employment $\rightarrow$ higher female participation. The results show:

\begin{enumerate}
\item More women in office: \textbf{Yes}. First stage is 2.74 pp ($p < 0.001$).
\item Different spending priorities: \textbf{No}. Social spending does not shift.
\item Female executive leadership: \textbf{No}. Female mayor probability unchanged.
\item Higher female employment: \textbf{No}. All labor outcomes null.
\end{enumerate}

The chain breaks at the second link. Communes above the threshold have more female councillors but do not allocate their budgets differently.

\subsection{Limited Local Fiscal Autonomy}

French communes near the 1,000 threshold have limited operational budgets, dominated by mandatory expenditures (personnel, school maintenance, road upkeep). The key determinants of female labor supply---childcare, parental leave, labor regulation---are set nationally. Even a council with strong gender-sensitive preferences lacks the fiscal room to act on them.

This contrasts with India, where village councils exercise substantial discretion over infrastructure investment---the margin through which \citet{chattopadhyayduflo2004} document effects.

\subsection{High Baseline Female Participation}

The mean female LFPR is 75.5\% and the employment rate is 68.2\%. In rural India, female LFPR was approximately 30\%. When most women who want to work are already working, no policy channel can generate large gains.

\subsection{Treatment Intensity}

The first stage of 2.74 pp is modest compared to Indian reservation studies, where female village head representation shifts from near zero to 100\% \citep{chattopadhyayduflo2004}. The fuzzy RD-IV estimates are imprecise precisely because the first stage is limited.

\subsection{Role Model and Pipeline Effects}

The null on female mayor probability suggests weak role-model effects. In Indian villages, a female \textit{pradhan} is highly visible \citep{beaman2012}. In French communes, individual councillors are largely invisible. The mayor is visible, but parity does not increase the probability of a female mayor.

\subsection{Reconciling with the Developing-Country Evidence}

The results do not contradict the Indian evidence---they delineate its boundary conditions. The mechanisms driving quota effects in India (discretionary spending, role models in settings of extreme inequality, network creation in thin markets) require institutional conditions absent in France. The returns to women's political empowerment appear to diminish with the level of baseline gender equality and local fiscal autonomy.


%% ═══════════════════════════════════════════════════════════════════════
\section{Discussion}\label{sec:discussion}
%% ═══════════════════════════════════════════════════════════════════════

\subsection{Relation to the Existing Literature}

These findings complement several strands of prior work. \citet{ferreiragyourko2014} show null effects of female U.S.\ mayors on policy, using close elections. \citet{baguescampa2020} find limited spillovers of Spanish quotas. \citet{bertrand2019} find that Norwegian board quotas did not ``trickle down'' to women below the C-suite. This paper adds intermediate channels---spending, political pipeline, entrepreneurship---showing the chain breaks everywhere.

\citet{besley2017} find that Swedish gender quotas improve the quality of male politicians by displacing mediocre incumbents; this suggests quotas could affect outcomes through selection rather than preferences. \citet{hessami2020} find female representation shifts childcare spending in German municipalities. It is consistent for spending to shift without affecting employment, but this paper finds that spending does \textit{not} shift---ruling out the intermediate step. \citet{casasarce2015} study Spanish quotas and find candidate selection effects but not policy effects. \citet{brollo2016} find larger effects of female leadership in Brazil, consistent with the development-contingent hypothesis. \citet{iyer2012} find effects of female representation on reported crime in India, operating through a different channel than economic participation.

\subsection{Power and the Informativeness of the Null}

The 95\% CI for female employment rate is approximately $[-0.018, 0.003]$. The MDE at 80\% power is 1.5 pp (2.1\% of the mean). For context, \citet{beaman2012} find 5--7 pp effects on aspirations, which this design would detect. The design is well-powered for developing-country-scale effects but not for the very small effects characteristic of developed-country interventions.

\subsection{Limitations}

First, the compound treatment at 1,000 cannot be fully disentangled. Second, the RP2021 outcome data are a rolling average of 2018--2022 survey cycles; approximately 40\% of observations predate the 2020 election, attenuating any treatment effect. Panel data separating pre- and post-election census vintages would strengthen the design. Third, commune-level outcomes may mask within-commune heterogeneity. Fourth, external validity extends only to communes near the threshold. Fifth, the spending classification uses account codes (nature-based) rather than functional classification, potentially misclassifying some social expenditure, and the net debit measure understates absolute spending levels. Sixth, the fuzzy RD-IV is underpowered due to the modest first stage (2.74 pp), limiting the informativeness of the IV null.


%% ═══════════════════════════════════════════════════════════════════════
\section{Conclusion}\label{sec:conclusion}
%% ═══════════════════════════════════════════════════════════════════════

This paper tests whether mandated gender parity in local government produces substantive changes beyond descriptive representation. Exploiting France's 1,000-inhabitant threshold, I find a strong first stage but precisely estimated null effects across four outcome families: labor markets, political pipeline, municipal spending, and entrepreneurship.

The chain from representation to economic outcomes breaks at every intermediate link. Communes above the threshold do not spend more on social services. They are not more likely to elect a female mayor. Female self-employment does not increase. Labor market outcomes are unchanged. The 3,500 validation confirms that female councillor share converges rapidly once parity is imposed. Fuzzy RD-IV estimates and Holm-corrected inference across seven labor outcomes reinforce the null, though the design lacks power to formally establish equivalence at the 1 percentage point threshold.

These results establish a boundary condition for the developing-country evidence on gender quotas. The mechanisms that drive effects in India---discretionary local spending, role models in settings of extreme inequality---do not operate in France. The returns to women's political empowerment diminish with development.

Mandated political parity remains worth pursuing on normative grounds. But the instrumental case that quotas will improve women's economic outcomes has a limited domain. In developed economies with centralized governance and high baseline equality, closing the remaining gender gap requires labor market reform, not changes to the gender composition of the town hall.

\vspace{1em}

\noindent\textbf{Project Repository:} \url{https://github.com/SocialCatalystLab/ape-papers}

\noindent\textbf{Contributors:} @olafdrw

\noindent\textbf{First Contributor:} \url{https://github.com/olafdrw}

\label{apep_main_text_end}
\newpage
\bibliography{references}

\newpage
\appendix

%% ═══════════════════════════════════════════════════════════════════════
\section{Additional Robustness Tests}\label{app:robustness}
%% ═══════════════════════════════════════════════════════════════════════

\subsection{Fuzzy RD-IV Estimates}

\begin{table}[H]
\centering
\caption{Fuzzy RD-IV: Effect of Female Councillor Share on Labor Outcomes}
\begin{threeparttable}
\begin{tabular}{lccccc}
\toprule
Outcome & IV Estimate & SE & $p$ & BW & $N$ \\
\midrule
Female Employment Rate (IV) & -0.6466 & (0.6524) & 0.272 & 131 & 2,119 \\
Female LFPR (IV) & -0.9143 & (1.0904) & 0.329 & 118 & 1,919 \\
Male Employment Rate (IV) & -0.1251 & (0.2934) & 0.667 & 165 & 2,700 \\
Gender Employment Gap (IV) & 0.2771 & (0.2099) & 0.140 & 191 & 3,136 \\
\bottomrule
\end{tabular}
\begin{tablenotes}[flushleft]
\small
\item Notes: Fuzzy RDD using the 1,000 threshold as an instrument for female councillor share (0--1 scale). Coefficients represent the effect of a 1-unit (i.e., 100 percentage point) increase in the share. To convert to per-1-pp effects, divide by 100 (e.g., $-0.65 / 100 = -0.0065$). CER-optimal bandwidths differ from the reduced-form estimates in Table 2 because \texttt{rdrobust} selects bandwidths separately for each specification. Local linear regression, triangular kernel.
\end{tablenotes}
\end{threeparttable}
\label{tab:fuzzy}
\end{table}


The fuzzy RD-IV is severely underpowered. The first-stage $F$-statistic of approximately 23 exceeds the \citet{stockyogo2005} threshold for 2SLS bias but the small magnitude of the first stage (2.74 pp on a 0--1 scale) inflates IV standard errors. For Female LFPR, the SE (1.09) exceeds the entire range of the outcome variable (0--1), confirming the specification is non-informative. These results are included for methodological completeness; all substantive inference relies on the reduced-form estimates in \Cref{tab:main}.

\subsection{Bandwidth Sensitivity Table}

\begin{table}[H]
\centering
\caption{Bandwidth Sensitivity: Female Employment Rate}
\begin{threeparttable}
\begin{tabular}{lcccc}
\toprule
Bandwidth & Estimate & SE & $N$ & 95\% CI \\
\midrule
100 & -0.0093 & (0.0060) & 1,600 & [-0.0210, 0.0023] \\
150 & -0.0068 & (0.0048) & 2,436 & [-0.0162, 0.0027] \\
200 & -0.0051 & (0.0042) & 3,295 & [-0.0134, 0.0032] \\
300 & -0.0042 & (0.0034) & 5,111 & [-0.0110, 0.0025] \\
400 & -0.0032 & (0.0029) & 7,231 & [-0.0089, 0.0026] \\
500 & -0.0045 & (0.0026) & 9,539 & [-0.0097, 0.0007] \\
600 & -0.0042 & (0.0024) & 12,119 & [-0.0090, 0.0005] \\
800 & -0.0053 & (0.0021) & 20,118 & [-0.0094, -0.0012] \\
\bottomrule
\end{tabular}
\begin{tablenotes}[flushleft]
\small
\item Notes: Local linear regression with HC1 standard errors at fixed bandwidths. Estimates differ from Table 2 because the main results use CER-optimal bandwidth selection with robust bias-corrected standard errors \citep{calonico2014}, while this table uses fixed bandwidths with conventional HC1 inference.
\end{tablenotes}
\end{threeparttable}
\label{tab:bandwidth}
\end{table}


\subsection{Polynomial and Kernel Sensitivity}

The quadratic specification yields $-0.008$ (SE $= 0.007$, $p = 0.21$), compared to the linear baseline of $-0.007$. The uniform kernel gives $-0.008$ ($p = 0.14$) and the Epanechnikov $-0.007$ ($p = 0.15$). The null is robust across specifications.

\subsection{Donut-Hole Estimates}

Excluding communes within $\pm 10$, $\pm 20$, and $\pm 50$ of the cutoff yields $-0.008$ ($p = 0.16$), $-0.010$ ($p = 0.14$), and $-0.009$ ($p = 0.18$), respectively.

\subsection{Heterogeneity by Density}

Splitting by population density yields null effects in both urban ($-0.014$, $p = 0.59$) and rural ($-0.007$, $p = 0.15$) subsamples. The null is not driven by heterogeneity cancellation.

\subsection{Female LFPR RDD}

\begin{figure}[H]
\centering
\includegraphics[width=\textwidth]{figures/fig7_female_lfpr.pdf}
\caption{Female LFPR at the Threshold}
\label{fig:female_lfpr}
\begin{figurenotes}
The borderline significant regression estimate ($p = 0.04$) is driven by a few bins above the cutoff; the visual pattern is continuity.
\end{figurenotes}
\end{figure}


%% ═══════════════════════════════════════════════════════════════════════
\section{Data Appendix}\label{app:data}
%% ═══════════════════════════════════════════════════════════════════════

\subsection{Data Cleaning}

The analysis merges five datasets using the Code Officiel G\'{e}ographique (COG). Commune mergers are handled using the most recent correspondence table. Matching rates exceed 99\% for core datasets and 98.6\% for spending data.

\subsection{Municipal Spending Classification}

The DGFIP Balances Comptables use M14/M57 accounting. I classify:
\begin{itemize}
\item \textit{Social spending:} Accounts 655--657 (grants, social transfers, welfare contributions).
\item \textit{Education:} Accounts 6551, 6553 (school and education grants).
\item \textit{Total operational:} All 6xx accounts (operating expenses).
\end{itemize}

This nature-based classification is conservative, likely underestimating social spending by missing expenditures coded elsewhere.

\subsection{Replication}

All code and data are in the project repository. Scripts run sequentially from \texttt{00\_packages.R} through \texttt{06\_tables.R}. All data are public.


\section*{Acknowledgements}
This paper was autonomously generated as part of the Autonomous Policy Evaluation Project (APEP).

\noindent\textbf{Contributors:} @olafdrw

\noindent\textbf{First Contributor:} \url{https://github.com/olafdrw}

\noindent\textbf{Project Repository:} \url{https://github.com/SocialCatalystLab/ape-papers}

\end{document}
