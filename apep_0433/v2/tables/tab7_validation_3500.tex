\begin{table}[H]
\centering
\caption{Validation: RDD at the 3,500-Inhabitant Threshold}
\begin{threeparttable}
\begin{tabular}{lccccc}
\toprule
Outcome & Estimate & SE & $p$ & BW & $N$ \\
\midrule
Female Share (3500) & 0.0126 & (0.0094) & 0.144 & 225 & 419 \\
Female Employment Rate (3500) & -0.0119 & (0.0113) & 0.265 & 372 & 697 \\
Female LFPR (3500) & -0.0100 & (0.0078) & 0.168 & 341 & 638 \\
Male Employment Rate (3500) & -0.0159 & (0.0108) & 0.115 & 347 & 650 \\
Gender Employment Gap (3500) & -0.0048 & (0.0061) & 0.396 & 363 & 677 \\
Female Mayor (3500) & -0.0778 & (0.0953) & 0.383 & 276 & 504 \\
\bottomrule
\end{tabular}
\begin{tablenotes}[flushleft]
\small
\item Notes: RDD at the 3,500 threshold using communes with population 2,000--5,000. Before 2014, communes above 3,500 used proportional list voting with parity (since 2000); those below used majority voting. After 2014, communes 1,000--3,500 also use PR with parity, so this threshold now separates communes that have had parity since 2000 from those that gained it in 2014. A null first stage at 3,500 supports the interpretation that parity is the binding component of the 1,000 regime change.
\end{tablenotes}
\end{threeparttable}
\label{tab:validation}
\end{table}
