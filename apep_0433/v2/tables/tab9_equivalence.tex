\begin{table}[H]
\centering
\caption{Equivalence Tests (TOST): Can We Reject Meaningful Effects?}
\begin{threeparttable}
\begin{tabular}{lccccc}
\toprule
Outcome & Estimate & SE & SESOI & TOST $p$ & Equivalent? \\
\midrule
Female Employment Rate & -0.0074 & (0.0052) & $\pm$0.01 & 0.309 & No \\
Female LFPR & -0.0079 & (0.0040) & $\pm$0.01 & 0.300 & No \\
Gender Employment Gap & 0.0050 & (0.0042) & $\pm$0.01 & 0.117 & No \\
\bottomrule
\end{tabular}
\begin{tablenotes}[flushleft]
\small
\item Notes: Two one-sided test (TOST) procedure. SESOI (smallest effect of scientific interest) set at 1 percentage point (0.01 on the 0--1 scale), following \citet{bertrand2019} and \citet{blaukahn2017}. TOST $p$ = max of the two one-sided $p$-values. ``Equivalent'' = we reject that the true effect exceeds $\pm$1 pp at the 5\%\ level. None of the three outcomes achieve formal equivalence, because the 95\%\ CIs extend beyond $\pm$0.01 (e.g., Female Employment Rate CI lower bound $= -0.018$). The design is powered for effects $\geq$ 1.5 pp but not for formal equivalence at the 1 pp SESOI.
\end{tablenotes}
\end{threeparttable}
\label{tab:equivalence}
\end{table}
