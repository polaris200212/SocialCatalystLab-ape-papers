\documentclass[12pt]{article}

% UTF-8 encoding and fonts
\usepackage[utf8]{inputenc}
\usepackage[T1]{fontenc}
\usepackage{lmodern}

% Page setup
\usepackage[margin=1in]{geometry}
\usepackage{setspace}
\onehalfspacing

% Typography
\usepackage{microtype}

% Math and symbols
\usepackage{amsmath,amssymb}

% Graphics
\usepackage{graphicx}
\usepackage{float}
\usepackage{subcaption}

% Tables
\usepackage{booktabs}
\usepackage{array}
\usepackage{multirow}
\usepackage{threeparttable}
\usepackage{longtable}
\usepackage{pdflscape}
\usepackage{siunitx}
\sisetup{detect-all=true, group-separator={,}, group-minimum-digits=4}

% Bibliography
\usepackage{natbib}
\bibliographystyle{aer}

% Hyperlinks
\usepackage{hyperref}
\hypersetup{
    colorlinks=true,
    linkcolor=blue,
    citecolor=blue,
    urlcolor=blue
}
\usepackage[nameinlink,noabbrev]{cleveref}

% Captions
\usepackage{caption}
\captionsetup{font=small,labelfont=bf}

% Section formatting
\usepackage{titlesec}
\titleformat{\section}{\large\bfseries}{\thesection.}{0.5em}{}
\titleformat{\subsection}{\normalsize\bfseries}{\thesubsection}{0.5em}{}

% Custom commands
\newcommand{\E}{\mathbb{E}}
\newcommand{\Var}{\text{Var}}
\newcommand{\Cov}{\text{Cov}}
\newcommand{\ind}{\mathbb{I}}
\newcommand{\sym}[1]{\ifmmode^{#1}\else\(^{#1}\)\fi}
\newenvironment{figurenotes}{\par\vspace{0.5em}\small\noindent\textit{Notes:} }{\par}

\title{Parity Without Payoff? Gender Quotas, Public Facilities, and the Channels\\from Representation to Economic Participation in France\thanks{This is a revision of APEP Working Paper apep\_0433 (\url{https://github.com/SocialCatalystLab/ape-papers/tree/main/apep_0433}).}}
\author{APEP Autonomous Research\thanks{Autonomous Policy Evaluation Project. Contributor: \texttt{@olafdrw}. Correspondence: scl@econ.uzh.ch}}
\date{\today}

\begin{document}

\maketitle

\begin{abstract}
\noindent
I exploit France's 1,000-inhabitant threshold---which bundles mandatory gender parity with proportional list voting---in a sharp regression discontinuity design. The regime change increases female councillor share by 2.74 percentage points but produces no detectable effects on female employment, labor force participation, or the gender gap. Testing six proximal channels, I find null effects on the executive pipeline (female mayor, deputy mayor positions), municipal spending composition (social, culture, sports), and public facility provision (childcare, social services, education). Holm-corrected inference across primary labor outcomes and a pre-specified outcome hierarchy reinforce the null. The 3,500-threshold validation confirms rapid convergence of female representation regardless of exposure duration. In developed economies with centralized governance, mandated parity achieves descriptive representation without measurably altering council policies or women's economic outcomes.
\end{abstract}

\vspace{0.5em}
\noindent\textbf{JEL Codes:} J16, D72, J21, H70, H72

\noindent\textbf{Keywords:} gender quotas, political representation, female labor force participation, regression discontinuity, municipal spending, public facilities, France

\newpage

%% ═══════════════════════════════════════════════════════════════════════
\section{Introduction}\label{sec:intro}
%% ═══════════════════════════════════════════════════════════════════════

In Indian villages, reserving council seats for women transformed local public spending, raised aspirations among girls, and shifted norms about female leadership \citep{chattopadhyayduflo2004, beaman2012, pande2003}. These results have shaped global policy: nearly half of all countries now mandate some form of gender quota in political representation \citep{blaukahn2017}. The theoretical logic is compelling---women in office may enact policies favorable to female labor supply, serve as role models, and build economic networks---but the evidence base is overwhelmingly concentrated in developing countries where women's political and economic exclusion was severe.

Do the mechanisms that connect political representation to economic empowerment operate in rich democracies? This question matters for the roughly 50 OECD countries that have adopted or are considering gender quotas in local government. If the transmission channels documented in India require extreme gender inequality to function, then quotas in developed countries may achieve descriptive representation---more women in office---without producing substantive policy changes or downstream economic effects.

This paper tests both the ultimate question (does parity affect female economic outcomes?) and a comprehensive set of intermediate channels (does it change what councils build, who leads them, and how they spend?). I exploit France's 1,000-inhabitant threshold, above which communes must use proportional list voting with strict gender alternation (the ``zipper'' system). This threshold creates a regression discontinuity in the gender composition of municipal councils, the electoral system, and potentially in council behavior and policy outputs.

The institutional setting is unusually informative. France has approximately 35,000 communes, providing a large sample. The running variable---legal population determined by INSEE census figures---cannot be manipulated by communes. And the 2013 law that lowered the threshold from 3,500 to 1,000 generates both a treatment discontinuity (at 1,000) and a validation opportunity (at 3,500, where proportional representation was already in place on both sides).

The 1,000-inhabitant threshold bundles two institutional changes: the switch from majority to proportional list voting, and the imposition of mandatory gender parity. As \citet{eggers2015} documents, proportional representation itself alters electoral competition and turnout; the reduced-form estimand captures this compound treatment. I address the bundling concern through three strategies: testing for PR-specific signatures (council fragmentation), validating at the 3,500 threshold where only exposure duration varies, and estimating a fuzzy RD-IV specification instrumenting for female councillor share.

This paper's central contribution is to expand the set of proximal policy channels tested beyond the prior literature. Where previous studies have examined spending or political selection in isolation, I test \textit{six} outcome families simultaneously, organized into a pre-specified hierarchy that addresses concerns about multiple testing:

\begin{itemize}
\item \textbf{Primary outcomes (Holm-corrected):} Female employment rate, female LFPR, and the gender employment gap.
\item \textbf{Secondary outcomes:} Executive pipeline (female mayor, female share of deputy mayors, female first deputy), spending composition (social, culture, sports, concentration index), and public facility provision (childcare, social services, education, sports per 1,000 inhabitants).
\item \textbf{Exploratory outcomes:} Female self-employment, council size, education spending.
\end{itemize}

The public facility channel is novel. Using the INSEE Base Permanente des \'{E}quipements (BPE), I test whether communes above the parity threshold provide more childcare facilities (cr\`{e}ches municipales), social service centers, or educational infrastructure per capita. These are precisely the types of public goods that the ``different preferences'' hypothesis predicts female-majority councils would prioritize \citep{chattopadhyayduflo2004, clotsfigueras2012}. The executive pipeline analysis goes beyond the female mayor indicator to examine deputy mayor (\textit{adjoint}) positions---the executive posts where real policy influence resides in French local government.

The main findings are as follows:

\textit{First stage.} The regime change increases female councillor share by 2.74 percentage points at the threshold ($p < 0.001$, BW = 200), confirming a strong institutional discontinuity.

\textit{Labor market outcomes.} There are precisely estimated null effects on female employment rate ($-0.74$ pp, $p = 0.14$), female labor force participation ($-0.79$ pp, $p = 0.04$, wrong direction), and the gender employment gap ($+0.50$ pp, $p = 0.21$). All estimates survive Holm correction for multiple testing.

\textit{Executive pipeline.} The regime change does not increase the probability of a female mayor ($+1.6$ pp, $p = 0.72$) and has no effect on the female share of deputy mayors ($+0.2$ pp, $p = 0.99$), though the probability of having a female first deputy is marginally significant ($+7.3$ pp, $p = 0.048$).

\textit{Municipal spending.} Total spending per capita shows a marginally significant positive discontinuity ($+6.1$ EUR, $p = 0.067$), but social spending ($-0.2$ EUR, $p = 0.79$), culture spending ($+0.2$ EUR, $p = 0.25$), and spending concentration (HHI) show no discontinuities. Councils with more women do not allocate budgets differently.

\textit{Public facilities.} Childcare facilities per 1,000 inhabitants ($-0.02$, $p = 0.54$), social service facilities ($+0.01$, $p = 0.97$), and total facilities ($+1.6$, $p = 0.30$) show no discontinuities. Education facilities show a marginally significant positive estimate ($+0.96$ per 1,000, $p = 0.015$), which is not adjusted for multiple comparisons within the secondary family.

\textit{Validation.} At the 3,500 threshold, the null first stage ($+1.26$ pp, $p = 0.14$) confirms rapid convergence of female councillor share once parity is imposed, regardless of exposure duration.

These results contribute to three literatures. First, they extend the test of the political representation--economic empowerment hypothesis to the broadest set of intermediate channels examined in a single study. The chain breaks comprehensively: parity does not shift spending composition, does not create an executive pipeline, does not alter public facility provision, and does not affect employment. This is a stronger null than showing only the absence of labor market effects, because it identifies \textit{where} the chain fails.

Second, the paper speaks to the external validity debate surrounding the Indian evidence. \citet{duflo2012} conjectures that the returns to women's political empowerment diminish with development. \citet{clotsfigueras2012} finds that female legislators in Indian states increase investment in public goods, while \citet{folkerickne2020} find that gender quotas in Sweden improve politician quality without generating policy divergence. This paper provides evidence that in France---where fiscal autonomy is limited and baseline equality is high---the null extends across all proximal channels, consistent with a development-contingent return.

Third, the paper demonstrates rigorous null reporting in an RDD framework: a pre-specified outcome hierarchy with appropriate multiple testing corrections, equivalence tests, minimum detectable effect analysis, and compound treatment validation.



%% ═══════════════════════════════════════════════════════════════════════
\section{Institutional Background}\label{sec:background}
%% ═══════════════════════════════════════════════════════════════════════

\subsection{French Communes and Municipal Government}

France has approximately 35,000 communes, the most municipally fragmented country in Europe. Each commune is governed by a \textit{conseil municipal} whose members are elected for six-year terms. The council elects the mayor (\textit{maire}) from among its members. The mayor appoints deputy mayors (\textit{adjoints}), who hold executive portfolios (finance, urban planning, social affairs, education). Council size varies with population: communes below 100 inhabitants elect 7 councillors, those between 100 and 499 elect 11, and the numbers increase stepwise \citep{codeelectoral}.

Municipal councils decide where to build the village cr\`{e}che, how to maintain the local school, whether to fund a sports complex or a social center---decisions that directly shape working mothers' daily lives. They exercise authority over urban planning, primary school infrastructure (not curriculum or teachers), local roads, water distribution, cultural facilities, and some social services. Fiscal autonomy is constrained: the bulk of revenue comes from national transfers (\textit{dotation globale de fonctionnement}), with limited tax-setting power. Small communes near the 1,000 threshold have limited discretionary spending, with most expenditure committed to mandatory services (personnel, school maintenance, road upkeep). This institutional fact is central to interpreting the spending and facility results.

\subsection{Electoral Rules and the Compound Treatment}

The electoral system depends on commune population, with a threshold that changed in 2013. Before 2014, communes above 3,500 used proportional list voting (\textit{scrutin de liste}), while smaller communes used majority voting (\textit{scrutin plurinominal majoritaire}). The Law of May 17, 2013 (no. 2013-403) lowered this threshold to 1,000, effective for the March 2014 elections.

Above the threshold, elections use proportional list voting with two rounds. Candidate lists must strictly alternate between men and women---the ``zipper'' system (\textit{alternance stricte}). Non-compliant lists are rejected by the prefecture. Seats are allocated proportionally with a majority bonus. Below 1,000, elections use majority voting with no parity requirement.

This institutional design creates a \textit{compound treatment}: crossing the 1,000 threshold triggers both the switch from majority to proportional list voting and the imposition of mandatory gender parity. As \citet{eggers2015} shows, proportional representation itself can affect electoral dynamics. I frame the estimand as the effect of this ``bundled electoral reform'' and pursue three strategies to disentangle the components (detailed in \Cref{sec:compound}).

\subsection{The 2000 Parity Law and the 3,500 Threshold}

The 2014 mandate was an evolution. The Law of June 6, 2000 (no. 2000-493) established parity in list-based elections, initially applying only to communes above 3,500. The 2013 law extended the proportional system---and therefore parity---down to 1,000.

This history creates a useful validation exercise. In the post-2014 regime, all communes above 1,000 use PR with parity, so there is no discrete policy change at 3,500. However, the two sides of 3,500 differ in \textit{exposure duration}: communes above 3,500 have been subject to PR and parity since 2000 (five election cycles by 2020), while those between 1,000 and 3,500 gained both only in 2014 (two cycles). A null first stage at 3,500---showing no difference in female councillor share---is consistent with the parity mandate rapidly achieving its mechanical effect within one or two election cycles.

\subsection{Why France}

France provides an unusually informative test case. The large number of communes yields precise estimates. The running variable cannot be manipulated. French labor law mandates equal pay, prohibits gender discrimination, and provides generous parental leave and subsidized childcare. If political representation were to have additional economic effects beyond what national institutions provide, France would be a best-case scenario. The null is therefore all the more informative.


%% ═══════════════════════════════════════════════════════════════════════
\section{Data}\label{sec:data}
%% ═══════════════════════════════════════════════════════════════════════

\subsection{Data Sources}

The analysis combines seven administrative datasets, all publicly available.

\paragraph{R\'{e}pertoire National des \'{E}lus (RNE).} The 2025 edition contains records for all currently serving municipal councillors, elected in March 2020 (serving the 2020--2026 mandate). The RNE is a stock file reflecting the current state of office-holders; the 2025 vintage captures the council composition resulting from the 2020 election, with only minor attrition from resignations or by-elections during the mandate. For each commune, I compute female councillor share, a female mayor indicator, and---new in this version---the gender composition of the deputy mayor (\textit{adjoint}) team. The RNE records each councillor's function (\textit{fonction}), allowing me to identify deputy mayors and their rank order. I construct: female share of all deputy mayors, whether the first deputy is female, and the female share among the top three deputy mayors. These executive pipeline variables capture whether parity in the council translates to influence in the executive team.

\paragraph{INSEE Recensement de la Population (RP2021).} The 2021 vintage of France's rolling census, covering 2018--2022 survey cycles. Commune-level tabulations of employment status by gender for the population aged 15--64. Outcomes: female employment rate, female LFPR, gender employment gap. Pre-treatment outcomes from the 2011 and 2016 censuses serve as placebos. The 2020 election occurred midway through the survey cycle: approximately 40\% of commune-level observations were collected in 2018--2019 (pre-election) and 60\% in 2020--2022 (post-election). Crucially, the pre-2020 observations were collected under councils elected in 2014, which were \textit{also subject to the same 1,000-inhabitant threshold} (established by the 2013 law). The RP2021 outcome variable therefore captures the cumulative effect of both the 2014 and 2020 mandates. The 40/60 pre/post split for the 2020 election attenuates any \textit{incremental} effect of the second mandate toward zero, making the null results conservative.

\paragraph{INSEE Communes Data.} Legal population (\textit{population l\'{e}gale}), geographic identifiers, and density classification.

\paragraph{DGFIP Balances Comptables des Communes (2019--2022).} Municipal budget data from the Direction G\'{e}n\'{e}rale des Finances Publiques, recording net operational debits by account code. I classify expenses using the M14/M57 nomenclature: accounts 655--657 for social spending, 6574 for culture, 6573 for sports, and 6575--6576 for environment. I compute a spending concentration index (Herfindahl-Hirschman Index across six categories) and the social spending share. Averaged across 2019--2022.

\paragraph{INSEE Base Permanente des \'{E}quipements (BPE 2024).} The BPE inventories all public and private facilities in France, geocoded to the commune level. The 2024 vintage contains 2.8 million facility records classified by type. I aggregate into six domains using the INSEE classification: childcare (cr\`{e}ches, halte-garderies---domain D1), social services (D2--D7), education (B1--B3), sports (F1--F3), health (C1--C6), and culture (A5). Per-capita rates are computed per 1,000 inhabitants. This is the first study to use BPE facility data as an outcome in a gender quota evaluation, testing directly whether more female representation translates to more family-relevant public infrastructure.

\paragraph{Historical Populations.} The running variable is the INSEE \textit{population l\'{e}gale} in force for the 2020 municipal elections, based on the 2017 census cycle. Communes near 1,000 inhabitants have slowly evolving legal populations (the running variable changes only with new census vintages), so the fraction of communes that cross the 1,000 threshold between the 2014 and 2020 elections is small. The design identifies the effect of the \textit{current} assignment to the proportional/parity regime, which for the vast majority of communes near the cutoff has been stable since 2014.

\subsection{Sample Construction}

From 34,955 metropolitan communes, I drop 606 with missing employment or councillor data, yielding 34,349 communes: 24,348 below and 10,001 above the threshold. Municipal spending data match for 33,883 communes (98.6\%). BPE facility data match for 34,346 communes (99.99\%). Deputy mayor data are available for 33,850 communes (98.5\%).

\subsection{Summary Statistics}

\Cref{tab:summary} reports summary statistics. The mean female councillor share is 40.1\% overall: 36.9\% below versus 47.7\% above. The mean female share of deputy mayors is 37.3\%, suggesting some pipeline attrition from council to executive positions. Communes have on average 0.10 childcare facilities per 1,000 inhabitants (0.04 below vs.\ 0.24 above the threshold), reflecting the strong size gradient in public infrastructure.

\textit{A note on sample sizes.} The RDD estimates use CER-optimal bandwidths \citep{cattaneo2020practical}, selected separately for each outcome. Because optimal bandwidths vary (from $h = 98$ for council size to $h = 254$ for male employment rate), the number of communes within the bandwidth differs across tables. This is a feature of the methodology.

\begin{table}[htbp]
\centering
\caption{Summary Statistics: New State vs Parent State Districts}
\label{tab:summary}
\begin{tabular}{lccc}
\hline\hline
 & New State & Parent State & $p$-value \\
\hline
Mean Nightlights & 8862.2 & 15587.7 & 0.000 \\
Mean Log(NL+1) & 8.215 & 9.160 & 0.000 \\
Population (2011, millions) & 1.25 & 2.37 & 0.000 \\
Literacy Rate & 0.583 & 0.556 & 0.071 \\
Ag. Worker Share & 0.362 & 0.434 & 0.001 \\
SC Share & 0.132 & 0.179 & 0.000 \\
ST Share & 0.276 & 0.083 & 0.000 \\
\hline
Districts & 55 & 159 & \\
\hline\hline
\end{tabular}
\begin{minipage}{0.9\textwidth}
\vspace{0.2cm}
\footnotesize \textit{Notes:} Pre-treatment means (1994--1999) for districts in newly created states (Uttarakhand, Jharkhand, Chhattisgarh) vs remaining districts in parent states (UP, Bihar, MP). Nightlights from DMSP calibrated luminosity. Population and sociodemographic characteristics from Census 2011. $p$-values from two-sample $t$-tests of equal means across districts.
\end{minipage}
\end{table}


\subsection{Variable Construction}

All outcome variables are constructed as ratios:
\begin{align}
\text{Female Employment Rate}_c &= \frac{\text{Employed Females}_{c,15\text{--}64}}{\text{Female Population}_{c,15\text{--}64}} \\[4pt]
\text{Female LFPR}_c &= \frac{\text{Active Females}_{c,15\text{--}64}}{\text{Female Population}_{c,15\text{--}64}} \\[4pt]
\text{Gender Employment Gap}_c &= \text{Male Emp Rate}_c - \text{Female Emp Rate}_c \\[4pt]
\text{Spending HHI}_c &= \sum_{k=1}^{6} s_{kc}^2, \quad s_{kc} = \frac{\text{Category } k \text{ spending}_c}{\text{Total spending}_c}
\end{align}

Facility variables are rates per 1,000 inhabitants. Binary indicators (has cr\`{e}che, has social centre) capture the extensive margin.

\subsection{Pre-Specified Outcome Hierarchy}

To address multiple testing concerns, I pre-specify the following hierarchy before estimation:

\textit{Primary family (Holm-corrected):} Female employment rate, female LFPR, and gender employment gap. These are the downstream economic outcomes central to the research question. I apply Holm's sequential procedure controlling the family-wise error rate.

\textit{Secondary families (raw $p$-values, clearly labeled):} Executive pipeline (4 outcomes), spending composition (6 outcomes), and facility provision (5 outcomes plus extensive margin). These are intermediate mechanisms tested individually, with no adjustment across families.

\textit{Exploratory outcomes:} Female self-employment share, council size, education spending. These are hypothesis-generating and reported without formal inference adjustment.


%% ═══════════════════════════════════════════════════════════════════════
\section{Empirical Strategy}\label{sec:method}
%% ═══════════════════════════════════════════════════════════════════════

\subsection{Regression Discontinuity Design}

The identification strategy exploits the sharp discontinuity at the 1,000-inhabitant threshold:
\begin{equation}\label{eq:rdd}
Y_c = \alpha + \tau \cdot \ind\{P_c \geq 1000\} + f(P_c - 1000) + \varepsilon_c
\end{equation}
where $Y_c$ is the outcome for commune $c$, $P_c$ is the legal population, $\ind\{P_c \geq 1000\}$ is the treatment indicator, and $f(\cdot)$ is a flexible function of the centered running variable. The parameter $\tau$ is the discontinuity---the reduced-form effect of the entire 1,000-inhabitant electoral regime change.

Under the continuity assumption \citep{imbenslemieux2008, leelemieux2010}:
\begin{equation}
\tau = \lim_{p \downarrow 1000} \E[Y_c | P_c = p] - \lim_{p \uparrow 1000} \E[Y_c | P_c = p]
\end{equation}

\subsection{Estimation}

I use the robust bias-corrected estimator of \citet{calonico2014}, with a local linear polynomial, triangular kernel, and CER-optimal bandwidth selection \citep{cattaneo2020practical}. The first-stage regression uses a fixed bandwidth of 200 with HC1 standard errors. Mass points in the running variable are addressed using the adjustment procedure in \texttt{rdrobust}.

\subsection{Reporting Conventions}

All rate and share variables are on a 0--1 scale in the regressions and tables (e.g., a coefficient of 0.0274 = 2.74 percentage points). In the text, I convert to percentage points for readability. Spending variables are in euros per capita. Facility variables are counts per 1,000 inhabitants.

\subsection{Addressing the Compound Treatment}\label{sec:compound}

The 1,000 threshold bundles parity with proportional representation. I address this in three ways:

\textit{Political outcome tests.} I test whether outcomes plausibly driven by PR itself---such as council size---show discontinuities distinct from outcomes driven by parity---female executive positions, spending on family services.

\textit{3,500 threshold validation.} Post-2014, both sides of 3,500 use PR with parity, differing only in exposure duration. A null first stage at 3,500 shows that female councillor share converges quickly, consistent with the mechanical nature of the parity mandate rather than a PR effect.

\textit{Fuzzy RD-IV.} The IV specification recovers the LATE of female councillor share, netting out the direct effect of PR (under the exclusion restriction that the threshold affects outcomes only through council gender composition).

\subsection{Equivalence Testing}

I complement standard tests with two one-sided tests (TOST) for equivalence. The smallest effect size of interest (SESOI) is set at 1 percentage point, following \citet{bertrand2019}.

\subsection{Threats to Identification}

\paragraph{Manipulation.} Legal population is determined by INSEE census methodology, not by communes. Verified by McCrary density test (\Cref{sec:robustness}).

\paragraph{Compound treatment.} Addressed through the three strategies above.

\paragraph{Other threshold policies.} No major policy threshold at exactly 1,000 besides the electoral system. Covariate balance tests address residual concerns.


%% ═══════════════════════════════════════════════════════════════════════
\section{Results}\label{sec:results}
%% ═══════════════════════════════════════════════════════════════════════

\subsection{First Stage: The Regime Change Increases Female Representation}

The gender parity mandate sharply increases female councillor share. \Cref{fig:first_stage} shows a clear discontinuity at the 1,000 threshold. At a bandwidth of 200, the discontinuity is 0.0274 (SE $= 0.0057$, $p < 0.001$), or 2.74 percentage points. Below the threshold, the mean female share is approximately 35\%; above, it jumps to near 48\%. The first-stage $F$-statistic ($\approx 23$) exceeds standard weak-instrument thresholds.

\begin{figure}[H]
\centering
\includegraphics[width=\textwidth]{figures/fig1_first_stage.pdf}
\caption{First Stage: Female Councillor Share at the 1,000-Inhabitant Threshold}
\label{fig:first_stage}
\begin{figurenotes}
Binned scatter plot with local linear fits. Each dot is a binned mean (bins of 20 inhabitants). Vertical line marks the threshold. The discontinuity is 2.74 pp ($p < 0.001$).
\end{figurenotes}
\end{figure}

The 2.74 pp first stage is modest compared to Indian reservation studies, where female village head representation shifts from near zero to 100\% \citep{chattopadhyayduflo2004}. Near the cutoff, the female councillor share distribution ranges from approximately 0.30 to 0.52 (10th to 90th percentiles), confirming meaningful variation but limited treatment intensity.

\subsection{Summary of All Outcomes}

Before presenting detailed results, \Cref{fig:multi_outcome} displays RDD estimates for all outcomes organized by family. No outcome in any family shows a robust positive discontinuity. The causal chain from mandated parity to economic outcomes breaks comprehensively.

\begin{figure}[H]
\centering
\includegraphics[width=\textwidth]{figures/fig2_multi_outcome_v3.pdf}
\caption{RDD Estimates Across All Outcome Families}
\label{fig:multi_outcome}
\begin{figurenotes}
Point estimates and 95\% robust bias-corrected CIs from separate RDDs. CER-optimal bandwidths. Primary outcomes (labor) are Holm-corrected; secondary and exploratory outcomes use raw $p$-values. The vertical dashed line marks zero.
\end{figurenotes}
\end{figure}

\subsection{Primary Outcomes: Labor Markets}

The regime change has no detectable effect on any primary labor outcome (\Cref{tab:main}). The female employment rate estimate is $-0.007$ ($p = 0.14$; Holm $p = 0.29$), with a 95\% CI ruling out positive effects larger than 0.3 pp. Female LFPR shows a borderline significant estimate ($-0.008$, $p = 0.04$) in the wrong direction, which loses significance after Holm correction ($p_{\text{Holm}} = 0.12$). The gender employment gap is insignificant ($+0.005$, $p = 0.21$; Holm $p = 0.29$).

\begin{table}[htbp]
\centering
\caption{Main Results: Effect of Energy Community Designation on Clean Energy Investment}
\label{tab:main_results}
\small
\begin{tabular}{lcccc}
\toprule
 & (1) & (2) & (3) & (4) \\
 & Sharp RDD & + Covariates & Quadratic & OLS (BW) \\
\midrule
Energy Community & -5.279 & -8.144 & -6.46 & -4.06 \\
 & (4.098) & (3.333) & (5.235) & (2.344) \\
 & [0.198] & [0.015] & [0.217] & \\
95\% CI & [-13.31, 2.75] & [-14.68, -1.61] & [-16.72, 3.8] & [-8.65, 0.53] \\
\midrule
Polynomial & Linear & Linear & Quadratic & Linear \\
Covariates & No & Yes & No & Yes \\
Bandwidth & 0.069 & 0.071 & 0.09 & 0.069 \\
N (left) & 27 & 28 & 35 & 27 \\
N (right) & 13 & 14 & 16 & 13 \\
\bottomrule
\end{tabular}
\begin{minipage}{0.95\textwidth}
\vspace{0.3em}
\footnotesize
\textit{Notes:} Dependent variable is post-IRA (2023+) clean energy generating capacity in megawatts per 1,000 employees. Columns (1)--(3) report robust bias-corrected estimates from \texttt{rdrobust} with Calonico-Cattaneo-Titiunik optimal bandwidth selection. Column (4) reports OLS within the optimal bandwidth. Standard errors in parentheses; $p$-values in brackets. Covariates include log population, median household income, percent with bachelor's degree, and percent white. Running variable: fossil fuel employment as percent of total employment (2021 CBP). Threshold: 0.17\% (IRA statutory cutoff). Sample: MSAs/non-MSAs with unemployment $\geq$ national average.
\end{minipage}
\end{table}


\Cref{fig:female_emp} displays RDD plots for female employment rate and LFPR. Neither shows a visible discontinuity at the threshold.

\begin{figure}[H]
\centering
\includegraphics[width=\textwidth]{figures/fig3_female_emp.pdf}
\caption{Female Employment Rate and LFPR at the 1,000-Inhabitant Threshold}
\label{fig:female_emp}
\begin{figurenotes}
Binned scatter plots with local linear fits and 95\% CIs. Bins of 25 inhabitants. INSEE RP2021.
\end{figurenotes}
\end{figure}

\subsection{Secondary: Executive Pipeline}

Does parity create a political pipeline to executive positions? \Cref{tab:mechanisms} Panel A reports results for four executive pipeline outcomes. The probability of a female mayor shows no discontinuity ($+1.6$ pp, $p = 0.72$). The female share of all deputy mayors is essentially zero ($+0.2$ pp, $p = 0.99$), as is the female share of top-three deputies ($+0.4$ pp, $p = 0.96$). The one suggestive finding is the probability of having a female \textit{first} deputy ($+7.3$ pp, $p = 0.048$), though this is not adjusted for multiple comparisons within the secondary family.

The mayoralty is the ultimate prize in commune politics---the office where real policy discretion resides. Yet parity does not reach it. Either the marginal 2.74 pp increase in female councillor share is too small to shift mayoral selection, or the factors that determine who becomes mayor---political experience, party networks, incumbency---operate orthogonally to gender composition. \citet{folkerickne2020} find a similar pattern in Sweden: quotas improved politician quality but did not generate policy divergence, suggesting selection effects rather than preference differences drive the impacts of increased representation.

\subsection{Secondary: Spending Composition}

If female councillors bring different policy preferences, the most direct intermediate outcome is spending composition. \Cref{tab:mechanisms} Panel B presents expanded spending results. Social spending per capita shows no discontinuity ($-0.2$ EUR, $p = 0.79$). Culture spending ($+0.2$ EUR, $p = 0.25$) and sports spending ($+0.05$ EUR, $p = 0.75$) are also null. Total spending shows a marginally significant positive estimate ($+6.1$ EUR, $p = 0.067$), possibly reflecting administrative costs of list elections.

The spending concentration index (HHI) provides a test of whether councils diversify their budget priorities. The estimate ($-0.02$, $p = 0.23$) is insignificant---no evidence that parity leads to more diversified spending. The social spending share shows a marginally negative estimate ($-2.3$ pp, $p = 0.06$), suggesting if anything a relative decline.

These results contrast with India, where \citet{chattopadhyayduflo2004} find that female \textit{pradhans} increase drinking water investment by 50--100\%. \citet{hessami2020} find that female representation shifts childcare spending in German municipalities. The difference is consistent with constrained fiscal autonomy: Indian village councils and German municipalities control substantial discretionary budgets, while French communes near 1,000 inhabitants have tightly constrained spending.

\subsection{Secondary: Public Facility Provision}

The novel test in this paper is whether parity affects the stock of public facilities. \Cref{tab:mechanisms} Panel C reports facility outcomes using the BPE. Childcare facilities per 1,000 inhabitants show no discontinuity ($-0.02$, $p = 0.54$). Social service facilities ($+0.01$, $p = 0.97$), sports facilities ($+0.4$, $p = 0.25$), and total facilities ($+1.6$, $p = 0.30$) are all insignificant. The binary indicator for whether the commune has any cr\`{e}che is also null ($-2.6$ pp, $p = 0.24$).

\begin{figure}[H]
\centering
\includegraphics[width=\textwidth]{figures/fig4_childcare_rdd.pdf}
\caption{Childcare Facilities per 1,000 Inhabitants at the Threshold}
\label{fig:childcare}
\begin{figurenotes}
BPE 2024 childcare domain (cr\`{e}ches, halte-garderies). Per-capita rates. No discontinuity visible.
\end{figurenotes}
\end{figure}

Education facilities per 1,000 inhabitants show a marginally significant positive estimate ($+0.96$, $p = 0.015$). This result should be interpreted cautiously: it is a single significant finding among 16 secondary outcomes, is not adjusted for multiple comparisons, and the education domain in the BPE includes facilities like driving schools and language centers that are private-sector services unrelated to council policy.

\begin{figure}[H]
\centering
\includegraphics[width=\textwidth]{figures/fig5_facilities_summary.pdf}
\caption{Public Facility Provision: Coefficient Estimates by Domain}
\label{fig:facilities}
\begin{figurenotes}
Point estimates and 95\% CIs from separate RDDs. BPE 2024. All outcomes are per 1,000 inhabitants except ``Has Cr\`{e}che'' (binary).
\end{figurenotes}
\end{figure}

The facility null is important because the BPE captures the cumulative stock of public infrastructure---it reflects years of policy decisions, not just current budgets. If parity had gradually shifted council priorities toward family-relevant services, this would appear in the facility stock. The null suggests that even over the medium term (two election cycles since 2014), parity does not alter the types of public goods communes provide.

\begin{table}[H]
\centering
\caption{Intermediate Mechanisms and Policy Channels at the 1,000 Threshold}
\begin{threeparttable}
\begin{tabular}{lcccccc}
\toprule
Outcome & Estimate & SE & 95\% CI & $p$ & BW & $N$ \\
\midrule
\multicolumn{7}{l}{\textit{Panel A: Executive Pipeline}} \\
Female Mayor Probability & 0.0157 & (0.0358) & [-0.057, 0.083] & 0.717 & 185 & 3,032 \\
Female Share of Adjoints & 0.0023 & (0.0230) & [-0.045, 0.045] & 0.989 & 91 & 1,419 \\
Female First Adjoint & 0.0726** & (0.0374) & [0.001, 0.147] & 0.048 & 227 & 3,727 \\
Female Share Top-3 Adjoints & 0.0041 & (0.0247) & [-0.047, 0.050] & 0.957 & 100 & 1,566 \\
\midrule
\multicolumn{7}{l}{\textit{Panel B: Spending Composition}} \\
Social Spending PC & -0.1821 & (0.7277) & [-1.617, 1.236] & 0.794 & 226 & 3,737 \\
Culture Spending PC & 0.1729 & (0.1569) & [-0.127, 0.488] & 0.250 & 301 & 5,102 \\
Sports Spending PC & 0.0479 & (0.1680) & [-0.276, 0.383] & 0.750 & 250 & 4,162 \\
Total Spending PC & 6.1287* & (3.5275) & [-0.463, 13.365] & 0.067 & 136 & 2,178 \\
Spending HHI & -0.0315 & (0.0284) & [-0.090, 0.022] & 0.232 & 166 & 2,700 \\
Social Spending Share & -0.0229* & (0.0128) & [-0.049, 0.001] & 0.061 & 201 & 3,278 \\
\midrule
\multicolumn{7}{l}{\textit{Panel C: Public Facility Provision (BPE)}} \\
Childcare Facilities PC & -0.0202 & (0.0349) & [-0.090, 0.047] & 0.542 & 219 & 3,620 \\
Social Facilities PC & 0.0128 & (0.5088) & [-0.977, 1.018] & 0.968 & 193 & 3,175 \\
Sports Facilities PC & 0.3841 & (0.3500) & [-0.285, 1.087] & 0.252 & 144 & 2,328 \\
Education Facilities PC & 0.9645** & (0.4091) & [0.191, 1.795] & 0.015 & 215 & 3,579 \\
Total Facilities PC & 1.6275 & (1.6793) & [-1.563, 5.020] & 0.303 & 151 & 2,436 \\
Has Crèche & -0.0257 & (0.0235) & [-0.074, 0.018] & 0.237 & 216 & 3,591 \\
\bottomrule
\end{tabular}
\begin{tablenotes}[flushleft]
\small
\item Notes: Each row is a separate RDD at the 1,000 threshold. Panel A: Executive positions from RNE (2025). Panel B: Per capita spending (EUR) from DGFIP (2019--2022 average); HHI measures spending concentration across 6 categories. Panel C: Facilities per 1,000 inhabitants from INSEE BPE 2024; ``Has cr\`eche'' is binary. All secondary outcomes; raw $p$-values reported. * $p<0.10$, ** $p<0.05$, *** $p<0.01$.
\end{tablenotes}
\end{threeparttable}
\label{tab:mechanisms}
\end{table}


\subsection{Exploratory Outcomes}

Female self-employment share shows no discontinuity ($+0.2$ pp, $p = 0.79$), ruling out entrepreneurship as a channel. Council size is smooth ($-0.03$, $p = 0.80$), confirming no mechanical threshold effect. Education spending per capita shows a marginally negative but insignificant estimate ($-0.2$ EUR, $p = 0.24$).


%% ═══════════════════════════════════════════════════════════════════════
\section{Validity and Robustness}\label{sec:robustness}
%% ═══════════════════════════════════════════════════════════════════════

\subsection{No Manipulation of the Running Variable}

The McCrary density test yields $T = 0.18$ ($p = 0.86$), confirming no manipulation. \Cref{fig:density} shows the smooth density.

\begin{figure}[H]
\centering
\includegraphics[width=\textwidth]{figures/fig2_density.pdf}
\caption{McCrary Density Test at the 1,000-Inhabitant Threshold}
\label{fig:density}
\begin{figurenotes}
Histogram with McCrary test: $T = 0.18$, $p = 0.86$.
\end{figurenotes}
\end{figure}

\subsection{Covariate Balance}

\Cref{tab:balance} reports RDD estimates for pre-determined covariates from the 2011 census (before the threshold was lowered). All covariates are smooth ($p > 0.4$), confirming local randomization.

\begin{table}[htbp]
\centering
\caption{Covariate Balance at PMGSY Population Threshold}
\label{tab:balance}
\begin{tabular}{lcccc}
\hline\hline
Covariate & Estimate & SE & $p$-value & $N_{\text{eff}}$ \\
\hline
Population (1991) & 1.9582 & (2.3704) & 0.370 & 95,241 \\
lit rate 91 & -0.0003 & (0.0030) & 0.987 & 84,385 \\
Female Share (2001) & 0.0003 & (0.0006) & 0.484 & 65,250 \\
SC Share (2001) & -0.0045 & (0.0031) & 0.136 & 135,836 \\
ST Share (2001) & -0.0013 & (0.0060) & 0.652 & 82,141 \\
\hline\hline
\end{tabular}
\begin{tablenotes}\small
\item \textit{Notes:} Each row reports an RDD estimate of the discontinuity in the pre-determined covariate at the 500 population threshold. MSE-optimal bandwidth, local linear polynomial, triangular kernel. Robust bias-corrected standard errors. No covariate shows a statistically significant discontinuity at conventional levels.
\end{tablenotes}
\end{table}


\subsection{Pre-Treatment Placebo}

Using 2011 census outcomes, female employment rate shows no discontinuity ($-0.004$, $p = 0.41$), nor does LFPR ($-0.003$, $p = 0.53$). The 2016 census (one election cycle after the threshold was lowered) similarly shows nulls for female employment ($-0.008$, $p = 0.14$) and LFPR ($-0.006$, $p = 0.16$). The absence of effects in both the pre-treatment and intermediate-treatment periods rules out pre-existing trends and suggests the null extends to earlier post-treatment vintages.

\subsection{Alternative Specifications}

\Cref{tab:robustness} reports estimates under five specifications: baseline linear, quadratic, uniform kernel, donut hole ($\pm 20$), and department fixed effects. All yield insignificant estimates ($-0.005$ to $-0.010$), confirming the null is robust.

\begin{table}
\centering
\begin{talltblr}[         %% tabularray outer open
caption={Robustness of the Visibility Premium},
note{}={* p \num{< 0.1}, ** p \num{< 0.05}, *** p \num{< 0.01}},
note{ }={Standard errors clustered as indicated in parentheses.},
note{  }={Outcome: annual change in deck condition rating.},
note{   }={All models include state x year FE, material FE, and engineering covariates.},
note{    }={Column (3) restricts to bridges aged 10+ years.},
note{     }={Column (4) excludes bridges with any reconstruction event.},
note{      }={* p < 0.10, ** p < 0.05, *** p < 0.01.},
]                     %% tabularray outer close
{                     %% tabularray inner open
colspec={Q[]Q[]Q[]Q[]Q[]Q[]},
column{2,3,4,5,6}={}{halign=c,},
column{1}={}{halign=l,},
hline{8}={1,2,3,4,5,6}{solid, black, 0.05em},
}                     %% tabularray inner close
\toprule
& Median Split & Top Quartile & Age 10+ & No Reconstruction & County Cluster \\ \midrule %% TinyTableHeader
High Initial ADT & --- & --- & 0.006 & 0.004 & 0.001 \\
& --- & --- & (0.006) & (0.005) & (0.002) \\
Above Median ADT & -0.000 & --- & --- & --- & --- \\
& (0.005) & --- & --- & --- & --- \\
Top Quartile ADT & --- & 0.002 & --- & --- & --- \\
& --- & (0.006) & --- & --- & --- \\
Num.Obs. & 5194414 & 5194414 & 4777000 & 4719893 & 5191291 \\
R2 & 0.025 & 0.025 & 0.023 & 0.023 & 0.025 \\
\bottomrule
\end{talltblr}
\label{tab:robustness}
\end{table}


\subsection{Equivalence Tests}

\Cref{tab:equivalence} reports TOST results with SESOI = 1 pp. All three TOST $p$-values exceed 0.05, meaning the design cannot formally establish equivalence. The binding constraint is the lower one-sided test: the 95\% CI for female employment rate extends to $-1.8$ pp, beyond the SESOI bound. This reflects the MDE of approximately 1.0--1.5 pp. The design rules out large effects but cannot demonstrate that effects are negligibly small.

\begin{table}[H]
\centering
\caption{Equivalence Tests (TOST): Can We Reject Meaningful Effects?}
\begin{threeparttable}
\begin{tabular}{lccccc}
\toprule
Outcome & Estimate & SE & SESOI & TOST $p$ & Equivalent? \\
\midrule
Female Employment Rate & -0.0074 & (0.0052) & $\pm$0.01 & 0.306 & No \\
Female LFPR & -0.0079 & (0.0040) & $\pm$0.01 & 0.294 & No \\
Gender Employment Gap & 0.0050 & (0.0042) & $\pm$0.01 & 0.115 & No \\
\bottomrule
\end{tabular}
\begin{tablenotes}[flushleft]
\small
\item Notes: Two one-sided test (TOST) procedure. SESOI set at 1 percentage point. ``Equivalent'' = we reject that the true effect exceeds $\pm$1 pp at the 5\%\ level.
\end{tablenotes}
\end{threeparttable}
\label{tab:equivalence}
\end{table}


\subsection{Multiple Hypothesis Correction}

For the three primary outcomes, Holm correction yields adjusted $p$-values of 0.29, 0.12, and 0.29 for female employment rate, female LFPR, and the gender gap, respectively. No primary outcome is significant after correction. Secondary outcomes are reported with raw $p$-values as pre-specified; the single marginally significant finding (female first deputy, $p = 0.048$; education facilities, $p = 0.015$) should be interpreted in light of 16 secondary tests.

\subsection{Minimum Detectable Effects}

At 80\% power, the MDE for female employment rate is 1.5 pp (2.1\% of the mean). For childcare facilities per 1,000, the MDE is 0.10. For context, \citet{beaman2012} find 5--7 pp effects on aspirations, which this design would detect. The design is well-powered for developing-country-scale effects but not for very small effects. \Cref{fig:mde} (Appendix) plots MDEs for all outcomes.


%% ═══════════════════════════════════════════════════════════════════════
\section{Mechanisms and Interpretation}\label{sec:mechanisms}
%% ═══════════════════════════════════════════════════════════════════════

The expanded analysis allows a precise diagnosis of \textit{why} mandated parity does not affect women's economic outcomes.

\subsection{The Chain Breaks Comprehensively}

The theoretical chain proceeds: more women in office $\rightarrow$ different spending/infrastructure priorities $\rightarrow$ reduced barriers to female employment $\rightarrow$ higher female participation. The results show:

\begin{enumerate}
\item More women in office: \textbf{Yes}. First stage is 2.74 pp ($p < 0.001$).
\item Different spending priorities: \textbf{No}. Social, culture, and sports spending do not shift. Spending concentration is unchanged.
\item More family-relevant infrastructure: \textbf{No}. Childcare facilities, social service centers, and total facilities show no discontinuity.
\item Female executive leadership: \textbf{No}. Female mayor probability and deputy mayor share are unchanged.
\item Higher female employment: \textbf{No}. All primary labor outcomes are null.
\end{enumerate}

The chain breaks at the second and third links. Communes above the threshold have more female councillors but do not allocate their budgets differently or build different types of public infrastructure.

\subsection{Facility Provision as a Policy Channel}

The facility result deserves particular attention. Cr\`{e}ches municipales are a canonical example of a public good that disproportionately benefits working mothers. In France, childcare is a shared responsibility between communes (which may establish cr\`{e}ches), \textit{d\'{e}partements} (which finance maternal and child protection), and the national government (CAF subsidies). For communes near 1,000 inhabitants, establishing a cr\`{e}che requires substantial investment and ongoing operating costs. Even if more female councillors preferred childcare investment, the fiscal constraints documented above may prevent action. The BPE stock captures the cumulative result of these investment decisions over many years---the null suggests that parity has not shifted priorities even at the margin.

\subsection{Limited Local Fiscal Autonomy}

French communes near the 1,000 threshold have limited operational budgets, dominated by mandatory expenditures. The mean social spending is approximately 6 EUR per capita, while total operational spending averages 52 EUR per capita. The key determinants of female labor supply---childcare, parental leave, labor regulation---are set nationally. Even a council with strong gender-sensitive preferences lacks the fiscal room to act on them. According to OECD data, French sub-national governments control approximately 20\% of total public spending, compared to over 50\% in Scandinavian countries and substantially more in India's \textit{panchayati raj} system \citep{duflo2012}.

\subsection{High Baseline Female Participation}

The mean female LFPR is 75.5\% and the employment rate is 68.2\%. In rural India, female LFPR was approximately 30\% when \citet{chattopadhyayduflo2004} documented effects. The ``different preferences'' channel requires that women in office prioritize policies that address binding constraints on female economic participation. When most women who want to work are already working, and when the remaining barriers (work-family balance, occupational segregation, pay gaps) are national rather than local phenomena, no local policy channel can generate large gains.

\subsection{Treatment Intensity and the Pipeline}

The first stage of 2.74 pp is modest. The null on deputy mayor positions suggests that increased female council membership does not translate to increased female executive power. This is consistent with \citet{lippmann2022}, who documents that in France, women in local politics face persistent barriers to advancement beyond council membership. The ``glass ceiling'' within local government may prevent the marginal female councillors---whose seats are mechanically generated by parity---from accessing the executive positions where real policy discretion resides.

\subsection{Reconciling with the Cross-Country Evidence}

The results do not contradict the Indian evidence---they delineate its boundary conditions. \citet{clotsfigueras2012} finds that female representation in Indian state legislatures increases spending on public health and education; the magnitude of those effects (5--10\%) requires both discretionary budgets and large gender gaps in service provision. \citet{folkerickne2020} find that Swedish quotas improve politician quality---through selection effects---without generating policy divergence, consistent with our spending null. \citet{lippmann2022} documents that French women in local politics face advancement barriers even under parity rules, consistent with our pipeline null.

The mechanisms driving quota effects in India (discretionary spending, role models in settings of extreme inequality, network creation in thin markets) require institutional conditions absent in France. The returns to women's political empowerment appear to diminish with the level of baseline gender equality and local fiscal autonomy.


%% ═══════════════════════════════════════════════════════════════════════
\section{Discussion}\label{sec:discussion}
%% ═══════════════════════════════════════════════════════════════════════

\subsection{Relation to the Existing Literature}

These findings complement several strands of prior work. \citet{ferreiragyourko2014} show null effects of female U.S.\ mayors on policy, using close elections. \citet{baguescampa2020} find limited spillovers of Spanish quotas. \citet{bertrand2019} find that Norwegian board quotas did not ``trickle down'' to women below the C-suite. This paper extends these findings to a broader set of intermediate channels---facility provision, executive pipeline, spending composition---showing the chain breaks comprehensively.

\citet{besley2017} find that Swedish gender quotas improve the quality of male politicians by displacing mediocre incumbents. \citet{hessami2020} find female representation shifts childcare spending in German municipalities. It is consistent for spending to shift without affecting employment, but this paper finds that spending does \textit{not} shift---and neither does facility provision---ruling out the intermediate step entirely. The German result may reflect greater municipal fiscal autonomy in Germany's federal system compared to France's centralized structure.

\subsection{Power and the Informativeness of the Null}

The 95\% CI for female employment rate is approximately $[-0.018, 0.003]$. The MDE at 80\% power is 1.5 pp (2.1\% of the mean). For context, \citet{beaman2012} find 5--7 pp effects on aspirations; such effects would be easily detectable here. The design is well-powered for developing-country-scale effects but not for the very small effects characteristic of developed-country interventions. The MDE for childcare facilities per 1,000 (0.10) represents roughly 1\% of the mean stock, suggesting the design can detect even modest infrastructure effects.

\subsection{Limitations}

First, the compound treatment at 1,000 cannot be fully disentangled, though the three validation strategies provide reassurance. Second, the RP2021 outcome data are a rolling average of 2018--2022 survey cycles; approximately 40\% of observations predate the 2020 election, attenuating any treatment effect. Third, the BPE captures the stock of facilities, not flow---a council that decided to build a cr\`{e}che in 2020 may not see it appear in the 2024 BPE if construction is ongoing. Fourth, commune-level outcomes may mask within-commune heterogeneity. Fifth, the spending classification uses account codes (nature-based) rather than functional classification. Sixth, the fuzzy RD-IV is underpowered due to the modest first stage. Seventh, external validity extends only to communes near the threshold.


%% ═══════════════════════════════════════════════════════════════════════
\section{Conclusion}\label{sec:conclusion}
%% ═══════════════════════════════════════════════════════════════════════

This paper tests whether mandated gender parity in local government produces substantive changes beyond descriptive representation. Exploiting France's 1,000-inhabitant threshold, I find a strong first stage but precisely estimated null effects across six outcome families: labor markets, executive pipeline, municipal spending, public facility provision, entrepreneurship, and council composition.

The chain from representation to economic outcomes breaks comprehensively. Communes above the threshold do not spend more on social services, culture, or sports. They do not provide more childcare or social service facilities. They are not more likely to elect a female mayor or appoint women to deputy mayor positions. Female self-employment does not increase. Labor market outcomes are unchanged. Pre-treatment placebos, the 3,500-threshold validation, and Holm-corrected inference across primary outcomes all reinforce the null.

These results establish a boundary condition for the developing-country evidence on gender quotas. The mechanisms that drive effects in India---discretionary local spending, role models in settings of extreme inequality, control over family-relevant infrastructure---do not operate in France, where fiscal autonomy is limited, national institutions already support female labor supply, and baseline gender gaps in political and economic participation are modest.

Mandated political parity remains worth pursuing on normative grounds. But the instrumental case that quotas will improve women's economic outcomes---by changing what councils build and how they spend---has a limited domain. In developed economies with centralized governance and high baseline equality, closing the remaining gender gap requires labor market reform and national policy, not changes to the gender composition of the town hall.

\vspace{1em}

\noindent\textbf{Project Repository:} \url{https://github.com/SocialCatalystLab/ape-papers}

\noindent\textbf{Contributors:} @olafdrw

\noindent\textbf{First Contributor:} \url{https://github.com/olafdrw}

\label{apep_main_text_end}
\newpage
\bibliography{references}

\newpage
\appendix

%% ═══════════════════════════════════════════════════════════════════════
\section{Additional Results}\label{app:results}
%% ═══════════════════════════════════════════════════════════════════════

\subsection{Fuzzy RD-IV Estimates}

\begin{table}[H]
\centering
\caption{Fuzzy RD-IV: Effect of Female Councillor Share on Labor Outcomes}
\begin{threeparttable}
\begin{tabular}{lccccc}
\toprule
Outcome & IV Estimate & SE & $p$ & BW & $N$ \\
\midrule
Female Employment Rate (IV) & -0.6466 & (0.6524) & 0.272 & 131 & 2,119 \\
Female LFPR (IV) & -0.9143 & (1.0904) & 0.329 & 118 & 1,919 \\
Gender Employment Gap (IV) & 0.2771 & (0.2099) & 0.140 & 191 & 3,136 \\
\bottomrule
\end{tabular}
\begin{tablenotes}[flushleft]
\small
\item Notes: Fuzzy RDD using the threshold as instrument for female councillor share. Local linear regression, triangular kernel, CER-optimal bandwidth.
\end{tablenotes}
\end{threeparttable}
\label{tab:fuzzy}
\end{table}


The fuzzy RD-IV is severely underpowered. The first-stage $F$-statistic of approximately 23 exceeds the \citet{stockyogo2005} threshold for 2SLS bias, but the small first stage (2.74 pp on a 0--1 scale) inflates IV standard errors mechanically. For Female LFPR, the SE (1.09) exceeds the entire 0--1 range of the outcome variable, rendering the confidence interval non-informative. These results are included for methodological completeness; all substantive inference relies on the reduced-form estimates in \Cref{tab:main}.

\subsection{Validation at the 3,500 Threshold}

\begin{table}[H]
\centering
\caption{Validation: RDD at the 3,500-Inhabitant Threshold}
\begin{threeparttable}
\begin{tabular}{lccccc}
\toprule
Outcome & Estimate & SE & $p$ & BW & $N$ \\
\midrule
Female Share (3500) & 0.0126 & (0.0094) & 0.144 & 225 & 419 \\
Female Employment Rate (3500) & -0.0119 & (0.0113) & 0.265 & 372 & 697 \\
Female LFPR (3500) & -0.0100 & (0.0078) & 0.168 & 341 & 638 \\
Gender Employment Gap (3500) & -0.0048 & (0.0061) & 0.396 & 363 & 677 \\
Female Mayor (3500) & -0.0778 & (0.0953) & 0.383 & 276 & 504 \\
\bottomrule
\end{tabular}
\begin{tablenotes}[flushleft]
\small
\item Notes: RDD at the 3,500 threshold using communes 2,000--5,000. Both sides had PR with parity by 2020 (since 2000 above, since 2014 below). A null first stage confirms rapid convergence of female councillor share.
\end{tablenotes}
\end{threeparttable}
\label{tab:validation}
\end{table}


Post-2014, both sides of 3,500 use PR with parity; the difference is exposure duration (since 2000 above, since 2014 below). Female councillor share shows no discontinuity ($+1.26$ pp, $p = 0.14$), confirming rapid convergence. All labor market outcomes are null at 3,500.

\begin{figure}[H]
\centering
\includegraphics[width=\textwidth]{figures/fig10_validation_3500.pdf}
\caption{Validation: Female Councillor Share at the 3,500-Inhabitant Threshold}
\label{fig:validation}
\begin{figurenotes}
Post-2014, both sides use PR with parity. No discontinuity despite 14 years of differential exposure.
\end{figurenotes}
\end{figure}

\subsection{Executive Pipeline RDD}

\begin{figure}[H]
\centering
\includegraphics[width=\textwidth]{figures/fig6_pipeline_rdd.pdf}
\caption{Female Share of Deputy Mayors at the Threshold}
\label{fig:pipeline}
\begin{figurenotes}
RNE 2025. Deputy mayors (\textit{adjoints}) are executive appointments by the mayor. No discontinuity visible.
\end{figurenotes}
\end{figure}


%% ═══════════════════════════════════════════════════════════════════════
\section{Robustness Details}\label{app:robustness}
%% ═══════════════════════════════════════════════════════════════════════

\subsection{Bandwidth Sensitivity}

\input{tables/tab_bandwidth.tex}

\Cref{fig:bw_sensitivity} plots the female employment rate estimate across bandwidths from 100 to 800. The coefficient is stable ($-0.003$ to $-0.009$), consistently including zero.

\begin{figure}[H]
\centering
\includegraphics[width=\textwidth]{figures/fig4_bw_sensitivity.pdf}
\caption{Bandwidth Sensitivity: Female Employment Rate}
\label{fig:bw_sensitivity}
\begin{figurenotes}
Point estimates and 95\% CIs. Local linear regression with HC1 SE.
\end{figurenotes}
\end{figure}

\subsection{Placebo Cutoffs}

\begin{figure}[H]
\centering
\includegraphics[width=\textwidth]{figures/fig6_placebo.pdf}
\caption{Placebo Cutoff Tests for Female Employment Rate}
\label{fig:placebo}
\begin{figurenotes}
RDD estimates at placebo and true thresholds. No placebo is significant.
\end{figurenotes}
\end{figure}

\subsection{Polynomial and Kernel Sensitivity}

The quadratic specification yields $-0.008$ (SE $= 0.007$, $p = 0.21$), compared to the linear baseline of $-0.007$. The uniform kernel gives $-0.008$ ($p = 0.14$) and the Epanechnikov $-0.008$ ($p = 0.14$). The null is robust across specifications.

\subsection{Donut-Hole Estimates}

Excluding communes within $\pm 10$, $\pm 20$, and $\pm 50$ of the cutoff yields $-0.008$ ($p = 0.16$), $-0.010$ ($p = 0.14$), and $-0.009$ ($p = 0.18$), respectively.

\subsection{Heterogeneity by Density}

Splitting by population density yields null effects in both urban ($-0.014$, $p = 0.59$) and rural ($-0.007$, $p = 0.15$) subsamples. The null is not driven by heterogeneity cancellation.

\subsection{Minimum Detectable Effects}

\begin{figure}[H]
\centering
\includegraphics[width=\textwidth]{figures/fig9_mde.pdf}
\caption{Minimum Detectable Effects Across All Outcome Families}
\label{fig:mde}
\begin{figurenotes}
MDE at 80\% power, $\alpha = 0.05$.
\end{figurenotes}
\end{figure}

\subsection{Female Mayor and Spending RDD Plots}

\begin{figure}[H]
\centering
\includegraphics[width=\textwidth]{figures/fig7_female_mayor.pdf}
\caption{Female Mayor Probability at the Threshold}
\label{fig:mayor}
\begin{figurenotes}
The mayor is elected by the council, not mechanically determined by the parity mandate. No discontinuity is visible.
\end{figurenotes}
\end{figure}

\begin{figure}[H]
\centering
\includegraphics[width=\textwidth]{figures/fig8_spending.pdf}
\caption{Social Spending per Capita at the Threshold}
\label{fig:spending}
\begin{figurenotes}
DGFIP accounts 655--657, averaged 2019--2022. No discontinuity visible.
\end{figurenotes}
\end{figure}

\subsection{Female LFPR RDD}

\begin{figure}[H]
\centering
\includegraphics[width=\textwidth]{figures/fig7_female_lfpr.pdf}
\caption{Female LFPR at the Threshold}
\label{fig:female_lfpr}
\begin{figurenotes}
The borderline significant regression estimate ($p = 0.04$) is driven by a few bins above the cutoff; the visual pattern is continuity.
\end{figurenotes}
\end{figure}


%% ═══════════════════════════════════════════════════════════════════════
\section{Data Appendix}\label{app:data}
%% ═══════════════════════════════════════════════════════════════════════

\subsection{Data Cleaning}

The analysis merges seven datasets using the Code Officiel G\'{e}ographique (COG). Commune mergers are handled using the most recent correspondence table. Matching rates exceed 99\% for core datasets, 98.6\% for spending, and 99.99\% for BPE facility data.

\subsection{Municipal Spending Classification}

The DGFIP Balances Comptables use M14/M57 accounting. I classify:
\begin{itemize}
\item \textit{Social spending:} Accounts 655--657 (grants, social transfers, welfare contributions).
\item \textit{Culture:} Account 6574 (cultural grants and subsidies).
\item \textit{Sports:} Account 6573 (sports grants and subsidies).
\item \textit{Environment:} Accounts 6575--6576 (environmental and sustainability spending).
\item \textit{Personnel:} Accounts 641--645 (salaries and social charges).
\item \textit{Total operational:} All 6xx accounts.
\end{itemize}

The spending HHI is computed as $\text{HHI} = \sum_k s_k^2$ where $s_k$ is the share of spending in category $k$. Higher HHI indicates more concentrated spending.

\subsection{BPE Equipment Classification}

The BPE uses a hierarchical classification with domains (A--G) and sub-domains. My mapping:
\begin{itemize}
\item \textit{Childcare:} Sub-domain D1 (cr\`{e}ches collectives, halte-garderies, micro-cr\`{e}ches).
\item \textit{Social services:} Sub-domains D2--D7 (social centers, youth services, elderly care).
\item \textit{Education:} Domain B (schools, training centers, higher education).
\item \textit{Sports:} Domain F (sports facilities, swimming pools, stadiums).
\item \textit{Health:} Domain C (medical facilities, hospitals, pharmacies).
\item \textit{Culture:} Sub-domain A5 (libraries, cinemas, museums, theaters).
\end{itemize}

\subsection{Replication}

All code and data are in the project repository. Scripts run sequentially from \texttt{00\_packages.R} through \texttt{06\_tables.R}. All data are public.


\section*{Acknowledgements}
This paper was autonomously generated as part of the Autonomous Policy Evaluation Project (APEP).

\noindent\textbf{Contributors:} @olafdrw

\noindent\textbf{First Contributor:} \url{https://github.com/olafdrw}

\noindent\textbf{Project Repository:} \url{https://github.com/SocialCatalystLab/ape-papers}

\end{document}
