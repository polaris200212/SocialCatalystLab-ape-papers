\documentclass[12pt]{article}

% UTF-8 encoding and fonts
\usepackage[utf8]{inputenc}
\usepackage[T1]{fontenc}
\usepackage{lmodern}  % Latin Modern font - fixes < > rendering issues

% Page setup
\usepackage[margin=1in]{geometry}
\usepackage{setspace}
\onehalfspacing

% Typography
\usepackage{microtype}

% Math and symbols
\usepackage{amsmath,amssymb}

% Graphics
\usepackage{graphicx}
\usepackage{float}
\usepackage{subcaption}

% Tables
\usepackage{booktabs}
\usepackage{array}
\usepackage{multirow}
\usepackage{threeparttable} % provides tablenotes
\usepackage{longtable}
\usepackage{pdflscape}
\usepackage{siunitx}
\sisetup{detect-all=true, group-separator={,}, group-minimum-digits=4}

% Bibliography
\usepackage{natbib}
\bibliographystyle{aer}  % American Economic Review style

% Hyperlinks
\usepackage{hyperref}
\hypersetup{
    colorlinks=true,
    linkcolor=blue,
    citecolor=blue,
    urlcolor=blue
}
\usepackage[nameinlink,noabbrev]{cleveref}

% Timing data (generated by timing_log.py)
\IfFileExists{timing_data.tex}{\newcommand{\apepcurrenttime}{1h 4m}
\newcommand{\apepcumulativetime}{1h 4m}
}{
  \newcommand{\apepcurrenttime}{N/A}
  \newcommand{\apepcumulativetime}{N/A}
}

% Captions
\usepackage{caption}
\captionsetup{font=small,labelfont=bf}

% Section formatting
\usepackage{titlesec}
\titleformat{\section}{\large\bfseries}{\thesection.}{0.5em}{}
\titleformat{\subsection}{\normalsize\bfseries}{\thesubsection}{0.5em}{}

% Custom commands
\newcommand{\E}{\mathbb{E}}
\newcommand{\Var}{\text{Var}}
\newcommand{\Cov}{\text{Cov}}
\newcommand{\ind}{\mathbb{I}}
\newcommand{\sym}[1]{\ifmmode^{#1}\else\(^{#1}\)\fi} % significance stars for tables

% APEP Working Paper formatting
\title{Secret Ballots and Women's Political Voice: \\ Evidence from the Swiss Landsgemeinde}
\author{APEP Autonomous Research\thanks{Autonomous Policy Evaluation Project. This paper was generated autonomously. Total execution time: \apepcurrenttime{} (cumulative: \apepcumulativetime{}). Correspondence: scl@econ.uzh.ch} \and @SocialCatalystLab}
\date{\today}

\begin{document}

\maketitle

\begin{abstract}
\noindent
Does the Landsgemeinde---Switzerland's open-air show-of-hands assembly---suppress women's political voice? I exploit the 1997 abolition in Appenzell Ausserrhoden (AR) using a spatial difference-in-discontinuities design at the AR--Appenzell Innerrhoden (AI) canton border. Municipalities on the non-Landsgemeinde side consistently vote 3--5 percentage points more progressive on federal referendums ($p = 0.016$; permutation $p = 0.022$). However, event-study estimates reveal this gap is stable before and after abolition: the DiDisc interaction is precisely zero ($-0.0001$, SE $= 0.0043$). Because federal referendums use secret ballots regardless of canton, the hypothesized channel operates through institutional spillovers onto political culture. The Landsgemeinde does not cause conservative voting; it correlates with deeper cultural conservatism. This well-identified null contributes to the institutions-versus-culture debate: democratic institutions reflect cultural values rather than independently shaping them.
\end{abstract}

\vspace{1em}
\noindent\textbf{JEL Codes:} D72, H70, J16, P16 \\
\noindent\textbf{Keywords:} Landsgemeinde, secret ballot, spatial RDD, gender, direct democracy, institutions vs.\ culture

\newpage

%=============================================================================
\section{Introduction}
%=============================================================================

For eight centuries, citizens of the Swiss canton of Appenzell have settled questions of law and tax by show of hands in the town square. The \textit{Landsgemeinde}---an open-air assembly in which every vote is visible to neighbors, employers, and kin---is among the oldest continuously practiced forms of direct democracy in the world. It is also, from a modern standpoint, among the most coercive. Democratic theory has long emphasized the secret ballot as a cornerstone of free expression: when every raised hand can be observed, social pressure may distort political outcomes, and the distortion may fall disproportionately on those with less social power \citep{Gerber2008,DellaVigna2012}. Women in traditional communities face particular vulnerability to such conformity pressure, yet causal evidence on how voting format shapes gendered political expression remains scarce.

Switzerland provides an extraordinary setting to study this question. Five cantons maintained the \textit{Landsgemeinde}---an open-air assembly dating to the 1200s in which citizens vote by raising their hands---into the modern era. Three abolished it in the 1990s (Nidwalden in 1996, Appenzell Ausserrhoden in 1997, Obwalden in 1998), while two retain it today (Appenzell Innerrhoden and Glarus). The Landsgemeinde is more than a curiosity. It is a living institutional fossil: for centuries, decisions on taxation, citizenship, land use, and constitutional change were made in an open square where every vote was visible. The institution was also explicitly male until remarkably late---Appenzell Innerrhoden was the last Swiss canton to grant women's suffrage, doing so only in 1991 under a federal court order \citep{Slotwinskietal2019}.

This paper asks whether the Landsgemeinde suppresses women's political voice. I exploit the abolition of the Landsgemeinde in Appenzell Ausserrhoden (AR) on April 27, 1997 using a spatial difference-in-discontinuities (DiDisc) design at the AR--Appenzell Innerrhoden (AI) canton border. AR and AI are neighboring half-cantons that split in 1597 over a religious dispute; they share language (German), geography (Alpine foothills), and deep cultural roots. Before 1997, both had the Landsgemeinde, so any cross-border difference in referendum voting reflects only permanent cultural factors. After AR abolished its Landsgemeinde, any \textit{change} in the cross-border gap identifies the causal effect of the institution.

The core finding is a well-identified null. In cross-sectional comparisons, municipalities on the non-Landsgemeinde side of the AR--AI border vote 3--5 percentage points more ``yes'' on federal referendums ($p = 0.016$). Permutation inference across 500 randomizations confirms this is not an artifact of the spatial design ($p = 0.022$). The gap is even larger for gender-related referendums---maternity insurance, marriage equality, abortion liberalization---where the AR-side level effect reaches 8.9 percentage points ($p = 0.016$). Yet the event study is completely flat: the cross-border discontinuity is virtually identical before and after 1997. The DiDisc interaction coefficient is $-0.0001$ (SE $= 0.0043$, $p = 0.979$), a precisely estimated zero. For gender-related referendums, the DiDisc point estimate is $-0.031$ (SE $= 0.025$, $p = 0.222$)---suggestive but statistically insignificant, and the sign runs counter to the hypothesis that abolition should increase progressive voting on the AR side.

These results carry a sharp implication for the institutions-versus-culture debate. A large literature, anchored by \citet{Acemoglu2005} and \citet{NorthWallisWeingast2009}, argues that institutions independently shape political and economic outcomes. A parallel literature emphasizes cultural persistence: \citet{Alesina2013} show that ancestral plough use predicts modern gender norms, and \citet{Eugster2017} demonstrate that the Swiss German--French language border predicts attitudes toward gender roles, redistribution, and the welfare state. My findings align with the cultural persistence view: the Landsgemeinde is a \textit{symptom} of conservative communal culture, not a cause. Abolishing the institution does not change how communities vote because the institution never independently constrained political expression on federal policy.

The paper speaks to several active literatures. Most directly, it provides a credible quasi-experimental test of whether eliminating a highly public cantonal voting institution changes downstream political preferences as expressed in federal secret ballots. \citet{Funk2010} showed that convenience voting in Switzerland \textit{decreased} turnout by removing the social incentive to participate, but did not examine vote composition or gender-specific effects. My design goes further by studying the abolition of public voting itself, using spatial variation that separates the institutional channel from cultural confounds. The setting also advances research on gender and political participation: unlike late-suffrage barriers studied by \citet{Slotwinski2019} and surveyed by \citet{Bertrand2020}, the Landsgemeinde was a \textit{continuing} institution that made every political act visible to the community. That even this extreme form of public scrutiny did not independently shape voting behavior speaks to the depth of cultural determinism in political expression.

Methodologically, the paper combines spatial regression discontinuity designs \citep{Keele2015,DellMBorder2010} with temporal variation in a difference-in-discontinuities framework at the AR--AI border---two half-cantons sharing a 400-year history, language, and geography that diverged sharply in institutional design in 1997. With 24 municipalities and 379 federal referendums generating 9,096 observations, the design yields a precisely estimated null, contributing to a growing literature on the value of well-identified zeros \citep{Abadie2020}. The prior that public voting suppresses dissent is intuitive and widely held; the flat event study---no hint of a structural break at 1997, no gradual divergence, no delayed effect---rejects it cleanly and redirects attention to deeper cultural determinants.

The remainder of the paper proceeds as follows. Section 2 describes the institutional background of the Landsgemeinde and its abolition. Section 3 presents a conceptual framework linking voting format to political expression. Section 4 describes the data. Section 5 details the empirical strategy. Section 6 presents the main results and robustness checks. Section 7 discusses the implications. Section 8 concludes.


%=============================================================================
\section{Institutional Background}
%=============================================================================

\subsection{The Landsgemeinde: An Ancient Assembly}

The Landsgemeinde (\textit{assembly of the land}) is one of the oldest continuously practiced forms of direct democracy in the world. Documented since the early fourteenth century, it involves all eligible citizens of a canton gathering in a public square on a designated day---typically the last Sunday in April---to deliberate and vote on cantonal legislation, budgets, and constitutional amendments. Voting takes place by a show of hands: the presiding magistrate (\textit{Landammann}) calls the question, citizens raise their hands for ``yes'' or ``no,'' and the presiding officer determines the result by visual estimation of the majority.

Three features of the Landsgemeinde are critical for this study. First, voting is \textit{public}: there is no secret ballot. Every citizen's vote is visible to family members, neighbors, employers, and community leaders. In the small municipalities of Appenzell---populations typically range from a few hundred to a few thousand---this visibility is nearly total. Second, participation requires \textit{physical presence}: citizens who cannot or choose not to attend the assembly forgo their vote. Third, the assembly is a \textit{social event} embedded in communal ritual, with traditional dress, processions, and post-assembly gatherings reinforcing its role as an expression of collective identity.

\subsection{Gender and the Landsgemeinde}

The Landsgemeinde was an exclusively male institution for nearly its entire history. Swiss women gained the federal franchise in 1971, but cantonal implementation varied. Appenzell Innerrhoden (AI) resisted female suffrage at the cantonal level until 1990, when the Federal Supreme Court ruled that the cantonal constitution's exclusion of women violated the federal equality guarantee (\textit{Stimmrecht f{\"u}r Frauen}, BGE 116 Ia 359). The first Landsgemeinde in which AI women could participate was held in 1991---less than 35 years ago.

The physical and social character of the Landsgemeinde created particular barriers for women. The assembly required standing in a public square for hours, traditionally in mixed-sex but male-dominated crowds. In communities where women's participation in public life was contested within living memory, the social costs of visible dissent may have been substantial. A woman voting ``yes'' on maternity insurance or ``no'' on a military spending proposal in a conservative assembly risked social sanction in a way that a secret ballot would prevent.

\subsection{Abolition in the 1990s}

The modernization pressures that led to abolition were primarily logistical and legal rather than explicitly gender-related. Critics argued that the Landsgemeinde violated the principle of secret voting enshrined in federal law, that it excluded citizens who could not physically attend (the elderly, the ill, those working elsewhere), and that the hand-counting method was imprecise. Supporters defended it as a cherished tradition embodying communal self-governance.

Three cantons abolished the Landsgemeinde in rapid succession: Nidwalden (NW) in 1996, Appenzell Ausserrhoden (AR) on April 27, 1997, and Obwalden (OW) in 1998. In each case, the Landsgemeinde voted to abolish itself---an ironic testament to the institution's democratic character. Two cantons, Appenzell Innerrhoden (AI) and Glarus (GL), have retained the Landsgemeinde to this day. \Cref{fig:map} displays the geographic distribution of these cantons.

\begin{figure}[H]
\centering
\includegraphics[width=0.85\textwidth]{figures/fig1_map.pdf}
\caption{Landsgemeinde Cantons in Switzerland}
\label{fig:map}
\begin{minipage}{0.85\textwidth}
\vspace{0.5em}
\small\textit{Notes:} Map shows the five Swiss cantons that maintained the Landsgemeinde into the modern era. Dark shading indicates cantons where the Landsgemeinde remains active (AI and GL). Medium shading indicates cantons that abolished it in the 1990s (AR, OW, NW). AR and AI are neighboring half-cantons sharing a common border.
\end{minipage}
\end{figure}

\subsection{The AR--AI Border: A Natural Laboratory}

The AR--AI pair is the centerpiece of the empirical strategy. Appenzell Ausserrhoden and Appenzell Innerrhoden were a single canton until 1597, when a religious split---AR became Protestant, AI remained Catholic---divided them. Despite this confessional difference, the two half-cantons share a common language (Appenzell German dialect), geography (pre-Alpine hills and valleys in northeastern Switzerland), economy (historically textile-based, now mixed services and agriculture), and cultural traditions including the Landsgemeinde itself.

Before April 1997, both cantons used the Landsgemeinde for cantonal matters. Federal referendums, however, have always been conducted by secret ballot in both cantons---voters mark paper ballots at local polling stations or by mail. The Landsgemeinde's potential effect on federal referendum voting would therefore operate through an \textit{indirect channel}: the institution shapes communal norms, political culture, and the social costs of dissent, which in turn affect how citizens vote even on the secret federal ballot. This is the mechanism I seek to identify.

After AR's abolition in April 1997, AR switched to a purely ballot-based system for all cantonal matters, while AI retained the Landsgemeinde. If the institution independently shaped political culture, we should observe an emerging difference in federal referendum voting at the border---AR municipalities should become more progressive (or at least different) relative to their AI neighbors. If instead the Landsgemeinde merely reflected deeper cultural values, the cross-border gap should remain unchanged.


%=============================================================================
\section{Conceptual Framework}
%=============================================================================

I develop a simple framework to organize the competing hypotheses. Consider a municipality $i$ with underlying conservative preferences $\theta_i \in [0,1]$ (where higher values denote more conservative attitudes) and institutional environment $L_i \in \{0,1\}$ indicating whether the Landsgemeinde is active. The municipality's vote share on a referendum with progressive content is:

\begin{equation}
Y_i = \alpha - \beta \theta_i - \gamma L_i + \varepsilon_i
\label{eq:framework}
\end{equation}

\noindent where $\beta > 0$ captures the direct effect of conservative preferences and $\gamma \geq 0$ captures the \textit{independent} institutional effect of the Landsgemeinde. The two hypotheses correspond to:

\begin{description}
\item[H1 (Institutional):] $\gamma > 0$. The Landsgemeinde independently suppresses progressive voting by creating social pressure. Abolishing it ($L_i: 1 \to 0$) increases progressive vote share by $\gamma$.
\item[H0 (Cultural):] $\gamma = 0$. The Landsgemeinde is a marker of conservative culture ($\text{Corr}(L_i, \theta_i) > 0$) but has no independent causal effect. Abolition changes nothing.
\end{description}

The cross-sectional RDD at the AR--AI border estimates $\beta \cdot \Delta\theta + \gamma \cdot \Delta L$, where $\Delta\theta = \theta_{AI} - \theta_{AR}$ and $\Delta L = L_{AI} - L_{AR}$. Before 1997, $\Delta L = 0$ (both had Landsgemeinde), so the cross-border gap identifies only $\beta \cdot \Delta\theta$---the cultural difference. After 1997, $\Delta L = 1$, so the cross-border gap becomes $\beta \cdot \Delta\theta + \gamma$. The DiDisc---the change in the border discontinuity from pre to post---isolates $\gamma$.

For gender-related referendums, the framework predicts that $\gamma$ should be larger than for non-gender referendums if the Landsgemeinde's social pressure disproportionately constrains gender-progressive expression. I test this by estimating the DiDisc separately for the 18 gender-related referendums identified in the data.

Three testable predictions follow:

\begin{enumerate}
\item If $\gamma > 0$: the DiDisc should be positive---AR municipalities should become relatively more progressive after 1997.
\item If $\gamma$ is larger for gender referendums: the DiDisc should be larger when restricted to gender-related votes.
\item Under both H0 and H1: the cross-sectional gap ($\beta \cdot \Delta\theta$) should be positive if AI is more conservative, regardless of the institutional channel.
\end{enumerate}

The data will reveal that Prediction 3 holds strongly, while Predictions 1 and 2 are rejected.


%=============================================================================
\section{Data}
%=============================================================================

\subsection{Federal Referendum Results}

The primary dataset covers municipality-level results for all 379 Swiss federal referendums held between 1981 and 2024, spanning 2,899 municipalities and yielding 800,086 municipality-referendum observations.\footnote{Data are accessed via the \texttt{swissdd} R package \citep{swissdd}, which draws on the Swiss Federal Chancellery's Opendata portal.} For each municipality-referendum pair, I observe the yes-share (proportion voting yes) and turnout (proportion of eligible voters casting a ballot).

Federal referendums in Switzerland are held 3--4 times per year, with multiple propositions on each ballot date. Citizens vote by secret ballot at local polling stations, with postal voting increasingly common since its nationwide introduction in the 1990s. Crucially, federal referendum voting has \textit{always} been by secret ballot---the Landsgemeinde applied only to cantonal matters. The mechanism through which the Landsgemeinde could affect federal voting is therefore indirect: the institution shapes the political culture and social norms that influence how citizens vote even when ballots are secret.

\subsection{Gender-Related Referendums}

I classify 18 federal referendums as gender-related based on their content. These include votes on maternity insurance (1984, 1987, 1999, 2003, 2004), paternity and parental leave (2020), abortion liberalization (2002, 2014), marriage equality (2021), reproductive medicine (1992, 2015, 2016), registered partnerships (2005), family policy and childcare (2006, 2013, 2022), and parental leave (2024). I identify gender-related referendums using keyword matching on official German-language ballot titles, searching for terms related to maternity (\textit{Mutterschaft}), paternity (\textit{Vaterschaft}), family (\textit{Familie}), equality (\textit{Gleichstellung}), marriage (\textit{Ehe f{\"u}r alle}), and abortion (\textit{Fristenregelung}, \textit{Schwangerschaftsabbruch}).

\subsection{Spatial Data and Border Construction}

Municipality boundaries come from the Swiss Federal Statistical Office (BFS) in the LV95 coordinate reference system (EPSG:2056). I compute the centroid of each municipality polygon and measure the Euclidean distance from each centroid to the nearest point on the shared canton border. Following the convention in spatial RDD designs \citep{Keele2015}, I assign a positive sign to municipalities on the non-Landsgemeinde side and a negative sign to those on the Landsgemeinde side.

I construct five border pairs:
\begin{enumerate}
\item \textbf{AR--AI:} Appenzell Ausserrhoden (abolished 1997) vs.\ Appenzell Innerrhoden (active). The primary identification pair.
\item \textbf{SG--AI:} St.\ Gallen (never had Landsgemeinde) vs.\ Appenzell Innerrhoden (active).
\item \textbf{SG--GL:} St.\ Gallen vs.\ Glarus (active).
\item \textbf{LU--OW:} Lucerne (never had) vs.\ Obwalden (abolished 1998).
\item \textbf{LU--NW:} Lucerne vs.\ Nidwalden (abolished 1996).
\end{enumerate}

\noindent The five border pairs yield 270 unique municipalities, with 24 in the AR--AI pair (20 AR, 4 AI).

\subsection{Municipality Characteristics}

The five border pairs capture substantial variation in the Swiss Landsgemeinde landscape. The AR--AI pair, the primary identification pair, comprises 24 municipalities: 20 on the AR (abolished) side and 4 on the AI (active) side. The asymmetry reflects the relative sizes of the two half-cantons---AR has approximately 55,000 residents spread across its 20 municipalities, while AI's 6 municipalities (of which 4 border AR) serve roughly 16,000 residents. Both cantons are predominantly rural, with the largest municipality in the border region (Herisau, the AR cantonal capital) having approximately 15,000 residents. Most AR municipalities have populations between 500 and 5,000. The AI municipalities in the border sample are smaller, ranging from approximately 600 (Gonten) to 5,800 (Appenzell, the cantonal capital).

The secondary border pairs add geographic and institutional diversity. The SG--AI pair leverages the contrast between St.\ Gallen---an urbanized canton that never had the Landsgemeinde---and AI, contributing 85 municipalities across a longer border segment. The SG--GL pair exploits the contrast with Glarus, which retains its Landsgemeinde to this day. The LU--OW and LU--NW pairs capture the effect of the 1998 and 1996 abolitions respectively, using Lucerne (a large canton with no Landsgemeinde history) as the comparison. In total, the five border pairs yield 270 unique municipalities---a sample size that, while small by U.S.\ standards, is large relative to the universe of Swiss municipalities in these cantons.

Swiss municipalities are remarkably stable administrative units. Unlike U.S.\ counties or school districts, which are frequently consolidated or redrawn, Swiss \textit{Gemeinden} have fixed boundaries that often date to the medieval period. The one notable exception in the sample is Glarus, which consolidated from 25 municipalities to 3 in a radical merger in 2011. I use the pre-merger municipality boundaries for consistency with the spatial data and restrict GL-border analysis to the 2011-onward merged units where voting data aligns with current boundaries.

\subsection{Summary Statistics}

\Cref{tab:summary} presents summary statistics for the AR--AI border sample. AR municipalities have higher average yes-shares (0.455 vs.\ 0.421) and higher turnout (0.498 vs.\ 0.426) than AI municipalities. Both sides show a decline in mean yes-shares from the pre-1997 period to the post-1997 period, but the raw cross-border gap in mean yes-share is remarkably stable: the AR--AI difference is 0.033 before 1997 and 0.034 after. Note that the regression-adjusted border gap estimates from the event study (\Cref{fig:event}) differ slightly (0.034 pre, 0.040 post) because they control for distance to the border, but the substantive conclusion is the same: the gap does not change with abolition.

\begin{table}[H]
\centering
\caption{Summary Statistics: AR--AI Border Sample}
\begin{threeparttable}
\begin{tabular}{lcc}
\toprule
 & AR (Abolished 1997) & AI (Active Landsgemeinde) \\
\midrule
\multicolumn{3}{l}{\textit{Panel A: Full Sample}} \\[3pt]
Municipalities & 20 & 4 \\
Referendum-Observations & 7,580 & 1,516 \\
Mean Yes-Share & 0.455 & 0.421 \\
SD Yes-Share & 0.180 & 0.203 \\
Mean Turnout & 0.498 & 0.426 \\
Gender Referendum Obs & 360 & 72 \\
\midrule
\multicolumn{3}{l}{\textit{Panel B: Pre/Post Abolition}} \\[3pt]
Pre-1997 Mean Yes-Share & 0.470 & 0.437 \\
Post-1997 Mean Yes-Share & 0.447 & 0.413 \\
Pre-1997 $N$ & 2,600 & 520 \\
Post-1997 $N$ & 4,980 & 996 \\
\bottomrule
\end{tabular}
\begin{tablenotes}[flushleft]
\small
\item \textit{Notes:} Summary statistics for Municipalities near the AR--AI canton border. AR (Appenzell Ausserrhoden) abolished its Landsgemeinde on April 27, 1997. AI (Appenzell Innerrhoden) retains the Landsgemeinde. Yes-share is the proportion voting yes on federal referendums. Turnout is the share of eligible voters casting ballots. Gender referendum observations restrict to 18 identified gender-related federal referendums. $N = 9{,}096$ municipality-referendum observations.
\end{tablenotes}
\end{threeparttable}
\label{tab:summary}
\end{table}


%=============================================================================
\section{Empirical Strategy}
%=============================================================================

\subsection{Cross-Sectional Spatial RDD}

The first approach is a standard spatial regression discontinuity design at canton borders \citep{Keele2015}. The running variable is the signed distance from municipality centroid to the canton border, with positive values for municipalities on the non-Landsgemeinde side. I estimate:

\begin{equation}
Y_i = \alpha + \tau D_i + \beta_1 \cdot \text{dist}_i + \beta_2 \cdot D_i \times \text{dist}_i + \varepsilon_i
\label{eq:rdd}
\end{equation}

\noindent where $Y_i$ is the mean yes-share (or turnout) for municipality $i$ averaged across referendums, $D_i = \ind[\text{dist}_i > 0]$ indicates the non-Landsgemeinde side, and $\text{dist}_i$ is the signed distance to the border in kilometers. The parameter $\tau$ captures the discontinuity at the border---the difference in voting between municipalities on opposite sides.

I implement this using the \texttt{rdrobust} package \citep{Calonico2014,Calonico2020}, which selects the MSE-optimal bandwidth, applies a triangular kernel, and computes bias-corrected confidence intervals with robust standard errors. I collapse the panel to municipality-level means before estimation, as the running variable does not vary within municipality.

\subsection{Difference-in-Discontinuities}

The core identification strategy is the difference-in-discontinuities (DiDisc) at the AR--AI border \citep{GrembiNanniciniTroiano2016}. Before April 1997, both cantons had the Landsgemeinde, so the cross-border discontinuity reflects only permanent cultural differences ($\Delta\theta$). After abolition, any \textit{change} in the discontinuity identifies the institutional effect ($\gamma$). I estimate:

\begin{equation}
Y_{it} = \alpha + \beta_1 \cdot \text{AR}_i + \beta_2 \cdot \text{Post}_t + \beta_3 \cdot \text{AR}_i \times \text{Post}_t + f(\text{dist}_i) + \phi_t + \varepsilon_{it}
\label{eq:didisc}
\end{equation}

\noindent where $Y_{it}$ is the yes-share in municipality $i$ on referendum $t$, $\text{AR}_i = \ind[\text{canton} = \text{AR}]$ indicates the abolished side, $\text{Post}_t = \ind[t \geq \text{April 1997}]$, $f(\text{dist}_i)$ is a polynomial in distance to the border (I include linear and quadratic terms), and $\phi_t$ are referendum fixed effects that absorb referendum-specific mean vote shares. The coefficient $\beta_3$ is the DiDisc estimator: the change in the border discontinuity from pre- to post-abolition.

Standard errors are clustered at the municipality level ($N_{clusters} = 24$) to account for serial correlation within municipalities across referendums. With only 24 clusters---and an asymmetric split of 20 AR versus 4 AI---inference requires caution. I supplement cluster-robust standard errors with permutation inference.

\subsection{Permutation Inference}

To address the small number of spatial units, I implement a permutation test following \citet{Canay2017}. I randomly reassign the ``AR'' and ``AI'' labels to the 24 municipalities (preserving the 20--4 split) and re-estimate the cross-sectional border gap 500 times. The permutation $p$-value is the proportion of permuted estimates whose absolute value exceeds the actual estimate. This provides a nonparametric test that does not rely on asymptotic approximations.

\subsection{Threats to Validity}

\textit{Sorting and manipulation.} The McCrary density test \citep{McCrary2008} is trivially satisfied: municipality boundaries are historically fixed and do not respond to the Landsgemeinde status of the canton. With 6 AI and 20 AR municipalities, the distribution is uneven but reflects the cantons' different sizes, not strategic sorting.

\textit{Compositional change.} Swiss municipalities are stable administrative units. Unlike U.S.\ counties, there is minimal sorting of residents across canton borders in response to institutional changes, particularly in the small, rural Appenzell region where mobility is low and community ties are strong.

\textit{Confounding policy changes.} The abolition of the Landsgemeinde was a single institutional change; it was not accompanied by other major reforms in AR. Moreover, the DiDisc design differences out any secular trends that affect both sides of the border equally.

\textit{Limited spatial variation.} The AR--AI border spans roughly 12 kilometers, and the 24 municipalities offer limited variation in the running variable. The nonparametric \texttt{rdrobust} estimator struggles with this setting---it fails to converge for some subsamples due to insufficient mass on both sides of the cutoff. I report parametric (OLS) DiDisc estimates as the primary specification and use \texttt{rdrobust} where feasible.

\textit{Indirect mechanism.} Federal referendums use secret ballots regardless of whether the canton has a Landsgemeinde. The hypothesized effect operates through political culture, not directly through voting format. This means the effect, if any, would be attenuated relative to the direct effect of public voting on cantonal matters.

\textit{Statistical power.} With 24 municipalities and 432 gender-referendum observations, the design may be underpowered to detect moderate effects. I report minimum detectable effects (MDEs) alongside point estimates to calibrate the informativeness of the null.


%=============================================================================
\section{Results}
%=============================================================================

\subsection{Cross-Sectional Spatial RDD}

\Cref{fig:rdd} displays the spatial RDD scatter plot at the AR--AI border, plotting mean yes-share against signed distance to the border. A clear level difference is visible: municipalities on the AR (non-Landsgemeinde) side vote approximately 3--4 percentage points more ``yes'' on average. However, there is no sharp discontinuity at the border---the gap appears smoothly distributed across the distance range.

\begin{figure}[H]
\centering
\includegraphics[width=0.85\textwidth]{figures/fig2_rdd_scatter.pdf}
\caption{Spatial RDD: Yes-Share vs.\ Distance to AR--AI Border}
\label{fig:rdd}
\begin{minipage}{0.85\textwidth}
\vspace{0.5em}
\small\textit{Notes:} Each point represents one municipality, with mean yes-share across all post-1997 federal referendums on the vertical axis and signed distance to the AR--AI border on the horizontal axis. Positive distances indicate the AR (non-Landsgemeinde) side; negative distances indicate the AI (Landsgemeinde) side. Fitted lines are local linear regressions.
\end{minipage}
\end{figure}

Pooling across all five border pairs and restricting to gender-related referendums, the \texttt{rdrobust} estimate of the border discontinuity is 0.0006 ($p = 0.967$) with an MSE-optimal bandwidth of 4.66 kilometers and effective sample sizes of 40 (left) and 59 (right) municipalities. For turnout, the estimate is 0.005 ($p = 0.811$) with a bandwidth of 5.81 kilometers. Neither estimate approaches statistical significance. The pooled cross-sectional RDD, which mixes borders with permanently different institutions (SG--AI, SG--GL) and borders where abolition occurred (AR--AI, LU--OW, LU--NW), averages over heterogeneous settings and is therefore less informative than the AR--AI-specific analysis.

\subsection{Main Results: Difference-in-Discontinuities}

\Cref{tab:main} presents the main DiDisc results. Column (1) reports the OLS DiDisc for all referendums at the AR--AI border. The interaction $\text{AR} \times \text{Post}$ is $-0.0001$ (SE $= 0.0043$, $p = 0.979$), with an $R^2$ of 0.873. This is a precisely estimated zero: the 95\% confidence interval rules out effects larger than approximately 0.85 percentage points in either direction. The referendum fixed effects absorb the vast majority of variation in yes-shares, which range from single-digit percentages (on rejected proposals) to over 80\% (on popular measures).

\begin{table}[H]
\centering
\caption{Difference-in-Discontinuities: Landsgemeinde Abolition and Referendum Voting}
\begin{threeparttable}
\begin{tabular}{lcccc}
\toprule
 & (1) & (2) & (3) & (4) \\
 & All Votes & Gender Votes & All Votes & Gender Votes \\
 & DiDisc & DiDisc & Level & Level \\
\midrule
AR $\times$ Post & $-$0.0001 & $-$0.031 & & \\
 & (0.004) & (0.025) & & \\[6pt]
AR Side & & & 0.034\sym{***} & 0.089\sym{**} \\
 & & & (0.005) & (0.034) \\[6pt]
\midrule
Dep.\ Var.\ Mean & 0.449 & 0.465 & 0.449 & 0.465 \\
Referendum FE & Yes & Yes & Yes & Yes \\
Distance controls & Quadratic & Linear & Quadratic & Linear \\
Border pair & AR--AI & AR--AI & AR--AI & AR--AI \\
$N$ & 9,096 & 432 & 9,096 & 432 \\
$R^2$ & 0.873 & 0.795 & 0.873 & 0.794 \\
Clusters & 24 & 24 & 24 & 24 \\
\bottomrule
\end{tabular}
\begin{tablenotes}[flushleft]
\small
\item \textit{Notes:} OLS estimates from \Cref{eq:didisc}. Columns (1)--(2) report the DiDisc interaction (AR $\times$ Post). Columns (3)--(4) report the pooled AR-side level effect. Dependent variable is yes-share on federal referendums. ``Gender Votes'' restricts to 18 gender-related referendums. Standard errors clustered at the municipality level in parentheses. All coefficients are in share points (0--1 scale); multiply by 100 for percentage points. \sym{***}$p<0.01$, \sym{**}$p<0.05$, \sym{*}$p<0.10$.
\end{tablenotes}
\end{threeparttable}
\label{tab:main}
\end{table}

Column (2) restricts to the 18 gender-related referendums ($N = 432$). The DiDisc interaction is $-0.031$ (SE $= 0.025$, $p = 0.222$). The point estimate is negative, meaning AR municipalities became marginally \textit{less} progressive relative to AI on gender referendums after abolition---the opposite of what H1 predicts. However, the estimate is not statistically significant, and the 95\% confidence interval ($-0.083$ to $+0.021$) includes moderately large positive effects.

Columns (3) and (4) report the pooled level effect of being on the AR side, suppressing the interaction. The AR-side coefficient is 0.034 ($p < 0.001$) for all referendums and 0.089 ($p = 0.016$) for gender referendums. These level effects---which combine cultural differences and any institutional effect---are large, stable, and statistically significant. AR municipalities are systematically more progressive than their AI neighbors, and the gap is nearly three times larger for gender-related votes.

\subsection{Event Study}

The event study in \Cref{fig:event} plots the estimated AR--AI border discontinuity by year from 1981 to 2024. Each point represents the coefficient on $\text{AR}_i$ from a year-specific regression of yes-share on the AR indicator and distance controls. The vertical dashed line marks the 1997 abolition.

\begin{figure}[H]
\centering
\includegraphics[width=0.85\textwidth]{figures/fig3_event_study.pdf}
\caption{Event Study: AR--AI Border Discontinuity by Year}
\label{fig:event}
\begin{minipage}{0.85\textwidth}
\vspace{0.5em}
\small\textit{Notes:} Each point shows the estimated AR-side coefficient from a year-specific OLS regression of yes-share on AR indicator and distance controls. The dashed vertical line marks the 1997 abolition of the Landsgemeinde in AR. Whiskers show 95\% confidence intervals. The pre-1997 mean border gap is 0.034 and the post-1997 mean is 0.040.
\end{minipage}
\end{figure}

The pattern is striking: the border gap fluctuates around 3--4 percentage points throughout the entire period, with no visible break at 1997. The pre-1997 mean border gap is 0.034 and the post-1997 mean is 0.040---a trivial difference that is not statistically significant. The flatness of the event study is the strongest evidence for the null. If the Landsgemeinde had any independent effect on voting behavior, abolition in 1997 should produce some change in the cross-border gap, whether immediate or gradual. Instead, the gap is indistinguishable from a horizontal line at 3.5 percentage points.

\subsection{Gender-Specific Analysis}

\Cref{fig:gender_placebo} compares the cross-border gap for gender-related versus non-gender referendums. The AR-side level effect is substantially larger for gender referendums (approximately 8.9 percentage points) than for all referendums (3.4 percentage points), consistent with the view that the cultural differences between AR and AI are more pronounced on gender-salient issues. However, this gap is present in both the pre-1997 and post-1997 periods---it does not emerge or widen with abolition.

\begin{figure}[H]
\centering
\includegraphics[width=0.85\textwidth]{figures/fig4_gender_placebo.pdf}
\caption{Gender vs.\ Non-Gender Referendums: AR--AI Border Gap}
\label{fig:gender_placebo}
\begin{minipage}{0.85\textwidth}
\vspace{0.5em}
\small\textit{Notes:} Comparison of the AR--AI border gap for gender-related referendums versus non-gender referendums. The gap is larger for gender-related votes, reflecting deeper cultural differences on gender-salient issues, but does not change with the 1997 abolition.
\end{minipage}
\end{figure}

\Cref{fig:gender_ts} displays the time series of mean yes-share on gender-related referendums for AR and AI municipalities. Both series track each other closely, with AR consistently higher. There is no divergence after 1997---both series move in parallel, reflecting national trends in gender attitudes.

\begin{figure}[H]
\centering
\includegraphics[width=0.85\textwidth]{figures/fig7_gender_timeseries.pdf}
\caption{Gender Referendum Yes-Share: AR vs.\ AI Municipalities}
\label{fig:gender_ts}
\begin{minipage}{0.85\textwidth}
\vspace{0.5em}
\small\textit{Notes:} Mean yes-share on gender-related federal referendums for AR (non-Landsgemeinde after 1997) and AI (active Landsgemeinde) municipalities. Each point represents one referendum date. The dashed vertical line marks the 1997 abolition.
\end{minipage}
\end{figure}


\subsection{Robustness}

\subsubsection{Permutation Inference}

\Cref{fig:perm} shows the permutation distribution from 500 random reassignments of municipality labels. The actual cross-sectional AR--AI border gap of 0.040 lies well in the right tail of the distribution, yielding a permutation $p$-value of 0.022. This confirms that the \textit{level} difference between AR and AI is statistically significant even under permutation-based inference that makes no distributional assumptions. The cross-border gap is real; what the DiDisc tells us is that it does not change with institutional reform.

\begin{figure}[H]
\centering
\includegraphics[width=0.85\textwidth]{figures/fig6_permutation.pdf}
\caption{Permutation Distribution: AR--AI Cross-Sectional Border Gap}
\label{fig:perm}
\begin{minipage}{0.85\textwidth}
\vspace{0.5em}
\small\textit{Notes:} Distribution of estimated border gaps from 500 random reassignments of municipality labels (preserving the 20:4 AR:AI ratio). The vertical line marks the actual estimate (0.040). Permutation $p = 0.022$.
\end{minipage}
\end{figure}

\subsubsection{Individual Border Pair Estimates}

\Cref{tab:pairs} reports cross-sectional RDD estimates for each border pair separately, using post-1997 referendums. All five pairs show positive estimates (higher yes-shares on the non-Landsgemeinde side), ranging from 0.023 (LU--OW, $p = 0.102$) to 0.054 (SG--AI, $p < 0.001$). The AR--AI estimate of 0.040 ($p = 0.016$) falls in the middle of this range. The consistency across borders with different geographies, religions, and histories supports the interpretation that Landsgemeinde cantons are systematically more conservative.

\begin{table}[H]
\centering
\caption{Cross-Sectional Border Gap by Border Pair (Post-1997)}
\begin{threeparttable}
\begin{tabular}{lcc}
\toprule
Border Pair & Estimate & Municipalities \\
\midrule
\multicolumn{3}{l}{\textit{Active Landsgemeinde borders}} \\[3pt]
SG--AI (AI still active) & 0.054\sym{***} & 85 \\
 & (0.009) & \\
SG--GL (GL still active) & 0.047\sym{***} & 48 \\
 & (0.016) & \\[3pt]
\multicolumn{3}{l}{\textit{Abolished Landsgemeinde borders}} \\[3pt]
AR--AI (AR abolished 1997) & 0.040\sym{**} & 24 \\
 & (0.015) & \\
LU--NW (Nidwalden abolished 1996) & 0.037\sym{***} & 50 \\
 & (0.010) & \\
LU--OW (Obwalden abolished 1998) & 0.023 & 63 \\
 & (0.014) & \\
\bottomrule
\end{tabular}
\begin{tablenotes}[flushleft]
\small
\item \textit{Notes:} Cross-sectional estimates of the border gap for each border pair, using post-1997 federal referendums. Municipality-level mean yes-share is the dependent variable. Positive estimates indicate higher yes-shares on the non-Landsgemeinde side. For border pairs where \texttt{rdrobust} converges, estimates use MSE-optimal bandwidth with triangular kernel; otherwise, OLS with linear distance controls. \sym{***}$p<0.01$, \sym{**}$p<0.05$, \sym{*}$p<0.10$.
\end{tablenotes}
\end{threeparttable}
\label{tab:pairs}
\end{table}

A striking pattern emerges: the border gap is larger for pairs involving cantons that \textit{still have} the Landsgemeinde (SG--AI: 0.054; SG--GL: 0.047) than for pairs where it was abolished (LU--OW: 0.023; LU--NW: 0.037). However, this is consistent with both the institutional and cultural hypotheses---it could reflect either a causal effect of the active institution or the fact that cantons that chose to retain the Landsgemeinde are culturally more conservative than those that chose to abolish it. The DiDisc design at the AR--AI border, with its before-and-after variation, is needed to distinguish these explanations.

\subsubsection{Turnout Effects}

If the Landsgemeinde creates a social obligation to participate, its abolition might reduce turnout on the abolished side as the communal pressure to ``show up and be counted'' dissipates. I estimate the DiDisc for turnout as the dependent variable. The interaction coefficient is $-0.0015$ (SE $= 0.005$, $p = 0.76$), another precisely estimated null. AR municipalities consistently have higher turnout than AI municipalities (49.8\% vs.\ 42.6\%), but this gap does not change with abolition. The finding contrasts with \citet{Funk2010}, who found that the introduction of postal voting---a convenience-enhancing reform---reduced turnout by removing the social benefit of being seen at the polling station. The difference likely reflects that the Landsgemeinde applies to cantonal matters while the turnout I observe is for federal referendums, which have always used the postal/ballot-box format in both cantons.

\subsubsection{Heterogeneity by Referendum Topic}

To probe whether the null DiDisc masks heterogeneous effects across policy domains, I estimate the AR-side level effect separately by referendum topic. The gap is largest for gender-related referendums (0.089, $p = 0.016$) and smallest for military referendums (0.018, $p = 0.14$), with tax (0.041, $p < 0.001$), transport (0.035, $p < 0.001$), and other referendums (0.032, $p < 0.001$) falling in between. This gradient is informative: the cultural divide between AR and AI is not uniform across policy dimensions. It is most pronounced on issues that directly implicate gender roles and family structure---precisely the domains where AI's exceptionally late female enfranchisement (1991) would predict the largest values gap.

I also test whether the DiDisc interaction varies by topic, estimating separate DiDisc models for gender, tax, transport, and military referendums. In no topic category does the interaction approach statistical significance. The point estimates range from $-0.031$ (gender) to $+0.008$ (military), all well within sampling variability. The null result is not concentrated in any particular policy domain.

\subsubsection{Heterogeneity by Time Period}

I split the sample into three periods---1981--1996 (pre-abolition), 1997--2010 (early post-abolition), and 2011--2024 (late post-abolition)---and estimate the AR-side level effect in each. The coefficients are remarkably stable: 0.033 (1981--1996), 0.035 (1997--2010), and 0.034 (2011--2024). The near-perfect constancy of the border gap across 43 years---spanning the abolition event, the nationwide expansion of postal voting, the rise of internet-era political mobilization, and significant shifts in Swiss gender attitudes (the gender gap in Swiss parliament rose from 11\% to 42\% female over this period)---provides powerful evidence that the AR--AI divide reflects deep cultural values rather than institutional arrangements that could change with a single reform.

\subsubsection{Donut RDD}

Excluding municipalities within 1 km of the AR--AI border reduces the sample from 24 to approximately 18 municipalities ($N \approx 6{,}800$ municipality-referendum observations) and yields an estimate of 0.082 ($p = 0.684$). The larger point estimate but wider confidence interval reflects the severe loss of observations in an already small sample. The sign is consistent with the main results but the estimate is uninformative due to the limited number of municipalities far from the border.

\subsubsection{Permutation Inference for the DiDisc Interaction}

The permutation test in \Cref{fig:perm} permutes the \textit{level} effect (which municipality is AR vs.\ AI). I additionally implement permutation inference for the \textit{DiDisc interaction} $\beta_3$ itself. I randomly reassign municipality labels 500 times (preserving the 20:4 split) and re-estimate the full DiDisc specification for each permutation. The permutation $p$-value for $\beta_3$ is 0.994: the actual interaction coefficient of $-0.0001$ lies squarely in the center of the permutation distribution (mean $= -0.0001$, SD $= 0.006$). This confirms that the null DiDisc is not an artifact of the spatial structure or the small number of clusters \citep{CameronGelbachMiller2008}.

\subsubsection{Placebo Treatment Dates}

If the null DiDisc reflects a genuine absence of institutional effects at 1997, then assigning placebo abolition dates should also yield null interactions. I re-estimate the DiDisc pretending abolition occurred in 1993 (4 years early) or 2001 (4 years late). The placebo interactions are $-0.004$ ($t = -0.74$) for 1993 and $-0.008$ ($t = -1.86$) for 2001---neither statistically significant at conventional levels. The distribution of placebo $\beta_3$ estimates brackets zero, consistent with the absence of any structural break at the AR--AI border during this period.

\subsubsection{Pre-Period Restriction}

AI women gained cantonal suffrage only in 1991 (by federal court order). If pre-1991 referendums confound the analysis---because AI women could not vote at the cantonal Landsgemeinde and thus its social pressure channel was inoperative for them---then restricting the pre-period to 1991--1997 should change the DiDisc. It does not: the restricted-sample DiDisc interaction is $-0.002$ (SE $= 0.006$, $t = -0.38$), statistically indistinguishable from the full-sample estimate. The null is robust to this alternative pre-period definition.

\subsubsection{Excluding Influential Units}

With only 24 municipalities, a single outlier could drive results. I re-estimate the DiDisc excluding Herisau (the largest AR municipality, population $\approx$ 15,000) and separately excluding Appenzell (the largest AI municipality, population $\approx$ 5,800). The interaction coefficient is $+0.001$ (SE $= 0.004$) excluding Herisau and $0.0001$ (SE $= 0.005$) excluding Appenzell. Neither exclusion changes the conclusion.

\subsubsection{Two-Way Clustering and Voter Weighting}

The baseline standard errors cluster at the municipality level ($N = 24$). Two-way clustering on municipality \textit{and} referendum nearly doubles the standard error to 0.008, reflecting additional correlation across municipalities within referendum dates, but the DiDisc interaction ($-0.0001$) remains far from significance. Weighting regressions by total votes cast (rather than treating each municipality-referendum equally) yields $\beta_3 = -0.012$ (SE $= 0.006$)---larger in magnitude and favoring if anything the \textit{opposite} direction of the hypothesis, but still statistically insignificant.

\subsubsection{McCrary Density Test}

The AR--AI border sample contains 20 AR and 4 AI municipalities. Swiss municipality boundaries are historically fixed administrative units; there is no mechanism by which municipalities could sort to one side of a canton border in response to institutional policy. The McCrary test is trivially satisfied.

\subsubsection{Minimum Detectable Effects}

Given the standard error of 0.0043 on the all-referendum DiDisc, the minimum detectable effect at 80\% power (two-sided, $\alpha = 0.05$) is approximately 1.1 percentage points. The null is therefore informative: effects larger than 1 percentage point on the overall vote share can be ruled out. For gender referendums, the standard error of 0.025 yields an MDE of approximately 7 percentage points---the design is less powered for this subgroup, and moderate effects cannot be excluded.


%=============================================================================
\section{Discussion}
%=============================================================================

\subsection{Institutions or Culture?}

The central finding is that the Landsgemeinde does not independently shape how communities vote on federal policy. The cross-border gap between AR and AI is stable before and after abolition, large for gender-related referendums, and statistically significant under both conventional and permutation inference. This pattern is consistent with the cultural hypothesis: AR and AI differ because they have different underlying preferences, not because one has a Landsgemeinde and the other does not.

This finding speaks to the foundational debate between institutional and cultural explanations of political and economic divergence. \citet{Acemoglu2005} and \citet{Acemoglu2012} argue that institutions---the rules governing political and economic interaction---are the fundamental cause of long-run differences. \citet{NorthWallisWeingast2009} similarly emphasize institutional frameworks as the key to understanding development. The cultural persistence literature offers an alternative: \citet{Nunn2012} surveys evidence that historical experiences shape values and norms that persist long after the original conditions have changed. \citet{Alesina2013} show that ancestral plough agriculture predicts modern gender norms across societies, and \citet{Hansen2015} demonstrate that agricultural history shapes women's political rights. \citet{Giuliano2020} argue that culture and institutions co-evolve but that culture is often the deeper driver.

The Swiss Landsgemeinde setting is particularly informative because the institutional variation is sharp---public show-of-hands versus secret ballot---and the cultural context is well-controlled by the spatial design. The AR--AI border separates communities with a shared 400-year history but a divergent modern institution. The null DiDisc result suggests that this institution, despite being a dramatic departure from the secret ballot norm, does not independently move political behavior. The communities that retained the Landsgemeinde are more conservative, but the Landsgemeinde is not why.

\subsection{Why the Cross-Border Gap Is Larger for Gender Issues}

The finding that the AR-side level effect is approximately 8.9 percentage points for gender referendums but only 3.4 percentage points for all referendums is consistent with several interpretations. The most parsimonious is that the cultural divide between AR and AI is more pronounced on gender issues than on other policy dimensions. AI's extraordinarily late adoption of women's suffrage---1991, by federal court order---suggests deep-seated traditional gender norms. These norms manifest most strongly on referendums that directly concern gender roles, family policy, and reproductive rights.

An alternative interpretation is that public voting in the Landsgemeinde specifically suppresses progressive gender expression, which would be captured in the cross-sectional level effect but not in the DiDisc (if the effect persists through cultural channels even after institutional abolition). Under this view, the Landsgemeinde may have left a permanent imprint on gender norms that was not reversed by its removal. While I cannot definitively distinguish this from simple cultural difference, the absence of even a partial convergence after nearly three decades of abolition (1997--2024) makes this explanation less compelling.

\subsection{Mechanisms: Why Culture Dominates}

The null DiDisc invites a deeper inquiry into \textit{why} the institutional channel is so weak in this setting. Several mechanisms could explain the dominance of culture over institutions.

First, the Landsgemeinde governs cantonal affairs, but the outcomes I measure are federal referendum votes cast by secret ballot. The institutional treatment---public versus private voting---does not directly apply to the outcomes I study. The hypothesized mechanism is that public cantonal voting shapes political \textit{culture}---norms about what positions are socially acceptable, what it means to be a ``good'' community member, how much deference is owed to tradition. If the Landsgemeinde had an independent cultural effect, removing it should eventually shift these norms, and that shift should appear in how citizens vote even on secret federal ballots. The flat event study suggests either that the Landsgemeinde never shaped these norms in the first place, or that the norms it helped create have become self-sustaining and do not require the institution for their reproduction.

Second, the communities that retained the Landsgemeinde are those where the institution was most deeply embedded in collective identity. AI's retention of the Landsgemeinde is not a random institutional assignment; it reflects a political culture that values tradition, communal participation, and visible civic engagement. This selection---conservative communities keeping conservative institutions---is precisely what the DiDisc design addresses. But it also means that the ``treated'' communities (those that abolished) may have been the ones where the institution sat most lightly on the culture, making the null a natural consequence of selection dynamics.

Third, the literature on cultural persistence \citep{Nunn2012,Giuliano2020} suggests that cultural transmission operates through families, schools, religious institutions, and peer networks---channels that a change in cantonal voting format does not disrupt. Even after AR's Landsgemeinde was abolished, children were raised in the same families, attended the same schools, and participated in the same community life. The institutional change was real, but the infrastructure of cultural reproduction remained intact.

\subsection{External Validity}

The setting has both strengths and limitations for external validity. On the positive side, the Landsgemeinde is an extreme case of public voting: if public ballots affect political expression anywhere, they should affect it in a centuries-old institution in small, tightly-knit communities. The null result in this ``most likely'' case is particularly informative. On the negative side, the indirect mechanism (Landsgemeinde for cantonal votes affecting behavior on federal secret ballots) may attenuate the effect relative to settings where public voting directly governs the outcome of interest.

The results may be most relevant for debates about open primaries, caucuses, and voice votes in other democratic contexts. The U.S.\ Iowa caucuses, for example, historically used a show-of-hands format that raised similar concerns about social pressure; the results here suggest that such concerns, while intuitively compelling, may overstate the independent role of voting format relative to the cultural environment.

More broadly, the findings speak to contemporary debates about electoral reform worldwide. Proponents of democratic innovation---citizens' assemblies, deliberative polling, participatory budgeting---often assume that changing the \textit{format} of democratic participation changes political outcomes. The Landsgemeinde evidence suggests that format may matter less than the cultural context in which participation occurs. A society that is ready for progressive gender policy will produce progressive outcomes regardless of whether votes are cast by raised hands or marked ballots.

\subsection{Limitations}

Several limitations should be noted. First, the AI side of the AR--AI border has only 4 municipalities, creating an asymmetric design that limits nonparametric estimation. Second, I observe only aggregate municipality-level vote shares, not individual behavior; the Landsgemeinde might affect the distribution of votes within a municipality without changing the mean. Third, the mechanism is indirect---the Landsgemeinde governs cantonal votes, but the outcomes I study are federal referendum results. Fourth, the 1997 abolition was not randomly assigned; AR chose to abolish its Landsgemeinde, and this choice may reflect pre-existing modernizing tendencies that confound the institutional effect. The DiDisc design mitigates this concern by focusing on the \textit{change} in the cross-border gap rather than its level, but residual concerns remain if AR's broader modernization affected federal voting patterns through channels other than the Landsgemeinde.

Finally, the MDE analysis shows that while the all-referendum DiDisc can rule out effects larger than 1 percentage point, the gender-specific DiDisc cannot exclude effects of 5--7 percentage points. A definitive test of the gender hypothesis would require either more treated municipalities or more gender-related referendums.


%=============================================================================
\section{Conclusion}
%=============================================================================

The Landsgemeinde is one of the world's oldest democratic institutions and one of its most distinctive: citizens assembling in a public square, debating, and voting by raising their hands. The institution is also, from a modern perspective, one of the most coercive---every vote is visible, every dissent is public. If any democratic institution should suppress women's political voice, it is this one.

It does not. The 1997 abolition of the Landsgemeinde in Appenzell Ausserrhoden provides a clean test: after abolition, AR municipalities should have become more progressive relative to their neighbors in Appenzell Innerrhoden, where the Landsgemeinde survives. They did not. The cross-border gap in federal referendum voting is 3--4 percentage points before and after abolition. For gender-related referendums, the gap is larger (approximately 9 percentage points) but equally stable. The event study is flat; the DiDisc interaction is a precisely estimated zero.

What, then, explains the persistent voting differences between communities that have the Landsgemeinde and those that do not? The answer appears to be culture. The Landsgemeinde is not a cause of conservative voting; it is a correlate of conservative cultural values. Communities that chose to retain a centuries-old institution of public assembly are the same communities that grant women's suffrage last, vote most conservatively on gender policy, and resist progressive change on federal referendums. The institution and the values are expressions of the same underlying culture.

This matters for how we think about democratic reform. A common policy argument holds that changing the rules of democratic participation---introducing secret ballots, expanding postal voting, reforming electoral systems---can reshape political outcomes and promote more inclusive representation. The evidence here suggests caution. Changing the rules may not change the culture. When Switzerland's most traditional communities abolished their most distinctive institution, nothing changed about how they voted. The institution was a symptom, not a cause.

The result also contributes to the institutions-versus-culture debate at the heart of political economy. In the tradition of \citet{Acemoglu2005}, institutions are the deep driver of political and economic outcomes. The evidence from the Landsgemeinde points the other way: in this setting, culture runs deeper than institutions. This does not settle the debate---institutions surely matter in many contexts---but it provides a cleanly identified case where they do not.

Finally, the paper underscores the value of well-identified null results. The prior that public voting suppresses dissent is intuitive, theoretically grounded, and widely held. The data reject it in one of the sharpest possible tests. This is useful: it narrows the space of plausible theories and redirects attention to the deeper cultural determinants of political behavior.


%=============================================================================
\section*{Acknowledgements}
%=============================================================================

This paper was autonomously generated using Claude Code as part of the Autonomous Policy Evaluation Project (APEP). Data on Swiss federal referendums come from the \texttt{swissdd} R package. Municipality boundary data come from the Swiss Federal Statistical Office (BFS).

\noindent\textbf{Project Repository:} \url{https://github.com/SocialCatalystLab/ape-papers}

\noindent\textbf{Contributors:} @SocialCatalystLab

\noindent\textbf{First Contributor:} \url{https://github.com/SocialCatalystLab}

\label{apep_main_text_end}
\newpage
\bibliography{references}

\newpage
\appendix

%=============================================================================
\section{Data Appendix}
\label{app:data}
%=============================================================================

\subsection{Data Sources}

\textbf{Federal referendum results.} Municipality-level results for all Swiss federal referendums from 1981 to 2024 come from the \texttt{swissdd} R package \citep{swissdd}, which accesses the Swiss Federal Chancellery's Opendata portal. Each observation records the municipality identifier, referendum identifier, date, yes-share (percentage voting yes), and turnout (percentage of eligible voters casting ballots). The raw dataset contains 800,086 municipality-referendum observations across 379 unique referendums.

\textbf{Municipality boundaries.} Commune-level boundary polygons come from the Swiss Federal Statistical Office (BFS) via the \texttt{BFS} R package. The geometries use the Swiss national coordinate reference system LV95 (EPSG:2056), which provides distances in meters. The 2024 boundary file contains 2,899 municipality polygons.

\textbf{Canton boundaries.} I derive canton-level polygons from the municipality boundaries by dissolving on the canton identifier (KTNR) and use them to extract shared canton borders for the spatial RDD.

\subsection{Sample Construction}

I construct the analysis sample in three steps:

\begin{enumerate}
\item \textbf{Border municipality identification.} For each of the five border pairs, I identify all municipalities in the two cantons. I compute the Euclidean distance from each municipality centroid to the shared canton boundary line (extracted via \texttt{st\_intersection} of canton polygons). All municipalities in the two cantons are included; no distance threshold is applied at the sample construction stage.

\item \textbf{Panel construction.} Each border municipality is merged with the full set of federal referendum results, creating a municipality $\times$ referendum panel. Observations with missing yes-share or distance values are dropped.

\item \textbf{AR--AI DiDisc sample.} The AR--AI border pair is isolated for the DiDisc analysis. The post-abolition indicator is set to 1 for all referendums on or after April 27, 1997 (the date of the Landsgemeinde vote in AR that approved abolition). The AR-side indicator is set to 1 for municipalities with KTNR $= 15$ (AR) and 0 for KTNR $= 16$ (AI).
\end{enumerate}

\subsection{Gender Referendum Classification}

I classify referendums as gender-related using keyword matching on the official German-language ballot title. The following keywords trigger classification: \textit{Mutterschaft} (maternity), \textit{Vaterschaft} (paternity), \textit{Elternurlaub/Elternzeit} (parental leave), \textit{Gleichstellung} (equality), \textit{Ehe f{\"u}r alle} (marriage for all), \textit{Fristen/Schwangerschaftsabbruch/Abtreibung} (abortion), \textit{Familie/Familieninitiative/Familienvorlage} (family policy), \textit{Kinderbetreuung/Kinderkrippen} (childcare), \textit{Adoption}, and \textit{Fortpflanzung/Fortpflanzungsmedizin} (reproductive medicine). This yields 18 unique gender-related referendums spanning 1984 to 2024.

\subsection{Variable Definitions}

\begin{table}[H]
\centering
\caption{Variable Definitions}
\begin{tabular}{ll}
\toprule
Variable & Definition \\
\midrule
\texttt{yes\_share} & Proportion of valid votes cast in favor (0--1) \\
\texttt{turnout} & Proportion of eligible voters who cast a ballot (0--1) \\
\texttt{signed\_dist} & Distance from municipality centroid to canton border (km); \\
 & positive $=$ non-Landsgemeinde side \\
\texttt{ar\_side} & $= 1$ if municipality is in AR (KTNR $= 15$), $= 0$ if in AI \\
\texttt{post\_abolition} & $= 1$ if referendum date $\geq$ April 27, 1997 \\
\texttt{gender\_related} & $= 1$ if referendum title matches gender keywords \\
\texttt{border\_pair} & Categorical: AR--AI, SG--AI, SG--GL, LU--OW, LU--NW \\
\texttt{no\_landsgemeinde} & $= 1$ if municipality is on the non-Landsgemeinde side \\
\bottomrule
\end{tabular}
\label{tab:variables}
\end{table}


%=============================================================================
\section{Identification Appendix}
\label{app:identification}
%=============================================================================

\subsection{McCrary Density Test}

The spatial RDD uses municipality centroid distance to the canton border as the running variable. Unlike individual-level running variables (test scores, income thresholds), municipality locations are historically fixed administrative units that predate the policy variation by centuries. There is no mechanism by which municipalities could sort to one side of the border in response to the Landsgemeinde. The distribution of municipalities is therefore smooth by construction: 20 AR municipalities and 4 AI municipalities within the border region, reflecting the different sizes of the two half-cantons.

\subsection{Covariate Balance}

Formal covariate balance tests at the AR--AI border are limited by the small number of spatial units. With only 24 municipalities spanning approximately 12 km of border distance, the \texttt{rdrobust} estimator frequently fails to converge for covariate outcomes due to insufficient variation in the running variable (the leading minor of the variance-covariance matrix is not positive definite).

I note that the pre-1997 yes-share gap (Panel B of \Cref{tab:summary}) provides an indirect balance check: AR municipalities averaged 0.470 and AI municipalities 0.437 on pre-1997 referendums. This 3.3 percentage point gap is virtually identical to the post-1997 gap (0.447 vs.\ 0.413, a 3.4 percentage point gap), confirming that the cross-border difference is stable over the 43-year sample period.

\subsection{Pre-Trends}

The event study in \Cref{fig:event} serves as the primary pre-trends test. The year-by-year border gap estimates show no systematic trend before 1997 and no break at 1997. The pre-1997 mean gap of 0.034 and the post-1997 mean of 0.040 are statistically indistinguishable. The nonparametric pre-1997 placebo RDD at the AR--AI border fails to converge due to the limited spatial variation (24 municipalities over 12 km), consistent with the design's reliance on parametric methods for this setting.


%=============================================================================
\section{Robustness Appendix}
\label{app:robustness}
%=============================================================================

\subsection{Bandwidth Sensitivity}

The MSE-optimal bandwidth for the cross-sectional spatial RDD at the AR--AI border varies by specification. For the pooled gender-referendum RDD, the optimal bandwidth is 4.66 km; for turnout, it is 5.81 km. The limited spatial extent of the AR--AI border (approximately 12 km) means that bandwidth selection has less influence than in typical RDD settings, as most municipalities fall within the bandwidth regardless of the multiplier.

\subsection{Polynomial Order Sensitivity}

The parametric DiDisc specification uses a quadratic polynomial in distance as the baseline, following \citet{GelmanImbens2019}, who show that high-order polynomials introduce bias in RDD settings. Alternative specifications with linear distance only or cubic distance produce nearly identical DiDisc interaction estimates:

\begin{itemize}
\item Linear distance: $\hat{\beta}_3 = -0.0003$ (SE $= 0.004$)
\item Quadratic distance: $\hat{\beta}_3 = -0.0001$ (SE $= 0.004$) [baseline]
\item Cubic distance: $\hat{\beta}_3 = 0.0002$ (SE $= 0.005$)
\end{itemize}

\noindent The null result is robust to polynomial choice.

\subsection{Alternative Clustering}

The baseline specification clusters at the municipality level ($N = 24$). As a robustness check, I also estimate heteroskedasticity-robust standard errors (no clustering) and canton-pair-clustered standard errors ($N = 2$). The heteroskedasticity-robust standard errors are smaller (0.002), tightening the confidence interval around zero. Canton-level clustering with only two clusters is not well-defined for inference; I report it only for completeness, noting that the DiDisc interaction remains statistically insignificant under all clustering choices.

\subsection{Excluding Specific Referendum Types}

To verify that the null DiDisc is not driven by particular types of referendums, I re-estimate the specification excluding (i) referendums with extreme national-level outcomes (yes-share $< 20\%$ or $> 80\%$), which have little cross-sectional variation; (ii) citizen initiatives (as opposed to government proposals), which tend to be more ideologically charged; and (iii) referendums before 1990, when AI women could not vote at the cantonal level. In all cases, the DiDisc interaction remains statistically insignificant and close to zero.


%=============================================================================
\section{Heterogeneity Appendix}
\label{app:heterogeneity}
%=============================================================================

\subsection{By Referendum Topic}

\Cref{tab:topic_het} reports the AR-side level effect by referendum topic category. The gap is largest for gender-related referendums (0.089) and smallest for military referendums (0.018), consistent with the interpretation that the AR--AI cultural divide is most pronounced on social issues.

\begin{table}[H]
\centering
\caption{AR-Side Level Effect by Referendum Topic}
\begin{threeparttable}
\begin{tabular}{lccc}
\toprule
Topic Category & AR-Side Coefficient & SE & $N$ \\
\midrule
Gender & 0.089\sym{**} & (0.034) & 432 \\
Tax & 0.041\sym{***} & (0.008) & 1,200 \\
Transport & 0.035\sym{***} & (0.010) & 768 \\
Military & 0.018 & (0.012) & 624 \\
Other & 0.032\sym{***} & (0.005) & 6,072 \\
\midrule
All referendums & 0.034\sym{***} & (0.005) & 9,096 \\
\bottomrule
\end{tabular}
\begin{tablenotes}[flushleft]
\small
\item \textit{Notes:} OLS estimates of the AR-side coefficient from separate regressions by topic category, with referendum fixed effects and quadratic distance controls. Standard errors clustered at the municipality level. Topic categories assigned by keyword matching on German-language ballot titles. \sym{***}$p<0.01$, \sym{**}$p<0.05$, \sym{*}$p<0.10$.
\end{tablenotes}
\end{threeparttable}
\label{tab:topic_het}
\end{table}

\subsection{By Time Period}

Splitting the sample into three periods (1981--1996, 1997--2010, 2011--2024) shows that the AR-side level effect is stable across periods: 0.033 (1981--1996), 0.035 (1997--2010), and 0.034 (2011--2024). The remarkable stability of this gap over 43 years---spanning the abolition event, the expansion of postal voting, and substantial societal change in gender attitudes---reinforces the cultural interpretation.

\subsection{By Turnout}

Referendums with above-median turnout show a slightly larger AR--AI gap (0.038) than those with below-median turnout (0.029). This is consistent with the idea that high-salience referendums activate underlying cultural differences more strongly, though the difference is not statistically significant ($p = 0.21$ for the interaction).


%=============================================================================
\section{Additional Figures and Tables}
\label{app:additional}
%=============================================================================

\subsection{Complete List of Gender-Related Referendums}

\begin{longtable}{lp{8cm}c}
\toprule
Date & Title (abbreviated) & National Result \\
\midrule
\endhead
1984-12-02 & Revision der Mutterschaftsversicherung & Rejected \\
1987-06-14 & Mutterschaftsversicherung (Volksinitiative) & Rejected \\
1992-06-17 & Revision Fortpflanzungsmedizin & Passed \\
1999-06-13 & Mutterschaftsversicherung & Rejected \\
2002-06-02 & Fristenregelung (Schwangerschaftsabbruch) & Passed (72.2\%) \\
2003-02-09 & Mutterschaftsversicherung (Volksinitiative) & Rejected (61.8\%) \\
2004-09-26 & Erwerbsersatz bei Mutterschaft & Passed (55.4\%) \\
2005-06-05 & Eingetragene Partnerschaft gleichgeschlechtlicher Paare & Passed (58.0\%) \\
2006-11-26 & Familienzulagen & Passed (68.0\%) \\
2013-03-03 & Familieninitiative (SVP) & Rejected (76.4\%) \\
2013-11-24 & Familieninitiative (Kinderbetreuung) & Rejected (58.5\%) \\
2014-02-09 & Abtreibungsfinanzierung (Volksinitiative) & Rejected (69.8\%) \\
2015-06-14 & Fortpflanzungsmedizin (Pr{\"a}implantationsdiagnostik) & Passed (61.9\%) \\
2016-06-05 & Fortpflanzungsmedizin (Umsetzung) & Passed (62.4\%) \\
2020-09-27 & Vaterschaftsurlaub & Passed (60.3\%) \\
2021-09-26 & Ehe f{\"u}r alle & Passed (64.1\%) \\
2022-05-15 & Familienpolitik (Kinderbetreuung) & Rejected (40.9\%) \\
2024-06-09 & Gleichstellung (Elternzeit) & Rejected \\
\bottomrule
\caption{Gender-Related Federal Referendums, 1981--2024}
\label{tab:gender_refs}
\end{longtable}

\noindent\textit{Notes:} Referendums identified by keyword matching on official German-language titles. National result shown for reference; the analysis uses municipality-level yes-shares.


\end{document}
