\begin{table}[!htbp]
\centering
\caption{Effect of PMGSY Eligibility on Female Employment Outcomes (Pooled RDD)}
\label{tab:main_rdd}
\small
\begin{tabular}{lccccc}
\toprule
Outcome & Coeff. & SE & $p$-value & BW & $N_{\text{eff}}$ \\
\midrule
Chg Female WPR & 0.0005 & (0.0029) & 0.853 & 160 & 119,851 \\
Chg Male WPR & 0.0013 & (0.0012) & 0.294 & 174 & 131,133 \\
Chg Gender Gap & 0.0010 & (0.0026) & 0.704 & 145 & 109,389 \\
Chg F Ag Labor & 0.0013 & (0.0015) & 0.363 & 193 & 144,220 \\
Chg F Cultivator & 0.0014 & (0.0020) & 0.482 & 176 & 132,581 \\
Chg F Other Work & -0.0001 & (0.0008) & 0.901 & 153 & 114,671 \\
Chg F Non-Worker & -0.0005 & (0.0029) & 0.853 & 160 & 119,851 \\
Chg F Literacy & -0.0023 & (0.0017) & 0.163 & 122 & 91,574 \\
Chg Child Sex Ratio & 0.0019$^{*}$ & (0.0010) & 0.067 & 165 & 123,567 \\
\bottomrule
\end{tabular}
\begin{tablenotes}
\small
\item \textit{Notes:} Local polynomial RDD estimates using \texttt{rdrobust} with
triangular kernel and MSE-optimal bandwidth. Running variable: Census 2001 population.
Threshold: 500. Sample: villages with population 50--2,000.
$^{***} p<0.01$, $^{**} p<0.05$, $^{*} p<0.1$.
\end{tablenotes}
\end{table}
