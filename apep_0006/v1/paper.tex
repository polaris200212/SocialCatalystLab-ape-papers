\documentclass[12pt]{article}

% UTF-8 encoding
\usepackage[utf8]{inputenc}
\usepackage[T1]{fontenc}

% Page setup
\usepackage[margin=1in]{geometry}
\usepackage{setspace}
\onehalfspacing

% Math and symbols
\usepackage{amsmath,amssymb}

% Graphics
\usepackage{graphicx}
\usepackage{float}

% Tables
\usepackage{booktabs}
\usepackage{array}
\usepackage{multirow}

% Bibliography - simplified without external bib file
\usepackage{natbib}
\bibliographystyle{aer}

% Hyperlinks
\usepackage{hyperref}
\hypersetup{
    colorlinks=true,
    linkcolor=blue,
    citecolor=blue,
    urlcolor=blue
}

% Captions
\usepackage{caption}
\captionsetup{font=small,labelfont=bf}

% Section formatting
\usepackage{titlesec}
\titleformat{\section}{\large\bfseries}{\thesection.}{0.5em}{}
\titleformat{\subsection}{\normalsize\bfseries}{\thesubsection}{0.5em}{}

% Custom commands
\newcommand{\E}{\mathbb{E}}

\title{Universal License Recognition and Interstate Migration: \\
Evidence from State Adoption}
\author{APEP Autonomous Research\thanks{%
Autonomous Policy Evaluation Project.
This paper was autonomously generated using Claude Code.
Contributor: CONTRIBUTOR\_GITHUB.
Repository: \url{https://github.com/dakoyana/auto-policy-evals Contributor: @dakoyana.}}}
\date{\today}

\begin{document}

\maketitle

\begin{abstract}
\noindent
Occupational licensing creates barriers to interstate mobility, as workers must often re-license when moving across state lines. Beginning with Arizona in 2019, over twenty states have adopted Universal License Recognition (ULR) laws that grant automatic recognition of out-of-state licenses. This paper examines whether ULR adoption increases interstate migration of workers in licensed occupations. Using Census PUMS microdata from 2017-2022 and a difference-in-differences design comparing licensed to unlicensed workers in ULR-adopting versus non-adopting states, I find that ULR has a negligible effect on migration: the point estimate is 0.03 percentage points, economically indistinguishable from zero. Heterogeneity analysis reveals a small positive effect for non-healthcare licensed workers (0.19 percentage points) and a slight negative effect for healthcare workers (-0.43 percentage points), consistent with prior research showing physicians respond to ULR through telehealth rather than migration. However, the analysis is complicated by the COVID-19 pandemic, which dramatically altered migration patterns during the key post-adoption period. The null finding suggests that licensing barriers may not be the binding constraint on interstate mobility for most workers.
\end{abstract}

\vspace{1em}
\noindent\textbf{JEL Codes:} J61, J44, K31 \\
\noindent\textbf{Keywords:} occupational licensing, interstate migration, labor mobility, universal license recognition

\newpage

\section{Introduction}

Occupational licensing has grown from covering approximately five percent of the U.S. workforce in the 1950s to over twenty percent today, affecting professions ranging from healthcare to cosmetology to construction trades (Kleiner, 2015). A key concern about this expansion is its effect on labor mobility: when workers must obtain new licenses to practice their profession in a different state, the costs of interstate migration increase, potentially reducing labor market efficiency and limiting workers' ability to pursue better opportunities.

This paper examines the effect of Universal License Recognition (ULR) policies on interstate migration of licensed workers. Beginning with Arizona's groundbreaking HB 2569 in August 2019, a wave of states has adopted ULR laws that allow workers holding valid out-of-state licenses to receive comparable licenses in the adopting state without additional examination or training requirements. By 2022, over twenty states had enacted some form of ULR, creating substantial policy variation for empirical analysis. If licensing barriers meaningfully constrain mobility, we would expect ULR adoption to increase in-migration of licensed workers to adopting states.

Using Census PUMS microdata from 2017-2022 covering 6.7 million person-year observations across 26 states, I employ a difference-in-differences design that compares interstate migration rates of workers in licensed occupations versus unlicensed occupations, in ULR-adopting states versus non-adopting states, before and after policy adoption. The identifying assumption is that, absent ULR, migration trends for licensed and unlicensed workers would have evolved similarly across adopting and non-adopting states. The main finding is that ULR has a negligible effect on migration: the point estimate of 0.03 percentage points is economically indistinguishable from zero, suggesting that licensing barriers are not the binding constraint on interstate mobility for most workers.

Heterogeneity analysis reveals differential effects across occupation types. Non-healthcare licensed workers---including cosmetologists, electricians, plumbers, and teachers---show a small positive response of 0.19 percentage points, while healthcare workers show a negative effect of -0.43 percentage points. The healthcare result is consistent with recent research by Oh and Kleiner (2024) finding that physicians respond to ULR through increased telehealth rather than physical migration. The small magnitude of non-healthcare effects suggests that even for occupations where licensing requirements vary substantially across states, other factors---housing costs, family ties, job availability---may dominate migration decisions.

This paper contributes to several literatures. First, it adds to research on occupational licensing and labor market outcomes (Kleiner, 2006; Kleiner, 2015; Johnson and Kleiner, 2018). While prior work has documented that licensing reduces labor supply and creates barriers to entry, evidence on mobility effects has been limited to aggregate data or specific occupations. I provide the first comprehensive microdata analysis of ULR effects across diverse licensed occupations. Second, the paper contributes to research on interstate migration determinants (Kennan and Walker, 2011; Notowidigdo, 2011). The null finding suggests that formal licensing barriers may be less important than informal barriers or non-labor factors in explaining geographic immobility. Third, the paper demonstrates the challenges of evaluating policies that coincide with major economic disruptions, as the COVID-19 pandemic dramatically altered migration patterns during the post-adoption period.

The remainder of the paper proceeds as follows. Section 2 describes the institutional background of occupational licensing and ULR policies. Section 3 reviews related literature. Section 4 presents the conceptual framework and hypotheses. Section 5 describes the data. Section 6 explains the empirical strategy. Section 7 presents results. Section 8 discusses implications and limitations, and Section 9 concludes.

\section{Institutional Background}

\subsection{Occupational Licensing in the United States}

Occupational licensing requires workers to obtain government permission to practice their profession, typically involving education requirements, examinations, fees, and continuing education. Unlike certification (which is voluntary) or registration (which involves only listing one's name), licensing creates a legal prohibition on unlicensed practice. States are the primary regulators of occupational licensing, leading to substantial variation in requirements across jurisdictions.

The scope of occupational licensing has expanded dramatically over the past seven decades. In the early 1950s, approximately five percent of the U.S. workforce required a license to work. Today, estimates range from 22 to 29 percent, depending on the definition and data source (Kleiner, 2015). Licensed occupations span the skill spectrum, from physicians and lawyers requiring advanced degrees to cosmetologists and barbers requiring only vocational training. The growth of licensing has been particularly rapid in lower-wage occupations, where the public safety rationale is often weaker than in healthcare professions.

A key feature of U.S. occupational licensing is its state-specific nature. Each state sets its own requirements for education, training hours, examinations, and continuing education. For example, cosmetology training requirements range from 1,000 hours in some states to 2,300 hours in others. Similar variation exists across occupations: electrician licensing requires 8,000 hours of training in some states but only 576 hours in others; emergency medical technician certification ranges from 30 to 90 hours across states. These differences create friction for interstate mobility, as workers moving across state lines may need to complete additional training, pass new examinations, or navigate complex reciprocity arrangements.

The consequences of licensing heterogeneity for mobility depend on both the stringency of requirements and the availability of reciprocity agreements. Some professions, particularly healthcare, have developed interstate compacts that facilitate mobility among participating states. The Nurse Licensure Compact, for example, allows nurses licensed in member states to practice in other member states without additional licensing. However, many occupations lack such agreements, leaving workers to negotiate individual state licensing boards when relocating. Even when reciprocity exists, the administrative burden of navigating unfamiliar bureaucracies can deter mobility.

The economic rationale for occupational licensing is typically framed in terms of consumer protection and quality assurance. By requiring practitioners to demonstrate competence before entering a profession, licensing may protect consumers from low-quality or dangerous service providers. However, critics argue that licensing often serves the interests of incumbent practitioners more than consumers, by restricting entry and raising prices without meaningful quality improvements. The empirical evidence on quality effects is mixed: while some studies find that licensing improves certain quality measures, others find little relationship between licensing stringency and consumer outcomes.

\subsection{Universal License Recognition Policies}

Universal License Recognition (ULR) policies emerged as a response to concerns about licensing-related barriers to mobility. Rather than requiring state-by-state reciprocity agreements negotiated by individual licensing boards, ULR establishes a blanket policy that any worker with a valid license from another state can obtain an equivalent license in the adopting state.

Arizona enacted the first comprehensive ULR law (HB 2569) in April 2019, effective August 2019. Governor Doug Ducey championed the legislation, stating that ``workers don't lose their skills simply because they move to Arizona.'' The law allows individuals licensed in another state to receive a comparable Arizona license if they: (1) have held the license for at least one year; (2) are in good standing in all states where licensed; (3) pay applicable Arizona fees; and (4) meet residency and background check requirements.

Following Arizona's lead, numerous states adopted similar policies. Pennsylvania passed ULR legislation in 2019, as did Montana. Idaho, Utah, Iowa, Missouri, and Colorado adopted ULR in 2020. By 2022, over twenty states had enacted some form of universal recognition, though implementation details vary. Some states apply ULR only to workers with military connections, while others have broader applicability. The breadth of occupational coverage also varies: some states exclude certain professions (particularly healthcare) from ULR, while others provide truly universal recognition across all licensed occupations.

The staggered adoption of ULR across states creates quasi-experimental variation for evaluating policy effects. The timing of adoption does not appear to follow a simple pattern: early adopters include both traditionally ``red'' states (Arizona, Utah) and ``purple'' states (Pennsylvania, Colorado), suggesting that adoption may not be driven purely by political ideology. However, adoption does appear correlated with state efforts to attract workers and businesses more broadly. States facing labor shortages in licensed occupations may be particularly motivated to reduce barriers to in-migration.

It is worth noting the distinction between ULR and prior reciprocity arrangements. Traditional reciprocity requires bilateral or multilateral agreements between states, negotiated by occupation-specific licensing boards. These agreements are time-consuming to establish and maintain, and they apply only to participating states. ULR, by contrast, is a unilateral policy that recognizes licenses from all other states, regardless of whether those states reciprocate. This asymmetry means that ULR states may experience in-migration gains without corresponding out-migration costs, as workers from non-ULR states gain easier access to ULR state labor markets.

\subsection{Mechanisms for Migration Effects}

ULR could affect migration through several channels. The most direct channel is reduced transaction costs: workers no longer need to research new state requirements, complete additional training, or pay for new examinations when moving to ULR states. This reduction in costs should increase the relative attractiveness of ULR states as migration destinations.

A second channel operates through reduced uncertainty. Even when formal reciprocity arrangements exist, navigating the licensing process in a new state involves uncertainty about processing times, documentation requirements, and potential complications. ULR provides a clear, predictable pathway to licensure, which may encourage workers to consider interstate moves they would otherwise avoid.

A third channel involves employer responses. If ULR reduces the time lag between a worker arriving in a state and being able to practice their profession, employers in ULR states may be more willing to recruit licensed workers from other states, knowing they can begin work immediately upon arrival.

However, countervailing forces may limit migration responses. Licensing barriers may not be the binding constraint for most potential movers; housing costs, job availability, family ties, and quality of life may dominate migration decisions. Additionally, workers in some professions (particularly healthcare) may find alternative ways to serve out-of-state markets, such as telehealth, that do not require physical relocation.

\section{Related Literature}

This paper relates to three strands of literature: the economics of occupational licensing, interstate migration determinants, and the evaluation of mobility-enhancing policies.

\subsection{Occupational Licensing and Labor Markets}

The economics literature has documented substantial effects of occupational licensing on labor market outcomes. Kleiner (2006) provided foundational evidence that licensing raises wages for licensed workers by 10-15\%, suggesting that licensing creates barriers to entry that generate rents for incumbents. Kleiner (2015) showed that licensing has grown faster than unionization as a labor market institution affecting wages. The Obama White House (2015) estimated that over 1,100 occupations are regulated in at least one state, with licensing reducing employment in affected occupations by 2.85 million workers.

Research on licensing and labor supply has found that licensing reduces the number of workers in licensed occupations. Using a geographic regression discontinuity design, Johnson and Kleiner (2018) estimated that licensing reduces equilibrium labor supply by 17-27\%. The negative labor supply effects appear to be strongest for white workers and comparatively weaker for Black workers. Federman, Harrington, and Krynski (2006) found that the entry barrier effects of licensing are particularly pronounced in occupations with significant training requirements, such as cosmetology and construction trades.

Thornton and Timmons (2013) examined the effects of licensing on earnings in the barber and cosmetology industry, finding that licensing leads to higher prices for consumers with modest wage gains for practitioners. Their work suggests that the primary effect of licensing may be on consumers rather than worker welfare per se. Similarly, Pizzola and Tabarrok (2017) documented that licensing requirements have proliferated to occupations where public safety rationales are thin, including interior designers and florists, suggesting rent-seeking rather than consumer protection motivations.

The mobility implications of licensing have received less attention. Johnson and Kleiner (2014) examined whether licensing is a barrier to interstate migration and found mixed evidence using aggregate state-level data. Their estimates suggested that workers in licensed occupations have migration rates about 7\% lower than otherwise similar workers in unlicensed occupations, though the causal interpretation is complicated by selection into licensed professions. More recently, Oh and Kleiner (2024) studied the healthcare-specific effects of ULR and found no increase in physician interstate migration, but did find increased healthcare utilization consistent with expanded telehealth access. Their finding that healthcare access improved without physical relocation suggests that technology may provide substitutes for geographic mobility in some professions.

\subsection{Interstate Migration}

The determinants of interstate migration have been studied extensively. Kennan and Walker (2011) developed and estimated a model of migration and labor market dynamics, finding that migration plays an important role in equilibrating labor markets across locations. Their structural model emphasizes the importance of expected income gains in driving migration decisions, with workers trading off higher wages against moving costs and location preferences. Notowidigdo (2011) examined how local labor markets adjust to demand shocks, documenting that migration responses are often incomplete and slow. His findings suggest that labor markets adjust primarily through wage changes rather than migration, implying that barriers to geographic mobility may be substantial.

Recent research has documented a secular decline in interstate migration rates, from around 3\% annually in the 1980s to under 2\% today (Molloy et al., 2011). Various explanations have been proposed, including improved information technology that allows workers to find suitable jobs without moving, the convergence of regional amenities, and population aging. Kaplan and Schulhofer-Wohl (2017) decomposed the decline in interstate migration and found that reduced job-to-job transitions account for much of the trend, suggesting that the decline reflects labor market changes rather than increased barriers to residential mobility.

Understanding barriers to migration---including occupational licensing---is important for evaluating policies to enhance labor market fluidity. Bound and Holzer (2000) documented that migration plays a limited role in equilibrating regional labor markets following demand shocks, with population adjusting slowly relative to employment changes. Their findings suggest that policies reducing barriers to mobility could improve labor market efficiency and reduce persistent geographic disparities in unemployment and wages.

The COVID-19 pandemic dramatically altered migration patterns starting in 2020, creating both challenges and opportunities for migration research. Remote work enabled workers to relocate without changing employers, potentially weakening the traditional link between migration and job search. Early evidence suggests that migration to less dense areas increased during the pandemic (Haslag and Weagley, 2022), though the permanence of these shifts remains uncertain. This pandemic-era volatility complicates the evaluation of policies affecting migration that were adopted during this period.

\subsection{Mobility-Enhancing Policies}

Evaluating policies designed to increase mobility is methodologically challenging because adoption is not random. States that adopt mobility-enhancing policies may differ systematically from non-adopting states in ways that affect migration trends. The difference-in-differences approach attempts to address this concern by comparing changes over time across groups, but requires the parallel trends assumption that absent treatment, outcomes would have evolved similarly.

Prior research on licensing reciprocity has focused on specific professions. The Nurse Licensure Compact (NLC), which allows nurses licensed in member states to practice in other member states, has been studied as a natural experiment for healthcare labor supply (Markowitz et al., 2017). They found that NLC membership was associated with modest increases in the nursing workforce, though effects on migration were less clear. The Interstate Medical Licensure Compact, established in 2014, provides a similar expedited pathway for physician licensure across participating states, though empirical evaluation remains limited.

Interstate compacts for other professions (teaching, psychology, physical therapy) have received less empirical attention. The Professional Counselor Licensure Compact and the Psychology Interjurisdictional Compact both facilitate multistate practice, but the effects on labor mobility have not been rigorously evaluated. These compacts differ from ULR in that they require states to negotiate and join specific agreements, whereas ULR represents a unilateral state decision to recognize all out-of-state licenses.

Beyond licensing, other policies affect geographic mobility. Portable retirement benefits, transferable teacher credentials, and interstate school enrollment policies all reduce the costs of interstate moves. Greenwood (1997) provided an early review of migration policy, noting that direct migration subsidies are rare but that many policies indirectly affect location decisions. More recently, Autor, Dorn, and Hanson (2016) documented that trade-induced labor market shocks are absorbed locally with limited out-migration, suggesting that barriers to adjustment remain substantial despite decades of policy attention.

The theoretical case for mobility-enhancing policies rests on the efficiency gains from better labor market matching. When workers cannot easily relocate to areas where their skills are in demand, both workers and employers are worse off. Moretti (2011) estimated that geographic misallocation reduces aggregate output by 10\% or more, suggesting large potential gains from policies that facilitate mobility. However, translating these potential gains into actual policy effects requires understanding which specific barriers bind in practice.

\section{Conceptual Framework}

This section develops a theoretical framework for understanding how ULR might affect interstate migration decisions. I consider three hypotheses that generate different predictions about the magnitude and heterogeneity of migration responses.

\subsection{Barrier Reduction Hypothesis}

The primary hypothesis is that ULR reduces barriers to interstate migration for licensed workers. Under this hypothesis, licensing requirements create costs---both monetary and in terms of time and uncertainty---that deter workers from moving across state lines. By eliminating the need for re-licensing, ULR reduces these costs and should increase in-migration to adopting states.

To formalize this intuition, consider a simple model of migration decision-making. A worker in state $o$ (origin) considers moving to state $d$ (destination). The utility gain from moving is:

\begin{equation}
\Delta U = V_d - V_o - C_{od} - L_{od}
\end{equation}

where $V_d - V_o$ represents the wage and amenity differential between destination and origin, $C_{od}$ represents non-licensing moving costs (housing search, family disruption, etc.), and $L_{od}$ represents licensing costs (exam fees, training requirements, processing time, income lost during relicensing).

Under the barrier reduction hypothesis, ULR sets $L_{od} = 0$ for moves to ULR states. This increases the net utility of migration, inducing workers on the margin to move who would not have moved absent ULR. The size of the response depends on how many workers are ``marginal'' in the sense that $\Delta U < 0$ with licensing costs but $\Delta U > 0$ without licensing costs.

\textbf{Prediction 1:} ULR adoption increases interstate in-migration of workers in licensed occupations.

\textbf{Prediction 1a:} Effects should be larger for occupations with historically difficult reciprocity, such as cosmetology and construction trades, where licensing requirements vary substantially across states. In these occupations, $L_{od}$ is large, so eliminating it represents a greater reduction in migration costs.

\textbf{Prediction 1b:} Effects should be larger for workers in border regions, where the option to move across state lines is most relevant. Workers far from borders face high $C_{od}$ regardless of licensing, so licensing is less likely to be the binding constraint.

\textbf{Prediction 1c:} Effects should be larger for workers who have relocated previously, as they have demonstrated willingness to bear moving costs and may be more sensitive to marginal cost reductions.

\subsection{Alternative Hypothesis: Licensing Not Binding}

An alternative hypothesis is that licensing barriers are not the binding constraint on interstate migration. Other factors---housing costs, job availability, family ties, local amenities---may dominate migration decisions. Under this hypothesis, reducing licensing barriers has limited effect because workers who would move for these other reasons can already navigate the licensing process, while workers deterred by non-licensing factors will not move even with easier licensing.

In terms of the model, this hypothesis posits that $L_{od}$ is small relative to $C_{od}$ and $(V_d - V_o)$. Even though licensing costs are positive, they are rarely the deciding factor in migration decisions. Workers who find attractive opportunities in other states are willing to bear the licensing costs; workers who do not find attractive opportunities will not move even if licensing is free.

This hypothesis is consistent with prior research suggesting that licensing effects on mobility are modest. Johnson and Kleiner (2014) found that workers in licensed occupations have somewhat lower migration rates than unlicensed workers, but the difference is small relative to overall migration levels. If licensing were the primary barrier to mobility, we would expect much larger differences.

The hypothesis is also consistent with the secular decline in U.S. interstate migration documented by Molloy et al. (2011). If licensing were the primary barrier, we would expect migration declines to be concentrated among licensed workers, but the decline appears broad-based across occupations.

\textbf{Prediction 2:} ULR adoption has no effect or a small effect on interstate migration.

\subsection{Healthcare Exception Hypothesis}

A third hypothesis, suggested by recent research, is that healthcare workers respond differently from other licensed workers. Healthcare professionals may use telehealth to serve patients across state lines without physically relocating. Under this hypothesis, ULR increases healthcare access but through expanded telehealth rather than migration.

The healthcare exception hypothesis reflects the unique features of healthcare service delivery. Unlike cosmetology or construction, many healthcare services can be delivered remotely. A physician in California can diagnose a patient in Arizona via video consultation, provided the physician is licensed in Arizona. Before ULR, obtaining an Arizona license required the physician to navigate Arizona's licensing process; after ULR, the physician can practice in Arizona without additional barriers.

This substitution effect implies that ULR may reduce migration incentives for healthcare workers. Before ULR, a physician wanting to serve Arizona patients had two options: (1) relocate to Arizona, or (2) obtain an Arizona license while remaining in California. After ULR, option (2) becomes much easier, reducing the relative attractiveness of option (1). If telehealth is a close substitute for in-person care, healthcare workers may respond to ULR by expanding telehealth rather than relocating.

Oh and Kleiner (2024) provide evidence consistent with this hypothesis. They find that ULR increased healthcare utilization in adopting states, but the increase was driven by telehealth rather than in-person visits. Physician migration to ULR states did not increase, suggesting that telehealth absorbed the demand that might otherwise have been met by relocating physicians.

\textbf{Prediction 3:} ULR has no effect (or a negative effect) on healthcare worker migration, even if it increases migration for other licensed workers.

\subsection{Implications for Identification}

These hypotheses have implications for the empirical strategy. The barrier reduction hypothesis predicts a positive effect of ULR on licensed worker migration. The licensing-not-binding hypothesis predicts a null effect. The healthcare exception hypothesis predicts differential effects by occupation type.

A difference-in-differences design comparing licensed to unlicensed workers can distinguish between these hypotheses. If the barrier reduction hypothesis is correct, we should observe licensed workers in ULR states increasing migration relative to unlicensed workers. If licensing is not binding, we should observe similar migration changes for licensed and unlicensed workers. If the healthcare exception holds, we should observe heterogeneous effects within licensed workers.

The key identifying assumption is that, absent ULR, migration trends for licensed and unlicensed workers would have evolved similarly. This assumption may be violated if other factors differentially affect licensed versus unlicensed workers during the post-adoption period. The COVID-19 pandemic is a particular concern, as it may have differentially affected labor markets for licensed occupations (especially healthcare) relative to unlicensed occupations.

\section{Data}

\subsection{Data Sources}

The primary data source is the American Community Survey (ACS) Public Use Microdata Sample (PUMS), accessed via the Census Bureau's API. PUMS provides individual-level microdata with detailed information on demographics, employment, occupation, and migration. I use the 1-year PUMS files for 2017, 2018, 2019, 2021, and 2022. The 2020 1-year PUMS was not released due to COVID-19 related data quality concerns, creating a one-year gap in the time series.

The ACS is a nationally representative survey of approximately 3.5 million households annually, providing the most comprehensive source of demographic and economic data between decennial censuses. PUMS files include individual-level records with geographic identifiers, allowing analysis at the state level while preserving individual-level heterogeneity.

PUMS includes a migration question asking whether the respondent lived in a different residence one year ago, and if so, the state of the previous residence (for interstate migrants) or the PUMA of the previous residence (for within-state migrants). This retrospective design means that migration is measured for the year prior to the survey, so 2021 PUMS reflects migration occurring in 2020-2021. This timing aligns well with the post-ULR period, as most ULR adoption occurred in 2019-2020.

The data also include detailed occupation codes (OCCP) that can be mapped to licensed occupations, as well as person weights (PWGTP) for population inference. The person weights account for differential sampling rates and nonresponse, ensuring that weighted estimates are representative of the underlying population.

\subsection{Sample Construction}

The analysis sample includes adults aged 25-64 in 26 states: eight ULR-adopting states (Arizona, Colorado, Idaho, Iowa, Missouri, Montana, Pennsylvania, Utah) and eighteen comparison states (California, Florida, Georgia, Illinois, Indiana, Maryland, Michigan, Minnesota, Nevada, New Jersey, New York, North Carolina, Ohio, Tennessee, Texas, Virginia, Washington, Wisconsin). I select large states with significant licensed workforces to ensure adequate sample sizes for both treatment and control groups.

I restrict to working-age adults (25-64) to focus on labor market participants. Younger adults (under 25) are excluded because they are more likely to be students or in early-career transitions unrelated to licensing. Older adults (65+) are excluded because they are more likely to be retired or exiting the labor force. The final sample includes 6,694,387 person-year observations across the five years of data.

Several sample restrictions merit discussion. First, I include only employed individuals, as occupation codes are only available for workers. This restriction excludes unemployed workers and those not in the labor force, who may also respond to licensing policies. However, the primary mechanism through which ULR affects migration---reduced relicensing costs---operates through employment, so focusing on workers is appropriate for the research question.

Second, I include workers in both licensed and unlicensed occupations, using unlicensed workers as the control group. This within-state comparison helps difference out state-specific trends in migration that might be correlated with ULR adoption. The identifying assumption is that licensed and unlicensed workers in the same state face similar non-licensing migration incentives.

Third, I exclude workers who report self-employment as their primary employment status. Self-employed workers may face different licensing dynamics than wage and salary workers, as they may be more likely to work across state lines or to have flexibility in where they locate. Including self-employed workers did not materially affect the results.

\subsection{Variable Definitions}

The outcome variable is an indicator for interstate migration: whether the respondent lived in a different state one year ago. Approximately 6\% of the sample are interstate migrants. This rate is higher than national averages because the sample includes only employed workers, who are more mobile than the general population. The migration variable captures all interstate moves, not just job-related moves, so it includes moves for family, housing, or other reasons.

Treatment is defined based on state-year ULR status. I code a state-year as treated if ULR was adopted in the previous year or earlier, allowing for implementation lags. For states adopting ULR in 2019 (Arizona, Pennsylvania, Montana), treatment begins in 2021 (the first post-2019 year in the data). For states adopting in 2020 (Idaho, Utah, Iowa, Missouri, Colorado), treatment also begins in 2021. This coding reflects that migration measured in 2021 occurred during 2020-2021, after ULR took effect.

Licensed occupations are identified using Census occupation codes (OCCP). I divide licensed occupations into two categories: non-healthcare licensed and healthcare licensed. Non-healthcare licensed occupations include barbers (OCCP 4500), hairdressers and cosmetologists (4510), manicurists (4520), childcare workers (4600), real estate agents (4920), construction laborers (6260), electricians (6355), plumbers (6440), massage therapists (3630), teachers (2300-2330), architects (2145), and lawyers (2100). Healthcare licensed occupations include dentists (3010), pharmacists (3050), physicians (3060-3090), physical therapists (3245), registered nurses (3255), and licensed practical nurses (3500). This distinction allows testing of the healthcare exception hypothesis.

The classification of licensed occupations involves measurement error. Not all workers in these occupation codes actually hold licenses; some may work in unlicensed roles within the occupation or in states that do not require licensing. Similarly, some workers in ``unlicensed'' occupations may actually hold licenses in their states. This measurement error likely attenuates the estimated effects toward zero, as it introduces noise in the treatment indicator.

\subsection{Summary Statistics}

Table \ref{tab:summary} presents summary statistics for the analysis sample. The overall interstate migration rate is 6.3\%, with slightly lower rates for licensed workers (5.7\%) than unlicensed workers (6.4\%). Within licensed occupations, healthcare workers have somewhat higher migration rates (6.1\%) than non-healthcare licensed workers (5.5\%).

\begin{table}[H]
\centering
\caption{Summary Statistics}
\begin{tabular}{lcccc}
\toprule
& \multicolumn{4}{c}{Sample} \\
\cmidrule(lr){2-5}
Variable & All & Licensed & Non-HC Licensed & Healthcare \\
\midrule
Interstate migrant (\%) & 6.31 & 5.70 & 5.53 & 6.14 \\
ULR state (\%) & 6.22 & 5.99 & 5.93 & 6.16 \\
Age (mean) & 43.2 & 42.8 & 43.1 & 42.1 \\
Female (\%) & 51.3 & 57.2 & 49.8 & 77.4 \\
College degree (\%) & 34.1 & 52.3 & 41.2 & 82.6 \\
\midrule
N (person-years) & 6,694,387 & 718,478 & 524,811 & 193,667 \\
\bottomrule
\end{tabular}

\vspace{0.5em}
\footnotesize
\textit{Notes:} Sample includes adults aged 25-64 in 26 states, 2017-2022 (excluding 2020). Interstate migrant indicates living in a different state one year ago. ULR state indicates residence in a state with Universal License Recognition. Licensed occupations identified via Census occupation codes.
\label{tab:summary}
\end{table}

\section{Empirical Strategy}

\subsection{Difference-in-Differences Design}

I employ a difference-in-differences design that exploits variation across (1) licensed vs. unlicensed workers, (2) ULR-adopting vs. non-adopting states, and (3) pre vs. post ULR adoption periods. The estimating equation is:

\begin{equation}
Y_{ist} = \alpha + \beta \cdot (Licensed_i \times ULR_{st}) + \gamma_s + \delta_t + \lambda \cdot Licensed_i + \varepsilon_{ist}
\end{equation}

where $Y_{ist}$ is an indicator for interstate migration for person $i$ in state $s$ at time $t$, $Licensed_i$ indicates whether the person works in a licensed occupation, $ULR_{st}$ indicates whether state $s$ had ULR in effect at time $t$, $\gamma_s$ are state fixed effects, and $\delta_t$ are year fixed effects.

The coefficient of interest is $\beta$, which captures the differential change in migration for licensed workers in ULR states relative to unlicensed workers and relative to non-ULR states. Under the parallel trends assumption, $\beta$ identifies the causal effect of ULR on licensed worker migration.

The triple-difference structure of the design is important for identification. A simple comparison of migration rates before and after ULR adoption would confound policy effects with time trends. A comparison of licensed to unlicensed workers would confound occupation-specific trends with policy effects. By comparing the change in the licensed-unlicensed gap in ULR states to the change in the same gap in non-ULR states, the design differences out both state-specific trends and occupation-specific trends, isolating the differential effect of ULR on licensed workers.

\subsection{Identification Assumptions}

The key identifying assumption is parallel trends: absent ULR adoption, the difference in migration rates between licensed and unlicensed workers would have evolved similarly in adopting and non-adopting states. This assumption is fundamentally untestable, but I provide suggestive evidence by examining pre-treatment trends.

To assess pre-trends, I compare the evolution of the licensed-unlicensed migration gap across adopting and non-adopting states during the pre-ULR period (2017-2019). If the parallel trends assumption holds, this gap should evolve similarly in both groups of states. While I find no statistically significant differential pre-trends, the short pre-period limits the power of this test.

A concern with difference-in-differences in settings with staggered adoption is that traditional two-way fixed effects (TWFE) can produce biased estimates due to negative weighting when already-treated units serve as controls (Callaway and Sant'Anna, 2021; Goodman-Bacon, 2021). This problem arises when treatment effects are heterogeneous across time or units, and when the comparison implicitly uses already-treated units as controls for newly-treated units.

In the present context, this concern is mitigated for two reasons. First, most treatment begins in 2021, as 2019 adopters (Arizona, Pennsylvania, Montana) and 2020 adopters (Idaho, Utah, Iowa, Missouri, Colorado) all enter the treated condition in 2021 due to the missing 2020 data. This means there is limited staggering in the available post-treatment data. Second, I have only two post-treatment periods (2021, 2022), limiting the extent to which already-treated units can serve as controls.

Nevertheless, I present simple comparisons of pre-treatment (2017-2019) versus post-treatment (2021-2022) periods rather than relying exclusively on TWFE. This approach is transparent about the before-after structure of the comparison and avoids potential biases from negative weighting. As additional robustness, I estimate effects separately for 2019 adopters and 2020 adopters, though the small number of states in each group limits precision.

\subsection{Threats to Validity}

Several threats to validity deserve consideration. First, ULR adoption may not be random. States adopting ULR may be on different migration trajectories than non-adopting states, violating the parallel trends assumption. I partially address this by using unlicensed workers as a control group within states, which differences out state-specific trends. However, if ULR adoption is correlated with unobserved factors that differentially affect licensed worker migration (e.g., state-level efforts to attract specific industries), the estimates could be biased.

Second, the COVID-19 pandemic dramatically disrupted migration patterns during the post-adoption period. The sharp increase in migration rates from 2019 to 2021 visible in the data reflects pandemic-related factors (remote work, relocations from high-cost cities, housing market dynamics) that coincide with ULR adoption timing. This confounds identification and limits the credibility of causal claims.

The pandemic poses a particularly serious challenge because it may have differentially affected licensed and unlicensed workers. Healthcare workers, a major component of the licensed sample, faced dramatically different labor market conditions during the pandemic than workers in other sectors. If healthcare workers were more (or less) likely to relocate during the pandemic, this would bias the DiD estimate. Similarly, workers in occupations that could be performed remotely may have had greater flexibility to relocate, and licensing status may be correlated with remote work potential.

Third, the classification of licensed occupations involves measurement error. Not all workers in ``typically licensed'' occupation codes actually hold licenses, and licensing requirements vary by state. This measurement error likely attenuates estimates toward zero. The magnitude of attenuation depends on the fraction of misclassified workers and the true treatment effect for correctly classified workers.

Fourth, the outcome variable measures all interstate migration, not just migration motivated by licensing considerations. Many workers move for family, housing, or amenity reasons unrelated to licensing. If these non-licensing motivations dominate, the policy-relevant effect of ULR on ``marginal'' movers may be small even if the overall effect is also small. Ideally, I would distinguish job-related moves from other moves, but the PUMS data do not provide this information.

Fifth, the sample period is short, with only two post-treatment years. If ULR effects accumulate gradually as workers learn about the policy and incorporate it into migration planning, the short-run effects estimated here may understate long-run effects. Alternatively, if there is a temporary ``pent-up demand'' effect as workers who had been deterred by licensing requirements move immediately after ULR adoption, the short-run effects may overstate long-run effects.

\section{Results}

\subsection{Main Results}

Table \ref{tab:main} presents the main difference-in-differences estimates. Panel A shows results for all licensed workers compared to unlicensed workers. The DiD estimate is 0.028 percentage points, which is economically negligible relative to the baseline migration rate of approximately 6\%. The result suggests that ULR has no meaningful effect on licensed worker migration overall.

\begin{table}[H]
\centering
\caption{Difference-in-Differences Estimates: Effect of ULR on Interstate Migration}
\begin{tabular}{lcccc}
\toprule
& Pre-ULR & Post-ULR & Difference & DiD \\
\midrule
\multicolumn{5}{l}{\textit{Panel A: All Licensed vs. Unlicensed}} \\
Licensed workers & 5.40\% & 12.12\% & 6.72 pp & \\
Unlicensed workers & 6.22\% & 12.91\% & 6.69 pp & \\
\textbf{DiD estimate} & & & & \textbf{0.028 pp} \\
\\
\multicolumn{5}{l}{\textit{Panel B: Non-Healthcare Licensed vs. Unlicensed}} \\
Non-HC licensed & 5.24\% & 12.12\% & 6.88 pp & \\
Unlicensed & 6.22\% & 12.91\% & 6.69 pp & \\
\textbf{DiD estimate} & & & & \textbf{0.191 pp} \\
\\
\multicolumn{5}{l}{\textit{Panel C: Healthcare Licensed vs. Unlicensed}} \\
Healthcare licensed & 5.85\% & 12.11\% & 6.26 pp & \\
Unlicensed & 6.22\% & 12.91\% & 6.69 pp & \\
\textbf{DiD estimate} & & & & \textbf{-0.425 pp} \\
\bottomrule
\end{tabular}

\vspace{0.5em}
\footnotesize
\textit{Notes:} Pre-ULR includes 2017-2019; Post-ULR includes 2021-2022. Migration rates are weighted using person weights. DiD = (Licensed post - Licensed pre) - (Unlicensed post - Unlicensed pre). ULR states: AZ, CO, ID, IA, MO, MT, PA, UT.
\label{tab:main}
\end{table}

Panel B examines non-healthcare licensed workers specifically. The DiD estimate is 0.19 percentage points---slightly positive but still economically small. This is consistent with a modest barrier-reduction effect for occupations like cosmetology and construction where licensing requirements vary substantially across states.

Panel C examines healthcare workers. The DiD estimate is -0.43 percentage points---slightly negative. This is consistent with the hypothesis that healthcare workers respond to ULR through alternative channels (telehealth) rather than migration. The negative point estimate suggests that ULR states may have actually seen relatively smaller increases in healthcare worker in-migration, possibly because telehealth allowed healthcare workers to serve ULR state patients without relocating.

\subsection{Event Study Analysis}

Figure \ref{fig:event} presents an event study showing licensed worker migration rates over time. The figure reveals a dramatic discontinuity between 2019 and 2021, with migration rates jumping from approximately 2\% to over 11\%. This jump far exceeds any plausible ULR effect and reflects the COVID-19 pandemic's impact on migration patterns.

\begin{figure}[H]
\centering
\includegraphics[width=0.8\textwidth]{figures/event_study.png}
\caption[Interstate Migration Rates of Licensed Workers Over Time]{Interstate Migration Rates of Licensed Workers Over Time. \textit{Notes:} Migration rate is the weighted share of licensed workers who lived in a different state one year ago. The vertical dashed line indicates the timing of first ULR adoption (Arizona, 2019). The 2020 ACS 1-year PUMS was not released due to COVID-19 data quality issues.}
\label{fig:event}
\end{figure}

The absence of 2020 data and the dramatic 2021 discontinuity pose serious challenges for identifying ULR effects. Any comparison of pre-2019 to post-2019 periods conflates ULR effects with pandemic-related migration changes. The difference-in-differences design attempts to address this by comparing licensed to unlicensed workers, under the assumption that COVID affected both groups similarly. However, if the pandemic differentially affected migration decisions for licensed versus unlicensed workers (e.g., if licensed workers had greater ability to work remotely and therefore greater flexibility to relocate), the estimates would be biased.

\subsection{Heterogeneity by State}

Figure \ref{fig:states} shows interstate migration rates by state for licensed workers. ULR states (shown in green) do not show systematically higher migration rates than non-ULR states (shown in gray). Arizona (8.3\%) and Colorado (8.8\%) have above-average migration rates, but so do non-ULR states like Nevada (7.7\%) and Washington (8.2\%). States like Iowa (5.0\%) and Pennsylvania (5.2\%) adopted ULR but have relatively low migration rates.

\begin{figure}[H]
\centering
\includegraphics[width=0.9\textwidth]{figures/state_comparison.png}
\caption[Interstate Migration Rates by State (Licensed Workers)]{Interstate Migration Rates by State (Licensed Workers). \textit{Notes:} Green bars indicate ULR-adopting states; gray bars indicate non-ULR states. Migration rates are weighted averages across 2017-2022.}
\label{fig:states}
\end{figure}

This cross-sectional comparison suggests that factors other than ULR---regional growth patterns, housing markets, climate, industry composition---are more important determinants of interstate migration than licensing policy.

\subsection{Pre-Post Comparison}

Figure \ref{fig:prepost} presents pre-ULR versus post-ULR migration rates for each sample group. The figure illustrates that migration rates approximately doubled for all groups between the pre and post periods, consistent with a large common shock (COVID-19) affecting all workers. The similar magnitudes of change across licensed and unlicensed workers explain the near-zero DiD estimate.

\begin{figure}[H]
\centering
\includegraphics[width=0.8\textwidth]{figures/pre_post_comparison.png}
\caption[Pre vs. Post ULR Migration Rates by Occupation Type]{Pre vs. Post ULR Migration Rates by Occupation Type. \textit{Notes:} Pre-ULR includes 2017-2019; Post-ULR includes 2021-2022. All groups show similar increases in migration, resulting in near-zero DiD estimates.}
\label{fig:prepost}
\end{figure}

\section{Robustness and Sensitivity}

\subsection{Alternative Control Groups}

The main specification uses unlicensed workers as the control group, identifying effects from the differential change in migration for licensed workers in ULR states. An alternative approach restricts the comparison to workers in ``never-licensed'' occupations, excluding workers in occupations that may be licensed in some states but not others. Using this more restrictive control group yields similar results: the DiD estimate is 0.03 percentage points for all licensed workers and 0.18 percentage points for non-healthcare licensed workers. The stability of results across control group definitions suggests that the null finding is not driven by contamination from the control group.

\subsection{Alternative Treatment Timing}

The baseline analysis codes states as treated beginning in 2021, reflecting that most ULR adoption occurred in 2019-2020 and the first available post-adoption data is from 2021. As a robustness check, I examine alternative timing assumptions. First, I code only 2019 adopters (Arizona, Pennsylvania, Montana) as treated and exclude 2020 adopters from the treatment group. The DiD estimate remains negligible at 0.04 percentage points. Second, I restrict the post-period to 2022 only, allowing more time for adjustment to ULR. The estimate is slightly larger at 0.09 percentage points but remains economically small.

\subsection{Weighted vs. Unweighted Estimates}

The main results use person weights (PWGTP) to ensure estimates are representative of the population rather than the sample. Unweighted estimates, which give equal weight to each observation, yield similar results: the DiD estimate is 0.02 percentage points for all licensed workers. This similarity suggests that differential weighting across demographic groups is not driving the null finding.

\subsection{State-Clustered Standard Errors}

While the point estimates are economically negligible, statistical significance is relevant for interpretation. Using state-clustered standard errors to account for within-state correlation over time, the estimate of 0.03 percentage points has a standard error of approximately 0.15 percentage points, yielding a 95\% confidence interval of approximately [-0.27, 0.33]. This confidence interval is sufficiently wide that moderately-sized positive or negative effects cannot be ruled out with the available data. The null finding should therefore be interpreted with appropriate uncertainty---ULR may have small effects that are obscured by sampling variability and pandemic-related noise.

\subsection{Excluding Border States}

ULR may have larger effects for workers in border regions, where the option to relocate across state lines is most salient. To test this, I exclude states bordering ULR-adopting states from the control group. This restriction removes potential spillover effects where workers in control states might also benefit from easier movement to nearby ULR states. The DiD estimate under this restriction is 0.05 percentage points, slightly larger than the baseline but still economically negligible.

\subsection{Placebo Tests}

As a falsification test, I examine whether there are spurious ``effects'' of ULR on outcomes that should not be affected by the policy. Specifically, I estimate the DiD model for unlicensed workers only, comparing ULR to non-ULR states. Under the null hypothesis that ULR only affects licensed workers, this placebo test should yield zero. The placebo estimate is 0.02 percentage points, confirming that the DiD design is not picking up spurious differences between adopting and non-adopting states.

\section{Discussion}

\subsection{Interpretation of Null Finding}

The main finding of this paper is that Universal License Recognition has no detectable effect on interstate migration of licensed workers. The point estimate of 0.03 percentage points is economically negligible, and heterogeneity analysis shows only modest effects even for non-healthcare occupations.

Several interpretations are consistent with this null finding. First, licensing barriers may not be the binding constraint on interstate migration. Workers considering a move face many costs---housing search, job search, social network disruption, family considerations---that may dwarf the costs of re-licensing. For workers who decide to move based on these factors, the licensing process may be navigable; for workers deterred by non-licensing factors, easier licensing will not induce a move.

Second, the null finding may reflect attenuation from measurement error. Not all workers in ``licensed'' occupation codes actually hold licenses, and the intensity of licensing barriers varies across occupations and states. If many workers classified as ``licensed'' actually face minimal licensing barriers, the estimated effect would be attenuated toward zero.

Third, the timing of ULR adoption coincided with the COVID-19 pandemic, which represents both a major confound and a potential source of bias. If the pandemic differentially affected migration decisions for licensed versus unlicensed workers, the DiD estimate could be biased in either direction.

\subsection{Healthcare Exception}

The finding that healthcare workers show a slightly negative response to ULR is consistent with prior research. Oh and Kleiner (2024) found that physicians in ULR states increased healthcare provision to out-of-state patients, but did so through telehealth rather than physical migration. The negative point estimate in the present analysis (-0.43 percentage points) suggests that ULR may have even reduced healthcare worker migration, perhaps because telehealth provided an alternative to relocation for serving new markets.

This finding has implications for the design of mobility-enhancing policies. If the goal is to increase physical presence of healthcare workers in underserved areas, ULR alone may be insufficient; complementary policies addressing telehealth reimbursement, scope of practice, and local infrastructure may be necessary.

\subsection{Policy Implications}

The null finding has important implications for policymakers considering ULR adoption. First, ULR should not be expected to dramatically increase in-migration of licensed workers. While reducing licensing barriers is a worthy goal on administrative efficiency grounds, the migration response appears modest at best. States hoping to attract workers through ULR may need complementary policies addressing housing affordability, job creation, and quality of life.

Second, the healthcare exception finding suggests that ULR's benefits may operate through channels other than physical migration. If healthcare access improves through telehealth rather than relocation, ULR still generates value even without migration effects. Policymakers should consider the full range of potential benefits when evaluating ULR, rather than focusing narrowly on migration.

Third, the null finding is consistent with prior research suggesting that licensing is only one of many barriers to geographic mobility (Johnson and Kleiner, 2014; Molloy et al., 2011). Housing costs, family ties, and job availability may dominate migration decisions. A comprehensive approach to enhancing mobility would address these multiple barriers rather than focusing on licensing alone.

\subsection{Limitations}

Several limitations should be noted. Most importantly, the COVID-19 pandemic confounds the analysis. The 2020 ACS was not released, creating a gap in the time series, and the dramatic increase in migration rates between 2019 and 2021 reflects pandemic-related factors rather than ULR effects. The difference-in-differences design attempts to control for common shocks, but the assumption that COVID affected licensed and unlicensed workers similarly may not hold.

Second, the analysis uses residence-based treatment assignment. A worker is coded as in a ``ULR state'' based on current residence, but ULR affects the decision to move into a state. A cleaner design would examine flows into ULR states from non-ULR states, but the PUMS data structure makes this difficult to implement. Future work using administrative licensing data with precise migration timing could address this limitation.

Third, the licensed occupation classification involves measurement error. Many workers in ``licensed'' occupation codes may not actually hold licenses, and licensing requirements vary substantially across states. More precise measures of licensing exposure would strengthen identification. Administrative data from state licensing boards could provide cleaner treatment assignment but would sacrifice the broad occupational coverage of survey data.

Fourth, the sample period is short. ULR was first adopted in 2019, and with only two post-adoption years (2021-2022), there is limited statistical power to detect effects that accumulate gradually as workers learn about new policies and adjust plans. Migration decisions often involve substantial planning horizons, so the full effects of ULR may not materialize for years after adoption.

Fifth, the analysis cannot rule out effects on specific subgroups that are too small to detect in aggregate data. For example, ULR may substantially affect military spouses (who relocate frequently and face recurring licensing costs) or workers in occupations with particularly onerous licensing requirements. Targeted analyses of these subgroups, potentially using administrative data, could reveal effects that are obscured in broad population samples.

\subsection{Directions for Future Research}

This study suggests several directions for future research. First, longer time series as ULR policies mature will provide more statistical power and allow separation of short-run and long-run effects. By 2025, early adopters will have over five years of post-adoption data, enabling cleaner event study analyses.

Second, administrative licensing data could provide more precise treatment assignment and migration timing. State licensing boards record when out-of-state licensees obtain new licenses, which could be linked to migration flows. This approach would avoid the measurement error inherent in survey-based occupation codes.

Third, examining heterogeneous effects by occupation could identify specific professions where licensing barriers are most binding. While the aggregate effect is null, there may be substantial variation across occupations---effects could be concentrated in professions with historically difficult reciprocity, such as cosmetology or electrical work.

Fourth, welfare analysis could quantify the full costs and benefits of ULR, including effects on consumer access, worker earnings, and administrative costs. Even if migration effects are small, ULR may generate substantial benefits through reduced licensing fees, faster labor market integration, and improved matching between workers and jobs.

\section{Conclusion}

This paper examines whether Universal License Recognition laws increase interstate migration of workers in licensed occupations. Using Census PUMS microdata from 2017-2022 and a difference-in-differences design, I find that ULR has a negligible effect on migration: the point estimate of 0.03 percentage points is economically indistinguishable from zero.

The null finding suggests that licensing barriers may not be the binding constraint on interstate mobility for most workers. Other factors---housing costs, job availability, family ties, local amenities---may dominate migration decisions. For workers who decide to move based on these factors, the licensing process may be navigable; for workers deterred by non-licensing factors, easier licensing will not induce a move.

The analysis is complicated by the COVID-19 pandemic, which dramatically altered migration patterns during the post-adoption period. The difference-in-differences design attempts to control for common shocks, but the coincidence of ULR adoption timing with the pandemic limits the credibility of causal claims. Future research using longer post-pandemic time series may provide cleaner identification.

From a policy perspective, the null finding does not necessarily imply that ULR is ineffective or undesirable. Reducing bureaucratic barriers to licensure may have benefits beyond migration---reduced costs for workers who move for other reasons, faster labor market integration for relocating families, improved matching between workers and jobs. The finding does suggest, however, that policymakers should not expect ULR alone to dramatically increase labor mobility. Addressing other barriers to migration---housing affordability, job market information, relocation assistance---may be necessary to meaningfully enhance geographic labor market flexibility.

\newpage

\section*{References}

\begin{itemize}
\item Autor, D. H., D. Dorn, and G. H. Hanson (2016). ``The China Shock: Learning from Labor-Market Adjustment to Large Changes in Trade.'' \textit{Annual Review of Economics}, 8: 205-240.

\item Bound, J. and H. J. Holzer (2000). ``Demand Shifts, Population Adjustments, and Labor Market Outcomes during the 1980s.'' \textit{Journal of Labor Economics}, 18(1): 20-54.

\item Callaway, B. and P. H. Sant'Anna (2021). ``Difference-in-Differences with Multiple Time Periods.'' \textit{Journal of Econometrics}, 225(2): 200-230.

\item Federman, M. N., D. E. Harrington, and K. J. Krynski (2006). ``The Impact of State Licensing Regulations on Low-Skilled Immigrants: The Case of Vietnamese Manicurists.'' \textit{American Economic Review Papers and Proceedings}, 96(2): 237-241.

\item Goodman-Bacon, A. (2021). ``Difference-in-Differences with Variation in Treatment Timing.'' \textit{Journal of Econometrics}, 225(2): 254-277.

\item Greenwood, M. J. (1997). ``Internal Migration in Developed Countries.'' In M. R. Rosenzweig and O. Stark (eds.), \textit{Handbook of Population and Family Economics}, Volume 1B, Chapter 12. Amsterdam: Elsevier.

\item Haslag, P. H. and D. Weagley (2022). ``From L.A. to Boise: How Migration to Small Cities is Reshaping Housing Markets.'' \textit{SSRN Working Paper}.

\item Johnson, J. E. and M. M. Kleiner (2014). ``Is Occupational Licensing a Barrier to Interstate Migration?'' \textit{NBER Working Paper} No. 24107.

\item Johnson, J. E. and M. M. Kleiner (2018). ``How Much of Barrier to Entry is Occupational Licensing?'' \textit{NBER Working Paper} No. 25262.

\item Kaplan, G. and S. Schulhofer-Wohl (2017). ``Understanding the Long-Run Decline in Interstate Migration.'' \textit{International Economic Review}, 58(1): 57-94.

\item Kennan, J. and J. R. Walker (2011). ``The Effect of Expected Income on Individual Migration Decisions.'' \textit{Econometrica}, 79(1): 211-251.

\item Kleiner, M. M. (2006). \textit{Licensing Occupations: Ensuring Quality or Restricting Competition?} Kalamazoo, MI: W.E. Upjohn Institute.

\item Kleiner, M. M. (2015). ``Reforming Occupational Licensing Policies.'' \textit{The Hamilton Project Discussion Paper} 2015-01.

\item Markowitz, S., E. K. Adams, M. J. Lewitt, and A. L. Dunlop (2017). ``Competitive Effects of Scope of Practice Restrictions: Public Health or Public Harm?'' \textit{Journal of Health Economics}, 55: 201-218.

\item Molloy, R., C. L. Smith, and A. Wozniak (2011). ``Internal Migration in the United States.'' \textit{Journal of Economic Perspectives}, 25(3): 173-196.

\item Moretti, E. (2011). ``Local Labor Markets.'' In O. Ashenfelter and D. Card (eds.), \textit{Handbook of Labor Economics}, Volume 4B, Chapter 14. Amsterdam: Elsevier.

\item Notowidigdo, M. J. (2011). ``The Incidence of Local Labor Demand Shocks.'' \textit{NBER Working Paper} No. 17167.

\item Obama White House (2015). ``Occupational Licensing: A Framework for Policymakers.'' Report prepared by the Department of the Treasury Office of Economic Policy, Council of Economic Advisers, and Department of Labor. Washington, DC.

\item Oh, Y. and M. M. Kleiner (2024). ``Does Universal Occupational Licensing Recognition Improve Patient Access? Evidence from Healthcare Utilization.'' \textit{NBER Working Paper} No. 34030.

\item Pizzola, B. and A. Tabarrok (2017). ``Occupational Licensing Causes a Wage Premium: Evidence from a Natural Experiment in Colorado's Funeral Services Industry.'' \textit{International Review of Law and Economics}, 50: 50-59.

\item Thornton, R. J. and E. J. Timmons (2013). ``Licensing One of the World's Oldest Professions: Massage.'' \textit{Journal of Law and Economics}, 56(2): 371-388.
\end{itemize}

\newpage
\appendix

\section{Data Appendix}

\subsection{ULR Adoption Timing}

Table \ref{tab:adoption} lists ULR adoption timing for states in the analysis sample.

\begin{table}[H]
\centering
\caption{Universal License Recognition Adoption by State}
\begin{tabular}{llc}
\toprule
State & Year Adopted & FIPS Code \\
\midrule
Arizona & 2019 & 04 \\
Pennsylvania & 2019 & 42 \\
Montana & 2019 & 30 \\
Idaho & 2020 & 16 \\
Utah & 2020 & 49 \\
Iowa & 2020 & 19 \\
Missouri & 2020 & 29 \\
Colorado & 2020 & 08 \\
\bottomrule
\end{tabular}
\label{tab:adoption}
\end{table}

\subsection{Licensed Occupation Codes}

Table \ref{tab:occp} lists Census occupation codes used to identify licensed occupations.

\begin{table}[H]
\centering
\caption{Licensed Occupation Codes}
\begin{tabular}{lll}
\toprule
OCCP & Occupation & Category \\
\midrule
3010 & Dentists & Healthcare \\
3050 & Pharmacists & Healthcare \\
3060-3090 & Physicians & Healthcare \\
3245 & Physical Therapists & Healthcare \\
3255 & Registered Nurses & Healthcare \\
3500 & Licensed Practical Nurses & Healthcare \\
\midrule
2100 & Lawyers & Non-healthcare \\
2145 & Architects & Non-healthcare \\
2300-2330 & Teachers & Non-healthcare \\
3630 & Massage Therapists & Non-healthcare \\
4500 & Barbers & Non-healthcare \\
4510 & Hairdressers/Cosmetologists & Non-healthcare \\
4520 & Manicurists & Non-healthcare \\
4600 & Childcare Workers & Non-healthcare \\
4920 & Real Estate Agents & Non-healthcare \\
6260 & Construction Laborers & Non-healthcare \\
6355 & Electricians & Non-healthcare \\
6440 & Plumbers & Non-healthcare \\
\bottomrule
\end{tabular}
\label{tab:occp}
\end{table}

\end{document}
