\documentclass[12pt]{article}

% UTF-8 encoding and fonts
\usepackage[utf8]{inputenc}
\usepackage[T1]{fontenc}
\usepackage{lmodern}

% Page setup
\usepackage[margin=1in]{geometry}
\usepackage{setspace}
\onehalfspacing

% Typography
\usepackage{microtype}

% Math and symbols
\usepackage{amsmath,amssymb}

% Graphics
\usepackage{graphicx}
\usepackage{float}
\usepackage{subcaption}

% Tables
\usepackage{booktabs}
\usepackage{array}
\usepackage{multirow}
\usepackage{threeparttable}
\usepackage{longtable}
\usepackage{pdflscape}
\usepackage{siunitx}
\sisetup{detect-all=true, group-separator={,}, group-minimum-digits=4}

% Bibliography
\usepackage{natbib}
\bibliographystyle{aer}

% Hyperlinks
\usepackage{hyperref}
\hypersetup{
    colorlinks=true,
    linkcolor=blue,
    citecolor=blue,
    urlcolor=blue
}
\usepackage[nameinlink,noabbrev]{cleveref}

% Timing data
\IfFileExists{timing_data.tex}{\newcommand{\apepcurrenttime}{1h 4m}
\newcommand{\apepcumulativetime}{1h 4m}
}{
  \newcommand{\apepcurrenttime}{N/A}
  \newcommand{\apepcumulativetime}{N/A}
}

% Captions
\usepackage{caption}
\captionsetup{font=small,labelfont=bf}

% Section formatting
\usepackage{titlesec}
\titleformat{\section}{\large\bfseries}{\thesection.}{0.5em}{}
\titleformat{\subsection}{\normalsize\bfseries}{\thesubsection}{0.5em}{}

% Custom commands
\newcommand{\E}{\mathbb{E}}
\newcommand{\Var}{\text{Var}}
\newcommand{\Cov}{\text{Cov}}
\newcommand{\ind}{\mathbb{I}}
\newcommand{\sym}[1]{\ifmmode^{#1}\else\(^{#1}\)\fi}

\title{Registered but Not Voting: Felon Voting Rights Restoration\\and the Limits of Civic Re-Inclusion}
\author{APEP Autonomous Research\thanks{Autonomous Policy Evaluation Project. Correspondence: scl@econ.uzh.ch} \and @ai1scl}
\date{\today}

\begin{document}

\maketitle

\begin{abstract}
\noindent
Does restoring felon voting rights increase Black political participation beyond the directly affected population? I exploit the staggered adoption of felon voting rights restoration laws across 22 US states from 1996 to 2024 using CPS Voting Supplement data and a difference-in-differences design with the \citet{callaway2021} estimator. Rights restoration increases Black voter registration by 2.3 percentage points relative to White registration (p$<$0.001), but the Black-White \textit{turnout} gap widens by 3.7 percentage points (p$=$0.015). A triple-difference exploiting within-race variation in felony risk finds no significant community-level spillovers. These results suggest that restoration removes a legal barrier to registration but does not produce the broader civic re-engagement predicted by ``civic chill'' theories. The divergence between registration and turnout effects implies that formal inclusion without supportive mobilization infrastructure is insufficient to close racial participation gaps.
\end{abstract}

\vspace{1em}
\noindent\textbf{JEL Codes:} D72, J15, K14, H75 \\
\noindent\textbf{Keywords:} felon disenfranchisement, voting rights, racial participation gap, difference-in-differences, civic engagement

\newpage

\section{Introduction}

Nearly six million Americans cannot vote because of a felony conviction. The impact falls disproportionately on Black communities: one in sixteen Black adults of voting age is disenfranchised, compared to one in fifty-nine for the general population \citep{uggen2022}. This disparity is not incidental. Felon disenfranchisement laws trace directly to the post-Reconstruction era, when Southern states crafted criminal codes and constitutional provisions explicitly designed to exclude Black citizens from the franchise \citep{behrens2003,manza2006}. The question of whether restoring these rights can reverse their damage---not just for the directly affected, but for entire communities---is both empirically open and politically urgent.

This paper tests whether felon voting rights restoration increases Black political participation in the United States, and whether its effects extend beyond the directly affected population to the broader Black community. The ``civic chill'' hypothesis posits that mass felon disenfranchisement---which removes a visible share of community members from the electorate---depresses political engagement among non-disenfranchised residents through norm erosion, stigma, and reduced mobilization infrastructure \citep{burch2013,lee2014}. If this mechanism operates, restoring voting rights should increase participation not only among ex-felons themselves but among Black citizens who were never disenfranchised.

I exploit the staggered adoption of felon voting rights restoration laws across 22 US states between 1996 and 2024. Using biennial data from the Current Population Survey (CPS) Voting and Registration Supplement---the gold standard for measuring political participation---I implement a difference-in-differences (DD) design comparing Black-White participation gaps in reform states before and after restoration, using never-reform states as controls. For heterogeneity-robust estimation under staggered adoption, I employ the \citet{callaway2021} estimator with doubly-robust inference. To separate community-level spillovers from direct effects, I design a triple-difference (DDD) that exploits demographic variation in felony risk: Black women aged 50+ and college-educated Black citizens have near-zero felony conviction rates, so any change in their participation after restoration cannot reflect direct rights restoration and must operate through community-level channels.

The results challenge the civic chill hypothesis. I find that rights restoration significantly increases Black voter \textit{registration} by 2.3 percentage points relative to White registration ($p < 0.001$). This is consistent with the direct effect of barrier removal: citizens who were previously ineligible can now register. However, this registration gain does not translate into turnout. The Black-White \textit{turnout} gap actually widens by 3.7 percentage points after restoration ($p = 0.015$), a result that is remarkably stable across specifications---including with or without reversal states, restricting to permanent legislative reforms, controlling for concurrent voting laws, and separately estimating presidential and midterm elections. The Callaway-Sant'Anna event study shows no clear pre-trends but a noisy post-treatment pattern with an overall ATT of +5.3 percentage points on the turnout gap (positive but insignificant), driven primarily by a few small treatment cohorts. The DDD triple interaction is positive but insignificant (+2.2 pp, $p = 0.14$), and a pure spillover test restricted to low-felony-risk Black citizens shows a significant \textit{decline} in relative turnout ($-2.6$ pp, $p = 0.009$). A placebo test on the Hispanic-White gap finds no significant effect ($-1.0$ pp, $p = 0.31$), supporting the interpretation that the results are specific to the Black population and not driven by concurrent voting law changes.

These findings shift the debate from whether felon disenfranchisement laws affect the directly targeted individuals \citep{miles2004,burch2012,gerber2015,white2019} to whether their reversal produces the broader civic re-engagement predicted by community-level theories \citep{burch2013}. It does not. The registration-turnout divergence also challenges the implicit assumption in the racial participation gap literature \citep{fraga2018,white2022} that removing legal barriers narrows gaps---formal inclusion is necessary but clearly insufficient. Methodologically, the setting provides a useful proving ground for modern staggered DiD estimators \citep{goodman2021,callaway2021,sun2021,roth2023trends} in a context where treatment heterogeneity across cohorts, reversals, and reform types generates substantively meaningful differences between estimators.


\section{Institutional Background and the Reform Movement}
\label{sec:background}

\subsection{The Origins and Scope of Felon Disenfranchisement}

Felon disenfranchisement in the United States has roots in colonial-era ``civil death'' doctrines, but its modern form was shaped decisively by the Reconstruction era. As \citet{behrens2003} document, Southern states adopted sweeping criminal disenfranchisement provisions in their post-Reconstruction constitutions---often targeting offenses believed to be disproportionately committed by Black citizens. The explicit racial intent is well-documented: Alabama's 1901 constitutional convention president stated that the new provisions were designed ``to establish white supremacy within the limits of the law'' \citep{manza2006}.

By the mid-twentieth century, every US state except Maine and Vermont had some form of felon disenfranchisement. The policies varied considerably in severity: some states restricted voting only during incarceration, while others imposed permanent disenfranchisement for any felony conviction. The expansion of mass incarceration beginning in the 1970s dramatically increased the scope of these laws. By 2020, an estimated 5.85 million Americans were disenfranchised due to felony convictions, with Black citizens constituting a vastly disproportionate share \citep{uggen2022}.

\subsection{The Reform Wave: 1996--2024}

Beginning in the late 1990s, a bipartisan reform movement emerged to restore voting rights to people with felony convictions. The reforms took multiple forms:

\textbf{Legislative restoration.} Most common, these statutes expanded automatic rights restoration after completion of sentence, eliminated waiting periods, or restored voting during probation or parole. Examples include Connecticut (2001, probation voting), Nebraska (2005, eliminated two-year waiting period), and Louisiana (2019, restored after sentence completion).

\textbf{Executive orders.} Several governors unilaterally expanded restoration through executive clemency powers. Virginia Governor McAuliffe's 2016 mass clemency for completed-sentence felons and Kentucky Governor Beshear's 2019 executive order are notable examples. These are institutionally fragile---Iowa demonstrates the vulnerability of executive action, where Governor Vilsack's 2005 expansion was reversed by Governor Branstad in 2011.

\textbf{Ballot initiatives.} Florida's Amendment 4 (2018) is the most consequential single reform, restoring voting rights to an estimated 1.4 million Floridians with felony convictions (excluding those convicted of murder or sexual offenses). California's Proposition 17 (2020) restored voting during parole.

\Cref{tab:reform_timing} presents the complete treatment timing for the 22 states in our sample. The staggered adoption pattern is well-suited for a difference-in-differences design: reforms span from 1998 (Texas) to 2024 (Minnesota), with substantial cross-cohort variation in timing and reform type.

\subsection{Treatment Reversals}

Two states experienced significant policy reversals during the study period. Florida expanded automatic restoration under Governor Crist in 2007, reversed under Governor Scott in 2011, and re-expanded via Amendment 4 in 2019. Iowa followed a similar trajectory: expansion under Governor Vilsack in 2005, reversal under Governor Branstad in 2011, and re-expansion under Governor Reynolds in 2020. These reversal states are excluded from the primary analysis and included as a robustness check.


\section{Conceptual Framework}
\label{sec:framework}

\subsection{Direct Effects: Barrier Removal}

The most straightforward prediction of rights restoration is that it increases participation among the directly affected population---those whose voting rights are restored. Citizens who were previously ineligible to register and vote can now do so. This direct effect should manifest primarily in \textit{registration} rates, since registration is the legal prerequisite for voting and the binding constraint for the disenfranchised. Whether newly registered ex-felons actually \textit{vote} depends on additional factors: information about eligibility, motivation, access to polling places, and whether the restoration process imposes frictions (e.g., requirements to demonstrate completion of sentence, pay outstanding fines).

\subsection{The Civic Chill Hypothesis: Community Spillovers}

The more ambitious prediction concerns community-level spillovers. \citet{burch2013} argues that concentrated felon disenfranchisement in Black neighborhoods depresses civic engagement among non-disenfranchised residents through three channels:

\textbf{Norm erosion.} When a visible share of community members cannot vote, voting is no longer perceived as a universal civic norm. Non-disenfranchised residents may internalize the message that ``people like us don't vote,'' reducing their own participation.

\textbf{Mobilization infrastructure.} Political parties, community organizations, and campaigns invest less in voter mobilization in neighborhoods with high disenfranchisement rates, reducing turnout among eligible voters.

\textbf{Stigma and information.} Confusion about eligibility---particularly the widespread misperception that any criminal record disqualifies one from voting---may deter eligible citizens from registering or attempting to vote \citep{lee2014}.

If these mechanisms operate, restoring voting rights should produce positive spillovers: as the disenfranchised population shrinks, norms shift, mobilization increases, and confusion dissipates, raising participation among citizens who were never personally disenfranchised.

\subsection{Testable Predictions}

The framework generates four testable predictions:

\textit{Prediction 1:} Rights restoration increases Black voter registration relative to White registration in the same state (direct effect of barrier removal).

\textit{Prediction 2:} Rights restoration increases Black voter turnout relative to White turnout (civic chill reversal).

\textit{Prediction 3:} The turnout effect is present among low-felony-risk Black citizens (women 50+, college-educated), who are unlikely to be directly affected by restoration. This would constitute evidence of community-level spillovers.

\textit{Prediction 4:} The Hispanic-White turnout gap is unaffected by felon voting rights restoration, since these policies are not racially targeted at Hispanic communities in the same way (placebo test).


\section{Data}
\label{sec:data}

\subsection{CPS Voting and Registration Supplement}

To track the political pulse of these communities, I draw on individual-level data from over one million citizens across fifteen biennial waves of the Current Population Survey (CPS) Voting and Registration Supplement---the gold standard for measuring American political participation \citep{census2021}.

I use all fifteen biennial waves from 1996 through 2024, providing a panel structure with up to fourteen pre-treatment periods for the earliest-treated states. The CPS provides individual-level data on voting behavior (voted in the November election), voter registration status, and demographic characteristics including state of residence, race, Hispanic origin, age, sex, education, and citizenship status. The supplement weight (PWSSWGT) adjusts for survey design and nonresponse.

I restrict the sample to US citizens aged 18 and older---the universe eligible to vote. I define three racial/ethnic groups: non-Hispanic White, non-Hispanic Black, and Hispanic. For years 1996--2002, the Hispanic origin variable is not separately available in the Census CPS API, so racial categories in those years include Hispanics of each race; this affects a very small share of the Black population ($<3\%$) and does not meaningfully alter the results.

\subsection{Treatment Timing Database}

I compile a state-level panel of felon voting rights reforms from the National Conference of State Legislatures (NCSL), the Brennan Center for Justice, and Ballotpedia. For each reform, I code the first November even-year election at which the reform was operative---the first election after which restored citizens could register and vote. If a reform takes effect before the voter registration deadline (typically 30 days before the election), that election is coded as the first treated observation; otherwise, the next even-year election is used.

\subsection{Concurrent Voting Law Panel}

To address the concern that felon voting rights reforms may coincide with other changes to voting infrastructure, I compile a panel of three concurrent voting law categories from NCSL databases: strict photo voter ID requirements, same-day voter registration, and automatic voter registration. These are included as controls in robustness checks.

\subsection{Sample Construction}

The analysis sample consists of 1,099,677 citizen respondents aged 18+ across 51 states (including DC) and 15 election years. I collapse the microdata to state $\times$ race $\times$ year cells for the primary DD analysis, yielding 1,524 cells (762 per race). For the DDD mechanism test, I further disaggregate by felony-risk subgroup (low-risk: women 50+ or college-educated; high-risk: men 25--44 without college), yielding 2,976 cells. I drop reversal states (Florida and Iowa) from the primary sample, retaining 46 states with 1,370 cells.

\subsection{Summary Statistics}

\begin{table}[htbp]
\centering
\caption{Summary Statistics: New State vs Parent State Districts}
\label{tab:summary}
\begin{tabular}{lccc}
\hline\hline
 & New State & Parent State & $p$-value \\
\hline
Mean Nightlights & 8862.2 & 15587.7 & 0.000 \\
Mean Log(NL+1) & 8.215 & 9.160 & 0.000 \\
Population (2011, millions) & 1.25 & 2.37 & 0.000 \\
Literacy Rate & 0.583 & 0.556 & 0.071 \\
Ag. Worker Share & 0.362 & 0.434 & 0.001 \\
SC Share & 0.132 & 0.179 & 0.000 \\
ST Share & 0.276 & 0.083 & 0.000 \\
\hline
Districts & 55 & 159 & \\
\hline\hline
\end{tabular}
\begin{minipage}{0.9\textwidth}
\vspace{0.2cm}
\footnotesize \textit{Notes:} Pre-treatment means (1994--1999) for districts in newly created states (Uttarakhand, Jharkhand, Chhattisgarh) vs remaining districts in parent states (UP, Bihar, MP). Nightlights from DMSP calibrated luminosity. Population and sociodemographic characteristics from Census 2011. $p$-values from two-sample $t$-tests of equal means across districts.
\end{minipage}
\end{table}


\Cref{tab:summary} presents summary statistics for the analysis sample. Mean voter turnout is higher for White non-Hispanic citizens than for Black non-Hispanic citizens across both reform and control states, with the gap slightly larger in control states. Registration rates show a similar pattern. The median state-race-year cell contains 74 Black respondents and 1,048 White respondents, reflecting the CPS sample design.


\section{Empirical Strategy}
\label{sec:strategy}

\subsection{Primary DD Specification}

The primary specification estimates the change in the Black-White participation gap associated with felon voting rights restoration:

\begin{equation}
\label{eq:dd}
Y_{rst} = \alpha + \beta(\text{Black}_r \times \text{Reform}_{st}) + \gamma \text{Black}_r + \delta_{st} + \varepsilon_{rst}
\end{equation}

where $r$ indexes race (Black vs. White), $s$ indexes state, $t$ indexes election year, and $\delta_{st}$ are state $\times$ election-year fixed effects. The outcome $Y_{rst}$ is the weighted mean turnout (or registration) rate for race $r$ in state $s$ in election $t$. The coefficient $\beta$ captures the change in the Black-White participation gap in reform states after restoration, relative to the gap in control states. State $\times$ year fixed effects absorb all time-varying state-level confounders that affect both races equally---including concurrent voting law changes, economic conditions, and political climate.

Standard errors are clustered at the state level, the unit of treatment assignment. Observations are weighted by cell size (number of CPS respondents) to account for precision differences across state-race-year cells.

\subsection{Callaway-Sant'Anna Staggered DiD}

Because treatment timing varies across states (staggered adoption), standard two-way fixed effects (TWFE) estimates are potentially biased due to negative weighting of already-treated units as controls for later-treated units \citep{goodman2021}. I implement the \citet{callaway2021} estimator, which computes group-time average treatment effects---ATT$(g,t)$ for each treatment cohort $g$ and time period $t$---using only not-yet-treated or never-treated units as controls.

For this estimator, I use the state-level Black-White turnout \textit{gap} as the outcome (rather than race-specific cells), with never-treated states as the control group. I employ doubly-robust estimation and bootstrap standard errors with 1,000 iterations. I aggregate group-time ATTs to an event study (dynamic effects by periods since treatment), an overall ATT, and cohort-specific ATTs.

\subsection{Triple-Difference Mechanism Test}

To distinguish community-level spillovers from direct effects, I exploit demographic variation in felony conviction risk within the Black population:

\begin{equation}
\label{eq:ddd}
Y_{rskt} = \alpha + \lambda(\text{Black}_r \times \text{LowRisk}_k \times \text{Reform}_{st}) + \text{lower-order interactions} + \phi_s + \tau_t + \varepsilon_{rskt}
\end{equation}

where $k$ indexes felony-risk subgroup. \textit{Low-risk} Black citizens are defined as women aged 50+ or citizens with a college degree---groups with near-zero felony conviction rates (BJS data shows Black women's incarceration rate is approximately one-tenth of Black men's, and college-educated adults have negligible felony rates). \textit{High-risk} Black citizens are men aged 25--44 without a college degree, the demographic most directly affected by disenfranchisement.

The triple interaction $\lambda$ captures the differential change in the low-risk Black-White gap (relative to the high-risk Black-White gap) after reform. If $\lambda > 0$, low-risk Black citizens---who cannot plausibly benefit from direct restoration---are experiencing positive spillovers.

I also estimate a ``pure spillover'' DD restricted to low-risk subgroups only, which provides the most transparent test of community-level effects.

\subsection{Identification Assumptions}

The DD identification relies on parallel trends: absent reform, the Black-White participation gap in reform states would have evolved similarly to the gap in control states. I assess this assumption through:

\begin{enumerate}
\item \textbf{Event study pre-trends.} The Callaway-Sant'Anna dynamic event study provides visual and statistical evidence on pre-treatment trends.
\item \textbf{Placebo test.} The Hispanic-White gap should not respond to felon voting rights restoration, which is not racially targeted at Hispanic communities.
\item \textbf{Concurrent policy controls.} Including controls for voter ID laws, same-day registration, and automatic voter registration addresses the concern that reform states simultaneously change other voting policies.
\item \textbf{Robustness to sample restrictions.} Results should be robust to dropping reversal states, restricting to permanent legislative reforms, and separating presidential from midterm elections.
\end{enumerate}


\section{Results}
\label{sec:results}

\subsection{Treatment Rollout}

\begin{table}[H]
\centering
\caption{Felon Voting Rights Restoration: Treatment Timing}
\label{tab:reform_timing}
\begin{threeparttable}
\small
\begin{tabular}{llclc}
\toprule
State & First Election & Type & Description & Reversal \\
\midrule
Texas & 1998 & Legislative & Automatic restoration upon completion &  \\
Connecticut & 2002 & Legislative & Restored voting during probation &  \\
New Mexico & 2002 & Legislative & Automatic restoration after completion &  \\
Wyoming & 2004 & Legislative & First-time nonviolent felons after completion &  \\
Montana & 2006 & Legislative & Automatic restoration after completion &  \\
Nebraska & 2006 & Legislative & Eliminated 2-year waiting period &  \\
Maryland & 2008 & Legislative & Restored voting during probation &  \\
Rhode Island & 2008 & Legislative & Restored voting during probation &  \\
Washington & 2010 & Legislative & Restored voting during probation &  \\
Delaware & 2014 & Legislative & Eliminated 5-year waiting period &  \\
Virginia & 2016 & Executive & McAuliffe mass clemency for completed sentences &  \\
New Jersey & 2018 & Legislative & Restored voting during probation &  \\
New York & 2018 & Executive & Parole voting by executive order &  \\
California & 2020 & Ballot & Prop 17: restored parole voting &  \\
Colorado & 2020 & Legislative & Restored voting during parole &  \\
Florida & 2020 & Ballot & Amendment 4 (passed Nov 2018, effective Jan 2019) & Yes \\
Kentucky & 2020 & Executive & Beshear executive order for nonviolent felons &  \\
Louisiana & 2020 & Legislative & Restored after completion (was 5-year wait) &  \\
Nevada & 2020 & Legislative & Automatic restoration upon release &  \\
North Dakota & 2020 & Legislative & Restored voting during probation &  \\
Oregon & 2022 & Legislative & Expanded: voting while incarcerated &  \\
Minnesota & 2024 & Legislative & Automatic restoration upon release &  \\
\bottomrule
\end{tabular}
\begin{tablenotes}[flushleft]
\small
\item Notes: Treatment coded as the first November even-year election at which the reform was operative. ``First Election'' is the first CPS Voting Supplement wave at which the state is coded as treated. Reversal states experienced policy rollback and are excluded from the main sample. Sources: NCSL, Brennan Center, Ballotpedia.
\end{tablenotes}
\end{threeparttable}
\end{table}


\Cref{tab:reform_timing} presents the treatment rollout. Twenty-two states adopted felon voting rights restoration between 1998 and 2024, with the largest cohort (seven states) in 2020. Reform types span legislative action (most common), executive orders, and ballot initiatives. \Cref{fig:map} displays the geographic distribution, showing broad coverage across regions---not concentrated in any single part of the country.

\begin{figure}[H]
\centering
\includegraphics[width=\textwidth]{figures/fig1_treatment_map.pdf}
\caption{Staggered Adoption of Felon Voting Rights Restoration, 1996--2024}
\label{fig:map}
\end{figure}

\begin{figure}[H]
\centering
\includegraphics[width=0.85\textwidth]{figures/fig6_rollout_timeline.pdf}
\caption{Treatment Rollout Timeline by State and Reform Type}
\label{fig:rollout}
\end{figure}

\subsection{Descriptive Evidence}

\Cref{fig:trends} plots raw voter turnout by race and reform status over time. Several patterns are visible. First, turnout fluctuates with the presidential election cycle (higher in presidential years), with this pattern more pronounced for White voters. Second, both reform and control states show similar trajectories for White turnout. Third, Black turnout in reform states tracks Black turnout in control states reasonably well in the pre-reform period but diverges modestly after the main wave of reforms begins around 2006--2008.

\begin{figure}[H]
\centering
\includegraphics[width=\textwidth]{figures/fig2_raw_trends.pdf}
\caption{Voter Turnout by Race and Reform Status, 1996--2024}
\label{fig:trends}
\end{figure}

\Cref{fig:gap} shows the Black-White turnout gap (Black minus White) separately for reform and control states. The gap is consistently negative---Black turnout is below White turnout---in both groups. In the early period (1996--2004), the gaps evolve roughly in parallel, supporting the parallel trends assumption. After the reform wave intensifies, the gap in reform states appears to widen modestly relative to control states.

\begin{figure}[H]
\centering
\includegraphics[width=\textwidth]{figures/fig3_turnout_gap.pdf}
\caption{Black-White Voter Turnout Gap by Reform Status, 1996--2024}
\label{fig:gap}
\end{figure}

\subsection{Main DD Results}


\begin{table}[htbp]
   \caption{\label{tab:main_dd} Effect of Felon Voting Rights Restoration on the Black-White Participation Gap}
   \bigskip
   \centering
   \begin{tabular}{lcccc}
      \toprule
       & Voter Turnout & Registration Rate & \multicolumn{2}{c}{Voter Turnout}\\
       & Turnout & Registration & \multicolumn{2}{c}{Turnout} \\ 
                                      & (1)            & (2)            & (3)            & (4)\\  
      \midrule 
      Black $\times$ Post-Reform      & -0.0373$^{**}$ & 0.0225$^{***}$ & -0.0315$^{**}$ & -0.0373$^{**}$\\   
                                      & (0.0148)       & (0.0056)       & (0.0145)       & (0.0148)\\   
      Black                           & -0.0128        & -0.0063$^{*}$  & -0.0133        & -0.0128\\   
                                      & (0.0088)       & (0.0033)       & (0.0089)       & (0.0088)\\   
      Post-Reform                     &                &                & 0.0091         &   \\   
                                      &                &                & (0.0088)       &   \\   
       \\
      R$^2$                           & 0.95433        & 0.96340        & 0.86441        & 0.95433\\  
      Observations                    & 1,370          & 1,370          & 1,375          & 1,370\\  
      State $\times$ Year FE          & Yes            & Yes            & No             & Yes\\  
      Separate State + Year FE        & No             & No             & Yes            & No\\  
      Voting Law Controls             & No             & No             & No             & Yes\\  
       \\
      state\_fips-year fixed effects  & $\checkmark$   & $\checkmark$   &                & $\checkmark$\\   
      state\_fips fixed effects       &                &                & $\checkmark$   & \\  
      year fixed effects              &                &                & $\checkmark$   & \\  
      \bottomrule
   \end{tabular}
   
   \par \raggedright 
   Standard errors clustered at the state level (49 clusters) in parentheses.\\
   * p$<$0.10, ** p$<$0.05, *** p$<$0.01.\\
   Outcome: weighted mean turnout/registration rate at the state-race-year level.\\
   Black $\times$ Post-Reform captures the change in the Black-White gap after reform.\\
   Reversal states (FL, IA) excluded. 22 reform states, 27 control states.\\
   Column (3) has 5 additional observations because additive FE do not drop singleton cells.
\end{table}




\Cref{tab:main_dd} presents the primary DD results. Column (1) reports the main specification with state $\times$ year fixed effects and voter turnout as the outcome. The coefficient on Black $\times$ Post-Reform is $-0.037$ ($\text{SE} = 0.015$, $p = 0.015$), indicating that the Black-White turnout gap widens by 3.7 percentage points after rights restoration. This is a precisely estimated, economically meaningful effect---roughly half the size of the overall Black-White turnout gap.

Column (2) reports the registration outcome. Here the coefficient reverses sign: $+0.023$ ($\text{SE} = 0.006$, $p < 0.001$). After restoration, Black registration increases by 2.3 percentage points relative to White registration. This is a strongly significant effect in the expected direction---consistent with the direct barrier-removal channel.

The contrast between columns (1) and (2) is the paper's central finding. Rights restoration removes the legal barrier to registration, producing a measurable increase in Black registration rates. But this registration gain does not translate into turnout---instead, the turnout gap actually widens. This pattern is consistent with ``registered but not voting'': newly registered citizens (predominantly ex-felons) have lower turnout propensity than the existing electorate, mechanically reducing the observed Black turnout rate even as registration rises.

Columns (3) and (4) show robustness to alternative fixed effect structures and concurrent voting law controls. Column (3) uses separate state and year fixed effects rather than their interaction; the coefficient is similar at $-0.031$ ($p = 0.035$). Column (4) includes controls for strict voter ID laws, same-day registration, and automatic voter registration; these are absorbed by the state $\times$ year FE (as they should be, since they vary at the state-year level), and the turnout coefficient is unchanged.

\subsection{Callaway-Sant'Anna Event Study}

\Cref{fig:event_study} presents the event study using the \citet{callaway2021} estimator with dynamic aggregation. The outcome is the state-level Black-White turnout gap, with never-treated states as controls. Pre-treatment coefficients at event times $-6$, $-4$, and $-2$ are close to zero ($-0.031$, $+0.039$, $-0.058$), though the $t=-2$ coefficient is negative and modestly large. Post-treatment effects are positive at event times $0$ ($+0.066$), $+2$ ($+0.047$), $+4$ ($+0.023$), and $+6$ ($+0.119$), but none individually reaches conventional significance. The overall ATT is $+0.053$ ($\text{SE} = 0.032$), positive but not statistically significant.

\begin{figure}[H]
\centering
\includegraphics[width=\textwidth]{figures/fig4_event_study.pdf}
\caption{Event Study: Effect of Rights Restoration on Black-White Turnout Gap}
\label{fig:event_study}
\end{figure}

The sign discrepancy between the cell-level DD ($-0.037$) and the CS overall ATT ($+0.053$) reflects three differences in estimation. First, the DD uses cell-size-weighted turnout rates (giving more influence to populous states), while the CS estimator uses unweighted state-level gaps. Second, the CS estimator requires a balanced panel per cohort and may drop state-year observations that the DD retains. Third, the CS aggregation weights treatment cohorts equally, so small states in single-state cohorts (e.g., Wyoming in 2004) can receive disproportionate influence. The overall pattern is consistent with a null or weak average effect whose sign depends on weighting choices---a common finding in staggered DiD with heterogeneous treatment effects across cohorts. The full event study coefficients are reported in \Cref{tab:event_study} in the appendix.

\subsection{DDD Mechanism Test: Community Spillovers}


\begin{table}[htbp]
   \caption{\label{tab:ddd} Triple-Difference Mechanism Test: Community Spillovers vs. Direct Effects}
   \bigskip
   \centering
   \begin{tabular}{lccc}
      \toprule
       & \multicolumn{3}{c}{Voter Turnout}\\
                                     & Full DDD        & Low-Risk DD     & High-Risk DD \\   
                                     & (1)             & (2)             & (3)\\  
      \midrule 
      triple                         & 0.0219          &                 &   \\   
                                     & (0.0146)        &                 &   \\   
      Black $\times$ Post-Reform     & -0.0484$^{**}$  & -0.0262$^{***}$ & -0.0359\\   
                                     & (0.0204)        & (0.0096)        & (0.0229)\\   
      Low-Risk $\times$ Post-Reform  & 0.0132          &                 &   \\   
                                     & (0.0093)        &                 &   \\   
      black\_low\_risk               & -0.0209$^{***}$ &                 &   \\   
                                     & (0.0074)        &                 &   \\   
      Black                          & 0.0328$^{***}$  & 0.0132$^{*}$    & 0.0244$^{**}$\\   
                                     & (0.0111)        & (0.0066)        & (0.0116)\\   
      low\_risk                      & 0.2976$^{***}$  &                 &   \\   
                                     & (0.0044)        &                 &   \\   
      Post-Reform                    & -0.0062         &                 &   \\   
                                     & (0.0110)        &                 &   \\   
       \\
      R$^2$                          & 0.90334         & 0.83893         & 0.71324\\  
      Observations                   & 2,856           & 1,440           & 1,416\\  
      Sample                         & All Subgroups   & Low-Risk Only   & High-Risk Only\\  
      State + Year FE                & Yes             & Yes             & Yes\\  
       \\
      state\_fips fixed effects      & $\checkmark$    & $\checkmark$    & $\checkmark$\\   
      year fixed effects             & $\checkmark$    & $\checkmark$    & $\checkmark$\\   
      \bottomrule
   \end{tabular}
   
   \par \raggedright 
   Standard errors clustered at the state level (49 clusters) in parentheses.\\
   * p$<$0.10, ** p$<$0.05, *** p$<$0.01.\\
   Low-risk: Black women 50+ or college-educated. High-risk: Black men 25-44, no college.\\
   Column 1: Full DDD. Column 2: DD among low-risk only (pure spillover test).\\
   Column 3: DD among high-risk only (direct + spillover). Reversal states excluded.
\end{table}




\Cref{tab:ddd} presents the DDD mechanism test. Column (1) reports the full triple-difference specification. The triple interaction (Black $\times$ Low-Risk $\times$ Post-Reform) is $+0.022$ ($\text{SE} = 0.015$, $p = 0.14$)---positive but not statistically significant. This provides no support for the civic chill hypothesis at conventional significance levels.

Column (2) provides the purest spillover test: a DD restricted to low-felony-risk subgroups only (women 50+, college-educated). If community-level spillovers operate, we should see low-risk Black citizens increasing their turnout relative to equivalent White citizens after reform. Instead, the coefficient is \textit{negative}: $-0.026$ ($\text{SE} = 0.010$, $p = 0.009$). Low-risk Black citizens' turnout actually falls relative to White counterparts after reform. Column (3) shows a similar pattern for high-risk subgroups ($-0.036$, $p = 0.12$), though the effect is imprecisely estimated.

\begin{figure}[H]
\centering
\includegraphics[width=\textwidth]{figures/fig5_ddd_subgroups.pdf}
\caption{Turnout by Race and Felony-Risk Subgroup in Reform States}
\label{fig:ddd}
\end{figure}

\Cref{fig:ddd} plots turnout trends by race and risk subgroup. Low-risk Black citizens (solid blue line) maintain substantially higher turnout than high-risk Black citizens (dashed blue line), as expected. The gap between low-risk Black and low-risk White citizens does not systematically narrow after the reform wave, consistent with the absence of spillover effects.

\subsection{Robustness}

\begin{table}[htbp]
\centering
\caption{Robustness: VIIRS 2020 RDD Estimates}
\label{tab:robustness}
\begin{tabular}{llcccc}
\hline\hline
Specification & & Estimate & SE & $p$-value & $N_{\text{eff}}$ \\
\hline
\multicolumn{6}{l}{\textit{Panel A: Bandwidth Sensitivity}} \\
& $h = 53.9 $ & -0.0211 & (0.0389) & 0.707 & 37,952 \\
& $h = 80.9 $ & -0.0291 & (0.0318) & 0.596 & 57,221 \\
& $h = 107.8 $ & -0.0253 & (0.0276) & 0.264 & 76,214 \\
& $h = 134.8 $ & -0.0205 & (0.0248) & 0.197 & 95,018 \\
& $h = 161.7 $ & -0.0185 & (0.0226) & 0.203 & 113,866 \\
& $h = 215.6 $ & -0.0158 & (0.0196) & 0.248 & 150,927 \\
\multicolumn{6}{l}{\textit{Panel B: Polynomial Order}} \\
& $p = 1 $ & -0.0253 & (0.0225) & 0.190 & 76,214 \\
& $p = 2 $ & -0.0272 & (0.0244) & 0.201 & 126,275 \\
& $p = 3 $ & -0.0322 & (0.0295) & 0.248 & 142,011 \\
\multicolumn{6}{l}{\textit{Panel C: Donut RDD ($\pm 25$ excluded)}} \\
& VIIRS 2015 & -0.0850 & (0.0621) & 0.099 & 37,025 \\
& VIIRS 2018 & -0.0601 & (0.0601) & 0.254 & 40,531 \\
& VIIRS 2021 & -0.0645 & (0.0617) & 0.202 & 39,153 \\
& VIIRS 2023 & -0.0500 & (0.0602) & 0.311 & 41,235 \\
\hline\hline
\end{tabular}
\begin{tablenotes}\small
\item \textit{Notes:} All specifications use $\text{asinh}(\text{VIIRS nightlights})$ as the dependent variable and Census 2001 population as the running variable. Panel A varies the bandwidth around the MSE-optimal choice ($h^* = 107.8$); bandwidths are forced via \texttt{rdrobust(h=...)}, which may yield different bias bandwidths and thus different SE/p-values than Table~\ref{tab:main_rdd} (which uses automatic bandwidth selection). Panel B varies the polynomial order with MSE-optimal bandwidth. Panel C excludes villages within $\pm 25$ persons of the 500 threshold to address heaping.
\end{tablenotes}
\end{table}


\Cref{tab:robustness} reports the Black $\times$ Post-Reform coefficient across alternative specifications.

\textbf{Including reversal states.} Adding Florida and Iowa to the sample yields $-0.036$ ($p = 0.017$), nearly identical to the main estimate. The reversal states' complex treatment histories do not drive the results.

\textbf{Permanent reforms only.} Restricting to legislative and ballot-initiative reforms (excluding executive orders, which are more easily reversed) yields $-0.040$ ($p = 0.021$). If anything, the effect is slightly larger for permanent institutional reforms.

\textbf{Placebo: Hispanic-White gap.} The placebo test on the Hispanic-White turnout gap yields $-0.010$ ($p = 0.31$)---economically small and statistically insignificant. This is reassuring: felon voting rights restoration, which disproportionately affects Black communities, should not systematically affect Hispanic turnout relative to White turnout. The null placebo supports the interpretation that our main results are specific to the Black population.

\textbf{Registration outcome.} The positive registration effect ($+0.023$, $p < 0.001$) confirms that restoration operates through the barrier-removal channel, increasing Black registration even as turnout declines.

\textbf{Unweighted estimation.} Without weighting by cell size, the turnout coefficient is $-0.040$ ($p = 0.05$), similar in magnitude and marginally significant.

\textbf{Presidential vs. midterm years.} The effect is virtually identical in presidential election years ($-0.038$, $p = 0.030$) and midterm years ($-0.038$, $p = 0.026$), ruling out composition effects from election-cycle variation.

\begin{figure}[H]
\centering
\includegraphics[width=\textwidth]{figures/fig7_robustness.pdf}
\caption{Robustness: Black $\times$ Post-Reform Coefficient Across Specifications}
\label{fig:robustness}
\end{figure}

\Cref{fig:robustness} summarizes the robustness analysis visually. The point estimates cluster tightly around $-0.037$, with confidence intervals that consistently exclude large positive effects. The registration coefficient stands apart as the only positive estimate, highlighting the divergence between formal inclusion (registration) and substantive participation (turnout).

\textbf{Placebo: Hispanic-White gap.}

\begin{figure}[H]
\centering
\includegraphics[width=\textwidth]{figures/fig8_placebo.pdf}
\caption{Placebo Test: Hispanic-White Voter Turnout Gap by Reform Status}
\label{fig:placebo}
\end{figure}

\Cref{fig:placebo} shows the Hispanic-White turnout gap over time for reform and control states. There is no systematic divergence between the two groups---the gap evolves similarly regardless of felon voting rights restoration, as predicted by the placebo logic.


\section{Discussion}
\label{sec:discussion}

\subsection{Interpreting the Registration-Turnout Divergence}

The central finding---rights restoration increases registration but not turnout---admits several interpretations.

\textbf{Composition effect.} The most mechanical interpretation is that newly registered ex-felons have lower turnout propensity than existing registrants. A back-of-envelope calculation suggests this channel is quantitatively plausible. The Sentencing Project estimates that approximately 3.5\% of Black voting-age adults are disenfranchised due to felony convictions \citep{uggen2022}. If restoration makes even half of these individuals eligible to register, and if their turnout rate is 20--30\% (consistent with evidence on recently restored citizens; \citealt{meredith2009}), while the existing Black turnout rate is approximately 60\%, the mechanical dilution of the Black turnout rate would be on the order of 1--2 percentage points---a meaningful fraction of the observed 3.7 pp decline. The remaining gap may reflect measurement (CPS over-reporting of turnout may interact with new registrant status) or additional behavioral channels. The key point is that the composition channel alone can plausibly account for much of the observed effect without invoking any change in existing voters' behavior.

\textbf{Mobilization deficit.} Even after formal restoration, practical barriers may prevent voting: confusion about eligibility, outstanding fines or fees (as in Florida's post-Amendment 4 implementation), difficulty navigating re-registration, or lack of information about polling locations. Restoration without accompanying mobilization infrastructure---voter education campaigns, registration drives, election-day transportation---may be insufficient to convert registration into turnout.

\textbf{Stigma persistence.} The civic chill may persist even after formal restoration. If years or decades of exclusion have eroded civic norms and institutional engagement in Black communities, simply changing the law may not reverse the accumulated damage. Norm change may require sustained community organizing, not just policy reform.

\subsection{Why the Cell DD and CS Estimator Disagree}

The cell-level DD and Callaway-Sant'Anna estimator yield different results: the DD finds a significant widening of the turnout gap ($-3.7$ pp), while the CS finds a positive but insignificant narrowing ($+5.3$ pp). Three factors explain this divergence. First, the DD weights by cell size (CPS respondents per state-race-year cell), giving populous states proportionally more influence. The CS estimator weights cohorts equally in its simple aggregation, so a single-state cohort receives the same weight as a seven-state cohort. Second, the DD identifies the effect through within-state-year variation across races, while the CS identifies it through across-state comparison of the turnout gap within a cohort, using never-treated states as controls. Third, the CS doubly-robust estimator includes propensity-score reweighting, which further changes the effective comparison group. The overall pattern is consistent with a null or weak average treatment effect that is sensitive to specification choices---a common finding in staggered DiD applications where treatment effects are heterogeneous across cohorts and time.

\subsection{Limitations}

Several limitations deserve acknowledgment. First, the CPS cannot identify individual felon status, so I cannot directly observe the ``treated'' population. The design relies on race-level averages, which combine direct and spillover effects. The DDD helps separate these, but the demographic proxies for felony risk are imperfect.

Second, the biennial frequency of the CPS Voting Supplement means that the analysis has at most 15 time periods, and some treatment cohorts contribute only one or two post-treatment observations. This limits power for the event study and cohort-specific estimates.

Third, selection into treatment is not random. States that adopt felon voting rights restoration may differ systematically from those that do not---they may be more politically progressive, have stronger civil rights organizations, or be responding to specific political moments. The parallel trends assumption mitigates this concern if the factors driving adoption do not differentially affect Black vs. White turnout, but this is ultimately untestable for the post-treatment period.

Fourth, the pre-2003 CPS data does not separately identify Hispanic origin, introducing a minor compositional difference in the definition of ``non-Hispanic Black'' across the early and later periods of the sample.


\section{Conclusion}
\label{sec:conclusion}

This paper tests whether restoring felon voting rights increases Black political participation beyond the directly affected population. Using the staggered adoption of restoration laws across 22 US states and the CPS Voting Supplement from 1996 to 2024, I find that restoration significantly increases Black voter registration (+2.3 pp relative to White, $p < 0.001$) but does not increase---and may actually decrease---Black voter turnout relative to White turnout ($-3.7$ pp, $p = 0.015$). A triple-difference exploiting within-race variation in felony risk finds no evidence of community-level spillovers on turnout.

These results carry a simple but important policy implication: removing the legal barrier to voting is necessary for inclusion but not sufficient for participation. The ``civic chill'' of mass disenfranchisement, if it exists, is not reversed by changing the law alone. Closing racial participation gaps requires more than formal eligibility---it requires the mobilization infrastructure, civic education, and sustained community engagement that translate registration into turnout. For the millions of Americans whose voting rights have been restored in recent years, the challenge has shifted from \textit{can they vote} to \textit{will they vote}. Answering that question requires investments that go well beyond the statute books.


\section*{Acknowledgements}

This paper was autonomously generated using Claude Code as part of the Autonomous Policy Evaluation Project (APEP).

\noindent\textbf{Project Repository:} \url{https://github.com/SocialCatalystLab/ape-papers}

\noindent\textbf{Contributors:} @ai1scl

\noindent\textbf{First Contributor:} \url{https://github.com/ai1scl}

\label{apep_main_text_end}
\newpage
\bibliography{references}

\newpage
\appendix

\section{Data Appendix}
\label{app:data}

\subsection{CPS Voting Supplement: Variable Construction}

The CPS Voting and Registration Supplement is administered in November of each even year as part of the Current Population Survey. I access the data through the Census Bureau's CPS API (\texttt{api.census.gov/data/\{year\}/cps/voting/nov}), which provides individual-level microdata for the voting supplement universe.

\textbf{Turnout.} Coded from the Census variable PES1: ``In any election held in November [year], did [you/name] vote?'' Responses of ``Yes'' (code 1) are coded as voted; ``No'' (code 2) as did not vote. Negative codes ($-1$, $-2$, $-3$, $-9$) indicating refusal, don't know, or not in universe are excluded.

\textbf{Registration.} Coded from PES2: ``Were [you/name] registered to vote?'' Same coding as turnout.

\textbf{Race.} Pre-2003: PERACE (1=White, 2=Black, 3=American Indian/Alaska Native, 4=Asian/Pacific Islander). Post-2003: PTDTRACE (same codes 1--4 for single-race, 5--26 for multi-race combinations). I use codes 1 and 2 consistently across all years.

\textbf{Hispanic origin.} PEHSPNON (1=Hispanic, 2=Non-Hispanic), available 2004--2024 only. For 1996--2002, race categories include Hispanics of each race.

\textbf{Citizenship.} PRCITSHP: codes 1--4 indicate US citizenship (native born or naturalized); code 5 indicates non-citizen. Only citizens are included in the analysis sample.

\textbf{Weights.} PWSSWGT (voting supplement weight) accounts for survey design, nonresponse, and post-stratification to population controls.

\subsection{Treatment Timing Construction}

Treatment timing is coded as the first November even-year election at which the reform was operative for voter registration. If the reform effective date precedes the voter registration deadline for a November election (typically 30 days before), that election is coded as the first treated observation. If the effective date falls after the registration deadline, the next even-year election is used.

For example: Florida's Amendment 4 passed in November 2018 and took effect January 8, 2019. The first eligible election is November 2020. Connecticut's 2001 probation voting restoration took effect before November 2002; the first eligible election is 2002.

\subsection{Sample Restrictions}

The analysis proceeds through the following sample filters:

\begin{enumerate}
\item Start: 1,889,620 CPS Voting Supplement respondents, 1996--2024
\item Restrict to valid voting responses (PES1 $>$ 0): drop 656,013 missing/refused
\item Restrict to positive supplement weights: negligible additional drops
\item Restrict to US citizens aged 18+: drop 133,930 non-citizens and minors
\item Restrict to non-Hispanic White, non-Hispanic Black, and Hispanic respondents
\item Final sample: 1,099,677 citizen respondents
\end{enumerate}

\section{Identification Appendix}
\label{app:id}

\subsection{Event Study Coefficients}

\begin{table}[H]
\centering
\caption{Event Study Coefficients: Callaway-Sant'Anna Estimator}
\label{tab:event_study}
\begin{threeparttable}
\begin{tabular}{cccc}
\toprule
Event Time & ATT & SE & 95\% CI \\
\midrule
-6 & -0.0313 & (0.0243) & [-0.0789, 0.0163] \\
-4 & 0.0393 & (0.0485) & [-0.0558, 0.1344] \\
-2 & -0.0577 & (0.0361) & [-0.1285, 0.0131] \\
0 & 0.0660 & (0.0461) & [-0.0245, 0.1564] \\
2 & 0.0473 & (0.0372) & [-0.0256, 0.1202] \\
4 & 0.0229 & (0.0294) & [-0.0347, 0.0805] \\
6 & 0.1194 & (0.0754) & [-0.0283, 0.2672] \\
\\
Overall ATT & \multicolumn{3}{c}{0.0532 (0.0309)} \\
\bottomrule
\end{tabular}
\begin{tablenotes}[flushleft]
\small
\item Notes: Callaway-Sant'Anna (2021) group-time ATTs aggregated to event time. Each unit of event time = one biennial election cycle (2 years). Bootstrap standard errors with 1,000 iterations. Control group: never-treated states.
\end{tablenotes}
\end{threeparttable}
\end{table}


\subsection{Pre-Trends Assessment}

The Callaway-Sant'Anna event study (\Cref{fig:event_study}) shows pre-treatment coefficients at event times $-6$, $-4$, and $-2$ of $-0.031$, $+0.039$, and $-0.058$ respectively. None are individually significant, though the $t=-2$ coefficient is the largest pre-treatment estimate. The pre-trend pattern does not suggest a systematic trend in one direction, though the oscillation raises some concern about noise in the state-level gap measure.

\subsection{Parallel Trends Sensitivity}

I conduct a HonestDiD sensitivity analysis \citep{rambachan2023} using the relative magnitudes approach. This asks: how large would violations of parallel trends need to be, relative to the observed pre-treatment coefficient pattern, to overturn the results? The analysis indicates that the positive CS overall ATT ($+5.3$ pp) would be overturned by moderate violations ($\bar{M} \approx 1$), confirming that the CS turnout result is fragile and should not be interpreted as strong evidence of a positive effect. By contrast, the cell-level DD estimate ($-3.7$ pp) and the registration effect ($+2.3$ pp) are robust across all specifications. The appropriate summary is that registration robustly increases while turnout effects are null or negative depending on weighting---the ``registered but not voting'' pattern holds regardless of estimator choice.

\section{Robustness Appendix}
\label{app:robust}

\subsection{Sun-Abraham Estimator}

As an alternative to Callaway-Sant'Anna, I implement the \citet{sun2021} interaction-weighted estimator via the \texttt{fixest::sunab()} function. The Sun-Abraham estimator yields a turnout coefficient of $-0.036$ ($\text{SE} = 0.023$, $p = 0.12$), consistent in sign and magnitude with the cell-level DD. Collinearity warnings arise from single-state treatment cohorts, but the overall estimate is well-identified.

\subsection{Concurrent Voting Law Controls}

The state $\times$ year fixed effects in the main specification absorb all state-time-varying confounders, including concurrent voting laws. When I replace state $\times$ year FE with separate state and year FE and add controls for strict voter ID, same-day registration, and automatic voter registration, the coefficient is similar ($-0.031$ vs. $-0.037$). The concurrent voting law controls do not meaningfully change the results because the DD structure---comparing Black vs. White within the same state-year---already differences out race-neutral policy shocks.

\subsection{Cell Size and Small-Sample Concerns}

Several state-race-year cells have very small samples (minimum 1 for Black cells in small states). I verify that results are robust to dropping cells with fewer than 20 respondents and to using unweighted estimation. Both alternatives yield similar coefficients (see \Cref{tab:robustness}).


\section{Additional Figures and Tables}

No additional figures or tables beyond those presented in the main text.

\end{document}
