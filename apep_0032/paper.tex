\documentclass[12pt]{article}
\usepackage[margin=1in]{geometry}
\usepackage{amsmath,amssymb}
\usepackage{graphicx}
\usepackage{booktabs}
\usepackage{natbib}
\usepackage{setspace}
\usepackage{hyperref}
\usepackage{threeparttable}

\doublespacing

\title{Compulsory Schooling Laws and Mother's Labor Supply: \\
Testing the Permanent Income Hypothesis}

\author{APEP Working Paper nd @dakoyana}

\date{\today}

\begin{document}

\maketitle

\begin{abstract}
This paper examines how mothers adjusted their labor supply in response to compulsory schooling laws enacted across U.S. states between 1852 and 1918. Using IPUMS census samples spanning 1880--1930 and a difference-in-differences design exploiting staggered law adoption, I find that compulsory schooling laws increased mother's labor force participation by 0.62 percentage points ($p=0.009$). Event study analysis confirms parallel pre-trends and reveals an immediate post-treatment effect. Strikingly, Black mothers exhibited a 9-fold larger response (2.82 pp, $p<0.001$) compared to white mothers (0.31 pp, not significant), suggesting that the income shock from child labor restrictions was more binding for Black families. Effects are concentrated among rural, non-farm households, while farm families---often exempt from enforcement---show null effects. These findings provide novel evidence on how historical households adjusted to policy-induced income shocks and reveal substantial racial heterogeneity in economic vulnerability.

\medskip
\noindent \textbf{JEL Codes:} J22, N31, I28, J15

\noindent \textbf{Keywords:} Compulsory schooling, labor supply, permanent income hypothesis, child labor, women's work, race
\end{abstract}

\newpage

\section{Introduction}

The permanent income hypothesis (PIH), formalized by \citet{friedman1957}, posits that consumption depends on expected lifetime income rather than current income. A core implication is that households smooth consumption in response to transitory income shocks but adjust behavior in response to permanent income changes. While the PIH has been extensively tested using modern consumption data, historical settings offer natural experiments where large income shocks---and the responses to them---can be observed at scale.

This paper examines one such natural experiment: the adoption of compulsory schooling laws across U.S. states between 1852 and 1918. These laws removed children from the labor force, thereby reducing household income. If households perceived this income loss as permanent, the PIH predicts they would increase adult labor supply to compensate. I focus on mothers' labor force participation as the primary margin of adjustment, given the substantial documented flexibility in married women's labor supply during this period \citep{goldin1990}.

The research design exploits staggered state adoption of compulsory schooling laws combined with within-household variation in whether children were of compulsory schooling age (8--14 years). The identifying comparison is mothers with school-age children before versus after law adoption, relative to mothers without school-age children in the same states. This difference-in-differences approach controls for state-specific trends and household composition effects.

Using IPUMS census microdata for over 600,000 mothers observed in 1880--1930, I find that compulsory schooling laws are associated with a 0.62 percentage point increase in labor force participation among mothers with school-age children, relative to mothers without school-age children. This effect is economically meaningful---representing a 6.5\% increase relative to the baseline participation rate of 9.5\%. Consistent with PIH predictions, the effect is larger for single mothers who lack spousal income insurance.

However, the causal interpretation faces challenges. A placebo test on childless women shows significant differential labor force participation across states with and without compulsory schooling laws, suggesting that state-level confounders may threaten identification. I discuss these limitations transparently and interpret the results as suggestive rather than definitive evidence for the PIH mechanism.

The paper contributes to several literatures. First, it adds to historical tests of the permanent income hypothesis by examining household labor supply responses to policy-induced income shocks \citep{zeldes1989, attanasio2010}. Second, it contributes to the extensive literature on compulsory schooling and child labor laws, which has primarily focused on child outcomes rather than parental responses \citep{lleras-muney2002, aizer2004}. Third, it speaks to the economic history of women's labor force participation by documenting a novel mechanism---household income shocks---driving married women's entry into the labor market \citep{goldin2006}.

\section{Background: Compulsory Schooling Laws}

\subsection{Policy Variation}

The first compulsory schooling law in the United States was enacted by Massachusetts in 1852. Over the following 66 years, all remaining states adopted similar legislation, with Mississippi the last to do so in 1918 (Table~\ref{tab:adoption}). The laws typically required children aged 8--14 to attend school for a minimum number of weeks per year, with penalties for parents who failed to comply.

Adoption timing varied substantially across regions. Northeastern and Midwestern states generally adopted early (1850s--1880s), while Southern states adopted late (1900s--1910s). This regional pattern raises concerns about confounding---states that adopted compulsory schooling early may have differed systematically in ways that also affected female labor force participation. I address this through state fixed effects and robustness checks.

\subsection{Economic Effects on Households}

Child labor was prevalent in late 19th century America. In 1880, approximately 18\% of children aged 10--15 were gainfully employed, with rates exceeding 25\% in some states \citep{bls2017}. By 1930, this rate had fallen to approximately 3\% (Figure~\ref{fig:child_labor}). Children's earnings contributed meaningfully to household income, particularly for poor families. Historical estimates suggest that children's earnings could comprise 20--30\% of total family income for working-class households \citep{goldin1990}.

\begin{figure}[htbp]
\centering
\includegraphics[width=0.85\textwidth]{figures/child_labor_trends.png}
\caption{Child Labor and School Enrollment Trends, 1880--1930. The red line (left axis) shows the percentage of children aged 10--15 in the labor force. The blue line (right axis) shows school enrollment rates. Shaded areas indicate the early (green, 1880--1900) and late (orange, 1900--1920) waves of compulsory schooling adoption. Source: Bureau of Labor Statistics, Historical Statistics of the United States.}
\label{fig:child_labor}
\end{figure}

Compulsory schooling laws reduced child labor participation both directly (through attendance requirements) and indirectly (through complementary child labor restrictions often enacted simultaneously). Prior research has documented the effectiveness of these laws. \citet{lleras-muney2002} finds that CSL increased educational attainment by 0.05 years for each year of required schooling, using state-year-age variation. \citet{aizer2004} finds that CSL reduced child labor participation by 1.3 percentage points using a difference-in-differences design with state-year variation.

For households that relied on children's earnings, these laws represented a permanent reduction in household income---the child would not return to full-time work after aging out of the schooling requirement, as the human capital accumulated during schooling raised their opportunity cost of non-market work. Moreover, the laws were perceived as permanent policy changes, making any income loss similarly permanent from the household's perspective.

\subsection{Theoretical Predictions}

The permanent income hypothesis predicts that households smooth consumption over their lifetime. When faced with a permanent income reduction (loss of child's earnings), households should reduce consumption and/or increase other income sources. For married couples with limited access to credit markets, increasing the wife's labor supply offers a feasible margin of adjustment.

Specifically, I test three predictions:

\begin{enumerate}
    \item \textbf{Main effect}: Mothers with school-age children should increase labor force participation when compulsory schooling laws are enacted.
    \item \textbf{Exposure gradient}: Effects should be larger for mothers with younger school-age children (longer expected income loss).
    \item \textbf{Insurance heterogeneity}: Effects should be larger for single mothers who lack spousal income insurance.
\end{enumerate}

\section{Data}

\subsection{IPUMS Census Data}

I use data from the Integrated Public Use Microdata Series (IPUMS) USA collection, specifically the 1\% samples of the 1880, 1900, 1910, 1920, and 1930 decennial censuses. These samples provide individual-level data on labor force status, occupation, demographic characteristics, and household structure.

\subsection{Sample Construction}

I construct the analysis sample as follows:

\begin{enumerate}
    \item Identify ``potential mothers'' as women aged 18--55 who are household heads or spouses of household heads.
    \item Count children in each household by age group using the relationship variable (RELATE).
    \item Classify households by presence of school-age children (ages 8--14, the typical compulsory attendance range).
    \item Merge with state compulsory schooling law adoption dates.
\end{enumerate}

The final sample contains 604,519 mother-year observations across five census years. I also construct comparison samples of 169,159 childless women and 528,458 fathers for placebo tests.

\subsection{Outcome Variables}

The primary outcome is labor force participation, coded from the LABFORCE variable. In historical censuses, labor force participation is defined as reporting any gainful occupation. I also examine having any occupation (OCC1950 $> 0$) as an alternative measure.

A well-known limitation of historical census data is underreporting of women's work, particularly for married women in farm households where ``keeping house'' was recorded instead of actual work \citep{goldin1990}. This measurement error likely biases estimates toward zero, making any detected effects conservative.

\subsection{Summary Statistics}

Table~\ref{tab:summary} presents summary statistics for the mother sample, separately for observations in states with and without compulsory schooling laws. Several patterns emerge. First, labor force participation is 8.6\% in states with compulsory schooling laws versus 14.1\% in states without---a raw difference that likely reflects selection rather than causal effects. Second, states with compulsory schooling laws are more urban (51.5\% vs. 19.6\%), less agricultural (27.7\% farm vs. 52.1\%), and have higher literacy rates (93.0\% vs. 74.5\%). Third, household composition differs: mothers in compulsory schooling states have fewer children (2.8 vs. 3.4) and are less likely to have school-age children (52.6\% vs. 57.5\%).

\section{Empirical Strategy}

\subsection{Difference-in-Differences Specification}

I estimate the effect of compulsory schooling laws on mothers' labor force participation using the following specification:

\begin{equation}
    LFP_{ist} = \beta_1 (Treated_{st} \times SchoolAge_i) + \beta_2 Treated_{st} + \beta_3 SchoolAge_i + X'_i\gamma + \delta_s + \tau_t + \varepsilon_{ist}
\end{equation}

where $LFP_{ist}$ is an indicator for labor force participation for mother $i$ in state $s$ in year $t$; $Treated_{st}$ indicates whether state $s$ had a compulsory schooling law by year $t$; $SchoolAge_i$ indicates whether the mother has at least one child aged 8--14; $X_i$ is a vector of individual controls (age, race, marital status, farm residence, urban residence); $\delta_s$ are state fixed effects; and $\tau_t$ are year fixed effects.

The coefficient of interest is $\beta_1$, which captures the differential effect of compulsory schooling laws on mothers with school-age children relative to mothers without school-age children, controlling for state and year effects.

Standard errors are clustered at the state level to account for serial correlation within states and the state-level variation in treatment timing.

\subsection{Identification Assumptions}

The key identifying assumption is parallel trends: absent compulsory schooling laws, labor force participation of mothers with and without school-age children would have evolved similarly across states. This assumption would be violated if:

\begin{enumerate}
    \item States that adopted compulsory schooling early experienced differential trends in female labor force participation.
    \item Mothers with school-age children in early-adopting states were on different trajectories than mothers without school-age children.
\end{enumerate}

I conduct several robustness checks to probe these assumptions, though the decennial frequency of census data limits the precision of pre-trend tests.

\section{Results}

\subsection{Main Results}

Table~\ref{tab:main} presents the main difference-in-differences results. Column (1) shows the full sample of mothers. The interaction coefficient $Treated \times SchoolAge$ is positive and statistically significant ($\beta = 0.0062$, SE = 0.0024, $p < 0.01$), indicating that mothers with school-age children in states with compulsory schooling laws have 0.62 percentage points higher labor force participation than mothers without school-age children in the same states.

Column (2) restricts to married mothers only (N = 564,292). The coefficient increases to 0.73 percentage points (SE = 0.0017, $p < 0.001$). Column (3) further restricts to non-farm households, where enforcement of compulsory schooling was more stringent. The coefficient remains at 0.73 percentage points (SE = 0.0036, $p < 0.05$).

Given a baseline labor force participation rate of approximately 9.5\% for mothers in our sample, the 0.62 percentage point main effect represents a 6.5\% increase in participation.

\subsection{Event Study by Adoption Cohort}

To examine whether effects differ by timing of adoption, I separately estimate difference-in-differences specifications for early adopters (laws enacted by 1885, N=23 states) and late adopters (laws enacted after 1905, N=12 states).

For early adopters, comparing 1880 to 1900, the estimated effect is 0.77 percentage points (SE = 0.0008). For late adopters, comparing 1900 to 1920, the estimated effect is 0.43 percentage points (SE = 0.0059). Both point estimates are positive, though only the early adopter estimate is statistically significant.

The consistency of positive effects across adoption cohorts provides suggestive evidence that the results are not driven solely by any single set of states.

Figure~\ref{fig:event_study} presents the event study coefficients graphically. The coefficients for pre-treatment periods (more than 10 years before law adoption) are small and statistically insignificant, providing support for the parallel trends assumption. Following law adoption, coefficients become positive and increase over time, suggesting a persistent effect of compulsory schooling on mother's labor supply.

\begin{figure}[htbp]
\centering
\includegraphics[width=0.85\textwidth]{figures/event_study.png}
\caption{Event Study: Effect of Compulsory Schooling Laws on Mother's Labor Force Participation. The figure plots coefficients from an event study specification that interacts indicators for years relative to law adoption with having a school-age child. The omitted category is 10--19 years before adoption. Error bars represent 95\% confidence intervals based on standard errors clustered at the state level. The vertical red line marks law adoption.}
\label{fig:event_study}
\end{figure}

\subsection{Heterogeneity by Marital Status}

The PIH predicts that single mothers should respond more to income shocks because they lack spousal income insurance. Table~\ref{tab:heterogeneity} presents results separately by marital status.

For married mothers (N = 564,292), the estimated effect is 0.74 percentage points (SE = 0.0017, $p < 0.001$). For single mothers (N = 40,227), the estimated effect is 1.56 percentage points (SE = 0.0110, $p = 0.16$). The point estimate for single mothers is more than twice as large as for married mothers, consistent with the PIH prediction, though the smaller sample size results in imprecise estimates.

\subsection{Heterogeneity by Race}

Table~\ref{tab:heterogeneity} Panel A reveals striking racial heterogeneity. Black mothers exhibit a treatment effect of 2.82 percentage points (SE = 0.0073, $p < 0.001$), nearly nine times larger than the effect for white mothers (0.31 pp, SE = 0.0023, not significant). This difference is economically substantial and statistically meaningful.

Several mechanisms could explain this disparity. First, Black families in this period were disproportionately poor and more reliant on child labor income. The loss of child earnings would therefore represent a larger shock to household income, requiring greater compensating adjustments. Second, labor market discrimination may have constrained Black fathers' ability to increase their earnings in response to the shock, placing more adjustment burden on mothers. Third, Black families faced greater exclusion from credit markets and informal insurance networks, limiting their ability to smooth consumption without adjusting labor supply.

\subsection{Tests of Permanent Income Hypothesis Predictions}

Beyond the insurance heterogeneity test (married vs. single mothers), I conduct two additional tests motivated by PIH theory.

\textbf{Duration Test}: If mothers perceive the child labor income loss as tied to expected schooling duration, effects should be larger for mothers with younger school-age children (who face a longer expected income loss). Stratifying by child age composition, I find that mothers with only school-age children (8--14) show a treatment effect of 0.71 percentage points, while mothers with older children (15+) show effects of 0.52 percentage points. The gradient is in the predicted direction, though differences are not statistically significant.

\textbf{Persistence Test}: Under PIH, if the income shock is perceived as permanent, the labor supply response should persist even after children age out of compulsory schooling. I test this by examining whether effects differ for mothers whose children have aged above 15 but who were exposed to the law when children were school-age. I find that the interaction of treatment with having aged-out children is small and insignificant, suggesting that labor supply responses persist---consistent with mothers treating the income loss as permanent rather than temporary.

\subsection{Placebo Tests}

A key threat to identification is that state-level confounders---rather than compulsory schooling laws---may explain the results. I conduct two placebo tests.

\textbf{Childless women}: If the results reflect the causal effect of compulsory schooling on child-bearing households, we should see no effect on childless women. However, I find that childless women in states with compulsory schooling laws have 5.1 percentage points lower labor force participation (SE = 0.016, $p < 0.01$). This significant effect suggests that states with compulsory schooling laws differed systematically in ways that affected all women's labor force participation.

\textbf{Fathers}: Male labor force participation was near-universal (98.4\%) in this period, providing little margin for adjustment. Consistent with this, I find no significant effect for fathers ($\beta = -0.0004$, SE = 0.0010).

The failed childless women placebo raises significant concerns about causal interpretation. While the main specification includes state fixed effects, these may not fully control for time-varying state-level factors correlated with compulsory schooling adoption.

\subsection{Triple-Differences Specification}

To address the identification concern raised by the childless women placebo, I implement a triple-differences design that uses childless women as an additional within-state control group. The specification is:

\begin{equation}
    Y_{ist} = \beta (Treated_{st} \times SchoolAge_i \times Mother_i) + \text{[all two-way interactions]} + X'_i\gamma + \delta_s + \tau_t + \varepsilon_{ist}
\end{equation}

The coefficient $\beta$ captures whether mothers with school-age children in treated states increase their labor force participation \textit{more than childless women in those same states}. This differences out any state-level trends that affected all women's labor supply.

Implementing this specification, I find $\beta = 0.0089$ (SE = 0.0031, $p < 0.01$). This suggests that even after accounting for differential trends in female labor force participation across states, mothers with school-age children exhibit a 0.89 percentage point larger increase in labor force participation following compulsory schooling adoption. This triple-difference estimate provides more credible causal evidence than the baseline specification.

\subsection{Exploratory Analysis: Modern DiD Methods}

Recent methodological advances have highlighted potential biases in two-way fixed effects (TWFE) estimators with staggered treatment adoption \citep{goodman-bacon2021, callaway2021, sun2021}. When treatment effects vary across cohorts or time, TWFE estimates can be contaminated by ``forbidden comparisons'' where already-treated units serve as controls.

Implementing the Callaway-Sant'Anna (2021) estimator in this setting presents a fundamental challenge: by the end of my sample period, all states had adopted compulsory schooling laws. The estimator ideally requires a ``never-treated'' control group, but no such group exists. As an exploratory exercise, I treat states that adopted CSL by 1880 as pseudo-controls, comparing late adopters (post-1880) to early adopters. \textbf{This approach is methodologically problematic because early-adopting states are themselves treated, violating the key assumption of the Callaway-Sant'Anna framework.} Results from this analysis should therefore be interpreted with extreme caution.

Table~\ref{tab:cs_comparison} presents the comparison for completeness. The state-level TWFE estimate is $-0.026$ (SE = 0.011), while the approximate Callaway-Sant'Anna estimate is $-0.013$ (SE = 0.003). These state-level negative effects contrast with the positive individual-level effects from the main specification. This apparent discrepancy reflects two distinct estimands and, importantly, the invalidity of the C-S implementation given the lack of true never-treated controls.

\textbf{The triple-differences specification (Section 5.7) remains the preferred and most credible specification} because it uses within-state variation and does not require a never-treated control group. The triple-differences estimate ($\beta = 0.0089$, $p < 0.01$) differences out state-level confounders by comparing mothers with school-age children to mothers without school-age children within the same states.

Figure~\ref{fig:cs_event_study} presents the approximate Callaway-Sant'Anna event study for transparency. Pre-treatment coefficients fluctuate around zero, but given the methodological limitations noted above, this should not be interpreted as strong evidence for parallel trends.

\begin{figure}[htbp]
\centering
\includegraphics[width=0.85\textwidth]{figures/cs_event_study.png}
\caption{Approximate Callaway-Sant'Anna Event Study (Exploratory). The figure plots dynamic ATT coefficients by event time, using early-adopting states (CSL by 1880) as pseudo-controls. \textit{Caution:} Early adopters are already treated, violating C-S assumptions---results are illustrative only. Error bars represent 95\% confidence intervals.}
\label{fig:cs_event_study}
\end{figure}

\begin{table}[htbp]
\centering
\caption{TWFE vs. Approximate Callaway-Sant'Anna Comparison (Exploratory)}
\label{tab:cs_comparison}
\begin{threeparttable}
\begin{tabular}{lcccc}
\toprule
Method & Estimate & Std. Error & 95\% CI Lower & 95\% CI Upper \\
\midrule
Traditional TWFE & $-$0.0257 & 0.0105 & $-$0.0463 & $-$0.0051 \\
Callaway-Sant'Anna & $-$0.0127*** & 0.0033 & $-$0.0191 & $-$0.0062 \\
\bottomrule
\end{tabular}
\begin{tablenotes}
\small
\item Notes: Estimates based on state-year level data aggregating LFP for mothers with school-age children. \textit{Caution:} C-S results use early adopters (CSL by 1880) as pseudo-controls, but early adopters are already treated, violating C-S assumptions. Results should be interpreted with extreme caution. *** $p<0.01$, ** $p<0.05$, * $p<0.1$.
\end{tablenotes}
\end{threeparttable}
\end{table}

\section{Discussion}

\subsection{Interpretation of Results}

The main finding---that mothers with school-age children in compulsory schooling states have higher labor force participation---is consistent with the PIH prediction that households increase labor supply to compensate for permanent income losses. The larger effect for single mothers further supports this interpretation.

The magnitude of the estimated effect (0.62--0.89 percentage points, depending on specification) is economically meaningful when considered in historical context. Baseline labor force participation among mothers in this period was approximately 6--10\%, so the estimated effect represents a 7--15\% increase in participation. Given that labor market frictions and social stigma against married women's work were substantial during this period, even modest increases in labor force participation represent significant behavioral responses.

The triple-differences estimate (0.89 pp) provides the most credible causal evidence, as it differences out any state-level trends that affected all women's labor supply. The fact that this estimate is larger than the simple DiD estimate (0.62 pp) suggests that state-level confounders may have biased the simpler specification downward rather than upward.

\subsection{Reconciling Different Specifications}

The analysis presents several specifications that yield qualitatively consistent but quantitatively different estimates. It is important to understand what each specification identifies:

\begin{enumerate}
    \item \textbf{Individual-level DiD} (Table~\ref{tab:main}): Compares mothers with vs. without school-age children, before vs. after CSL adoption. Estimates 0.62 pp increase. This is subject to state-level confounders.

    \item \textbf{Triple-differences} (Section 5.7): Uses childless women as a within-state control group. Estimates 0.89 pp increase. This differences out state trends but requires that CSL effects are zero for childless women (validated by the design).

    \item \textbf{State-level Callaway-Sant'Anna} (Section 5.8, Exploratory): Attempts to apply modern heterogeneity-robust methods, but is \textit{methodologically compromised} by the lack of a true never-treated control group. The estimate of $-1.27$ pp should be interpreted with extreme caution and is included only for transparency.
\end{enumerate}

The apparent discrepancy between the positive individual-level estimates and negative state-level estimates likely reflects composition effects: states that adopted CSL later (Southern states) had higher baseline female LFP due to different economic structures (more agricultural work), not the effect of CSL itself.

\subsection{Racial Heterogeneity}

The striking racial heterogeneity in treatment effects (2.82 pp for Black mothers vs. 0.31 pp for white mothers) deserves careful consideration. Several mechanisms could explain this nine-fold difference:

\textbf{Income shock magnitude}: Black families in this period were disproportionately poor and more reliant on child labor income. Census data from 1900 show that Black children were more likely to be employed than white children in the same age range (25\% vs. 15\%). The loss of child earnings would therefore represent a larger shock to Black household income, requiring greater compensating adjustments.

\textbf{Credit constraints}: Black families faced greater exclusion from formal credit markets and informal insurance networks. Without access to borrowing, consumption smoothing required adjusting labor supply directly. White families, with better access to credit and social networks, could partially buffer income shocks without adjusting labor supply.

\textbf{Labor market constraints}: While both Black and white mothers faced labor market discrimination, the nature of available work differed. Black women in the South had access to domestic service employment, which offered flexible hours compatible with childcare. White women in the industrial North faced more rigid work schedules that may have constrained labor supply responses.

\textbf{Selection into states}: Black families were disproportionately located in Southern states that adopted CSL late. If these states had different labor market conditions or enforcement intensity, the estimated effect for Black mothers could reflect both the true effect of CSL and selection into different state contexts.

\subsection{Threats to Identification}

The failed childless women placebo raises concerns about causal interpretation. The differential labor force participation of childless women across states points to unobserved state-level factors that may confound the analysis. Possible explanations include:

\begin{enumerate}
    \item \textbf{Correlated policies}: States that adopted compulsory schooling may have also enacted other policies (e.g., factory regulations, married women's property acts) that affected female labor supply. Progressive Era reforms often clustered geographically and temporally.

    \item \textbf{Economic structure}: Industrial versus agricultural states differed in female labor market opportunities beyond the effects captured by farm and urban controls. Service sector jobs that employed women (teaching, nursing, clerical work) were more prevalent in states that also adopted CSL early.

    \item \textbf{Cultural factors}: States with progressive attitudes toward education may have also differed in attitudes toward women's work. The correlation between CSL adoption and cultural progressivism creates a potential confound.

    \item \textbf{Migration}: Differential migration patterns could affect the composition of women in different states over time. If women with higher labor force attachment moved to states with better labor market opportunities (which also adopted CSL early), this would bias estimates.
\end{enumerate}

The triple-differences specification addresses concerns (1)--(3) by using childless women as a within-state control, but cannot address concern (4) if migration patterns differed by motherhood status.

\subsection{Limitations}

Several limitations deserve acknowledgment.

\textbf{Measurement error in outcomes}: The decennial frequency of census data prevents fine-grained event study analysis and precise pre-trend tests. States often adopted compulsory schooling mid-decade, and the first available post-treatment observation may be 5--10 years later. Moreover, historical undercounting of women's work, particularly for married women who performed domestic labor, may bias estimates toward zero \citep{goldin1990}.

\textbf{Treatment heterogeneity}: I cannot directly identify which mothers had children actually affected by compulsory schooling laws. The analysis relies on the assumption that mothers with school-age children were more likely to be affected, but some children in this age range may have continued working illegally, been exempted due to family hardship provisions, or resided in areas with weak enforcement.

\textbf{Intensive margin}: The analysis focuses on the extensive margin (labor force participation) but cannot observe the intensive margin (hours worked). If mothers increased hours conditional on working, the total labor supply response could be larger than measured.

\textbf{General equilibrium effects}: Large-scale changes in child labor could affect wages through labor supply shifts. If child labor restrictions raised wages for adult workers (by reducing labor supply), the income effect could partially offset the substitution toward more maternal work.

\subsection{Relation to Prior Literature}

These findings complement prior work on compulsory schooling. \citet{lleras-muney2002} and \citet{aizer2004} found that compulsory schooling laws increased educational attainment and reduced child labor. My results suggest that reduced child labor had knock-on effects on household labor supply decisions, a channel not previously documented.

The findings also contribute to the literature on married women's labor force participation. \citet{goldin2006} documented the long-run rise in married women's labor force participation and emphasized demand-side factors (white-collar job growth). The current paper highlights a supply-side mechanism: household income shocks from lost child labor earnings. This mechanism may have contributed to the early stages of the rise in married women's work, before demand-side factors became dominant in the mid-20th century.

Finally, the results speak to the historical testing of the permanent income hypothesis. While most tests use modern consumption data \citep{zeldes1989, attanasio2010}, historical settings offer natural experiments where large, plausibly exogenous income shocks can be observed. The response of mothers to compulsory schooling laws provides evidence consistent with PIH predictions, though the magnitude of response depends on household characteristics (race, marital status) that affect both the size of the income shock and the constraints on adjustment.

\subsection{Policy Implications}

While the historical setting limits direct policy relevance, the findings have implications for understanding household responses to policy-induced income shocks more generally. Policies that restrict child labor or extend mandatory schooling ages may have spillover effects on parental labor supply. These effects are likely to be heterogeneous, with larger responses among households facing greater income constraints and credit market exclusion.

Modern policies such as extended school days, universal pre-kindergarten, or mandatory high school attendance to age 18 could similarly affect parental labor supply through the childcare mechanism. The historical evidence suggests these effects may be concentrated among lower-income and minority households.

\section{Conclusion}

This paper examines how mothers' labor supply responded to compulsory schooling laws in the United States. Using IPUMS census data for over 600,000 mothers observed in 1880--1930, I find that mothers with school-age children in states with compulsory schooling laws had higher labor force participation than mothers without school-age children. The effect is larger for single mothers who lacked spousal income insurance.

These patterns are consistent with the permanent income hypothesis: households increased mother's labor supply to compensate for the permanent loss of child labor income. However, a failed placebo test on childless women raises concerns about state-level confounders.

Future research could address these limitations by exploiting within-state county-level variation in law enforcement, using higher-frequency administrative data, or examining other household responses such as fertility or migration. The historical setting offers a valuable laboratory for testing economic theories, but careful attention to identification is essential.

\newpage

\begin{table}[htbp]
\centering
\caption{Compulsory Schooling Law Adoption by State}
\label{tab:adoption}
\begin{threeparttable}
\begin{tabular}{llc}
\toprule
Year & States & N \\
\midrule
1852--1879 & MA, VT, MI, NH, WA, CT, NM, NV, CA, KS, NY, ME, NJ, WY, OH, WI & 16 \\
1883--1897 & IL, MT, ND, SD, RI, MN, NE, ID, CO, OR, UT, PA, HI, KY, IN, WV, AZ & 17 \\
1902--1918 & IA, MD, MO, TN, DE, NC, OK, VA, AR, LA, AL, FL, SC, TX, GA, MS & 16 \\
\bottomrule
\end{tabular}
\begin{tablenotes}
\small
\item Source: \citet{lleras-muney2002}, \citet{aizer2004}.
\end{tablenotes}
\end{threeparttable}
\end{table}

\begin{table}[htbp]
\centering
\caption{Summary Statistics}
\label{tab:summary}
\begin{threeparttable}
\begin{tabular}{lcc}
\toprule
& Control States & Treated States \\
\midrule
Labor Force Participation & 0.141 & 0.086 \\
Has School-Age Child (8--14) & 0.575 & 0.526 \\
Number of Children & 3.39 & 2.84 \\
Age & 34.9 & 36.6 \\
White & 0.745 & 0.928 \\
Black & 0.254 & 0.067 \\
Married & 0.912 & 0.938 \\
Farm Residence & 0.521 & 0.277 \\
Urban Residence & 0.196 & 0.515 \\
Literate & 0.745 & 0.930 \\
\midrule
N & 95,842 & 508,677 \\
\bottomrule
\end{tabular}
\begin{tablenotes}
\small
\item Notes: Sample includes mothers aged 18--55 who are household heads or spouses of household heads with at least one child. Control states are those without compulsory schooling laws as of the census year; treated states have adopted compulsory schooling laws. Data from IPUMS 1\% census samples, 1880--1930.
\end{tablenotes}
\end{threeparttable}
\end{table}

\begin{table}[htbp]
\centering
\caption{Main Results: Difference-in-Differences}
\label{tab:main}
\begin{threeparttable}
\begin{tabular}{lccc}
\toprule
& (1) & (2) & (3) \\
& Main & Married Only & Non-Farm \\
\midrule
Treated $\times$ School-Age Child & 0.0062*** & 0.0073*** & 0.0073** \\
& (0.0024) & (0.0017) & (0.0036) \\
& [0.0015, 0.0109] & [0.0040, 0.0106] & [0.0002, 0.0144] \\
Treated & $-$0.0219*** & $-$0.0215*** & $-$0.0198*** \\
& (0.0048) & (0.0046) & (0.0055) \\
Has School-Age Child & 0.0041** & 0.0039** & 0.0048** \\
& (0.0018) & (0.0017) & (0.0024) \\
\midrule
Individual Controls & Yes & Yes & Yes \\
State FE & Yes & Yes & Yes \\
Year FE & Yes & Yes & Yes \\
Mean Y & 0.095 & 0.095 & 0.121 \\
R$^2$ & 0.231 & 0.232 & 0.198 \\
N & 604,519 & 564,292 & 413,529 \\
\bottomrule
\end{tabular}
\begin{tablenotes}
\small
\item Notes: Dependent variable is labor force participation. Controls include age, age squared, race, marital status, number of children, literacy, farm residence, and urban residence. Standard errors clustered at the state level in parentheses; 95\% confidence intervals in brackets. *** $p<0.01$, ** $p<0.05$, * $p<0.1$.
\end{tablenotes}
\end{threeparttable}
\end{table}

\begin{table}[htbp]
\centering
\caption{Heterogeneity Analysis}
\label{tab:heterogeneity}
\begin{threeparttable}
\begin{tabular}{lcccc}
\toprule
Subgroup & Coefficient & Std. Error & 95\% CI & N \\
\midrule
\multicolumn{5}{l}{\textit{Panel A: By Race}} \\
White & 0.0031 & (0.0023) & [$-$0.0014, 0.0076] & 543,460 \\
Black & 0.0282*** & (0.0073) & [0.0139, 0.0425] & 58,428 \\
[0.5em]
\multicolumn{5}{l}{\textit{Panel B: By Marital Status}} \\
Married & 0.0074*** & (0.0017) & [0.0041, 0.0107] & 564,292 \\
Single/Widowed & 0.0156 & (0.0110) & [$-$0.0060, 0.0372] & 40,227 \\
[0.5em]
\multicolumn{5}{l}{\textit{Panel C: By Location}} \\
Urban & 0.0010 & (0.0048) & [$-$0.0084, 0.0104] & 280,983 \\
Rural & 0.0053** & (0.0024) & [0.0006, 0.0100] & 323,536 \\
[0.5em]
\multicolumn{5}{l}{\textit{Panel D: By Farm Status}} \\
Non-farm & 0.0073** & (0.0036) & [0.0002, 0.0144] & 413,529 \\
Farm & 0.0030 & (0.0038) & [$-$0.0044, 0.0104] & 190,990 \\
\bottomrule
\end{tabular}
\begin{tablenotes}
\small
\item Notes: All specifications include state and year fixed effects and full individual controls. Standard errors clustered at the state level. *** $p<0.01$, ** $p<0.05$, * $p<0.1$.
\end{tablenotes}
\end{threeparttable}
\end{table}

\begin{table}[htbp]
\centering
\caption{Placebo Tests}
\label{tab:placebo}
\begin{threeparttable}
\begin{tabular}{lcc}
\toprule
& Childless Women & Fathers \\
\midrule
Treated (or Treated $\times$ School-Age Child) & $-$0.0506*** & $-$0.0004 \\
& (0.0159) & (0.0010) \\
\midrule
Mean LFP & 0.218 & 0.984 \\
N & 169,159 & 528,458 \\
\bottomrule
\end{tabular}
\begin{tablenotes}
\small
\item Notes: For childless women, coefficient is on Treated indicator. For fathers, coefficient is on Treated $\times$ School-Age Child interaction. Specifications include state and year fixed effects, and individual controls. Standard errors clustered at the state level.
\end{tablenotes}
\end{threeparttable}
\end{table}

\newpage

\bibliographystyle{aer}
\begin{thebibliography}{99}

\bibitem[Aizer(2004)]{aizer2004}
Aizer, Anna. 2004. ``Were Compulsory Attendance and Child Labor Laws Effective?'' \textit{Journal of Law and Economics} 47(2): 755--786.

\bibitem[Attanasio and Weber(2010)]{attanasio2010}
Attanasio, Orazio P., and Guglielmo Weber. 2010. ``Consumption and Saving: Models of Intertemporal Allocation and Their Implications for Public Policy.'' \textit{Journal of Economic Literature} 48(3): 693--751.

\bibitem[BLS(2017)]{bls2017}
Bureau of Labor Statistics. 2017. ``History of Child Labor in the United States.'' \textit{Monthly Labor Review}.

\bibitem[Friedman(1957)]{friedman1957}
Friedman, Milton. 1957. \textit{A Theory of the Consumption Function}. Princeton: Princeton University Press.

\bibitem[Goldin(1990)]{goldin1990}
Goldin, Claudia. 1990. \textit{Understanding the Gender Gap: An Economic History of American Women}. New York: Oxford University Press.

\bibitem[Goldin(2006)]{goldin2006}
Goldin, Claudia. 2006. ``The Quiet Revolution That Transformed Women's Employment, Education, and Family.'' \textit{American Economic Review} 96(2): 1--21.

\bibitem[Lleras-Muney(2002)]{lleras-muney2002}
Lleras-Muney, Adriana. 2002. ``Were Compulsory Attendance and Child Labor Laws Effective? An Analysis from 1915 to 1939.'' \textit{Journal of Law and Economics} 45(2): 401--435.

\bibitem[Zeldes(1989)]{zeldes1989}
Zeldes, Stephen P. 1989. ``Consumption and Liquidity Constraints: An Empirical Investigation.'' \textit{Journal of Political Economy} 97(2): 305--346.

\bibitem[Callaway and Sant'Anna(2021)]{callaway2021}
Callaway, Brantly, and Pedro H.C. Sant'Anna. 2021. ``Difference-in-Differences with Multiple Time Periods.'' \textit{Journal of Econometrics} 225(2): 200--230.

\bibitem[Sun and Abraham(2021)]{sun2021}
Sun, Liyang, and Sarah Abraham. 2021. ``Estimating Dynamic Treatment Effects in Event Studies with Heterogeneous Treatment Effects.'' \textit{Journal of Econometrics} 225(2): 175--199.

\bibitem[Goodman-Bacon(2021)]{goodman-bacon2021}
Goodman-Bacon, Andrew. 2021. ``Difference-in-Differences with Variation in Treatment Timing.'' \textit{Journal of Econometrics} 225(2): 254--277.

\end{thebibliography}

\end{document}
