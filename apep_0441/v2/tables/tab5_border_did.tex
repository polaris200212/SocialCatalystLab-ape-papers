\begin{table}[htbp]
\centering
\caption{Border Discontinuity Design: Full Sample vs. Border Subsample}
\label{tab:border}
\begin{tabular}{lccccc}
\hline\hline
 & (1) & (2) & (3) & (4) & (5) \\
 & Full Sample & Border 100km & Border 150km & Border 200km & Border 300km \\
\hline
\addlinespace
New State $\times$ Post & 0.7993*** & 0.7120*** & 0.6947** & 0.6534*** & 0.7007*** \\
 & (0.1986) & (0.2517) & (0.2805) & (0.2437) & (0.2323) \\
\addlinespace
Spatial RDD & \multicolumn{5}{c}{1.3674*** (0.2919), BW = 61 km} \\
\addlinespace
\hline
District FE & Yes & Yes & Yes & Yes & Yes \\
Year FE & Yes & Yes & Yes & Yes & Yes \\
Districts & 214 & 81 & 103 & 129 & 158 \\
Treated Districts & 55 & 32 & 41 & 50 & 55 \\
Observations & 4,280 & 1,620 & 2,060 & 2,580 & 3,160 \\
\hline\hline
\end{tabular}
\begin{minipage}{0.95\textwidth}
\vspace{0.2cm}
\footnotesize \textit{Notes:} Column (1) reproduces the full-sample baseline TWFE. Columns (2)--(5) restrict to districts whose centroids lie within the specified distance of the nearest new state boundary. Distances computed from GADM district centroids to the shared state boundary. The Spatial RDD row reports the \texttt{rdrobust} local polynomial estimate using MSE-optimal bandwidth with a triangular kernel; the running variable is signed distance to the boundary (positive = new state side). Standard errors clustered at the state level where feasible, at the district level otherwise. $^{*}p<0.10$, $^{**}p<0.05$, $^{***}p<0.01$.
\end{minipage}
\end{table}
