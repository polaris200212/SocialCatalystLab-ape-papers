% ============================================================================
% Paper 64: The Pence Effect
% Main LaTeX File
% ============================================================================

\documentclass[12pt]{article}

% Packages
\usepackage[utf8]{inputenc}
\usepackage[T1]{fontenc}
\usepackage{amsmath,amssymb,amsthm}
\usepackage{graphicx}
\usepackage{booktabs}
\usepackage{natbib}
\usepackage{hyperref}
\usepackage{geometry}
\usepackage{setspace}
\usepackage{subfiles}
\usepackage{float}
\usepackage{caption}
\usepackage{subcaption}
\usepackage{xcolor}
\usepackage{appendix}
\usepackage{pifont}

% Page setup
\geometry{margin=1in}
\doublespacing

% Hyperref setup
\hypersetup{
    colorlinks=true,
    linkcolor=blue,
    citecolor=blue,
    urlcolor=blue
}

% Custom commands
\newcommand{\cmark}{\ding{51}}
\newcommand{\xmark}{\ding{55}}

% Title
\title{\textbf{The Pence Effect: Did \#MeToo Reduce Female Employment in High-Harassment Industries?}}

\author{
    APEP Research Team\thanks{Autonomous Policy Evaluation Project. Correspondence: apep@example.edu. We thank the EEOC for providing publicly available harassment charge data and the Census Bureau for the Quarterly Workforce Indicators. All errors are our own.}
}

\date{\today}

\begin{document}

\maketitle

% Abstract
\subfile{sections/01_abstract}

\newpage

% Introduction (4+ pages, 20+ citations)
\subfile{sections/02_introduction}

% Background and Literature
\subfile{sections/03_background}

% Data
\subfile{sections/04_data}

% Empirical Strategy
\subfile{sections/05_methods}

% Results
\subfile{sections/06_results}

% Robustness
\subfile{sections/07_robustness}

% Mechanisms
\subfile{sections/08_mechanisms}

% Discussion
\subfile{sections/09_discussion}

% Conclusion
\subfile{sections/10_conclusion}

% References
\newpage
\bibliographystyle{aer}
\bibliography{references}

% Appendix
\newpage
\appendix
\subfile{sections/11_appendix}

\end{document}
