\documentclass[../main.tex]{subfiles}

\begin{document}

\section{Mechanisms}

Our main results document a decline in female employment in high-harassment industries following \#MeToo. This section explores potential mechanisms underlying this pattern.

\subsection{Hiring vs. Separations}

The employment decline we document could result from reduced hiring of women, increased separations of women, or both. To distinguish these channels, we decompose the employment effect into its constituent flows.

Table \ref{tab:flows} presents triple-difference estimates for hires and separations separately. Column 1 shows that female hires in high-harassment industries fell by 2.8 percent relative to the counterfactual. Column 2 shows that female separations increased by 0.6 percent, though this effect is smaller and less precisely estimated. The majority of the employment decline appears to operate through reduced hiring rather than increased turnover.

\begin{table}[htbp]
\centering
\caption{Decomposition of Employment Effects: Hires vs. Separations}
\label{tab:flows}
\begin{tabular}{lcc}
\toprule
 & (1) & (2) \\
 & Hires & Separations \\
\midrule
Female $\times$ High Harass. $\times$ Post & $-$0.028$^{***}$ & 0.006$^{**}$ \\
 & (0.004) & (0.003) \\
\midrule
Fixed Effects & Full & Full \\
Observations & 77,520 & 77,520 \\
R$^2$ & 0.912 & 0.887 \\
\bottomrule
\end{tabular}
\caption*{\footnotesize Note: Both columns include state$\times$time, industry$\times$time, industry$\times$gender, and gender$\times$time fixed effects. Standard errors clustered by state$\times$industry. The dependent variable in Column (1) is log hires; in Column (2), log separations. $^{***}$p$<$0.01, $^{**}$p$<$0.05, $^{*}$p$<$0.1.}
\end{table}


This pattern is consistent with the ``Pence Effect'' mechanism. If male managers avoid interacting with female job candidates, this should reduce female hiring more than it increases separations of existing employees, since hiring decisions require active engagement while retention of current employees may be more passive.

\subsection{Firm Entry and Exit}

An alternative mechanism is that \#MeToo affected firm dynamics in high-harassment industries. For example, firms owned by women may have entered these industries at lower rates, or male-owned firms may have exited at higher rates due to harassment scandals.

We examine this possibility using firm counts from the QWI. Results (available in the Appendix) show no significant differential change in the number of firms in high-harassment industries after \#MeToo. The employment effects we document appear to reflect within-firm hiring decisions rather than changes in industry composition.

\subsection{Occupational Composition}

Another mechanism is that \#MeToo may have affected the occupational composition of female employment within industries. If women were disproportionately moved out of client-facing roles or supervisory positions, this could manifest as changes in employment patterns.

Using occupation-by-industry data from the ACS, we examine changes in female representation in managerial occupations within high-harassment industries. We find some evidence of a decline in female managers, though the estimates are imprecise due to smaller sample sizes. More detailed administrative data would be needed to fully characterize these occupational shifts.

\subsection{Geographic Heterogeneity}

We examine whether the employment effects vary by state characteristics that might be associated with different responses to \#MeToo. Specifically, we interact our triple-difference specification with measures of state political orientation (Democratic vs. Republican vote share in 2016) and state policy responses (whether the state enacted harassment training mandates).

Results suggest that the negative employment effects are slightly larger in Republican-leaning states, though the difference is not statistically significant. There is no evidence that state-level policy responses moderated or amplified the employment effects, though this null result should be interpreted cautiously given the limited post-policy period for states that enacted mandates in 2018-2019.

\subsection{Alternative Interpretations}

While our results are consistent with the ``Pence Effect'' hypothesis, alternative interpretations warrant consideration.

\paragraph{Demand-Side Changes.} Consumer preferences may have shifted following \#MeToo, affecting demand for services in high-harassment industries. For example, consumers may have avoided restaurants or retailers associated with harassment scandals. However, if demand changes were the driver, we would expect similar effects on male and female employment, which we do not observe.

\paragraph{Supply-Side Changes.} Women may have chosen to exit high-harassment industries following increased awareness of harassment risks. This ``exit'' mechanism is conceptually distinct from the ``exclusion'' mechanism we emphasize, though both would produce similar employment patterns. Survey evidence that women report \textit{wanting} to work in these industries but facing reduced opportunities supports the exclusion interpretation.

\paragraph{Reporting Changes.} Our harassment exposure measure uses pre-\#MeToo charge rates, but post-\#MeToo changes in reporting behavior could still affect employment if firms anticipated increased liability. This mechanism is not mutually exclusive with male avoidance behavior and may reinforce it.

\end{document}
