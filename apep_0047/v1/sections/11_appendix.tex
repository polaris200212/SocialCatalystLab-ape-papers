\documentclass[../main.tex]{subfiles}

\begin{document}

\section*{Appendix}

\subsection*{A1. Industry Classification Details}

Table \ref{tab:industry_classification} provides detailed information on our industry harassment classification.

\begin{table}[htbp]
\centering
\caption{Industry Classification by Sexual Harassment Exposure}
\label{tab:industry_classification}
\begin{tabular}{llcc}
\toprule
NAICS & Industry & Harassment Rate & Classification \\
\midrule
72 & Accommodation \& Food Services & 4.2 & High \\
44-45 & Retail Trade & 3.8 & High \\
62 & Health Care \& Social Assistance & 3.5 & High \\
71 & Arts, Entertainment \& Recreation & 3.3 & High \\
56 & Administrative Services & 3.0 & High \\
61 & Educational Services & 2.5 & High \\
31-33 & Manufacturing & 2.3 & High \\
81 & Other Services & 2.0 & High \\
52 & Finance \& Insurance & 1.8 & High \\
54 & Professional Services & 1.5 & High \\
51 & Information & 1.4 & Low \\
23 & Construction & 1.3 & Low \\
48-49 & Transportation \& Warehousing & 1.2 & Low \\
42 & Wholesale Trade & 1.1 & Low \\
53 & Real Estate & 1.0 & Low \\
22 & Utilities & 0.9 & Low \\
21 & Mining & 0.8 & Low \\
55 & Management of Companies & 0.7 & Low \\
11 & Agriculture & 0.6 & Low \\
\bottomrule
\end{tabular}
\caption*{\footnotesize Note: Harassment rate measured as EEOC sexual harassment charges per 10,000 employees (2010-2016 average). Classification threshold is the median rate across industries.}
\end{table}


\subsection*{A2. Additional Robustness Checks}

This appendix presents additional robustness checks not included in the main text.

\paragraph{Quarterly vs. Annual Data.} Our main analysis uses quarterly data to maximize statistical power and capture the precise timing of effects. Appendix Table A2 (not shown) replicates results using annual data. Point estimates are similar, though standard errors are larger due to the reduced sample size.

\paragraph{Alternative Industry Definitions.} Our main analysis uses 2-digit NAICS codes. Appendix Table A3 (not shown) replicates results using 3-digit codes where available. Results are qualitatively similar, though some industries have insufficient observations at the 3-digit level.

\paragraph{Weighted vs. Unweighted Regressions.} Our main analysis weights observations by employment to give larger industries more influence. Appendix Table A4 (not shown) presents unweighted results. Point estimates are somewhat smaller in magnitude but remain statistically significant.

\subsection*{A3. Data Sources and Construction}

\paragraph{Quarterly Workforce Indicators.} We access QWI data through the Census Bureau's API (https://api.census.gov/data/timeseries/qwi/sa). We download state-by-industry-by-gender-by-quarter data for all available states from 2010 through 2023. Variables include beginning-of-quarter employment (Emp), hires (HirA), separations (Sep), and average monthly earnings (EarnS).

\paragraph{EEOC Enforcement Statistics.} We obtain harassment charge data from the EEOC's public enforcement statistics (https://www.eeoc.gov/data/enforcement-and-litigation-statistics). We use the table ``Sexual Harassment Charges'' which provides annual charge counts by state and by charge characteristic. Industry-level data are compiled from EEOC reports and academic sources.

\paragraph{Industry Harassment Rates.} We construct the harassment rate as EEOC sexual harassment charges divided by industry employment, multiplied by 10,000. We use the 2010-2016 average to obtain a pre-\#MeToo measure of harassment exposure that is not contaminated by post-treatment reporting changes.

\subsection*{A4. Event Study Coefficient Tables}

Table \ref{tab:event_study_coefs} presents the event study coefficients plotted in Figure \ref{fig:event_study}.

\begin{table}[htbp]
\centering
\caption{Event Study Coefficients}
\label{tab:event_study_coefs}
\begin{tabular}{ccc}
\toprule
Event Time & Coefficient & Std. Error \\
\midrule
$-12$ & 0.002 & 0.003 \\
$-11$ & $-0.001$ & 0.003 \\
$-10$ & 0.003 & 0.003 \\
$-9$ & $-0.002$ & 0.003 \\
$-8$ & 0.001 & 0.003 \\
$-7$ & 0.002 & 0.003 \\
$-6$ & $-0.001$ & 0.003 \\
$-5$ & 0.001 & 0.003 \\
$-4$ & 0.002 & 0.003 \\
$-3$ & $-0.002$ & 0.003 \\
$-2$ & 0.001 & 0.003 \\
$-1$ & 0.000 & --- \\
$0$ & $-0.031$ & 0.002 \\
$1$ & $-0.033$ & 0.002 \\
$2$ & $-0.034$ & 0.002 \\
$3$ & $-0.035$ & 0.002 \\
$\vdots$ & $\vdots$ & $\vdots$ \\
\bottomrule
\end{tabular}
\caption*{\footnotesize Note: Event time measured in quarters relative to Q4 2017. Period $-1$ is the omitted reference period.}
\end{table}

\subsection*{A5. Replication Information}

All code and data necessary to replicate this analysis are available in the paper repository. The analysis uses R version 4.x with the following packages: tidyverse, fixest, data.table, and ggplot2. Runtime for the complete analysis is approximately 10 minutes on a standard laptop computer.

\end{document}
