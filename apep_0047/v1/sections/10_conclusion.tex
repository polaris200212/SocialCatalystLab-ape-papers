\documentclass[../main.tex]{subfiles}

\begin{document}

\section{Conclusion}

This paper provides quasi-experimental evidence on the \#MeToo movement's potential effects on female employment. Using a triple-difference design that exploits the October 2017 timing of the movement along with cross-industry variation in harassment exposure, we find a relative decline in female employment in high-harassment industries of approximately 3.4 percentage points. This effect emerges immediately after \#MeToo, shows no evidence of pre-trends, and persists through 2023. However, given that treatment varies across only 19 industries, readers should interpret our precision estimates with appropriate caution.

Our findings are consistent with the ``Pence Effect'' hypothesis: that heightened awareness of harassment risk may have caused some men in gatekeeping positions to reduce professional interactions with women. Survey evidence documents widespread adoption of avoidance behaviors following \#MeToo. However, we cannot rule out alternative mechanisms, and the reduced-form nature of our estimates does not allow us to definitively establish the causal channel. The employment patterns we document are suggestive of---but do not prove---unintended consequences from awareness campaigns.

The implications extend beyond \#MeToo to the broader challenge of combating workplace discrimination. Information-based interventions that raise awareness of misconduct may be necessary but insufficient. When awareness increases perceived liability without changing underlying behavior, gatekeepers may respond through avoidance rather than reform. Effective policy requires complementary interventions---structured mentoring programs, transparent advancement criteria, and manager accountability---that maintain engagement while addressing misconduct.

Our findings should not be interpreted as criticism of the \#MeToo movement or as an argument against harassment awareness. The movement achieved important goals, including documenting widespread misconduct, removing serial harassers from positions of power, and shifting cultural norms toward accountability. These benefits are real and important. Our contribution is to document one cost---reduced female employment in high-harassment industries---that policymakers should address through complementary interventions.

More broadly, our research highlights the importance of evaluating anti-discrimination policies for unintended consequences. Good intentions do not guarantee good outcomes. Understanding how policies affect behavior---including the behavior of potential discriminators---is essential for designing interventions that achieve their objectives without generating harmful side effects.

\end{document}
