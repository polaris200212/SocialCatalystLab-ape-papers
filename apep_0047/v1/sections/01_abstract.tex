\documentclass[../main.tex]{subfiles}

\begin{document}

\begin{abstract}

The \#MeToo movement transformed workplace culture by raising awareness of sexual harassment, but did it inadvertently harm women's labor market outcomes? We examine patterns consistent with the ``Pence Effect'' hypothesis---that heightened harassment awareness may cause men to reduce professional interactions with women. Using a triple-difference design that exploits the October 2017 timing of \#MeToo along with cross-industry variation in pre-existing harassment rates, we find that female employment in high-harassment industries declined by approximately 3.4 percentage points relative to low-harassment industries, controlling for male employment trends. This effect is concentrated in accommodation, retail, and healthcare---sectors with historically high harassment rates. Event study estimates show no differential pre-trends prior to October 2017 and an immediate, persistent effect thereafter. However, with treatment varying across only 19 industries, readers should interpret precision with appropriate caution. Robustness checks using alternative harassment measures, placebo treatment dates, and different clustering approaches yield qualitatively similar results. Our findings are consistent with---though do not definitively establish---the hypothesis that awareness campaigns may generate unintended labor market consequences.

\vspace{0.5cm}
\noindent \textbf{Keywords:} Sexual harassment, \#MeToo, female employment, difference-in-differences, labor market discrimination

\vspace{0.3cm}
\noindent \textbf{JEL Codes:} J16, J71, K31, M51

\end{abstract}

\end{document}
