\documentclass[../main.tex]{subfiles}

\begin{document}

\section{Discussion}

\subsection{Interpretation of Findings}

Our results document a significant decline in female employment in high-harassment industries following the \#MeToo movement. The preferred interpretation, consistent with survey evidence and the ``Pence Effect'' hypothesis, is that heightened awareness of harassment risk caused men in gatekeeping positions to reduce professional interactions with women, manifesting as reduced female hiring and employment.

This interpretation carries important implications. First, it suggests that awareness-raising campaigns, while valuable for documenting misconduct and supporting survivors, may generate unintended consequences that policymakers must anticipate and address. Second, it highlights the limitations of information-based interventions when underlying incentives favor avoidance over behavioral change. Third, it underscores the need for complementary policies that maintain accountability while preventing male withdrawal from mentoring and sponsoring women.

\subsection{Policy Implications}

Our findings suggest several policy responses to mitigate the backlash effects of harassment awareness:

\paragraph{Structured Mentoring Programs.} Organizations could implement formal mentoring and sponsorship programs that match senior leaders with junior employees regardless of gender. Structured programs reduce the discretion that enables avoidance behavior and create accountability for developing diverse talent pipelines.

\paragraph{Transparent Advancement Criteria.} Clear, documented criteria for hiring and promotion decisions leave less room for gender-based exclusion. When advancement depends on objective metrics rather than informal relationships, the consequences of male withdrawal from informal mentoring are reduced.

\paragraph{Manager Accountability.} Organizations could track and evaluate managers on their success in developing diverse teams. If avoidance behavior becomes visible and carries career consequences, the incentives favoring withdrawal may be attenuated.

\paragraph{Training Reform.} Evidence suggests that traditional harassment training may backfire by triggering defensive reactions among men. Bystander intervention training, which positions men as allies rather than potential perpetrators, may be more effective at changing behavior without generating avoidance.

\subsection{Limitations}

Several limitations of our analysis warrant acknowledgment.

\paragraph{Data Limitations.} Our employment data come from the Quarterly Workforce Indicators, which measure total employment by industry, state, and gender. We cannot observe individual-level employment transitions, wages, occupational changes, or working conditions. More detailed administrative data would enable richer analysis of mechanisms.

\paragraph{Measurement of Harassment Exposure.} Our harassment exposure measure is based on EEOC charge data, which represent a small and potentially non-representative sample of actual harassment incidents. Alternative measures based on survey data or firm-level characteristics might yield different results.

\paragraph{Causal Identification.} While our triple-difference design controls for many confounders, we cannot rule out all threats to identification. Industry-specific shocks coincident with \#MeToo, changes in industry composition, or other factors could contribute to our results. The dose-response relationship and placebo tests provide supporting evidence, but definitive causal claims require caution.

\paragraph{External Validity.} Our findings pertain to aggregate employment patterns in U.S. industries. Results may differ in other countries, in specific occupations or firms, or for individual-level outcomes like wages and advancement. Generalizing to these contexts requires additional research.

\paragraph{Welfare Implications.} While we document employment declines, we cannot assess overall welfare effects. The \#MeToo movement generated benefits---including increased reporting of misconduct, removal of serial harassers from positions of power, and cultural shifts toward greater accountability---that our analysis does not capture. A full welfare assessment would require weighing these benefits against the employment costs we document.

\subsection{Future Research}

Several extensions would advance understanding of these issues. First, firm-level administrative data could illuminate within-firm dynamics, including changes in hiring criteria, promotion rates, and gender pay gaps. Second, linked employer-employee data could track individual women's career trajectories before and after \#MeToo, distinguishing voluntary exits from involuntary separations. Third, experimental interventions could test whether policy responses (e.g., structured mentoring, manager accountability) mitigate avoidance behavior. Fourth, cross-country comparisons could assess whether similar dynamics occurred in other countries that experienced \#MeToo-like movements.

\end{document}
