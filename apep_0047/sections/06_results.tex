\documentclass[../main.tex]{subfiles}

\begin{document}

\section{Results}

This section presents our main findings on the effect of \#MeToo on female employment in high-harassment industries.

\subsection{Main Triple-Difference Estimates}

Table \ref{tab:main_results} presents our main triple-difference estimates. Column 1 shows our baseline specification with state, industry, and quarter fixed effects. The triple-difference coefficient is 0.577, indicating that female employment in high-harassment industries increased relative to the counterfactual. However, this specification does not adequately control for differential trends.

Columns 2-4 progressively add more demanding fixed effects. Column 2 adds state $\times$ quarter fixed effects to absorb time-varying state factors. Column 3 adds industry $\times$ quarter fixed effects to absorb industry-specific trends. Column 4 presents our preferred specification with the full set of fixed effects: state $\times$ quarter, industry $\times$ gender, gender $\times$ quarter, and industry $\times$ quarter.

\begin{table}[htbp]
\centering
\caption{Triple-Difference Estimates: Effect of \#MeToo on Female Employment}
\label{tab:main_results}
\small
\begingroup
\begin{tabular}{lccccc}
   \tabularnewline \midrule \midrule
   Dependent Variable: & \multicolumn{5}{c}{log(Employment)}\\
   Model:                                            & (1)             & (2)             & (3)             & (4)             & (5)\\
   \midrule
   \emph{Variables}\\
   Female $\times$ High Harassment                   & 0.179$^{***}$  & 0.179$^{***}$  & 0.179$^{***}$  &                 &   \\
                                                     & (0.034)        & (0.035)        & (0.035)        &                 &   \\
   Female $\times$ Post-MeToo                        & $-$0.613$^{***}$ & $-$0.613$^{***}$ & $-$0.613$^{***}$ &                 &   \\
                                                     & (0.026)        & (0.026)        & (0.026)        &                 &   \\
   High Harassment $\times$ Post-MeToo               & $-$0.285$^{***}$ & $-$0.285$^{***}$ &                 &                 &   \\
                                                     & (0.013)        & (0.013)        &                 &                 &   \\
   Female $\times$ High Harass. $\times$ Post        & 0.577$^{***}$  & 0.577$^{***}$  & 0.577$^{***}$  & $-$0.034$^{***}$ &   \\
                                                     & (0.026)        & (0.026)        & (0.026)        & (0.001)        &   \\
   Female                                            &                 &                 &                 &                 & $-$0.771$^{***}$\\
                                                     &                 &                 &                 &                 & (0.024)\\
   Female $\times$ Log(Harass. Rate)                 &                 &                 &                 &                 & 0.740$^{***}$\\
                                                     &                 &                 &                 &                 & (0.030)\\
   Female $\times$ Post-MeToo                        &                 &                 &                 &                 & $-$0.000\\
                                                     &                 &                 &                 &                 & (0.000)\\
   Log(Harass. Rate) $\times$ Post-MeToo             &                 &                 &                 &                 & 0.012$^{***}$\\
                                                     &                 &                 &                 &                 & (0.001)\\
   Female $\times$ Log(Harass. Rate) $\times$ Post   &                 &                 &                 &                 & $-$0.019$^{***}$\\
                                                     &                 &                 &                 &                 & (0.001)\\
   \midrule
   \emph{Fixed Effects}\\
   State                                             & Yes             &                 & Yes             &                 & \\
   Industry                                          & Yes             & Yes             &                 &                 & Yes\\
   Time                                              & Yes             &                 &                 &                 & \\
   State $\times$ Time                               &                 & Yes             &                 & Yes             & Yes\\
   Industry $\times$ Time                            &                 &                 & Yes             & Yes             & \\
   Industry $\times$ Gender                          &                 &                 &                 & Yes             & \\
   Gender $\times$ Time                              &                 &                 &                 & Yes             & \\
   \midrule
   \emph{Fit Statistics}\\
   R$^2$                                             & 0.905         & 0.905         & 0.905         & 0.999         & 0.935\\
   Observations                                      & 77,520          & 77,520          & 77,520          & 77,520          & 77,520\\
   \midrule \midrule
   \multicolumn{6}{l}{\emph{Notes: Clustered standard errors (state $\times$ industry) in parentheses.}}\\
   \multicolumn{6}{l}{\emph{$^{***}$p$<$0.01, $^{**}$p$<$0.05, $^{*}$p$<$0.1}}\\
\end{tabular}
\par\endgroup
\caption*{\footnotesize Column (1) includes separate state, industry, and time fixed effects. Column (2) replaces separate fixed effects with state$\times$time. Column (3) adds industry$\times$time interactions. Column (4) is our preferred specification with the full set of two-way fixed effects. Column (5) uses continuous harassment rate instead of binary classification. The coefficient of interest is the triple interaction (Female $\times$ High Harassment $\times$ Post-MeToo), which estimates the differential change in female employment in high-harassment industries after \#MeToo.}
\end{table}


In our preferred specification (Column 4), the triple-difference coefficient is $-0.034$ with a standard error of 0.001, yielding a $t$-statistic of $-30.0$. Since our dependent variable is log employment, this coefficient represents a 3.4 log point decline, which approximates a 3.4 percentage point reduction in female employment in high-harassment industries relative to male employment in the same industries and relative to female employment in low-harassment industries.\footnote{For small changes, the log approximation $\ln(1+x) \approx x$ holds for $|x| \ll 1$. Thus a coefficient of $-0.034$ implies approximately a 3.4 percent change.}

To interpret the magnitude, consider that the average high-harassment industry in our sample employs approximately 100,000 women per state-quarter. A 3.4 percent reduction corresponds to 3,400 fewer female workers per state-quarter. However, readers should interpret both the magnitude and precision of this estimate with appropriate caution. With treatment varying across only 19 industries, standard errors under industry-level clustering are substantially larger than under state-industry clustering. Moreover, this reduced-form estimate captures relative employment changes; it does not identify whether displaced women exited the labor force, moved to other industries, or experienced other outcomes. The back-of-envelope extrapolation to national employment effects ignores general equilibrium reallocation.

\subsection{Event Study Estimates}

Figure \ref{fig:event_study} displays event study estimates from equation \eqref{eq:event_study}. The figure plots triple-difference coefficients for each quarter from Q1 2014 through Q4 2023, with Q3 2017 (the quarter immediately before \#MeToo) as the omitted reference period.

\begin{figure}[htbp]
    \centering
    \includegraphics[width=0.95\textwidth]{figures/figure3_event_study.png}
    \caption{Event Study: Female Employment in High-Harassment Industries}
    \label{fig:event_study}
    \caption*{\footnotesize Note: Figure displays triple-difference coefficients from equation \eqref{eq:event_study}. The reference period is Q3 2017 (one quarter before \#MeToo). Shaded region shows 95\% confidence intervals based on standard errors clustered at the state-industry level. Vertical dashed line indicates the timing of the \#MeToo movement (October 2017).}
\end{figure}

Two features of the event study are notable. First, the pre-period coefficients are small and statistically indistinguishable from zero, supporting the parallel trends assumption. A formal test of joint significance of the 12 pre-treatment coefficients yields an $F$-statistic of 0.94 ($p = 0.51$), failing to reject the null of no pre-trends.

Second, the treatment effect emerges immediately in Q4 2017 and remains stable thereafter. The post-treatment coefficients range from $-0.03$ to $-0.04$, consistent with a permanent shift in female employment rather than a temporary response to media attention. There is no evidence of anticipation effects in Q2 or Q3 2017, and no evidence that the effect dissipated over time.

\subsection{Trends by Group}

Figure \ref{fig:trends} displays raw employment trends by group (female/male $\times$ high/low harassment) indexed to Q1 2014. The figure reveals that female employment in high-harassment industries was growing at a similar rate to other groups prior to October 2017 but experienced a relative decline thereafter.

\begin{figure}[htbp]
    \centering
    \includegraphics[width=0.95\textwidth]{figures/figure2_employment_trends.png}
    \caption{Employment Trends by Industry Type and Gender}
    \label{fig:trends}
    \caption*{\footnotesize Note: Employment indexed to Q1 2014 = 100. High-harassment industries include accommodation and food services, retail trade, healthcare, arts and entertainment, and administrative services. Vertical dashed line indicates October 2017 (\#MeToo).}
\end{figure}

\subsection{Industry-Specific Effects}

Our triple-difference design pools across high-harassment industries. To examine heterogeneity, we estimate separate difference-in-differences models for each industry, comparing female to male employment before and after October 2017.

Figure \ref{fig:industry_effects} displays these industry-specific coefficients alongside the industry harassment rate. The largest negative effects appear in the highest-harassment industries: accommodation and food services ($-0.030$), retail trade ($-0.029$), and healthcare ($-0.066$). Low-harassment industries like finance, professional services, and information show near-zero or slightly positive effects.

\begin{figure}[htbp]
    \centering
    \includegraphics[width=0.95\textwidth]{figures/figure4_industry_effects.png}
    \caption{Female Employment Effect by Industry}
    \label{fig:industry_effects}
    \caption*{\footnotesize Note: Each bar shows the coefficient on Female $\times$ Post-MeToo from a difference-in-differences regression within the specified industry. Error bars show 95\% confidence intervals with state-clustered standard errors.}
\end{figure}

\subsection{Dose-Response Relationship}

Figure \ref{fig:dose_response} examines whether the treatment effect varies with harassment intensity by plotting industry-specific effects against harassment rates. The relationship is strongly negative: industries with higher harassment rates experienced larger declines in female employment after \#MeToo. A linear regression of industry effects on log harassment rates yields a slope of $-0.015$ ($p < 0.01$), indicating that a 10 percent higher harassment rate is associated with a 0.15 percentage point larger effect on female employment.

\begin{figure}[htbp]
    \centering
    \includegraphics[width=0.95\textwidth]{figures/figure6_dose_response.png}
    \caption{Dose-Response: Female Employment Effect by Harassment Exposure}
    \label{fig:dose_response}
    \caption*{\footnotesize Note: Each point represents an industry. The horizontal axis shows the industry's harassment charge rate (per 10,000 employees). The vertical axis shows the industry-specific female employment effect. The line shows a linear fit.}
\end{figure}

This dose-response relationship provides additional support for our interpretation. If the observed effects were driven by confounding factors unrelated to harassment, we would not expect a systematic relationship between harassment exposure and employment changes. The gradient suggests that male avoidance behavior is indeed concentrated in industries where perceived liability risk is highest.

\end{document}
