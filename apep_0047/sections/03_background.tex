\documentclass[../main.tex]{subfiles}

\begin{document}

\section{Background}

\subsection{The \#MeToo Movement}

The \#MeToo movement emerged in October 2017 as a watershed moment in public discourse about workplace sexual harassment. While activist Tarana Burke had coined the phrase ``Me Too'' in 2006 to support survivors of sexual violence, the hashtag went viral on October 15, 2017, when actress Alyssa Milano encouraged women to share their experiences following the \textit{New York Times} exposé on Harvey Weinstein \citep{johnson2019metoo}. Within 24 hours, the hashtag had been used more than 500,000 times on Twitter, and within two weeks, it had appeared in 85 countries \citep{frye2020effect}.

The movement's impact extended far beyond social media. High-profile accusations led to the resignations or firings of prominent figures including Matt Lauer, Charlie Rose, Kevin Spacey, and numerous executives across industries. By December 2017, \textit{Time} magazine had named ``The Silence Breakers'' as its Person of the Year. The movement catalyzed policy responses at multiple levels: Congress passed legislation banning mandatory arbitration for sexual harassment claims in 2022, and numerous states implemented new harassment training mandates, disclosure requirements, and limitations on non-disclosure agreements \citep{ford2021metoo}.

\subsection{Workplace Sexual Harassment: Prevalence and Consequences}

Sexual harassment remains pervasive in American workplaces. The Equal Employment Opportunity Commission (EEOC) receives approximately 7,000-8,000 sexual harassment charges annually, though experts estimate this represents only 6-13 percent of actual incidents \citep{chai2019eeoc}. Survey data suggest that 25-85 percent of women experience some form of workplace harassment during their careers, with rates varying by industry, occupation, and definition \citep{fitzgerald2018sexual}.

The consequences of harassment extend beyond immediate psychological harm. Victims experience increased job turnover, reduced earnings, and diminished career advancement \citep{mclaughlin2017economic}. At the firm level, harassment incidents reduce productivity, increase legal liability, and damage organizational reputation \citep{willness2007meta}. \citet{folke2022sexual} provide causal evidence that harassment causes women to exit firms and reduces their subsequent earnings by 3-5 percent.

\subsection{The ``Pence Effect'' Hypothesis}

Alongside the \#MeToo movement's successes emerged concerns about male backlash. The term ``Pence Effect'' references Vice President Mike Pence's publicized practice of refusing to dine alone with women other than his wife---a behavior that critics argued could constitute gender discrimination in professional contexts \citep{miller2017pence}.

Survey evidence documents widespread adoption of avoidance behaviors following \#MeToo. The Lean In organization found that the percentage of male managers uncomfortable working one-on-one with women increased from 46 percent in 2018 to 60 percent in 2019 \citep{leanin2019}. A University of Houston study documented that 27 percent of men avoid one-on-one meetings with women at work, 21 percent are reluctant to hire women for positions requiring close interpersonal interaction, and 19 percent are reluctant to hire ``attractive'' women \citep{atwater2019looking}. A Harvard Business Review survey found that 64 percent of senior men were reluctant to mentor junior women one-on-one after \#MeToo \citep{bower2019hbr}.

The mechanism underlying these avoidance behaviors appears to be fear of false accusations. While data on actual false accusations are limited, survey evidence suggests that many men perceive the risk as substantial \citep{atwater2019looking}. This perception may be amplified by media coverage of high-profile cases and uncertainty about the boundary between acceptable and unacceptable behavior.

\subsection{Industry Variation in Harassment Exposure}

Sexual harassment rates vary substantially across industries. EEOC data reveal that the highest-risk industries include accommodation and food services, retail trade, healthcare, and arts and entertainment \citep{eeoc2016task}. These industries share characteristics that may facilitate harassment: female-majority workforces, customer-facing roles, tipping structures that create power imbalances, and relatively weak formal HR systems \citep{good2016hospitality}.

This cross-industry variation provides the identifying variation for our empirical strategy. If male avoidance behavior following \#MeToo is driven by perceived harassment liability, we would expect the largest employment effects in industries with historically high harassment rates. Men in these industries have the most reason to believe that interactions with female colleagues could lead to accusations.

\subsection{Policy Responses to \#MeToo}

States responded to \#MeToo with a variety of legislative initiatives. Table \ref{tab:policy_timeline} summarizes the major policy changes:

\begin{table}[htbp]
\centering
\caption{State Policy Responses to \#MeToo}
\label{tab:policy_timeline}
\begin{tabular}{lll}
\toprule
State & Year & Policy \\
\midrule
New York & 2018 & Mandatory harassment training for all employers \\
California & 2019 & Expanded training requirements; NDA limitations \\
Illinois & 2019 & Training mandates; disclosure requirements \\
Connecticut & 2019 & Training mandates \\
New Jersey & 2019 & NDA limitations \\
Delaware & 2019 & Training mandates for large employers \\
Maine & 2019 & Training mandates \\
\bottomrule
\end{tabular}
\caption*{\footnotesize Note: Table shows major state-level policies enacted in response to \#MeToo.}
\end{table}

These policies create additional identifying variation that future research could exploit. Our paper focuses on the national-level \#MeToo shock and cross-industry variation, leaving state policy evaluation for future work.

\end{document}
