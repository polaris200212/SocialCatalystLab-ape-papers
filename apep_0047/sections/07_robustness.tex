\documentclass[../main.tex]{subfiles}

\begin{document}

\section{Robustness}

We conduct extensive robustness checks to assess the sensitivity of our findings to alternative specifications, sample restrictions, and measurement choices.

\subsection{Alternative Specifications}

Table \ref{tab:robustness} presents results from alternative specifications. Column 1 reproduces our main result for reference. Column 2 excludes accommodation and food services, the largest high-harassment industry. The coefficient remains negative ($-0.031$) and highly significant, indicating that our results are not driven by a single industry.

Column 3 excludes healthcare, which may have experienced unique shocks during the COVID-19 pandemic. The coefficient is $-0.028$, somewhat smaller than our main estimate but still statistically significant. Column 4 restricts the sample to the pre-COVID period (before 2020), yielding a coefficient of $-0.029$.

Column 5 uses state-level clustering instead of state-industry clustering. Standard errors increase somewhat, but the coefficient remains highly significant ($t = -12.4$).

\begin{table}
\centering
\begin{talltblr}[         %% tabularray outer open
caption={Robustness: Alternative Estimators},
note{}={* p \num{< 0.1}, ** p \num{< 0.05}, *** p \num{< 0.01}},
note{ }={Column 1: Simple 2x2 DiD. Columns 2-3: Two-way fixed effects.},
note{  }={Column 3 includes state-specific linear time trends.},
]                     %% tabularray outer close
{                     %% tabularray inner open
colspec={Q[]Q[]Q[]Q[]},
column{2,3,4}={}{halign=c,},
column{1}={}{halign=l,},
hline{6}={1,2,3,4}{solid, black, 0.05em},
}                     %% tabularray inner close
\toprule
& (1) Simple 2x2 & (2) TWFE & (3) TWFE + Trends \\ \midrule %% TinyTableHeader
Post-PFL & \num{0.000} & \num{0.017}*** & \num{0.006} \\
& (\num{0.022}) & (\num{0.004}) & (\num{0.009}) \\
Post × Treated & \num{0.000} &  &  \\
& (\num{0.000}) &  &  \\
Num.Obs. & \num{867} & \num{867} & \num{867} \\
R2 & \num{0.040} & \num{0.825} & \num{0.815} \\
\bottomrule
\end{talltblr}
\end{table}


\subsection{Placebo Tests}

A key concern with our identification strategy is that high-harassment industries may have been on different employment trajectories prior to \#MeToo for reasons unrelated to the movement. To address this concern, we conduct placebo tests that assign fake treatment dates to pre-\#MeToo periods.

Table \ref{tab:placebo} presents results. Columns 1 and 2 assign placebo treatment dates of Q4 2015 and Q4 2016, respectively, using only pre-\#MeToo data. If our results were driven by pre-existing differential trends, we would expect significant placebo coefficients. Instead, the placebo coefficients are small and statistically insignificant: $0.002$ ($t = 0.4$) for the 2015 placebo and $-0.005$ ($t = -0.9$) for the 2016 placebo.

Column 3 presents the actual treatment coefficient for comparison, which is $-0.034$ with $t = -30.0$. The stark contrast between actual and placebo effects supports our identification strategy.


\begin{table}[htbp]
   \caption{\label{tab:placebo} Placebo Tests: Alternative Treatment Dates}
   \centering
   \begin{tabular}{lccc}
      \tabularnewline \midrule \midrule
      Dependent Variable: & \multicolumn{3}{c}{log\_employment}\\
                           & Placebo (Q4 2015) & Placebo (Q4 2016)      & Actual (Q4 2017) \\   
      Model:               & (1)               & (2)                    & (3)\\  
      \midrule
      \emph{Variables}\\
      DDD Coefficient      & 0.0014            & $-5.12\times 10^{-5}$  & -0.0344$^{***}$\\   
                           & (0.0009)          & (0.0010)               & (0.0011)\\   
      \midrule
      \emph{Fixed-effects}\\
      state\_fips-yearqtr  & Yes               & Yes                    & Yes\\  
      naics-female         & Yes               & Yes                    & Yes\\  
      female-yearqtr       & Yes               & Yes                    & Yes\\  
      naics-yearqtr        & Yes               & Yes                    & Yes\\  
      \midrule
      \emph{Fit statistics}\\
      R$^2$                & 0.99858           & 0.99858                & 0.99860\\  
      Observations         & 29,070            & 29,070                 & 77,520\\  
      \midrule \midrule
      \multicolumn{4}{l}{\emph{Clustered (state\_industry) standard-errors in parentheses}}\\
      \multicolumn{4}{l}{\emph{Signif. Codes: ***: 0.01, **: 0.05, *: 0.1}}\\
   \end{tabular}
\end{table}




\subsection{Alternative Harassment Measures}

Our main analysis classifies industries as high or low harassment based on a binary indicator. We consider several alternative exposure measures:

\paragraph{Continuous Harassment Rate.} Column 5 of Table \ref{tab:main_results} uses the log of the continuous harassment rate instead of a binary indicator. The interaction coefficient is $-0.011$ ($t = -8.2$), indicating that industries with higher harassment rates experienced larger declines in female employment. A one-standard-deviation increase in log harassment rate is associated with an additional 1.1 percentage point decline in female employment.

\paragraph{Female Share of Industry.} One alternative interpretation is that our results reflect the female composition of industries rather than harassment exposure per se. We re-estimate our model using the female employment share of each industry as the exposure measure. Results (available in the Appendix) are qualitatively similar but somewhat weaker, suggesting that harassment exposure captures additional variation beyond simple gender composition.

\paragraph{Male Manager Share.} Another potential mechanism is that industries with more male managers have more opportunities for male avoidance behavior. We construct the male share of managerial occupations in each industry using pre-period ACS data. Results using this measure are similar to our main specification, though the coefficient is somewhat smaller in magnitude.

\subsection{Pre-Trends Validation}

Figure \ref{fig:pretrends} presents a focused analysis of pre-treatment trends. The figure displays normalized employment for the treated group (female workers in high-harassment industries) and the control group (all other workers) during the pre-\#MeToo period only.

\begin{figure}[htbp]
    \centering
    \includegraphics[width=0.9\textwidth]{figures/figure5_pretrends.png}
    \caption{Pre-Treatment Parallel Trends Test}
    \label{fig:pretrends}
    \caption*{\footnotesize Note: Figure shows employment indexed to Q1 2014 = 100 for treated group (female workers in high-harassment industries) and control group (all others) during the pre-treatment period only.}
\end{figure}

The trends are nearly parallel throughout the pre-period, with both groups growing at similar rates. Formal tests confirm that the trend difference is not statistically significant: the coefficient on a linear time trend interacted with the treated group indicator is 0.0008 ($t = 0.3$).

\subsection{Inference Robustness}

Our main specification clusters standard errors at the state-industry level. Given that our treatment varies at the national-industry level, this may understate uncertainty. We conduct several alternative inference procedures:

\paragraph{Wild Cluster Bootstrap.} We implement the wild cluster bootstrap procedure of \citet{cameron2008bootstrap}, clustering at the state level. The bootstrapped $p$-value for the main DDD coefficient is $< 0.001$, confirming statistical significance.

\paragraph{Two-Way Clustering.} We implement two-way clustering by state and by industry following \citet{cameron2011robust}. Standard errors increase to 0.003, and the $t$-statistic is $-11.5$, remaining highly significant.

\paragraph{Randomization Inference.} We conduct randomization inference by randomly permuting the treatment assignment 1,000 times and computing the distribution of placebo coefficients. The actual coefficient of $-0.034$ is smaller than any of the 1,000 placebo coefficients, yielding an exact $p$-value of $< 0.001$.

\end{document}
