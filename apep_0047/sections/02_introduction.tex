\documentclass[../main.tex]{subfiles}

\begin{document}

\section{Introduction}

On October 5, 2017, the \textit{New York Times} published an exposé detailing decades of sexual harassment allegations against film producer Harvey Weinstein. Within days, actress Alyssa Milano encouraged women to share their own experiences using the hashtag \#MeToo, originally created by activist Tarana Burke in 2006. The resulting movement fundamentally transformed public discourse about workplace sexual harassment, leading to the downfall of powerful figures across entertainment, media, politics, and business, and spurring policy responses ranging from mandatory harassment training to limitations on non-disclosure agreements \citep{frye2020effect, johnson2019metoo}.

Yet alongside the movement's successes emerged a troubling concern: the ``Pence Effect,'' named after Vice President Mike Pence's practice of refusing to dine alone with women other than his wife. Survey evidence suggests that many men responded to heightened harassment awareness not by improving their behavior, but by withdrawing from professional interactions with women altogether. A 2019 Lean In survey found that 60 percent of male managers reported being uncomfortable mentoring, working alone, or socializing with women---up from 46 percent in 2018 \citep{leanin2019}. University of Houston research documented that 27 percent of men avoid one-on-one meetings with women at work, and 21 percent hesitate to hire women for jobs requiring close interpersonal interaction \citep{atwater2019looking}.

This paper asks a stark question: Did the \#MeToo movement, despite its laudable goals, inadvertently harm women's labor market outcomes? If men systematically withdrew from mentoring, hiring, and promoting women in the wake of \#MeToo, the movement may have generated exactly the discrimination it sought to combat. Understanding whether this backlash occurred---and in which contexts---is essential for designing policies that protect workers from harassment without creating new barriers to women's advancement.

We answer this question using a triple-difference research design that exploits three sources of variation. First, we compare outcomes before and after October 2017, when the \#MeToo movement emerged. Second, we compare industries with historically high sexual harassment rates to industries with low rates, on the premise that male avoidance behavior should be concentrated where the perceived liability risk is greatest. Third, we compare female to male employment within industries, controlling for industry-specific shocks that affect both genders equally. This design identifies the differential effect of \#MeToo on female employment in high-harassment industries relative to both male employment and female employment in low-harassment industries.

Our empirical analysis yields three main findings. First, we document a statistically significant and economically meaningful decline in female employment in high-harassment industries following \#MeToo. Our preferred specification, which includes state-by-quarter, industry-by-gender, gender-by-quarter, and industry-by-quarter fixed effects, estimates that female employment fell by 3.4 percent relative to the triple-difference counterfactual ($t = -30.0$). This effect is robust to alternative specifications, including different fixed effects structures, continuous harassment exposure measures, and exclusion of specific industries.

Second, event study estimates reveal no evidence of differential pre-trends in the 15 quarters prior to October 2017, supporting our identifying assumption that high-harassment and low-harassment industries would have evolved similarly absent the \#MeToo movement. The treatment effect emerges immediately in the fourth quarter of 2017 and persists throughout our sample period, consistent with a permanent behavioral shift rather than a transitory response to media attention.

Third, we find substantial heterogeneity across industries. The largest employment effects appear in accommodation and food services, retail trade, and healthcare---precisely the sectors where prior research documents the highest harassment rates \citep{hersch2011compensating, folke2022sexual}. Industries like finance, professional services, and information show near-zero effects, consistent with our identification strategy.

These findings contribute to several literatures. Most directly, we contribute to the nascent economics literature on the \#MeToo movement. Prior work has documented that \#MeToo increased sex crime reporting \citep{levy2020effects}, affected co-authorship patterns in academic economics \citep{bourveau2022}, and influenced corporate governance practices \citep{egan2022metoo}. We provide the first quasi-experimental evidence on \#MeToo's effects on female employment, finding that the movement's labor market consequences were more complex than prior work has suggested.

We also contribute to the broader literature on workplace sexual harassment and labor market outcomes. Economists have long recognized that harassment imposes costs on workers, firms, and the broader economy \citep{willness2007meta, mclaughlin2017economic}. Recent work by \citet{folke2022sexual} documents that harassment reduces women's labor earnings and increases firm exit. Our findings suggest that policies designed to address harassment may generate their own costs if they cause men to disengage from interactions with female colleagues.

Additionally, we contribute to research on unintended consequences of anti-discrimination policy. A substantial literature documents that diversity training and harassment training programs often fail to achieve their stated objectives \citep{kalev2006best, dobbin2019promise}. Laboratory studies find that mandatory training can actually increase discriminatory attitudes among some men \citep{chang2019does}. Our work extends this literature by documenting labor market consequences of heightened harassment awareness, suggesting that the backlash phenomenon is not limited to attitudes but manifests in hiring and employment decisions.

Finally, our paper speaks to policy debates about how to address workplace harassment. The post-\#MeToo period has seen a wave of state-level policies, including mandatory harassment training, limitations on non-disclosure agreements, and extended statutes of limitations \citep{ford2021metoo}. Our findings suggest that awareness-raising alone may be insufficient---and potentially counterproductive---without complementary policies that prevent male withdrawal from mentoring and sponsoring women. Possible interventions include structured mentoring programs, transparent promotion criteria, and accountability systems that reward managers for developing diverse teams.

The remainder of the paper proceeds as follows. Section 2 provides background on the \#MeToo movement and reviews related literature on harassment, training programs, and gender gaps in employment. Section 3 describes our data sources, including the Quarterly Workforce Indicators and EEOC harassment charge statistics. Section 4 develops our triple-difference empirical strategy and discusses identification assumptions. Section 5 presents main results, while Section 6 provides robustness checks including placebo tests and alternative specifications. Section 7 explores mechanisms and heterogeneity. Section 8 discusses implications and limitations, and Section 9 concludes.

\subsection{Related Literature}

Our paper connects to several strands of economic research on gender, workplace dynamics, and policy evaluation.

\paragraph{Sexual Harassment and Labor Market Outcomes.} A growing body of evidence documents the labor market consequences of workplace harassment. \citet{hersch2011compensating} finds that workers receive compensating wage differentials for exposure to harassment risk, implying that harassment functions as a disamenity in the labor market. \citet{mclaughlin2017economic} shows that harassment leads to financial stress, job changes, and career disruptions for victims. Most recently, \citet{folke2022sexual} use Swedish administrative data to document that harassment incidents cause women to exit firms and reduce their subsequent earnings. Our paper extends this literature by examining how \textit{awareness} of harassment---distinct from harassment itself---affects female employment.

\paragraph{The \#MeToo Movement.} Economic research on \#MeToo is still emerging. \citet{levy2020effects} use a triple-difference design to show that \#MeToo increased reporting of sex crimes in the United States by approximately 10 percent. \citet{bourveau2022} document changes in academic co-authorship patterns, finding that senior male economists reduced new collaborations with junior women after \#MeToo. \citet{egan2022metoo} show that investors respond to harassment revelations by demanding governance changes. We contribute to this literature by providing the first estimates of \#MeToo's effects on employment outcomes.

\paragraph{Diversity and Harassment Training.} A substantial literature evaluates workplace training programs. \citet{kalev2006best} find that diversity training rarely increases the representation of women and minorities in management, while mandatory training programs can actually reduce diversity. \citet{dobbin2019promise} document that harassment training fails to prevent harassment and can trigger backlash among men who perceive themselves as unfairly targeted. Laboratory experiments by \citet{chang2019does} find that training increases harassment-supportive attitudes among men who score high on measures of ``social dominance orientation.'' Our findings are consistent with this literature's suggestion that awareness-raising without complementary interventions may be counterproductive.

\paragraph{Gender Gaps in Employment and Advancement.} The gender gap in employment, wages, and leadership remains a central concern in labor economics \citep{goldin2014grand, bertrand2018coase}. Research has identified numerous contributing factors, including occupational segregation \citep{blau2017gender}, differential returns to job attributes \citep{wiswall2018preference}, and discrimination \citep{goldin2000orchestrating}. Our paper suggests an additional mechanism: male avoidance behavior following harassment awareness may create barriers to women's employment and advancement, particularly in industries where men hold gatekeeping positions.

\paragraph{Unintended Consequences of Policy.} Economists have long studied how well-intentioned policies can generate perverse outcomes \citep{peltzman1975effects}. In the context of anti-discrimination policy, \citet{autor2007does} shows that wrongful discharge laws reduced employment of workers they were designed to protect. Our findings parallel this literature by documenting potential unintended consequences of harassment awareness campaigns.

\end{document}
