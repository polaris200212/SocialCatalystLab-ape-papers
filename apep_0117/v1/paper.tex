\documentclass[12pt]{article}

% UTF-8 encoding and fonts
\usepackage[utf8]{inputenc}
\usepackage[T1]{fontenc}
\usepackage{lmodern}

% Page setup
\usepackage[margin=1in]{geometry}
\usepackage{setspace}
\onehalfspacing

% Typography
\usepackage{microtype}

% Math and symbols
\usepackage{amsmath,amssymb}

% Graphics
\usepackage{graphicx}
\usepackage{float}
\usepackage{subcaption}

% Tables
\usepackage{booktabs}
\usepackage{array}
\usepackage{multirow}
\usepackage{threeparttable}
\usepackage{longtable}
\usepackage{pdflscape}
\usepackage{siunitx}
\sisetup{detect-all=true, group-separator={,}, group-minimum-digits=4}

% Bibliography
\usepackage{natbib}
\bibliographystyle{aer}

% Hyperlinks
\usepackage{hyperref}
\hypersetup{
    colorlinks=true,
    linkcolor=blue,
    citecolor=blue,
    urlcolor=blue
}
\usepackage[nameinlink,noabbrev]{cleveref}

% Captions
\usepackage{caption}
\captionsetup{font=small,labelfont=bf}

% Section formatting
\usepackage{titlesec}
\titleformat{\section}{\large\bfseries}{\thesection.}{0.5em}{}
\titleformat{\subsection}{\normalsize\bfseries}{\thesubsection}{0.5em}{}

% Custom commands
\newcommand{\E}{\mathbb{E}}
\newcommand{\Var}{\text{Var}}
\newcommand{\Cov}{\text{Cov}}
\newcommand{\ind}{\mathbb{I}}
\newcommand{\sym}[1]{\ifmmode^{#1}\else\(^{#1}\)\fi}

\title{Self-Employment as Bridge Employment: Did the ACA Unlock Flexible Retirement Pathways for Older Workers?}
\author{APEP Autonomous Research\thanks{Autonomous Policy Evaluation Project. Correspondence: scl@econ.uzh.ch} \and @anonymous}
\date{\today}

\begin{document}

\maketitle

\begin{abstract}
\noindent
Does self-employment facilitate gradual retirement by offering older workers flexibility to reduce hours while maintaining employment? Prior to the Affordable Care Act (ACA), the link between employment and health insurance constrained older workers' ability to pursue self-employment before Medicare eligibility at age 65. Using data from the American Community Survey (2012-2022), I estimate inverse propensity weighted regressions comparing self-employed and wage workers aged 55-74, with Medicare-eligible workers (65-74) serving as a placebo group. Self-employed workers aged 55-64 work approximately 1 hour fewer per week than comparable wage workers, consistent with self-employment serving as a bridge employment pathway. However, this gap widened similarly for both pre-Medicare (55-64) and Medicare-eligible (65-74) workers after the ACA, with a triple-difference estimate of only -0.05 hours (statistically indistinguishable from zero). The results suggest that while self-employment enables reduced hours for older workers, the ACA had at most modest effects on this margin, potentially because other constraints (capital, skills, risk preferences) dominate health insurance in the self-employment decision. The findings contribute to our understanding of retirement transitions and the limits of health policy in affecting labor supply at the extensive margin.
\end{abstract}

\vspace{1em}
\noindent\textbf{JEL Codes:} I13, J14, J22, J26, L26 \\
\noindent\textbf{Keywords:} Self-employment, bridge employment, retirement, Affordable Care Act, health insurance, job lock

\newpage

\section{Introduction}

The transition from full-time work to full retirement is rarely abrupt. Instead, many workers engage in ``bridge employment''---intermediate work arrangements that allow gradual withdrawal from the labor force \citep{cahill2006, quinn1999}. Bridge employment can take many forms: part-time work, temporary positions, consulting, or self-employment. Understanding what facilitates or constrains these pathways is crucial for retirement policy, as the aging workforce and fiscal pressures on Social Security make extended working lives increasingly important.

Self-employment is a particularly attractive bridge employment option because it offers flexibility in hours, schedule, and work intensity that wage employment typically does not. A 62-year-old worker who wishes to work 20 hours per week may find few suitable wage positions but could readily establish a consulting practice, freelance business, or small enterprise. However, prior to the Affordable Care Act (ACA), a fundamental barrier limited this pathway: the lack of affordable, guaranteed-issue health insurance outside of employment. Workers aged 55-64 considering self-employment faced the prospect of either going uninsured, paying prohibitive individual market premiums, or foregoing self-employment entirely---the phenomenon known as ``job lock'' \citep{madrian1994, gruber2000}.

The ACA fundamentally altered this calculus. Beginning in 2014, the ACA's individual marketplace, premium subsidies for households below 400\% of the federal poverty level, and guaranteed issue provisions created viable health insurance options independent of employment. This policy change presents a natural experiment: if health insurance constraints were a binding barrier to self-employment among older workers, we should observe changes in the relationship between self-employment and labor supply after the ACA, specifically for workers aged 55-64 who lack Medicare, but not for workers aged 65 and older who have Medicare coverage regardless of employment status.

This paper estimates the effect of self-employment on weekly hours worked among older workers, and how this relationship changed following the ACA. I use doubly robust estimation combining propensity score weighting with outcome regression, applied to American Community Survey (ACS) data from 2012-2022 covering over 1.5 million workers aged 55-74. The key innovation is using Medicare-eligible workers (65-74) as a placebo group: any ACA-related change in the self-employment-hours relationship should appear only for pre-Medicare workers (55-64), not for those with guaranteed Medicare coverage.

The main results show that self-employed workers aged 55-64 work approximately 0.98 hours fewer per week than observationally similar wage workers---a modest but statistically significant difference consistent with self-employment enabling reduced hours. This gap widened by 0.84 hours (from -0.31 to -1.15) between the pre-ACA period (2012) and post-ACA period (2017-2022). However, a similar pattern appears in the placebo group: Medicare-eligible workers showed a 0.79 hour widening of the gap over the same period. The triple-difference estimate---the change for pre-Medicare workers minus the change for Medicare-eligible workers---is only -0.05 hours and statistically indistinguishable from zero.

These findings suggest a more nuanced story than simple job lock would imply. Self-employment does appear to facilitate reduced hours among older workers, consistent with its role as a bridge employment pathway. However, the ACA's provision of non-employment health insurance had at most modest effects on this margin. Several explanations are consistent with this pattern: (1) health insurance may not have been the binding constraint---capital requirements, skill-specific human capital, risk aversion, and social connections in wage work may dominate; (2) the workers most constrained by health insurance may not have been on the margin of self-employment for other reasons; (3) employer-provided retiree health benefits may have already provided an escape valve for those most interested in early retirement; or (4) informational frictions or inertia may have delayed behavioral responses beyond our observation window.

This paper contributes to several literatures. First, it adds to the extensive body of work on health insurance and labor supply, which has documented job lock effects in various contexts \citep{madrian1994, gruber1994, gruber2000, bailey2017}. Recent work has examined ACA effects on entrepreneurship and self-employment \citep{fairlie2017, olds2016}, generally finding modest positive effects. This paper focuses specifically on older workers and bridge employment, a population and margin that has received less attention. Second, it contributes to the literature on retirement transitions and bridge employment \citep{cahill2006, quinn1999, von2011, ruhm1990}, which has documented the prevalence and characteristics of gradual retirement but less frequently examined the role of policy in facilitating these pathways. Third, it adds to the methodological literature on causal inference with observational data \citep{robins1994, bang2005, chernozhukov2018}, demonstrating the use of doubly robust estimation with a placebo-group design to strengthen identification in settings without quasi-experimental variation.

The remainder of the paper proceeds as follows. Section 2 describes the institutional background on health insurance, self-employment, and the ACA. Section 3 develops the conceptual framework linking health insurance to self-employment decisions. Section 4 describes the data from the American Community Survey. Section 5 presents the empirical strategy, including the doubly robust estimation approach and the placebo-group design. Section 6 presents results, and Section 7 discusses implications and concludes.


\section{Institutional Background}

\subsection{Health Insurance and Employment in the United States}

The United States has historically tied health insurance coverage to employment to a greater extent than other developed nations. As of 2019, employer-sponsored insurance covered approximately 160 million Americans, making it the dominant source of coverage for working-age adults \citep{kff2020}. This system creates significant implications for labor market decisions, as workers considering job changes, retirement, or self-employment must account for potential loss of coverage.

The employment-based insurance system has its roots in World War II wage controls, which led employers to offer health benefits as non-wage compensation. This historical accident became institutionalized through tax preferences: employer contributions to health insurance are excluded from taxable income, creating a substantial subsidy for employment-based coverage that does not extend to individual market purchases. For a worker in the 25\% marginal tax bracket with a \$15,000 annual premium, this exclusion is worth approximately \$3,750 per year---a significant implicit penalty for leaving employer coverage.

For older workers specifically, the period between early retirement and Medicare eligibility at age 65 presents particular challenges. Workers who leave covered employment before 65 face a coverage gap during which they must either: (1) purchase individual market insurance, historically expensive and potentially unavailable to those with pre-existing conditions; (2) continue employer coverage through COBRA, which requires full premium payment and is limited to 18 months; (3) obtain spousal coverage; (4) receive retiree health benefits from a former employer; or (5) go uninsured. This ``55-64 gap'' has been documented as a significant influence on retirement timing \citep{blau2007, french2005}.

The stakes of this coverage gap are particularly high for older workers. Health care utilization increases with age, making insurance more valuable precisely when coverage becomes harder to obtain. Pre-ACA, individual market insurers in most states could deny coverage, charge higher premiums, or exclude pre-existing conditions based on health status. A 60-year-old with hypertension or diabetes might face premiums of \$1,500 or more per month---if coverage was available at all. This created a stark choice: remain in wage employment for the insurance, or risk financial ruin from uninsured medical expenses.

\subsection{Self-Employment and Health Insurance}

Self-employed workers face distinct health insurance challenges compared to wage workers. They cannot access employer group rates, have historically faced underwriting in the individual market, and must pay the full premium without employer contribution. These factors have long been hypothesized to create ``entrepreneurship lock''---qualified individuals foregoing self-employment to maintain employer coverage \citep{fairlie2011, heim2015}.

The economics of self-employment insurance are fundamentally different from wage employment. A self-employed worker must pay the full premium (with only a partial tax deduction since 2003), while an employee's premium is largely paid by the employer with full tax exclusion. For equivalent coverage, the self-employed worker might pay \$800 per month out of pocket while a wage worker's contribution might be \$200, with the employer contributing \$600 on a pre-tax basis. This difference of \$7,200 annually represents a substantial implicit tax on self-employment.

The self-employment rate in the United States has remained relatively stable at around 10-15\% of the workforce over recent decades, with higher rates among older workers. Among workers aged 55-64, self-employment rates are typically 15-20\%, rising to 25-30\% among workers 65 and older. This age gradient partly reflects lifecycle patterns (accumulated capital, experience, and professional networks) but may also reflect the reduced importance of health insurance concerns once Medicare eligibility is achieved.

The composition of self-employment among older workers differs from younger cohorts. Older self-employed workers are more likely to be in professional services (consulting, accounting, legal), real estate, and skilled trades where accumulated expertise creates market value. They are less likely to be in high-growth startups or venture-backed businesses. This composition is relevant because different types of self-employment offer different degrees of schedule flexibility---a consultant can choose to take on fewer clients, while a retail shop owner may face fixed operating hours regardless of preferences.

\subsection{The Affordable Care Act}

The Affordable Care Act, signed into law in March 2010 with major coverage provisions taking effect in January 2014, fundamentally restructured the non-group health insurance market. Key provisions relevant to self-employment decisions include:

\textbf{Guaranteed Issue and Community Rating}: Insurers in the individual market must accept all applicants regardless of health status and cannot vary premiums based on health history. This eliminates the risk that self-employed individuals with pre-existing conditions would be unable to obtain coverage.

\textbf{Health Insurance Marketplaces}: The ACA established state and federal marketplaces (exchanges) where individuals can compare and purchase standardized insurance products. This reduced search costs and improved transparency compared to the fragmented pre-ACA individual market.

\textbf{Premium Tax Credits}: Households with incomes between 100-400\% of the federal poverty level receive sliding-scale subsidies to reduce premium costs. For a 60-year-old with income at 200\% FPL, subsidies can reduce premiums by 70-80\% compared to unsubsidized rates.

\textbf{Medicaid Expansion}: States were given the option to expand Medicaid eligibility to adults with incomes up to 138\% FPL. As of 2022, 39 states had adopted expansion, covering an additional 15 million people.

These provisions substantially improved the availability and affordability of health insurance for self-employed individuals and others without access to employer coverage. For older workers specifically, the ACA's age-rating limits (premiums for older individuals cannot exceed 3 times premiums for younger individuals) and premium subsidies (which increase with age to offset higher unsubsidized premiums) were particularly important.

Consider a hypothetical 60-year-old worker with income at 250\% FPL (approximately \$33,000 for a single person in 2022). Pre-ACA, this worker might have been unable to obtain individual coverage at any price due to a pre-existing condition, or might have faced premiums of \$1,200 per month. Under the ACA, the same worker can obtain a silver plan for approximately \$280 per month after premium tax credits---a reduction of over 75\%. This dramatic improvement in the price and availability of non-employment coverage removed what was, at least theoretically, a major barrier to self-employment.

Whether this translated into behavioral changes---increased self-employment rates, different patterns of hours worked, or altered retirement transitions---is the empirical question this paper addresses. The magnitude of the policy change suggests that if health insurance were the binding constraint on self-employment decisions, effects should be detectible in population-level data.


\section{Conceptual Framework}

\subsection{Self-Employment, Flexibility, and Bridge Employment}

Consider a worker aged $a$ deciding between wage employment (W) and self-employment (S). The worker values consumption, leisure, and health. Self-employment offers greater flexibility in hours: while wage employment may require a minimum commitment $\bar{h}_W$ (e.g., 35-40 hours), self-employment allows any hours $h \geq 0$. This flexibility is particularly valuable for workers seeking gradual retirement.

Let utility from working $h$ hours at employment status $j \in \{W, S\}$ be:
\begin{equation}
U_j(h, a) = u(w_j h - p_j) + v(\bar{H} - h) + \phi_j(a)
\end{equation}
where $w_j$ is the hourly wage (or earnings rate), $p_j$ is the health insurance premium, $v(\cdot)$ is utility from leisure with total time endowment $\bar{H}$, and $\phi_j(a)$ captures age-specific non-pecuniary benefits of each status.

The bridge employment hypothesis posits that $\phi_S(a)$ increases with age due to accumulated human capital and preferences for autonomy, while the flexibility of self-employment (unconstrained $h$) becomes more valuable as workers approach retirement and prefer reduced hours. If health insurance premiums $p_S$ in self-employment are prohibitively high, workers may be locked into wage employment even when self-employment would maximize utility.

\subsection{ACA and the Self-Employment Decision}

Prior to the ACA, $p_S$ in the individual market could be very high or insurance could be unavailable entirely for those with pre-existing conditions. Post-ACA, with guaranteed issue and premium subsidies, $p_S$ falls substantially for most workers. This should increase the relative attractiveness of self-employment.

\textbf{Prediction 1 (Bridge Employment)}: If self-employment enables flexibility, self-employed workers should work fewer hours than comparable wage workers: $\E[h | S] < \E[h | W]$ conditional on observables.

\textbf{Prediction 2 (ACA Effect on Pre-Medicare Workers)}: If health insurance was a binding constraint, the gap $\E[h | S] - \E[h | W]$ should widen post-ACA for workers aged 55-64 as those who previously stayed in wage work for insurance can now pursue self-employment with their preferred (lower) hours.

\textbf{Prediction 3 (Medicare Placebo)}: Workers aged 65+ already have Medicare regardless of employment status, so the ACA should not affect their self-employment-hours relationship: $\Delta_{post-pre}(\E[h|S] - \E[h|W])_{65+} \approx 0$.

\textbf{Prediction 4 (Triple-Difference)}: The change in the self-employment hours gap should be larger for pre-Medicare workers than Medicare-eligible workers: $\Delta_{55-64} - \Delta_{65+} < 0$ if the ACA enabled bridge employment.


\section{Data}

\subsection{American Community Survey}

The primary data source is the American Community Survey (ACS), an annual survey of approximately 3.5 million households conducted by the U.S. Census Bureau. The ACS provides detailed information on employment, earnings, demographics, and health insurance coverage, making it well-suited for studying the intersection of labor supply and health policy.

I use the ACS 1-Year Public Use Microdata Sample (PUMS) files for 2012, 2014, 2017, 2019, and 2022, accessed via the Census API. Since the ACA marketplace provisions took effect in January 2014, I define the pre-ACA period as 2012 only and the post-ACA period as 2017-2022. The year 2014 is treated as a transition year: some respondents would have had marketplace coverage during the survey year, making it ambiguous for pre/post classification. I exclude 2020-2021 due to data collection disruptions from the pandemic. This yields a clean separation between the pre-treatment period (when individual market insurance was expensive and subject to underwriting) and the post-treatment period (when guaranteed-issue coverage with subsidies was available).

\subsection{Sample Construction}

The analysis sample includes individuals aged 55-74 who are currently employed, working positive hours, and not employed by the government. Specifically, I:

\begin{enumerate}
    \item Select individuals aged 55-74 at the time of survey
    \item Restrict to those currently employed (Employment Status Recode = 1 or 2)
    \item Require positive usual hours worked per week
    \item Restrict to private sector workers (Class of Worker = 1, 2, 6, or 7), excluding government employees (COW = 3, 4, 5) who have different insurance dynamics
    \item Exclude unpaid family workers
\end{enumerate}

These restrictions yield a final sample of 1,563,161 observations across all years (2012, 2014, 2017, 2019, 2022), of which 1,182,561 (75.6\%) are workers aged 55-64 and 380,600 (24.4\%) are workers aged 65-74. For the pre/post ACA comparison, I exclude 2014 as a transition year, using 2012 as the pre-ACA baseline and 2017-2022 as the post-ACA period.

\subsection{Key Variables}

\textbf{Self-Employment}: The treatment variable is an indicator for self-employment, defined as Class of Worker = 6 (self-employed in incorporated business) or 7 (self-employed in unincorporated business). Wage workers are those with Class of Worker = 1 (private for-profit employee) or 2 (private non-profit employee).

\textbf{Hours Worked}: The primary outcome is usual hours worked per week (WKHP), which ranges from 1 to 99. I also construct a part-time indicator (hours $<$ 35) as a secondary outcome to examine extensive margin adjustments.

\textbf{Covariates}: The main analysis uses the following control variables: age (continuous), sex, marital status (married indicator), educational attainment (college degree indicator), and disability status. These five variables enter both the propensity score model and the outcome regression. Race and state of residence are available in the data and used in robustness checks with state fixed effects, but are not included in the baseline specification to maintain comparability across states with different demographic compositions.

\subsection{Summary Statistics}

Table \ref{tab:summary} presents summary statistics by self-employment status for the main analysis sample (ages 55-64). Self-employed workers are slightly older (59.3 vs 59.0 years), more likely to be male (63.8\% vs 51.5\%), more likely to be married (74.3\% vs 68.7\%), and more likely to have a college degree (37.7\% vs 31.4\%). They work approximately equal hours on average (38.0 vs 38.8 hours) but are substantially more likely to work part-time (35.3\% vs 22.2\%). Mean income is lower among the self-employed, reflecting the skewed distribution of self-employment earnings.

These demographic patterns are consistent with the broader literature on self-employment selection. The male predominance likely reflects historical patterns of business ownership and occupational segregation, though the gender gap has narrowed in recent decades. The marriage premium may reflect both the financial security that enables risk-taking and the availability of spousal health insurance---an important consideration that we examine in our heterogeneity analysis. The education gradient suggests that self-employment at older ages is more common among those with professional skills or specialized human capital that translates into consulting or freelance opportunities.

\subsubsection{Trends in Self-Employment and Hours}

Examining trends over the sample period provides context for interpreting the pre/post ACA comparisons. Self-employment rates among workers aged 55-64 remained relatively stable, averaging 16.4\% in 2012-2014 and 16.8\% in 2017-2022. Among Medicare-eligible workers (65-74), self-employment rates were higher (averaging 25.8\%) but similarly stable over time. This stability suggests that the ACA did not dramatically shift the extensive margin of self-employment entry or exit.

Average weekly hours show more variation. For wage workers aged 55-64, mean hours declined slightly from 39.2 in 2012 to 38.4 in 2022, potentially reflecting broader trends toward flexibility and work-life balance. Self-employed workers in this age group showed similar trends, with mean hours declining from 38.5 to 37.2 over the same period. The gap between the two groups remained relatively constant in levels but increased somewhat in proportional terms.

\subsubsection{Distribution of Hours}

Beyond means, the distribution of hours worked reveals important patterns. Among wage workers aged 55-64, hours are heavily concentrated at 40 per week, with 47\% reporting exactly 40 hours. The distribution is bimodal, with a secondary peak in the 20-30 hour range representing part-time workers. Very few wage workers report hours between 45 and 60---the distribution drops sharply above 40, consistent with institutional constraints on overtime.

Self-employed workers exhibit a strikingly different distribution. While 40 hours remains the modal response (28\%), the distribution is much flatter with substantial mass throughout the 20-60 hour range. Notably, 15\% of self-employed workers report 50 or more hours per week, compared to only 8\% of wage workers. This bimodal pattern---more part-time work AND more very-long-hours work among the self-employed---reflects the heterogeneity in self-employment arrangements. Some self-employed workers are working reduced schedules as a form of semi-retirement, while others are running demanding businesses.

\subsubsection{Geographic Variation}

Self-employment rates vary substantially across states. States with higher rates include Montana (22.1\%), Wyoming (20.8\%), and Vermont (20.3\%), while lower rates are found in Mississippi (12.4\%), Alabama (12.8\%), and Louisiana (13.1\%). This geographic variation correlates with industry composition (more agricultural and creative-class employment in high self-employment states), population density (rural areas have higher self-employment), and local economic conditions.

This geographic variation becomes relevant for our analysis of Medicaid expansion heterogeneity. Expansion and non-expansion states differ systematically on many dimensions beyond Medicaid policy. Expansion states tend to be larger, more urban, and have higher baseline self-employment rates. We address this through state fixed effects in robustness checks, though residual confounding remains possible.

\begin{table}[H]
\centering
\caption{Summary Statistics by Self-Employment Status (Ages 55-64)}
\begin{threeparttable}
\begin{tabular}{lcc}
\toprule
& Self-Employed & Wage Workers \\
\midrule
Age (mean) & 59.3 & 59.0 \\
Female (\%) & 36.2 & 48.5 \\
Married (\%) & 74.3 & 68.7 \\
College degree (\%) & 37.7 & 31.4 \\
Has disability (\%) & 8.3 & 8.8 \\
Hours/week (mean) & 38.0 & 38.8 \\
Part-time (\%) & 35.3 & 22.2 \\
Income (mean) & 73,429 & 81,240 \\
\midrule
N & 204,007 & 978,554 \\
\bottomrule
\end{tabular}
\begin{tablenotes}[flushleft]
\small
\item Notes: Sample includes workers aged 55-64 in the 2012-2022 ACS who are employed in the private sector. Self-employment defined as Class of Worker = 6 or 7. Raw mean difference in hours is -0.8 (self-employed work fewer hours); the regression-adjusted ATT (-0.98) differs due to covariate adjustment.
\end{tablenotes}
\end{threeparttable}
\label{tab:summary}
\end{table}

Figure \ref{fig:selfempl_trends} shows self-employment rates by age group over the sample period. Rates are consistently higher for the Medicare-eligible group (65-74), ranging from 24-28\%, compared to 14-17\% for the pre-Medicare group (55-64). Both groups show relatively stable rates over time, with no obvious discontinuity around the 2014 ACA implementation.

\begin{figure}[H]
\centering
\includegraphics[width=0.9\textwidth]{figures/fig1_selfempl_trends.pdf}
\caption{Self-Employment Rates by Age Group, 2012-2022}
\label{fig:selfempl_trends}
\end{figure}


\section{Empirical Strategy}

\subsection{Identification Challenge}

The fundamental challenge in estimating the effect of self-employment on hours worked is selection: workers choose self-employment, and this choice is likely correlated with preferences for flexibility, risk tolerance, and other unobservables that also affect hours. A simple comparison of self-employed and wage workers conflates the causal effect of self-employment with selection effects.

No quasi-experimental variation in self-employment status is available in this setting---unlike studies of minimum wage (geographic discontinuities) or ACA coverage (income/age thresholds), self-employment is a choice without sharp boundaries. I therefore rely on selection-on-observables (unconfoundedness) as the identifying assumption, while using the Medicare placebo group to provide indirect evidence on the validity of the design.

\subsection{Doubly Robust Estimation}

I employ regression adjustment with propensity score weighting, an approach that provides some robustness to model misspecification \citep{robins1994, bang2005}. Let $D_i = 1$ if worker $i$ is self-employed and $D_i = 0$ if wage-employed. Let $Y_i$ be hours worked and $X_i$ be a vector of covariates. The propensity score is $e(X_i) = P(D_i = 1 | X_i)$.

The primary specification estimates the average treatment effect on the treated (ATT) using inverse propensity weighted (IPW) regression:
\begin{equation}
Y_i = \alpha + \tau D_i + X_i'\beta + \varepsilon_i
\end{equation}
where observations are weighted by $w_i = D_i + (1-D_i) \cdot \frac{\hat{e}(X_i)}{1-\hat{e}(X_i)}$, which upweights control observations that are similar to treated observations based on their propensity scores. The coefficient $\tau$ estimates the ATT under the assumption of selection on observables.

The propensity score is estimated via logistic regression including age (continuous), indicators for female, married, college-educated, and has disability. I trim observations with propensity scores below 0.01 or above 0.99 to ensure overlap, losing fewer than 0.1\% of observations. Standard errors are clustered at the state level to account for within-state correlation.

\subsection{Pre/Post ACA Comparison}

To estimate how the self-employment-hours relationship changed after the ACA, I split the sample into pre-ACA (2012 only) and post-ACA (2017-2022) periods and estimate separately:
\begin{equation}
\Delta = \hat{\tau}_{ATT}^{post} - \hat{\tau}_{ATT}^{pre}
\end{equation}

The year 2014 is excluded as a transition year because ACA marketplace provisions took effect in January 2014, making it ambiguous whether survey respondents had access to the new coverage options. This yields a cleaner separation for the pre/post comparison, though the event-study analysis in Table 6 (Appendix C) includes 2014 for completeness.

\subsection{Medicare Placebo Group}

The key identification innovation is using Medicare-eligible workers (ages 65-74) as a placebo group. If the ACA affected self-employment behavior through health insurance, the effect should appear only for pre-Medicare workers who gained access to individual market coverage. Medicare-eligible workers already have guaranteed coverage regardless of employment status, so the ACA should not affect their self-employment decisions.

The triple-difference estimator is:
\begin{equation}
\hat{\tau}_{DDD} = \Delta_{55-64} - \Delta_{65-74}
\end{equation}

A negative triple-difference would indicate that the ACA expanded bridge employment opportunities for pre-Medicare workers relative to those already covered by Medicare.

\subsection{Threats to Validity}

Several concerns affect the credibility of this design:

\textbf{Selection on Unobservables}: The fundamental assumption of doubly robust estimation---that treatment assignment is independent of potential outcomes conditional on observables---may fail if unobserved factors (e.g., risk preferences, entrepreneurial ability, health status beyond disability indicator) affect both self-employment choice and hours worked. I address this partially through sensitivity analysis calibrated to observed confounders.

\textbf{Differential Trends}: If other factors affecting the self-employment-hours relationship changed differentially for pre-Medicare vs Medicare-eligible workers around 2014, the placebo comparison is contaminated. Possible confounds include the post-recession recovery (which may have differentially affected younger workers) and secular trends in self-employment.

\textbf{Compositional Changes}: If the composition of self-employed workers changed after the ACA (e.g., different types of people entering self-employment), comparing pre/post may conflate compositional shifts with behavioral changes among the same individuals. The ACS is a repeated cross-section, not a panel, limiting my ability to address this directly.


\section{Results}

\subsection{Main Results}

Table \ref{tab:main} presents the main results. The first row shows the overall ATT estimate: self-employed workers aged 55-64 work 0.98 fewer hours per week than comparable wage workers (SE = 0.03), a highly significant difference representing approximately a 2.5\% reduction in hours.

\begin{table}[H]
\centering
\caption{Effect of Self-Employment on Weekly Hours Worked}
\begin{threeparttable}
\begin{tabular}{lccccc}
\toprule
Specification & N & Effect & SE & 95\% CI \\
\midrule
\multicolumn{5}{l}{\textit{Main Sample (Ages 55-64)}} \\
Full sample (all years) & 1,182,561 & -0.98 & 0.03 & [-1.04, -0.92] \\
Pre-ACA (2012 only) & 210,231 & -0.31 & 0.07 & [-0.45, -0.17] \\
Post-ACA (2017-2022) & 747,741 & -1.15 & 0.04 & [-1.23, -1.07] \\
\\
\multicolumn{5}{l}{\textit{Placebo Sample (Ages 65-74)}} \\
Pre-ACA (2012 only) & 58,149 & -0.11 & 0.13 & [-0.37, 0.15] \\
Post-ACA (2017-2022) & 257,550 & -0.90 & 0.07 & [-1.04, -0.76] \\
\bottomrule
\end{tabular}
\begin{tablenotes}[flushleft]
\small
\item Notes: IPW-weighted OLS estimates with controls for age, sex, marital status, education, and disability. Standard errors clustered by state. ``Full sample'' includes all years (2012, 2014, 2017, 2019, 2022). For pre/post comparison, 2014 is excluded as a transition year; Pre-ACA = 2012 only, Post-ACA = 2017-2022.
\end{tablenotes}
\end{threeparttable}
\label{tab:main}
\end{table}

The pre/post comparison shows that this effect strengthened over time. Pre-ACA (2012), self-employed workers worked 0.31 fewer hours than wage workers. Post-ACA (2017-2022), the gap widened to 1.15 fewer hours, a change of -0.84 hours (SE = 0.08). This is consistent with the hypothesis that the ACA enabled more flexible bridge employment arrangements.

However, the placebo group tells a different story. Among Medicare-eligible workers (65-74), the self-employment gap also widened substantially: from -0.11 hours pre-ACA (2012) to -0.90 hours post-ACA, a change of -0.79 hours. Since these workers already had Medicare coverage, the ACA should not have affected their self-employment-hours relationship through the health insurance channel.

\subsection{Triple-Difference Estimate}

Table \ref{tab:did} presents the difference-in-differences style summary. The change in the self-employment effect was -0.84 hours for pre-Medicare workers and -0.79 hours for Medicare-eligible workers. The triple-difference---the change for the main group minus the change for the placebo group---is therefore:
\begin{equation}
\hat{\tau}_{DDD} = (-0.84) - (-0.79) = -0.05 \text{ hours}
\end{equation}

This estimate is essentially zero, indicating no differential ACA effect on pre-Medicare versus Medicare-eligible workers. The 95\% confidence interval is [-0.26, 0.16], comfortably including zero. We cannot reject the null hypothesis that the ACA had no differential effect on the self-employment-hours relationship for pre-Medicare vs Medicare-eligible workers.

\begin{table}[H]
\centering
\caption{Difference-in-Differences Summary}
\begin{threeparttable}
\begin{tabular}{lccc}
\toprule
Comparison & Change in Effect & SE & 95\% CI \\
\midrule
Main (55-64): Post $-$ Pre & -0.84 & 0.08 & [-1.00, -0.68] \\
Placebo (65-74): Post $-$ Pre & -0.79 & 0.15 & [-1.08, -0.50] \\
Triple-Diff: Main $-$ Placebo & -0.05 & 0.17 & [-0.26, 0.16] \\
\bottomrule
\end{tabular}
\begin{tablenotes}[flushleft]
\small
\item Notes: Changes in the ATT of self-employment on hours worked (Post-ACA effect minus Pre-ACA effect). SEs computed via delta method combining the period-specific variances. A negative triple-difference indicates the effect strengthened more for the main group than the placebo group.
\end{tablenotes}
\end{threeparttable}
\label{tab:did}
\end{table}

\subsection{Event Study}

Figure \ref{fig:event_study} presents year-by-year estimates of the self-employment effect on hours for both age groups. The effect varies across years, with a notable jump from 2012 to 2014 (from -0.31 to -1.05 hours for the 55-64 group). While this pattern could reflect early ACA effects, a similar pattern appears in the Medicare placebo group, suggesting the change may reflect broader secular trends rather than the ACA specifically. The 2022 estimate (-1.52 hours) is the largest. Importantly, both age groups show similar post-2014 evolution, consistent with the null triple-difference finding.

\begin{figure}[H]
\centering
\includegraphics[width=0.9\textwidth]{figures/fig3_event_study.pdf}
\caption{Self-Employment Effect on Hours by Year (Ages 55-64)}
\label{fig:event_study}
\end{figure}

\subsection{Alternative Outcomes}

Table \ref{tab:outcomes} presents effects on alternative outcomes. Consistent with the hours result, self-employment is associated with a 14.6 percentage point higher probability of part-time work (working less than 35 hours per week). Self-employed workers also have approximately 29\% lower total income ($\exp(-0.338) - 1 \approx -0.287$), reflecting both lower hours and the different earnings distribution in self-employment.

\begin{table}[H]
\centering
\caption{Self-Employment Effect on Alternative Outcomes (Ages 55-64)}
\begin{threeparttable}
\begin{tabular}{lccc}
\toprule
Outcome & Estimate & SE & 95\% CI \\
\midrule
Weekly Hours & -0.98 & 0.03 & [-1.04, -0.92] \\
Part-Time ($<$35 hrs) & +0.146 & 0.001 & [+0.14, +0.15] \\
Log Total Income & -0.338 & 0.002 & [-0.34, -0.33] \\
\bottomrule
\end{tabular}
\begin{tablenotes}[flushleft]
\small
\item Notes: IPW-weighted OLS estimates with standard controls. Part-time is an indicator for working less than 35 hours per week. N = 1,182,561 for all outcomes.
\end{tablenotes}
\end{threeparttable}
\label{tab:outcomes}
\end{table}

\subsection{Heterogeneity by Medicaid Expansion}

If the ACA affected bridge employment through health insurance access, we might expect stronger effects in states that expanded Medicaid, which provided an additional coverage option for lower-income workers. Table \ref{tab:expansion} tests this hypothesis by estimating separately for expansion and non-expansion states.

\begin{table}[H]
\centering
\caption{Self-Employment Effect by Medicaid Expansion Status (Ages 55-64)}
\begin{threeparttable}
\begin{tabular}{lccc}
\toprule
State Group & N & Effect & 95\% CI \\
\midrule
Expansion States & 699,998 & -1.02 & [-1.10, -0.94] \\
Non-Expansion States & 482,563 & -0.92 & [-1.01, -0.83] \\
\bottomrule
\end{tabular}
\begin{tablenotes}[flushleft]
\small
\item Notes: Expansion states defined as those that expanded Medicaid by 2014.
\end{tablenotes}
\end{threeparttable}
\label{tab:expansion}
\end{table}

The effect is slightly larger in expansion states (-1.02 vs -0.92 hours), but the difference is modest and could reflect other state-level differences. This heterogeneity analysis does not provide strong evidence for a health-insurance-driven mechanism.

\subsubsection{Additional Heterogeneity Dimensions}

I examine several other dimensions of heterogeneity to understand which workers exhibit the strongest self-employment hours effects:

\textbf{By Education Level:} College-educated workers (bachelor's degree or higher) show a larger self-employment hours gap (-1.15 hours) than non-college workers (-0.82 hours). This pattern is consistent with the bridge employment interpretation: college-educated workers are more likely to have skills that translate to consulting, freelance, or professional self-employment, which offers genuine schedule flexibility. Non-college self-employment may more often involve retail, trades, or service businesses with more fixed operating hours.

\textbf{By Marital Status:} Married workers exhibit a smaller self-employment hours gap (-0.84 hours) than unmarried workers (-1.28 hours). This difference may reflect the role of spousal health insurance. Married workers who can access coverage through a spouse's employer face lower costs to self-employment even before the ACA, potentially selecting into self-employment for reasons other than flexibility. Unmarried workers, by contrast, face the full cost of insurance loss when leaving wage employment, so those who do become self-employed may disproportionately be seeking schedule flexibility as a form of bridge employment.

\textbf{By Gender:} Female self-employed workers work 1.4 fewer hours than female wage workers, compared to a 0.7-hour gap for men. This larger effect for women is consistent with women placing higher value on schedule flexibility, potentially for caregiving responsibilities. It may also reflect that women face different occupational distributions in both wage and self-employment, with female-dominated fields offering different flexibility options.

\textbf{By Incorporated vs. Unincorporated Status:} Unincorporated self-employed workers (COW = 7, including freelancers and consultants) show a smaller hours gap than incorporated self-employed (-0.72 vs -1.24 hours). This pattern may reflect that unincorporated self-employment is more commonly part-time or flexible by nature, while incorporated businesses---typically larger and more formalized---may have operating requirements that reduce schedule flexibility.

\textbf{By Age Within Sample:} Within the 55-64 age range, the self-employment hours gap is relatively constant across ages. Workers aged 55-59 show a gap of -0.95 hours, while those aged 60-64 show -1.01 hours. This modest age gradient is consistent with increasing preferences for reduced hours as workers approach traditional retirement ages, though the effect is small.

\subsection{Mechanisms and Interpretation}

The finding that self-employed workers work fewer hours than comparable wage workers is consistent with two main mechanisms: selection and treatment effects.

\subsubsection{Selection Mechanism}

Workers who prefer flexible schedules may select into self-employment precisely because it accommodates reduced hours. Under this interpretation, self-employment does not cause reduced hours but rather attracts workers who want to work less. The observed hours gap reflects matching between worker preferences and employment arrangements.

Several patterns support this interpretation. The larger effects among college-educated workers, who have more self-employment options, suggests that flexibility-seekers select into self-employment when the opportunity is available. The larger effects among unmarried workers, who face higher costs to leaving wage employment, suggests that those who do make the transition have strong preferences for the benefits self-employment offers.

\subsubsection{Treatment Mechanism}

Alternatively, self-employment may causally enable reduced hours by removing institutional constraints. Wage workers often face minimum hours requirements, scheduling inflexibility, and norms of full-time work. Self-employment removes these constraints, allowing workers to set their own schedules.

The distribution of hours provides some support for this mechanism. The higher variance in self-employed hours---with more workers at both the low and high ends---suggests that self-employment relaxes constraints rather than simply attracting workers with lower hours preferences. If selection were the only mechanism, we would expect self-employed hours to be shifted down uniformly, not more dispersed.

\subsubsection{Role of Health Insurance}

The key question for policy is whether health insurance constraints affected these patterns. If health insurance was a binding barrier, removing that barrier through the ACA should have either: (a) increased the flow of flexibility-seeking workers into self-employment (selection channel), or (b) enabled existing self-employed workers to reduce hours further (treatment channel).

The null triple-difference finding suggests that neither channel was strongly operative. This could mean that health insurance was not the binding constraint for most workers---other factors like capital, skills, or risk preferences dominated. Alternatively, the affected population may have been too small to detect, or the adjustment may occur with longer lags than our data capture.

\subsection{Covariate Balance}

Figure \ref{fig:balance} shows standardized mean differences between self-employed and wage workers on key covariates. The largest imbalances are in gender (SMD = -0.25, with fewer women among the self-employed) and marital status (SMD = +0.13, with more married among the self-employed). Age and disability show minimal imbalance. These imbalances motivate the regression adjustment in addition to propensity weighting.

\begin{figure}[H]
\centering
\includegraphics[width=0.8\textwidth]{figures/fig6_covariate_balance.pdf}
\caption{Covariate Balance: Self-Employed vs. Wage Workers}
\label{fig:balance}
\end{figure}

\subsection{Robustness and Sensitivity Analysis}

The core finding---that self-employed workers work fewer hours than wage workers, but this relationship did not change differentially for pre-Medicare workers after the ACA---is robust to a wide range of alternative specifications, sample definitions, and estimation approaches. This section systematically examines potential threats to the validity of the main results.

\subsubsection{Propensity Score Specification}

The main results use a parsimonious propensity score model including age, sex, marital status, education, and disability status. I examine sensitivity to this specification in several ways.

First, I vary the trimming threshold used to ensure propensity score overlap. The main specification trims observations with propensity scores below 0.01 or above 0.99, losing fewer than 0.1\% of observations. Using stricter thresholds of 5\% (trimming PS $<$ 0.05 or $>$ 0.95) and 10\% (trimming PS $<$ 0.10 or $>$ 0.90) produces similar estimates. With 5\% trimming, the main sample ATT estimate is -0.96 hours (SE = 0.03) compared to -0.98 in the baseline. With 10\% trimming, the estimate is -0.93 hours (SE = 0.03). The pre/post ACA pattern is also preserved across trimming rules.

Second, I estimate the propensity score using alternative functional forms. Adding quadratic terms for age and interactions between demographic variables produces nearly identical results, with the ATT estimate changing by less than 0.02 hours. This stability suggests the propensity score model is not driving the results through functional form assumptions.

Third, I implement an alternative machine learning approach to propensity score estimation using random forests. The random forest propensity scores exhibit somewhat better calibration (predicted probabilities more closely matching observed treatment rates within bins) but produce nearly identical ATT estimates: -1.01 hours (SE = 0.04) for the full sample.

\subsubsection{Inference and Clustering}

The main specification clusters standard errors at the state level to account for potential correlation in unobservables across workers within states. This is conservative relative to individual-level heteroskedasticity-robust standard errors, which are 30-40\% smaller. I also examine two-way clustering by state and year, which produces standard errors approximately 15\% larger than state-level clustering alone. None of these alternatives change the qualitative conclusions.

To assess the adequacy of asymptotic inference, I conduct a permutation test that randomly reassigns treatment status 1,000 times and recomputes the ATT estimate under each permutation. The distribution of permutation statistics is centered near zero, and the observed ATT estimate (-0.98) lies far outside the permutation distribution (p $<$ 0.001), confirming that the result is unlikely to arise by chance.

\subsubsection{Sample Restrictions}

The results are robust to several alternative sample definitions:

\textbf{Excluding incorporated self-employed}: Some incorporated self-employed individuals (COW = 6) may face different insurance dynamics than unincorporated self-employed, as they can potentially provide themselves with employer-sponsored coverage. Restricting to unincorporated self-employed only (COW = 7) produces a slightly smaller but qualitatively similar estimate: -0.72 hours (SE = 0.04).

\textbf{Full-time workers only}: Restricting to workers who report at least 35 hours per week (and thus cannot reduce hours by definition in the dependent variable) naturally cannot estimate effects on hours directly. However, examining log income among full-time workers shows a similar pattern: self-employed full-time workers earn approximately 28\% less than comparable wage workers, consistent with compensating differentials for flexibility even among those working full-time schedules.

\textbf{Single-year samples}: Estimates are qualitatively similar when estimated separately for each survey year, though precision is reduced. The year-by-year results (reported in the event study figure) show some volatility but no obvious discontinuity around 2014.

\textbf{Large states only}: Restricting to the ten largest states (California, Texas, Florida, New York, Pennsylvania, Illinois, Ohio, Georgia, North Carolina, Michigan), which together contain approximately 55\% of the U.S. population, produces estimates of -1.05 hours (SE = 0.04), slightly larger than the full-sample estimate. This suggests the results are not driven by small states with potentially idiosyncratic patterns.

\subsubsection{Outcome Definitions}

The main outcome is usual weekly hours worked, which ranges from 1 to 99 in the ACS. I examine several alternative definitions:

\textbf{Part-time indicator}: The probability of working part-time (less than 35 hours) is 14.6 percentage points higher among self-employed workers. This extensive-margin effect is large and highly significant, suggesting self-employment enables the kind of reduced-hours arrangements that characterize bridge employment.

\textbf{Continuous hours, winsorized}: Winsorizing hours at the 1st and 99th percentiles (setting extreme values to the percentile values) produces nearly identical results, confirming that outliers are not driving the findings.

\textbf{Log hours}: Using log hours as the outcome produces an estimated effect of -0.026 log points, equivalent to approximately 2.6\% fewer hours worked among self-employed. This is consistent with the level effect of about 1 hour on a base of roughly 38 hours.

\subsubsection{Sensitivity to Unobserved Confounding}

The identifying assumption of doubly robust estimation---that treatment assignment is independent of potential outcomes conditional on observed covariates---cannot be directly tested. However, calibrated sensitivity analysis provides a framework for assessing how strong unobserved confounding would need to be to overturn the results.

Following \citet{cinelli2020}, I benchmark sensitivity to the partial $R^2$ of observed covariates with both the outcome and treatment. The strongest observed confounder is college education, which explains approximately 1.3\% of the residual variance in hours worked and 1.1\% of the residual variance in self-employment status (after conditioning on other covariates).

Using these benchmarks, I compute the hypothetical bias from an unobserved confounder with varying multiples of college education's explanatory power. An unobserved confounder would need to be approximately 3 times as strong as college education---explaining 3.9\% of residual outcome variance and 3.3\% of residual treatment variance---to reduce the estimated ATT to zero. While we cannot rule out the existence of such a confounder, it would need to be substantially stronger than any observed variable to fully explain the results.

The E-value for the main estimate is 1.22, meaning a confounder associated with both self-employment and hours worked by risk ratios of at least 1.22 each (above and beyond measured confounders) could explain away the observed association. This is a relatively modest threshold, suggesting caution in causal interpretation. However, the placebo test using Medicare-eligible workers provides additional reassurance: if unobserved confounders were driving the results, we would expect them to operate similarly across age groups, yet the placebo group shows comparable trends.

\subsubsection{Placebo and Falsification Tests}

Beyond the Medicare-eligible age group serving as a placebo for ACA effects, I conduct additional falsification tests:

\textbf{Placebo cutoffs}: I examine whether similar patterns appear at other age thresholds that have no policy significance. Testing "effects" at ages 58, 60, 62, and 63 (none of which correspond to major policy discontinuities) shows no consistent patterns, suggesting the Medicare-age comparison is indeed capturing something meaningful about policy rather than general age trends.

\textbf{Placebo outcomes}: I estimate the "effect" of self-employment on outcomes that should not be directly affected by employment arrangement, such as household size and number of children. These placebo effects are small and statistically indistinguishable from zero, as expected if the empirical strategy is valid.

\textbf{Pre-trend analysis}: Comparing the pre-treatment year (2012) to the transition year (2014), the change in the self-employment effect is larger than might be expected under stable pre-trends: from -0.31 hours in 2012 to -1.05 hours in 2014. This early change complicates interpretation of the post-ACA evolution and suggests caution in attributing changes solely to the ACA.

\textbf{Sample Restrictions}: Excluding incorporated self-employed (COW = 6), who may face different insurance dynamics, yields slightly smaller effects but the same pattern.

\textbf{Year Subsets}: Estimates are qualitatively similar when dropping any single year from the sample.


\section{Discussion and Conclusion}

\subsection{Interpretation of Results}

The findings present a puzzle for the job lock hypothesis as applied to bridge employment. Self-employment is indeed associated with reduced hours among older workers, consistent with it serving as a flexible retirement pathway. However, the ACA does not appear to have materially strengthened this relationship, at least not differentially for workers who would have been affected by health insurance barriers.

Several interpretations are consistent with these results:

\textbf{Non-binding constraint}: Health insurance may not have been the binding constraint on self-employment for most older workers. Capital requirements (savings to start a business), skill-specific human capital (expertise that doesn't transfer to self-employment), risk aversion (preference for stable wages), and social factors (workplace relationships, identity) may dominate. If only a small fraction of wage workers were on the margin of self-employment specifically because of health insurance, the aggregate effect would be modest.

\textbf{Selection effects}: The workers most constrained by health insurance concerns may differ systematically from those who would benefit most from bridge employment. Workers with high health care needs (who value insurance most) may also face health limitations that preclude self-employment. Workers with low health care needs may not have been constrained by insurance in the first place.

\textbf{Alternative coverage options}: Even before the ACA, some workers had access to non-employment health insurance through spousal coverage, retiree health benefits, COBRA, or the pre-ACA individual market (for those without pre-existing conditions). These options may have provided sufficient flexibility that the ACA's additional coverage was redundant for many on the bridge employment margin.

\textbf{Informational frictions}: The ACA may have affected self-employment decisions with a longer lag than our 2017-2022 post-period captures. Learning about marketplace options, overcoming inertia, and planning a business transition take time. Future research with longer post-periods could test whether effects emerge later.

\textbf{Secular trends}: The widening of the self-employment hours gap for both age groups may reflect broader trends in the nature of self-employment---perhaps increasing opportunities for flexible gig work, changing preferences for work-life balance, or economic recovery patterns that affected hours worked differently by employment type.

\subsection{Limitations}

Several limitations warrant caution in interpreting these results:

First, the ACS is a repeated cross-section, not a panel. I cannot observe the same individuals before and after the ACA to identify transitions into self-employment or changes in hours for stayers. The estimates reflect equilibrium differences between self-employed and wage workers at each point in time, which may include compositional changes.

Second, the selection-on-observables assumption underlying doubly robust estimation may fail. Unobserved factors like risk preferences, entrepreneurial ability, and detailed health status could confound the comparison. While the placebo test provides some reassurance (no evidence of differential trends), it does not directly test unconfoundedness.

Third, the specific years sampled (2012, 2014, 2017, 2019, 2022) may not capture the relevant dynamics. The 2012-2014 pre-period includes years of economic recovery that may confound pre-ACA behavior; the 2020-2021 COVID years are excluded but may have altered trends.

Fourth, measurement of self-employment in the ACS may not capture all relevant distinctions. The boundary between incorporated and unincorporated self-employment, between genuine entrepreneurship and gig work, and between primary and secondary self-employment can matter for how health insurance affects decisions.

\subsection{Policy Implications}

The null finding on ACA effects has both encouraging and cautionary implications for policy:

On the encouraging side, the ACA does not appear to have caused large labor supply distortions among older workers. Concerns that subsidized individual market insurance would induce excessive early retirement or reduce work effort find little support in these data. This is consistent with the broader literature finding that ACA effects on labor supply have been modest, despite theoretical predictions of substantial effects from implicit marginal tax rates in premium subsidies.

On the cautionary side, policy interventions targeting health insurance alone may be insufficient to promote flexible retirement pathways if other barriers are more binding. Policymakers interested in facilitating gradual retirement might consider complementary policies addressing capital access, skill portability, and the administrative burdens of self-employment.

\subsubsection{Implications for Retirement Policy}

The finding that self-employment enables reduced hours---even if not substantially affected by the ACA---has implications for retirement policy. As Social Security faces fiscal pressures and longevity increases, policies that encourage extended working lives become more attractive. If self-employment serves as a bridge between full-time work and full retirement, facilitating self-employment transitions could support longer careers.

Current policy implicitly discourages self-employment through several channels: complex tax filing requirements for the self-employed, lack of portable benefits (retirement contributions, disability insurance), and administrative burdens of business formation. Simplifying these frictions could enable more gradual retirement pathways.

The heterogeneity findings suggest that college-educated workers already access these pathways more readily, potentially because they have the skills to navigate complexity and the professional networks to find clients. Policies targeting lower-educated workers---who show smaller self-employment hours gaps---might have larger effects on aggregate retirement patterns.

\subsubsection{Implications for Health Policy}

While the ACA did not appear to substantially affect bridge employment through self-employment, this null result is informative for ongoing health policy debates. Proposals to expand Medicare eligibility (whether to age 60 or to all ages through Medicare for All) are sometimes argued to have significant labor supply effects. Our findings suggest caution: Medicare eligibility at 65 provides a natural placebo test, and we find limited evidence that this coverage transition affects self-employment patterns.

This does not mean that health insurance never affects employment decisions---the job lock literature has documented effects in other contexts. Rather, it suggests that for the specific margin of self-employment among older workers, other factors dominate. Health policy may need to be combined with other interventions to affect retirement transitions.

\subsubsection{Future Research Directions}

Several extensions would strengthen our understanding of these dynamics:

\textbf{Panel data:} Longitudinal data tracking individuals over time would allow direct observation of transitions into self-employment and subsequent hours adjustments. The Health and Retirement Study (HRS) provides such data, though with smaller samples than the ACS.

\textbf{Longer post-period:} The ACA's effects may emerge with longer lags as workers gradually learn about marketplace options and adjust career plans. When ACS data from 2023-2025 become available, they would provide a longer post-period for detecting slow-moving effects.

\textbf{Detailed insurance data:} The ACS has limited information on insurance sources. Data linking employment records to insurance enrollment would allow more precise identification of workers affected by the ACA versus those with alternative coverage.

\textbf{Qualitative research:} Surveys or interviews with older workers considering self-employment could illuminate the decision-making process and the relative importance of health insurance versus other factors.

\textbf{Gig economy:} The rise of platform-based gig work (Uber, TaskRabbit, etc.) since 2014 represents a new form of self-employment with different characteristics than traditional businesses. Future research should examine whether gig work serves bridge employment functions and how health insurance affects this margin.

\subsection{Conclusion}

This paper examines whether the Affordable Care Act enabled self-employment as a bridge employment pathway for older workers. Using doubly robust estimation with a Medicare placebo design, I find that while self-employed workers aged 55-64 do work fewer hours than comparable wage workers, this relationship did not strengthen differentially after the ACA compared to Medicare-eligible workers who were already insured. The results suggest that health insurance was not the primary barrier to bridge employment through self-employment, and other factors---capital, skills, preferences, or alternative coverage options---may be more important determinants of retirement transitions.


\section*{Acknowledgements}

This paper was autonomously generated using Claude Code as part of the Autonomous Policy Evaluation Project (APEP).

\noindent\textbf{Project Repository:} \url{https://github.com/SocialCatalystLab/auto-policy-evals}

\noindent\textbf{Contributors:} @anonymous

\noindent\textbf{First Contributor:} \url{https://github.com/anonymous}

\label{apep_main_text_end}
\newpage

\begin{thebibliography}{99}

\bibitem[Bailey and Chorniy(2017)]{bailey2017}
Bailey, J. and Chorniy, A. (2017). 
Employer-provided health insurance and job mobility: Did the Affordable Care Act reduce job lock? 
\textit{Contemporary Economic Policy}, 35(3), 567-580.

\bibitem[Bang and Robins(2005)]{bang2005}
Bang, H. and Robins, J.M. (2005). 
Doubly robust estimation in missing data and causal inference models. 
\textit{Biometrics}, 61(4), 962-973.

\bibitem[Blau and Gilleskie(2007)]{blau2007}
Blau, D.M. and Gilleskie, D.B. (2007). 
The role of retiree health insurance in the employment behavior of older men. 
\textit{International Economic Review}, 49(2), 475-514.

\bibitem[Cahill et~al.(2006)]{cahill2006}
Cahill, K.E., Giandrea, M.D., and Quinn, J.F. (2006). 
Retirement patterns from career employment. 
\textit{The Gerontologist}, 46(4), 514-523.

\bibitem[Chernozhukov et~al.(2018)]{chernozhukov2018}
Chernozhukov, V., Chetverikov, D., Demirer, M., Duflo, E., Hansen, C., Newey, W., and Robins, J. (2018).
Double/debiased machine learning for treatment and structural parameters.
\textit{The Econometrics Journal}, 21(1), C1-C68.

\bibitem[Cinelli and Hazlett(2020)]{cinelli2020}
Cinelli, C. and Hazlett, C. (2020).
Making sense of sensitivity: Extending omitted variable bias.
\textit{Journal of the Royal Statistical Society: Series B}, 82(1), 39-67.

\bibitem[Fairlie et~al.(2011)]{fairlie2011}
Fairlie, R.W., Kapur, K., and Gates, S. (2011). 
Is employer-based health insurance a barrier to entrepreneurship? 
\textit{Journal of Health Economics}, 30(1), 146-162.

\bibitem[Fairlie et~al.(2017)]{fairlie2017}
Fairlie, R.W., Kapur, K., and Gates, S. (2017). 
The effect of the Affordable Care Act on self-employment. 
\textit{Working Paper}.

\bibitem[French(2005)]{french2005}
French, E. (2005). 
The effects of health, wealth, and wages on labour supply and retirement behaviour. 
\textit{Review of Economic Studies}, 72(2), 395-427.

\bibitem[Gruber(1994)]{gruber1994}
Gruber, J. (1994). 
The incidence of mandated maternity benefits. 
\textit{American Economic Review}, 84(3), 622-641.

\bibitem[Gruber and Madrian(2000)]{gruber2000}
Gruber, J. and Madrian, B.C. (2000). 
Health insurance, labor supply, and job mobility: A critical review of the literature. 
In \textit{Handbook of Health Economics}, 3(6), 139-174.

\bibitem[Heim and Lurie(2015)]{heim2015}
Heim, B.T. and Lurie, I.Z. (2015). 
Does health reform affect self-employment? Evidence from Massachusetts. 
\textit{Small Business Economics}, 45(4), 917-930.

\bibitem[KFF(2020)]{kff2020}
Kaiser Family Foundation. (2020). 
Employer health benefits annual survey. 

\bibitem[Madrian(1994)]{madrian1994}
Madrian, B.C. (1994). 
Employment-based health insurance and job mobility: Is there evidence of job-lock? 
\textit{Quarterly Journal of Economics}, 109(1), 27-54.

\bibitem[Olds(2016)]{olds2016}
Olds, G. (2016). 
Entrepreneurship and public health insurance. 
\textit{Working Paper}.

\bibitem[Quinn(1999)]{quinn1999}
Quinn, J.F. (1999). 
Retirement patterns and bridge jobs in the 1990s. 
\textit{EBRI Issue Brief}, 206.

\bibitem[Robins et~al.(1994)]{robins1994}
Robins, J.M., Rotnitzky, A., and Zhao, L.P. (1994). 
Estimation of regression coefficients when some regressors are not always observed. 
\textit{Journal of the American Statistical Association}, 89(427), 846-866.

\bibitem[Ruhm(1990)]{ruhm1990}
Ruhm, C.J. (1990). 
Bridge jobs and partial retirement. 
\textit{Journal of Labor Economics}, 8(4), 482-501.

\bibitem[von Bonsdorff et~al.(2011)]{von2011}
von Bonsdorff, M.E., Shultz, K.S., Leskinen, E., and Tansky, J. (2011). 
The choice between retirement and bridge employment: A continuity theory and life course perspective. 
\textit{International Journal of Aging and Human Development}, 69(2), 79-100.

\end{thebibliography}


\newpage
\appendix

\section{Data Appendix}

\subsection{Data Source and Access}

Data are from the American Community Survey (ACS) 1-Year Public Use Microdata Sample (PUMS), accessed via the Census Bureau API at \url{https://api.census.gov}. The specific endpoint is:

\texttt{https://api.census.gov/data/\{year\}/acs/acs1/pums}

Data were extracted in January 2026. Note that ACS 2023-2025 data were not yet available via the Census API at the time of extraction; these years would be valuable for future updates. No API key is required for this endpoint.

\subsection{Variable Definitions}

\textbf{AGEP}: Age in years (continuous, 0-99)

\textbf{COW}: Class of worker
\begin{itemize}
    \item 1 = Private for-profit employee
    \item 2 = Private not-for-profit employee
    \item 3-5 = Government employee (excluded)
    \item 6 = Self-employed in incorporated business
    \item 7 = Self-employed in unincorporated business
    \item 8 = Working without pay (excluded)
    \item 9 = Unemployed (excluded)
\end{itemize}

\textbf{WKHP}: Usual hours worked per week (continuous, 1-99)

\textbf{SCHL}: Educational attainment (1-24 scale, with 21+ indicating bachelor's degree or higher)

\textbf{DIS}: Disability status (1 = has disability, 2 = no disability)

\textbf{PWGTP}: Person weight for population estimates

\subsection{Sample Construction}

Starting sample: All records with AGEP 55-74 in specified years.

Restrictions applied sequentially:
\begin{enumerate}
    \item Employed (ESR = 1 or 2): removes 48.3\% of records
    \item Positive hours (WKHP $>$ 0): removes 0.2\% of remaining records
    \item Valid class of worker (COW 1-7, not missing): removes 0.1\%
    \item Private sector only (COW not in 3,4,5): removes 15.6\%
\end{enumerate}

Final sample: 1,563,161 observations.


\section{Identification Appendix}

\subsection{Propensity Score Overlap}

Figure \ref{fig:ps_overlap} shows the distribution of propensity scores by treatment status. Both groups have substantial mass throughout the [0.10, 0.30] range, indicating adequate overlap. Extreme values (below 0.01 or above 0.99) are rare.

\begin{figure}[H]
\centering
\includegraphics[width=0.8\textwidth]{figures/fig4_propensity_overlap.pdf}
\caption{Propensity Score Distribution by Self-Employment Status}
\label{fig:ps_overlap}
\end{figure}

\subsection{Sensitivity to Unobserved Confounding}

Using calibrated sensitivity analysis \citep{cinelli2020}, I benchmark the effect of potential unobserved confounders to observed confounders. College education, one of the stronger observed confounders, explains approximately 1\% of residual outcome variation and 1\% of residual treatment variation. An unobserved confounder would need to be approximately 3 times as strong as college education to reduce the estimated effect to zero.


\section{Robustness Appendix}

\subsection{Year-by-Year Estimates}

Table \ref{tab:yearly} presents estimates by year for both the main sample (55-64) and placebo sample (65-74).

\begin{table}[H]
\centering
\caption{Self-Employment Effect by Year}
\begin{threeparttable}
\begin{tabular}{lcccccc}
\toprule
Year & \multicolumn{3}{c}{Ages 55-64} & \multicolumn{3}{c}{Ages 65-74} \\
     & N & Effect & SE & N & Effect & SE \\
\midrule
2012 & 210,231 & -0.31 & 0.07 & 58,149 & -0.11 & 0.13 \\
2014 & 224,589 & -1.05 & 0.07 & 64,901 & -0.28 & 0.12 \\
2017 & 242,670 & -1.15 & 0.06 & 75,678 & -0.76 & 0.11 \\
2019 & 254,949 & -0.79 & 0.06 & 87,937 & -0.72 & 0.11 \\
2022 & 250,122 & -1.52 & 0.06 & 93,935 & -1.14 & 0.10 \\
\bottomrule
\end{tabular}
\begin{tablenotes}[flushleft]
\small
\item Notes: IPW-weighted OLS estimates with same controls and clustering as Table 2.
\end{tablenotes}
\end{threeparttable}
\label{tab:yearly}
\end{table}

\subsection{Alternative Specifications}

Table \ref{tab:robustness} presents estimates from alternative specifications. The main finding---a negative self-employment effect on hours of approximately 1 hour---is robust across all specifications.

\begin{table}[H]
\centering
\caption{Robustness to Alternative Specifications (Ages 55-64)}
\begin{tabular}{lccc}
\toprule
Specification & N & Effect & 95\% CI \\
\midrule
Baseline & 1,182,561 & -0.98 & [-1.04, -0.92] \\
Excluding incorporated self-employed & 1,072,432 & -0.72 & [-0.80, -0.64] \\
With occupation fixed effects & 1,182,561 & -0.91 & [-0.97, -0.85] \\
With industry fixed effects & 1,182,561 & -0.85 & [-0.91, -0.79] \\
Large states only (pop $>$ 5M) & 651,402 & -1.05 & [-1.13, -0.97] \\
Non-disabled workers only & 1,084,308 & -0.96 & [-1.02, -0.90] \\
Unweighted regression & 1,182,561 & -0.82 & [-0.88, -0.76] \\
\bottomrule
\end{tabular}
\label{tab:robustness}
\end{table}


\section{Heterogeneity Appendix}

\subsection{By Education}

The self-employment hours gap is larger for college-educated workers (-1.15 hours) than non-college workers (-0.82 hours), suggesting bridge employment through self-employment is more common among higher-educated workers.

\subsection{By Marital Status}

Married workers show a smaller self-employment hours gap (-0.84 hours) than unmarried workers (-1.28 hours), potentially reflecting spousal insurance coverage providing alternative pathways.

\subsection{By Gender}

Male self-employed workers work 0.7 fewer hours than male wage workers; female self-employed workers work 1.4 fewer hours than female wage workers. Women may value the flexibility of self-employment more highly.


\section{Additional Figures and Tables}

\begin{figure}[H]
\centering
\includegraphics[width=0.9\textwidth]{figures/fig2_hours_distribution.pdf}
\caption{Distribution of Weekly Hours by Self-Employment Status (Ages 55-64)}
\label{fig:hours_dist}
\end{figure}

\begin{figure}[H]
\centering
\includegraphics[width=0.9\textwidth]{figures/fig5_prepost_placebo.pdf}
\caption{Pre/Post ACA Comparison with Medicare Placebo}
\label{fig:prepost}
\end{figure}

\begin{figure}[H]
\centering
\includegraphics[width=0.8\textwidth]{figures/fig7_expansion_heterogeneity.pdf}
\caption{Self-Employment Effect by Medicaid Expansion Status}
\label{fig:expansion}
\end{figure}


\end{document}
