\begin{table}[htbp]
\centering
\caption{Summary Statistics}
\label{tab:summary}
\begin{tabular}{lccc}
\toprule
 & Full Sample & \multicolumn{2}{c}{Near Threshold ($|\text{CIV} - c| \leq 15$)} \\
\cmidrule(lr){3-4}
 & & At-Risk & Distressed \\
\midrule
\textit{N (county-years)} & 3,317 & 568 & 440 \\
 & & & \\
\textit{Panel A: Outcomes} & & & \\[3pt]
Unemployment rate (\%) & 8.28 & 9.03 & 9.73 \\
 & (2.30) & (2.16) & (2.44) \\
Per capita market income (\$) & 18,988 & 17,650 & 16,298 \\
 & (3,806) & (2,499) & (2,551) \\
Poverty rate (\%) & 18.92 & 20.46 & 23.00 \\
 & (4.78) & (3.19) & (3.59) \\
 & & & \\
\textit{Panel B: Running Variable} & & & \\[3pt]
Composite Index Value (CIV) & 145.8 & 155.3 & 170.3 \\
 & (23.8) & (6.7) & (7.1) \\
\midrule
Counties & 369 & \multicolumn{2}{c}{174} \\
Fiscal years & 2007---2017 & \multicolumn{2}{c}{2007---2017} \\
\bottomrule
\end{tabular}
\begin{tablenotes}[flushleft]
\small
\item \textit{Notes:} Standard deviations in parentheses. Full sample includes all ARC county-years within $\pm 50$ CIV points of the Distressed threshold. Near-threshold sample restricted to $\pm 15$ CIV points. Unemployment rate and poverty rate are ARC three-year averages used in the CIV calculation. Per capita market income is in nominal dollars.
\end{tablenotes}
\end{table}
